\documentclass{../exp}
\usepackage{../../ikany}

\title{Category Theory for Homological Algerba}

\begin{document}
\maketitle


\section{Additive category}

There are three main concepts we need to catch:
\begin{cond}
\item zero morphisms and zero object;
\item biproduct;
\item additive functors and enriched functors.
\end{cond}

\subsection{Zeros}
We get started from definitions.
\begin{defn}
A \emph{zero object} is an object which is initial and terminal.
\end{defn}
\begin{defn}
A category is said to have \emph{zero morphisms} if every hom-set contains a specified morphism denoted by 0 that satisfies $0\circ f=0$ and $f\circ 0=0$ for all morphisms $f$.
\end{defn}


In other words, we can say that the existence of zero morphisms is equivalent to $\cat{Set_*}$-enrichment.


Here are definitions of categories in which we are interested.
\begin{defn}
A \emph{pointed category} is a cateogry with a zero object.
A \emph{semiadditive category} is a $\cat{CMon}$-enriched category with a zero object.
A \emph{additive category} is a $\cat{Ab}$-enriched category with a zero object.
A \emph{preadditive category} is another name of $\cat{Ab}$-enriched category.
\end{defn}
Note that we have of course several possible ways to give some equivalent definitions.
For example, the following proposition is well-admitted as the definition of semiadditive category.



\begin{cd}
\text{Additive category} \ar{d}\ar{r} & \text{Semiadditive category} \ar{d}\ar{r} & \text{Pointed category} \ar{d} \\
\text{$\cat{Ab}$-enriched category} \ar{r}\lds{u}{\text{zero object}} & \text{$\cat{CMon}$-enriched category} \ar{r}\lds{u}{\text{zero object}} & \text{$\cat{Set_*}$-enriched category} \lds{u}{\text{zero object}}\\
\end{cd}


\subsection{Biproducts}
Simply saying, biproducts is something that is both product and coproduct.
To define biproduct, we need zero morphisms, so the $\cat{Set_*}$-enrichment will be assumed in all statements in this subsection.
\begin{defn}
A \emph{canonical morphism from coproduct to product} is a morphism blabla
\end{defn}
\begin{defn}[Biproduct]
If the canonical morphism from coproduct to product is an isomorphism, then we call it by the \emph{biproduct}.
\end{defn}
There is a counterexample that coproduct and product are isomorphic but it is not the biproduct.




The following two theorems must be highlightened.
The first theorem captures the familiar isomorphisms between direct sum and direct product of finitely many modules.
\begin{thm}
In a $\cat{CMon}$-enriched cateogry, finitary product is also the coproduct so that it forms the biproduct, and vice versa.
\end{thm}

\begin{thm}
A category is semiadditive if and only if it has all finite biproducts.
\end{thm}



\end{document}