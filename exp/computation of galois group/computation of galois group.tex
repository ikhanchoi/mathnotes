\documentclass{../exp}
\usepackage{../../ikany}
\title{Galois Theory}

\begin{document}
\maketitle

\section{Elementary field theory}

\begin{thm}
Let $E/F$ be a field extension.
Then, $E$ is a vector space over $F$.
\end{thm}

\begin{defn}
A \emph{degree} of a field extension $E/F$ is the dimension of the vector space $E$ over $F$ and denoted by $[E:F]$.
\end{defn}

\begin{defn}
A field extension is called \emph{finite} if its degree is finite.
\end{defn}

\begin{thm}
If $K$ is an intermediate field in a field extension $E/F$, then
\[[E:F]=[E:K][K:F].\]
\end{thm}
\begin{pf}
Boring basis counting.
\end{pf}

\begin{cor}
Finite extension of finite extension is finite.
\end{cor}

\begin{thm}
Let $E/F$ be a finite extension.
There is a tower of simple extensions.
\end{thm}



\subsection{Simple extensions}

\begin{defn}
A field extension $E/F$ is called \emph{simple} if there is an element $\alpha\in E$ such that $E$ is the smallest field containing $\alpha$ and $F$.
In this case, we write $E=F(\alpha)$.
\end{defn}




\section{Algebraic extension}

\subsection{Algebraic elements}

\begin{defn}
Let $E/F$ be a field extension.
An element $\alpha\in E$ is \emph{algebraic over $F$} if $F(\alpha)/F$ is finite.
\end{defn}

\begin{thm}
Let $E/F$ be a field extension and $\alpha\in E$.
Then, $\alpha$ is algebraic over $F$ iff there is a polynomial $f\in F[x]$ such that $f(\alpha)=0$.
\end{thm}
\begin{pf}
Since $d=[F(\alpha):F]<\oo$, we can find linearly dependent finite subset of $\{1,\alpha,\alpha^2,\cdots\}$.
The coefficients construct the polynomial.

Conversely, if there is such $f$, every element of $F(\alpha)$ is represented as a linear combination of $\{1,\alpha,\cdots,\alpha^{\deg f-1}\}$.
\end{pf}

\begin{prop}
Let $E/F$ be a field extension and $\alpha\in E$is algebraic over $F$.
Then there is a unique monic irreducible polynomial $\mu_\alpha\in F[x]$ such that $\mu_\alpha(\alpha)=0$.
\end{prop}
\begin{pf}
The polynomials satisfying $\alpha$ form an ideal of $F[x]$.
Since $F[x]$ is a PID, there is a generator which can be taken to be monic.
Since the ideal is prime, the generator is prime(=irreducible), and it is the only irreducible in the ideal.
\end{pf}

\begin{defn}
Let $E/F$ be a field extension and $\alpha\in E$ is algebraic.
A monic irreducible polynomial $\mu_\alpha\in F[x]$ satisfying $\mu_\alpha(\alpha)=0$ is called the \emph{minimal polynomial} of $\alpha$ over $F$.
\end{defn}

\begin{thm}
Let $E/F$ be a field extension and $\alpha\in E$ is algebraic.
$[F(\alpha):F]=\deg_F\mu_\alpha$.
\end{thm}
\begin{pf}
\end{pf}


\begin{ex}
A complex number is called an \emph{algebraic number} if it is algebraic over $\Q$.
\end{ex}



\subsection{Algebraic extensions}
\begin{defn}
A field extension $E/F$ is called \emph{algebraic} if all elements $\alpha\in E$ is algebraic over $F$.
\end{defn}

Equivalently,
\begin{defn}
A field extension is called \emph{algebraic} if it is a direct limit of finite extensions.
\end{defn}

\begin{thm}
Algebraic extension of algebraic extension is algebraic.
\end{thm}
\begin{pf}
Suppose $E/K$ and $K/F$ are algebraic.
Take $\alpha\in E$.
There is a polynomial 
\end{pf}

\subsection{Algebraic closure}











\section{Separable extension}



\section{Normal extension}







\section{Computation of Galois groups}

\subsection{Quartic}
In this section, we assume the following setting:
\begin{itemize}
\item $F$ is a perfect field, %유한체는 사실 잘 모르겠음
\item $f$ is an irreducible quartic over $F$,
\item $E$ is the splitting of $f$ over $F$,
\item $G=\Gal(E/F)$,
\item $H=G\cap V_4$. 
\end{itemize}
\begin{thm}
There are only five isomorphic types of transitive subgroups of the symmetric group $S_4$.
\end{thm}
\begin{cor}
$G\cong S_4,\ A_4,\ D_4,\ V_4,\ or\ C_4$.
\end{cor}
\begin{prop}
Two groups $A_4$ and $V_4$ are only transitive normal subgroups of $S_4$.
\end{prop}

Now we define our resolvent polynomial.
\begin{prop}
Let $K$ be the fixed field of $H$.
Then,
\[K=F(\alpha_1\alpha_2+\alpha_3\alpha_4,\ \alpha_1\alpha_3+\alpha_2\alpha_4,\ \alpha_1\alpha_4+\alpha_2\alpha_3).\]
\end{prop}
\begin{defn}
Let $K$ be the fixed field of $H$.
A \emph{resolvent cubic} is a cubic $R_3$ that has $K$ as the splitting field over $F$.
\end{defn}

\begin{thm}
We have
\begin{cond}
\item $G\cong S_4$ if $R_3$ is irreducible and ,
\item $G\cong A_4$ if $R_3$ is irreducible and ,
\item $G\cong D_4$ if $R_3$ has only one root in $K$ and $f$ is irreducible over $K$,
\item $G\cong C_4$ if $R_3$ has only one root in $K$ and $f$ is reducible over $K$,
\item $G\cong V_4$ if $R_3$ splits in $K$.
\end{cond}
\end{thm}
\begin{pf}
There are five possible cases:
\[(G,H)=(S_4,V_4),\ (A_4,V_4),\ (D_4,V_4),\ (V_4,V_4),\ (C_4,C_2).\]
We have
\[[K:F]=|G/H|,\qquad[E:K]=|H|.\]

If $f$ is reducible over $K$, then $\Gal(E/K)$ is no more a transitive subgroup of $S_4$ so that $H\ne V_4$ and $G\cong C_4$.
\end{pf}
\begin{cd}
E \ar{rrrr}\ds{dd} & & & & 1 \ds{d}{2} &\\
&&&& C_2 \ds{dl}[swap]{2}\ds{dr}{2}&\\
K=F_{V_4\cap G}\ds{d}&&& V_4 \ds{dl}[swap]{3}\ds{dr}{2} && C_4 \ds{dl}{2}\\
F \ar{r}& ? & A_4 \ds{dr}[swap]{2} && D_4 \ds{dl}{3} &\\
&&&S_4&&
\end{cd}




\end{document}