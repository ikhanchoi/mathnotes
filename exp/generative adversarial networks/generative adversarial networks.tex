\documentclass{../exp}
\usepackage{../../ikany}

\title{Generative Adversarial Networks}

\begin{document}
\maketitle

The AI paradigm changes when a new approximating method is discovered.

\section{Maximum likelihood estimate}

\section{Gradient descent method}
ascending stochastic gradient


\section{Minimax game}

Minimax is a \emph{decision policy} in a competitive game.

\section{Generative adversarial networks}
Let $X$ be the set of images.
Suppose the data distribution $p_{data}$ on $X$ which embodies learning materials is given.
If $x\in X$ is an image that looks like a real human face, then the distribution function $p_{data}$ has nonnegligible values near the point $x$.
We cannot describe the distribution function $p_{data}$ completely, but only can sample from it.

Let $p_g$ be a distribution on $X$.
The generator $G:\Omega\to X$ is just an arbitrarily taken random variable satisfying $p_g$ for sampling.
The discriminator $D:X\to[0,1]$ is a function
Our purpose is to construct a new method for approximating $p_g\to p_{data}$ by simultaneously updating the discriminator function $D$.


Balancing the convergence rates between $p_g$ and $D$ is important.




\end{document}