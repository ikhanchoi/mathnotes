\documentclass{../exp}
\usepackage{../../ikany}

\title{The Wightman axioms}

\begin{document}
\maketitle

\section{Definition}
\begin{defn}
A quantum field is defined as an operator valued tempered distribution satisfying:
\end{defn}


\iffalse
이차양자화: 임베드 L(H) -> L(F(H))
생성소멸: f를 left tensor prod하는 작용소로 바꾸기
		H의 원소를 L(F(H))


일차양자화에서 R이 L(H)로 바뀐 것처럼,
이차양자화도   R이 L(H)로 바뀐 것.


근데 다체론에서 H 대신 F를 쓰고 싶은데, L(H) 에서 L(F(H))로 확장시키는 방법이 있어서 L(F)로 바꿔도 됨.


양자장은 H에 작용해서 입자를 만들거나 없애는 작용소이며, 물리량과 조화롭게 연관됨

Wightman quantum field theory:
quantum field = operator valued tempered distribution
F carries a unitary representation of restricted orthochronous Poincare group

이 양자장조차 미분가능해 어머

RQFT satisfies: invariance of c, Poincare covariance


time-translation covariant quantum field에 대해서 hamiltonian과의 bracket이 미분과 같다고 식을 세우면 이게 양자장의 방정식.




H는 해공간
L(H)는 미분연산자공간

물리과 3학년 버전 파동함수: H = C valued L2 space		L(H) = C coeff pdd valued fields (물리량)
디랙입자에 대한 고전스피너: H = DA irrep valued L2 space	L(H) = DA coeff pdd valued fields?? (물리량)

M이 국소적으로 민코프스키 메트릭을 가지면 그 접공간은 STA irrep
spinor로서의 DA irrep은 four vector가 아냐 (이 STA irrep과 다른 거야, 접공간이 아냐)


디랙방정식의 해 in H 를 classical Dirac field, free Dirac spinor라고 함

STA = 실수 16차원    DA = 복소 16차원


creation operator 하나로 된 field에 vaccum state를 넣으면 state vector가 나온다.
\fi





\end{document}
















