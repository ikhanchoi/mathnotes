\documentclass{../exp}
\usepackage{../../ikany}

\title{Totally bounded uniform space}

\begin{document}
\maketitle

\section{Our problem}
In this article, we will discuss the following theorem:
\begin{thm}
Every net in a totally bounded uniform space has a Cauchy subnet.
\end{thm}
This theorem is just one direction of the following famous theorem:
\begin{thm}
A uniform space is compact if and only if it is totally bounded and complete.
\end{thm}
\begin{pf}[assuming Theorem 1.1]\,

($\Leftarrow$)
Recall that compactness is equivalent to the existence of convergence subnet of an arbitrarily taken net.
Then, it is a corollary of Theorme 1.1.

($\Rightarrow$)
Totally boundedness is easily shown by the definition of compactness.
To prove completeness, let $x:\cA\to X$ be a Cauchy net on a compact uniform space $X$.
Since $X$ is compact, $x$ has a convergent subnet $xh:\cB\to{h}\cA\to X$.
Let $xh$ converges to $x_0$.
%%%
\end{pf}


\section{Wrong proof}
The following is the copied half of proof of Lemma 39.8, written in page 262, on the book General topology of Stephen Willard:
\bigskip

\noindent\textbf{Lemma 39.8.}
$X$ is totally bounded iff each net in $X$ has a Cauchy subnet.
\begin{pf}
Let $(x_\lambda)$ be a net in the totally bounded space $X$.
Now given any $D\in\cD$, there is a set $U_D\subset X$ such that $U_D\x U_D\subset D$ and $(x_\lambda)$ is frequently in $U_D$.
Let $\Gamma=\{(\lambda,D)\mid D\in\cD\text{ and }x_\lambda\in U_D\}$, directed by $(\lambda_1,D_1)\le(\lambda_2,D_2)$ iff $\lambda_1\le\lambda_2$ and $U_{D_1}\supset U_{D_2}$.
For each $(\lambda,D)\in\Gamma$ define $x_{(\lambda,D)}=x_\lambda$.
Then, $(x_{(\lambda,D)})$ is a subset of $(x_\lambda)$ and, given $D_0\in\cD$, pick $\lambda_0\in\Lambda$ so that $(\lambda_0,D_0)\in\Gamma$.
Then,
\[(\lambda,D),(\lambda',D')\ge(\lambda_0,D_0)\impl(x_\lambda,x_{\lambda'})\in U_D\x U_{D'}\subset U_{D_0}\x U_{D_0}\subset D_0,\]
so that $(x_{(\lambda,D)})$ is a Cauchy subnet of $(x_\lambda)$.

On the other hand, ($\cdots$)
\end{pf}
\bigskip
This proof is wrong because the set $\Gamma$ is not directed.
If we choose $(\lambda_1,D_1),(\lambda_2,D_2)\in\Gamma$ and if $(\lambda_0,D_0)$ is an upper bound of them, then $U_{D_0}$ must be contained in both $U_{D_1}$ and $U_{D_2}$.
The thing is, however, $U_{D_1}$ and $U_{D_2}$ can be taken to be disjoint.

Let us give an example.
Our uniform space is the set of real numbers $\R$ and the uniformity is the standard $\{(x,y):|x-y|<\e\}_{\e\in(0,\oo)}$.
Let $(x_n)_{n=1}^\oo$ be a sequence in $\R$ defined by $x_n=(-1)^n$.
For two entourage $\e,\e'$ less than 1, we can define $U_\e\ni-1$, $U_{\e'}\ni1$.
Even though the sequence $x_n$ is frequently in both $U_\e$ and $U_{\e'}$, we have $U_\e\cap U_{\e'}=\mt$.





\section{Metric space case}









\section{Our proof using Zorn's lemma}

Then, the problem is:
How can we deal with the diagonal arguement in non-countable situation?
Concretely, which strategy for existence proof can replace the diagonal subsequence?
The answer is the axiom of choice.

As we have seen in the Section 1, the main obstacle is that non-directedness of $\Gamma$.
The reason why $\Gamma$ is not directed was there can be several possible limit points of subnets of a given net.
Therefore, we are required to ``choose'' one of them and focus on it in order to make a directed set.
Exactly in this step, Zorn's lemma will be used.
 
We give an original proof that takes a Cauchy subnet by constructing a monotone cofinal map with Zorn's lemma.
It may complement the gap in the Willard's proof.

\begin{pf}[1 of Theorem 1.1]
Let $X$ be a totally bounded uniform space.
Let $\cU$ and $\cT$ be the uniformity and topology of $X$ respectively.
Let $x:\fA\to X$ be a net in $X$.
We are going to show $x$ has a Cauchy subnet.

\Step[1]{Applying Zorn's lemma}

Define a subset $\fA'\subset\fA\x\cT\x\cU$ by
\[(\alpha,U,E)\in\fA'\iff x_\alpha\in U,\ U^2\subset E,\text{ and }x^{-1}(U)\text{ is cofinal in }\fA.\]
The third condition is a necessary condition for $U$ to contain a limit of a subnet of $x$.
Note that the order relation on $\fA'$ is defined as
\[(\alpha,U,E)\prec(\alpha',U',E')\iff \alpha\prec\alpha',\ U\supset U',\ E\supset E'.\]
Define a subset $Z\subset\cP(\fA')$ by
\[\fB\in Z\iff \pi_\cT(\fB)\subset\cT\text{ is directed}.\]
The image of $\pi_\cT$ will play a similar role like a ``filter''.
We apply Zorn's lemma on $Z$ to make an ``ultrafilter''.

First, we claim $Z\ne\mt$.
For an entourage $E\in\cU$, using totally boundedness, we can find a finite open cover $\{U_i\}_i$ of $X$ such that $U_i^2\subset E$ for all $i$.
Since $\bigcup_ix^{-1}(U_i)=x^{-1}(X)=\fA$, there is at least one $i$ such that $x^{-1}(U_i)$ is a cofinal subset of $\fA$.
If we choose any $\alpha\in x^{-1}(U_i)$, then the singleton $\{(\alpha,U_i,E)\}$ is an element of $Z$, because singleton can be always said to be directed.
(We cannot choose $U=X$ since no entourages may contain $X^2$ without totally boundedness.)

The upper bound of each chain is obtained by union.
Therefore, there is a maximal element in $Z$.
Let it denoted by $\fM$.
Here are several facts about $\fM$:
\begin{cond}
\item $\fM\subset\fA'$ inherits the order relation from $\fA\x\cT\x\cU$,
\item if $U\in\pi_\cT(\fM)$, then $\alpha\in x^{-1}(U)$ and $U^2\subset E$ imply $(\alpha,U,E)\in\fM$ by the maximality,
\item $\fM\in Z$, i.e. $\pi_\cT(\fM)$ is directed,
\item $\fM\subset\fA'$, i.e. $x^{-1}(U)$ is cofinal for $U\in\pi_\cT(\fM)$.
\end{cond}

\Step[2]{Verification of Cauchy subnet}

The goal is to show $x\o\pi_\fA:\fM\to X$ is a subnet that is Cauchy.
So, we need to show the three conditions: directedness of $\fM$, monotone cofinality of $\pi_\fA|_\fM$, and finally Cauchyness.

(directedness)
For $(\alpha,U,E),(\alpha',U',E')\in\fM$ we can find $V\in\pi_\cT(\fM)$ such that $U\cup U'\supset V$ by the directedness $\pi_\cT(\fM)$.
Since $x^{-1}(V)$ is cofinal, there is $\beta\in x^{-1}(V)$ satisfies $\alpha,\alpha'\prec\beta$.
Then, $(\beta,V,E\cap E')\in\fM$ gives an upper bound.

(monotone cofinality)
Monotonicity is trivial.
Take any $(\alpha,U,E)\in\fM$.
For any $\alpha'\in\fA$ there is $(\beta,U,E)\in\fM$ such that $\alpha,\alpha'\prec\beta$ since $x^{-1}(U)$ is cofinal.

(Cauchyness)
We claim $\pi_\cU(\fM)=\cU$.
Assume $E\notin\pi_\cU(\fM)$.
Let $\{V_i\}_i$ be a finite open cover of $X$ such that $V_i^2\subset E$ for all $i$.
Suppose for every $i$ there exists $U_i\in\pi_\cT(\fM)$ such that $x^{-1}(U_i\cap V_i)$ is bounded above, i.e. not cofinal.
If we let $U\in\pi_\cT(\fM)$ be an upper bound of $\{U_i\}_i$, then $x^{-1}(U\cap V_i)\subset x^{-1}(U_i\cap V_i)$ is clearly bounded above for all $i$, so
\[\bigcup_ix^{-1}(U\cap V_i)=x^{-1}(U\cap X)=x^{-1}(U)\]
is also bounded above, which gives a contradiction to the cofinality of $x^{-1}(U)$.
This implies the existence of an open set $V$ such that $V^2\subset E$ and $x^{-1}(U\cap V)$ is cofinal for all $U\in\pi_\cT(\fM)$.
With this $V$, we can deduce that the cofinality of $x^{-1}(U\cap V)$ lets the following collection
\[\fM\cup\{\,(\alpha,U\cap V,E):U\in\pi_\cT(\fM),\ \alpha\in U\cap V,\ (U\cap V)^2\in E\,\}\]
be a subset of $\fA'$.
Furthermore, it is contained in $Z$ as an element because $\{U\cap V\}_{U\in\pi_\cT(\fM)}$ is directed.
It is a contradiction to the maximality of $\fM$, therefore, $\pi_\cU(\fM)=\cU$.
The Cauchyness follows easily from this.
\end{pf}







\section{Another proof using Tychonoff's theorem}
This proof is by DL Frank, Columbia university, 1965 [ ].
The proof used Tychonoff's theorem to avoid the extremely complicated application of Zorn's lemma.

\begin{pf}[2 of Theorem 1.1]

\end{pf}


\section{Another proof using universal nets}
Joshi?

\end{document}