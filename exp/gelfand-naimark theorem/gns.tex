\documentclass{../exp}
\usepackage{../../ikany}

\title{The Gelfand-Naimark Theorems}

\begin{document}
\maketitle

This note was organized in order to read the book``characterizations of $C^*$-algebras'' by Robert S. Doran and Victor A. Belfi.

\section{Simple histories and statements}
The first abstract treatment of normed linear space was given in Banach's 1920 thesis.
$C^*$-algebras made their first appearance in 1943 in the now famous paper of Gelfad and Naimark.
The present term ``Banach algebra'' was used for the first time in 1945 by W. Ambrose.

One of early results in Banach algebra theory was generalize the classical theorem of Frobenius to the Gelfand-Mazur theorem.
\begin{thm}[Frobenius]
Every finite dimensional division algebra over $\C$ is isomorphic to $\C$.
\end{thm}
\begin{thm}[Gelfand-Mazur, 1938]
Every normed division algebra over $\C$ is isomorphic to $\C$.
\end{thm}
\begin{thm}[Mazur, 1938]
Every normed division algebra over $\R$ is isomorphic to $\R$, $\C$, or $\H$.
\end{thm}

Many important Banach algebras carry a natural involution.

In 1943, Gelfand and Naimark proved that an involutive unital Banach algebra satisfying the following three conditions
\begin{cond}
\item $\|x^*x\|=\|x^*\|\|x\|$,\hfill($C^*$-identity)
\item $\|x^*\|=\|x\|$,
\item $e+x^*x$ is invertible
\end{cond}
is same with(isometrically isomorphic to) $C^*$-algebra.
They immediately asked in a footnote if the second and third conditions are able to be deleted, which indeed turned out to be true later; the proof is very hard.
For simplicity, we often introduce the identity $\|x^*x\|=\|x\|^2$ to define $C^*$-algebras. (Since it deduces the other two conditions quite clearly.)
However, we will choose the $C^*$-condition $\|x^*x\|=\|x^*\|\|x\|$ for the definition of $C^*$-algebras, and prove the other conditions in the next section.
\begin{defn}
A \emph{$C^*$-algebra} is a involutive Banach algerba satisfying the $C^*$-identity.
\end{defn}


The followings are the statements of the Gelfand-Naimark theorem.
\begin{thm}[Gelfand-Naimark I]
A commutative $C^*$-algebra is isometrically $^*$-isomorphic to $C_0(X)$ for a locally compact Hausdorff space $X$.
\end{thm}
\begin{thm}[Gelfand-Naimark II]
A $C^*$-algebra is isometrically $^*$-isomorphic to a closed $^*$-subalgebra of $B(H)$ for a Hilbert space $H$.
\end{thm}



\section{Why does $\|x^*x\|=\|x^*\|\|x\|$ imply $\|x^*\|=\|x\|$?}



\section{Commutative Banach algebras}
The Gelfand representation $A\to C_0(\hat A)$ can be defined for commutative Banach algebras.

\begin{defn}
A \emph{symmetric} Banach algebra is an involutive Banach algebra for which the Gelfand representation preserves the involution.
We will not consider non-symmetric involutive Banach algebras in this section.
\end{defn}
Notice the following implication:
\begin{cd}
& \text{symmetirc Banach algebra} \ar[dr] &\\
\text{$C^*$-algebra} \ar[ur]\ar[dr]&& \text{Banach algebra}.\\
& \text{semisimple Banach algebra} \ar[ur] &
\end{cd}

Let $A$ be a commutative Banach algebra.
\begin{thm}
If $A$ is semisimple, then the Gelfand representation is a monomorphism; it is injective.
\end{thm}
\begin{pf}
It is because the kernel is given by the Jacobson radical.
\end{pf}
\begin{thm}
If $A$ is symmetric, then the Gelfand representation is an epimorphism; it has a dense range.
\end{thm}
\begin{pf}
The image is closed under all operations except involution, separates points, and vanishes nowhere.
If $A$ is symmetric, then the image is closed under involution.
Thus, by the Stone-Weierstrass theorem, we get the result.
\end{pf}
$C^*$-algebras are semisimple and symmetric (even if it is noncommutative), but furethermore,
\begin{thm}
If $A$ satisfies $C^*$-identity, then the Gelfand representation is isometric.
\end{thm}
Since an isometry is injective and has a closed range, therefore, it should be isometric $^*$-isomorphism.






\section{Gelfand-Naimark-Segal construction}

The study of Gelfand representation is far more richer for noncommutatives.
For noncommutative Banach algebras, we will not care about the cases not satisfying $C^*$-condition.
From now, when we mention the Gelfand representation, all Banach algebras we are going to discuss will be $C^*$-algebras.

Refer an operator algebra note by Bruce Blackadar!

\end{document}