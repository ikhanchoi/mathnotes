\documentclass{../exp}
\usepackage{../../ikany}

\title{Binary quadratic forms}

\begin{document}
\maketitle

\section{Equivalence}


\begin{defn}
Two forms are called \emph{equivalent} if they are in a same oribit with respect to $\GL_2(\Z)$-action.
\end{defn}

\begin{defn}
Two forms are called \emph{properly equivalent} if they are in a same oribit with respect to $\SL_2(\Z)$-action.
\end{defn}


\section{Definite forms}


\begin{prop}
The $\SL_2(\Z)$-action on the definite forms is not faithful, i.e. the kernel is given by a nontrivial group $\{\pm I\}$.
\end{prop}
\begin{prop}
The $\PSL_2(\Z)$-action on the definite forms is faithful.
\end{prop}

\subsection{Positive definite forms}
\begin{prop}
The set of positive definite forms admits the $\SL_2(\Z)$-action(also $\PSL_2(\Z)$-action).
\end{prop}
\begin{prop}
The $\PSL_2(\Z)$-actions on positive definite forms and negative definite forms are isomorphic.
\end{prop}

\section{Indefinite forms}


\section{Class group}



\end{document}