\documentclass{../../small}
\usepackage{../../ikhanchoi}

\begin{document}
\title{Operator Algebra Seminar Note II}
\author{Ikhan Choi}
\maketitle
\tableofcontents


\section*{Acknowledgement}
This note has been written based on the first-year graduate seminar presented at the University of Tokyo in the 2023 Autumn semester.
Each seminar was delivered for 105 minutes.
I gratefully acknowledge advice of Prof. Yasuyuki Kawahigashi and support of my colleagues Futaba Sato, Yusuke Suzuki.



\newpage
\section{October 18}

\subsection{Countably decomposable von Neumann algebras}

\begin{defn}[Countably decomposable von Neumann algebras]
Let $M$ be a von Neumann algebra.
A projection $p\in M$ is called \emph{countably decomposable} if mutually orthogonal non-zero projections majorized by $p$ are at most countable, and we say $M$ is \emph{countably decomposable} if the identity is.
\end{defn}
\begin{prop}
For a von Neumann algebra $M$, the followings are all equivalent.
\begin{parts}
\item $M$ is countably decomposable.
\item $M$ admits a faithful normal state.
\item $M$ admits a faithful normal non-degenerate representation with a cyclic and separating vector.
\item The unit ball of $M$ is metrizable in the four strong topologies.
\end{parts}
\end{prop}
\begin{pf}
(a)$\Leftrightarrow$(b)
Suppose $M$ is countably decomposable.
Let $\{\xi_i\}\subset H$ be a maximal family of unit vectors such that $\bar{M'\xi_i}$ are mutually orthogonal subspaces, taken by Zorn's lemma.
If we let $p_i$ be the projection on $\bar{M'\xi_1}$, then $p_izp_i=zp_i$ for $z\in M'$ implies $p_i\in M''=M$.
By the assumption, the family $\{\xi_i\}$ is countable.
Define a state $\omega$ of $M$ such that
\[\omega(x):=\sum_{i=1}^\infty\omega_{2^{-i}\xi_i}(x),\qquad x\in M.\]
It converges due to $\|\omega_{2^{-i}\xi_i}\|=2^{-i+1}$.
It is normal since the sequence $(2^{-i}\xi_i)$ belongs to $\ell(\N,H)$, and it is faithful because $\omega(x^*x)=0$ implies $x\xi_i=0$ for all $i$, which deduces that $x=\sum_ixp_i=0$.

Conversely, if $\omega$ is a faithful normal state, then for a mutually orthogonal family of non-zero projections $\{p_i\}\subset M$, we have
\[\{p_i\}=\bigcup_{n=1}^\infty\{p_i:\f(p_i)>n^{-1}\}\]
the countable union of finite sets.
Thus $M$ is countable decomposable.

(b)$\Leftrightarrow$(c)
Let $\omega$ be a faithful normal state of $M$.
Consider any faithful normal nondegenerate representation in which $\omega$ is a vector state so that the corresponding vector is a separating vector.
Examples include the GNS representation of $\omega$, and the composition with the diagonal map $B(H)\to B(\ell^2(\N,H))$.
Then, $\bar{M\Omega}$ admits a cyclic and separating vector $\Omega$ of $M$.
The converse is immediate, i.e.~the vector state $\omega_\Omega$ is a faithful normal state of $M$.

(a)$\Leftrightarrow$(d)
Suppose $M$ is countably decomposable and take $\{\xi_i\}_{i=1}^\infty$ and $\{p_i\}_{i=1}^\infty$ as we did.
Define
\[d(x,y):=\sum_{i=1}^\infty2^{-i}\|(x-y)\xi_i\|.\]
Clearly it generates a topology coarser than strong topology.
It is also finer because if a bounded net $x_\alpha$ in $M$ converges to zero in the metric $d$ so that $x\xi_i\to0$ for all $i$, then $H=\bigoplus_iM'\xi_i$ implies that for every $\xi\in H$ and $\e>0$ we have $\|\xi-\sum_{k=1}^nz_k\xi_{i_k}\|<\e$ for some $z_k\in M'$ so that
\[\|x_\alpha\xi\|\le\|x_\alpha(\xi-\sum_{k=1}^nz_k\xi_{i_k})\|+\sum_{k=1}^n\|x_\alpha z_k\xi_{i_k}\|<\e+\sum_{k=1}^n\|z_k\|\|x_\alpha\xi_{i_k}\|\to\e.\]
Since on the bounded part the strong and $\sigma$-strong topologies coincide, the two topologies on the unit ball are metrizable.
We can do similar for the strong$^*$ and the $\sigma$-strong$^*$ topologies.

Conversely, for a mutually orthogonal family of non-zero projections $\{p_i\}_{i\in I}\subset M$, since the net of finite partial sums $p_F:=\sum_{i\in F}p_i$ is an non-decreasing net in the closed unit ball whose supremum is the identity of $M$, there is a convergent subsequence $p_{F_n}\uparrow1$ by the metrizability, which implies $I=\bigcup_{n=1}^\infty F_n$, the countable union of finite sets.
\end{pf}


\subsection{Semi-cyclic representations}


\begin{defn}[Weights]
Let $M$ be a von Neumann algebra.
A \emph{weight} is a function $\f:M^+\to[0,\infty]$ such that
\[\f(x+y)=\f(x)+\f(y),\qquad\f(\lambda x)=\lambda\f(x),\qquad x,y\in M^+,\ \lambda\in\R^{\ge0},\]
where we use $0\cdot\infty=0$.
A weight $\f$ is said to be \emph{normal} if
\[\f(\sup_\alpha x_\alpha)=\sup_\alpha\f(x_\alpha)\]
for any bounded non-decreasing net $(x_\alpha)$ in $M^+$.
\end{defn}
\begin{defn}
Let $\f$ be a weight on a von Neumann algebra $M$.
Define a left ideal of $M$
\[\fn:=\{x\in M:\f(x^*x)<\infty\},\]
and a hereditary $*$-subalgebra of $M$
\[\fm:=\fn^*\fn=\{\sum_{i=1}^ny_i^*x_i:(x_i),(y_i)\in\fn^n\}.\]
\end{defn}


\begin{lem}
If $x,y\in M$ satisfies $y^*y\le x^*x$, then there is a unique $s\in B(H)$ such that $y=sx$ and $s=sp$, where $p$ is the range projection of $x$, and $s\in M$.
\end{lem}
\begin{pf}
Suppose $\id_H\in M\subset B(H)$.
The operator $s_0:\bar{xH}\to\bar{yH}:x\xi\mapsto y\xi$ is well defined because
\[\|y\xi\|^2=\<y^*y\xi,\xi\>\le\<x^*x\xi,\xi\>=\|x\xi\|^2.\]
Let $p$ be the range projection of $x$ and let $s:=s_0p$.
Then, $y\xi=sx\xi$ for all $\xi\in H$.
If $y=s'x$ and $s'=s'p$, then
\[x^*(s-s')^*(s-s')x=(y-y)^*(y-y)=0\]
implies
\[0=p(s-s')^*(s-s')p=(s-s')^*(s-s').\]
Therefore, $s$ is unique in $B(H)$.
If $u\in M'$ is unitary, then $usu^*$ satisfies the same property $y=usu^*x$ and $usu^*=usu^*p$, so $us=su$.
Since the unitary span the whole C$^*$-algebra, we have $s\in M''=M$.
\end{pf}

\begin{prop}
Let $\f$ be a weight on a von Neumann algebra $M$.
\begin{parts}
\item Every element of $\fm^+$ can be written to be $x^*x$ for some $x\in\fn$.
\item Every element of $\fm$ can be written to be $y^*x$ for some $x,y\in\fn$.
\end{parts}
\end{prop}
\begin{pf}
(a)
Let $a:=\sum_{i=1}^ny_i^*x_i\in\fm^+$ for some $x_i,y_i\in\fn$.
The polarization writes
\[a=\frac14\sum_{i=1}^n\sum_{k=0}^3i^k|x_i+i^ky_i|^2\]
and $a^*=a$ implies
\[a=\frac12\sum_{i=1}^n(|x_i+y_i|^2-|x_i-y_i|^2)\le\frac12\sum_{i=1}^n|x_i+y_i|^2\]
implies
\[\f(a)\le\frac12\sum_{i=1}^n\f(|x_i+y_i|^2)<\infty.\]
Therefore, if $x:=a^{\frac12}\in\fn$, then $a=x^*x$.

(b)
Let $a:=\sum_{i=1}^ny_i^*x_i\in\fm$ for some $x_i,y_i\in\fn$.
Let $x:=(\sum_{i=1}^nx_i^*x_i)^{\frac12}\in\fn$.
Since $x_i^*x_i\le x^2$, we have $s_i\in M$ such that $x_i=s_ix$.
If we let $y:=\sum_{i=1}^ns_i^*y_i\in\fn$, then
\[a=\sum_{i=1}^ny_i^*x_i=\sum_{i=1}^ny_i^*s_ix=(\sum_{i=1}^ns_i^*y_i)x=y^*x.\qedhere\]
\end{pf}


\begin{defn}[Semi-cyclic representations]
Let $\f$ be a weight on a von Neumann algebra.
Let $H$ be the Hilbert space defined by the separation and completion of a sesquilinear form
\[\fn\times\fn\to\C:(x,y)\mapsto\f(y^*x)\]
and let $\psi:\fn\to H$ be the canonical image map.
The pair $(\pi,\psi)$ is called the \emph{semi-cyclic representation} associated to $\f$.
\end{defn}


\begin{prop}
Let $\f$ be a weight on a von Neumann algebra and $(\pi,\psi)$ be the associated semi-cyclic representation to $\f$.
Consider a map
\[\Theta:\fm\times\pi(M)'\to\C:(y^*x,z)\mapsto\<z\psi(x),\psi(y)\>\]
and define
\[\theta:\fm\to(\pi(M)')_*,\qquad\theta^*:\pi(M)'\to\fm^\#\]
such that $\Theta(x,z)=\theta(x)(z)=\theta^*(z)(x)$ for $x\in\fm$ and $z\in\pi(M)'$.
\begin{parts}
\item $\Theta$ is a well-defined bilinear form.
\item $\theta^*$ is bijective onto the space of linear functionals on $\fm$ whose absolute value is majorized by $\f$. (bounded Radon-Nikodym theorem)
\end{parts}
\end{prop}
\begin{pf}
(a)
The linearity in the second argument is obvious.
Fix $z\in\pi(M)'$.
We first check the well-definedness on $\fm^+$.
Let $x^*x=y^*y\in\fm^+$ for $x,y\in\fn$.
Then, there is $s\in M$ such that $y=sx$ and $s=sp$, where $p$ is the range projection of $x$, so
\[x^*(1-s^*s)x=x^*x-y^*y=0\]
implies
\[0=p(1-s^*s)p=p-s^*s\]
and $x=px=s^*sx=s^*y$.
The well-definedness follows from
\[\Theta(x^*x,z)=\<z\psi(x),\psi(x)\>=\<\pi(s)z\pi(s^*)\psi(y),\psi(y)\>=\<z\psi(ss^*y),\psi(y)\>=\Theta(y^*y,z).\]

The homogeneity is clear, so now we prove the addivitiy.
Let $x^*x,y^*y\in\fm^+$ for some $x,y\in\fn$.
Let $a:=(x^*x+y^*y)^{\frac12}$ and take $s,t\in M$ such that $x=sa$, $y=ta$, $s=sa$, and $t=ta$, where $p$ is the range projection of $a$.
Then,
\[a(1-s^*s-t^*t)a=a^*a-x^*x-y^*y=0\]
implies
\[p(1-s^*s-t^*t)p=p-s^*s-t^*t.\]
It follows that
\begin{align*}
\Theta(x^*x+y^*y,z)
&=\<z\psi(a),\psi(a)\>=\<z\pi(p)\psi(a),\psi(a)\>\\
&=\<z\pi(s^*s)\psi(a),\psi(a)\>+\<z\pi(t^*t)\psi(a),\psi(a)\>\\
&=\<z\psi(x),\psi(x)\>+\<z\psi(y),\psi(y)\>\\
&=\Theta(x^*x,z)+\Theta(y^*y,z).
\end{align*}
Now the $\Theta(\cdot,z)$ is linearly extendable to $\fm$.

(b)
The linear map $\theta^*$ is injective since $\psi$ has dense range.
Take $z\in\pi(M)'$ and consider $\theta^*(z)$, which maps $x^*x$ to $\<z\psi(x),\psi(x)\>$ for $x\in\fn$.
The image is majorized by $\f$ as
\[|\<z\psi(x),\psi(x)\>|\le\|z\|\|\psi(x)\|^2=\|z\|\f(x^*x).\]
Conversely, let $l\in\fm^\#$ is a linear functional majorized by $\f$, i.e.~there is a constant $C>0$ such that
\[|l(x^*x)|\le C\f(x^*x),\qquad x\in\fn.\]
Define a sesquilinear form $\sigma:\fn\times\fn\to\C$ such that $\sigma(x,y):=l(y^*x)$.
It is well-defined after separation of $\fn$ and is bounded by the Cauhy-Schwartz inequality
\[|\sigma(x,y)|^2=|l(y^*x)|^2\le\|l(x^*x)\|\|l(y^*y)\|\le\f(x^*x)\f(y^*y)=\|\psi(x)\|^2\|\psi(y)\|^2.\]
Therefore, $\sigma$ defines a bounded linear operator $z\in\pi(M)'$ such that
\[\sigma(x,y)=\<z\psi(x),\psi(y)\>,\]
exactly meaning $\theta^*(z)(y^*x)=l(y^*x)$ for $x,y\in\fn$.
\end{pf}

Note that we have a commutative diagram
\[\begin{tikzcd}
\fn \ar{r}{\psi}\ar[swap]{dd}{|\cdot|^2} & H \ar{d}{\omega} \\
& B(H)_* \ar{d}{\mathrm{res}}\\
\fm^+ \ar{r}{\theta} & (\pi(M)')_*.
\end{tikzcd}\]
In particular, for $x\in\fn^+$ we have
\[\|\theta(x^2)\|=\|\omega_{\psi(x)}\|=\|\psi(x)\|^2=\f(x^2).\]


\begin{lem}
Let 
For $z\in\fm^{sa}$, we have
\[\inf\{\f(a):z\le a\in\fm^+\}\le\|\theta(z)\|.\]
In particular, for $x,y\in\fn^+$ and for any $\e>0$ there is $a\in\fm^+$ such that $x^2-y^2\le a$ and
\[\f(a)\le\|\theta(x^2-y^2)\|+\e=\|\omega_{\psi(x)}-\omega_{\psi(y)}\|+\e.\]
\end{lem}
\begin{pf}
Denote by $p(z)$ the left-hand side of the inequality.
Then, we can check $p:\fm^{sa}\to\R_{\ge0}$ is a semi-norm such that $p(z)=\f(z)$ for $z\ge0$.
(If we take $p(z):=\f(z^+)$, then it seems to be dangerous when checking the sublinearity. I could not find the counterexample.)

Fix any non-zero $z_0\in\fm^{sa}$.
By the Hahn-Banach extension, there is an algebraic real linear functional $l:\fm^{sa}\to\R$ such that
\[l(z_0)=p(z_0),\qquad |l(z)|\le p(z),\qquad z\in\fm^{sa}.\]
Extend linearly $l$ to be $l:\fm\to\C$.
Since $|l(z)|\le\f(z)$ for $z\in\fm^+$, the linear functional $l$ is contained in the image of the closed unit ball under the injective map
\[\theta^*:\pi(M)'\to\fm^\#.\]
If we let $a\in(\pi(M)')_1$ be the corresponding operator such that $\theta^*(a)=l$, then we get
\[p(z_0)=l(z_0)=\theta^*(a)(z_0)=\theta(z_0)(a)\le\|\theta(z_0)\|.\]
Since $z_0\in\fm^{sa}$ is aribtrary, we are done.
\end{pf}




\subsection{$\sigma$-weak lower semi-continuity}


\begin{thm}
Let $M$ be a countably decomposable von Neumann algebra.
Then,  normal weight on $M$ is $\sigma$-weakly lower semi-continuous.
\end{thm}
\begin{pf}
Let $\f$ be a normal weight on $M$ and let $(\pi,\psi)$ be the associated semi-cyclic representation.

In the spirit of the Krein-\v Smulian theorem, the $\sigma$-weak lower semi-continuity is equivalent to the $\sigma$-weak closedness of the intersection with the ball
\begin{align*}
\f^{-1}([0,1])_1
&=\{x\in M^+:\f(x)\le1,\ \|x\|\le1\}\\
&=\{x\in M^+:\|\psi(x^{\frac12})\|\le1,\ \|x^{\frac12}\|\le1\}.
\end{align*}
Since that the $\sigma$-weak and $\sigma$-strong closedness of a convex set are equivalent and that the square root operation on $M^+_1$ is $\sigma$-strongly continuous, we are enough to show the set
\[(\f^{-1}([0,1])_1)^{\frac12}=\{x\in M^+:\|\psi(x)\|\le1,\ \|x\|\le1\}\]
is $\sigma$-weakly closed.
This set, if we denote the graph of $\psi:\fn\to H$ by $\Gamma_\psi$, is the image of the positive part of the unit ball
\[(\Gamma_\psi)^+_1=\{(x,\psi(x))\in M^+\oplus_\infty H:\|\psi(x)\|\le1,\ \|x\|\le1\}\]
under the projection $M\oplus_\infty H\to M$.
Observing $M\oplus_\infty H\cong(M_*\oplus_1H)^*$, if we prove $(\Gamma_\psi)_1^+$ is weakly$^*$ closed, then we are done by its compactness.

Consider a linear functional $l:M\oplus_\infty H\to\C$ that is continuous with respect to $(\sigma s,\|\cdot\|)$.
If we define $l_1:M\to\C$ and $l_2:H\to\C$ such that $l_1(x):=l(x,0)$ and $l_2(\xi)=(0,\xi)$, then they satisfy $l(x,\xi)=l_1(x)+l_2(\xi)$, and are continuous in $\sigma$-strong and norm topologies, hence to $\sigma$-weak and weak topologies, respectively.
Since a net $(x_\alpha,\xi_\alpha)$ converges to $(x,\xi)$ weakly$^*$ if and only if $x_\alpha\to x$ $\sigma$-weakly and $\xi_\alpha\to\xi$ weakly, $l$ is weakly$^*$ continuous.
Because $(\Gamma_\psi)_1^+$ is convex, we will now show that $(\Gamma_\psi)^+_1$ is closed in $(M,\sigma s)\times(H,\|\cdot\|)$.

Note that the unit ball $M_1$ is metrizable in $\sigma$-strong topology since $M$ is countably decomposable.
Suppose a sequence $x_n\in\fn^+_1$ satisfies $x_n\to x$ $\sigma$-strongly and $\psi(x_n)\to\xi$ in $H$.
Then, it suffices to show the following two statements: $x\in\fn^+_1$ and $\psi(x)=\xi$.
We first observe that since $\psi(x_n)$ is Cauchy, so is $\omega_{\psi(x_n)}$ in $(\pi(M)')_*$.

Consider for a while, a family of functions
\[f_a(t):=\frac t{1+at},\qquad t\in(-a^{-1},\infty),\]
parametrized by $a>0$.
They have several properties.
At first, they are operator monotone.
Next, they are $\sigma$-strongly continuous on a closed subset of its domain due to the boundedness of $f_a$, as we can see in the proof of the Kaplansky density theorem.
Finally, for each $x\in M_+$, the increasing limit $f_a(x)\uparrow x$ in norm as $a\to0$ implies that $\sup_af_a(x)=x$.

First we show $x\in\fn^+_1$.
It is clear that $x\in M^+_1$, so it is enough to show $\f(x^2)<\infty$.
By taking a subsequence, we may assume $\|\omega_{\psi(x_{n+1})}-\omega_{\psi(x_n)}\|<\frac1{2^n}$.
In order to dominate $x_n$ with an monotone sequence, find $a_n\in\fm^+$ such that
\[x_{n+1}^2-x_n^2\le a_n,\qquad \f(a_n)<\frac1{2^n},\]
using the previous lemma.
Then, we can write
\[x_{n+1}^2\le x_1^2+\sum_{k=1}^n(x_{k+1}^2-x_k^2)\le x_1+\sum_{k=1}^na_k.\]
Here the right-hand side is non-decreasing but not a bounded sequence so we take $f_a$ to get the $\sigma$-strong limit
\[f_a(x^2)\le\sup_nf_a(x_1^2+\sum_{k=1}^na_k).\]
Then, by the normality of $\f$, we have
\begin{align*}
\f(f_a(x^2))
&\le\sup_n\f(f_a(x_1^2+\sum_{k=1}^na_k))\\
&\le\sup_n\f(x_1^2+\sum_{k=1}^na_k)\\
&=\f(x_1^2)+\sum_{k=1}^\infty\f(a_k)\\
&<\f(x_1^2)+1<\infty
\end{align*}
which implies by sending $a\to0$ that $\f(x^2)<\infty$, whence $x\in\fn$.

Next we show $\psi(x)=\xi$.
If we prove $\f((x_n-x)^2)\to0$, then
\[\|\xi-\psi(x)\|\le\|\xi-\psi(x_n)\|+\|\psi(x_n)-\psi(x)\|=\|\xi-\psi(x_n)\|+\f((x_n-x)^2)^{\frac12}\to0\]
deduces the desired result.
By taking a subsequence, since $\psi(x_n-x)$ is Cauchy, we may assume
\[\|\omega_{\psi(x_n-x)}-\omega_{\psi(x_{n+1}-x)}\|<\frac1{2^n}.\]
Let $b_n\in\fm^+$ such that
\[(x_n-x)^2-(x_{n+1}-x)^2\le b_n,\qquad\f(b_n)<\frac1{2^n}\]
As we did previously, we have
\[f_a((x_n-x)^2)\le f_a((x_{m+1}-x)^2)+f_a(\sum_{k=n}^mb_k)\to\sup_mf_a(\sum_{k=n}^mb_k)\]
as $m\to\infty$ and
\[\f(f_a((x_n-x)^2))
\le\sup_m\f(f_a(\sum_{k=n}^mb_k))
\le\sup_m\f(\sum_{k=n}^mb_k)<\frac1{2^{n-1}}.\]
Therefore,
\[\f((x_n-x)^2)\le\frac1{2^{n-1}}\to0.\qedhere\]
\end{pf}


\begin{thm}
Let $M$ be an arbitrary von Neumann algebra.
Then, a normal weight on $M$ is $\sigma$-weakly lower semi-continuous.
\end{thm}
\begin{pf}
Let $\f$ be a normal weight of $M$.
Let $\Sigma$ be the set of all countably decomposable projections of $M$ and let $M_0:=\bigcup_{p\in\Sigma}pMp$.
The equivalent condition for $x\in M$ to belong to $M_0$ is that the left and right support projections of $x$ are countably decomposable.
Since then the left support projection $p$ and the right support projection $q$ of $x$ are Murray-von Neumann equivalent so that there is a $*$-isomorphism between $pMp$ and $qMq$, the countable decomposability is equivalent for $p$ and $q$.
It implies that $M_0$ is an algebraic ideal of $M$.
(Moreover, $M_0$ is $\sigma$-weakly sequentially closed in $M$ since if a sequence $x_n\in M_0$ converges to $x\in M$ $\sigma$-weakly, then for $p_n\in\Sigma$ such that $x_n=p_nx_np_n$, we have $p\in\Sigma$ with $p_n\le p$ so that $x_n=px_np$ converges to $x=pxp$ $\sigma$-weakly.
This fact is not needed in the proof.)

We first claim that $\f^{-1}([0,1])_1$ is relatively $\sigma$-weakly closed in $M_0$.
Let $y\in\bar{\f^{-1}([0,1])_1}^{\sigma w}\cap M_0$ so that there is a net $y_\alpha\in\f^{-1}([0,1])_1$ converges $\sigma$-weakly to $y$, and there is $p\in\Sigma$ such that $pyp=y$.
Note that the previous theorem states that $\f^{-1}([0,1])\cap pMp$ is $\sigma$-weakly closed.
Since $py_\alpha p$ is a net in $\f^{-1}([0,1])_1\cap pMp$ that also converges $\sigma$-weakly to $pyp=y$, we have $y\in\f^{-1}([0,1])$.
The claim proved.

We now claim that $\f^{-1}([0,1])_1$ is $\sigma$-weakly closed in $M$.
Suppose a net $x_\alpha\in\f^{-1}([0,1])_1$ converges to $x\in M$ $\sigma$-weakly.
Clearly $x\in M_1^+$.
Let $\{p_i\}_{i\in I}$ be a maximal mutually orthogonal projections in $\Sigma$, and let $p_J:=\sum_{i\in J}p_i$ for finite sets $J\subset I$ so that $\sup_Jp_J=1$.
It clearly follows that for each $\alpha$ we have
\[x_\alpha^{\frac12}p_Jx_\alpha^{\frac12}\in\f^{-1}([0,1])_1.\]
Then, we can show easily with boundedness of $x_\alpha$ that
\[x^{\frac12}p_Jx^{\frac12}\in\bar{\f^{-1}([0,1])_1}^{\sigma w}.\]
Because $p_J\in M_0$ and $M_0$ is an ideal, 
\[x^{\frac12}p_Jx^{\frac12}\in\bar{\f^{-1}([0,1])_1}^{\sigma w}\cap M_0.\]
By the above claim,
\[x^{\frac12}p_Jx^{\frac12}\in\f^{-1}([0,1])_1.\]
By the normality of $\f$, we finally obtain
\[x\in\f^{-1}([0,1])_1.\]
Therefore, $\f^{-1}([0,1])_1$ is $\sigma$-weakly closed.
\end{pf}


\subsection{Pointwise supremum of normal positive linear functionals}

Endow a partial order on the set of all weights.
Then, every set of monotonically increasing subadditive homogeneous functions $\f:M^+\to[0,\infty]$ always have its supremum given by its pointwise supremum.
Since if $\f$ is the supremum of $\sigma$-weakly lower semi-continuous $\f_i$, then
\[\f^{-1}([0,1])=\bigcap_i\f_i^{-1}([0,1])\]
implies the $\sigma$-weak lower semi-continuity of $\f$.
Conversly, the following theorem holds.

\begin{thm}
Let $M$ be a von Neumann algebra.
Then, a $\sigma$-weakly lower semi-continuous monotonically increasing additivie homogeneous function $\f:M^+\to[0,\infty]$ is given by the pointwise supremum of a set of normal positive linear functionals.
\end{thm}
\begin{pf}
Let $F:=\f^{-1}([0,1])$.
It is a hereditary closed convex subset of the real locally convex space $(M^{sa},\sigma w)$.
Denote by the superscript circle the real polar set.
Since
\[F^{\circ+}=\{\omega\in M_*^+:\omega\le\f\},\qquad
F^{\circ+\circ+}=\{x\in M^+:\sup_{\omega\le\f,\ \omega\in M_*^+}\omega(x)\le1\},\]
it is enough to show $F^{\circ+\circ+}=F$.
The positive part of the real polar of $F$ is generally written as
\[F^{\circ+}=F^\circ\cap M_*^+=F^\circ\cap(-M^+)^\circ=(F\cup-M^+)^\circ=(F-M^+)^\circ.\]
Consider a sequence of inclusions
\[F\subset\bar F\subset\bar{(F-M^+)^+}\subset\bar{(F-M^+)}^+\subset(F-M^+)^{\circ\circ+}=F^{\circ+\circ+}.\]
The first, second, and forth inclusions are in fact surjective becuase $F$ is closed, hereditary, and convex.
So we claim that the reverse of the third inclusion $\bar{(F-M^+)}^+\subset\bar{(F-M^+)^+}$...
\end{pf}





\newpage
\section{November 10}
\subsection{Hilbert algebras}

Let $A$ be a $*$-algebra together with an inner product, and $H$ be its closure.
Then, we have a $*$-homomorphism $\lambda:A\to B(H)$, and a densely defined anti-linear operator $S:A\to H$.
We say $A$ is a \emph{left Hilbert algebra} if $\lambda$ is non-degenerate and $S$ is closable...?
(closable and closure notations....)

The faithfulness of $\lambda$ is deduced after the density of $A'^2$ in $H$:
\[0=\<\lambda(\xi)\eta,\zeta\>=\<\xi,\zeta F\eta\>,\qquad\eta\zeta\in A'.\]

\begin{defn}[Left Hilbert algebra]
A \emph{left Hilbert algebra} is a $*$-algebra $A$ together with an inner product such that $A^2$ is dense in $A$ and the involution is closable.
Then, we can induce the following additional devices:
\begin{enumerate}[(i)]
\item the left multiplication defines a non-degenerate $*$-homomorphism $\lambda:A\to B(H)$, where $H:=\bar A$,
\item the involution defines by its closure a closed and densely defined anti-linear operator $S$ on $H$.
\end{enumerate}
The \emph{left von Neumann algebra} of a left Hilbert algebra $A$ is defined as $\lambda(A)''$.
\end{defn}


\begin{defn}[Right Hilbert algebra]
Let $A$ be a left Hilbert algebra.
For $\eta\in H$, let $\rho(\eta):A\to H$ be a densely defined operator such that $\rho(\eta)\xi:=\lambda(\xi)\eta$ for $\xi\in A$, and let $F$ be the adjoint of $S$.
Define
\begin{align*}
B'&:=\{\eta\in H\mid A\to A:\xi\mapsto\lambda(\xi)\eta\text{ is bounded}\}=\{\eta\in H\mid\rho(\eta)\text{ is bounded}\},\\
D'&:=\{\eta\in H\mid A\to\C:\xi\mapsto\<\eta,S\xi\>\text{ is bounded}\}=\dom F,\\
A'&:=B'\cap D'.
\end{align*}
\end{defn}

\begin{prop}
Let $A$ be a left Hilbert algebra.
\begin{parts}
\item $\rho(B')\subset\lambda(A)'$.
\item $\rho:B'\to B(H)$ preserves multiplication.
\item $\rho:D'\to\{\text{closed densely defined operators}\}$ preserves involution.
\item The multiplication and the involution are compatible on $A'$.
\item $B'$ is a module over $\lambda(A)'$ and $\rho(B')$ is a left ideal of $\lambda(A)'$.
\item $\rho(B')^*\rho(B')\subset\rho(A')$.
\item $A'^2$ is dense in $H$.
\item $\lambda(A)''=\rho(A')'$.
\end{parts}
In particular, (d) means $A'$ is a $*$-algebra, (b), (c) mean that $\rho:A'^{\operatorname{op}}\to B(H)$ is a $*$-homomorphism, and (g) implies that $\rho:A'^{\operatorname{op}}\to B(H)$ is non-degenerate and $F$ is densely defined, i.e.~$A'$ is indeed a right Hilbert algebra.
\end{prop}
\begin{pf}
(a)
It follows from
\[\lambda(\xi)\rho(\eta)\xi_0=\lambda(\xi)\lambda(\xi_0)\eta=\lambda(\xi\xi_0)\eta=\rho(\eta)\xi\xi_0=\rho(\eta)\lambda(\xi)\xi_0,\qquad\xi_0\in A.\]

(b)
By the part (a),
\[\rho(\eta\zeta)\xi=\rho(\rho(\zeta)\eta)\xi=\lambda(\xi)\rho(\zeta)\eta=\rho(\zeta)\lambda(\xi)\eta=\rho(\zeta)\rho(\eta)\xi,\qquad\eta,\zeta\in B',\ \xi\in A.\]

(c)
It follows from
\begin{align*}
\<\rho(F\eta)\xi,\xi\>&=\<\lambda(\xi)F\eta,\xi\>=\<F\eta,\lambda(S\xi)\xi\>=\<F\eta,(S\xi)\xi\>\\
&=\<S((S\xi)\xi),\eta\>=\<(S\xi)\xi,\eta\>=\<\lambda(S\xi)\xi,\eta\>\\
&=\<\xi,\lambda(\xi)\eta\>=\<\xi,\rho(\eta)\xi\>=\<\rho(\eta)^*\xi,\xi\>,\qquad\eta\in D',\ \xi\in A.
\end{align*}

(d)
By the part (c),
\begin{align*}
\<F(\eta\zeta),\xi\>&=\<F(\rho(\zeta)\eta),\xi)\>=\<S\xi,\rho(\zeta)\eta\>=\<\rho(F\zeta)S\xi,\eta\>\\
&=\<\lambda(S\xi)F\zeta,\eta\>=\<F\zeta,\lambda(\xi)\eta\>=\<F\zeta,\rho(\eta)\xi\>\\
&=\<\rho(F\eta)F\zeta,\xi\>=\<(F\zeta)(F\eta),\xi\>,\qquad \eta,\zeta\in A',\ \xi\in A.
\end{align*}

(e) Note that
\[\rho(y\eta)\xi=\lambda(\xi)y\eta=y\lambda(\xi)\eta=y\rho(\eta)\xi,\qquad y\in\lambda(A)',\ \eta\in B',\ \xi\in A.\]
Hence $B'$ is a module over $\lambda(A)'$ because
\[\|\rho(y\eta)\xi\|=\|y\rho(\eta)\xi\|\le\|y\|\|\rho(\eta)\|\|\xi\|\]
implies $y\eta\in B'$ for $y\in \lambda(A)'$ and $\eta\in B'$, and $\rho(B')$ is a left ideal of $\lambda(A)'$ because $y\rho(\eta)=\rho(y\eta)\in\rho(B')$.

(f)
We first claim $\rho(B')^*B'\subset A'$.
For fixed $\eta,\zeta\in B'$, we have $\rho(\zeta)^*\eta\in D'$ since
\begin{align*}
|\<\rho(\zeta)^*\eta,S\xi\>|&=|\<\eta,\rho(\zeta)S\xi\>|=|\<\eta,\lambda(S\xi)\zeta\>|=|\<\lambda(\xi)\eta,\zeta\>|\\
&=|\<\rho(\eta)\xi,\zeta\>|=|\<\xi,\rho(\eta)^*\zeta\>|\le\|\rho(\eta)^*\zeta\|\|\xi\|,\qquad\xi\in A,
\end{align*}
and we have $\rho(\zeta)^*\eta\in B'$ since $B'$ is a module over $\lambda(A)'$, hence $\rho(\zeta)^*\eta\in A'$.
Now we have $\rho(B')^*\rho(B')\subset\rho(A')$ since
\[\rho(\zeta)^*\rho(\eta)\xi=\rho(\zeta)^*\lambda(\xi)\eta=\lambda(\xi)\rho(\zeta)^*\eta=\rho(\rho(\zeta)^*\eta)\xi,\qquad\xi\in A.\]


(g)
Since $D'$ is dense in $H$, it suffices to verify $D'\subset\bar{A'^2}$.
Let $\eta\in D'$.
Then, by (c), $\rho(\eta)$ has a densely defined adjoint $\rho(F\eta)$, it is closable.
Denoting its closure by the same notation $\rho(\eta)$, we can write down the polar decomposition $\rho(\eta)=vh=kv$.
To control the unboundedness of $\rho(\eta)$, we introduce $f\in C_c((0,\infty))^+$ to cutoff $\rho(\eta)$.

First, we have $f(h)F\eta\in B'$ since
\begin{align*}
\|\lambda(\xi)f(h)F\eta\|&=\|f(h)\lambda(\xi)F\eta\|=\|f(h)\rho(F\eta)\xi\|\\
&=\|f(h)\rho(\eta)^*\xi\|=\|f(h)hv^*\xi\|\le\sup_{t\ge0}tf(t)\|\xi\|,\qquad\xi\in A.
\end{align*}
We have $vf(h)F\eta\in B'$ since $B'$ is a module over $\lambda(A)'$.
Since
\[\rho(vf(h)F\eta)=v\rho(f(h)F\eta)=vf(h)(vh)^*=vf(h)hv^*=f(k)k,\]
we have $kf(k)\in\rho(B')$, and in fact we also have $f(k)\in\rho(B')$ because $\rho(B')$ is a left ideal of $\lambda(A)'$ and $f(t)=g(t)(tf(t))$ if we take $g\in C_c((0,\infty))$ such that $g(t)=t^{-1}$ on the support of $f$.
Applying the above arguments for $f^{\frac14}\in C_c((0,\infty))$, we may assume
\[f(k)=(f(k)^{\frac14})^4\in\rho(B)^*\rho(B)\rho(B)^*\rho(B)\subset\rho(A'^2).\]
Take $\zeta_1,\zeta_2\in A'$ such that $f(k)=\rho(\zeta_1\zeta_2)$.
For $\eta,\zeta_1\in D'$, we have $\eta\zeta_1\in D'$ because $D'$ is multiplicatively closed, and $\eta\zeta_1\in B'$ because $B'$ is a left module of $\lambda(A)'$, so $\eta\zeta_1\in A'$ by definition of $A'$.
Then,
\[f(k)\eta=\rho(\zeta_1\zeta_2)\eta=(\eta\zeta_1)\zeta_2\in A'^2.\]

If we construct a non-decreasing net $f_\alpha\in C_c((0,\infty))$ such that $\sup_\alpha f_\alpha=\1_{(0,\infty)}$, then the strong limit implies
\[\lim_\alpha f_\alpha(k)\eta=\1_{(0,\infty)}(k)\eta=s(k)\eta=s_r(\rho(F\eta))\eta=s_l(\rho(\eta))\eta.\]
Here we use the non-degeneracy of $\lambda$ to verify $\eta$ belongs to the closure of the range of $\rho(\eta)$, i.e.~since $\id_H\in\lambda(A)''$, we have a net $\xi_\alpha\in A$ such that $\lambda(\xi_\alpha)\to\id_H$ strongly so that $\lambda(\xi_\alpha)\eta\to\eta$.
It implies that $\eta\in\bar{\lambda(A)\eta}=\bar{\rho(\eta)A}$ and $s_l(\rho(\eta))\eta=\eta$.
Therefore, $\eta=s_l(\rho(\eta))\eta\in\bar{A'^2}$.

(h)
One direction is clear, i.e.~$\rho(A')\subset\rho(B')\subset \lambda(A)'$ implies $\rho(A')''\subset\lambda(A)'$.
Conversely, let $y\in\lambda(A)'$.
By the part (g), $\rho:A'^{\operatorname{op}}\to B(H)$ is non-degenerate and the $\sigma$-weak closure of $\rho(A')$ contains $\id_H$, so there is a net $\eta_\alpha\in B'$ such that $y=\lim_\alpha\rho(\eta_\alpha)^*y\rho(\eta_\alpha)\in(\rho(B')^*\rho(B'))''\subset\rho(A')''$, so we are done.
\end{pf}

\begin{defn}[Full Hilbert algebra]
Let $A$ be a left Hilbert algebra.
Symmetrically as above, starting from the right Hilbert algebra $A'$, we can construct a left Hilbert algebra $A''$.
We say $A$ is \emph{full} if $A=A''$.
\end{defn}

\begin{ex}[Commutative full Hilbert algebras]
If $(X,\mu)$ is a $\sigma$-finite measure space, then $L^2(X)\cap L^\infty(X)$ is a full Hilbert algebra.
If $G$ is a locally compact abelian group, then $A:=\cF^{-1}(L^2(\hat G)\cap L^\infty(\hat G))$ is a full Hilbert algebra, where $\cF:L^2(G)\to L^2(\hat G)$ is the Fourier transform.
\end{ex}







\subsection{Modular operator and conjugation}

\begin{defn}[Modular operator and conjugation]
Let $A$ be a left Hilbert algebra.
Denote the polar decomposition of $S$ by $S=J\Delta^{\frac12}$.
The unbounded operators $\Delta$ and $J$ are called the \emph{modular operator} and the \emph{modular conjugation}.
\end{defn}

\begin{cor}
From the polar decomposition theorem for unbounded (anti-)linear operators, we have
\begin{parts}
\item $S$ is injective with $S=S^{-1}$ and $D=\dom S=\dom\Delta^{\frac12}$.
\item $F$ is injective with $F=F^{-1}$ and $D'=\dom F=\dom\Delta^{-\frac12}$.
\item $\Delta$ is an injective positive self-adjoint operator.
\item $J$ is a conjugation, i.e.~an anti-linear isometric involution.
\item $S=J\Delta^{\frac12}=\Delta^{-\frac12}J$, $F=J\Delta^{-\frac12}=\Delta^{\frac12}J$, and $J\Delta J=\Delta^{-1}$.
\end{parts}
\end{cor}




\begin{ex}[Group von Neumann algebra]
For a locally compact group $G$, the set $A:=C_c(G)$ together with a left Haar measure on $G$ has the following left Hilbert algebra structure
\[\<\xi_1,\xi_2\>:=\int\bar{\xi_2(s)}\xi_1(s)\,ds,\qquad(\xi_1\xi_2)(s):=\int_G\xi_1(t)\xi_2(t^{-1}s)\,dt,\qquad\xi^*(s):=\Delta(s^{-1})\bar{\xi(s^{-1})}.\]
We have $S$, $F$, $\Delta$, and $J$ given by
\[S\xi(s):=\Delta(s^{-1})\bar{\xi(s^{-1})},\qquad F\xi(s)=\bar{\xi(s^{-1})},\]
\[\Delta\xi(s)=\Delta(s)\xi(s),\qquad J\xi(s)=\Delta(s)^{-\frac12}\bar{\xi(s^{-1})},\]
and they have the following norm formulas
\[\|S\xi\|_2=\|\Delta^{\frac12}\xi\|_2,\quad\|F\xi\|_2=\|\Delta^{-\frac12}\xi\|_2,\quad\|S\xi\|_1=\|\xi\|_1,\quad\|F\xi\|_1=\|\Delta^{-1}\xi\|_1.\]
The left von Neumann algebra $\lambda(A)''$ is called the \emph{group von Neumann algebra}.
\iffalse
The left involution $S$ is an isometric anti-linear automorphism of $L^1(G)=L^1(G,ds)$, but the right involution $F$ defines an isometric anti-linear isomorphism between $L^1(G,ds)$ and $L^1(G,ds^{-1})$.
To sum up,
\[\begin{array}{lrl}
\xi\in H &\Leftrightarrow& \|\xi\|_2<\infty,\\
\xi\in D &\Leftrightarrow& \|S\xi\|_2+\|\xi\|_2<\infty,\\
\xi\in A &\Rightarrow& \|\lambda(\xi)\|+\|S\xi\|_2+\|\xi\|_2<\infty,\\
\xi\in B' &\Leftrightarrow& \|\rho(\xi)\|+\|\xi\|_2<\infty,\\
\xi\in D' &\Leftrightarrow& \|F\xi\|_2+\|\xi\|_2<\infty,\\
\xi\in A' &\Leftrightarrow& \|\rho(\xi)\|+\|F\xi\|_2+\|\xi\|_2<\infty,\\
\xi\in B &\Leftrightarrow& \|\lambda(\xi)\|+\|\xi\|_2<\infty,\\
\xi\in A'' &\Leftrightarrow& \|\lambda(\xi)\|+\|S\xi\|_2+\|\xi\|_2<\infty.
\end{array}\]
\fi
\end{ex}


\begin{ex}[Cyclic separating vector]
Let $M$ be a von Neumann algebra on $H$ together with a cyclic separating vector $\Omega\in H$.
Then, $A:=M\Omega$ has the following left Hilbert algebra structure:
\[\<x\Omega,y\Omega\>\text{ is defined as it is},\qquad (x\Omega)(y\Omega):=xy\Omega,\qquad (x\Omega)^*:=x^*\Omega.\]
There is no specific description of $\Delta$ and $J$ in general, but it is known that
\[D=\{c\Omega:c\text{ closed densely defined on $H$ and affiliated with $M$},\ \Omega\in\dom c\cap\dom c*\}.\]
\end{ex}



\subsection{Faithful normal semi-finite weights}

\begin{defn}
Let $\f$ be a weight on a von Neumann algebra $M$.
We say $\f$ is \emph{faithful} if $\f(x^*x)=0$ implies $x=0$ for $x\in\fn$.
We say $\f$ is \emph{semi-finite} if $\fm$ is $\sigma$-weakly dense in $M$.
Recall that a weight $\f$ on a von Neumann algebra $M$ is normal if and only if it is obtained by the pointwise supremum of a set of normal positive linear functionals.
\end{defn}

In the proofs of theorems of this section, the following diagram might be helpful: 
\[\begin{tikzcd}
\fm=\fn^*\fn \rar[phantom,"\subset"] & \fn\cap\fn^* \rar[phantom,"\subset"] & \fn \rar[phantom,"\subset"]\dar[shift left]{\psi} & \pi(M) \rar[phantom,"\subset"] & B(H)\\
& A \rar[phantom,"\subset"]\uar{\lambda} & B \rar[phantom,"\subset"]\uar[shift left]{\lambda} & H.
\end{tikzcd}\]
Recall that for a weight $\f$ on a von Neumann algebra $M$ and its semi-cyclic representation $(\pi,\psi)$ of $M$ we have $\f(x^*x)=\|\psi(x)\|^2$ for $x\in\fn$.


\begin{thm}
Let $M$ be a von Neumann algebra.
If $A$ is a full left Hilbert algebra together with a faithful normal non-degenerate representation $\pi:M\to B(H)$ such that $\lambda(A)''=\pi(M)$, then
\[\f(x^*x):=\begin{cases}\|\xi\|^2&\text{ if }\pi(x)=\lambda(\xi)\in\lambda(B),\\\infty&\text{ otherwise},\end{cases}\]
is a faithful normal semi-finite weight on $M$.
\end{thm}
\begin{pf}
We use the notation $\pi(x)=x$.
We first check that the weight $\f$ is well-defined.
Let $x_1=\lambda(\xi_1),x_2=\lambda(\xi_2)\in\lambda(B)$ such that $x_1^*x_1=x_2^*x_2$.
Since $x_1,x_2\in M$, we have a partial isometry $v\in M$ such that $x_2=vx_1$ and $v^*v=s_l(x_1)$, and it is not diffcult to see $\xi_2=v\xi_1$.
As we have seen in the proof of the part (g) of Proposition 2.3, we know $s_l(x)\xi_1=\xi_1$, so
\[\|\xi_2\|^2=\<\xi_2,\xi_2\>=\<v\xi_1,v\xi_1\>=\<v^*v\xi_1,\xi_1\>=\<\xi_1,\xi_1\>=\|\xi_1\|^2,\]
which proves the well-definedness.

With this weight $\f$, we can see
\[\fn=\lambda(B),\qquad\fn\cap\fn^*=\lambda(A),\qquad\fm=\lambda(B)^*\lambda(B).\]
The first one is by definition of $\f$, and the third one is by definition of $\fm$.
Since $A$ is full so that $A=B\cap D$, $\lambda$ is injective, $\lambda(A)^*=\lambda(A)$, and $\lambda(D)^*=\lambda(D)$, we have $\lambda(A)=\lambda(B)\cap\lambda(D)=\lambda(B)^*\cap\lambda(D)$, which implies $\lambda(A)=\lambda(B)\cap\lambda(B)^*\cap\lambda(D)$.
Because $\lambda(\xi_1)=\lambda(\xi_2)^*$ implies
\begin{align*}
|\<\xi_1,F(\eta\zeta)\>|
&=|\<\xi_1,\rho(F\eta)F\zeta\>|=\<\rho(\eta)\xi_1,F\zeta\>|=|\<\lambda(\xi_1)\eta,F\zeta\>|\\
&=|\<\eta,\lambda(\xi_2)F\zeta\>|=|\<\eta,\rho(F\zeta)\xi_2\>|=|\<\rho(\zeta)\eta,\xi_2\>|\\
&=|\<\eta\zeta,\xi_2\>|\le\|\eta\zeta\|\|\xi_2\|,\qquad\eta,\zeta\in A',
\end{align*}
we have $\xi_1\in D$ due to the density of $A'^2$ in $H$, so $\lambda(B)\cap\lambda(B)^*\subset\lambda(D)$, hence the second equality follows.
From now in the rest of proof, we will always denote $y=\rho(\eta)$ and $z=\rho(\zeta)$ for $y,z\in\fn'$.

The weight $\f$ is clearly faithful, and semi-finiteness is because $x\in M$ is approximated by a net $\lambda(\xi_\alpha)x\lambda(\xi_\alpha)\in\lambda(B)^*\lambda(B)=\fm$, where $\lambda(\xi_\alpha)\in\lambda(B)$ converges $\sigma$-weakly to $\id_H$.
To verify the normality of $\f$, we will show
\[\f(x^*x)=\sup_{y\in\fn'_1}\omega_{\eta}(x^*x),\qquad x\in\fn,\]
where $\fn':=\rho(B')$.

($\ge$)
We may assume $x=\lambda(\xi)\in\fn=\lambda(B)$ so that $\f(x^*x)<\infty$.
Since the unit ball $\fn'_1$ has a net $y_\alpha$ that converges to $\id_H$ strongly by the Kaplansky density theorem, we have an inequality
\[\omega_{\eta_\alpha}(x^*x)=\|x\eta_\alpha\|^2=\|\lambda(\xi)\eta_\alpha\|^2=\|\rho(\eta_\alpha)\xi\|^2=\|y_\alpha\xi\|^2\le\|\xi\|^2=\f(x^*x),\]
in which the equality condition is attained at its limit.

($\le$)
Suppose $x\in M$ is taken such that the right-hand side $\sup_{y\in\fn'_1}\omega_{\eta}(x^*x)$ is finite.
If we show $x\in\fn$, then we are done from $\f(x^*x)<\infty$ by the previous argument.
To motivate the strategy, consider the opposite weight
\[\f'(y^*y):=\begin{cases}\|\eta\|^2&\text{ if }y\in\rho(B'),\\\infty&\text{ otherwise},\end{cases}\]
and the associated linear map
\[\theta'^*:M\to\fm'^\#:x^*x\mapsto(z^*y\mapsto\<x^*x\eta,\zeta\>),\qquad y,z\in\fn',\]
where we can check $\fm'=\rho(B')^*\rho(B')$.
The idea is to show a well-designed linear functional $l\in\fm'^\#$ such that $l=\theta'^*(x^*x)$ is contained in the image $\theta'^*(\fm)$ using the assumption that the right-hand side is finite to verify $x\in\fn$.

Define a linear functional
\[l:\fm'\to\C:z^*y\mapsto\<x^*x\eta,\zeta\>.\]
Then, by the assumption we have
\[\|l\|=\sup_{y\in\fn'_1}\<x^*x\eta,\eta\>=\sup_{y\in\fn'_1}\omega_\eta(x^*x)<\infty,\]
and
\[|l(y)|\le\|l\|l(y^*y)^{\frac12}=\|l\|\|x\eta\|,\qquad y\in\fn'\]
implies the well-definedness as well as boundedness of the linear functional $\bar{xH}\to\C:x\eta\mapsto l(y)$ for any $\eta\in H$, and it follows the existence of $\xi\in\bar{xH}$ such that
\[l(y)=\<x\eta,\xi\>,\qquad y\in\fn'\]
by the Riesz representation theorem on $\bar{xH}$.
We have $\lambda(\xi)\zeta\in\bar{xH}$ and
\begin{align*}
\<x\eta,x\zeta\>&=l(z^*y)=\<x\rho^{-1}(z^*y),\xi\>=\<xz^*\eta,\xi\>\\
&=\<z^*x\eta,\xi\>=\<x\eta,z\xi\>=\<x\eta,\rho(\zeta)\xi\>=\<x\eta,\lambda(\xi)\zeta\>,\qquad y,z\in\fn',
\end{align*}
hence $x=\lambda(\xi)$.
The vector $\xi$ is left bounded by definition and $x=\lambda(\xi)\in\lambda(B)=\fn$.
\end{pf}

\begin{thm}
Let $M$ be a von Neumann algebra.
If $\f$ is a faithful normal semi-finite weight on $M$ and $(\pi,\psi)$ is the associated semi-cyclic representation of $M$, then $A:=\psi(\fn\cap\fn^*)$ is a full left Hilbert algebra with
\[\<\psi(x_1),\psi(x_2)\>:=\f(x_2^*x_1),\qquad \psi(x_1)\psi(x_2):=\psi(x_1x_2),\qquad\psi(x)^*:=\psi(x^*),\]
such that $\lambda(A)''=\pi(M)$.
\end{thm}
\begin{pf}
We use the notation $\pi(x)=x$.
It does not make any confusion because the semi-cyclic representation $\pi:M\to B(H)$ is always unital and is faithful due to the assumption that $\f$ is faithful.
We clearly see that $A$ is a $*$-algebra and the left multiplication provides a $*$-homomorphism $\lambda:A\to B(H)$.
By the construction of the semi-cyclic representation associated to $\f$, $A$ is dense in $H$.
We need to show the non-degeneracy of $\lambda$, the closability of the involution, and the fullness of $A$.

(non-degeneracy)
Since $\f$ is semi-finite, there is a net $x_\alpha$ in $(\fn\cap\fn^*)_1$ converges strongly to the identity of $M$ by the Kaplansky density theorem.
Then, it follows that $\lambda$ is non-degenerate from
\[\lambda(\psi(x_\alpha))\psi(x)=\psi(x_\alpha)\psi(x)=\psi(x_\alpha x)=x_\alpha\psi(x)\to\psi(x),\qquad x\in\fn\cap\fn^*.\]

(closability)
We need to prove the domain of the adjoint
\[D':=\{\eta\in H\mid A\to\C:\psi(x)\mapsto\<\eta,\psi(x^*)\>\text{ is bounded}\}\]
is dense in $H$.
Let
\[\Phi:=\{\omega\in M_*^+:(1+\e)\omega\le\f\text{ for some }\e>0\}.\]
Note that the normality of $\f$ says that $\f(x^*x)=\sup_{\omega\in\Phi}\omega(x^*x)$ for any $x\in M$.
For each $\omega\in\Phi$, by the bounded Radon-Nikodym theorem, there is $h_\omega\in M'^+$ such that $\|h_\omega\|<1$ and
\[\omega(x^*x)=\<h_\omega\psi(x),\psi(x)\>,\qquad x\in\fn.\]
Also, if we take a net $x_\alpha\in\fn_1$ that converges $\sigma$-strongly to the identity of $M$ using the strong density of $\fn$ in $M$, the Kaplansky density, and the coincidence of strong and the $\sigma$-strong topologies on the bounded part, then we have a $\sigma$-weak limit $\lim_{\alpha,\beta}|x_\alpha-x_\beta|^2=0$ so that by the normality of $\omega$ we obtain
\[\lim_{\alpha,\beta}\|h_\omega^{\frac12}\psi(x_\alpha)-h_\omega^{\frac12}\psi(x_\beta)\|^2=\lim_{\alpha,\beta}\omega(|x_\alpha-x_\beta|^2)=0.\]
Thus, the vector $\psi_\omega:=\lim_\alpha h_\omega^{\frac12}\psi(x_\alpha)$ can be defined, and in particular, we have $h_\omega^{\frac12}\psi(x)=x\psi_\omega$ for $x\in\fn$ and $\omega=\omega_{\psi_\omega}$.

If $\eta=h_{\omega_2}^{\frac12}y\psi_{\omega_1}$ for some $y\in M'$ and $\omega_1,\omega_2\in\Phi$, then
\begin{align*}
|\<\eta,\psi(x^*)\>|
&=|\<h_{\omega_2}^{\frac12}y\psi_{\omega_1},\psi(x^*)\>|=|\<y\psi_{\omega_1},h_{\omega_2}^{\frac12}\psi(x^*)\>|=|\<y\psi_{\omega_1},x^*\psi_{\omega_2}\>|\\
&=|\<yx\psi_\omega,\psi_{\omega_2}\>|=|\<yh_{\omega_1}^{\frac12}\psi(x),\psi_{\omega_2}\>|=|\<\psi(x),h_{\omega_1}^{\frac12}y^*\psi_{\omega_2}\>|\\
&\le\|\psi(x)\|\|h_{\omega_1}^{\frac12}y^*\psi_{\omega_2}\|,\qquad x\in\fn\cap\fn^*,
\end{align*}
which deduces that $\eta\in D'$.
Therefore, it suffices to show the following space is dense in $H$:
\[\{h_{\omega_2}^{\frac12}y\psi_{\omega_1}:\omega_1,\omega_2\in\Phi,\ y\in M'\}.\]
Thanks to the normality of $\f$, we can write
\begin{align*}
\<\psi(x),\psi(x)\>&=\|\psi(x)\|^2=\f(x^*x)=\sup_{\omega\in\Phi}\omega(x^*x)\\
&=\sup_{\omega\in\Phi}\|x\psi_\omega\|^2=\sup_{\omega\in\Phi}\|h_\omega^{\frac12}\psi(x)\|^2=\sup_{\omega\in\Phi}\<h_\omega\psi(x),\psi(x)\>,\qquad x\in\fn\cap\fn^*.
\end{align*}
Because $A$ in $H$, for any $\xi\in H$ and $\e>0$ there is $x\in\fn\cap\fn^*$ such that $\|\xi-\psi(x)\|<\e$, so the inequality
\[\<(1-h_\omega)\xi,\xi\>\le\e(\|\xi\|+\|\psi(x)\|)+\<(1-h_\omega)\psi(x),\psi(x)\>\]
deduces $\inf_{\omega\in\Phi}\<(1-h_\omega)\xi,\xi\>=0$ by limiting $\e\to0$ and taking infinimum on $\omega\in\Phi$.
In other words, for each $\xi\in H$ and $\e>0$, we can find $\omega\in\Phi$ such that $\<(1-h_\omega)\xi,\xi\><\e$.
At this point, we claim the set $\{h_\omega:\omega\in\Phi\}$ is upward directed.
If the claim is true, then we can construct an increasing net $\omega_\alpha$ in $\Phi$ such that $h_{\omega_\alpha}$ converges weakly to the identity of $M$, and also strongly by the nature of increasing nets.
To prove the claim, take $h_1=h_{\omega_1}$ and $h_2=h_{\omega_2}$ with $\omega_1,\omega_2\in\Phi$.
Introduce a operator monotone function $f(t):=t/(1+t)$ and its inverse $f^{-1}(t)=t/(1-t)$ to define 
\[h_0:=f(f^{-1}(h_1)+f^{-1}(h_2)).\]
Then, we have $h_0\ge h_1$, $h_0\ge h_2$, and $\|h_0\|<1$.
Consider a linear functional
\[\omega_0:\fn\to\C:x\mapsto\<h_0\psi(x),\psi(x)\>.\]
Write
\begin{align*}
\omega_0(x^*x)
&\le\<f^{-1}(h_1)\psi(x),\psi(x)\>+\<f^{-1}(h_2)\psi(x),\psi(x)\>\\
&\le(1-\|h_1\|)^{-1}\<h_1\psi(x),\psi(x)\>+(1-\|h_2\|)^{-1}\<h_2\psi(x),\psi(x)\>\\
&=(1-\|h_1\|)^{-1}\omega_1(x^*x)+(1-\|h_2\|)^{-1}\omega_2(x^*x),\qquad x\in\fn.
\end{align*}
Then, since $\omega_1$ and $\omega_2$ are normal, we can define $\psi_0:=\lim_\alpha h_0^{\frac12}\psi(x_\alpha)\in H$ for a $\sigma$-strongly convergent net $x_\alpha\in\fn_1$ to the identity of $M$ as we have taken above, and we have the vector functional $\omega_0=\omega_{\psi_0}$.
Henceforth, $\omega_0$ is extended to a normal positive linear functional on the whole $M$, and finally the norm condition $\|h_0\|<1$ tells us that $\omega_0\in\Phi$, so the claim is true.

Now the problem is reduced to the density of $\{y\psi_{\omega}:\omega\in\Phi,\ y\in M'\}$ in $H$.
Let $p\in B(H)$ be the projection to the closure of this space.
Then, $p\psi_\omega=\psi_\omega$ for every $\omega\in\Phi$.
Since the space is left invariant under the action of the self-adjoint set $M'$, we have $p\in M$.
Then,
\[\f(1-p)=\sup_{\omega\in\Phi}\omega(1-p)=\sup_{\omega\in\Phi}\<(1-p)\psi_\omega,\psi_\omega\>=0\]
implies $p=1$, hence the density.

(fullness)
We have $\lambda(\psi(x))=x$ for $x\in\fn\cap\fn^*$ since $\psi(\fn\cap\fn^*)=A$ is dense in $H$ and
\[x_1\psi(x_2)=\psi(x_1x_2)=\psi(x_1)\psi(x_2)=\lambda(\psi(x_1))\psi(x_2),\qquad x_1,x_2\in\fn\cap\fn^*.\]
Also we have for $\xi=\psi(x)\in A$ that
\[\psi(\lambda(\xi))=\psi(\lambda(\psi(\xi)))=\psi(x)=\xi.\]
For $\xi\in B$ so that $\lambda(\xi)\in M$, since
\[\f(\lambda(\xi)^*\lambda(\xi))=\|\psi(\lambda(\xi))\|^2=\|\xi\|^2<\infty,\]
we get $\lambda(B)\subset\fn$.
Therefore, $A$ is full by
\[\lambda(A'')=\lambda(B)\cap\lambda(B)^*\subset\fn\cap\fn^*=\lambda(A).\qedhere\]
\end{pf}

\begin{cor}
The operations giving a faithful normal semi-finite weight and a full left Hilbert algebra in the above two theorems are mutually inverses of each other.
\end{cor}

\begin{prop}
Every von Neumann algebra admits a faithful normal semi-finite weight.
\end{prop}
\begin{pf}
Let $M$ be a von Neumann algebra and let $\{\omega_i\}_{i\in I}$ be a maximal family of normal states on $M$ with orthogonal support projections $p_i:=s(\omega_i)$.
Here, the support projection $s(\omega)$ of a normal state $\omega$ is the minimal projection $p$ such that $\omega(px)=\omega(x)=\omega(xp)$ for all $x\in M$.
Since every countably decomposable projection $p$ is a support of a normal state, a faithful normal state on $pMp$, we have $\sum_ip_i=1$.
Define a weight $\f$ by
\[\f(x):=\sum_i\omega_i(x).\]
It is faithful because $\f(x)=0$ with $x\ge0$ means $\omega_i(x)=0$ and $p_ixsp_i=0$ for all $i$, and it implies
\[x^{\frac12}=x^{\frac12}\sum_ip_i=\sum_ix^{\frac12}p_i=0.\]
It is normal because it is completely additive.
It is semi-finite because $p_J\uparrow1$ with $\f(p_J)<\infty$ as $J\to I$, where $p_J:=\sum_{i\in J}p_i$ and $J$ runs through finite subsets of $I$.
\end{pf}


\begin{ex}
For a locally compact abelian group $G$, the corresponding f.n.s.~weight is a suitably normalized Haar measure on the Pontryagin dual group $\hat G$, called the Plancherel measure, not the Haar measure on the original group $G$.
For a locally compact non-abelian group $G$, there is no characterization of the corresponding f.n.s.~weight as a measure because the left Hilbert algebra $(C_c(G),*)$ is not commutative.
\end{ex}







\newpage
\section{December 20}


\subsection{Tomita-Takesaki commutation theorem}
1.17, 1.18, 1.20, 1.21, 1.22

\begin{thm}[Tomita-Takesaki commutation theorem]
Goal: $\Delta^{it}R_l(A)\Delta^{-it}=R_l(A)$ and $JR_l(A)J=R_l(A)'$.
1.19
\end{thm}


modular automorphism groups and Tomita algebras
centralizer, Connes cocyle

Chapter IX: standard form, unitary implementation
Chapter X: crossed product duality, W$^*$-dynamical system



\newpage
\section{January 17}

abelian group

\newpage
\section{February 9}

Type III




\end{document}