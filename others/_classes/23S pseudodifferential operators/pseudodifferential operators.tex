\documentclass{../../small}
\usepackage{../../ikhanchoi}

\begin{document}
\title{Pseudodifferential Operators}
\author{Ikhan Choi\\Lectured by Kenichi Ito\\The University of Tokyo, Spring 2023}
\maketitle
\tableofcontents

\newpage
\section{Day 1: April 11}


Notation

$D_j=(-1)\pd_j$
\[\xi^\alpha\cF u=\cF D^\alpha u,\quad\xi^\alpha\cF^* u=\cF^*(-D)^\alpha u,\quad D^\alpha\cF u=\cF(-x)^\alpha u\]

Let
\[A=\sum a_\alpha(x)D^\alpha,\quad a(x,\xi)=\sum a_\alpha(x)\xi^\alpha.\]
Then,
\begin{align*}
Au(x)=\cF^*M_{a(x,\xi)}\cF u(x)\\
&=(2\pi)^{-1}\int e^{ix\xi}a(x,\xi)\int e^{-iy\xi}u(y)\,dy\,d\xi\\
&=(2\pi)^{-1}\iint e^{i(x-y)\xi}a(x,\xi)u(y)\,dy\,d\xi.
\end{align*}
If $a$ has a polynomial growth in $\xi$, then the integrand $e^{i(x-y)\xi}a(x,\xi)u(y)$ is not integrable in $(y,\xi)$, so we need to justify it as an oscillatory integral.

\bigskip


Japanese bracket, originated by Kitada or Kumano-go (akumade setsu)
\[\<x+y\>^{-2}\le4\<x\>^2\<y\>^{-2}\]
\[\<x^2+x\>\asymp\<x^2\>\asymp\<x\>^2\]

\bigskip

Here we define the amplitude functions as
\[|\pd^\alpha a(x)|\lesssim\<x\>^{m+\delta|\alpha|},\quad\forall\alpha\in\Z_{\ge0}^d\]

\begin{ex*}[Justification of a qudaratic oscillation]
Let $Q$ be a nondegenerate real quadratic form, then for $a\in A_\delta^m(\R^d)$ and $\chi\in\cS(\R^d)$ with $\chi(0)=1$,
\[I_Q(a):=\lim_{\e\to0}\int_{\R^d}e^{i\frac12Q(x)}\chi(\e x)a(x)\,dx\]
exists.
The term $e^{i\frac12Q(x)}$ oscillates fast where $|x|\gg1$, the term $\chi(\e x)$ becomes flatten as $\e\to0$.
When we do integrate by parts, we integrate the oscillating term, and differentiate the cutoff and amplitude.
If we differentiate the amplitude, the integrability is enhanced.
\end{ex*}
\begin{pf}
We compute for $Q=I$ as an example.
Since
\[De^{i\frac12x^2}=xe^{i\frac12x^2},\]
we have
\[(1+x\cdot D)e^{i\frac12x^2}=(1+x^2)e^{i\frac12x^2}.\]
Define a differential operator $L$ such that
\[^tL:=\frac1{1+x^2}+\frac x{1+x^2}\cdot D=\<x\>^{-2}+\<x\>^{-2}x\cdot D,\]
that is,
\begin{align*}
L&=\frac1{1+x^2}-D\cdot\frac x{1+x^2}-\frac x{1+x^2}\cdot D\\
&=\frac1{1+x^2}+i(\frac d{1+x^2}-\frac{2x^2}{(1+x^2)^2})-\frac x{1+x^2}\cdot D\\
&=(1+(d+2)i)\<x\>^{-2}-2\<x\>^{-4}-\<x\>^{-2}x\cdot D.
\end{align*}
Then $^tL$ fixes $e^{i\frac12x^2}$, so
\[\int_{\R^d}e^{i\frac12x^2}\chi(\e x)a(x)\,dx=\int_{\R^d}e^{i\frac12x^2}L^k[\chi(\e x)a(x)]\,dx\]
for any $k\ge0$.
Since
\[L=c_0(x)+c_j(x)\pd_j,\qquad c_0\in A_{-1}^{-2},\quad c_j\in A_{-1}^{-1},\]
we have
\[|L^k[\chi(\e x)a(x)]|\lesssim|L^k[a(x)]| \lesssim|a|_k\<x\>^{m-\min\{1-\delta,2\}k}\]
bounded $\e>0$.


Then, 
\[|L^k[\chi(\e x)a(x)]|\xrightarrow{\e\to0} L^k[a(x)]\quad\text{pointwise}.\]


\end{pf}


\section{Day 2: April 18}
\begin{itemize}
\item Lemma 1.3:
Coordinate changes and integration by parts work.
Also we can check even if we do coordiante change and differeniation (of oscillating term), we also have oscillatory integral.
\item Theorem 1.4:
For amplitude functions with $\delta<1$, an operator \emph{defined} by the multiplier
\[e^{i\frac12Q(D)}:\cS'\to\cS'\]
have an explicit expression.
\item Theorem 1.5:
The above multiplier also can be defined by the Taylor expansion.
This kind of theorems may be called a expansion formula (I think).
The last part of the proof holds from the integral by parts.
\item Corollary 1.7:
We want to have an extension with a parameter.
The parameter $h$ is called s semiclassical parameter.
As $h\to0$, the oscillation goes rapid.
The name stationary phase is implied by the origin zero is the only critical points of the phase function.
\item Lemma 1.6:
Here we introduce a sequence of Schwarz functions which converges in $\cS'$
\[e^{-\e x^2}e^{i\frac12Q(x)}\xrightarrow{\e\to0}e^{i\frac12Q(x)}.\]
Between the second row and the third row in the aligned equations, we have used
\[\cF(e^{i\frac12Q(x)}e^{-\e x({}^tP^{-1}P^{-1})x})(P^{-1}\eta)=(2\pi)^{-d/2}\int_{\R^d}e^{-ix\cdot P^{-1}\eta}e^{i\frac12Q(x)}e^{-\e x({}^tP^{-1}P^{-1})x}\,dx.\]
\end{itemize}

\newpage
\section{Day 3: April 25}

\begin{itemize}
\item $x$ 쪽의 디케이는 신경쓰지 않음! 위의 진폭함수와 다르게 델타가 x 편미분에 의한 xi 디케이를 보는 패러미터임에 주의 따라서 거의 델타를 0으로 두는 것.
\item x에 대해 컴팩트 셋을 잡아서 심볼 클래스를 정의한 것에 대하여: 아마 tempered distribution 도입 없이 슈바르츠 함수만으로 하려고 한 듯. 그러지 않아도 컷오프를 곱해서 tempered distribution sense에서 근사시켜 생각하면 됨.
\item Some references only use $S^m=S_{1,0}^m$. % 반대로 Kohn-Nirenberg calculus보다 훨씬 일반적인 심볼 클래스들을 생각할 수야 있겠지만 그건 좀 에바임
\item Remakr 1. % 변수변환이 안 되거나 하는 상황이 생기거나 함
\item homogeneous polynomial에 대한 euler relation(xi로 미분해서 xi편도함수의 동차성을 확인가능), a의 도메인에 대하여 xi를 구면상을 제한시켜 생각할 수 있음
\item smoothing operator는 사실상 제로와 같이 버려도 되는 operator라는 느낌
\item Theorem 2.1. 증명 첫줄
\[\<\xi\>^{-2}\<D_y\>^2e^{i(x-y)\xi}=e^{i(x-y)\xi}\]
\item Theorem 2.2. 전개식의 각 항을 먼저 주고 전체 심볼을 컨스트럭트할 수 있다는 것. rho와 delta는 고정하고 m만 신경쓰자. Borel 정리(거의 안쓰임)와 증명 및 의미를 비교해보자.
\item 유한에서 적당히 스무스하다 가정하고 무한에서의 디케이만 보는 개념이다
\end{itemize}



\newpage
\section{Day 4: May 2}



For a fixed cutoff $\chi\in\cS$,
\[\partial^\alpha(\chi(\e x)a(x))
=\begin{cases}
\partial^\alpha a(x),&\text{ if }|x|<\e^{-1}\\
\partial^\alpha a(x)+O_\alpha(\e),&\text{ if }\e^{-1}\le|x|<2\e^{-1}\\
0,&\text{ if }|x|\ge 2\e^{-1}
\end{cases}.\]
and
\[\partial^\alpha([1-\chi(\e x)]a(x))
=\begin{cases}
0,&\text{ if }|x|<\e^{-1}\\
\partial^\alpha a(x)+O_\alpha(\e),&\text{ if }\e^{-1}\le|x|<2\e^{-1}\\
\partial^\alpha a(x),&\text{ if }|x|\ge 2\e^{-1}
\end{cases}.\]
We have if
\[|\partial_\xi^\beta a(\xi)|\lesssim\<\xi\>^{m-\rho|\beta|},\]
then
\[|\partial_\xi^\beta(\chi(\e\xi)a(\xi))|\lesssim\<\xi\>^{m-\rho|\beta|}+\e\quad\text{ with }\quad\<\xi\>\lesssim\e^{-1},\]
and
\[|\partial_\xi^\beta([1-\chi(\e\xi)]a(\xi))|\lesssim\<\xi\>^{m-\rho|\beta|}+\e\quad\text{ with }\quad\e^{-1}\lesssim\<\xi\>,\]

\begin{pf}
Suppose we have a sequence $(a_j)_{j=0}^\infty$ of functions such that $a_j\in S_{\rho,\delta}^{m_j}(\R^{2d})$ with $m_j\downarrow-\infty$.
Then, there exists $a\in S_{\rho,\delta}^{m_0}$ such that the asymptotic expansion formula holds:
\[a-\sum_{j=0}^{k-1}a_j\in S_{\rho,\delta}^{m_k}.\]
In addition, we have a uniqueness result and a support description.

Note that
\[a(x,\xi)=\sum_{j=0}^\infty a_j(x,\xi)\]
will not converges, so we introduce a cutoff function $\chi$ to define
\[a(x,\xi):=\sum_{j=0}^\infty[1-\chi(\e_j\xi)]a_j(x,\xi)\]
with $\e_j\downarrow0$ so that the summation is locally finite.
Then, the error of the expansion formula is decomposed into
\[a-\sum_{j=0}^{k-1}a_j=-\sum_{j=0}^{k-1}\chi(\e_j\xi)a_j(x,\xi)+\sum_{j=k}^\infty[1-\chi(\e_j\xi)]a_j(x,\xi).\]

\[|\partial_x^\alpha\partial_\xi^\beta([1-\chi(\e_j\xi)]a_j(x,\xi))|\lesssim\<\xi\>^{m+\delta|\alpha|-\rho|\beta|}+\e_j\quad\text{ with }\quad\e_j^{-1}\lesssim\<\xi\>\]
The first term vanishes 


\end{pf}
\begin{itemize}
\item formal adjoint: 내 이해 방식으로는 transpose 라 하는 게 좋을 거 같다
\item 다이아몬드 보일 때 편미분이 적분 안으로 들어가는 거는 진동적분 이용해서 진폭과 그 도함수가 적분가능이 될 만큼 미분해 바꾼 다음 편미분을 집어넣는 것
\item 네 개의 컷오프로 쪼개는 것에 대해 y와 eta가 도메인, xi는 패러미터라 보자. $(y,\eta)=0$에서 stationary이므로 $\chi_1$에서 유의미한 기여가 있을 것이고 나머지 영역에서는 0으로 보낼 예정.
\end{itemize}



\end{document}