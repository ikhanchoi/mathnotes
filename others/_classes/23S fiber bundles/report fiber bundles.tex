\documentclass{../../../small}
\usepackage{../../../ikhanchoi}

\begin{document}

\title{Fiber Bundles and Characteristic Classes}
\author{Ikhan Choi}
\maketitle


\section{Topology of hypersurfaces of $\CP^3$}
Throughout this section, let $X$ be a smooth projective hypersurface embedded in $\CP^3$ of degree $d$.
Our goal is to compute topological invariants of $X$, including the cohomology groups and the intersection form.

Here we assume many facts on complex and algebraic geometry and focus on the actual computation of characteristic classes, for example,
\begin{itemize}
\item the Lefschetz hyperplane theorem,
\item the Veronese embedding,
\item the adjunction formula,
\item the Hirzeburch signature theorem, etc.
\end{itemize}
We also have some remarks on compact complex surfaces:
\begin{itemize} 
\item Although not every compact complex surface is algebraic, many important examples include algebraic surfaces.
\item The Enriques-Kodaira classification describes the big forest on the main problems regarding compact complex surfaces.
\item If a compact complex surface is algebraic, then it is birational to a complex projective surface.
\item Every complex projective surface is embedded in $\CP^5$.
\item The topology of hypersurfaces of $\CP^3$ uniquely determined by the degree of defining polynomial by the Ehresman theorem.
\end{itemize}
We will always consider smooth manifolds and smooth varieties.

An \emph{intersection form} is a bilinear form $Q:H^2(X)\times H^2(X)\to\Z$ defined by $Q(\alpha,\beta):=(\alpha\smile\beta)[X]$.
In our setting of complex surfaces, the intersection form fully describes the ring structure of the cohomology $H^*(X)$.
More generally, a \emph{lattice} is a finitely generated free abelian group together with a symmetric bilinear form, and a lattice is called \emph{integral} if the values of the bilinear form is always an integer.
An integral lattice is called \emph{unimodular} if the unit volume, more precisely the absolute value of the determinant of the Gram matrix constructed from any basis of the lattice, is one.

Consider the classification problems of compact oriented simply connected 4-manifolds.
An intersection form is always a unimodular lattice, and conversely, by Freedman, any unimodular lattice is given in this way.
In fact, by the Milnor-Whitehead, the topology of such manifolds are uniquely determined up to oriented homotopy by the intersection form.
The Kirby-Seibermann invariant is the only required invariant to classify further in the sense of homeomorphism classes only when the intersection form is odd.
Henceforth, we are concerned with the classification of unimodular lattices.

We have two significant invariants for unimodular lattice $Q$:
\begin{enumerate}[(1)]
\item Signature: an integer $\sigma$ called the \emph{signature} is defined by the signature of the realified symmetric bilinear form $Q\otimes_\Z\R$ in the Sylvester law of inertia.
When we write $n_{\pm}$ to denote the number of positive/negative eigenvalues, then the rank is $b_2=n_++n_-$ and the signature is $\sigma:=n_+-n_-$.
\item Parity: we say $Q$ is \emph{even} if $Q(\alpha,\alpha)$ is always even, and \emph{odd} if not.
\end{enumerate}
If $Q$ is indefinite, i.e.~$|\sigma|<b_2$, then these two invariants form a complete set of invariants of unimodular lattices.
We will show later that in our case of $X$ the intersection form is always indefinite.

\subsection{Euler class}
We first see the relation of the Euler characteristic and the cohomology group $H^k(X)$.
If $e(X)\in H^4(X)$ is the Euler class, then the Euler characteristic is equal to the evaluation of the fundamental class $\chi(X)=e[X]$.

\begin{thm}[Lefchetz hyperplane theorem]
Let $Y$ be an $n$-dimensional complex projective variety in $\CP^m$, and $H$ a hyperplane of $\CP^m$.
If $Y\setminus H$ is smooth, then we have
\begin{parts}
\item $H_k(Y\cap H,\Z)\to H_k(Y,\Z)$ is an isomorphism for $k\le n-2$ and surjective for $k=n-1$,
\item $H^k(Y,\Z)\to H^k(Y\cap H,\Z)$ is an isomorphism for $k\le n-2$ and surjective for $k=n-1$.
\end{parts}
\end{thm}

If we consider the Veronese embedding $\CP^3\to\CP^m$, where $m={d+3\choose d}-1$, then $X$ is a smooth hyperplane section in $\CP^m$ with $\CP^3$.
Note that we can compute the following table
\[\begin{array}{c|ccccc}
n&0&1&2&3&4\\\hline
H_n(\CP^3)&\Z&0&\Z&0&\Z\\
H^n(\CP^3)&\Z&0&\Z&0&\Z
\end{array}\]
by applying the Gysin sequence.
Then we obtain
\[\begin{array}{c|ccccc}
n&0&1&2&3&4\\\hline
H_n(X)&\Z&0&?&0&\Z\\
H^n(X)&\Z&0&?&0&\Z
\end{array}\]
by the Lefschetz hyperplane theorem and the Poincar\'e duality.
By the universal coefficient theorem, we have a short exact sequence of abelian groups
\[0\to\Ext^1(H_1(X),\Z)\to H^2(X)\to\Hom(H_2(X),\Z)\to0,\]
which implies the second cohomology group $H^2(X)\cong\Hom(H_2(X),\Z)$ is torsion-free since $\Z$ is torsion-free.
Therefore, by the definition of the Betti number we have
\[H^2(X)\cong H_2(X)\cong\Z^{b_2(X)}.\]
Since
\[\chi(X)=b_0(X)-b_1(X)+b_2(X)-b_3(X)+b_4(X)=2+b_2(X),\]
the cohomology group is completely determined if we compute the Euler characteristic.


\subsection{Chern class}

The Chern class itself gives an important invariant of $X$, but also is used in computations of other characteristic classes.
We first see the Chern class of the projective space $\CP^3$.
The Euler sequence reads a short exact sequence of complex vector bundles
\[0\to\e^1\to(\bar\gamma^1)^4\to\tau^3\to0\]
over the projective space $\CP^3$, which gives the Chern class of the tangent bundle
\[c(T\CP^3)=\frac{c(\bar\gamma^1)^4}{c(\e^1)}=\frac{(1+c_1(\bar\gamma^1))^4}{1}=(1+x)^4\]
in $H^*(\CP^3)\cong\Z[x]/x^4$, where $x:=c_1(\bar\gamma^1)\in H^2(\CP^3)$ is the Poincar\'e dual of a generic hyperplane of $\CP^3$, because a section of $\bar\gamma^1=\cO(1)$ is given by a linear homogeneous polynomial.

To see the hypersurface $X$, consider the short exact sequence of complex vector bundles
\[0\to TX\to T\CP^3|_X\to N_{\CP^3/X}\to0,\]
where $N_{\CP^3/X}$ denotes the normal bundle.
Note that the normal bundle of the hypersurface of degree $d$ is the restriction of the line bundle $\cO(d)$ to $X$ as in the adjunction formula.
If we use the same notation $x:=i^*x$, where $i:X\to\CP^3$ is the inclusion, then the total Chern class is computed as
\begin{align*}
c(TX)&=\frac{c(T\CP^3|_X)}{c(\cO(d)|_X)}=\frac{i^*c(\CP^3)}{i^*c(\cO(d))}=\frac{(1+x)^4}{1+dx}\\
&=(1+x)^4(1-dx+(dx)^2)\\
&=1+(4-d)x+(d^2-4d+6)x^2
\end{align*}
in $H^*(X)$, in which $x^3=0$.
Therefore, since $x^2\in H^4(\CP^3)$ is the Poincar\'e dual of the intersection $\CP^1\subset\CP^3$ of two generic hyperplanes, we have
\[x^2[X]=\CP^1\cdot X=d\]
if we write $[X]$ for the fundamental class of $X$.
Thus we have
\begin{align*}
e(X)=c_2(X)&=(d^2-4d+6)x^2,\\
\chi(X)=e[X]&=d^3-4d^2+6d,\\
b_2(X)&=d^3-4d^2+6d-2.
\end{align*}

\subsection{Pontryagin class}

Now we prove the intersection form is indefinite by seeing the Pontryagin class.
We will apply the following theorem.
\begin{thm}[Hirzeburch signature theorem]
Let $X$ be a closed oriented 4-manifold.
Then, $\sigma(X)=p_1[X]/3$.
\end{thm}
The Pontryagin class can be computed by
\begin{align*}
1-p_1&=(1-c_1+c_2)(1+c_1+c_2)\\
&=1+2c_2-c_1^2\\
&=1+2(d^2-4d+6)x^2-((4-d)x)^2\\
&=1+(d^2-4)x^2.
\end{align*}
Therefore, we have
\[\sigma(X)=\frac{(4-d^2)x^2[X]}3=-\frac{(d-2)d(d+2)}3,\]
which implies $|\sigma(X)|<b_2(X)$ for $d\ge2$.

\subsection{Stiefel-Whitney and Wu classes}

If we tensor $\Z/2\Z$ on the cohomology groups to see the parity, then the quadratic form induced from the intersection form boils down to a cohomology operation called the \emph{Steenrod square}
\[\operatorname{Sq}^2:H^2(X,\Z/2\Z)\to H^4(X,\Z/2\Z):\alpha\mapsto\alpha\smile\alpha.\]
Since $\operatorname{Sq}^2$ is a $\Z/2\Z$-linear functional on the $\Z/2\Z$-linear space $H^2(X,\F^2)$ when we see $\Z/2\Z$ as a finite field of two elements, we have an element $\nu_2\in H^2(X,\Z/2\Z)$ such that $\operatorname{Sq}^2(\alpha)=\nu_2\smile\alpha$.
The element $\nu_2$ is the second Wu class, and has a relation to the Stiefel-Whitney class $w=\operatorname{Sq}(\nu)$, called the Wu formula.
We can also write as
\[\nu_1=v_1,\qquad \nu_2=w_1^2+w_2,\qquad \nu_3=w_1w_2,\qquad \nu_4=w_4+w_1w_3+w_2^2+w_1^4,\cdots.\]
Since $H^1(X)\cong0$ implies $\nu_2(X)=w_2(X)$, the intersection form is even if and only if $w_2(X)=0$.

The computation of the second Stiefel-Whitney class of a complex vector bundle follows from the first Chern class.
Since $c_1$ generates $H^2(BU(1),\Z)\cong\Z[c_1]$ and $w_2$ generates $H^2(BSO(2),\Z/2\Z)\cong\Z/2\Z[w_2]$, we have $w_2\equiv c_1$ modulo two.
The generalization for complex vector bundles of higher rank, we may apply the splitting principle, but we will skip.
Since $c_1(X)=(4-d)x=dx$ in $H^2(X,\Z/2\Z)$, $w_2(X)$ vanishes if and only if $d$ is even.


\bigskip

To summarize, we have
\[\begin{array}{c|cccc}
d&b_2&\sigma&\text{parity}&X\\\hline
1&1&1&\text{odd}&\CP^2\\
2&2&0&\text{even}&\CP^1\times\CP^1\\
3&7&-5&\text{odd}&\text{rational}\\
4&22&-16&\text{even}&\text{K3}\\
5&53&-35&\text{even}&\text{general type}
\end{array}\]

\bigskip

We mainly refer to the blog post [https://qchu.wordpress.com/2014/06/16/hypersurfaces-4-manifolds-and-characteristic-classes/].
Other references we have refered to include [Milnor-Stasheff], [Griffiths-Harris], [May].



\end{document}