\documentclass{../../../small}
\usepackage{../../../ikhanchoi}

\newcommand{\GrAut}{\operatorname{GrAut}}
\newcommand{\gldim}{\operatorname{gldim}}
\renewcommand{\pd}{\operatorname{pd}}
\newcommand{\GrMod}{\operatorname{GrMod}}
\newcommand{\Tors}{\operatorname{Tors}}
\newcommand{\Tails}{\operatorname{Tails}}
\newcommand{\Proj}{\operatorname{Proj}}
\newcommand{\grmod}{\operatorname{grmod}}
\newcommand{\tors}{\operatorname{tors}}
\newcommand{\tails}{\operatorname{tails}}
\newcommand{\Pic}{\operatorname{Pic}}
\begin{document}
\title{Noncommutative Algebraic Geometry}
\author{Ikhan Choi\\Lectured by Izuru Mori\\University of Tokyo, Autumn 2023}
\maketitle

\section{Algebras}



\begin{itemize}
\item 1987: Artin-Schelter, regular algebra.
\item 1990: Artin-Tate-Bergh, three dimensional, geometrically classified.
\item 1994: Artin-Zhang, noncommutative scheme, categorical perspective.
\end{itemize}

\subsection{}
Let $k$ be an algebraically closed field of characteristic zero.
Examples of $k$-algebras include the free algebra $T:=k\<x_1,\cdots,x_n\>$, which is noncommutative for $n\ge2$.
It consists of linear combinations of monomials, and there are $2^n$ monomials of degree $n$ in $T$, and $T$ is $k$-isomorphic to the tensor algebra constructed from $n$-dimensional vector space $k^n$.
Note that $(x+y)^2=x^2+xy+yx+y^2$ in $T$.
An algebra $R$ is finitely generated if and only if $R\cong T/I$ for some $n$ and some ideal $I$ of $R$.
If $n\ge2$, then $T$ is not right noetherian, $I=\sum_{i=0}^\infty x^iyR$ is a right ideal which is not finitely generated for exmaple(not easy to show finitely generatedness).
Is $k\<x,y\>/(yx,y^2)$ noetherian?
It is known that it is left notherian, but not right noetherian.



\subsection{}
Let $R$ be a ring and let $\sigma\in\Aut(R)$.
An additive map $\delta:R\to R$ is called a \emph{$\sigma$-derivation} if $\delta(ab)=\sigma(a)\delta(b)+\delta(a)b$ for $a,b\in R$.
We define a ring $R[x;\sigma,\delta]$, called the \emph{Ore extension}, as an additive group $R[x]$ together with multiplication defined by
\[xa:=\sigma(a)x+\delta(a),\qquad a\in R.\]
\begin{ex}\,
\begin{parts}
\item We can compute
\begin{align*}
(ax+b)(cx+d)
&=axcx+axd+bcx+bd\\
&=a(\sigma(c)x+\delta(c))x+a(\sigma(d)x+\delta(d))+bcx+bd\\
&=a\sigma(c)x^2+(a\delta(c)+a\sigma(d)+bc)x+a\delta(d)+bd.
\end{align*}
\item We have $R[x;\id_R,0]\cong R[x]$ as rings.
\item If $\sigma(f(x)):=f(\alpha x)$ for some non-zero $\alpha\in k$, then $k[x][y;\sigma,0]\cong k\<x,y\>/(\alpha xy-yx)$ since $yx=\sigma(x)y-\delta(x)=\alpha xy$.
\item If $\delta(f(x)):=f'(x)$, then $k[x][y;\id_{k[x]},\delta]\cong k\<x,y\>/(xy-yx+1)$, called the \emph{Weyl algebra}, since $yx=\sigma(x)y+\delta(x)=xy+1$.
\item How can we find a $k$-automorphism $\sigma$ of $k[x]$ and a $\sigma$-derivation $\delta$ such that $k\<x,y\>/(xy-yx+x^2)\cong k[x][y;\sigma,\delta]$? What should $\delta(x^i)$ be? One answer is $\sigma=\id_{k[x]}$ and $\delta(f(x))=x^2f'(x)$.
\end{parts}
\end{ex}
\begin{thm}
Let $R$ be a ring and $S:=R[x;\sigma,\delta]$ be an Ore extension.
\begin{parts}
\item If $R$ is right noetherian, then so is $S$.
\item If $R$ is a domain, then so is $S$.
\item If $R$ is of finite global dimension, then so is $S$.
\end{parts}
\end{thm}
As examples, we have $k\<x,y\>/(\alpha xy-yx)$ and $\dim k\<x,y\>/(xy-yx+1)$ are noetherian domains of global dimensions 2 and 1, respectively.
There is a result that left and right global dimensions coincide when $R$ is two-sided noetherian.





\subsection{}
\begin{thm}
If $R$ is a $k$-algebra and $a_1,\cdots,a_n\in R$, then there is a unique $k$-algebra homomorphism $\f:k\<x_1,\cdots,x_n\>\to R$ such that $\f(x_i)=a_i$.
If a $k$-algebra homomorphism $\f:S\to R$ satisfies $\f(I)=0$ for an ideal $I$ of $S$, then it factors through $S/I$.
\end{thm}
With the above theorem we can construct an $k$-algebra isomorphism $k[x]\cong k\<x,y\>/(x^2-y)$.
As an another example, for $\char k\ne2$, then
\[k\<x,y\>/(x^2+y^2,xy+yx)=k\<x+y,x-y\>/((x+y)^2,(x-y)^2)\cong k\<x,y\>/(x^2+y^2).\]








\subsection{}
We now consider grading, a direct sum decomposition over a monoid.
The free $k$-algebra $T=k\<x_1,\cdots,x_n\>$ is $\N$-graded by degree.
Let $A=\bigoplus A_i$ be a graded ring.
We can define homogeneous ideals of $A$, and the quotient can be written as $A/I\cong\bigoplus A_i/I_i$, where $I_i:=I\cap A_i$.
Also, graded homomorphisms between graded rings or graded modules are able to be introduced.
Let $I$ and $J$ be homogeneous ideal of $T_n:=k\<x_1,\cdots,x_n\>$ and $T_m:=k\<y_1,\cdots,y_m\>$ such that $J_0=J_1=0$.
Then, a graded algebra homomorphism $\f:T_n\to T_m$ is uniquely determined by $\f(x_i)=a_{ij}y_j$ for $(a_{ij})\in M_{nm}(k)$.
Let $\GrAut(A)$ be the group of graded algebra automorphisms of $A$.
Then,
\[\GrAut(T_n)\cong\GrAut(k[x_1,\cdots,x_n])\cong\GL(n,k),\]
and if $I$ is a homogeneous ideal of $T_n$ such that $I_0=I_1=0$, then $\GrAut(T_n/I)$ is a subgroup of $\GL(n,k)$.
For example, we have
\[\GrAut(k\<x,y\>/(x^2))\cong\left\{\mat{a&0\\c&d}:a,d\in k^\times\right\}\]
and for $\alpha\ne\pm1$ we have
\[\GrAut(k\<x,y\>/(\alpha xy-yx))\cong\left\{\mat{a&0\\0&d}:a,d\in k^\times\right\}\]
since $\alpha\f(x)\f(y)-\f(y)\f(x)=(\alpha-1)(acx^2+bdy^2)+(\alpha^2-1)bcxy$.

Fix $\theta\in\GrAut(A)$.
Define an algebra $A^\theta:=A$ as sets and multiplication $a*b:=a\theta^i(b)$ on $A^\theta$ for $a\in A_i$ and $b\in A$.
It is called the \emph{twist} of $A$ by $\theta$, and it is also graded.
For example, if we let $A=k[x,y]$, then
\[\text{If }\theta=\mat{\alpha&0\\0&1},\text{ then }\quad A^\theta\cong k\<x,y\>/(\alpha xy-yx)\]
and
\[\text{If }\theta=\mat{1&0\\-1&1},\text{ then }\quad A^\theta\cong k\<x,y\>/(xy-yx+x^2).\]
Note that $\f(xy-yx)=(ad-bc)(xy-yx)$ if $\theta=\mat{a&b\\c&d}$.

\begin{thm}
Let $A$ be a graded ring and $\theta\in\GrAut(A)$.
\begin{parts}
\item If $A$ is right noetherian, then so is $A^\theta$.
\item If $A$ is a domain, then so is $A^\theta$.
\item If $A$ is of finite global dimension, then so is $A^\theta$.
\end{parts}
\end{thm}





\newpage
\section{Quantum polynomial algebras}

\subsection{}

Today, let $A:=k\<x_1,\cdots,x_n\>/I$ be a finitely generated graded algebra such that $I$ is a homogeneous ideal satisfying $I_0=I_1=0$, i.e.~$I$ is an admissible ideal.
Let $M$ be a graded right $A$-module, $M_{\ge n}:=\bigoplus_{i\ge n}M_i$ be a graded submodule of $M$, and $M(n)$ be a graded module such that $M(n):=M$ as a set but $M(n)_i:=M_{n+i}$.
With this notation, $\fm:=A_{\ge1}$ is the unique maximal homogeneous ideal of $A$.
A \emph{free} graded right $A$-module is a graded right $A$-module of the form $\oplus_s A(n_s)$.
A finitely generated graded right $A$-module is free if and only if projective. 
A function $\f:A(l)\to A(m)$ is a graded right $A$-module homomorphism if and only if $\f=a\cdot$ for some $a\in A_{m-l}$.
Therefore, between free right $A$-modules, $\f:\oplus A(l_s)\to\oplus A(m_t)$ is a graded right $A$-module homomorphism if and only if $\f=(a_{st})\cdot$, for some $a_{st}\in A_{m_t-l_s}$.
A free resolution
\[\cdots\to F^2\to F^1\to F^0\to M\to0\]
is called \emph{minimal} if the map $\f_i:F^i\to F^{i-1}$ is given by the left multiplication of a matrix whose entries are in $A_1$.
We can define the projective dimension of a module as the minimal length of free resolution, and the global dimension of $A$ as the supremum of the projective dimension of graded right $A$-modules.
\begin{lem}$\gldim A=\pd(k)$.\end{lem}
For example, $A=k\<x,y\>$, then $k=A/(xA+yA)$, so $\pd(k)=1$, hence $\gldim A=1$, and in generally $\gldim A=1$ for $I=0$.


\subsection{}
Let $M$ be a finitely generated graded right $A$-module.
Suppose further $M$ is locally finite, i.e.~$\dim_kM_i<\infty$ for each $i$.
Then,
\[H_M(t):=\sum_{i\in\Z}(\dim_kM_i)t^i\in\Z[[t,t^{-1}]]\]
is called the \emph{Hilbert series} of $M$.
For example, letting $M=A$,
\[H_{k[x_1,\cdots,x_n]}(t)=\sum_{i=0}^\infty{n+i-1 \choose n-1}t^i=(1-t)^{-n},\]
and
\[H_{k\<x_1,\cdots,x_n\>}(t)=\sum_{i=0}^\infty n^it^i=(1-nt)^{-1}.\]
\begin{lem}
Let $M$ be a finitely generated graded right $A$-module.
\begin{parts}
\item $H_{M^{\oplus r}}(t)=rH_M(t)$.
\item $H_{M(n)}(t)=t^{-n}H_M(t)$.
\item If $0\to M^r\to\cdots\to M^1\to M^0\to0$ is exact, then $\sum_{i=0}^r(-1)^iH_{M_i}(t)=0$.
\end{parts}
\end{lem}
For example for (c), consider
\[0\to A(-1)^{\oplus2}\to A\to k\to0.\]
Then, we can check $H_A(t)=(1-2t)^{-1}$ from
\[0=H_k(t)-H_A(t)+H_{A(-1)^{\oplus2}}(t)=1-H_A(t)+2tH_A(t).\]



\subsection{}
\begin{defn}[Artin-Schelter]
We say $A$ is a $d$-dimensional \emph{quantum polynomial algebra}(QPA) if $\gldim A=d<\infty$, $H_A(t)=(1-t)^{-d}$, and $\Ext_A^i(k,A)=\delta_{di}\cdot k(d)$.
The last condition is called the \emph{Gorenstein condition}.
\end{defn}
If a QPA is commutative, then it is isomorphic to the polynomial algebra.
The above two conditions are equivalent to have the minimal free resolution of the graded right $A$-module $k$
\[0\to A(-d)\to\oplus A(-d+1)\to\cdots\to\oplus A(-1)\to A\to k\to0,\]
where $\phi^i:\oplus A(-i)\to\oplus A(-i+1)$ is the left multiplication of a matrix whose components are in $A_1$.
The Gorenstein condition is equivalent to the transpose
\[0\leftarrow k(d)\leftarrow\oplus A(d)\leftarrow\cdots\leftarrow\oplus A(1)\leftarrow A\leftarrow0\]
is a minimal free resolution of left $A$-module $k(d)$, where the arrows are right multiplications of matrices whose compoenents are in $A_1$.
Ranks of each free modules must be determined by the Hilbert series.

For example, $A=k\<x,y\>/(\alpha xy-yx)$ is a 2-dimensional QPA for all non-zero $\alpha\in k$.
The classification up to dimension two is easy:
\begin{lem}
Let $A$ be a QPA over an algebraically closed field $k$.
\begin{parts}
\item $\gldim A=0$ iff $A\cong k$,
\item $\gldim A=1$ iff $A\cong k[x]$,
\item $\gldim A=2$ iff $A\cong k[x,y]^\theta$ for some $\theta\in\GL(2,k)$.
\end{parts}
\end{lem}


\subsection{}
We can describe three-dimensional QPAs are classified in terms of derivation quotient algebras.
\begin{defn}
Let $V=k^n$ and let
\[\f:V^{\otimes m}\to V^{\otimes m}:v_1\otimes\cdots\otimes v_m\mapsto v_2\otimes\cdots\otimes v_1.\]
We say $w\in V^{\otimes m}$ is called a \emph{superpotential}(SP) if $\f(w)=w$, and a \emph{twisted superpotential}(TSP) if $(\sigma\otimes\id^{\otimes(m-1)})\f(w)=w$ for all $\sigma\in\GL(V)$.
\end{defn}
\begin{ex}
Let $V=kx+ky$, and $w=\alpha x^2+\beta xy+\gamma yx+\delta y^2\in V^{\otimes 2}$.
Then, $w$ is SP iff $\beta=\gamma$ and $SP^2(V)=kx^2+k(xy+yx)+ky^2\subset V^{\otimes2}$.
\end{ex}
\begin{defn}
For $\dim_kV=n$ and $w\in V^{\otimes m}$, we can define $\partial_iw,w\partial_i\in V^{\otimes(m-1)}$ such that $w=\sum x_i\otimes\partial_iw=\sum w\partial_i\otimes x_i$.
\emph{Derivation quotient algebras} are
\[D_l(w):=k\<x_1,\cdots,x_n\>/(\partial_1w,\cdots,\partial_nw),\qquad D_r(w):=k\<x_1,\cdots,x_n\>/(w\partial_1,\cdots,w\partial_n).\]
\end{defn}
\begin{lem}
\,
\begin{parts}
\item $w$ is SP iff $\partial_iw=w\partial_i$.
\item $w$ is TSP iff $D_l(w)=D_r(w)=:D(w)$ (ideals quotiented are same as sets.)
\end{parts}
\end{lem}
\begin{ex}
If $V=kx+ky$, and $w=\alpha x^2+\beta xy+\gamma yx+\delta y^2\in V^{\otimes 2}$, then
\[\partial_xw=\alpha x+\beta y,\qquad w\partial_x=\alpha x+\gamma y.\]
\end{ex}

\begin{thm}\,
\begin{parts}
\item If $\omega$ is TSP with $m=n=3$, then $D(w)$ is a three-dimensional QPA.
\item The converse holds.
\end{parts}
\end{thm}


\begin{ex}[Sklyanin algebra]
For $\alpha,\beta,\gamma\in k$,
\[w=\alpha(xyz+yzx+zxy)+\beta(xzy+yxz+zyx)+\gamma(x^3+y^3+z^3)\]
is a superpotential.
$D(w)$ is called the Sklyanin algebra.
We can construct with $M=\mat{\gamma x&\beta z&\alpha y\\\alpha z&\gamma y&\beta x\\\beta y&\alpha x&\gamma z}$ the minimal free resolutions of $k$ and $k(3)$.
\end{ex}


There is $\theta\in\GrAut(k\<x,y\>/(\alpha xy-yx))$ such that
\[(k\<x,y\>/(\alpha xy-yx))^\theta\cong k\<x,y\>/(xy-yx+x^2)\]
if and only if $\alpha=1$.
We can see this for $\alpha=-1$ by computing $\GrAut$.
Note that
\[(k\<x,y\>/(\alpha xy-yx))^\theta\cong k\<x,y\>/(\alpha\theta(x)y-\theta(y)x)\]
If $\alpha\ne\pm1$....?


\newpage
\section{}

Artin-Tate-van den Bergh classification of 3-dimensional QPA.

Point varieties.


\newpage
\section{}

\subsection{}

\begin{defn}
A \emph{noncommutative scheme} is a pair $X=(\Mod X,\cO_X)$ of an abelian category $\Mod X$ and an object $\cO_X$ in it.
A \emph{morphism} between noncommutative schemes $X$ and $Y$ is an adjoint of pair of functors $f_*:\Mod X\to\Mod Y$ and $f^*:\Mod Y\to\Mod X$ such that $f^*\cO_Y=\cO_X$.
\end{defn}

For a scheme $X$, then $X$ can be considered as a noncommutative scheme by the pair of the category of quasi-coherent sheaves and the structure sheaf.

Consider the noncommutative affine schemes.
For a ring $R$, we define its noncommutative spectrum as $\Spec_{nc}R:=(\Mod R,R)$.
Note that for a ring homomorphism $\f:R\to S$, $S$ can be seen as $R$-$S$-bimodule.
Here the morphism $f:\Spec_{nc}R\to\Spec_{nc}S$ can be given as the pair of
\[f^*:\Mod R\to\Mod S:M\mapsto M\otimes_RS,\qquad f_*:\Mod S\to\Mod R:N\mapsto\Hom_S(S,N),\]
which we can check they are adjoint and $f^*S=R$.
In general, an equivalence of the category of modules does not imply the isomorphism between rings.
However, if two noncommutative schemes are isomorphic by $f:\Spec_{nc}R\to\Spec_{nc}S$, then
\[R=\End_R(R)\cong\End_S(f^*(R))=\End_S(S)=S.\]

\subsection{}
For the rest of today, a graded ring is an $\N$-graded ring.
Let $A$ be a graded ring.
Let $\GrMod A$ be the category of graded right $A$-modules.
A graded right $A$-module $M$ is called a torsion module if $mA$ is right bounded for all $m\in M$.
We define $\Tails A:=\GrMod A/\Tors A$.
For $M\in\GrMod A$, we will use somtimes $\cM=\pi(M)\in\Tails A$ to denote the image of $M$ under the projection functor $\pi:\GrMod A\to\Tails A$.

\begin{thm}[Serre]
Let $A$ be a commutative graded ring finitely generated in degree one.
Then, $\Tails A\cong\Mod(\Proj A):\pi(A)\mapsto\cO_{\Proj A}$.
\end{thm}
\begin{defn}
Let $A$ be a graded ring.
We define $\Proj_{nc}(A):=(\Tails A,\pi(A))$.
For $A$ an $n$-dimensional QPA, $\Proj_{nc}A$ is called the quantum $\P^{n-1}$.
\end{defn}

If $A$ is right noetherian graded ring, then we define $\grmod A$ as teh category of finitely generated modules.
It is an abelian category.
We may also define $\tors(A):=\Tors(A)\cap\grmod(A)$ and $\tails:=\grmod/\tors$.

\begin{thm}
Let $A$ be a connected graded right coherent algebra.
Then,
\[\Hom_{\tails A}(\cM,\cN)\cong\lim_{n\to\infty}\Hom_A(M_{\ge n},N).\]
In particular, for \emph{finitely generated} $M$ and $N$, $\pi(M)\cong\pi(N)$ if there is $n$ such that $M_{\ge n}\cong N_{\ge n}$.
\end{thm}


\subsection{}
Morita theory asks $\Mod R\cong\Mod R'$, and Artin-Zhang theory asks $\Mod A\cong\Mod A'$.
\begin{thm}
An abelian category $\cC$ is equivalent to $\Mod R$ if there is $\cO\in\Mod R$ such that
\begin{enumerate}[(i)]
\item $\Hom_\cC(\cO,-)$ preserves small coproducts,
\item every $M\in\cC$ admits an epi $\oplus\cO\to M$,
\item for epi $M\to N$, $\Hom_\cC(\cO,M)\to\Hom_\cC(\cO,N)$ is epi,
\end{enumerate}
and $\cC$ has small coproducts.
In particular, the equivalence is given as $\Hom_\cC(\cO,-):\cC\to\Mod R$, where $R:=\End_\cC(\cO)$, and in this case, $\Spec_{nc}R\cong(\cC,\cO)$.
The object $\cO$ is called the compact projective generator.
\end{thm}
compact? small coproducts?
\begin{cor}
$\Mod R\cong\Mod R'$ iff there is a finitely generated projective generator $P\in\Mod R$ such that $R'=\End_R(P)$.
\end{cor}
\begin{ex}
Let $R$ be a ring.
$R$ and $M_n(R)$ are Morita equivalent since $R^n$ is a finitely generated projective generator for $\Mod R$, but their schemes are not isomorphic in general.
\end{ex}


\begin{defn}[Twisting systems]
Let $A$ be a graded ring.
A twisting system is a family $\{\theta_i\}_{i\in\Z}$ such that $\theta_i:A\to A$ is a graded abelian group isomorphisms such that $\theta_i(a\theta_j(b))=\theta_i(a)\theta_{i+j}(b)$.
We can define twists $A^{\{\theta_i\}}$ and $M^{\{\theta_i\}}$.
\end{defn}
\begin{thm}[Zhang]\,
\begin{parts}
\item $\GrMod A\cong\GrMod A^{\{\theta_i\}}$. In particular, $\Proj_{nc}(A)\cong\Proj_{nc}(A^{\{\theta_i\}})$.
\item If $A,A'$ are finitely generated in degree one, and if $\GrMod A\cong\GrMod A'$, then there is a twisting system $\{\theta_i\}$ such that $A'\cong A^{\{\theta_i\}}$.
\end{parts}
\end{thm}

Let $A_\alpha:=k\<x,y\>/(\alpha xy-yx)$ and $A_J:=k\<x,y\>/(xy-yx+x^2)$.
Although we have seen that there is no twist $\theta\in\GrAut A_\alpha$ such that $A_\alpha^\theta\cong A_J$, but there is a twisting system $A_\alpha^{\{\theta_i\}}\cong A_J$.
We do not exactly know how to construct $\{\theta_i\}$ in general, but we can give the concrete computation in this case:
\[\theta_i:=\mat{\alpha&0\\0&1}^{-i}\mat{1&0\\-1&1}^i\]
since $A_\alpha=k[x,y]^{\left(\mat[small]{\alpha&0\\0&1}\right)}$ and $A_J=k[x,y]^{\left(\mat[small]{1&0\\-1&1}\right)}$.
Generalizing this, there is a theorem that if $\{\theta_i\}$ and $\{\theta_i'\}$ are twisting systems of $A$ and $A^{\{\theta_i\}}$ then there is a twisting system $\{\theta_i''\}$ such that $A^{\{\theta_i''\}}\cong(A^{\{\theta_i\}})^{\{\theta'_i\}}$.


\subsection{}

Consider an abelian category $\cC$, an object $\cO$ in $\cC$, and an autoequivalence $s$ of $\cC$.
We call the triple as an algebraic triple here.
Then,
\[B(\cC,\cO,s):=\bigoplus_{i\in\Z}\Hom_\cC(\cO,s^i\cO).\]
For example, if $X$ is a projective scheme, and $\cL\in\Pic$ is very ample, then we have a graded ring
\[B:=B(\Mod X,\cO_X,-\otimes_X\cL)=\bigoplus\Gamma(X,\cL^{\otimes i}),\]
gives $X=\Proj B$.
In other words, $B$ is the homogeneous coordinate ring of $X$.
\begin{defn}
For an algebraic triple $(\cC,\cO,s)$, we say $(\cO,s)$ is ample for $\cC$ if
\begin{enumerate}[(i)]
\item for every $M\in\cC$, there is $\{p_i\}_{i=1}^m$ with $\oplus_is^{-p_i}\cO\twoheadrightarrow M$.
\item for every epi $\cM\to\cN$ in $\cC$, there is an integer $n_0$ such that $\Hom_\cC(s^{-n}\cO,\cM)\to\Hom_\cC(s^{-n}\cO,\cN)$ is epi for $n\ge n_0$.
\end{enumerate}
\end{defn}


\begin{thm}[Artin-Zhang]
If $(\cO,s)$ is ample for $\cC$, then $\cC\cong\tails B(\cC,\cO,s):\cO\mapsto B$, i.e.~$(\cC,\cO)\cong\Proj_{nc}B$.
\end{thm}
\begin{cor}[Veronese algebra]
If $A$ is finitely generated in degree one, then $\Proj_{nc} A^{(r)}\cong \Proj_{nc}A$.
\end{cor}
For examples, $A=k[x,y]$ with $\Proj A=\P^1$ and $A^{(2)}=k[x^2,xy,y^2]\cong k[s,t,u]/(su-t^2)$, we can check manually $\Proj k[s,t,u]/(su-t^2)\cong\P^1$.


For a graded algebra $A$ and $B$, we can define the Segre product
\[A\circ B:=\bigoplus_iA_i\otimes B_i.\]
If $A,B$ are commutative and f.g.~in degree one, then $\Proj(A\circ B)\cong\Proj A\times\Proj B$.



\begin{ex}
We show that for $A:=k\<x,y\>/(x^2y-yx^2,xy^2-y^2x)$ that $\Proj_{nc}A\cong\P^1\times\P^1$.
Note that the second Veronese algebra is
\[A^{(2)}=k\<x^2,xy,yx,y^2\>/(x^2y-yx^2,xy^2-y^2x).\]
By observing eight relations
\[x(x^2y-yx^2)=0,\quad(x^2y-yx^2)x=0,\quad y(x^2y-yx^2)=0,\ \cdots\ (xy^2-y^2x)y=0,\]
and by letting $s=x^2$, $t=xy$, $u=yx$, $v=y^2$, we can conclude the existence of a surjection
\[B:=k[s,t,u,v]/(sv-tu)\twoheadrightarrow A^{(2)}.\]
The monomaial of $(A^{(2)})_i$ can be reduced to an element of the form $y^a(xy)^bx^c$ up to scalar multiple, where $a+2b+c=2i$, so $\dim_k(A^{(2)})_i=(i+1)^2$.
For $B$, if we count the dimension of the space spanned by monomials without $tu$, then we can see $\dim_kB_i=2\cdot{i+1\choose2}-{i+1\choose1}=(i+1)^2$.
Thus $H_{A^{(2)}}(t)=H_B(t)$ implies $A^{(2)}\cong B$

On the other hands, we also have for the Segre product
\[B=k[s,t,u,v]/(sv-tu)\twoheadrightarrow k[x_1,y_1]\circ k[x_2,y_2]=:C.\]
We can easily see that $H_C(t)=\sum_i(i+1)^2t^i$, so $B\cong C$.
Consequently,
\[\Proj_{nc}A\cong\Proj_{nc}A^{(2)}\cong\Proj A^{(2)}\cong\Proj C\cong\Proj k[x,y]\times\Proj k[x,y]\cong\P^1\times\P^1.\]
\end{ex}


\newpage
\pagenumbering{gobble}
\begin{center}
{\large\textbf{Noncommutative algebraic geometry}}\\
\bigskip
University of Tokyo M1 Ikhan Choi\\
November 28
\end{center}

\noindent
\textbf{1.}
Show that $A=k\<x,y\>/(yx,y^2)$ is left noetherian but not right noetherian.
\begin{sol}
Let $J$ be a left ideal of $k\<x,y\>$ containing $(yx,y^2)$, generated by $\{f_i\}_i\subset k\<x,y\>$ as a left $k\<x,y\>$-module.
Since $k\<x,y\>=k[x]+k[x]y+(yx,y^2)$, there are $g_i,h_i\in k[x]$ such that $f_i(x,y)\equiv g_i(x)+h_i(x)y$ modulo $(yx,y^2)$ for each $i$.
Then, we have
\[J=\sum_ik\<x,y\>f_i=\sum_i(k\<x,y\>g_i+k\<x,y\>h_iy)=\sum_i(k[x]g_i+k[x]g_i(0)+k[x]h_iy)=J_1+J_2y,\]
where $J_1$ and $J_2$ are ideals of $k[x]$ generated by $\{g_i,g_i(0)\}$ and $\{h_i\}$ respectively.
Because $k[x]$ is noetherian, the ideals $J_1$ and $J_2$ are finitely generated over $k[x]$, so $J$ is finitely generated over $k\<x,y\>$.
Thus, $A$ is left noetherian.

Let $J_n:=\sum_{i=0}^nx^iyk+(yx,y^2)$.
Since $x^iyk\<x,y\>=x^iyk+(yx,y^2)$, we can see that $J_n$ is an increasing sequence of right ideals of $k\<x,y\>$ containing $(yx,y^2)$ for all $n$.
Because the sequence $J_n$ does not terminate, $A$ is not right noetherian.
\end{sol}

\bigskip
\noindent
\textbf{2.}
Compute the Hilbert series of $A=k\<x,y\>/(x^2)$.
\begin{sol}
Note that $A_{i+2}=A_{i+1}y+A_iyx$ and $A_{i+1}y\cap A_iyx=0$ imply that the dimensions of the homogeneous modules satisfy the recurrence relation of Fibonacci sequence: $\dim_kA_{i+2}=\dim_kA_{i+1}+\dim_kA_i$ for all $i\ge0$.
Since $\dim_kA_1=2$, $\dim_kA_0=1$, and $\dim_kA_i=0$ for $i<0$, we have with the generating function that
\[H_A(t)=1+2t+3t^2+5t^3+8t^4+\cdots=\fbox{\displaystyle\frac{1+t}{1-t-t^2}}.\qedhere\]
\end{sol}

\bigskip
\noindent
\textbf{6.}
Compute the point variety of $A=k\<x,y\>/(yx)$.
\begin{sol}
Note that $\Gamma_1=\P^1$.
Suppose $((a_1,b_1),\cdots,(a_n,b_n))\in\Gamma_n$ for $n\ge2$.
If $a_n=0$, then since $b_n\ne0$ we have $((a_1,b_1),\cdots,(a_{n-1},b_{n-1}))\in\Gamma_{n-1}$ because
\[g((a_1,b_1),\cdots,(a_{n-1},b_{n-1}))b_n=f((a_1,b_1),\cdots,(a_n,b_n))=0\]
implies $g((a_1,b_1),\cdots,(a_{n-1},b_{n-1}))=0$ for all $g\in(yx)_{n-1}$, where $f:=gy$ belongs to $(yx)_n$.
If $a_n\ne0$, then since for each $1\le i<n$ we have $a_i\ne0$ or $b_i\ne0$, so $c_1c_2\cdots c_{n-1}a_n\ne0$, where $c_i\in\{a_i,b_i\}$.
If $c_i=b_i$ for some $i$, then implies that there is $1\le i<n$ such that $b_ia_{i+1}\ne0$, which leads a contradiction to the definitioin of $\Gamma_n$, so $c_i=a_i\ne0$ and $b_i=0$ for all $1\le i<n$.
Consequently, we have $\Gamma_n\subset(\Gamma_{n-1}\times\{0\})\cup(\{\infty\}^{\times(n-1)}\times\P^1)$.

Conversely, if $((a_1,b_1),\cdots,(a_{n-1},b_{n-1}))\in\Gamma_{n-1}$ and $a_n=0$, then every monomial $f\in(yx)_n$ satisfies
\[f((a_1,b_1),\cdots,(a_n,b_n))=g((a_1,b_1),\cdots,(a_n,b_n))c_n=0,\qquad c_n\in\{a_n,b_n\},\]
so $((a_1,b_1),\cdots,(a_n,b_n))\in\Gamma_n$, and if $b_1=\cdots b_{n-1}=0$, then every monomial $f\in(yx)_n$ satisfies
$f((a_1,b_1),\cdots,(a_n,b_n))=0$ clearly.
Thus the inverse inclusion holds so that $\Gamma_n=(\Gamma_{n-1}\times\{0\})\cup(\{\infty\}^{\times(n-1)}\times\P^1)$ for all $n\ge2$.

Therefore, the point variety of $A$ is
\[\Gamma_A=\lim_{\substack{\longleftarrow\\N}}\Gamma_N=\lim_{\substack{\longleftarrow\\N}}\ \bigcup_{n=0}^N\ \{\infty\}^n\times\P^1\times\{0\}^{N-n-1}=\fbox{\displaystyle\bigcup_{n=0}^\infty\ \{\infty\}^n\times\P^1\times\{0\}^\infty}.\qedhere\]
\end{sol}


\end{document}