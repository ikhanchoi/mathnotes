\documentclass{../../../small}
\usepackage{../../../ikhanchoi}

\begin{document}
\title{Operator Algebra Seminar Note I}
\author{Ikhan Choi}
\maketitle
\tableofcontents


\section*{Acknowledgement}
This note has been written based on the first-year graduate seminar presented at the University of Tokyo in the 2023 Spring semester.
Each seminar was delivered for 105 minutes.
The main reference for this seminar was Brown-Ozawa, and detailed gaps were filled with the aids of other books such as Takesaki, Murphy, and Paulsen whenever required.
I gratefully acknowledge advice of Prof. Yasuyuki Kawahigashi and support of my colleagues Futaba Sato, Yusuke Suzuki, Hiroto Nishikawa, Taiki Maeda.



\newpage
\section{April 14}

\subsection{Completely positive maps}

\begin{defn}
Let $A$ and $B$ be C$^*$-algebras.
A linear map $\f:A\to B$ is said to be \emph{completely positive} (c.p.) if the inflation $\f_n:M_n(A)\to M_n(B):[a_{ij}]\mapsto[\f(a_{ij})]$ is positive for each $n\ge1$.
\end{defn}

\begin{rmk}
For the positivity in matrix algebras, the following equivalent statements are useful.
\begin{parts}
\item $[a_{ij}]\in M_n(A)$ is positive.
\item $[a_{ij}]=[b_{ij}]^*[b_{ij}]=[b_{ji}^*][b_{ij}]=[\sum_kb_{ki}^*b_{kj}]$ for some $[b_{ij}]\in M_n(A)$.
\item $\sum_{i,j}\<\pi(a_{ij})\xi_j,\xi_i\>_H\ge0$ for $[\xi_i]\in H^n$, for a faithful representation $\pi:A\to B(H)$.
\item $\sum_{i,j}\<\pi(a_{ij})\xi_j,\xi_i\>_H\ge0$ for $[\xi_i]\in H^n$, for every representation $\pi:A\to B(H)$.
\end{parts}
Every positive element of $M_n(A)$ is the sum of elements of the form $[a_i^*a_j]$, so for $\f:A\to B$ to be completely positive, it is necessary and suffices to have
\[\sum_{i,j}\<\f(a_i^*a_j)\xi_j,\xi_i\>\ge0,\qquad [a_i]\in A^n,\ [\xi_i]\in H^n,\]
where $B$ is embedded in $B(H)$.
\end{rmk}

\begin{ex}\,
\begin{parts}
\item A $*$-homomorphism is c.p.
\item A state is c.p.
\item A conjugation $B(\hat H)\to B(H):a\mapsto V^*aV$ is c.p. for every bounded linear $V:H\to\hat H$.
\item The transpose $M_2(\C)\to M_2(\C)$ is not c.p.
\item The convex combination, composition, restriction of c.p. maps is c.p.
\end{parts}
\end{ex}
\begin{pf}
(a)
A $*$-homomorphism is positive, and its inflations are all $*$-homomorphisms.

(b)
Let $\rho:A\to\C$ be a state.
Then, we have for $[c_i]\in\ell_2^n$ that
\[\sum_{i,j}\<\rho(a_i^*a_j)c_j,c_i\>_\C
=\rho(\sum_{i,j}\bar{c_i}a_i^*a_jc_j)
=\rho((\sum_ic_ia_i)^*(\sum_jc_ja_j))\ge0.\]

(c)
We have for $[\xi_i]\in H^n$ that
\[\sum_{i,j}\<V^*a_i^*a_jV\xi_j,\xi_i\>
=\<\sum_ja_jV\xi_j,\sum_ia_iV\xi_i\>\ge0.\]

(d)
We have a counterexample for $M_2(M_2(\C))\to M_2(M_2(\C))$:
\[\mat{1&0&0&1\\0&0&0&0\\0&0&0&0\\1&0&0&1}\mapsto\mat{1&0&0&0\\0&0&1&0\\0&1&0&0\\0&0&0&1}.\]
The former has an eigenvalues $\{2,0\}$, and the latter has $\{\pm1\}$.

(e) Clear.
\end{pf}

\begin{thm}[Stinespring dilation]
Let $A$ be a unital C$^*$-algebra and $\f:A\to B(H)$ be a c.p.~map.
Then, there is a representation $\pi:A\to B(\hat H)$ and a bounded linear operator $V:H\to\hat H$ such that the following diagram commutes:
\[\begin{tikzcd}
 A \ar{r}{\f} \ar[swap]{d}{\pi} & B(H) \\
B(\hat H) \ar[swap]{ur}{V^*\cdot V} &
\end{tikzcd}\]
\end{thm}
\begin{pf}
Define a sesquilinear form on the algebraic tensor product $A\odot H$ as
\[\<\sum_ja_j\otimes\xi_j,\sum_ib_i\otimes\eta_i\>:=\sum_{i,j}\<\f(b_i^*a_j)\xi_j,\eta_i\>.\]
It is positive since
\[\sum_{i,j}\<a_i^*a_j\xi_j,\xi_i\>=\sum_{i,j}\<a_j\xi_j,a_i\xi_i\>=\|\sum_ia_i\xi_i\|^2\ge0\]
implies
\[\<\sum_ja_j\otimes\xi_j,\sum_ia_i\otimes\xi_i\>=\sum_{i,j}\<\f(a_i^*a_j)\xi_j,\xi_i\>\ge0.\]
Taking quotient by the left kernel $N$ and completion, we obtain a hilbert space $\hat H:=(A\odot H/N)^-$.

Define $\pi:A\to B(\hat H)$ such that
\[\pi(a)(b\otimes\xi+N):=ab\otimes\xi+N,\]
and define $V:H\to\hat H$ such that
\[V\xi:=1_ A\otimes\xi+N.\]
Then for any $\xi,\eta\in H$,
\[\<V^*\pi(a)V\xi,\eta\>=\<\pi(a)(1_ A\otimes\xi+N),1_ A\otimes\xi+N\>=\<a_ A\otimes\xi+N,1_ A\otimes\xi+N\>=\<\f(a)\xi,\eta\>.\qedhere\]
\end{pf}

\begin{rmk}\,
\begin{parts}
\item If $\f$ is unital, then $V$ is an isometry since $V^*V=V^*\pi(1)V=\f(1)=1$.
\item If $\f$ is unital and $H=\C$, then it is just the GNS-construction with the cyclic vector $V1_\C$.
\item If $\f:A\to B$ is c.p., then by embedding $B$ into $B(H)$ and applying the Stinespring dilation,
\[\|\f(a)\|=\|V^*\pi(a)V\|\le\|V\|\|a\|\|V\|=\|a\|\|V^*V\|=\|a\|\|\f(1)\|\]
implies $\|\f\|\le\|\f(1)\|$, hence $\|\f\|=\|\f(1)\|$.
\item It has a physical meaning: a unital completely positive map is called quantum channel or quantum operation in quantum information theory. They are interpreted as an evolution in open quantum system, and taking $\hat H$ means introducing a closed ambient system in which unitary evolution occurs.
\end{parts}
\end{rmk}

\begin{thm}[Completely positive maps for matrix algebras]
Let $A$ be a C$^*$-algebra.
Let $e_i\in\ell_2^n$ be standard orthonormal basis and let $e_{ij}=e_i\otimes e_j=|e_i\>\<e_j|\in M_n(\C)$ be unit matrix elements.
\begin{parts}
\item
There is a 1-1 correspondence
\[\mathrm{CP}(M_n(\C), A)\to M_n(A)_+:\psi\mapsto[\psi(e_{ij})].\]
\item
Let $A$ be unital.
There is a 1-1 correspondence
\[\mathrm{CP}(A,M_n(\C))\to M_n(A)^*_+:\f\mapsto(\hat\f:[a_{ij}]\mapsto\sum_{i,j}\<\f(a_{ij})e_j,e_i\>).\]
\end{parts}
\end{thm}
\begin{pf}
(a)
Fix $A\to B(H)$ a faithful representation and just write $A\subset B(H)$.

Suppose $\psi:M_n(\C)\to A$ is a c.p.~map.
Identify $M_n(\C)=B(\ell_2^n)$.
Since $[e_{ij}]\in M_n(B(\ell_2^n))_+$ is positive because
\[\sum_{i,j}\<e_{ij}\xi_j,\xi_i\>=\sum_{i,j}\<e_j,\xi_j\>\<\xi_i,e_i\>=|\sum_i\<e_i,\xi_i\>|^2\ge0,\qquad\forall[\xi_i]\in(\ell_2^n)^n,\]
it follows that $[\psi(e_{ij})]\in M_n(A)_+$ by the complete positivity of $\psi$.

Conversely, let $[\psi(e_{ij})]=[\sum_kb_{ki}^*b_{kj}]\in M_n(B(H))_+$,
For $T=[t_{ij}]\in M_n(\C)$ and $\xi,\eta\in H$, write
\begin{align*}
\<\psi(T)\xi,\eta\>
&=t_{ij}\<\psi(e_{ij})\xi,\eta\>\\
&=t_{ij}\<b_{kj}\xi,b_{ki}\eta\>\\
&=t_{ij}\delta_{kl}\<b_{lj}\xi,b_{ki}\eta\>\\
&=\<Te_j,e_i\>\<e_l,e_k\>\<b_{lj}\xi,b_{ki}\eta\>\\
&=\<(T\otimes1\otimes1)(e_j\otimes e_l\otimes(b_{lj}\xi)),(e_i\otimes e_k\otimes(b_{ki}\eta))\>.
\end{align*}
The summation symols are omitted in each row.
Then, if we define
\[V:H\to\ell_2^n\otimes\ell_2^n\otimes H:\xi\mapsto\sum_{i,k}e_i\otimes e_k\otimes(b_{ki}\eta),\]
we have an expression
\[\<\psi(T)\xi,\eta\>=\<V^*(T\otimes1\otimes1)V\xi,\eta\>,\]
which implies that $\psi$ is c.p.~because $T\mapsto T\otimes1_{\ell_2^n}\otimes1_H$ is a $*$-homomorphism.

(b)
Suppose $\f:A\to M_n(\C)$ is a c.p.~map.
Then, $\hat\f$ is positive since $[a_{ij}]\in M_n(A)_+$ implies
\[\hat\f([a_{ij}])=\sum_{i,j}\<\f(a_{ij})e_j,e_i\>\ge0.\]

Conversely, let $\hat\f\in M_n(A)^*_+$.
By the GNS-construction, we have a cyclic representation $\pi:M_n(A)\to B(H)$ with a cyclic vector $\psi\in H$ such that
\[\hat\f([a_{ij}])=\<\pi([a_{ij}])\psi,\psi\>.\]
For $\xi=\sum_j\xi_je_j,\eta=\sum_i\eta_ie_i\in\ell_2^n$, write
\begin{align*}
\<\f(a)\xi,\eta\>
&=\sum_{i,j}\<\f(a)\xi_je_j,\eta_ie_i\>
=\sum_{i,j}\<\f(\bar{\eta_i}a\xi_j)e_j,e_i\>\\
&=\hat\f([\bar{\eta_i}a\xi_j])
=\<\pi([\bar{\eta_i}a\xi_j])\psi,\psi\>
=\<\pi([\delta_{ij}\eta_i1_ A]^*[a][\delta_{ij}\xi_j1_ A])\psi,\psi\>\\
&=\<\pi([a])\pi([\delta_{ij}\xi_j1_ A])\psi,\pi([\delta_{ij}\eta_i1_ A])\psi\>.
\end{align*}
If we define
\[V:\ell_2^n\to H:\xi\mapsto\pi([\delta_{ij}\xi_j1_ A])\psi,\]
then
\[\<\f(a)\xi,\eta\>=\<V^*\pi([a])V\xi,\eta\>,\]
so $\f$ is c.p.~since $A\to M_n(A):a\mapsto[a]$ is a $*$-homomorphism.
\end{pf}

\begin{thm}[Arveson extension]
Let $A\subset B$ be C$^*$-algebras such that $1_ B\in A$.
Then, every c.p.~map $\f:A\to B(H)$ has an norm-preserving c.p.~extension $\tilde\f: B\to B(H)$, i.e. $\|\tilde\f\|=\|\f\|$.
\end{thm}
\begin{pf}
Let $p_\alpha$ be the net of projections of finite rank $n_\alpha$ in $B(H)$ with the image $V_\alpha$, which strongly converges to $\id_H$.
Fix $\alpha$ temporarily and let $\f_\alpha:=p_\alpha\f|_{V_\alpha}:A\to B(V_\alpha)$.
Choosing an any orthonormal basis of each $V_\alpha$, we can rewrite as $\f_\alpha:A\to M_{n_\alpha}(\C)$.
By the above theorem, we have the associated linear functional $\hat\f_\alpha\in M_{n_\alpha}(A)$.
Then, the Hahn-Banach extension provides an extension $(\hat\f_\alpha)^\sim\in M_{n_\alpha}(B)$, and we can define $\tilde\f_\alpha: B\to M_{n_\alpha}(\C)$ as the associated completely positive map.
Via the identification $B(V_\alpha)=M_{n_\alpha}(\C)$ we used to write $\f_\alpha:A\to M_{n_\alpha}(\C)$, we have $\tilde\f_\alpha: B\to B(V_\alpha)$.
We can check $\tilde\f_\alpha$ actually extends $\f_\alpha$, i.e. $\tilde\f_\alpha(a)=\f_\alpha(a)$ for $a\in A$, by putting $[a\delta_{ik}\delta_{jl}]_{i,j}\in M_{n_\alpha}(A)$ and comparing matrix components for each $k,l$.

Since $\|\tilde\f_\alpha\|=\|\tilde\f_\alpha(1)\|=\|\f_\alpha(1)\|=\|\f_\alpha\|\le\|\f\|$, the net $\tilde\f_\alpha$ is bounded in $B(B,B(H))$.
The norm-closed unit ball is compact in the point-$\sigma$-weak topology $\sigma(B(B,B(H)), B\odot L^1(H))$ because it is coarser than the weak$^*$ topology $\sigma(B(B,B(H)), B\hat\otimes_\pi L^1(H))$.
By taking a convergent subnet, we have a limit point $\tilde\f: B\to B(H)$.
It is easily seen to be completely positive and extend $\f$, and satisfies $\|\f\|=\|\f(1)\|=\|\tilde\f(1)\|=\|\tilde\f\|$.
\end{pf}



\subsection{Enveloping von Neumann algebras}
\begin{defn}
For a representation $\pi:A\to B(H)$ of a C$^*$-algebra $A$, we define a von Neumann algebra $\cM(\pi):=\pi(A)''$ associated to $\pi$.
\end{defn}

\begin{thm}[Sherman-Takeda]
Let $A$ be a C$^*$-algebra and $\pi_u:A\to B(H_u)$ the universal representation, the direct sum of all the GNS-representations of states of $A$.
Consider the following three maps
\[\pi_u:A\to(\cM(\pi_u),\sigma w),\qquad\pi_u^*:\cM(\pi_u)_*\to A^*,\qquad\tilde\pi_u:=\pi_u^{**}:A^{**}\to\cM(\pi_u),\]
constructed by adjoints, where $\cM(\pi_u)_*$ denotes the set of $\sigma$-weakly continuous(= normal) linear functionals on $\cM(\pi_u)$.
\begin{parts}
\item $\pi_u^*$ is isometric.
\item $\pi_u^*$ is surjective.
\item $\tilde\pi_u$ is an isometric isomorhpism(w.r.t. norms), and is an homeomorphism(w.r.t. weak$^*$-topologies).
\item $A^{**}$ enjoys a universal property in the sense that for every $*$-homomorphism $\f:A\to\cM$ to a von Neumann algebra $\cM$, there exists a unique normal extension $\tilde\f:A^{**}\to\cM$ of $\f$.
\end{parts}
\end{thm}
\begin{pf}
(a)
It holds for any representation of $\pi:A\to B(H)$.
For each $l\in\cM(\pi)_*$, we have
\[\|\pi^*(l)\|=\sup_{\substack{\|a\|\le1\\a\in A}}|l(\pi(a))|=\sup_{\substack{\|b\|\le1\\b\in\cM(\pi)}}|l(b)|=\|l\|\] by the Kaplansky density theorem and the $\sigma$-weak continuity of $l$.

(b)
Although $\pi_u$ is an injective $*$-homomorphism and hence is isometric so that its dual $\cM(\pi_u)^*\to A^*$ is surjective by the Hahn-Banach extension, it does not guarantee the $\sigma$-weak continuity of the extended linear functional.
We claim that every state of $A$ has a normal extension on $\cM(\pi_u)$.
If the claim is true, then the Jordan decomposition can be applied to show that every bounded linear functional has a normal extension.

Let $\rho$ be a state of $A$.
If we let $\psi$ be the canonical cyclic vector of the GNS representation $\pi_\rho:A\to B(H_\rho)$, then the state $\rho$ can be represented as a vector state $\omega_\psi$.
Since $\pi_\rho$ is a subrepresentation of $\pi_u$, the unit vector $\psi$ can be seen as an element of $H_u$, and it defines a normal state of $\cM(\pi_u)$.

(c)
It is is clear from (a) and (b).

(d)
We can define $\tilde\f$ as the bitranspose of $\f:A\to(\cM,\sigma w)$, and it is a unique extension because $A$ is $\sigma$-weakly dense in $A^{**}$.
\end{pf}
\begin{rmk}
The bidual $A^{**}$ is frequently viewed as a von Neumann algebra, and we call it the \emph{enveloping von Neumann algebra} of a C$^*$-algebra $A$.
By the universal property, we have a normal $*$-homomorphism $\cM(\pi_u)\to\cM(\pi)$ that is in fact surjective for every representation $\pi$ of $A$, and it fails to be injective even for faithful representations.
\end{rmk}

\begin{thm}[Tomiyama]
Let $B\subset A$ be C$^*$-algebras.
Let $\f:A\to B$ be a \emph{conditional expectation}, i.e. a contractive idempotent linear map.
\begin{parts}
\item $\f$ is $B$-bimodule map.
\item $\f$ is completely positive.
\end{parts}
\end{thm}
\begin{pf}
Since each conclusion of (a) and (b) still holds for restriction, we may assume $A$ and $B$ are von Neumann algebras by thinking of the bitranspose $\f^{**}:A^{**}\to B^{**}$.

(a)
Since the linear span of projections is $\sigma$-weakly dense in a von Neumann algebra, we are enough to show $p\f(a)=\f(pa)$ and $\f(ap)=\f(a)p$ for any projection $p\in B$.

Let $p\in B$ be a projection and let $a\in A$.
Note that we have
\[p\f(a)=pp\f(a)=p\f(p\f(a))\]
and
\[(a-pa)^*(p\f(a-pa))=(p\f(a-pa))^*(a-pa)=0.\]
Then,
\begin{align*}
(1+t)^2\|p\f(a-pa)\|^2
&=\|p\f(a-pa)+tp\f(a-pa)\|^2\\
&=\|p\f((a-pa)+tp\f(a-pa))\|^2\\
&\le\|(a-pa)+tp\f(a-pa)\|^2\\
&=\|a-pa\|^2+t^2\|p\f(a-pa)\|^2
\end{align*}
implies $p\f(a-pa)=0$ by letting $t\to\infty$.
Putting $1_ B-p$ and $1_ B$ instead of $p$, we obtain $(1_ B-p)\f(a-1_ B a+pa)=0$ and $\f(a-1_ B a)=0$, so
\[p\f(a)=p\f(pa)=\f(pa).\]
Similarly, we can show $\f(a-ap)p=0$ and $\f(ap)(1-p)=0$, we are done.

(b)
Let $[a_{ij}]\in M_n(A)_+$.
Let $\pi: B\to B(H)$ be a cyclic representation with a cyclic vector $\psi$.
Then, $[\xi_i]\in H^n$ can be replaced to $[\pi(b_i)\psi]$, so we can check the positivity of inflations $\f_n$ as
\[\sum_{i,j}\<\pi(\f(a_{ij}))\pi(b_j)\psi,\pi(b_i)\psi\>=\<\pi(\f(\sum_{i,j}b_i^*a_{ij}b_j))\psi,\psi\>\ge0,\]
because it follows $\sum_{i,j}b_i^*a_{ij}b_j\ge0$ by the positivity of $a_{ij}$ from
\[\<\pi_ A(\sum_{i,j}b_i^*a_{ij}b_j)\xi,\xi\>=\sum_{i,j}\<\pi_ A(a_{ij})\pi_ A(b_j)\xi,\pi_ A(b_i)\xi\>\ge0,\]
where $\pi_ A$ is any representation of $A$.
\end{pf}

\begin{thm}[Sakai]
Suppose $A$ is a C$^*$-algebra which admits a predual $F$.
\begin{parts}
\item There is an injective $*$-homomorphism $\pi:A\to A^{**}$ with weakly$^*$ closed image.
\item $\pi$ is a topological embedding w.r.t. $\sigma(A,F)$ and $\sigma(A^{**}, A^*)$.
\item The predual $F$ is unique in $A^*$.
\end{parts}
In particular, there is a faithful representation $A\to B(H)$ whose image is ($\sigma$-)weakly closed.
\end{thm}
\begin{pf}
(a)
By taking the adjoint for the inclusion $i:F\hookrightarrow A^*$, we have a conditional expectation $\e:A^{**}\twoheadrightarrow A$.
Its kernel is a $A$-bimodule, and by the $\sigma$-weak density of $A$ in $A^{**}$ and the continuity of $\e$ between weak$^*$ topologies, so it is in fact a $A^{**}$-bimodule, which means it is a $\sigma$-weakly closed ideal of $A^{**}$.
Thus we have a central projection $z\in A^{**}$ such that $\ker\e=(1-z) A^{**}$.

Define $\pi:A\to A^{**}$ such that $\pi(a):=za$.
It is clearly a $*$-homomorphism.
The injectivity follows from $a=\e(a)=\e(za)$ for $a\in A$.
The image is weakly$^*$ closed because $\e(x-\e(x))=0$ implies $z(x-\e(x))=0$ for $x\in A^{**}$ so that $z A^{**}=z A$.

(b)
Since $\<a,f\>=\<\e(za),f\>=\<za,f\>$ for $a\in A$ and $f\in F$, in which the second equality holds by the definition of $\e$, it is enough to show $\sigma(z A, A^*)=\sigma(z A,F)$.

For $l\in A^*$, we claim there exists $f$ such that $\<za,l\>=\<za,f\>$.
Define $\tilde l\in A^*$ such that $\<x,\tilde l\>:=\<zx,l\>$ for $x\in A^{**}$.
Then, $\<zx,l\>=\<z^2x,l\>=\<zx,\tilde l\>$ for $x\in A^{**}$.
Suppose $\tilde l\notin F$.
Because $F$ is closed in $A^*$, there is $x\in A^{**}$ such that $\<x,\tilde l\>\ne0$ and $\<x,f\>=0$ for all $f\in F$ by the Hahn-Banach separation.
Then, $0=\<x,f\>=\<x,i(f)\>=\<\e(x),f\>$ implies $\e(x)=0$ so that $zx=0$, which leads a contradiction $\<x,\tilde l\>=\<zx,l\>=0$, so we have $\tilde l\in F$.

(c)
If closed subspaces $F_1$ and $F_2$ of $A^*$ are preduals of $A$, then $\sigma(A,F_1)=\sigma(A,F_2)$ by the part (b).
If $l\in F_1$, which is obviously continuous on $\sigma(A,F_1)$, and the continuity in $\sigma(A,F_2)$ implies that $l$ is contained in a linear span of some finitely many elements of $F_2$, hence $F_1\subset F_2$.
\end{pf}


% 더 할 수 있으면 좋은 것 같은 이야기들
%  normality 에 관한 이야기
%  quasi-equivalent representations
% 


\newpage
\section{May 12}



\subsection{Nuclear C$^*$-algebras}

\begin{prop}[Maximal and minimal tensor products]
Let $A_1, A_2, B, B_1, B_2$ be C$^*$-algebras and $H_1,H_2$ be Hilbert spaces.
\begin{parts}
\item (Continuity of tensor product maps)
For any $*$-homomorphisms $\f_1:A_1\to B_1$ and $\f_2:A_2\to B_2$, the $*$-homomorphisms
\[\f_1\otimes\f_2:A_1\otimes_{\max} A_2\to B_1\otimes_{\max} B_2\]
and
\[\f_1\otimes\f_2:A_1\otimes_{\min} A_2\to B_1\otimes_{\min} B_2\]
are well-defined.
The same holds for c.p.~maps(see Corollary 3.3 for $\otimes_{\max}$).
\item (Universal property)
For any $*$-homomorphisms $\f_1:A_1\to B$ and $\f_2:A_2\to B$ whose images are commuting, there exists a unique $*$-homomorhpism
\[\f_1\times\f_2:A_1\otimes_{\max} A_2\to B\]
such that $\f_1\times\f_2(a_1\otimes a_2)=\f_1(a_1)\f_2(a_2)$.
\item (Minimal norm is spatial)
There is a natural $*$-monomorphism $B(H_1)\otimes_{\min}B(H_2)\hookrightarrow B(H_1\otimes H_2)$.
\end{parts}
\end{prop}
\begin{pf}
Omitted.
\end{pf}

\begin{defn}
A C$^*$-algebra $A$ is called \emph{nuclear} if the canonical surjection $A\otimes_{\max} B\to A\otimes_{\min} B$ is injective for any C$^*$-algebra $B$.
\end{defn}


\begin{ex}\,
\begin{parts}
\item Every finite-dimensional C$^*$-algebra is nuclear.
\item Every abelian C$^*$-algebra is nuclear.
\item A non-unital C$^*$-algebra is nuclear if and only if its unitization is nuclear.
\item A quotient of a nuclear C$^*$-algebra is nuclear.
\item The inductive limit of nuclear C$^*$-algebras is nuclear.
\item The tensor product of nuclear C$^*$-algebra is nuclear.
\end{parts}
\end{ex}
\begin{pf}
(a), (b), (e) See Theorem 6.3.9, 6.4.15, and 6.3.10 of [Murphy].
\end{pf}


\subsection{Completely positive approximation property}

\begin{defn}
Let $\theta:A\to B$ be a c.c.p.~map between C$^*$-algebras.
We say $\theta$ is \emph{factorable} if it factors through a matrix algebra $M_n(\C)$.
We say $\theta$ is \emph{approximable} or \emph{nuclear} if it is a limit of factorable maps in the point-norm topology.
When $B$ is a von Neumann algebra, we say $\theta$ is \emph{weakly approximable} or \emph{weakly nuclear} if it is a limit of factorable maps in the point-$\sigma$-weak topology.
\end{defn}

\begin{prop}
Let $A$ and $B$ be C$^*$-algebras, and $\cM\subset B(H)$ a von Neumann algebra.
Let $\cF\subset B(A, B)$ or $B(A,\cM)$ be the set of factorabl maps.
\begin{parts}
\item $\cF$ is convex.
\item In $B(A, B)$, we have
\[\bar\cF^{co}=\bar\cF^{pt-\|\cdot\|}=\bar\cF^{pt-w}.\]
\item In $B(A,\cM)$, we have
\[\bar{\R_{\ge0}\cF}^{pt-\sigma w}\cap B=\bar\cF^{pt-\sigma w}=\bar\cF^{pt-wot}=\bar\cF^{pt-sot},\]
where $B$ denotes the closed unit ball of $B(A,\cM)$.
\end{parts}
\end{prop}
\begin{pf}[Sketch]
(a)
Let $A\xrightarrow{\psi_i}M_{n_i}(\C)\xrightarrow{\f_i} B$ be c.c.p.~maps for $i\in\{0,1\}$.
Then, for $t\in[0,1]$ we have a diagram
\[\begin{tikzcd}
 A \ar{rrr}{(1-t)\psi_0\circ\f_0+t\psi_1\circ\f_1}\ar{d}&&& B\\
 A\oplus A\ar[swap]{r}{\f_0\oplus\f_1}&M_{n_0}(\C)\oplus M_{n_1}(\C)\ar[swap]{rr}{((1-t)\psi_0)\oplus(t\psi_1)}&& B\oplus B\ar{u}
\end{tikzcd}\]
which is commutative, so we are done.

(b), (c)
When comparing the strong (operator) topology and the weak (operator) topology, we can use the fact that continuous functionals with respect to both topologies are same and apply the Hahn-Banach separation because $\cF$ is convex.

When comparing the weak topology and the $\sigma$-weak topology, we can use the compactness of the closed unit ball in the $\sigma$-weak topology and use the fact that weak topology is weaker than the $\sigma$-weak topology to prove a homeomoprhism.

Finally, for the first equality of (b), see Proposition 3.8.2 in [Brown-Ozawa], which assumes unital maps.
\end{pf}


\begin{thm}
Let $A$ be a C$^*$-algebra.
Then, the identity $A\to A$ is approximable if and only if the inclusion $A\to A^{**}$ is weakly approximable.
\end{thm}
\begin{pf}
($\Rightarrow$)
Clear.

($\Leftarrow$)
Let $E\subset A$ and $F\subset A^*$ be any finite subsets and fix $\e>0$.
We may assume $E$ and $F$ are bounded by one and $F$ is positive.
We want to construct c.c.p.~maps $A\xrightarrow{\f}M_n(\C)\xrightarrow{\psi} A$ such that
\[|l(a-\psi\circ\f(a))|<\e,\qquad a\in E,\ l\in F.\]

By the assumption, we have a net of c.c.p.~maps $A\xrightarrow{\f'_\alpha}M_{n_\alpha}(\C)\xrightarrow{\psi'_\alpha} A^{**}$ satisfying
\[|l(a-\psi'_\alpha\circ\f'_\alpha(a))|\to0,\qquad a\in A,\ l\in A^*.\]
If we choose $0\le e\le1$ in $A$ such that $l(1-e)<\frac\e8$ for all $l\in F$, then
\begin{align*}
l(1-\psi'_\alpha(\id))
&=l(1-e)+l(e-\psi'_\alpha\circ\f'_\alpha(e))+l(\psi'_\alpha(\f'_\alpha(e)-\id))\\
&<\frac\e8+l(e-\psi'_\alpha\circ\f'_\alpha(e))+0
\end{align*}
implies that we have a c.c.p.~map $A\xrightarrow{\f'}M_n(\C)\xrightarrow{\psi'} A^{**}$ in the net such that
\[|l(a-\psi'\circ\f'(a))|<\frac\e4,\qquad a\in E,\ l\in F\tag{1}\]
and
\[l(1-\psi'(\id))<\frac\e4,\qquad l\in F.\tag{1'}\]
Now we let $\f:=\f'$ and try to deform $\psi'$ so that $A$ contains the codomain.

Define $\psi'':M_n(\C)\to A^{**}$ such that
\[\psi''(T):=\frac1n\Tr(T)(1-\psi'(\id))+\psi'(T).\]
Then, $\psi''$ is a u.c.p.~map and (1') implies
\[|l(\psi'\circ\f(a)-\psi''\circ\f(a))|<\frac\e4,\qquad a\in E,\ l\in F.\tag{2}\]

Now consider the associated matrix element $[\psi''(e_{ij})]\in M_n(A^{**})_+$ and the correspondence
\[M_n(A^*)\xrightarrow{\sim}M_n(A)^*:[l_{ij}]\mapsto([a_{ij}]\mapsto\sum_{i,j}l_{ij}(a_{ij})).\]
Since $M_n(A)$ is $\sigma$-weakly dense in $M_n(A)^{**}=M_n(A^*)^*=M_n(A^{**})$, the Kaplansky density theorem implies that the closed ball of $M_n(A)_+$ is $\sigma$-weakly dense in the closed ball of $M_n(A^{**})_+$.
For each pair $(a,l)\in E\times F$, if we define $[l_{ij}]\in M_n(A^*)$ such that $l_{ij}:=t_{ij}l$ and $\f(a)=[t_{ij}]\in M_n(\C)$, then we can take $[a_{ij}]\in M_n(A)$ such that
\[|l(\psi''(\f(a))-\sum_{i,j}t_{ij}a_{ij})|=|\sum_{i,j}l_{ij}(\psi''(e_{ij})-a_{ij})|=|[l_{ij}]([\psi''(e_{ij})-a_{ij}])|<\frac\e4\]
for all $a\in E$ and $l\in F$.
Now we define $\psi''':M_n(\C)\to A$ by $\psi'''(e_{ij}):=a_{ij}$ to get
\[|l(\psi''\circ\f(a)-\psi'''\circ\f(a))|<\frac\e4,\qquad a\in E,\ l\in F\tag{3}\]
and $\psi'':M_n(\C)\to A$ is c.p.
Note that we may not assume $\psi''$ is contractive because the correspondence $\mathrm{CP}(M_n(\C), A)\cong M_n(A)_+$ dose not preserve the norm.

However, when we take $\psi'''$ we can insert an additional condition
\[\|1-\psi'''(\id)\|=\|\psi''(\id)-\psi'''(\id)\|<\frac\e4\tag{3'}\]
using Mazur's lemma.
Define $\psi'''':M_n(\C)\to A$ by $\psi'''':=\psi'''/\|\psi'''\|$.
Then, (3') implies
\[|l((\psi'''-\psi'''')\circ\f(a))|=|l((\|\psi'''\|-1)\psi'''\circ\f(a))|\le|\|\psi'''\|-1|<\frac\e4,\qquad a\in E,\ l\in F.\tag{4}\]

Combining (1)\sim(4), we finally obtain $\psi:=\psi''''$ such that
\[|l(a-\psi''''\circ\f(a))|<\frac\e4\cdot4=\e,\qquad a\in E,\ l\in F,\]
so we are done.
In summary, we have constructed the following maps
\begin{center}
\begin{tabular}{rlc}
$\psi'$&$:M_n(\C)\to A^{**}$ & c.c.p.\\
$\psi''$&$:M_n(\C)\to A^{**}$ & u.c.p.\\
$\psi'''$&$:M_n(\C)\to A$ & c.p.\\
$\psi''''$&$:M_n(\C)\to A$ & u.c.p.
\end{tabular}
\end{center}
\end{pf}


\subsection{Choi-Effros-Kirchberg characterization}


\begin{lem}[Bounded Radon-Nikodym theorem]
Let $A$ be a C$^*$-algebra.
For $F$ a finite dimensional subspace of $A^*$, there is a cyclic representation $\pi:A\to B(H)$ with the cyclic vector $\Omega$ such that there is a linear map $\pi':F\to\pi(A)'$ satisfying $l(a)=\<\pi(a)\pi'(l)\Omega,\Omega\>$ for every $l\in F$.
The operator $\pi'(l)$ is called the Radon-Nikodym derivative of $l$ with respect to the vector state $\omega_\Omega$.
\end{lem}
\begin{pf}
Choose a basis $l_1,\cdots,l_n$ of $F$.
With the Jordan decomposition $l_i=l_{i,1}-l_{i,2}+i(l_{i,3}-l_{i,4})$, define a state
\[l_0:=\frac1{4n}\sum_{i=1}^n\sum_{j=1}^4\frac{l_{i,j}}{\|l_{i,j}\|}\]
by averaging, and let $\pi:A\to B(H)$ be the GNS-representation of $l_0$ with cyclic vector $\Omega\in H$.
Then for each $l\in F$,
\[\sigma(\pi(a)\Omega,\pi(b)\Omega):=l(b^*a)\]
are extended to well-defined bounded sesquilinear forms on $H$.
If we write by $\pi'(l)$ the bounded linear operator associated to the sesquilinear form $\sigma$, then for $l\in F$ we have
\[l(b^*a)=\<\pi'(l)\pi(a)\Omega,\pi(b)\Omega\>=\<\pi(b^*)\pi'(l)\pi(a)\Omega,\Omega\>,\]
and by putting $a=1$ and $b=1$ respectively, we can conclude $\pi'(l)\in\pi(A)'$ and the desired result.
\end{pf}

\begin{lem}[Fell's theorem]
Let $\pi:A\to B(H)$ be a faithful representation of a C$^*$-algebra.
Then, every state of $A$ is a weak$^*$ limit of the convex combination of pullbacks of vector states of $B(H)$.
\end{lem}
\begin{pf}
Suppose not and let $\omega$ be a counterexample.
By the Hahn-Banach separation, there is $a\in A$ and $r\in\R$ such that
\[\Re\omega_\xi(a)\le r<\Re\omega(a)\]
for all unit vector $\xi\in H$.
Defining $h=(a+a^*)/2$, rewrite the above inequality as
\[\omega_\xi(h)\le r<\omega(h).\]
Then, $r-h\ge0$ by the left inequality, which contradicts to the right inequality.
\end{pf}

\begin{thm}[Choi-Effros-Kirchberg]
Let $A$ be a C$^*$-subalgebra.
\begin{parts}
\item The identity $A\to A$ is approximable.
\item $A$ is nuclear.
\item $\pi\times i:A\odot_{\min}\pi(A)'\to B(H)$ is continuous.
\end{parts}
\end{thm}
\begin{pf}
(a)$\Rightarrow$(b)
Recall that every finite-dimensional C$^*$-algebra is nuclear.
For any C$^*$-algebra $B$, we have (by Corollary 3.3) a diagram
\[\begin{tikzcd}[row sep=10pt]
 A\otimes_{\max} B \ar{rr}\ar[dashed]{rd}\ar{dd} &&  A\otimes_{\max} B\\
&M_{n_\alpha}(\C)\otimes_{\max} B \ar[dashed]{ru}\ar[dashed,leftrightarrow]{dd}&\\
 A\otimes_{\min} B \ar[dashed]{rd} &&\\
&M_{n_\alpha}(\C)\otimes_{\min} B&
\end{tikzcd}\]
in which the square at lower left side algebraically commutes for each $\alpha$ and the upper triangle approximately commutes in the point-norm topology because
\[\|a\otimes b-\psi_\alpha\circ\f_\alpha(a)\otimes b\|_{\max}=\|a-\psi_\alpha\circ\f_\alpha(a)\|\|b\|\to0\]
for each $a\in A$, $b\in B$.
Here the dashed arrows mean that they vary as $\alpha$ goes to limit.
Hence we have an approximately commuting diagram
\[\begin{tikzcd}[row sep=10pt]
 A\otimes_{\max} B \ar{rr}\ar{rd} &&  A\otimes_{\max} B\\
& A\otimes_{\min} B \ar[dashed]{ur}&
\end{tikzcd}\]
so that we can verify the injectivity of $A\otimes_{\max} B\to A\otimes_{\max} B$ (because it is just the identity, so we do not have to take care of the failure of inclusion for maximal tensor products) implies the injectivity of $A\otimes_{\max} B\to A\otimes_{\min} B$.

(b)$\Rightarrow$(c)
Clear.

(c)$\Rightarrow$(a)
Let $E\subset A$ and $F\subset A^*$ be finite subsets and fix $\e>0$.
We want to find c.c.p.~maps $A\xrightarrow{\f}M_n(\C)\xrightarrow{\psi} A$ such that
\[|l(a)-l(\psi\circ\f(a))|<\e\]
for $a\in E$ and $l\in F$.
To implement the approximation, we would like to regard the inclusion operator as a state of a tensor product C$^*$-algebra via the correspondence
\[B(A, A)\subset B(A, A^{**})\xrightarrow{\sim}(A\otimes_\pi A^*)^*,\]
which maps the identity map $A\to A$ to the linear functional characterized by $a\otimes l\mapsto l(a)$.
Since $A^*$ is not a C$^*$-algebra, we think a ``representation'' of $\pi':F\to\pi(A)'$ through the above Radon-Nikodym type result.
Let $\pi:A\to B(H)$ be the cyclic representation obtained from the above Radon-Nikodym theorem and $\Omega$ the cyclic vector such that $l(a)=\<\pi(a)\pi'(l)\Omega,\Omega\>$ for $a\in E$ and $l\in F$.

By the assumption, we have a representation
\[\pi\times i:A\otimes_{\min}\pi(A)'\to B(H).\]
Consdier any faithful representation $\rho:A\to B(K)$ and the tensor representation
\[\rho\otimes i:A\otimes_{\min}\pi(A)'\to B(K\otimes H),\]
which is also faithful.
By Fell's theorem, the state $\omega_\Omega\circ(\pi\times i)$ on $A\otimes_{\min}\pi(A)'$ can be approximated by convex combinations of vector states in $B(K\otimes H)$.
In particular, by the density of $\pi(A)\Omega$ in $H$, we have tensors $(\tau_k)_{k=1}^m\subset K\odot\pi(A)\Omega$ such that
\[\Bigl|\omega_\Omega((\pi\times i)(a\otimes\pi'(l)))-\sum_{k=1}^m\lambda_k\omega_{\tau_k}((\rho\otimes i)(a\otimes\pi'(l)))\Bigr|<\e\tag{\dagger}\]
for all $a\in E$ and $l\in F$, where $\lambda_k\ge0$, $\sum_{k=1}^m\lambda=1$.

If we write each element $\tau\in K\odot\pi(A)\Omega$ as
\[\tau=\sum_{i=1}^n\eta_i\otimes\pi(b_i)\Omega,\]
then
\begin{align*}
\omega_\tau((\rho\otimes i)(a\otimes\pi'(l)))
&=\left\<(\rho(a)\otimes\pi'(l))\Bigl(\sum_{j=1}^n\eta_j\otimes\pi(b_j)\Omega\Bigr),\Bigl(\sum_{i=1}^n\eta_i\otimes\pi(b_i)\Omega\Bigr)\right\>\\
&=\sum_{i,j=1}^n\<\rho(a)\eta_j,\eta_i\>\<\pi'(l)\pi(b_i^*b_j)\Omega,\Omega\>\\
&=l\Bigl(\sum_{i,j=1}^n\<\rho(a)\eta_j,\eta_i\>b_i^*b_j\Bigr).
\end{align*}
If we define c.c.p.~maps $A\xrightarrow{\f}M_n(\C)\xrightarrow{\psi} A$ for each $\tau$ such that
\[\f(a):=[\<\rho(a)\eta_j,\eta_i\>],\quad\psi([e_{ij}]):=b_i^*b_j,\]
then we have $\omega_\tau(a\otimes\pi'(l))=l(\psi\circ\f(a))$.

Since $\mu(a\otimes\pi'(l))=l(a)$ and since the c.c.p.~maps that factor through a matrix algebra form a convex set, we have c.c.p.~maps $A\xrightarrow{\f}M_n(\C)\xrightarrow{\psi} A$ such that the inequality (\dagger) is rewritten as
\[|l(a)-l(\psi\circ\f(a))|<\e,\]
so we are done.
\end{pf}






\newpage
\section{May 26}



In the next two subsections, we extend our investigation of famous theorems related to copmletely positive maps: the Stinespring dilation theorem and the Arveson extension theorem.

\subsection{Stinespring dilation revisited}

Here is a Stinespring dilation theorem for non-unital C$^*$-algebras.
We also discuss shortly the minimal Stinespring representation.

\begin{thm}[Stinespring dilation]
Let $A$ be a C$^*$-algebra and $\f:A\to B(H)$ is a c.p.~map.
Then, there is a representation $\pi:A\to B(\hat H)$ and a bounded linear operator $V:H\to\hat H$ such that $\f(a)=V^*\pi(a)V$ for all $a\in A$.
Moreover, we can take $\pi$ to be \emph{minimal} in the sense that $(\pi(A)VH)^-=\hat H$.
\end{thm}
\begin{pf}
In the first day we have defined $\pi:A\to B(\hat H)$ and $V:H\to\hat H$, where $\hat H:=(A\odot H/N)^-$, such that
\[\pi(a)(b\otimes\xi+N):=(ab)\otimes\xi+N,\qquad V\xi:=1_ A\otimes\xi+N.\]
Since $A$ may not have its unit, we want to adjust the definition of $V$ with approximate unit.
For an approximate unit $e_\alpha$ of $A$, let $V_\alpha:H\to\hat H$ be such that
\[V_\alpha\xi:=e_\alpha\otimes\xi+N.\]
Recall that if $\omega$ is a positive linear functional, then $\lim_\alpha\omega(e_\alpha)=\omega(1)$.
For fixed $\xi\in H$, because $\omega_\xi\circ\f$ is a positive linear functional and products of approximate units are also approximate units, we have
\begin{align*}
\|(V_\alpha-V_\beta)\xi\|^2=\<\f((e_\alpha-e_\beta)^2)\xi,\xi\>&=\<\f(e_\alpha^2-e_\alpha e_\beta-e_\beta e_\alpha+e_\beta^2)\xi,\xi\>\\
&\to\<(\f-\f-\f+\f)\xi,\xi\>=0,
\end{align*}
so the net $V_\alpha\xi$ is Cauchy.
Define $V\xi:=\lim_\alpha V_\alpha\xi$.
Then,
\begin{align*}
\<V^*\pi(a)V\xi,\eta\>
&=\lim_\alpha\<\pi(a)V_\alpha\xi,V_\alpha\eta\>\\
&=\lim_\alpha\<(ae_\alpha)\otimes\xi+N,e_\alpha\otimes\eta+N\>\\
&=\lim_\alpha\<\f(e_\alpha ae_\alpha)\xi,\eta\>\\
&=\<\f(a)\xi,\eta\>.
\end{align*}
It is easy to check the condition $(\pi(A)VH)^-=\hat H$ holds.

(Alternatively, we can define $V\in B(H,\hat H)\cong H^*\otimes\hat H$ by
\[\<V\xi,a\otimes\eta+N\>:=\<\f(a^*)\xi,\eta\>.\]
The vector $V\xi$ in $\hat H$ for each $\xi\in H$ is well-defined because
\[|\<\f(a^*)\xi,\eta\>|^2=|\<\xi,\f(a)\eta\>|^2\le\|\xi\|^2\<\f(a^*)\f(a)\eta,\eta\>\le\|\xi\|^2\|\f\|\<\f(a^*a)\eta,\eta\>=\|\xi\|^2\|\f\|\|a\otimes\eta+N\|^2.\]
(For this proof, we have to establish (a) in Proposition 3.2 by applying the Cauchy-Schwarz inequality on the second inflation of $\f$(Maybe..?).))
\end{pf}

\begin{rmk*}
For a c.p.~map $\f:A\to B(H)$, we can define a \emph{Stinespring dilation} of $\f$ as a triplet $(\pi,K,V)$, where $\pi:A\to K$ is a representation and $V:H\to K$ is a bounded linear operator, such that $\f(a)=V^*\pi(a)V$.
For two Stinespring dilations $(\pi_1,K_1,V_1)$ and $(\pi_2,K_2,V_2)$ of $\f$, we can define morphisms by a bounded linear operator $U:K_2\to K_1$ such that $\pi_2(a)=U^*\pi_1(a)U$ and $V_2=UV_1$.
The two conditions imply that the following diagram commutes:
\[\begin{tikzcd}
 A\ar{rr}{\f}\ar{dr}{\pi_2}\ar[swap]{dddr}{\pi_1}&&B(H)\\
&B(K_2)\ar{ur}{V_2^*\cdot V_2}&\\
&&\\
&B(K_1)\ar[swap]{uuur}{V_1^*\cdot V_1}\ar{uu}[description]{U^*\cdot U}&
\end{tikzcd}\]

For arbitrary Stinespring dilation, we may always assume $\pi$ acts non-degenerately on the subspace $VH$ of $K$ by changing $V$ to $\lim_\alpha\pi(e_\alpha)V$ to make $\im V\subset(\pi(A)VH)^-$.
Even we add this in the definition of a Stinespring dilation, the definition of morphisms still left unchanged.
Here, the non-denegeracy condition $(\pi(A)VH)^-=\hat H$ for the \emph{minimal} Stinespring dilation is equivalent to that the Stinespring dilation enjoys the universal property, and hence unique up to morhpisms.
A nice fact, the isomorhpism $U$ is unitary, which means the minimal Stinespring is unique up to unitary equivalence.
\end{rmk*}

\begin{prop}[Multiplicative domain]
Let $\f:A\to B$ be a c.c.p.~map between C$^*$-algebras.
\begin{parts}
\item $|\f(a)|^2\le\f(|a|^2)$ for $a\in A$.
\item If $|\f(a)|^2=\f(|a|^2)$ and $|\f(a^*)|^2=\f(|a^*|^2)$, then $\f(ba)=\f(b)\f(a)$ and $\f(ab)=\f(a)\f(b)$, respectively.
\item $\{a\in A:|\f(a)|^2=\f(|a|^2)\}$ is a closed under the multiplication.
In particular, $\{a\in A:|\f(a)|^2=\f(|a|^2),\ |\f(a^*)|^2=\f(|a^*|^2)\}$ is a C$^*$-subalgebra of $A$ on which the restriction of $\f$ is a $*$-homomorphism.
\end{parts}
\end{prop}
\begin{pf}
(a)
Let $(\pi,H,V)$ be the minimal Stinespring dilation of $\f$.
Since $\f$ is contractive, so is $V$.
Then,
\[\f(|a|^2)-|\f(a)^2|=V^*\pi(a)^*(1_H-VV^*)\pi(a)V\ge0.\]

(b)
For $a\in A$, the equality $|\f(a)|^2=\f(|a|^2)$ holds if and only if $(\id_H-VV^*)^{\frac12}\pi(a)V=0$.
Then,
\[\f(ba)-\f(b)\f(a)=V^*\pi(b)^*(1_H-VV^*)\pi(a)V=0.\]
Similar for $a^*$.

(c)
If $a$ and $b$ satisfy the equality, then
\begin{align*}
(\id_H-VV^*)^{\frac12}\pi(ab)V
&=(\id_H-VV^*)^{\frac12}\pi(a)\pi(b)V\\
&=(\id_H-VV^*)^{\frac12}\pi(a)\pi(b)V-[(\id_H-VV^*)^{\frac12}\pi(a)V]V^*\pi(b)V\\
&=(\id_H-VV^*)^{\frac12}\pi(a)(\id_H-VV^*)\pi(b)V\\
&=(\id_H-VV^*)^{\frac12}\pi(a)(\id_H-VV^*)^{\frac12}[(\id_H-VV^*)^{\frac12}\pi(b)V]=0.\qedhere
\end{align*}
\end{pf}

\begin{lem}[Restriction of product representation]
Let $A$ and $B$ be C$^*$-algebras. and $\Pi:A\odot B\to B(H)$ be a $*$-homomorphism.
Then, there are $*$-homomorphisms $\pi:A\to B(H)$ and $\pi': B\to B(H)$ with commuting ranges such that $\Pi=\pi\times\pi'$.
\end{lem}
\begin{pf}
Let $K:=(\Pi(A\odot B)H)^-$.
We first claim that a map $\pi:A\to B(K)$ defined by
\[\pi(a)(\Pi(\sum_ia_i\otimes b_i)\xi):=\Pi(\sum_i(aa_i)\otimes b_i)\xi\]
for $\xi\in H$ is indeed well-defined and bounded for each $a\in A$.
We can define $\pi'$ in the same manner.

We may assume $\|a\|\le1$.
It suffices to show the inequality
\[\|\Pi(\sum_i(aa_i)\otimes b_i)\xi\|^2\le\|\Pi(\sum_ia_i\otimes b_i)\xi\|^2\]
for $\xi\in H$
The left hand side is
\[\|\Pi(\sum_i(aa_i)\otimes b_i)\xi\|^2
=\<\Pi(\sum_{i,j}(a_i^*a^*aa_j)\otimes(b_i^*b_j))\xi,\xi\>\]
and the right hand side is
\[\|\Pi(\sum_ia_i\otimes b_i)\xi\|^2
=\<\Pi(\sum_{i,j}(a_i^*a_j)\otimes(b_i^*b_j))\xi,\xi\>\]
so that there difference is positive because
\[\sum_{i,j}(a_i^*(1-a^*a)a_j)\otimes(b_i^*b_j)=|\sum_i((1-a^*a)^{\frac12}a_i)\otimes b_i|^2\ge0,\]
where 1 is the unit in the unitization of $A$ for the continuous functional calculus.

Note that the restrictions $\pi$ and $\pi'$ are clearly $*$-homomorphisms, and commuting ranges of $\pi$ and $\pi'$ also can be immediately checked.
By definition of the restrictions, we have
\[\pi(a)\pi'(b)\eta=\Pi(a\otimes b)\eta\]
for all $a\in A$, $b\in B$, and $\eta\in K$.
If we extend the domain from $K$ to $H$ by letting $\pi(a)\eta^\perp=\pi'(b)\eta^\perp:=0$ for $\eta^\perp\in H\ominus K$, then since
\[\<\Pi(a\otimes b)\eta^\perp,\xi\>=\<\eta^\perp,\Pi(a^*\otimes b^*)\xi\>=0\]
for $\xi\in H$ implies $\Pi(a\otimes b)\eta^\perp=0$, we have
\[\pi(a)\pi'(b)(\eta+\eta^\perp)=\pi(a)\pi'(b)\eta=\Pi(a\otimes b)\eta=\Pi(a\otimes b)(\eta+\eta^\perp),\]
hence $\pi\times\pi'=\Pi$.
\end{pf}


\begin{prop}[Maximal tensor products of c.c.p.~maps]
Let $A, B$ and $ C$ be C$^*$-algebras and $\f:A\to B$ be a c.c.p.~map.
Then, $\f\otimes i:A\otimes_{\max} C\to B\otimes_{\max} C$ is well-defined.
\end{prop}
\begin{pf}
First assume $B\otimes_{\max} C\subset B(H)$ by taking a faithful representation $b\otimes c\mapsto bc$.
By restriction, $B$ and $ C$ can be seen as commuting C$^*$-subalgebras of $B(H)$.
Let $\pi:A\to B(\hat H)$ be the minimal Stinespring representation of the c.c.p. map $\f:A\to B\subset B(H)$.
\[\begin{tikzcd}
 A\ar{d}{\f}\ar{r}{\pi}&B(\hat H)\ar{d}{V^*\cdot V}& C\ar[equal]{d}\ar[dashed,swap]{l}{\rho}\\
 B\ar[hook]{r}&B(H)& C\ar[hook']{l}
\end{tikzcd}\]
Now we want to lift the inclusion $ C\subset B'\subset B(H)$ to obtain a representation $\rho: C\to\pi(A)'\subset B(\hat H)$.

Since $\pi(A)VH$ is dense in $\hat H$, we can try to define $\rho$ by
\[\rho(c)(\sum_i\pi(a_i)V\xi_i):=\sum_i\pi(a_i)Vc\xi_i.\]
It is indeed well-defined from the following inequality:
\begin{align*}
\|\sum_i\pi(a_i)Vc\xi_i\|^2
&=\sum_{i,j}\<c^*\f(a_i^*a_j)c\xi_j,\xi_i\>\\
&=\<([c\delta_{ij}]^*[\f(a_i^*a_j)][c\delta_{ij}])[\xi_i],[\xi_i]\>_{H^n}\\
&=\<([\f(a_i^*a_j)]^{\frac12}[c\delta_{ij}]^*[c\delta_{ij}][\f(a_i^*a_j)]^{\frac12})[\xi_i],[\xi_i]\>_{H^n}\\
&\le\|[c\delta_{ij}]\|_{M_n( C)}^2\<([\f(a_i^*a_j)]^{\frac12}[\f(a_i^*a_j)]^{\frac12})[\xi_i],[\xi_i]\>_{H^n}\\
&=\|c\|^2\|\sum_i\pi(a_i)V\xi_i\|^2.
\end{align*}
Now then we can easily deduce that $\rho$ is linear and preserves the multiplication, and is a $*$-homomorhpism from checking
\[\<\sum_j\pi(a_j)Vc\xi_j,\sum_i\pi(a_i)V\xi_i\>=\<\sum_j\pi(a_j)V\xi_j,\sum_i\pi(a_i)Vc^*\xi_i\>.\]
The commutation with $\pi(A)'$ is clear.

Using the universality, we have a $*$-homomorphism $\pi\times\rho:A\otimes_{\max} C\to B\otimes_{\max} C\subset B(\hat H)$, which satisfies
\[V^*(\pi\times\rho)(a\otimes c)V\xi=V^*\pi(a)\rho(c)V\xi=V^*\pi(a)Vc\xi=\f(a)c\xi.\]
Since the product $\f(a)c$ has been identified with the simple tensor $\f(a)\otimes c\in B\otimes_{\max} C$ when we embed $B$ and $ C$ into $B(H)$, the above equality implies that $V^*(\pi\times\rho)V=\f\otimes i$.
\end{pf}

\begin{rmk*}
For completely positive maps $\f_1:A_1\to\B_1$ and $\f_2:A_2\to B_2$, the tensor product map
\[\f_1\otimes\f_2:A_1\odot A_2\to B_1\odot B_2:a_1\otimes a_2\mapsto\f_1(a_1)\otimes\f_2(a_2)\]
can be extended to both minimal and maximal tensor products: $A_1\otimes_{\min} A_2\to B_1\otimes_{\min} B_2$ and $A_1\otimes_{\max} A_2\to B_1\otimes_{\max} B_2$, respectively.
For $\otimes_{\max}$, it is justified by applying Proposition 3.4 twice, and for $\otimes_{\min}$, we can show it by the Stinespring dilation and the spatial property.
\end{rmk*}


\subsection{Arveson extension revisited}

\begin{prop}[Representation extensioin]
Let $A$ be a closed ideal of a C$^*$-algebra $B$.
For a representation $\pi:A\to B(H)$, there is a representation $\tilde\pi: B\to B(H)$ which extends $\pi$.
If $\pi$ is non-degenerate, the extension is unique and $\pi(e_\alpha b)\to\tilde\pi(b)$ and $\pi(be_\alpha)\to\tilde\pi(b)$ strongly for $b\in B$, where $e_\alpha$ is an approximate unit of $A$.
\end{prop}
\begin{pf}
Let $K:=(\pi(A)H)^-$.
Define $\tilde\pi: B\to B(H)$ such that
\[\tilde\pi(b)(\pi(a)\xi+\eta):=\pi(ba)\xi\]
for $\xi\in H$, $\eta\in H\ominus K$.
The well-definedness is from
\[\|\pi(ba)\xi\|^2=\<\pi(a^*b^*ba)\xi,\xi\>\le\|b\|^2\<\pi(a^*a)\xi,\xi\>=\|b\|^2\|\pi(a)\xi\|^2.\]
It is clearly a $*$-homomorphism and extends $\pi$ since $\<\pi(a')\eta,\zeta\>=\<\eta,\pi(a'^*)\zeta\>=0$ for all $\zeta\in H$ implies
\[\tilde\pi(a')(\pi(a)\xi+\eta)=\pi(a'a)\xi=\pi(a')(\pi(a)\xi+\eta).\]

For the uniqueness, if $\pi$ is non-degenerate and $\tilde\pi$ is a $*$-homomorphism which extends $\pi$, then
\[\tilde\pi(b)(\pi(a)\xi)=\tilde\pi(b)\tilde\pi(a)\xi=\tilde\pi(ba)\xi=\pi(ba)\xi,\]
which is unique by the density of $\pi(A)H$ in $H$.
\end{pf}

\begin{thm}[Arveson extension]
Let $A\subset B$ be C$^*$-algebras.
Let $\f:A\to B(H)$ be a c.p.~map and consider the following diagram:
\[\begin{tikzcd}
 B\ar[dashed]{dr}{\tilde\f}&\\
 A\ar{u}\ar[swap]{r}{\f}&B(H).
\end{tikzcd}\]
\begin{parts}
\item The n.p.c.p.(norm preserving c.p.)~extension $\tilde\f$ of $\f$ exists if $B$ is unital and $1_ B\in A$.
\item The n.p.c.p.~extension $\tilde\f$ of $\f$ exists if $A$ is unital and $B= A\oplus\C$.
\item The n.p.c.p.~extension $\tilde\f$ of $\f$ exists if $A$ is non-unital and $B=\tilde A$.
\item The n.p.c.p.~extension $\tilde\f$ of $\f$ always exists.
\end{parts}
\end{thm}
\begin{pf}
(a) We have proved on the first day.

(b)
Define $\tilde\f(a+\lambda)=\f(a)$.
Then, $\tilde\f:A\oplus\C\to B(H)$ is a norm-preserving c.p.~extension of $\f$ since
\[\|\tilde\f(a+\lambda)\|=\|\f(a)\|\le\|\f\|\|a\|\le\|\f\|\|a+\lambda\|\]
and
\[\|\f(a)\|=\|\tilde\f(a)\|\le\|\tilde\f\|\|a\|.\]

(c)
Let $\pi:A\to B(\hat H)$ be the minimal and hence non-degenerate Stinespring representation of $\f$.
Since $A$ is a closed ideal of $B=\tilde A$, we can apply the representation extension for $\pi$ to get $\tilde\pi$ as follows:
\[\begin{tikzcd}
\tilde A\ar{dr}{\tilde\pi}&\\
 A\ar{u}\ar{r}{\pi}&B(\hat H)\ar{d}{V^*\cdot V}\\
 A\ar[equal]{u}\ar{r}{\f}&B(H).
\end{tikzcd}\]
Then, the c.p.~extension is given by $\tilde\f(b):=V^*\tilde\pi(b)V$ for $b\in B$.
To check the norm is preserved, take an approximate unit $e_\alpha$ of $A$ so that the net $\f(e_\alpha b)=V^*\pi(e_\alpha b)V$ converges to $\tilde\f(b)=V^*\tilde\pi(b)V$ strongly and weakly, which implies
\[\|\tilde\f(1)\xi\|=\lim_\alpha\|\f(e_\alpha)\xi\|\le\|\f\|\|\xi\|.\]
Hence $\|\tilde\f\|\le\|\f\|$.

(d)
By the part (c), we may assume $B$ is unital.
Let $\tilde A= A+1_ B\C$ be the C$^*$-subalgebra of $B$ generated by $A$ and $1_ B$.
Since $A$ is an ideal of $\tilde A$ with codimension at most one, we have three cases:
\begin{enumerate}[(i)]
\item $\tilde A= A$, if $1_ B\in A$,
\item $\tilde A= A\oplus\C$, if $A$ is unital but $1_ B\notin A$
\item $\tilde A$ is the unitization of $A$, if $A$ is non-unital.
\end{enumerate}
For the last case reduces to the part (a). 
By using (b) or (c), we have an extension to $\tilde A$, and the other two cases also reduce to the part (a).
\end{pf}

We introduce a  more controlled version of the Arveson extension, the Trick, named after [Brown-Ozawa].
It is useful when we want to restrict the codomain of an Arveson extension.
\begin{thm}[The Trick]
Let $A\subset B$ and $ C$ be C$^*$-algebras, and let $\pi:A\to B(H)$ be a representation.
\begin{enumerate}[(i)]
\item If there are C$^*$-norm $\alpha$ and $\beta$ on $A\odot C$ and $B\odot C$ respectively such that $A\otimes_\alpha C\to B\otimes_\beta C$ is injective,
\item and if there is a representation $\pi': C\to B(H)$ such that $\pi\times\pi':A\odot_\alpha C\to B(H)$ is continuous,
\end{enumerate}
then there is a norm-preserving c.c.p. extension $\f: B\to\pi'( C)'$ of $\pi:A\to\pi'( C)'$.
\end{thm}
\begin{pf}
We first claim that we may assume $B$ and $ C$ are unital.
To see this, we must verify the existence of a C$^*$-norm $\tilde\beta$ on $B\odot\tilde C$ such that $B\otimes_\beta C\to B\otimes_{\tilde\beta}\tilde C$ is injective.
We can do same thing for $B$ with symmetry.
Take a faithful representation $\rho\times\rho': B\otimes_\beta C\to B(K)$, where $\rho: B\to B(K)$ and $\rho': C\to B(K)$ are restrictions of $\rho\times\rho'$.
Then, $\rho'$ can be extended to a faithful representation $\tilde\rho':\tilde C\to B(K)$, and it commutes with $\rho$ since the new element $\tilde\rho'(1)$ is the strong limit of elements of $\rho'( C)$.
Now consider the product representation $\rho\times\tilde\rho': B\odot\tilde C\to B(K)$.
We want to this is faithful.
Suppose $(\rho\times\tilde\rho')(\sum_{i=1}^nb_i\otimes c_i)=0$ and $b_i$ are linearly independent.
For approximate units $e_\alpha$ and $f_\alpha$ of $B$ and $ C$ respectively, we have
\[(\rho\times\rho')(\sum_ib_ie_\alpha\otimes c_if_\alpha)=(\rho\times\tilde\rho')((\sum_ib_i\otimes c_i)(e_\alpha\otimes f_\alpha))=0,\]
which implies $\sum_ib_ie_\alpha\otimes c_if_\alpha=0$ for each $\alpha$.
For every sufficiently large $\alpha$, the finite set $b_ie_\alpha$ should be linearly independent.
(Define $W_i:=\spn(\{b_1,\cdots,b_n\}\setminus\{b_i\})$ and $d_{i,\alpha}:=\inf\{\|b_ie_\alpha-be_\alpha\|:b\in W_i\}$. Then, we can deduce $\min_id_{i,\alpha}>0$ for every sufficiently small $\|e_\alpha\|$ from the linear independence of $b_i$.)
Therefore, $c_if_\alpha=0$.
By $\lim_\alpha$, we can conclude $\rho\times\tilde\rho'$ is faithful so that we have $B\otimes_\beta C\to B\otimes_{\tilde\beta}\tilde C$ injective with the C$^*$-norm induced from $B(K)$.

Now let $B$ and $ C$ be unital.
Then, by the Arveson extension theorem, there is a c.c.p.~map $\Phi: B\otimes_\beta C\to B(H)$ which extends $\pi\times\pi'$.
Define $\f: B\to B(H)$ by $\f(b):=\Phi|_ B(b)=\Phi(b\otimes1)$.
Then, $\f$ clearly extends $\pi$.
Since $\Phi|_{\C1\otimes C}=\pi'$ is a $*$-homomorphism, $1\otimes c$ belongs to the multiplicative domain of $\Phi$, hence we can compute
\[\f(b)\pi'(c)=\Phi(b\otimes1)\Phi(1\otimes c)=\Phi((b\otimes1)(1\otimes c))=\Phi(1\otimes c)\Phi(b\otimes1)=\pi'(c)\f(b).\qedhere\]
\end{pf}


\begin{cor}[Inclusion problem]
Let $A\subset B$ be C$^*$-algebras.
Then, the following statements are all equivalent:
\begin{parts}
\item There is a c.c.p.~map $\f: B\to A^{**}$ such that
\[\begin{tikzcd}
 B\ar[dashed]{dr}{\f}&\\
 A\ar{u}\ar{r}& A^{**}
\end{tikzcd}\]
is commutative.
\item There is a c.c.p.~map $\f: B\to\pi(A)''$ such that
\[\begin{tikzcd}
 B\ar[dashed]{dr}{\f}&\\
 A\ar{u}\ar{r}{\pi}&\pi(A)''
\end{tikzcd}\]
is commutative for every representation $\pi:A\to B(H)$.
\item The map $A\otimes_{\max} C\to B\otimes_{\max} C$ is injective for every C$^*$-algebra $ C$.
\end{parts}
\end{cor}
\begin{pf}
(a)$\Rightarrow$(b)
For any representation $\pi:A\to B(H)$ admits a normal extension $\tilde\pi:A^{**}\to B(H)$ by the universal property of the enveloping von Neumann algebra.
Then, our c.c.p.~map is $\tilde\pi\circ\f$, where $\f: B\to A^{**}$ is a c.c.p.~map constructed from the assumption (a), because the following diagram commutes:
\[\begin{tikzcd}
 B\ar[dashed]{dr}{\f}&\\
 A\ar{u}\ar{r}& A^{**}\ar{d}{\tilde\pi}\\
 A\ar[equal]{u}\ar{r}{\pi}&\pi(A)''.
\end{tikzcd}\]

(b)$\Rightarrow$(a)
Clear.

(b)$\Rightarrow$(c)
Taking any faithful representation of $A\otimes_{\max} C$ and by restriction, we may assume $A$, $ C$, and $A\otimes_{\max} C$ are C$^*$-subalgebras of $B(H)$ such that $A\subset C'$.
Then, by the assumption (b) we can define a c.c.p.~map $\f\otimes\id$ such that
\[\begin{tikzcd}
 B\otimes_{\max} C\ar[dashed]{dr}{\f\otimes\id}&\\
 A\otimes_{\max} C\ar{u}\ar{r}& A''\otimes_{\max} C\subset B(H).
\end{tikzcd}\]
commutes.
Since the horizontal arrow is injective, the vertial arrow is also injective.

(c)$\Rightarrow$(b)
This step is a corollary of ``The Trick''.
\end{pf}



\subsection{Exact C$^*$-algebras}


Let $\cI$ be a closed ideal of a C$^*$-algebra $A$.
For a C$^*$-algebra $C$, do we have an exact sequence
\[0\to\cI\otimes C\to A\otimes C\to(A/\cI)\otimes C\to0,\]
where $\otimes=\otimes_{\min}$ or $\otimes_{\max}$?
The answer is given as follows:
\begin{center}
\begin{tabular}{c|ccc}
exact&at left&at middle&at right\\\hline
$\otimes_{\min}$&True&False&True\\
$\otimes_{\max}$&True&True&True
\end{tabular}
\end{center}

\begin{defn}
A C$^*$-algebra $ C$ is called \emph{exact} if the following sequence exact for every C$^*$-algerba $A$ and its closed ideal $\cI$:
\[0\to\cI\otimes_{\min} C\to A\otimes_{\min} C\to(A/\cI)\otimes_{\min} C\to0.\]
\end{defn}

\iffalse
\[(A/\cI)\otimes_\alpha B\xrightarrow{\sim}\frac{ A\otimes_{\max} B}{\cI\otimes_{\max} B}\to(A/\cI)\otimes_{\max} B.\]
\fi


% 3.7.7. exactness passing to subalgebras
% Effros-Haagerup lifting theorem









\newpage
\section{June 16}

\subsection{Group C$^*$-algebras}

Let $G$ be a locally compact group, which is possibly not $\sigma$-compact.
We denote by $L^p(G)$ for $1\le p<\infty$ the $L^p$ space with respect to the \emph{left} Haar measure on $G$.
Let $L^\infty(G)$ be the Kolmogov quotient of the set of all locally Borel measurable functions on $G$ that is bounded except on locally null subsets together with the norm of essential supremum.
Then, we have a representation $L^1(G)^*\cong L^\infty(G)$.
The Fubini theorem for integrable functions are always fine since their support are $\sigma$-finite.


\begin{defn}[Left regular representations]
We will consider the actions $\alpha:G\to\Aut(L^1(G))$ and $\lambda:G\to B(L^2(G))$ defined as
\[\alpha_sf(t):=f(s^{-1}t),\qquad \lambda_s\xi(t):=\xi(s^{-1}t),\qquad \lambda_sx\lambda_s^*(t):=x(s^{-1}t)\]
for $f\in L^1(G)$, $\xi\in L^2(G)$, $x\in L^\infty(G)$.
The actions $\alpha$ and $\lambda$, called the \emph{left regular representations}, are continuous with respect to the strong operator topologies, i.e. the point-norm topology and the strict topology.
\end{defn}

\begin{defn}[Group C$^*$-algebras]
Let $G$ be a locally compact group.
Note that $C_c(G)$ has a (generally noncommutative) $*$-algebra structure
\[f*g(s):=\int f(t)g(t^{-1}s)\,d\mu(t),\qquad f^*(s):=\Delta(s^{-1})\bar{f(s^{-1})}.\]
The \emph{reduced group C$^*$-algebra} $C_r^*(G)$ and the \emph{full group C$^*$-algebra} $C^*(G)$ are defined as the completion of $C_c(G)$ with the following norms respectively:
\[\|f\|_{C_r^*(G)}:=\|\lambda(f)\|_{B(L^2(G))},\qquad\|f\|_{C^*(G)}:=\sup\{\|\pi(f)\|\mid\pi:C_c(G)\to B(H)\text{ $*$-hom}\},\]
where $\lambda:C_c(G)\to B(L^2(G))$ is given by $\lambda(f)\xi:=f*\xi$ for $\xi\in L^2(G)$.
The \emph{group von Neumann algebra} $L(G)$ is defined to be the double commutant $C_r^*(G)''$.
\end{defn}

\begin{rmk*}\,
\begin{itemize}
\item By convention, we consider $C_r^*(G)$ as a C$^*$-subalgebra of $B(L^2(G))$.
\item We have $\|f\|_{C_r^*(G)}\le\|f\|_{C^*(g)}\le\|f\|_{L^1(G)}$ because a homomorphism from a Banach algebra to a C$^*$-algebra is contractive. We can show the injectivity of $\lambda:L^1(G)\to C_r^*(G)$ using measure theory, and hence also $L^1(G)\subset C^*(G)$. Due to the closed image of $*$-homomorphisms, we have surjection $\lambda:C^*(G)\to C_r^*(G)$.
\item There is a natural one-to-one correspondences among (non-degenerate, unitary) representations of $G$, $L^1(G)$, and $C^*(G)$.
\item $L^1(G)$ has two order structure; as a $*$-subalgebra of group C$^*$-algebras, as a predual of $L^\infty(G)$.
\item If $\rho:C_c(G)\to B(L^2(G))$ is the right regular representation, then it is known that $L(G)=\rho(C_c(G))'$, so in particular we have $\lambda_s\in L(G)$ for $s\in G$. We will skip the proof. See Theorem 6.1.4 in [Brown-Ozawa].
\end{itemize}	
\end{rmk*}

\begin{lem}[Fell's absorption principle]
Let $G$ be a locally compact group.
\begin{parts}
\item For every unitary representation $\rho:G\to U(H)$, we have the unitary equivalence $\lambda\otimes\rho\cong\lambda\otimes\id_H$.
%\item The comultiplication $\Delta:C_c(G)\to M(C_c(G)\otimes C_c(G))$ can be extended to $\Delta:C_r^*(G)\to M(C_r^*(G)\otimes_{\min}C^*(G))$.(?)
\end{parts}
\end{lem}


\begin{defn}[Positive definite functions]
Let $G$ be a locally compact group.
A function $\f$ is called \emph{positive definite} if the matrix $[\f(s_i^{-1}s_j)]_{i,j}\in M_n(\C)$ is always positive semi-definite for all $n$ and $[s_i]\in G^n$.
The \emph{Fourier-Stieltjes algebra} $B(G)$ is the linear span of continuous positive definite functions.
We can check $B(G)$ is a commutative $*$-algebra.
\end{defn}

\begin{prop}
Let $G$ be a locally compact group and let $\f\in B(G)_{+,1}$, i.e.~$\f$ is continuous and positive definite satisfying $\f(e)=1$.
Define a linear functional $\omega:C^*(G)\to\C$ and a linear operator $M:C^*(G)\to C^*(G)$ called multiplier such that
\[\omega(f):=\int\f(s)f(s)\,\mu(s),\qquad M(f)(s):=\f(s)f(s),\qquad f\in C_c(G).\]
Endow on $B(G)_{+,1}$ the topology of uniform convergence on compact sets.
\begin{parts}
\item If $\f$ is compactly supported, then there is $\xi\in L^2(G)$ such that $\f(s)=\<\lambda_s\xi,\xi\>_{L^2(G)}$.
\item $\f\mapsto\omega$ defines a homeomorphism $B(G)_{+,1}\to S(C^*(G))$ with respect to the weak$^*$ topology.
\item $\f\mapsto M$ defines a continuous map $B(G)_{+,1}\to B(C^*(G))$ with respect to the point-norm topology.
\item If $\f$ is compactly supported, then $\omega$ and $M$ factor through $\lambda:C^*(G)\to C_r^*(G)$.
\end{parts}
\end{prop}
\begin{pf}
(
Since
\[\<\eta*\bar\f,\eta\>=\int\eta^**\eta(s)\bar{\f(s)}\,d\mu(s)\ge0,\]
the convolution operator $(-*\bar\f)\in B(L^2(G))$ is positive.
If we assume $\f(s)=\<\lambda_s\xi,\xi\>$ as an ansatz, then we have $\<\eta*\bar\f,\eta\>=\<\eta*\xi,\eta*\xi\>$.
For an approximate unit $e_\alpha$ of $C^*(G)$ in $C_c(G)$, the net $(-*\bar\f)^{\frac12}e_\alpha$ is Cauchy because
\[\|(-*\bar\f)^{\frac12}(e_\alpha-e_\beta)\|_{L^2(G)}^2=\<(e_\alpha-e_\beta)*\bar\f,e_\alpha-e_\beta\>\to0,\]
which follows from that we can take large $\alpha,\beta$ for any $\e>0$ such that
\[|\<e_\alpha*\bar\f,e_\beta\>-\bar{\f(e)}|\le\iint e_\alpha(s)|\f(s^{-1}t)-\f(e)|e_\beta(t)\,d\mu(s)\,d\mu(t)<\e.\]
Define $\xi:=\lim_\alpha(-*\bar\f)^{\frac12}e_\alpha$ so that $\eta*\xi=(-*\bar\f)^{\frac12}\eta$.
Then we can check $\f(s)=\<\lambda_s\xi,\xi\>$ by integration with arbitrarily taken $f\in C_c(G)$.

(b)
Skipped.
See Theorem A.10 in [Williams].

(c)
Suppose $\f_\alpha\to0$ uniformly on compacts.
Then, $\f_\alpha f\to0$ in $L^1(G)$ for $f\in C_c(G)$, so we have $\|\f_\alpha f\|_{C^*(G)}\le\|\f_\alpha f\|_{L^1(G)}\to0$.
By the density of $C_c(G)$ in $C^*(G)$, the continuity follows.

(d)
For the states, it is easy to check from (a) and the Fubini theorem that $\omega(f)=\<\lambda(f)\xi,\xi\>$.
For the multipliers, consdier the following commutative diagram:
\[\begin{tikzcd}
C^*(G) \ar{r}{\Delta}\ar{d}{\lambda} & M(C^*(G)\otimes_{\min}C^*(G)) \ar{r}{\omega\otimes\id}\ar{d}{\lambda\otimes\id} & C^*(G) \ar[equals]{d}\\
C_r^*(G) \ar{r}{\Delta} & M(C_r^*(G)\otimes_{\min}C^*(G)) \ar{r}{\omega_\xi\otimes\id} & C^*(G).
\end{tikzcd}\]
The diagonal map $\Delta$ on the second row is due to Fell's absorption principle.
Since $M$ can be realized as the compositioin $(\omega\otimes\id)\Delta$ at the first row, we are done. (We need more detail!)
\end{pf}



\subsection{Amenable groups}
\begin{defn}
Let $G$ be a locally compact group.
A \emph{mean} on $G$ is a state of $L^\infty(G)$.
We say $G$ is \emph{amenable} if there is an invariant mean $m$ on $L^\infty(G)$, i.e.~$m(\lambda_sx\lambda_s^*)=m(x)$ for $x\in L^\infty(G)$ and $s\in G$.
It is known to be equivalent to that $m(\lambda(f)x\lambda(f)^*)$ for $x\in L^\infty(G)$ and $f\in C_c(G)_{+,1}$.
See [Runde] or [Williams].
\end{defn}

\begin{thm}
Let $G$ be a locally compact group with a left Haar measure $\mu$.
Then, the following are equivalent:
\begin{parts}
\item $G$ is amenable.
\item $G$ satisfies the Day-Reiter condition: there is a net $f_\alpha\subset L^1(G)_{+,1}$ such that for each compact $K\subset G$,
\[\lim_\alpha\sup_{s\in K}\|\alpha_sf_\alpha-f_\alpha\|_{L^1(G)}=0\]
\item $G$ satisfies the F\o lner condition: there is a net $F_\alpha\subset G$ of Borel sets with $0<\mu(F_\alpha)<\infty$ such that for each compact $K\subset G$,
\[\lim_\alpha\sup_{s\in K}\frac{\mu(sF_\alpha\triangle F_\alpha)}{\mu(F_\alpha)}=0.\]
\item $G$ satisfies the Dixmier condition: there is a net of unit vectors $\xi_\alpha\in L^2(G)$ such that for each compact $K\subset G$,
\[\lim_\alpha\sup_{s\in K}\|\lambda_s\xi_\alpha-\xi_\alpha\|_{L^2(G)}=0.\]
\item There is a net of compactly supported continuous positive definite functions $\f_\alpha:G\to\C$ such that for each $K\subset G$,
\[\lim_\alpha\sup_{s\in K}|\f_\alpha(s)-1|=0.\]
\item $C^*(G)=C_r^*(G)$.
\end{parts}
\end{thm}
\begin{pf}
(a)$\Rightarrow$(b)
Let $m\in L^\infty(G)^*_{+,1}$ be an invariant mean.
Since the set $L^1(G)_{+,1}$ is exactly the set of normal states of $L^\infty(G)$, and since a state of a von Neumann algebra is $\sigma$-weakly approximated by normal states(we can show this similarly as Lemma 2.8), there exists a net $f_\alpha\in L^1(G)_{+,1}$ such that
\[\lim_\alpha|f_\alpha(x)-m(x)|=0,\qquad x\in L^\infty(G).\]

Fix any $h\in L^1(G)_{+,1}$ so that $h*f\in L^1(G)_{+,1}$ for any $f\in L^1(G)_{+,1}$.
Since for every $x\in L^\infty(G)$ we have
\begin{align*}
\lim_\alpha|h*f_\alpha(x)-f_\alpha(x)|
&\le\lim_\alpha|h*f_\alpha(x)-h*m(x)|+\lim_\alpha|m(x)-f_\alpha(x)|\\
&=\lim_\alpha|f_\alpha(h^**x)-m(h^**x)|+\lim_\alpha|m(x)-f_\alpha(x)|=0,
\end{align*}
the Mazur lemma implies that the closed convex hull of the set $\{h*f-f:f\in L^1(G)_{+,1}\}$ contains the zero.
In other words, we have a net $f_\alpha\in L^1(G)_{+,1}$ such that
\[\lim_\alpha\|h*f_\alpha-f_\alpha\|_{L^1(G)}=0\qquad h\in L^1(G)_{+,1}.\]

Let $K\subset G$ be compact and take $\e>0$.
Take an open neighborhood $U$ of the unit of $G$ such that $t^{-1}s\in U$ implies $\|\alpha_sh-\alpha_th\|_{L^1(G)}<\e$.
Using compactness, take a finite sequence $(s_i)\subset K$ such that for each $s\in K$ there is $s_i$ with $s_i^{-1}s\in U$.
Then,
\begin{align*}
\|\alpha_s(h*f_\alpha)-h*f_\alpha\|_{L^1(G)}
&\le\|(\alpha_sh-\alpha_{s_i}h)*f_\alpha\|_{L^1(G)}+\|(\alpha_{s_i}h)*f_\alpha-f_\alpha\|+\|f_\alpha-h*f_\alpha\|\\
&<\e+\max_i\|(\alpha_{s_i}h)*f_\alpha-f_\alpha\|+\|f_\alpha-h*f_\alpha\|
\end{align*}
implies
\[\lim_\alpha\sup_{s\in K}\|\alpha_s(h*f_\alpha)-h*f_\alpha\|_{L^1(G)}\le\e\]
for arbitrary $\e>0$, so the net $h*f_\alpha$ satisfies our desired condition.

(b)$\Rightarrow$(c)
For $f\in L^1(G)_{+,1}$ and $r>0$, define $F(f,r):=\{s\in G:f(s)>r\}$.
Then,
\begin{align*}
\|\alpha_sf-f\|_{L^1(G)}
&=\int|\alpha_sf(t)-f(t)|\,d\mu(t)\\
&=\int\int_0^\infty|1_{F(\alpha_sf,r)}(t)-1_{F(f,r)}(t)|\,dr\,\mu(t)\\
&=\int_0^\infty\int|1_{sF(f,r)}(t)-1_{F(f,r)}(t)|\,d\mu(t)\,dr\\
&=\int_0^\infty\mu(sF(f,r)\triangle F(f,r))\,dr,
\end{align*}
and
\begin{align*}
\int_0^\infty\mu(F(f,r))\,dr
&=\int_0^\infty\int1_{F(f,r)}(t)\,d\mu(t)\,dr\\
&=\int\int_0^{f(t)}\,dr\,d\mu(t)\\
&=\int f(t)\,d\mu(t)=1.
\end{align*}

Suppose a net $f_\alpha$ satisfies the Day-Reiter condition.
Then, for arbitrary $\e>0$ we have sufficiently large $\alpha$ such that
\[\int_0^\infty\mu(sF(f_\alpha,r)\triangle F(f_\alpha,r))\,dr=\|\alpha_sf-f\|_{L^1(G)}<\e=\e\int_0^\infty\mu(F(f_\alpha,r))\,dr\]
so there is $r_\alpha>0$ such that
\[\mu(sF(f_\alpha,r_\alpha)\triangle F(f_\alpha,r_\alpha))<\e\mu(F(f_\alpha,r_\alpha)).\]
We are done if we take $F_\alpha:=F(f_\alpha,r_\alpha)$.

(c)$\Rightarrow$(d)
Let $F_\alpha$ be a net of subsets satisfying the F\o lner condition.
Define $\xi_\alpha:=\mu(F_\alpha)^{-\frac12}\in L^2(G)$.
Then,
\[\|\lambda_s\xi_\alpha-\xi_\alpha\|_{L^2(G)}^2=\int\mu(F_\alpha)^{-1}|1_{sF_\alpha}(t)-1_{F_\alpha}(t)|\,d\mu(t)=\frac{\mu(sF_\alpha\triangle F_\alpha)}{\mu(F_\alpha)}\]
proves the claim.

(d)$\Rightarrow$(e)
Let $\xi_\alpha$ be a net of unit vectors in $L^2(G)$ satisfying the Dixmier condition.
Take compact sets $F_\alpha\subset G$ such that $0<\mu(F_\alpha)<\infty$ and $\|\xi_\alpha-\xi_\alpha1_{F_\alpha}\|<\e_\alpha$ for each $\alpha$, where $\e_\alpha\downarrow0$.
Since if we consider $\tilde\xi_\alpha:=\xi_\alpha1_{F_\alpha}/\|\xi_\alpha1_{F_\alpha}\|$, then we have
\[\|\lambda_s\tilde\xi_\alpha-\tilde\xi_\alpha\|<\|\lambda_s\xi_\alpha-\xi_\alpha\|+4\e_\alpha,\]
we may assum $\xi_\alpha$ is compactly supported.
Define $\f_\alpha:G\to\C$ such that $\f_\alpha:=\<\lambda_s\xi_\alpha,\xi_\alpha\>$, which is clearly continuous, positive definite, and compactly supported function.
We can also check $|\f_\alpha(s)|\le1$ and $\|\lambda_s\xi_\alpha-\xi_\alpha\|_{L^2(G)}^2=2-2\Re\f_\alpha(s)$ imply the compact convergence of $\f_\alpha\to1$.

(e)$\Rightarrow$(f)
Let $\f_\alpha$ be the net of compactly supported continuous positive definite functions satisfying the condition of (e).
Since $\f_\alpha$ has the compact support, the associated multiplication operator $M_\alpha$ factors through $\lambda:C^*(G)\to C_r^*(G)$.
Since the constant unity function $1\in B(G)_{+,1}$ corresponds to the identity operator in $B(C^*(G))$, the net $M_\alpha$ converges to the identity strongly.
Therefore, if $a\in C^*(G)$ satisfies $\lambda(a)=0$, then
\[a=\lim_\alpha M_\alpha(a)=0.\]
The injectivity of $\lambda$ implies the equality $C^*(G)=C_r^*(G)$.

(f)$\Rightarrow$(a)
Take any one-dimensional representation $L^1(G)\to\C:f\mapsto\int f(s)\,d\mu(s)$ and extend to obtain a $*$-homomorphism $C_r^*(G)=C^*(G)\to\C$.
It is extended again to a state $\f:B(L^2(G))\to\C$, whose multiplicative domain contains $C_r^*(G)$.
Since $\lambda(f)\in C_r^*(G)$ for any $f\in L^1(G)$, we have
\[\f(\lambda(f)x\lambda(f)^*)=\f(\lambda(f))\f(x)\f(\lambda(f)^*)=\f(x),\qquad x\in L^\infty(G).\]
Therefore, $\f|_{L^\infty(G)}$ is a left-invariant mean.
\end{pf}


\begin{lem}[Markov-Kakutani fixed point theorem]
Let $K$ be a compact convex subset of a topological vector space $X$.
Let $\cT$ be a commuting family of continuous linear operators $T:X\to X$ such that $TK\subset K$.
Then, $K$ has a point $x$ such that $Tx=x$ for all $T\in\cT$.
\end{lem}
\begin{pf}
We may assume $\cT$ is convex and closed under the multiplication.
Let
\[T^{(n)}:=\frac1n\sum_{k=0}^{n-1}T^k\]
Since
\[\left\{T^{(n)}K:T\in\cT,n\in\N\right\}\subset K\]
satisfies the finite intersection property.
Take $x$ in the intersection.
We claim that this $x$ is a fixed point.

Fix $T\in\cT$ and take any neighborhood of $U$ of zero in $X$.
Since $K-K$ is bounded, we have an integer $n$ such that $K-K\subset nU$.
With this $n$, because $x\in T^{(n)}K$, we have $y\in K$ such that $T^{(n)}y=x$ so that
\[Tx-x=\frac1n(T^ny-y)\in\frac1n(K-K)\subset U.\]
Since $U$ is arbitrary, we have $Tx=x$.
\end{pf}

\begin{prop}
Let $G$ be a locally compact group.
\begin{parts}
\item If $G$ is compact, then it is amenable.
\item If $G$ is abelian, then it is amenable.
\end{parts}
\end{prop}
\begin{pf}
(a)
The normalized left Haar measure gives rise to a left-invariant state of $L^\infty(G)$.

(b)
Define linear operators $T_s:L^\infty(G)^*\to L^\infty(G)^*$ such that $T_sm(x):=m(\lambda_sx\lambda_s^*)$ for $s\in G$ and $x\in L^\infty(G)$.
We can check that $T_s(S(L^\infty(G)))\subset S(L^\infty(G))$ and is continuous with respect to the weak$^*$ topology because it is given by the dual operator in $B(L^\infty(G))$.
Then, since $\{T_s:s\in G\}$ is commuting, by the Markov-Kakutani fixed point theorem, we have a left-invariant mean for $G$.
\end{pf}

\begin{prop}
Let $G$ be a locally compact group and let $H\le G$ and $N\lhd G$ be closed.
Then, we have the following permanence properties.
\begin{parts}
\item If $G$ is amenable, then so is $H$.
\item If $G$ is amenable, then so is $G/N$.
\item If $N$ and $G/N$ are amenable, then so is $G$.
\item The direct limit of amenable groups is amenable.
\end{parts}
\end{prop}
\begin{pf}
See [Paterson].
\end{pf}

\subsection{Discrete amenable groups}


Let $\Gamma$ be a discrete group.
Then,
\begin{itemize}
\item $C_r^*(\Gamma)$ and $C^*(\Gamma)$ are unital.
\item $c_0(\Gamma)^*\cong\ell^1(\Gamma),
\quad\ell^1(\Gamma)^*\cong\ell^\infty(\Gamma),
\quad\ell^\infty(\Gamma)^*\cong ba(\Gamma)$.
We skip the proof of this.
\item Any function $\Gamma\to X$ is continuous.
\item We can write $C_c(C, A)= A\Gamma$.
\end{itemize}
A mean on $\Gamma$ is identified to a finitely additive probability measure on $\Gamma$.

\begin{ex}[Subexponential growth]
\end{ex}

\begin{ex}[Elementary amenable groups]
Every finite group and every abelian group is an discrete amenable group.
Their finite number of elementary operations of four types (subgroups, quotients, extensions, direct limits) are also amenable, and they are called \emph{elementary amenable groups}.
\end{ex}

\begin{ex}[Nonabelian free groups]
The free group $F_2$ is not amenable.
In particular, every discrete group containing $F_2$, for example, $F_n$ or $\SL(n,\Z)$ for $n\ge2$, is not amenable.
\end{ex}
\begin{pf}
Suppose there is a left-invariant finitely additive probability measure $m$ on $F_2=\<a,b\>$.
Its value $m(S)$ of finite set $S$ is zero.
Let $S(a)$ be the set of all reduced words in $F_2$ starting with $a$, and similarly for other generators.
Then, $F_2=S(a)\sqcup aS(a^{-1})$ implies $m(S(a))+m(S(a^{-1}))=1$ and
\[F_2=\{e\}\sqcup S(a)\sqcup S(a^{-1})\sqcup S(b)\sqcup S(b^{-1})\]
implies
\begin{align*}
1=m(F_2)=m(S(a))+m(S(a^{-1}))+m(S(b))+m(S(b^{-1}))=2,
\end{align*}
a contradiction.
\end{pf}

\begin{rmk}
We have the following implications for a discrete group $\Gamma$:
\[\text{elementary amenable}\quad\Longrightarrow\quad\text{amenable}\quad\Longrightarrow\quad\text{no non-abelian free subgroup}\]
The question about the converses had been a longstanding problem, which is also called the \emph{von Neumann-Day problem}.
Grigorchuk's group(1984) is an amenable that is not elementary amenable, and the Tarski monster group(1980) is a non-amenable group without non-abelian free subgroup.
The Tits alternative states that the three statements are equivalent for a linear group over a field.
\end{rmk}


\begin{thm}[Nuclearity]
Let $\Gamma$ be a discrete group.
Then, the following are equivalent:
\begin{parts}
\item $\Gamma$ is amenable.
\item $C_r^*(\Gamma)$ is nuclear.
\item $C^*(\Gamma)$ is nuclear.
\end{parts}
\end{thm}
\begin{pf}
(a)$\Rightarrow$(b)
Take a F\o lner sequence $F_k\subset\Gamma$.
Define $e_{s,t},p_k\in B(\ell^2(\Gamma))$ such that
\[e_{s,t}\xi:=\<\xi,\delta_t\>\delta_s,\qquad p_k:=\sum_{s\in F_k}e_{s,s}.\]
Define $C_r^*(\Gamma)\xrightarrow{\f_k}p_kB(\ell^2(\Gamma))p_k\xrightarrow{\psi_k}C_r^*(\Gamma)$ such that
\[\f(a):=p_kap_k,\qquad\psi(e_{s,t}):=\frac1{|F_k|}\lambda_{st^{-1}}.\]
Since
\begin{align*}
(p_k\lambda_sp_k)\delta_t
&=p_k\lambda_s\delta_t1_{F_k}(t)
=p_k\delta_{st}1_{F_k}(t)
=\delta_{st}1_{F_k}(t)1_{F_k}(st)
=1_{sF_k\cap F_k}(st)\delta_{st}\\
&=\sum_{p\in sF_k\cap F_k}\<\delta_{st},\delta_p\>\delta_p
=\sum_{p\in sF_k\cap F_k}e_{p,s^{-1}p}\delta_t,
\end{align*}
we have
\[\psi_k\circ\f_k(\lambda_s)=\psi_k\left(\sum_{p\in sF_k\cap F_k}e_{p,s^{-1}p}\right)=\sum_{p\in sF_k\cap F_k}e_{p,s^{-1}p}\frac1{|F_k|}\lambda_s=\frac{|sF_k\cap F_k|}{|F_k|}\lambda_s\]
for all $s\in\Gamma$.
Therefore, $\psi_k\circ\f_k$ converges to the identity in the point-norm topology, and $C_r^*(\Gamma)$ is nucelar.

(b)$\Rightarrow$(a)
Consider a net of c.c.p.~maps $C_r^*(\Gamma)\xrightarrow{\f_\alpha}M_{n_\alpha}(\C)\xrightarrow{\psi_\alpha}C_r^*(\Gamma)$ which converges to the identity in the point-norm topology.
By the Arveson extension theorem, we have $\tilde\f_\alpha:B(\ell^2(\Gamma))\to M_{n_\alpha}$ which extends $\f_\alpha$.
Then, the bounded net $\psi\circ\tilde\f_\alpha:B(\ell^2(\Gamma))\to C_r^*(\Gamma)\subset L(\Gamma)$ has a limit point in the point-$\sigma$-weak topology.
By taking a subnet, denote by $\Phi$ the limit.
Then, $\Phi(a)=a$ for $a\in C_r^*(\Gamma)$.
See the following diagram.
\[\begin{tikzcd}
B(\ell^2(\Gamma)) \ar[dashed]{rddd}{\tilde\f_\alpha}\ar{rrrdd}{\Phi:=\lim_\alpha\psi\circ\tilde\f_\alpha}&&&\\
&&&\\
C_r^*(\Gamma)\ar{rr}\ar[swap,dashed]{rd}{\f_\alpha}\ar[hook]{uu}&&C_r^*(\Gamma)\ar[hook]{r}&L(\Gamma)=C_r^*(\Gamma)''.\\
&M_{n_\alpha}(\C)\ar[swap,dashed]{ur}{\psi_\alpha}&&
\end{tikzcd}\]

Consider a vector state $\omega_{\delta_e}:x\mapsto\<x\delta_e,\delta_e\>$ on $L(\Gamma)\subset B(\ell^2(\Gamma))$.
It is tracial because the right-hand side of
\[\omega_{\delta_e}(ab)=\<ab\delta_e,\delta_e\>=\sum_{s,t}\<a_sb_t\delta_{st},\delta_e\>=\sum_sa_sb_{s^{-1}}\]
is symmetric.
Then, the state $m:=\omega_{\delta_e}\circ\Phi$ on $B(\ell^2(\Gamma))$ is left-invariant since $C_r^*(\Gamma)$ belongs to the multiplicative domain of $m$ so that
\[m(\lambda_sx\lambda_s^*)=\omega_{\delta_e}(\Phi(\lambda_sx\lambda_s^*))=\omega_{\delta_e}(\lambda_s\Phi(x)\lambda_s^*))=\omega_{\delta_e}(\Phi(x))=m(x)\]
for $x\in B(\ell^2(\Gamma))$.
Therefore, $m|_{\ell^\infty(\Gamma)}$ is an invariant mean on $\Gamma$.

(a)$\Rightarrow$(c)
If $\Gamma$ is amenable, then $C_r^*(\Gamma)$ is amenable and $C^*(\Gamma)=C_r^*(\Gamma)$, so we are done.

(c)$\Rightarrow$(a)
It is known that a quotient of a nuclear C$^*$-algebra is nuclear, which requires a rather difficult proof in this stage, so we give a direct proof using the trick.

Suppose $C^*(\Gamma)$ is nuclear.
Here we denote by $C_\lambda^*(\Gamma)$ and $C_\rho^*(\Gamma)$ the reduced group C$^*$-algebras in $B(\ell^2(\Gamma))$ generated by the image of the left and right regular representations.
Then the maximal tensor products in the following diagram can be changed into the minimal tensor products:
\[\begin{tikzcd}
B(\ell^2(\Gamma))\otimes_{\max}C^*(\Gamma) &\\
C_\lambda^*(\Gamma)\otimes_{\max}C^*(\Gamma)\ar[hook]{r}{i\times\rho} \ar{u} & B(\ell^2(\Gamma)),
\end{tikzcd}\]
where $\rho:C^*(\Gamma)\to C_\rho^*(\Gamma)\subset B(\ell^2\Gamma)$ is the right regular representation.
Then, we have two facts: the vertical arrow is injective because the minimal tensor products can be implemented spatially in a large Hilbert space(i.e.~the conclusion (c) of the inclusion problem in Corollary 3.8 always holds for minimal tensor product), and $\rho(C^*(\Gamma))'=C_\rho^*(\Gamma)'=L(\Gamma)$.
By the trick, the embedding $C_\lambda^*(\Gamma)\hookrightarrow L(\Gamma)$ is extended to $B(\ell^2(\Gamma))\to L(\Gamma)$.
Then, as we did in the proof of (b)$\Rightarrow$(a), pullback the tracial vector state $\omega_{\delta_e}$ of $L(\Gamma)$ to get a left-invariant state of $B(\ell^2(\Gamma))$, which contains $\ell^\infty(\Gamma)$.
The restriction on $\ell^\infty(\Gamma)$ implies that $\Gamma$ is amenable.
\end{pf}

\begin{rmk}
It is known that a locally compact group $G$ is amenable if and only if $C_r^*(G)$ is nuclear and admits a tracial state[Ng, 2015].
Note that if $\Gamma$ is discrete, then the reduced group C$^*$-algebra $C_r^*(\Gamma)$ has a tracial state $a\mapsto\<a\delta_e,\delta_e\>$.
\end{rmk}

\iffalse
General locally compact case:
For $G$ a connected Lie group, $C_r^*(G)$ is nuclear, but $G$ may not be amenable.

Theorem 3.7.11: $\Gamma$ with property (F) is amenable iff $C^*(\Gamma)$ is exact.
Chapter 5: When is $-\otimes_{\min}C_r^*(\Gamma)$ or $-\rtimes_r\Gamma$ exact?
\fi




\iffalse

\newpage
\subsection{Crossed products}

\begin{defn}[C$^*$-dynamical system]
A \emph{C$^*$-dynamical system} is a continuous group homomorphism $\alpha:G\to\Aut(A)$, where $G$ is a locally compact group, $A$ is a C$^*$-algebra, and $\Aut(A)$ is the group of $*$-automorphisms with the point-norm topology.
\end{defn}
\begin{defn}[Covariant representation]
Let $\alpha:G\to\Aut(A)$ be a C$^*$-dynamical system.
A \emph{covariant representation} of $\alpha$ is a pair $(\pi,\rho)$ of representations $\pi:A\to B(H)$ and $\rho:G\to U(H)$ such that $\pi(\alpha_sa)=\rho_s\pi(a)\rho_s^*$ for every $a\in A$ and $s\in G$.
\end{defn}

locally compact left $G$-spaces <-> commutative C$^*$-dynamical systems


\begin{defn}[Induced representation]

\end{defn}
$C_c(G, A)$ is a $*$-algebra
with
\[f*g(s):=\int_Gf(t)\alpha_tg(t^{-1}s)\,dt,\qquad f^*(s):=\Delta(s^{-1})(\alpha_{s^{-1}}f(s^{-1}))^*.\]

covariant representations <-> $L^1$-contractive $*$-homs $C_c(G, A)\to B(H)$
\begin{defn}

\end{defn}


Day's fixed point theorem

discrete: $A\rtimes\Gamma$ contains copies of $A$ and $\Gamma$.

\begin{thm}
Let $\alpha:\Gamma\to\Aut(A)$ be a discrete C$^*$-dynamical system.
Then, the following are equivalent:
\begin{parts}
\item The action $\alpha$ is amenable.
\item $A\rtimes_\alpha\Gamma= A\rtimes_{\alpha,r}\Gamma$.
\item $A$ is nuclear if and only if $A\rtimes_\alpha\Gamma$ is nuclear.
\item $A$ is exact if and only if $A\rtimes_\alpha\Gamma$ is exact.
\end{parts}
\end{thm}
\fi


\newpage
\section{August 4}

\subsection{Hilbert C$^*$-modules}

\begin{defn}
Let $A$ be a C$^*$-algebra.
An (right) \emph{inner product $A$-module} is a complex vector space $E$ together with
\begin{enumerate}[(i)]
\item a ring anti-homomorphism $A^{\mathrm{op}}\to\End_\C(E)$
\item an $A$-valued inner product $\<-,-\>:E\times E\to A$.
\end{enumerate}
On a inner product $A$-module $E$ we have a scalar-valued norm given by $\|\xi\|:=\|\<\xi,\xi\>\|^{\frac12}$ for $\xi\in E$, and we say $E$ is a \emph{Hilbert $A$-module} if the norm is complete.
\end{defn}

Note that we have the sesquilinear map to be linear in the second argument because $E$ is a right $A$-module.
Boundedness follows from the following proposition.

\begin{prop}
$A$-valued norm $|\xi|:=\<\xi,\xi\>^{\frac12}$.
\begin{parts}
\item $|\<\xi,\eta\>|\le\|\xi\||\eta|$ in $A$. In particular, $\|\<\xi,\eta\>\|\le\|\xi\|\|\eta\|$
\item $|\xi a|\le\|\xi\||a|$ in $A$. In particular, $\|\xi a\|\le\|\xi\|\|a\|$.
%\item $E A$ is dense in $A$.
%\item $E\<E,E\>$ is dense in $E$, but generally $\<E,E\>$ is not dense in $A$(non-full case).
\end{parts}
\end{prop}
\begin{pf}
(a)
If we let $\|\xi\|=1$ and $a:=\<\xi,\eta\>$, then it is clear from
\begin{align*}
0&\le\<\xi a-\eta,\xi a-\eta\>\\
&=a^*\<\xi,\xi\>a-\<\eta,\xi\>a-a^*\<\xi,\eta\>+\<\eta,\eta\>\\
&\le a^*a-\<\eta,\xi\>a-a^*\<\xi,\eta\>+\<\eta,\eta\>\\
&\le-|\<\xi,\eta\>|^2+\<\eta,\eta\>.
\end{align*}

(b)
If we let $\|\xi\|=1$, then $\<\xi a,\xi a\>\le a^*\<\xi,\xi\>a\le a^*a$.
\end{pf}

Therefore, a right Hilbert $A$-module is a right Banach $A$-module, i.e.~we have a homomorphism $A^{\mathrm{op}}\to\cL(E)$.
Also, the right action $A^{\mathrm{op}}\to\cL(E)$ is always non-degenerate, and faithful if and only if $E$ is full, i.e.~$\<E,E\>$ is dense in $A$.


\begin{defn}[Adjointable operators]
Let $A$ be a C$^*$-algebra.
Let $E$ and $E'$ be Hilbert $A$-modules.
An \emph{adjointable operator} is a linear operator $t:E\to E'$ such that there is a linear operator $t^*:E'\to E$ satisfying $\<t\xi,\eta\>=\<\xi,t^*\eta\>$ for all $\xi\in E$ and $\eta\in E'$.
The set of adjointable operators are denoted by $B(E,E')$.
\end{defn}

\begin{prop}[Adjointable operators]
Let $A$ be a C$^*$-algebra.
Let $E$ and $E'$ be a Hilbert $A$-modules.
\begin{parts}
\item The sum, scalar multiplication, compositioin, and adjoint of adjointable operators are adjointable.
\item An adjointable operator is $A$-linear and bounded.
\item $B(E):=B(E,E)$ is a C$^*$-algebra.
\item $|t\xi|\le\|t\||\xi|$ in $A$ for $t\in B(E,E')$. In particular, $\|t\xi\|\le\|t\|\|\xi\|$.
\end{parts}
% $B(E)\leftarrow A$ is a $*$-anti-homomorphism? what is it?
\end{prop}
\begin{pf}
(a)
Clear.

(b)
Let $t$ be an adjointable operator from $E$ to $E'$.
It is $A$-linear because
\[\<t(\xi a),\eta\>=\<\xi a,t^*\eta\>=a^*\<\xi,t^*\eta\>=a^*\<t\xi,\eta\>=\<(t\xi)a,\eta\>\]
for any $\eta\in E'$ implies $t(\xi a)=(t\xi)a$.

Consider $\{\<t\xi,-\>:\|\xi\|\le1\}\subset B(E', A)$ a family of operators between Banach spaces.
Since it is pointwise bounded because $\|\<t\xi,\eta\>\|\le\|t^*\eta\|$ for each $\eta\in E'$, by the uniform boundedness principle, $\|\<t\xi,\eta\>\|\le C\|\eta\|$.
Then, $t$ is bounded because $\|t\xi\|^2\le\|\<t\xi,t\xi\>\|\le C\|t\xi\|$ for $\|\xi\|\le1$.

(c)
Note that
\[\|t^*t\|\ge\sup\frac{\|\<t^*t\xi,\xi\>\|}{\|\xi\|^2}=\sup\frac{\|\<t\xi,t\xi\>\|}{\|\xi\|^2}=\|t\|^2.\]
Therefore, the C$^*$-identity holds and we have $\|t\|=\|t^*\|$.
Since if a sequence of adjointable operators $t_n$ is Cauchy then the sequence of their adjoints $t_n^*$ is also Cauchy, $B(E,E)$ is a closed subspace of $L(E,E)$.

(d) (geometric series of Cauchy-Schwarz inequality).
\end{pf}


\iffalse
\begin{defn}[Strict topology]
The locally convex topology generated by seminorms $t\mapsto\|t\xi\|$ and $t\mapsto\|t^*\xi'\|$ indexed by $\xi\in E$ and $\xi'\in E'$.
\end{defn}
\begin{prop}[Density theorem]
\begin{parts}
\item The unit ball of $K(E,E')$ is strictly dense in the unit ball of $B(E,E')$.
\item The unit ball of any strictly dense C$^*$-subalgebra of $B(E)$ is dense in the unit ball of $B(E)$.(A generalization of Kaplansky density)
\end{parts}
\end{prop}
\begin{thm}[Riesz representation]
$t_\xi:\eta\mapsto\<\xi,\eta\>$ is an adjointable operator $E\to A$, and defines an isometry $E\to K_ A(E, A)$. Conversely, every element of $K_ A(E, A)$ is given by $t_\xi$ for some $\xi$.
\end{thm}
self-dual Hilbert C$^*$-modules, in which possibly non-adjointable operator $E\to A$ can be given by an inner product.
\fi






\begin{ex}[Hilbert $C(X)$-modules]
Let $X$ be a compact Hausdorff space.
A \emph{Banach bundle} over $X$ is a continuous map $p:E\to X$ from a topological space $E$ such that each fiber $p^{-1}(x)$ is given a compatible Banach space structure and satisfies
\begin{enumerate}[(i)]
\item the summation $E\times_X E\to E$ is continuous,
\item the scalar multiplication $E\times\F\to E$ is continuous,
\item the norm function $X\to\R_{\ge0}$ is continuous,
\item for a net $e_i\in E$ if $p(e_i)\to x$ and $\|e_i\|\to0$, then $e_i\to 0\in E_x$.
\end{enumerate}
If each fiber is a Hilbert space, then we call the Banach bundle as a \emph{Hilbert bundle}.
\begin{parts}
\item There is an equivalence between the category of Hilbert bundles over $X$ is equivalent to the category of Hilbert $C(X)$-modules,
via the functor $(E\to X)\mapsto\Gamma(E)$, the set of continuous sections of $E$.
\item Not every Hilbert bundle is locally trivial.
\item If the fibers have a common finite dimension, then the Hilbert bundle is locally trivial.
\end{parts}
\end{ex}
\begin{pf}[Proof sketch for (a)]
For every Hilbert bundle $E$ over $X$, we can easily check the space of sections $\Gamma(E)$ is a Hilbert $C(X)$-module.
Conversely, let $E$ be a Hilbert $C(X)$-module.
Consider the evaluation map $\mathrm{ev}_x:C(X)\to\C$ and let $\fm_x:=\mathrm{ev}^{-1}(0)$ be the maximal ideal at $x\in X$.
Then, we can show that $E\fm_x$ is a Hilbert $C(X)$-submodule of $E$.
If $\xi_nf_n$ converges to $\eta\in E$, then we can show that
\[\eta=\left(\lim_{n\to\infty}\eta(n^{-1}+|\eta|^{\frac12})^{-1}\right)|\eta|^{\frac12}\in E\fm_x.\]
Alternitively, we may apply the Cohen factorization theorem.

Now we define $E_x:=E/E\fm_x$.
We topologize the set $E:=\bigcup_{x\in X}E_x$ by the quotient topology with quotient map $E\times X\to E$.
Then, we can check the four conditions for Banach bundles.
\end{pf}

\begin{ex}
Let $A:=C([0,1])$ and $\cI:=C_0((0,1])$ be Hilbert $A$-modules.
The inclusion $\iota:\cI\to A$ is not adjointable because $i=\<\iota i,1\>=\<i,\iota^* 1\>$ implies $\iota^*1=1$, which is impossible.
The $A\oplus\cI\to A\oplus\cI$ that maps $(\xi_1,\xi_2)$ to $(\xi_2,0)$ is bounded but not adjointable.
\end{ex}

\begin{ex}[Hilbert modules for noncommutative C$^*$-algebras]
Let $A$ be a C$^*$-algebra.
\begin{parts}
\item (Scalar change) Every Hilbert space $H$ is exactly same to the Hilbert $\C$-module. The algebraic vector space tensor product $H\odot_\C A$ has a natural inner product $A$-module structure. The completion $H_ A:=H\otimes A=H\otimes_\C A$ is a Hilbert $A$-module.
\item (Trivial Hilbert module) $A^n$ is a $A$-module.
\item (Standard Hilbert module) $\ell^2(\N, A)$.
%\item proper actions.
%\item groupoid algebra
\end{parts}
\end{ex}

\iffalse
\begin{ex}[Basic constructions]
product, coproducts, pullbacks, pushouts, direct limits
complementations(sub,quo), tensor products........??
localization
\begin{parts}
\item a
\item (Localization) 
Let $A\subset B$ be C$^*$-algebras and let $1_ B\in A$.
Let $\f: B\to A$ be a conditional expectation.
Suppose 
Let $E$ be a Hilbert $A$-mmodule.
\end{parts}
\end{ex}
\fi


\begin{defn}[Compact adjointable operators]
Define $K(E,E')$ to be the closed linear subspace of $B(E,E')$ spanned by adjointable operators of the form $\theta_{\xi,\eta}:=\xi\<\eta,-\>$.
The elements of $K(E,E')$ are often called \emph{compact operators} even though they are not compact operators between Banach spaces.
\end{defn}

\begin{prop}[Compact adjointable operators]\,
\begin{parts}
\item $K(E)$ is a closed ideal of $B(E)$.
\item $K(A)\cong A$.
%\item $M_{m\times n}(K(E,E'))\cong K(E^n,E'^m)$ and $M_{m\times n}(B(E,E'))\cong B(E^n,E'^m)$.
\end{parts}
\end{prop}
\begin{pf}
(a)
Note that
\[\theta_{\xi,\eta}\theta_{\xi',\eta}=\theta_{\xi\<\eta,\xi'\>,\eta'},\qquad t\theta_{\xi,\eta}=\theta_{t\xi,\eta},\qquad\theta_{\xi,\eta}t=\theta_{\xi,t^*\eta}.\]
It shows that the linear span of $\theta_{\xi,\eta}$ is an ideal, and its closure is closed ideal of $B(E)$.

(b)
Let $\cT\subset A$ be the linear span of $\theta_{a,b}$. 
Consider the isometry $\cT\to A:\theta_{a,b}\mapsto ab^*$.
Since $\{ab^*\}$ is dense in $A$, we are done.

\end{pf}









\subsection{Multipliers}

Let $A$ be a non-unital C$^*$-algebra.

\begin{defn}[Multiplier algebra]
Let $A$ be a C$^*$-algebra.
The \emph{multiplier algebra} is a maximal C$^*$-algebra denoted by $M(A)$ which contains $A$ as an essential ideal.
We will show that $M(A)$ always uniquely exists up to $*$-isomorphism.
\end{defn}

Stone-\v Cech compactification
double centralizer approach


\begin{prop}[Ideal extension of representation]
Let $A, B, C$ be C$^*$-algebras, and suppose $A$ is a closed ideal of $B$.
Let $E$ be a Hilbert $ C$-module and $\pi:A\to B_ C(E)$ be a representation.
\begin{parts}
\item
There is a representation $\tilde\pi: B\to B_ C(E)$ which extends $\pi$.
\item If $\pi$ is non-degenerate, then the extension is unique.
\item If $\pi$ is faithful and $A$ is essential in $B$, then the extension is faithful.
\end{parts}
\end{prop}
\begin{pf}
(a)

\end{pf}

\begin{prop}[Idealizer description of multiplier algebra]
Let $\pi:A\to B_ A(E)$ be a non-degenerate injective representation.
Let
\[ B:=\{b\in B_ A(E):b\pi(a),\pi(a)b\in\pi(A)\ \forall a\in A\}.\]
Then, $\pi$ extends to a $*$-isomorphism between $M(A)$ and $B$
\end{prop}
\begin{pf}

\end{pf}

\begin{ex}\,
\begin{parts}
\item
Since the inclusion representation $K_ A(E)\to B_ A(E)$ is non-degenerate, the above proposition implies that $B_ A(E)$ is the multiplier algebra of $K_ A(E)$(Kasparov).
\item $M(K(H))\cong B(H)$.
\item $M(C_0(X))\cong C_b(X)$ from $C_0(X)\to B(\ell^2(X))$.
\end{parts}
\end{ex}

\begin{defn}
A \emph{morphism} from $A$ to $B$ is a non-degenerate $*$-homomorphism $A\to M(B)$.
Every morphism uniquely extends to a $*$-homomorphism $M(A)\to M(B)$, and the injectivity is preserved.
The composition is well-defined in the sense of extensions, and the non-degeneracy is proved by the following proposition.
\end{defn}
\begin{lem}[Non-degeneracy]
For a representation $\pi:A\to B_ B(E)$, TFAE:
\begin{parts}
\item $\pi$ is non-degenerate
\item $\pi$ is the restriction to $A$ of a unital $*$-homomorphism $\tilde\pi:M(A)\to B_ B(E)$ which is strictly continuous on the unit ball.
\item $\pi(e_\alpha)\to\id_E$ strictly.
\end{parts}
\end{lem}





complemented
Mi\u s\u cenko's closed range and complemented submodule theorem
isometric surjective <=> unitary

tensor product



We have $K_ A(H\otimes A)\cong K(H)\otimes A$. Note that $K(H)=\lim_\to M_n(\C)$ is nuclear.





\subsection{C$^*$-correspondence}



\begin{defn}[C$^*$-correspondences]
Let $A$ and $B$ be C$^*$-algebras.
A \emph{C$^*$-correspondence} over $A$ and $B$, or a \emph{Hilbert $A$-$B$-bimodule}, is a Hilbert $B$-module $E$ together with a $*$-homomorphism $A\to B_B(E):a\mapsto(\xi\mapsto a\xi)$, called the \emph{left action}.
The C$^*$-correspondence $E$ is called \emph{faithful} or \emph{non-degenerate} if the left action is faithful or non-degenerate respectively.
\end{defn}


\begin{defn}[Representations of a C$^*$-correspondence]
For a C$^*$-correspondence $E$ over $A$, a \emph{representation} of $E$ on $B$ is a pair $(\pi,\tau)$ of a $*$-homomorphism $\pi:A\to B$ and a complex linear map $\tau:E\to B$ such that
\[\pi(\<\xi,\eta\>)=\tau(\xi)^*\tau(\eta),\qquad\tau(a\xi)=\pi(a)\tau(\xi),\qquad\tau(\xi a)=\tau(\xi)\pi(a),\qquad\xi\in E,\ a\in A.\]
In fact, the third condition is redundant because it follows from the first two, and $\tau$ is automatically contractive.
The C$^*$-subalgebra of $B$ generated by $\pi(A)$ and $\tau(E)$ is denoted by $C^*(\pi,\tau)$.
\end{defn}

From now on, let $E$ be a C$^*$-correspondence over a C$^*$-algebra $A$, and let $(\pi,\tau)$ be a representation of $E$ on a C$^*$-algebra $B$.

\begin{defn}
We define complex linear maps
\[\tau_n:E^{\otimes n}\to B,\qquad\psi:K(E)\to B,\qquad\psi_n:K(E^{\otimes n})\to B,\]
defined by
\[\tau_n(\xi_1\otimes\cdots\xi_n):=\tau(\xi_1)\cdots\tau(\xi_n),\qquad\xi_1,\cdots,\xi_n\in E\]
and
\[\psi(\theta_{\xi,\eta}):=\tau(\xi)\tau(\eta)^*,\qquad\xi,\eta\in E.\]

We have a representation $(\pi,\tau_n)$ of the interior tensor product $E^{\otimes n}$.
In particular, $\tau_0=\pi$ and $\tau_1=\tau$.
Since $\tau_n$ is also a representation, we can also define $\psi_n:K(E^{\otimes n})\to B$.
\end{defn}
\begin{rmk}
For the well-definedness of $\psi$, see Proposition 4.6.3 in [Brown-Ozawa].
\end{rmk}


\begin{lem}[Injectivity on representations]
If $\pi$ is injective, then $\tau_n$ and $\psi_n$ are isometric.
\end{lem}


\begin{defn}[Core]
Let $\psi_n:K(E^{\otimes n})\to B$ be the associated $*$-homomorphism for $n\ge0$.
We define
\[ B_n:=\im\psi_n=\psi_n(K(E^{\otimes n}))\subset B.\]
In particular, $B_0=\pi(A)$ and $B_1=\psi(K(E))$.

\end{defn}


For a representation $(\pi,\tau)$, we define the \emph{Katsura ideal}
\[I_{(\pi,\tau)}:=\pi^{-1}(B_0\cap B_1)=\pi^{-1}(\pi(A)\cap\psi(K(E))).\]


\subsection{Pimsner algebras}


\begin{prop}[Universal representation]
We say a representation $(\tilde\pi,\tilde\tau)$ of $E$ on $C^*(\tilde\pi,\tilde\tau)$ is \emph{universal} if for every representation $(\pi,\tau)$ of $E$ there is a unique $*$-homomorphism $C^*(\tilde\pi,\tilde\tau)\to C^*(\pi,\tau)$ such that $\tilde\pi(a)$ and $\tilde\tau(a)$ are mapped to $\pi(a)$ and $\tau(a)$.
\begin{parts}
\item A universal representation always exists.
\item A universal representation satisfies $I_{(\pi,\tau)}=0$, that is, it is faithful and $B_0\cap B_1=0$.
\item A universal representation admits a gauge action $\beta:\T\to\Aut(C^*(\pi,\tau))$.
\end{parts}
\end{prop}
\begin{pf}
(a)
Consider the $\ell^\infty$-direct sum
\[\bigoplus_{(\pi,\tau)}C^*(\pi,\tau)\]
where $(\pi,\tau)$ runs through all representations of $E$.
Then, it satisfies the universal property.

(b), (c)
Assume that there is a special representation $(\pi,\tau)$ that satisfies each condition.
(We will see the existence later)
Let $\phi:C^*(\tilde\pi,\tilde\tau)\to C^*(\pi,\tau)$ be the $*$-homomotphism obtained by the universality.

Suppose $I_{(\pi,\tau)}=0$.
Since $(\pi,\tau)$ is faithful, $(\tilde\pi,\tilde\tau)$ is also faithful.
Since $\pi(A)\cap\psi(K(E))=0$ and
\[\phi(\tilde\pi(a))=\pi(a),\qquad\phi(\tilde\psi(\theta_{\xi,\eta}))=\phi(\tilde\tau(\xi)\tilde\tau(\eta)^*)=\tau(\xi)\tau(\eta)^*=\theta_{\xi,\eta}\]
imply that $\tilde\pi(A)\cap\tilde\psi(K(E))=0$.
\end{pf}


\begin{defn}[Fock representation]
Let
\[\cF(E):=\bigoplus_{n=0}^\infty E^{\otimes n}.\]
Then, we have a representation $(\pi,\tau)$ of $E$ on $B(\cF(E))$ as follows:
\end{defn}


\begin{thm}[Gauge-invariant uniqueness theorem I]
If a representation $(\pi,\tau)$ satisfies $I_{(\pi,\tau)}=0$ and admits a gauge action $\beta:\T\to\Aut(C^*(\pi,\tau))$, then $(\pi,\tau)$ is universal, i.e.~$C^*(\pi,\tau)\cong\cT(E)$.
\end{thm}
\begin{pf}[Sketch]
Step 1: $B_{n+1}$ is a closed ideal of $B_n$ and $B_{\le n}\cap B_{n+1}=0$.
\[0\to B_{\le n}\to B_{\le n+1}\to B_{n+1}\to0.\]

Step 2:
\[C^*(\pi,\tau)^\beta=\bar{\bigcup_{n\ge0} B_{\le n}}\]

Step 3: Enough to show injectivity only on the fixed point algebra.
\end{pf}






\bigskip

Denote the left action map by $\f:A\to B(E)$.
We define the ideal
\[I(E):=\{a\in A:\f(a)\in K(E), ab=0\text{ for all }b\in\ker\f\}.\]
In other words, $I(E)$ is the largest ideal on which $\f$ is faithfully mapped into $K(E)$.

\begin{defn}[Covariant representation]
We say a representation $(\pi,\tau)$ of $E$ is said to be \emph{covariant} if $\pi(a)=\psi(\f(a))$ on $a\in I(E)$.
\end{defn}
\begin{thm}[Gauge-invariant uniqueness theorem II]
If a covariant representation $(\pi,\tau)$ is faithful and admits a gauge action $\beta:\T\to\Aut(C^*(\pi,\tau))$, then $(\pi,\tau)$ is universal, i.e.~$C^*(\pi,\tau)\cong\cO(E)$.
\end{thm}

second quantization representation $B_ A(E)\to B_ A(\cF(E))$.

\begin{defn}
Let $E$ be a C$^*$-correspondence ove $A$.
The \emph{Toeplitz-Pimsner algebra} $\cT(E)$ is defined as the image of the universal representation of $E$.
The \emph{Cuntz-Pimsner algebra} $\cO(E)$ is defined as the C$^*$-generated by the image of the universal covariant representation of $E$.
\end{defn}

The name of Toeplitz from the C$^*$-correspondence $C(S^1)$ over $C(S^1)$.
The name of Cuntz from $_\C{\ell_2^n}_\C$.



gauge action,
universality,
examples(crossed products, graph and Cuntz-Krieger, groupoid),
nuclearity and exactness




\end{document}




熊谷 山崎 318 数理物理
倉橋 新井 421 基礎論
小松 木田 418 에르고딕
近藤 伊藤 402 シュレディンガー
佐藤 317 PDE Hamilton-Jacobi
佐藤 318 量子群
高橋 新井 421
田渕 会田

友永 井山 357 表現論
永井 伊藤 代数幾何
中浦 新井 570 モデル
中村 宮本 402
星屋 伊藤 557 pで
宮内 小林 406
Liu 志甫 326 数論幾何
渡辺 志甫 557 数論幾何
会沢 松井 261

原田 山崎 457



21
