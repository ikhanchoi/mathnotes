\documentclass{../../../small}
\usepackage{../../../ikhanchoi}

\begin{document}
\title{Operator Algebra Seminar Note II}
\author{Ikhan Choi}
\maketitle
\tableofcontents


\section*{Acknowledgement}
This note has been written based on the first-year graduate seminar presented at the University of Tokyo in the 2023 Autumn semester.
Each seminar was delivered for 105 minutes.
I gratefully acknowledge advice of Prof. Yasuyuki Kawahigashi and support of my colleagues Futaba Sato, Yusuke Suzuki.



\newpage
\section{October 18}



\begin{defn}[Countably decomposable von Neumann algebras]
Let $M$ be a von Neumann algebra.
A projection $p\in M$ is called \emph{countably decomposable} if mutually orthogonal non-zero projections majorized by $p$ are at most countable, and we say $M$ is \emph{countably decomposable} if the identity is.
\end{defn}
\begin{prop}
For a von Neumann algebra $M$, the followings are all equivalent.
\begin{parts}
\item $M$ is countably decomposable.
\item $M$ admits a faithful normal state.
\item $M$ admits a faithful normal non-degenerate representation with a cyclic and separating vector.
\item The unit ball of $M$ is metrizable in the six locally convex topology.
\end{parts}
\end{prop}
\begin{pf}
(a)$\Leftrightarrow$(b)
Suppose $M$ is countably decomposable.
Let $\{\xi_i\}\subset H$ be a maximal family of unit vectors such that $\bar{M'\xi_i}$ are mutually orthogonal subspaces, taken by Zorn's lemma.
If we let $p_i$ be the projection on $\bar{M'\xi_1}$, then $p_izp_i=zp_i$ for $z\in M'$ implies $p_i\in M''=M$.
By the assumption, the family $\{\xi_i\}$ is countable.
Define a state $\omega$ of $M$ such that
\[\omega(x):=\sum_{i=1}^\infty\omega_{2^{-i}\xi_i}(x),\qquad x\in M.\]
It converges due to $\|\omega_{2^{-i}\xi_i}\|=2^{-i+1}$.
It is normal since the sequence $(2^{-i}\xi_i)$ belongs to $\ell(\N,H)$, and it is faithful because $\omega(x^*x)=0$ implies $x\xi_i=0$ for all $i$, which deduces that $x=\sum_ixp_i=0$.

Conversely, if $\omega$ is a faithful normal state, then for a mutually orthogonal family of non-zero projections $\{p_i\}\subset M$, we have
\[\{p_i\}=\bigcup_{n=1}^\infty\{p_i:\f(p_i)>n^{-1}\}\]
the countable union of finite sets.
Thus $M$ is countable decomposable.

(b)$\Leftrightarrow$(c)
Let $\omega$ be a faithful normal state of $M$.
Consider any faithful normal nondegenerate representation in which $\omega$ is a vector state so that the corresponding vector is a separating vector.
Examples include the GNS representation of $\omega$, and the composition with the diagonal map $B(H)\to B(\ell^2(\N,H))$.
Then, $\bar{M\Omega}$ admits a cyclic and separating vector $\Omega$ of $M$.
The converse is immediate, i.e.~the vector state $\omega_\Omega$ is a faithful normal state of $M$.

(a)$\Leftrightarrow$(d)
Suppose $M$ is countably decomposable and take $\{\xi_i\}_{i=1}^\infty$ and $\{p_i\}_{i=1}^\infty$ as we did.
Define
\[d(x,y):=\sum_{i=1}^\infty2^{-i}\|(x-y)\xi_i\|.\]
Clearly it generates a topology coarser than strong topology.
It is also finer because if a bounded net $x_\alpha$ in $M$ converges to zero in the metric $d$ so that $x\xi_i\to0$ for all $i$, then $H=\bigoplus_iM'\xi_i$ implies that for every $\xi\in H$ and $\e>0$ we have $\|\xi-\sum_{k=1}^nz_k\xi_{i_k}\|<\e$ for some $z_k\in M'$ so that
\[\|x_\alpha\xi\|\le\|x_\alpha(\xi-\sum_{k=1}^nz_k\xi_{i_k})\|+\sum_{k=1}^n\|x_\alpha z_k\xi_{i_k}\|<\e+\sum_{k=1}^n\|z_k\|\|x_\alpha\xi_{i_k}\|\to\e.\]
Since on the bounded part the strong and $\sigma$-strong topologies coincide, the two topologies on the unit ball are metrizable.
We can do similar for the weak and strong$^*$ topologies.

Conversely, for a mutually orthogonal family of non-zero projections $\{p_i\}_{i\in I}\subset M$, since the net of finite partial sums $p_F:=\sum_{i\in F}p_i$ is an increasing net in the closed unit ball whose supremum is the identity of $M$, there is a convergent subsequence $p_{F_n}\uparrow1$ by the metrizability, which implies $I=\bigcup_{n=1}^\infty F_n$, the countable union of finite sets.
\end{pf}


\subsection{Semi-cyclic representations}


\begin{defn}[Weights]
Let $M$ be a von Neumann algebra.
A \emph{weight} is a function $\f:M^+\to[0,\infty]$ such that
\[\f(x+y)=\f(x)+\f(y),\qquad\f(\lambda x)=\lambda\f(x),\qquad x,y\in M^+,\ \lambda\in\R^{\ge0},\]
where we use $0\cdot\infty=0$.
A weight $\f$ is said to be \emph{normal} if
\[\f(\sup_\alpha x_\alpha)=\sup_\alpha\f(x_\alpha)\]
for any bounded increasing net $(x_\alpha)$ in $M^+$.
\end{defn}
\begin{defn}
Let $\f$ be a weight on a von Neumann algebra $M$.
Define a left ideal of $M$
\[\fn:=\{x\in M:\f(x^*x)<\infty\},\]
and a hereditary $*$-subalgebra of $M$
\[\fm:=\fn^*\fn=\{\sum_{i=1}^ny_i^*x_i:(x_i),(y_i)\in\fn^n\}.\]
\end{defn}


\begin{lem}
If $x,y\in M$ satisfies $y^*y\le x^*x$, then there is a unique $s\in B(H)$ such that $y=sx$ and $s=sp$, where $p$ is the range projection of $x$, and $s\in M$.
\end{lem}
\begin{pf}
Suppose $\id_H\in M\subset B(H)$.
The operator $s_0:\bar{xH}\to\bar{yH}:x\xi\mapsto y\xi$ is well defined because
\[\|y\xi\|^2=\<y^*y\xi,\xi\>\le\<x^*x\xi,\xi\>=\|x\xi\|^2.\]
Let $p$ be the range projection of $x$ and let $s:=s_0p$.
Then, $y\xi=sx\xi$ for all $\xi\in H$.
If $y=s'x$ and $s'=s'p$, then
\[x^*(s-s')^*(s-s')x=(y-y)^*(y-y)=0\]
implies
\[0=p(s-s')^*(s-s')p=(s-s')^*(s-s').\]
Therefore, $s$ is unique in $B(H)$.
If $u\in M'$ is unitary, then $usu^*$ satisfies the same property $y=usu^*x$ and $usu^*=usu^*p$, so $us=su$.
Since the unitary span the whole C$^*$-algebra, we have $s\in M''=M$.
\end{pf}

\begin{prop}
Let $\f$ be a weight on a von Neumann algebra $M$.
\begin{parts}
\item Every element of $\fm^+$ can be written to be $x^*x$ for some $x\in\fn$.
\item Every element of $\fm$ can be written to be $y^*x$ for some $x,y\in\fn$.
\end{parts}
\end{prop}
\begin{pf}
(a)
Let $a:=\sum_{i=1}^ny_i^*x_i\in\fm^+$ for some $x_i,y_i\in\fn$.
The polarization writes
\[a=\frac14\sum_{i=1}^n\sum_{k=0}^3i^k|x_i+i^ky_i|^2\]
and $a^*=a$ implies
\[a=\frac12\sum_{i=1}^n(|x_i+y_i|^2-|x_i-y_i|^2)\le\frac12\sum_{i=1}^n|x_i+y_i|^2\]
implies
\[\f(a)\le\frac12\sum_{i=1}^n\f(|x_i+y_i|^2)<\infty.\]
Therefore, if $x:=a^{\frac12}\in\fn$, then $a=x^*x$.

(b)
Let $a:=\sum_{i=1}^ny_i^*x_i\in\fm$ for some $x_i,y_i\in\fn$.
Let $x:=(\sum_{i=1}^nx_i^*x_i)^{\frac12}\in\fn$.
Since $x_i^*x_i\le x^2$, we have $s_i\in M$ such that $x_i=s_ix$.
If we let $y:=\sum_{i=1}^ns_i^*y_i\in\fn$, then
\[a=\sum_{i=1}^ny_i^*x_i=\sum_{i=1}^ny_i^*s_ix=(\sum_{i=1}^ns_i^*y_i)x=y^*x.\qedhere\]
\end{pf}


\begin{defn}[Semi-cyclic representations]
Let $\f$ be a weight on a von Neumann algebra.
Let $H$ be the Hilbert space defined by the separation and completion of a sesquilinear form
\[\fn\times\fn\to\C:(x,y)\mapsto\f(y^*x)\]
and let $\psi:\fn\to H$ be the canonical image map.
The pair $(\pi,\psi)$ is called the \emph{semi-cyclic representation} associated to $\f$.
\end{defn}


\begin{prop}
Let $\f$ be a weight on a von Neumann algebra and $(\pi,\psi)$ be the associated semi-cyclic representation to $\f$.
Consider a map
\[\Theta:\fm\times\pi(M)'\to\C:(y^*x,z)\mapsto\<z\psi(x),\psi(y)\>\]
and define
\[\theta:\fm\to(\pi(M)')_*,\qquad\theta^*:\pi(M)'\to\fm^\#\]
such that $\Theta(x,z)=\theta(x)(z)=\theta^*(z)(x)$ for $x\in\fm$ and $z\in\pi(M)'$.
\begin{parts}
\item $\Theta$ is a well-defined bilinear form.
\item $\theta^*$ is bijective onto the space of linear functionals on $\fm$ whose absolute value is majorized by $\f$. (bounded Radon-Nikodym theorem)
\end{parts}
\end{prop}
\begin{pf}
(a)
The linearity in the second argument is obvious.
Fix $z\in\pi(M)'$.
We first check the well-definedness on $\fm^+$.
Let $x^*x=y^*y\in\fm^+$ for $x,y\in\fn$.
Then, there is $s\in M$ such that $y=sx$ and $s=sp$, where $p$ is the range projection of $x$, so
\[x^*(1-s^*s)x=x^*x-y^*y=0\]
implies
\[0=p(1-s^*s)p=p-s^*s\]
and $x=px=s^*sx=s^*y$.
The well-definedness follows from
\[\Theta(x^*x,z)=\<z\psi(x),\psi(x)\>=\<\pi(s)z\pi(s^*)\psi(y),\psi(y)\>=\<z\psi(ss^*y),\psi(y)\>=\Theta(y^*y,z).\]

The homogeneity is clear, so now we prove the addivitiy.
Let $x^*x,y^*y\in\fm^+$ for some $x,y\in\fn$.
Let $a:=(x^*x+y^*y)^{\frac12}$ and take $s,t\in M$ such that $x=sa$, $y=ta$, $s=sa$, and $t=ta$, where $p$ is the range projection of $a$.
Then,
\[a(1-s^*s-t^*t)a=a^*a-x^*x-y^*y=0\]
implies
\[p(1-s^*s-t^*t)p=p-s^*s-t^*t.\]
It follows that
\begin{align*}
\Theta(x^*x+y^*y,z)
&=\<z\psi(a),\psi(a)\>=\<z\pi(p)\psi(a),\psi(a)\>\\
&=\<z\pi(s^*s)\psi(a),\psi(a)\>+\<z\pi(t^*t)\psi(a),\psi(a)\>\\
&=\<z\psi(x),\psi(x)\>+\<z\psi(y),\psi(y)\>\\
&=\Theta(x^*x,z)+\Theta(y^*y,z).
\end{align*}
Now the $\Theta(\cdot,z)$ is linearly extendable to $\fm$.

(b)
The linear map $\theta^*$ is injective since $\psi$ has dense range.
Take $z\in\pi(M)'$ and consider $\theta^*(z)$, which maps $x^*x$ to $\<z\psi(x),\psi(x)\>$ for $x\in\fn$.
The image is majorized by $\f$ as
\[|\<z\psi(x),\psi(x)\>|\le\|z\|\|\psi(x)\|^2=\|z\|\f(x^*x).\]
Conversely, let $l\in\fm^\#$ is a linear functional majorized by $\f$, i.e.~there is a constant $C>0$ such that
\[|l(x^*x)|\le C\f(x^*x),\qquad x\in\fn.\]
Define a sesquilinear form $\sigma:\fn\times\fn\to\C$ such that $\sigma(x,y):=l(y^*x)$.
It is well-defined after separation of $\fn$ and is bounded by the Cauhy-Schwartz inequality
\[|\sigma(x,y)|^2=|l(y^*x)|^2\le\|l(x^*x)\|\|l(y^*y)\|\le\f(x^*x)\f(y^*y)=\|\psi(x)\|^2\|\psi(y)\|^2.\]
Therefore, $\sigma$ defines a bounded linear operator $z\in\pi(M)'$ such that
\[\sigma(x,y)=\<z\psi(x),\psi(y)\>,\]
exactly meaning $\theta^*(z)(y^*x)=l(y^*x)$ for $x,y\in\fn$.
\end{pf}

Note that we have a commutative diagram
\[\begin{tikzcd}
\fn \ar{r}{\psi}\ar[swap]{dd}{|\cdot|^2} & H \ar{d}{\omega} \\
& B(H)_* \ar{d}{\mathrm{res}}\\
\fm^+ \ar{r}{\theta} & (\pi(M)')_*.
\end{tikzcd}\]
In particular, for $x\in\fn^+$ we have
\[\|\theta(x^2)\|=\|\omega_{\psi(x)}\|=\|\psi(x)\|^2=\f(x^2).\]


\begin{lem}
Let 
For $z\in\fm^{sa}$, we have
\[\inf\{\f(a):z\le a\in\fm^+\}\le\|\theta(z)\|.\]
In particular, for $x,y\in\fn^+$ and for any $\e>0$ there is $a\in\fm^+$ such that $x^2-y^2\le a$ and
\[\f(a)\le\|\theta(x^2-y^2)\|+\e=\|\omega_{\psi(x)}-\omega_{\psi(y)}\|+\e.\]
\end{lem}
\begin{pf}
Denote by $p(z)$ the left-hand side of the inequality.
Then, we can check $p:\fm^{sa}\to\R_{\ge0}$ is a semi-norm such that $p(z)=\f(z)$ for $z\ge0$.
(If we take $p(z):=\f(z^+)$, then it seems to be dangerous when checking the sublinearity. I could not find the counterexample.)

Fix any non-zero $z_0\in\fm^{sa}$.
By the Hahn-Banach extension, there is an algebraic real linear functional $l:\fm^{sa}\to\R$ such that
\[l(z_0)=p(z_0),\qquad |l(z)|\le p(z),\qquad z\in\fm^{sa}.\]
Extend linearly $l$ to be $l:\fm\to\C$.
Since $|l(z)|\le\f(z)$ for $z\in\fm^+$, the linear functional $l$ is contained in the image of the closed unit ball under the injective map
\[\theta^*:\pi(M)'\to\fm^\#.\]
If we let $a\in(\pi(M)')_1$ be the corresponding operator such that $\theta^*(a)=l$, then we get
\[p(z_0)=l(z_0)=\theta^*(a)(z_0)=\theta(z_0)(a)\le\|\theta(z_0)\|.\]
Since $z_0\in\fm^{sa}$ is aribtrary, we are done.
\end{pf}




\subsection{$\sigma$-weak lower semi-continuity}


\begin{thm}
Let $M$ be a countably decomposable von Neumann algebra.
Then,  normal weight on $M$ is $\sigma$-weakly lower semi-continuous.
\end{thm}
\begin{pf}
Let $\f$ be a normal weight on $M$ and let $(\pi,\psi)$ be the associated semi-cyclic representation.

In the spirit of the Krein-\v Smulian theorem, the $\sigma$-weak lower semi-continuity is equivalent to the $\sigma$-weak closedness of the intersection with the ball
\begin{align*}
\f^{-1}([0,1])_1
&=\{x\in M^+:\f(x)\le1,\ \|x\|\le1\}\\
&=\{x\in M^+:\|\psi(x^{\frac12})\|\le1,\ \|x^{\frac12}\|\le1\}.
\end{align*}
Since that the $\sigma$-weak and $\sigma$-strong closedness of a convex set are equivalent and that the square root operation on $M^+_1$ is $\sigma$-strongly continuous, we are enough to show the set
\[(\f^{-1}([0,1])_1)^{\frac12}=\{x\in M^+:\|\psi(x)\|\le1,\ \|x\|\le1\}\]
is $\sigma$-weakly closed.
This set, if we denote the graph of $\psi:\fn\to H$ by $\Gamma_\psi$, is the image of the positive part of the unit ball
\[(\Gamma_\psi)^+_1=\{(x,\psi(x))\in M^+\oplus_\infty H:\|\psi(x)\|\le1,\ \|x\|\le1\}\]
under the projection $M\oplus_\infty H\to M$.
Observing $M\oplus_\infty H\cong(M_*\oplus_1H)^*$, if we prove $(\Gamma_\psi)_1^+$ is weakly$^*$ closed, then we are done by its compactness.

Consider a linear functional $l:M\oplus_\infty H\to\C$ that is continuous with respect to $(\sigma s,\|\cdot\|)$.
If we define $l_1:M\to\C$ and $l_2:H\to\C$ such that $l_1(x):=l(x,0)$ and $l_2(\xi)=(0,\xi)$, then they satisfy $l(x,\xi)=l_1(x)+l_2(\xi)$, and are continuous in $\sigma$-strong and norm topologies, hence to $\sigma$-weak and weak topologies, respectively.
Since a net $(x_\alpha,\xi_\alpha)$ converges to $(x,\xi)$ weakly$^*$ if and only if $x_\alpha\to x$ $\sigma$-weakly and $\xi_\alpha\to\xi$ weakly, $l$ is weakly$^*$ continuous.
Because $(\Gamma_\psi)_1^+$ is convex, we will now show that $(\Gamma_\psi)^+_1$ is closed in $(M,\sigma s)\times(H,\|\cdot\|)$.

Note that the unit ball $M_1$ is metrizable in $\sigma$-strong topology since $M$ is countably decomposable.
Suppose a sequence $x_n\in\fn^+_1$ satisfies $x_n\to x$ $\sigma$-strongly and $\psi(x_n)\to\xi$ in $H$.
Then, it suffices to show the following two statements: $x\in\fn^+_1$ and $\psi(x)=\xi$.
We first observe that since $\psi(x_n)$ is Cauchy, so is $\omega_{\psi(x_n)}$ in $(\pi(M)')_*$.

Consider for a while, a family of functions
\[f_a(t):=\frac t{1+at},\qquad t\in(-a^{-1},\infty),\]
parametrized by $a>0$.
They have several properties.
At first, they are operator monotone.
Next, they are $\sigma$-strongly continuous on a closed subset of its domain due to the boundedness of $f_a$, as we can see in the proof of the Kaplansky density theorem.
Finally, for each $x\in M_+$, the increasing limit $f_a(x)\uparrow x$ in norm as $a\to0$ implies that $\sup_af_a(x)=x$.

First we show $x\in\fn^+_1$.
It is clear that $x\in M^+_1$, so it is enough to show $\f(x^2)<\infty$.
By taking a subsequence, we may assume $\|\omega_{\psi(x_{n+1})}-\omega_{\psi(x_n)}\|<\frac1{2^n}$.
In order to dominate $x_n$ with an increasing sequence, find $a_n\in\fm^+$ such that
\[x_{n+1}^2-x_n^2\le a_n,\qquad \f(a_n)<\frac1{2^n},\]
using the previous lemma.
Then, we can write
\[x_{n+1}^2\le x_1^2+\sum_{k=1}^n(x_{k+1}^2-x_k^2)\le x_1+\sum_{k=1}^na_k.\]
Here the right-hand side is increasing but not a bounded sequence so we take $f_a$ to get the $\sigma$-strong limit
\[f_a(x^2)\le\sup_nf_a(x_1^2+\sum_{k=1}^na_k).\]
Then, by the normality of $\f$, we have
\begin{align*}
\f(f_a(x^2))
&\le\sup_n\f(f_a(x_1^2+\sum_{k=1}^na_k))\\
&\le\sup_n\f(x_1^2+\sum_{k=1}^na_k)\\
&=\f(x_1^2)+\sum_{k=1}^\infty\f(a_k)\\
&<\f(x_1^2)+1<\infty
\end{align*}
which implies by sending $a\to0$ that $\f(x^2)<\infty$, whence $x\in\fn$.

Next we show $\psi(x)=\xi$.
If we prove $\f((x_n-x)^2)\to0$, then
\[\|\xi-\psi(x)\|\le\|\xi-\psi(x_n)\|+\|\psi(x_n)-\psi(x)\|=\|\xi-\psi(x_n)\|+\f((x_n-x)^2)^{\frac12}\to0\]
deduces the desired result.
By taking a subsequence, since $\psi(x_n-x)$ is Cauchy, we may assume
\[\|\omega_{\psi(x_n-x)}-\omega_{\psi(x_{n+1}-x)}\|<\frac1{2^n}.\]
Let $b_n\in\fm^+$ such that
\[(x_n-x)^2-(x_{n+1}-x)^2\le b_n,\qquad\f(b_n)<\frac1{2^n}\]
As we did previously, we have
\[f_a((x_n-x)^2)\le f_a((x_{m+1}-x)^2)+f_a(\sum_{k=n}^mb_k)\to\sup_mf_a(\sum_{k=n}^mb_k)\]
as $m\to\infty$ and
\[\f(f_a((x_n-x)^2))
\le\sup_m\f(f_a(\sum_{k=n}^mb_k))
\le\sup_m\f(\sum_{k=n}^mb_k)<\frac1{2^{n-1}}.\]
Therefore,
\[\f((x_n-x)^2)\le\frac1{2^{n-1}}\to0.\qedhere\]
\end{pf}


\begin{thm}
Let $M$ be an arbitrary von Neumann algebra.
Then, a normal weight on $M$ is $\sigma$-weakly lower semi-continuous.
\end{thm}
\begin{pf}
Let $\f$ be a normal weight of $M$.
Let $\Sigma$ be the set of all countably decomposable projections of $M$ and let $M_0:=\bigcup_{p\in\Sigma}pMp$.
The equivalent condition for $x\in M$ to belong to $M_0$ is that the left and right support projections are countably decomposable.
Since then the left support projection $p$ and the right support projection are Murray-von Neumann equivalent so that there is a $*$-isomorphism between $pMp$ and $qMq$, the countable decomposability is equivalent for $p$ and $q$.
It implies that $M_0$ is an algebraic ideal of $M$.
Moreover, $M_0$ is $\sigma$-weakly sequentially closed in $M$ since if a sequence $x_n\in M_0$ converges to $x\in M$ $\sigma$-weakly, then for $p_n\in\Sigma$ such that $x_n=p_nx_np_n$, we have $p\in\Sigma$ with $p_n\le p$ so that $x_n=px_np$ converges to $x=pxp$ $\sigma$-weakly.

We claim that $\f^{-1}([0,1])_1$ is relatively $\sigma$-weakly closed in $M_0$.
Let $y\in\bar{\f^{-1}([0,1])_1}^{\sigma w}\cap M_0$ so that there is a net $y_\alpha\in\f^{-1}([0,1])_1$ converges $\sigma$-weakly to $y$, and there is $p\in\Sigma$ such that $pyp=y$.
Since
\[py_\alpha p\in\f^{-1}([0,1])_1\cap pMp\]
also converges $\sigma$-weakly to $pyp=y$ and by the previous theorem $\f^{-1}([0,1])\cap pMp$ is $\sigma$-weakly closed for each $p\in\Sigma$, we have $y\in\f^{-1}([0,1])$.
The claim proved.

Suppose $x_\alpha\in\f^{-1}([0,1])_1$ converges to $x\in M_1^+$ $\sigma$-weakly.
Let $\{p_i\}_{i\in I}$ be a maximal mutually orthogonal projections in $\Sigma$, and let $p_F:=\sum_{i\in F}p_i$ for finite sets $F\subset I$ so that $\sup_Fp_F=1$.
It clearly follows that
\[x_\alpha^{\frac12}p_Fx_\alpha^{\frac12}\in\f^{-1}([0,1])_1.\]
Because $M_0$ is an ideal of $M$,
\[x^{\frac12}p_Fx^{\frac12}\in\bar{\f^{-1}([0,1])_1}^{\sigma w}\cap M_0.\]
By the above claim,
\[x^{\frac12}p_Fx^{\frac12}\in\f^{-1}([0,1])_1.\]
By the normality of $\f$, we finally obtain
\[x\in\f^{-1}([0,1])_1.\]
Therefore, $\f^{-1}([0,1])_1$ is $\sigma$-weakly closed.
\end{pf}


\subsection{Supremum of positive linear functionals}




\newpage
\section{November 10}



\newpage
\section{December 20}



\newpage
\section{January 17}



\newpage
\section{February 9}





\end{document}