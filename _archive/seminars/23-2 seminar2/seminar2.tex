\documentclass{../../../small}
\usepackage{../../../ikhanchoi}

\newcommand{\sech}{\operatorname{sech}}

\begin{document}
\title{Operator Algebra Seminar Note II}
\author{Ikhan Choi}
\maketitle
\tableofcontents


\section*{Acknowledgement}
This note has been written based on the first-year graduate seminar presented at the University of Tokyo in the 2023 Autumn semester.
Each seminar was delivered for 105 minutes.
I gratefully acknowledge advice of Prof. Yasuyuki Kawahigashi and support of my colleagues Futaba Sato and Yusuke Suzuki.



\newpage
\section{October 18}

\subsection{Countably decomposable von Neumann algebras}

\begin{defn}[Countably decomposable von Neumann algebras]
Let $M$ be a von Neumann algebra.
A projection $p\in M$ is called \emph{countably decomposable} if mutually orthogonal non-zero projections majorized by $p$ are at most countable, and we say $M$ is \emph{countably decomposable} if the identity is.
\end{defn}
\begin{prop}
For a von Neumann algebra $M$, the followings are all equivalent.
\begin{parts}
\item $M$ is countably decomposable.
\item $M$ admits a faithful normal state.
\item $M$ admits a faithful normal non-degenerate representation with a cyclic and separating vector.
\item The unit ball of $M$ is metrizable in one of the following topologies: $\sigma$-strong$^*$, $\sigma$-strong, strong$^*$, strong.
\end{parts}
\end{prop}
\begin{pf}
(a)$\Leftrightarrow$(b)
Suppose $M$ is countably decomposable.
Let $\{\xi_i\}\subset H$ be a maximal family of unit vectors such that $\bar{M'\xi_i}$ are mutually orthogonal subspaces, taken by Zorn's lemma.
If we let $p_i$ be the projection on $\bar{M'\xi_1}$, then $p_izp_i=zp_i$ for $z\in M'$ implies $p_i\in M''=M$.
By the assumption, the family $\{\xi_i\}$ is countable.
Define a state $\omega$ of $M$ such that
\[\omega(x):=\sum_{i=1}^\infty\omega_{2^{-i}\xi_i}(x),\qquad x\in M.\]
It converges due to $\|\omega_{2^{-i}\xi_i}\|=2^{-i+1}$.
It is normal since the sequence $(2^{-i}\xi_i)$ belongs to $\ell(\N,H)$, and it is faithful because $\omega(x^*x)=0$ implies $x\xi_i=0$ for all $i$, which deduces that $x=\sum_ixp_i=0$.

Conversely, if $\omega$ is a faithful normal state, then for a mutually orthogonal family of non-zero projections $\{p_i\}\subset M$, we have
\[\{p_i\}=\bigcup_{n=1}^\infty\{p_i:\f(p_i)>n^{-1}\}\]
the countable union of finite sets.
Thus $M$ is countable decomposable.

(b)$\Leftrightarrow$(c)
Let $\omega$ be a faithful normal state of $M$.
Consider any faithful normal nondegenerate representation in which $\omega$ is a vector state so that the corresponding vector is a separating vector by the faithfullness of $\omega$.
Examples include the GNS representation of $\omega$, and the composition with the diagonal map $B(H)\to B(\ell^2(\N,H))$.
Then, $\bar{M\Omega}$ admits a cyclic and separating vector $\Omega$ of $M$.
The converse is immediate, i.e.~the vector state $\omega_\Omega$ is a faithful normal state of $M$.

(a)$\Leftrightarrow$(d)
Suppose $M$ is countably decomposable and take $\{\xi_i\}_{i=1}^\infty$ and $\{p_i\}_{i=1}^\infty$ as we did.
Define
\[d(x,y):=\sum_{i=1}^\infty2^{-i}\|(x-y)\xi_i\|.\]
Clearly it generates a topology coarser than strong topology.
It is also finer because if a bounded net $x_\alpha$ in $M$ converges to zero in the metric $d$ so that $x\xi_i\to0$ for all $i$, then $H=\bigoplus_iM'\xi_i$ implies that for every $\xi\in H$ and $\e>0$ we have $\|\xi-\sum_{k=1}^nz_k\xi_{i_k}\|<\e$ for some $z_k\in M'$ so that
\[\|x_\alpha\xi\|\le\|x_\alpha(\xi-\sum_{k=1}^nz_k\xi_{i_k})\|+\sum_{k=1}^n\|x_\alpha z_k\xi_{i_k}\|<\e+\sum_{k=1}^n\|z_k\|\|x_\alpha\xi_{i_k}\|\to\e.\]
Since on the bounded part the strong and $\sigma$-strong topologies coincide, the two topologies on the unit ball are metrizable.
We can do similar for the strong$^*$ and the $\sigma$-strong$^*$ topologies.

Conversely, for a mutually orthogonal family of non-zero projections $\{p_i\}_{i\in I}\subset M$, since the net of finite partial sums $p_F:=\sum_{i\in F}p_i$ is an non-decreasing net in the closed unit ball whose supremum is the identity of $M$, there is a convergent subsequence $p_{F_n}\uparrow1$ by the metrizability, which implies $I=\bigcup_{n=1}^\infty F_n$, the countable union of finite sets.
\end{pf}

\begin{prop}
For a von Neumann algebra $M$, the followings are all equivalent.
\begin{parts}
\item $M$ has the separable predual.
\item $M$ admits a faithful normal non-degenerate representation on a separable Hilbert space.
\item $M$ is countably decomposable and countably generated.
\item The unit ball of $M$ is metrizable in one of the following topologies: $\sigma$-weak, weak.
\end{parts}
\end{prop}


\subsection{Weights and semi-cyclic representations}


\begin{defn}[Weights]
Let $M$ be a von Neumann algebra.
A \emph{weight} is a function $\f:M^+\to[0,\infty]$ such that
\[\f(x+y)=\f(x)+\f(y),\qquad\f(\lambda x)=\lambda\f(x),\qquad x,y\in M^+,\ \lambda\ge0,\]
where we use the convention $0\cdot\infty=0$.
\end{defn}

\begin{defn}
Let $\f$ be a weight on a von Neumann algebra $M$.
Define
\[\fn:=\{x\in M:\f(x^*x)<\infty\},\qquad\fa:=\fn^*\cap\fn,\qquad\fm:=\fn^*\fn.\]
It easily follows that $\fn$ is a left ideal of $M$ with a sesquilinear form $\<,\>_\f:\fn\times\fn\to\C$ such that
\[\<x,y\>_\f:=\f(y^*x),\qquad x,y\in\fn,\]
$\fa$ is a $*$-subalgebra of $M$, and $\fm$ is a hereditary $*$-subalgebra of $M$ with a positive linear functional $\f:\fm\to\C$ such that
\[\f(y^*x):=\sum_{k=0}^3i^k\f((x+i^ky)^*(x+i^ky)),\qquad x,y\in\fn,\]
which extends the original $\f$.
\end{defn}


\begin{prop}
Let $\f$ be a weight on a von Neumann algebra $M$.
\begin{parts}
\item Every element of $\fm^+$ can be written to be $x^*x$ for some $x\in\fn$.
\item Every element of $\fm$ can be written to be $y^*x$ for some $x,y\in\fn$.
\end{parts}
\end{prop}
\begin{pf}
(a)
Let $a:=\sum_{i=1}^ny_i^*x_i\in\fm^+$ for some $x_i,y_i\in\fn$.
The polarization writes
\[a=\frac14\sum_{i=1}^n\sum_{k=0}^3i^k|x_i+i^ky_i|^2\]
and $a^*=a$ implies
\[a=\frac12\sum_{i=1}^n(|x_i+y_i|^2-|x_i-y_i|^2)\le\frac12\sum_{i=1}^n|x_i+y_i|^2\]
implies
\[\f(a)\le\frac12\sum_{i=1}^n\f(|x_i+y_i|^2)<\infty.\]
Therefore, if $x:=a^{\frac12}\in\fn$, then $a=x^*x$.

(b)
Let $a:=\sum_{i=1}^ny_i^*x_i\in\fm$ for some $x_i,y_i\in\fn$.
Let $x:=(\sum_{i=1}^nx_i^*x_i)^{\frac12}\in\fn$.
Since $x_i^*x_i\le x^2$, we have $s_i\in M$ such that $x_i=s_ix$.
If we let $y:=\sum_{i=1}^ns_i^*y_i\in\fn$, then
\[a=\sum_{i=1}^ny_i^*x_i=\sum_{i=1}^ny_i^*s_ix=(\sum_{i=1}^ns_i^*y_i)x=y^*x.\qedhere\]
\end{pf}


\begin{defn}[Semi-cyclic representations]
Let $A$ be a $*$-algebra.
A \emph{semi-cyclic representation} of $A$ is a representation $\pi:A\to B(H)$ and a together with a left $M$-module homomorphism $\Lambda:\dom\Lambda\to H$ of dense image.
In particular, $\dom\Lambda$ is a left ideal of $M$.
\end{defn}

\begin{prop}
Let $\f$ be a weight on a von Neumann algebra and $(\pi,\Lambda)$ be the associated semi-cyclic representation to $\f$.
For $h\in\pi(M)'$, the linear functional $\fm\to\C:y^*x\mapsto\<h\Lambda(x),\Lambda(y)\>$ is well-defined.
\end{prop}
\begin{pf}
We first check the well-definedness on $\fm^+$.
Let $x^*x=y^*y\in\fm^+$ for $x,y\in\fn$.
Then, there is $s\in M$ such that $y=sx$ and $s=sp$, where $p$ is the range projection of $x$, so $x^*(1-s^*s)x=x^*x-y^*y=0$ implies $0=p(1-s^*s)p=p-s^*s$ and $x=px=s^*sx=s^*y$.
The well-definedness follows from
\[\<z\Lambda(x),\Lambda(x)\>=\<\pi(s)z\pi(s^*)\Lambda(y),\Lambda(y)\>=\<z\Lambda(ss^*y),\Lambda(y)\>=\<z\Lambda(y),\Lambda(y)\>.\]

The homogeneity is clear, so now we prove the addivitiy.
Let $x^*x,y^*y\in\fm^+$ for some $x,y\in\fn$.
Let $a:=(x^*x+y^*y)^{\frac12}$ and take $s,t\in M$ such that $x=sa$, $y=ta$, $s=sa$, and $t=ta$, where $p$ is the range projection of $a$.
Then, $a(1-s^*s-t^*t)a=a^*a-x^*x-y^*y=0$ implies $p(1-s^*s-t^*t)p=p-s^*s-t^*t$.
It follows that
\begin{align*}
\<z\Lambda(a),\Lambda(a)\>
&=\<z\pi(p)\Lambda(a),\Lambda(a)\>\\
&=\<z\pi(s^*s)\Lambda(a),\Lambda(a)\>+\<z\pi(t^*t)\Lambda(a),\Lambda(a)\>\\
&=\<z\Lambda(x),\Lambda(x)\>+\<z\Lambda(y),\Lambda(y)\>.
\end{align*}
Now the $\Theta(\cdot,z)$ is linearly extendable to $\fm$.
\end{pf}

\begin{prop}[Radon-Nikodym affiliated with commutant]
Let $\f$ be a weight on a von Neumann algebra and $(\pi,\Lambda)$ be the associated semi-cyclic representation to $\f$.
Let $\psi$ be a ...
There is $h$ such that
\[\psi(y^*x)=\<h\Lambda(x),\Lambda(y)\>\]


In particular, if $l\in\fm^\#$ satisfies $|l|\le C\f$ for some $C>0$, then there is $h\in\pi(M)'$ such that $\|h\|\le C$ and $l(y^*x)=\<h\Lambda(x),\Lambda(y)\>$ for $x,y\in\fn$.
\end{prop}
\begin{pf}
(a)

(b)
The linear map $\theta^*$ is injective since $\Lambda$ has dense range.
Take $z\in\pi(M)'$ and consider $\theta^*(z)$, which maps $x^*x$ to $\<z\Lambda(x),\Lambda(x)\>$ for $x\in\fn$.
The image is majorized by $\f$ as
\[|\<z\Lambda(x),\Lambda(x)\>|\le\|z\|\|\Lambda(x)\|^2=\|z\|\f(x^*x).\]
Conversely, let $l\in\fm^\#$ is a linear functional majorized by $\f$, i.e.~there is a constant $C>0$ such that
\[|l(x^*x)|\le C\f(x^*x),\qquad x\in\fn.\]
Define a sesquilinear form $\sigma:\fn\times\fn\to\C$ such that $\sigma(x,y):=l(y^*x)$.
It is well-defined after separation of $\fn$ and is bounded by the Cauhy-Schwartz inequality
\[|\sigma(x,y)|^2=|l(y^*x)|^2\le\|l(x^*x)\|\|l(y^*y)\|\le\f(x^*x)\f(y^*y)=\|\Lambda(x)\|^2\|\Lambda(y)\|^2.\]
Therefore, $\sigma$ defines a bounded linear operator $z\in\pi(M)'$ such that
\[\sigma(x,y)=\<z\Lambda(x),\Lambda(y)\>,\]
exactly meaning $\theta^*(z)(y^*x)=l(y^*x)$ for $x,y\in\fn$.
\end{pf}

Note that we have a commutative diagram
\[\begin{tikzcd}
\fn \ar{r}{\Lambda}\ar[swap]{dd}{|\cdot|^2} & H \ar{d}{\omega} \\
& B(H)_* \ar{d}{\mathrm{res}}\\
\fm^+ \ar{r}{\theta} & \pi(M)'_*.
\end{tikzcd}\]
In particular, for $x\in\fn^+$ we have
\[\|\theta(x^2)\|=\|\omega_{\Lambda(x)}\|=\|\Lambda(x)\|^2=\f(x^2).\]





\subsection{Normal weights and normal semi-cyclic representations}




\begin{defn}[Normal semi-cyclic representations]
Let $M$ be a von Neumann algebra.
We say a weight $\f$ on $M$ is \emph{normal} if it is the supremum of normal positive linear functionals.
We say a semi-cyclic representation $(\pi,\Lambda)$ of $M$ is \emph{normal} if $\pi$ is normal and $\Lambda$ is closed with respect to $\sigma$-weak topology of $M$ and weak topology of $H$.
\end{defn}

\begin{prop}
Let $\f$ be a weight on $M$.
Let $H_\f$ be the Hilbert space defined by the separation and completion of a sesquilinear form $\fn_\f\times\fn_\f\to\C:(x,y)\mapsto\f(y^*x)$, and let $\Lambda_\f:\fn_\f\to H_\f$ be the canonical map.

Let $(\pi,\Lambda)$ be a semi-cyclic representation of $M$.
Let
\[\cF_\Lambda:=\{\omega\in M_*^+:\omega(x^*x)\le\|\Lambda(x)\|^2,\ x\in\dom\Lambda\}\]
and $\f_\Lambda(x^*x):=\sup_{\omega\in\cF_\Lambda}\omega(x^*x)$ for $x\in M$.
Then, it is clear that $\f_\Lambda$ is a weight.
\begin{parts}
\item If $\f$ is normal, then $(\pi_\f,\Lambda_\f)$ is a normal semi-cyclic representation such that $\f=\f_{(\pi_\f,\Lambda_\f)}$.
\item If $(\pi,\Lambda)$ is normal, then $\f_\Lambda$ is a normal weight such that there is a unitary $u:H\to H_{\f_\Lambda}$ satisfying $\pi_{\f_\Lambda}=(\Ad u)\pi$ and $\Lambda_{\f_\Lambda}=u\Lambda$.
\item For a normal $\f$, $\f$ is faithful if and only if $\Lambda$ is injective.
\item For a normal $\f$, $\f$ is semi-finite if and only if $\Lambda$ is $\sigma$-weakly densely defined.
\end{parts}
\end{prop}
\begin{pf}
(a)
We first show $\pi_\f$ is normal.
The proof is almost same as the normality of cyclic representation associated to normal states.
Consider $\pi_\f^*:B(H)_*\to M^*$, which is bounded.
Since
\[\pi_\f^*(\omega_{\Lambda_\f(y)})(x)=\<\pi_\f(x)\Lambda_\f(y),\Lambda_\f(y)\>=\f(y^*xy),\qquad x\in M, y\in\fn_\f,\]
and $\f$ is order continuous, we can see that $\pi_\f^*(\omega_{\Lambda_\f(y)})$ is also order continuous, so it is contained in $M_*$.
Because the image of $\Lambda_\f$ is dense, the linear span of states of the form $\omega_{\Lambda_\f(y)}$ for $y\in\fn_\f$ is norm-dense in $B(H)_*$ by the inequality
\[\|\omega_\xi-\omega_\eta\|\le\|\xi-\eta\|(\|\xi\|+\|\eta\|),\qquad\xi,\eta\in H.\]
Since $M_*$ is norm-closed $M^*$, so $\pi_\f^*(B(H)_*)\to M_*$ and $\pi_\f$ is normal.

For closedness of $\Lambda$,



(b)
First we show $\dom\Lambda=\fn_{\f_{(\pi,\Lambda)}}$.
One direction $\dom\Lambda\subset\fn_{\f_{(\pi,\Lambda)}}$ is clear because $x\in\dom\Lambda$ implies $\f_{(\pi,\Lambda)}(x^*x)\le\|\Lambda(x)\|^2$ by definition of $\cF_{(\pi,\Lambda)}$ and $\f_{(\pi,\Lambda)}$.
Conversely, we let $x\in\fn_{\f(\pi,\Lambda)}$ and claim $x\in\dom\Lambda$.
We may assume $x\ge0$ and $\f_{(\pi,\Lambda)}(x^2)=1$.
If we consider the $\sigma$-weak and weak topologies on $\dom\Lambda$ and $H$ respectively, then since the graph of $\Lambda$ is closed and the projection $\dom\Lambda\times H_1\to\dom\Lambda$ is a closed map due to the tube lemma, the set $\{y\in\dom\Lambda:\|\Lambda(y)\|\le1\}$ and its positive part is $\sigma$-weakly closed.
Since the square root is strongly continuous, if we temporarily consider a sufficiently large representation of $M$ in which every normal state is a vector state so that a strong and $\sigma$-strong topology coincide on $M$, we can conclude that $C:=\{y^2:\|\Lambda(y)\|^2\le1,\ y\in(\dom\Lambda)^+\}$ is $\sigma$-weakly closed with its convexity.
If $x^2\notin C$, then there is $\omega\in M_*^{sa}$ such that
\[\sup_{y^2\in C}\omega(y^2)\le1<\omega(x^2)\]
by the Hahn-Banach separation.
Since all functional arguments in the above inequality are all positive, we may assume $\omega$ is positive.
Then, for every $y\in\dom\Lambda$ we have
\[\omega(y^*y)=\|\Lambda(y)\|^2\omega(\frac{y^*y}{\|\Lambda(y)\|^2})\le\|\Lambda(y)\|^2\]
because $y^*y/\|\Lambda(y)\|^2\in C$, which means $\omega\in\cF_{(\pi,\Lambda)}$.
Thus, $\omega(x^2)\le\f_{(\pi,\Lambda)}(x^2)=1$ by definition of $\f_{(\pi,\Lambda)}$, which leads a contradiction, so $x^2\in C$ and $x\in\dom\Lambda$.


Next, fixing $x\in\dom\Lambda=\fn_{\f_{(\pi,\Lambda)}}$, we can check $\|\Lambda(x)\|^2=\f(x^*x)=\|\Lambda_{\f_{(\pi,\Lambda)}}(x)\|^2$.
The rest is routine.

(c)
Suppose $\f$ is faithful.
If $x\in\fn$ satisfies $\Lambda(x)=0$, then $\f(x^*x)=\|\Lambda(x)\|^2=0$ implies $x=0$, so $\Lambda$ is injective.

Suppose $\Lambda$ is injective.
Take a non-zero $x\in\fn$ so that $\|\Lambda(x)\|^2>0$.
We claim $\f(x^*x)\ne0$.

(d)
Also clear.
\end{pf}


\begin{thm}
Let $\f$ is a weight on a von Neumann algebra $M$.
Then, $\f$ is normal if and only if $\f$ is $\sigma$-weakly lower semi-continuous.
\end{thm}
\begin{pf}
($\Rightarrow$)
Endow a partial order on the set of all weights.
Then, every set of monotonically increasing subadditive homogeneous functions $\f:M^+\to[0,\infty]$ always have its supremum given by its pointwise supremum.
Since if $\f$ is the supremum of $\sigma$-weakly lower semi-continuous $\f_i$, then
\[\f^{-1}([0,1])=\bigcap_i\f_i^{-1}([0,1])\]
implies the $\sigma$-weak lower semi-continuity of $\f$.
Conversly, the following theorem holds.

($\Leftarrow$)
Let $F:=\f^{-1}([0,1])$.
It is a hereditary closed convex subset of the real locally convex space $(M^{sa},\sigma w)$.
Denote by the superscript circle the real polar set.
Since
\[\cF_\f=F^{\circ+}=\{\omega\in M_*^+:\omega\le\f\},\qquad
F^{\circ+\circ+}=\{x\in M^+:\sup_{\omega\in\cF_\f}\omega(x)\le1\},\]
it is enough to show $F^{\circ+\circ+}=F$.
The positive part of the real polar of $F$ is generally written as
\[F^{\circ+}=F^\circ\cap M_*^+=F^\circ\cap(-M^+)^\circ=(F\cup-M^+)^\circ=(F-M^+)^\circ.\]
Consider a sequence of inclusions
\[F\subset\bar F\subset\bar{(F-M^+)^+}\subset\bar{(F-M^+)}^+\subset(F-M^+)^{\circ\circ+}=F^{\circ+\circ+}.\]
The first, second, and forth inclusions are in fact full because $F$ is closed, hereditary, and convex.
The forth one uses the bipolar theorem.
So we claim that the reverse of the third inclusion $\bar{(F-M^+)}^+\subset\bar{(F-M^+)^+}$.

Let $x\in\bar{(F-M^+)}^+$.
For arbitrary $\e>0$, it is enough to show $f_\e(x)\in F-M^+$ because $x\ge0$ implies $f_\e(x)\ge0$ and $f_\e(x)\uparrow x$ as $\e\to0$.
Let $x_\alpha$ be a net in $F-M^+$ that converges to $x$ $\sigma$-strongly, which can be done by the convexity of $F-M^+$.
Let $y_\alpha$ be a net in $F$ such that $f_{\e/2}(x_\alpha)\le y_\alpha$.
Since $f_{\e/2}(y_\alpha)$ is a bounded net, we may assume it is $\sigma$-weakly convergent.
By the $\sigma$-strong continuity of $f_\e$, the net $f_\e(x_\alpha)$ converges to $f_\e(x)$ $\sigma$-strongly, hence $\sigma$-weakly.
Therefore, by the closedness of $F$,
\[f_\e(x)=\lim_\alpha f_\e(x_\alpha)\le\lim_\alpha f_{\e/2}(y_\alpha)\in F,\]
so we conclude $f_\e(x)\in F-M^+$.
\end{pf}



\begin{lem}
For $z\in\fm^{sa}$, we have
\[\inf\{\f(a):z\le a\in\fm^+\}\le\|\theta(z)\|.\]
In particular, for $x,y\in\fn^+$ and for any $\e>0$ there is $a\in\fm^+$ such that $x^2-y^2\le a$ and
\[\f(a)\le\|\theta(x^2-y^2)\|+\e=\|\omega_{\Lambda(x)}-\omega_{\Lambda(y)}\|+\e.\]
\end{lem}
\begin{pf}
Denote by $p(z)$ the left-hand side of the inequality.
Then, we can check $p:\fm^{sa}\to\R_{\ge0}$ is a semi-norm such that $p(z)=\f(z)$ for $z\ge0$.
(If we take $p(z):=\f(z^+)$, then it seems to be dangerous when checking the sublinearity. I could not find the counterexample for $(z_1+z_2)^+\le z_1^++z_2^+$.)

Fix any non-zero $z_0\in\fm^{sa}$.
By the Hahn-Banach extension, there is an algebraic real linear functional $l:\fm^{sa}\to\R$ such that
\[l(z_0)=p(z_0),\qquad |l(z)|\le p(z),\qquad z\in\fm^{sa}.\]
Extend linearly $l$ to be $l:\fm\to\C$.
Since $|l(z)|\le\f(z)$ for $z\in\fm^+$, by the bounded Radon-Nikodym theorem, we have a corresponding operator $a\in\pi(M)'_1$ such that $\theta^*(a)=l$, hence
\[p(z_0)=l(z_0)=\theta^*(a)(z_0)=\theta(z_0)(a)\le\|\theta(z_0)\|.\]
Since $z_0\in\fm^{sa}$ is aribtrary, we are done.
\end{pf}



\begin{thm}
Let $\f$ is a weight on a von Neumann algebra $M$.
Then, $\f$ is $\sigma$-weakly lower semi-continuous if and only if $\f$ is order continuous.
\end{thm}
\begin{pf}
($\Rightarrow$)
Easy

($\Leftarrow$)
Let $\f$ be an order continuous weight on $M$.
We first claim that the associated semi-cyclic representation $(\pi,\Lambda)$ to $\f$ is normal if $M$ is countably decomposable.

Suppose a sequence $x_n\in\fn_1$ satisfies $x_n\to x$ $\sigma$-strongly in $M$ and $\Lambda(x_n)\to\xi$ in $H$.
Since $\Lambda(x_n)$ is Cauchy and bounded, $\omega_{\Lambda(x_n)}$ is also Cauchy in the norm topology of $B(H)_*$, so we may assume $\|\omega_{\Lambda(x_{n+1})}-\omega_{\Lambda(x_n)}\|<2^{-n}$.
In order to dominate $x_n$ with an monotone sequence, we take $a_n\in\fm^+$ such that $|x_{n+1}|^2-|x_n|^2\le a_n$ and $\f(a_n)<2^{-n}$ using the previous lemma.
Since the limit of the increasing sequence $\sum_{k=1}^n a_k$ in $n\to\infty$ may not exist, we introduce the cutoff $f_\e(t):=t(1+\e t)^{-1}$.
By taking the limit $\e\to0$ on the inequality
\[\f(f_\e(|x|^2))=\f(\lim_{n\to\infty}f_\e(|x_n|^2))\le\f(\sup_nf_\e(|x_1|^2+\sum_{k=1}^na_k))=\sup_n\f(f_\e(|x_1|^2+\sum_{k=1}^na_k))<\f(|x_1|^2)+1,\]
we have $x\in(\fn_\f)_1$ and $\Lambda(x)\in H_1$.
Next, since $\Lambda(x_n-x)$ is Cauchy, we may assume $\|\omega_{\Lambda(x_n-x)}-\omega_{\Lambda(x_{n+1}-x)}\|<2^{-n}$.
Take $b_n\in\fm^+$ such that $|x_n-x|^2-|x_{n+1}-x|^2\le b_n$ and $\f(b_n)<2^{-n}$.
As we did previously, by taking $\e\to0$ on the inequality
\[\f(f_\e(|x_m-x|^2))=\f(\lim_{n\to\infty}f_\e(|x_m-x|^2-|x_n-x|^2))\le\f(\sup_nf_\e(\sum_{k=m}^nb_k))=\sup_n\f(f_\e(\sum_{k=m}^nb_k))<2^{-(m-1)},\]
we have $\|\Lambda(x_n)-\Lambda(x)\|^2=\f(|x_n-x|^2)\to0$ and $\xi=\lim_{n\to\infty}\Lambda(x_n)=\Lambda(x)$.
Thus $(\pi,\Lambda)$ is normal.


In the spirit of the Krein-\v Smulian theorem, the $\sigma$-weak lower semi-continuity is equivalent to the $\sigma$-weak closedness of the bounded part of the inverse image of the closed interval
\begin{align*}
\f^{-1}([0,1])_1
&=\{x\in M^+:\f(x)\le1,\ \|x\|\le1\}\\
&=\{x^*x\in\fm^+:\|\Lambda(x)\|\le1,\ \|x\|\le1\}.
\end{align*}
Since the $\sigma$-weak and strong closedness of a bounded convex set are equivalent and that the square root operation is strongly continuous, we are enough to show the square root
\[\f^{-1}([0,1])_1^{\frac12}=\{x\in\fn^+:\|\Lambda(x)\|\le1,\ \|x\|\le1\}\]
is $\sigma$-weakly closed.
This set, if we denote the graph of $\Lambda:\fn\to H$ by $\Gamma_\Lambda$, is exactly the image of the positive part of the unit ball
\[(\Gamma_\Lambda)^+_1=\{(x,\Lambda(x))\in\fn^+\oplus_\infty H:\|\Lambda(x)\|\le1,\ \|x\|\le1\}\]
under the projection $M\oplus_\infty H\to M$.
Observe $(x_\alpha,\xi_\alpha)$ converges to $(x,\xi)$ weakly$^*$ in $M\oplus_\infty H\cong(M_*\oplus_1H)^*$ if and only if $x_\alpha\to x$ $\sigma$-weakly and $\xi_\alpha\to\xi$ weakly.
Since the graph of $\Lambda$ and the closed ball in $M\oplus_\infty H$ is a closed with respect to the $\sigma$-weak topology of $\fn$ and the weak topology of $H$, their intersection $(\Gamma_\Lambda)_1^+$ is weakly$^*$ closed.
Therefore, by its compactness, $\f$ is $\sigma$-wealy lower semi-continuous done provided $M$ is countably decomposable.


Now, let $M$ be an arbitrary von Neumann algebra, and let $\f$ be a order continuous weight on $M$.
Let $\Sigma$ be the set of all countably decomposable projections of $M$ and let $M_0:=\bigcup_{p\in\Sigma}pMp$.
The equivalent condition for $x\in M$ to belong to $M_0$ is that the left and right support projections of $x$ are countably decomposable.
Since then the left support projection $p$ and the right support projection $q$ of $x$ are Murray-von Neumann equivalent so that there is a $*$-isomorphism between $pMp$ and $qMq$, the countable decomposability is equivalent for $p$ and $q$.
It implies that $M_0$ is an algebraic ideal of $M$.
(Moreover, $M_0$ is $\sigma$-weakly sequentially closed in $M$ since if a sequence $x_n\in M_0$ converges to $x\in M$ $\sigma$-weakly, then for $p_n\in\Sigma$ such that $x_n=p_nx_np_n$, we have $p\in\Sigma$ with $p_n\le p$ so that $x_n=px_np$ converges to $x=pxp$ $\sigma$-weakly.
This fact is not needed in the proof.)

We first claim that $\f^{-1}([0,1])_1$ is relatively $\sigma$-weakly closed in $M_0$.
Let $y\in\bar{\f^{-1}([0,1])_1}^{\sigma w}\cap M_0$ so that there is a net $y_\alpha\in\f^{-1}([0,1])_1$ converges $\sigma$-weakly to $y$, and there is $p\in\Sigma$ such that $pyp=y$.
Note that the previous theorem states that $\f^{-1}([0,1])\cap pMp$ is $\sigma$-weakly closed in $pMp$, and hence in $M$.
Since $py_\alpha p$ is a net in $\f^{-1}([0,1])_1\cap pMp$ that also converges $\sigma$-weakly to $pyp=y$, we have $y\in\f^{-1}([0,1])$.
The claim proved.

We now claim that $\f^{-1}([0,1])_1$ is $\sigma$-weakly closed in $M$.
Suppose a net $x_\alpha\in\f^{-1}([0,1])_1$ converges to $x\in M$ $\sigma$-weakly.
Clearly $x\in M_1^+$.
Let $\{p_i\}_{i\in I}$ be a maximal mutually orthogonal projections in $\Sigma$, and let $p_J:=\sum_{i\in J}p_i$ for finite sets $J\subset I$ so that $\sup_Jp_J=1$.
It clearly follows that for each $\alpha$ we have
\[x_\alpha^{\frac12}p_Jx_\alpha^{\frac12}\in\f^{-1}([0,1])_1.\]
Then, we can show easily with boundedness of $x_\alpha$ that
\[x^{\frac12}p_Jx^{\frac12}\in\bar{\f^{-1}([0,1])_1}^{\sigma w}.\]
Because $p_J\in M_0$ and $M_0$ is an ideal, 
\[x^{\frac12}p_Jx^{\frac12}\in\bar{\f^{-1}([0,1])_1}^{\sigma w}\cap M_0.\]
By the above claim,
\[x^{\frac12}p_Jx^{\frac12}\in\f^{-1}([0,1])_1.\]
By the complete additivity (or the order continuity) of $\f$, we finally obtain
\[x\in\f^{-1}([0,1])_1.\]
Therefore, $\f^{-1}([0,1])_1$ is $\sigma$-weakly closed.
\end{pf}





\newpage
\section{November 10}
\subsection{Hilbert algebras}

\begin{defn}[Left Hilbert algebra]
A \emph{left Hilbert algebra} is a $*$-algebra $A$ together with an inner product such that the involution is closable on $H$ and the square $A^2$ is dense in $H$, where $H:=\bar A$.
A left Hilbert algebra $A$ has the following additional devices:
\begin{enumerate}[(i)]
\item a closable densely defined anti-linear operator $S:A\to H$, defined by the involution,
\item a faithful non-degenerate $*$-homomorphism $\lambda:A\to B(H)$, defined by the left multiplication.
\end{enumerate}
The \emph{associated von Neumann algebra} of a left Hilbert algebra $A$ is defined as $M:=\lambda(A)''$.
\end{defn}

\begin{defn}[Right Hilbert algebra]
Let $A$ be a left Hilbert algebra.
For $\eta\in H$, define:
\begin{enumerate}[(i)]
\item a linear functional $F\eta:A\to\C$ such that $F\eta(\xi):=\<\eta,S\xi\>$ for $\xi\in A$,
\item a linear operator $\rho(\eta):A\to H$ such that $\rho(\eta)\xi:=\lambda(\xi)\eta$ for $\xi\in A$.
\end{enumerate}
Define also:
\[D':=\{\eta\in H\mid F\eta\text{ is bounded}\},\qquad
B':=\{\eta\in H\mid \rho(\eta)\text{ is bounded}\},\qquad
A':=B'\cap D'.\]
Then, for $\eta\in D'$, we can identify $F\eta$ with a vector in $H$ by the Riesz representation theorem, and for $\eta\in B'$, we can identify $\rho(\eta)$ with an element of $B(H)$.
\end{defn}


\begin{prop}
Let $A$ be a left Hilbert algebra.
\begin{parts}
\item $A'$ is a $*$-algebra such that $\eta^*:=F\eta$ and $\eta\zeta:=\rho(\zeta)\eta$.
\item $\rho(A')A'$ is dense in $H$.
\item $A'$ is a right Hilbert algebra such that $\bar{A'}=H$.
\end{parts}
\end{prop}
\begin{pf}
(a)
Combining from (i) to (iv) in the below, the claim follows clearly:

(i)
For $\eta\in D'$, we have $FF\eta=\eta$ in $H$ by
\[FF\eta(\xi)=\<F\eta,S\xi\>=\<SS\xi,\eta\>=\<\xi,\eta\>,\qquad\xi\in A.\]
Therefore, if $\eta\in D'$, then $F\eta\in D'$.

(ii)
For $\eta\in D'$, we have $\rho(F\eta)=\rho(\eta)^*$ on $A$ by
\begin{align*}
\<\rho(F\eta)\xi,\xi\>
&=\<\lambda(\xi)F\eta,\xi\>
=\<F\eta,\lambda(\xi)^*\xi\>
=\<S\lambda(\xi)^*\xi,\eta\>\\
&=\<\lambda(\xi)^*\xi,\eta\>
=\<\xi,\lambda(\xi)\eta\>
=\<\xi,\rho(\eta)\xi\>
=\<\rho(\eta)^*\xi,\xi\>
,\qquad\xi\in A.
\end{align*}
Therefore, if $\eta\in A'$, then $F\eta\in B'$.

(iii)
For $\eta,\zeta\in B'$, we have $F(\rho(\eta)^*\zeta)=\rho(\zeta)^*\eta$ in $H$ by
\begin{align*}
\<F(\rho(\eta)^*\zeta),\xi\>
&=\<S\xi,\rho(\eta)^*\zeta\>
=\<\rho(\eta)S\xi,\zeta\>
=\<\lambda(\xi)^*\eta,\zeta\>\\
&=\<\eta,\lambda(\xi)\zeta\>
=\<\eta,\rho(\zeta)\xi\>
=\<\rho(\zeta)^*\eta,\xi\>
,\qquad\xi\in A.
\end{align*}
Therefore, if $\eta,\zeta\in B'$, then $\rho(\eta)^*\zeta\in D'$.

(iv)
For $\eta\in B'$ and $\zeta\in H$, we have $\rho(\rho(\eta)^*\zeta)=\rho(\eta)^*\rho(\zeta)$ on $A$ by
\begin{align*}
\<\rho(\rho(\eta)^*\zeta)\xi,\xi\>
&=\<\lambda(\xi)\rho(\eta)^*\zeta,\xi\>
=\<\zeta,\rho(\eta)\lambda(\xi)^*\xi\>
=\<\zeta,\lambda(\lambda(\xi)^*\xi)\eta\>\\
&=\<\zeta,\lambda((S\xi)\xi)\eta\>
=\<\zeta,\lambda(\xi)^*\lambda(\xi)\eta\>
=\<\lambda(\xi)\zeta,\lambda(\xi)\eta\>\\
&=\<\rho(\zeta)\xi,\rho(\eta)\xi\>
=\<\rho(\eta)^*\rho(\zeta)\xi,\xi\>
,\qquad\xi\in A.
\end{align*}
Therefore, if $\eta,\zeta\in B'$, then $\rho(\eta)\zeta\in B'$.


(b)
Since $D'$ is dense in $H$ by the closability of $S$, it suffices to verify the inclusion $D'\subset\bar{\rho(A')A'}$.
Let $\eta\in D'$.
Since $\rho(\eta)$ has densely defined adjoint $\rho(F\eta)$, we may assume $\rho(\eta)$ to be closed and densely defined by taking closure, so we can write down the polar decomposition
\[\rho(\eta)=vh=kv,\qquad h:=|\rho(\eta)|,\quad k:=|\rho(\eta)^*|.\]
To control the unboundedness of $\rho(\eta)$, we introduce $f\in C_c((0,\infty))^+$ to cutoff $\rho(\eta)$.
Let $\acute f(t):=tf(t)$ and $\grave f(t):=t^{-1}f(t)$.
Now we have $f(k)\in\rho(B')$ since $f(k)$ is bounded and
\[f(k)=f(vhv^*)=vf(h)v^*=v\grave f(h)\rho(\eta)^*=\rho\left(v\grave f(h)F\eta\right).\]
We also have $f(k)\eta\in B'$ since
\[\rho(f(k)\eta)=f(k)\rho(\eta)=\acute f(k)v\]
is bounded.
Applying the above arguments for $f^{\frac13}\in C_c((0,\infty))$,
\[f(k)\eta=(f(k)^{\frac13})^3\eta\in\rho(B')^*\rho(B')\rho(B')^*B'.\]
Because $\rho(B')^*B'\subset A'$ and $\rho(B')^*\rho(B)\subset\rho(A')$ by (iii) and (iv) in the part (a), we have $f(k)\eta\in\rho(A')A'$.

If we construct a non-decreasing net $f_\alpha\in C_c((0,\infty))$ such that $\sup_\alpha f_\alpha=1_{(0,\infty)}$, then the strong limit implies
\[\lim_\alpha f_\alpha(k)\eta=1_{(0,\infty)}(k)\eta=s(k)\eta=s_l(\rho(\eta))\eta.\]
Here we use the non-degeneracy of $\lambda$ to verify $\eta$ belongs to the closure of the range of $\rho(\eta)$, i.e.~since $M$ contains the identity operator on $H$, we have a net $\xi_\alpha\in A$ such that $\lambda(\xi_\alpha)$ converges to the identity strongly so that $\lambda(\xi_\alpha)\eta\to\eta$.
It implies that $\eta\in\bar{\lambda(A)\eta}=\bar{\rho(\eta)A}$ and $s_l(\rho(\eta))\eta=\eta$.
Therefore, $\eta=s_l(\rho(\eta))\eta\in\bar{\rho(A')A'}$.

(c)
The involution $F:A'\to H$ is a closable densely defined anti-linear operator because $A'$ is dense in $H$ by (b) and the closability follows from the dense domain of its adjoint $S$.
The right multiplication $\rho:A'^\op\to B(H)$ is a faithful non-degenerate $*$-homomorphism because $\rho(A')H$ is dense in $H$ by (b) and the faithfulness follows from the non-degeneracy of $\lambda$.
Therefore, $A'$ is a right Hilbert algebra with $\bar A'=H$.
\end{pf}


\begin{cor}
$\rho(A')'=M$.
\end{cor}
\begin{pf}
One direction is clear, i.e.~$\rho(A')\subset M'$ implies $\rho(A')''\subset M'$.
Conversely, let $y\in M'^+$.
Since $\rho:A'^{\operatorname{op}}\to B(H)$ is non-degenerate, there is a net $\eta_\alpha\in A'$ such that $\rho(\eta_\alpha)$ converges to the identity $\sigma$-weakly.
Then,
\[\rho(\eta_\alpha)^*y\rho(\eta_\alpha)=\rho(y^{\frac12}\eta_\alpha)^*\rho(y^{\frac12}\eta_\alpha)\in\rho(B')^*\rho(B')\subset\rho(A')\]
converges to $y$ $\sigma$-weakly, hence $y\in\rho(A')''$.
\end{pf}

\begin{defn}[Full Hilbert algebra]
Let $A$ be a left Hilbert algebra.
Symmetrically as above, starting from the right Hilbert algebra $A'$, we can construct a left Hilbert algebra $A''$.
We say $A$ is \emph{full} if $A=A''$.
\end{defn}


\begin{defn}[Modular operator and conjugation]
Let $A$ be a left Hilbert algebra.
Denote the polar decomposition of $S$ by $S=J\Delta^{\frac12}$.
The unbounded operators $\Delta$ and $J$ are called the \emph{modular operator} and the \emph{modular conjugation}.
\end{defn}

\begin{cor}
From the polar decomposition theorem for unbounded (anti-)linear operators, we have
\begin{parts}
\item $S$ is injective with $S=S^{-1}$ and $D=\dom S=\dom\Delta^{\frac12}$.
\item $F$ is injective with $F=F^{-1}$ and $D'=\dom F=\dom\Delta^{-\frac12}$.
\item $\Delta$ is an injective positive self-adjoint operator.
\item $J$ is a conjugation, i.e.~an anti-linear isometric involution.
\item $S=J\Delta^{\frac12}=\Delta^{-\frac12}J$, $F=J\Delta^{-\frac12}=\Delta^{\frac12}J$, and $J\Delta J=\Delta^{-1}$.
\end{parts}
\end{cor}





\subsection{Faithful semi-finite normal weights}

\begin{defn}
Let $\f$ be a weight on a von Neumann algebra $M$.
We say $\f$ is \emph{faithful} if $\f(x^*x)=0$ implies $x=0$ for $x\in\fn$.
We say $\f$ is \emph{semi-finite} if $\fn$ is $\sigma$-weakly dense in $M$.
Recall that a weight $\f$ on a von Neumann algebra $M$ is normal if and only if it is obtained by the pointwise supremum of a set of normal positive linear functionals.
\end{defn}

\begin{prop}
For a semi-finite weight $\f$ of a von Neumann algebra $M$, there is a net $e_\alpha$ in $\fm^+$ such that $e_\alpha\uparrow1$.
\end{prop}
\begin{pf}
Since $\fn$ is convex, we may suppose $\fn$ is $\sigma$-strongly dense in $M$.
For a projection $p\in M$, we have a net $x_\alpha\in\fn$ such that $x_\alpha\to p$ $\sigma$-strongly.
Because $(x_\alpha+x_\alpha^*)/2\to p$ $\sigma$-strongly, $\fa$ is $\sigma$-strongly dense in $M$ because the linear span of projections is norm dense in $M$.
Since $\fa$ is a $*$-subalgebra of $M$, by the Kaplansky density theorem, there is a net $e_\alpha\in\fa_1^+$ such that $e_\alpha\to1$ $\sigma$-strongly.
We may assume $e_\alpha\in\fa_{<1}^+$.
Because $e_\alpha$ is bounded, we have $e_\alpha\to1$ $\sigma$-strongly, so we may assume $e_\alpha\in\fm_{<1}^+$.
Since $\fm$ is a hereditary $*$-subalgebra of $M$, the partially ordered set $\fm_{<1}^+$ is directed, so we are done.
\end{pf}


In the proofs of theorems of this section, the following diagram might be helpful: 
\[\begin{tikzcd}
\fm:=\fn^*\fn \rar[phantom,"\subset"] & \fa:=\fn\cap\fn^* \rar[phantom,"\subset"]\dar[shift left]{\Lambda} & \fn \rar[phantom,"\subset"]\dar[shift left]{\Lambda} & \pi(M) \rar[phantom,"\subset"] & B(H)\\
& A \rar[phantom,"\subset"]\uar[shift left]{\lambda} & B \rar[phantom,"\subset"]\uar[shift left]{\lambda} & H.
\end{tikzcd}\]
Recall that for a weight $\f$ on a von Neumann algebra $M$ and its semi-cyclic representation $(\pi,\Lambda)$ of $M$ we have $\f(x^*x)=\|\Lambda(x)\|^2$ for $x\in\fn$.


\begin{thm}
Let $M$ be a von Neumann algebra.
If $A$ is a full left Hilbert algebra together with a faithful normal non-degenerate representation $\pi:M\to B(H)$ such that $\lambda(A)''=\pi(M)$, then
\[\f(x^*x):=\begin{cases}\|\xi\|^2&\text{ if }\pi(x)=\lambda(\xi)\in\lambda(B),\\\infty&\text{ otherwise},\end{cases}\]
is a faithful semi-finite normal weight on $M$.
\end{thm}
\begin{pf}
We use the notation $x=\pi(x)$.
We first check that the weight $\f$ is well-defined.
Let $x_1=\lambda(\xi_1),x_2=\lambda(\xi_2)\in\lambda(B)$ such that $x_1^*x_1=x_2^*x_2$.
Since $x_1,x_2\in M$, we have a partial isometry $v\in M$ such that $x_2=vx_1$ and $v^*v=s_l(x_1)$, and it is not diffcult to see $\xi_2=v\xi_1$.
As we know $s_l(x)\xi_1=\xi_1$,
\[\|\xi_2\|^2=\<\xi_2,\xi_2\>=\<v\xi_1,v\xi_1\>=\<v^*v\xi_1,\xi_1\>=\<\xi_1,\xi_1\>=\|\xi_1\|^2,\]
which proves the well-definedness.

With this weight $\f$, we can see
\[\fn=\lambda(B),\qquad\fa=\lambda(A),\qquad\fm=\lambda(B)^*\lambda(B).\]
The first one is by definition of $\f$, and the third one is by definition of $\fm$.
Since $A$ is full so that $A=B\cap D$, $\lambda$ is injective, $\lambda(A)^*=\lambda(A)$, and $\lambda(D)^*=\lambda(D)$, we have $\lambda(A)=\lambda(B)\cap\lambda(D)=\lambda(B)^*\cap\lambda(D)$, which implies $\lambda(A)=\lambda(B)\cap\lambda(B)^*\cap\lambda(D)$.
If $\xi_1,\xi_2\in B$ satisfy $\lambda(\xi_1)=\lambda(\xi_2)^*$, then
\begin{align*}
S\xi_1(\rho(\eta)^*\zeta)
&=\<F\rho(\eta)^*\zeta,\xi_1\>
=\<\rho(\zeta)^*\eta,\xi_1\>
=\<\eta,\rho(\zeta)\xi_1\>
=\<\eta,\lambda(\xi_1)\zeta\>\\
&=\<\lambda(\xi_2)\eta,\zeta\>
=\<\rho(\eta)\xi_2,\zeta\>
=\<\xi_2,\rho(\eta)^*\zeta\>,
\qquad\eta,\zeta\in A'.
\end{align*}
We have $\xi_1\in D$ by the density of $A'^2$ in $H$, so $\lambda(B)\cap\lambda(B)^*\subset\lambda(D)$, hence the second equality follows.

The weight $\f$ is clearly faithful, and semi-finiteness follows from the assumption $\fa''=\lambda(A)''=M$ that a net $e_\alpha$ in $\fa_1^+$ convergent $\sigma$-strongly to the identity has a $\sigma$-strong limit $x=\lim_\alpha e_\alpha xe_\alpha\in\fm''$ for $x\in M$.
To verify the normality of $\f$, we will show
\[\f(x^*x)=\sup_{y\in\fn'_1}\omega_{\Lambda'(y)}(x^*x),\qquad x\in M,\]
where $\fn':=\rho(B')$.

($\ge$)
We may assume $x\in\fn$ so that $\f(x^*x)<\infty$.
Since $\fm'$ has a net $e_\alpha$ such that $e_\alpha\uparrow1$, we have an inequality
\[\omega_{\Lambda(e_\alpha)}(x^*x)=\|x\Lambda'(e_\alpha)\|^2=\|\lambda(\Lambda(x))\Lambda'(e_\alpha)\|^2=\|\rho(\Lambda'(e_\alpha))\Lambda(x)\|^2=\|e_\alpha\Lambda(x)\|^2\le\|\Lambda(x)\|^2=\f(x^*x),\]
in which the equality condition is attained at its limit.

($\le$)
Suppose $x\in M$ is taken such that the right-hand side $\sup_{y\in\fn'_1}\omega_{\Lambda'(y)}(x^*x)$ is finite.
If we show $x\in\fn$, then we are done from $\f(x^*x)<\infty$ by the previous argument.
To motivate the strategy, consider the opposite weight
\[\f'(y^*y):=\begin{cases}\|\Lambda'(y)\|^2&\text{ if }y\in\rho(B'),\\\infty&\text{ otherwise},\end{cases}\]
and the associated linear map
\[\theta'^*:M\to\fm'^\#:x^*x\mapsto(z^*y\mapsto\<x^*x\Lambda'(y),\Lambda'(z)\>),\qquad y,z\in\fn',\]
where we can check $\fm'=\rho(B')^*\rho(B')$.
The idea is to show a well-designed linear functional $l\in\fm'^\#$ such that $l=\theta'^*(x^*x)$ is contained in the image $\theta'^*(\fm)$ using the assumption that the right-hand side is finite to verify $x\in\fn$.

Define a linear functional
\[l:\fm'\to\C:z^*y\mapsto\<x^*x\Lambda'(y),\Lambda'(z)\>.\]
Then, by the assumption we have
\[\|l\|=\sup_{y\in\fn'_1}\<x^*x\Lambda'(y),\Lambda'(y)\>=\sup_{y\in\fn'_1}\omega_\eta(x^*x)<\infty,\]
and
\[|l(y)|\le\|l\|l(y^*y)^{\frac12}=\|l\|\|x\Lambda'(y)\|,\qquad y\in\fn'\]
implies the well-definedness as well as boundedness of the linear functional $\bar{xH}\to\C:x\Lambda'(y)\mapsto l(y)$ for any $\Lambda'(y)\in H$, and it follows the existence of $\xi\in\bar{xH}$ such that
\[l(y)=\<x\Lambda'(y),\xi\>,\qquad y\in\fn'\]
by the Riesz representation theorem on $\bar{xH}$.
We have $\lambda(\xi)\Lambda'(z)\in\bar{xH}$ and
\begin{align*}
\<x\Lambda'(y),x\Lambda'(z)\>&=l(z^*y)=\<x\rho^{-1}(z^*y),\xi\>=\<xz^*\Lambda'(y),\xi\>\\
&=\<z^*x\Lambda'(y),\xi\>=\<x\Lambda'(y),z\xi\>=\<x\Lambda'(y),\rho(\Lambda'(z))\xi\>\\
&=\<x\Lambda'(y),\lambda(\xi)\Lambda'(z)\>,\qquad y,z\in\fn',
\end{align*}
hence $x=\lambda(\xi)$.
The vector $\xi$ is left bounded by definition, so $x\in\fn$.
\end{pf}

\begin{thm}
Let $M$ be a von Neumann algebra.
If $\f$ is a faithful semi-finite normal weight on $M$ and $(\pi,\Lambda,H)$ is the associated semi-cyclic representation of $M$, then $A:=\Lambda(\fa)$ is a full left Hilbert algebra with
\[\Lambda(x_1)\Lambda(x_2):=\Lambda(x_1x_2),\qquad\Lambda(x)^*:=\Lambda(x^*),\]
such that $\lambda(A)''=\pi(M)$.
\end{thm}
\begin{pf}
We use the notation $x=\pi(x)$.
It does not make any confusion because the semi-cyclic representation $\pi:M\to B(H)$ is always unital and is faithful due to the assumption that $\f$ is faithful.
We clearly see that $A$ is a $*$-algebra and the left multiplication provides a $*$-homomorphism $\lambda:A\to B(H)$.
By the construction of the semi-cyclic representation associated to $\f$, $A$ is dense in $H$.
We need to show the non-degeneracy of $\lambda$, the closability of the involution, and the fullness of $A$.

(non-degeneracy)
Since $\f$ is semi-finite, there is a net $e_\alpha$ in $\fa_1^+$ converges $\sigma$-strongly to the identity of $M$ by the Kaplansky density theorem.
Then, it follows that $\lambda$ is non-degenerate from
\[\lambda(\Lambda(e_\alpha))\Lambda(x)=\Lambda(e_\alpha x)=e_\alpha\Lambda(x)\to\Lambda(x),\qquad x\in\fa.\]

(closability)
We need to prove the domain of the adjoint
\[D':=\{\eta\in H\mid A\to\C:\Lambda(x)\mapsto\<\eta,\Lambda(x^*)\>\text{ is bounded}\}\]
is dense in $H$.
Let
\[\cG:=\{\omega\in M_*^+:(1+\e)\omega\le\f\text{ for some }\e>0\}.\]
Note that the normality of $\f$ says that $\f(x^*x)=\sup_{\omega\in\cG}\omega(x^*x)$ for any $x\in M$.
For each $\omega\in\cG$, by the bounded Radon-Nikodym theorem, there is $h_\omega\in M'^+$ such that $\|h_\omega\|<1$ and
\[\omega(x^*x)=\<h_\omega\Lambda(x),\Lambda(x)\>,\qquad x\in\fn.\]
Also, if we take a net $e_\alpha\in\fn_1^+$ that converges $\sigma$-strongly to the identity of $M$ using the strong density of $\fn$ in $M$, the Kaplansky density, and the coincidence of strong and the $\sigma$-strong topologies on the bounded part, then we have a $\sigma$-weak limit $\lim_{\alpha,\beta}|e_\alpha-e_\beta|^2=0$ so that by the normality of $\omega$ we obtain
\[\lim_{\alpha,\beta}\|h_\omega^{\frac12}\Lambda(e_\alpha)-h_\omega^{\frac12}\Lambda(e_\beta)\|^2=\lim_{\alpha,\beta}\omega(|e_\alpha-e_\beta|^2)=0.\]
Thus, the vector $\Lambda_\omega:=\lim_\alpha h_\omega^{\frac12}\Lambda(e_\alpha)$ can be defined, and in particular, we have $h_\omega^{\frac12}\Lambda(x)=x\Lambda_\omega$ for $x\in\fn$ and $\omega=\omega_{\Lambda_\omega}$.

If $\eta:=h_{\omega_2}^{\frac12}y\Lambda_{\omega_1}$ for some $y\in M'$ and $\omega_1,\omega_2\in\cG$, then
\begin{align*}
|\<\eta,\Lambda(x^*)\>|
&=|\<h_{\omega_2}^{\frac12}y\Lambda_{\omega_1},\Lambda(x^*)\>|=|\<y\Lambda_{\omega_1},h_{\omega_2}^{\frac12}\Lambda(x^*)\>|=|\<y\Lambda_{\omega_1},x^*\Lambda_{\omega_2}\>|\\
&=|\<yx\Lambda_\omega,\Lambda_{\omega_2}\>|=|\<yh_{\omega_1}^{\frac12}\Lambda(x),\Lambda_{\omega_2}\>|=|\<\Lambda(x),h_{\omega_1}^{\frac12}y^*\Lambda_{\omega_2}\>|\\
&\le\|\Lambda(x)\|\|h_{\omega_1}^{\frac12}y^*\Lambda_{\omega_2}\|,\qquad x\in\fa,
\end{align*}
which deduces that $\eta\in D'$.
Therefore, it suffices to show the following space is dense in $H$:
\[\{h_{\omega_2}^{\frac12}y\Lambda_{\omega_1}:\omega_1,\omega_2\in\cG,\ y\in M'\}.\]
Thanks to the normality of $\f$, we can write
\begin{align*}
\<\Lambda(x),\Lambda(x)\>&=\|\Lambda(x)\|^2=\f(x^*x)=\sup_{\omega\in\cG}\omega(x^*x)\\
&=\sup_{\omega\in\cG}\|x\Lambda_\omega\|^2=\sup_{\omega\in\cG}\|h_\omega^{\frac12}\Lambda(x)\|^2=\sup_{\omega\in\cG}\<h_\omega\Lambda(x),\Lambda(x)\>,\qquad x\in\fa.
\end{align*}
Because $A$ in $H$, for any $\xi\in H$ and $\e>0$ there is $x\in\fn\cap\fn^*$ such that $\|\xi-\Lambda(x)\|<\e$, so the inequality
\[\<(1-h_\omega)\xi,\xi\>\le\e(\|\xi\|+\|\Lambda(x)\|)+\<(1-h_\omega)\Lambda(x),\Lambda(x)\>\]
deduces $\inf_{\omega\in\Phi}\<(1-h_\omega)\xi,\xi\>=0$ by limiting $\e\to0$ and taking infinimum on $\omega\in\cG$.
In other words, for each $\xi\in H$ and $\e>0$, we can find $\omega\in\cG$ such that $\<(1-h_\omega)\xi,\xi\><\e$.
At this point, we claim the set $\{h_\omega:\omega\in\cG\}$ is upward directed.
If the claim is true, then we can construct an increasing net $\omega_\alpha$ in $\cG$ such that $h_{\omega_\alpha}$ converges weakly to the identity of $M$, and also strongly by the nature of increasing nets.
To prove the claim, take $h_1=h_{\omega_1}$ and $h_2=h_{\omega_2}$ with $\omega_1,\omega_2\in\cG$.
Introduce a operator monotone function $f(t):=t/(1+t)$ and its inverse $f^{-1}(t)=t/(1-t)$ to define 
\[h_0:=f(f^{-1}(h_1)+f^{-1}(h_2)).\]
Then, we have $h_0\ge h_1$, $h_0\ge h_2$, and $\|h_0\|<1$.
Consider a linear functional
\[\omega_0:\fn\to\C:x\mapsto\<h_0\Lambda(x),\Lambda(x)\>.\]
Write
\begin{align*}
\omega_0(x^*x)
&\le\<f^{-1}(h_1)\Lambda(x),\Lambda(x)\>+\<f^{-1}(h_2)\Lambda(x),\Lambda(x)\>\\
&\le(1-\|h_1\|)^{-1}\<h_1\Lambda(x),\Lambda(x)\>+(1-\|h_2\|)^{-1}\<h_2\Lambda(x),\Lambda(x)\>\\
&=(1-\|h_1\|)^{-1}\omega_1(x^*x)+(1-\|h_2\|)^{-1}\omega_2(x^*x),\qquad x\in\fn.
\end{align*}
Then, since $\omega_1$ and $\omega_2$ are normal, we can define $\Lambda_0:=\lim_\alpha h_0^{\frac12}\Lambda(x_\alpha)\in H$ for a $\sigma$-strongly convergent net $x_\alpha\in\fn_1$ to the identity of $M$ as we have taken above, and we have the vector functional $\omega_0=\omega_{\Lambda_0}$.
Henceforth, $\omega_0$ is extended to a normal positive linear functional on the whole $M$, and finally the norm condition $\|h_0\|<1$ tells us that $\omega_0\in\cG$, so the claim is true.

Now the problem is reduced to the density of $\{y\Lambda_{\omega}:\omega\in\cG,\ y\in M'\}$ in $H$.
Let $p\in B(H)$ be the projection to the closure of this space.
Then, $p\Lambda_\omega=\Lambda_\omega$ for every $\omega\in\cG$.
Since the space is left invariant under the action of the self-adjoint set $M'$, we have $p\in M$.
Then,
\[\f(1-p)=\sup_{\omega\in\cG}\omega(1-p)=\sup_{\omega\in\cG}\<(1-p)\Lambda_\omega,\Lambda_\omega\>=0\]
implies $p=1$, hence the density.

(fullness)
We have $\lambda(\Lambda(x))=x$ for $x\in\fa$ since $\Lambda(\fa)=A$ is dense in $H$ and
\[x_1\Lambda(x_2)=\Lambda(x_1x_2)=\Lambda(x_1)\Lambda(x_2)=\lambda(\Lambda(x_1))\Lambda(x_2),\qquad x_1,x_2\in\fn\cap\fn^*.\]
Also we have for $\xi=\Lambda(x)\in A$ that
\[\Lambda(\lambda(\xi))=\Lambda(\lambda(\Lambda(\xi)))=\Lambda(x)=\xi.\]
For $\xi\in B$ so that $\lambda(\xi)\in M$, since
\[\f(\lambda(\xi)^*\lambda(\xi))=\|\Lambda(\lambda(\xi))\|^2=\|\xi\|^2<\infty,\]
we get $\lambda(B)\subset\fn$.
Therefore, $A$ is full by
\[\lambda(A'')=\lambda(B)\cap\lambda(B)^*\subset\fa=\lambda(A).\qedhere\]
\end{pf}

\begin{cor}
The operations giving a faithful semi-finite normal weight and a full left Hilbert algebra in the above two theorems are mutually inverses of each other.
\end{cor}

\begin{prop}
Every von Neumann algebra admits a faithful semi-finite normal weight.
\end{prop}
\begin{pf}
Let $M$ be a von Neumann algebra and let $\{\omega_i\}_{i\in I}$ be a maximal family of normal states on $M$ with orthogonal support projections $p_i:=s(\omega_i)$.
Here, the support projection $s(\omega)$ of a normal state $\omega$ is the minimal projection $p$ such that $\omega(px)=\omega(x)=\omega(xp)$ for all $x\in M$.
Since every countably decomposable projection $p$ is a support of a normal state, a faithful normal state on $pMp$, we have $\sum_ip_i=1$.
Define a weight $\f$ by
\[\f(x):=\sum_{i\in I}\omega_i(x)=\sup_{J\Subset I}\sum_{i\in J}\omega_i(x).\]
It is faithful because $\f(x)=0$ with $x\ge0$ means $\omega_i(x)=0$ and $p_ixsp_i=0$ for all $i$, and it implies
\[x^{\frac12}=x^{\frac12}\sum_ip_i=\sum_ix^{\frac12}p_i=0.\]
It is normal because it is the supremum of normal positive linear functionals $\omega_J=\sum_{i\in J}\omega_i$.
It is semi-finite because $p_J\uparrow1$ with $\f(p_J)<\infty$ as $J\to I$, where $p_J:=\sum_{i\in J}p_i$ and $J$ runs through finite subsets of $I$.
\end{pf}



\subsection{Examples}




\begin{ex}[Locally compact groups]
Let $G$ be a locally compact group and $ds$ be a left Haar measure.
The modular function is a continuous group homomorphism $\Delta:G\to\R_{>0}^\times$ characterized by the Radon-Nikodym derivative $\Delta(s)=ds/ds^{-1}$, where $ds^{-1}$ is a right Haar measure.

The space $L^1(G)$ of integrable functions has a Banach $*$-algebra structure
\[(f*g)(s)=\int_Gf(t)g(t^{-1}s)\,dt,\qquad f^*(s)=\Delta(s^{-1})\bar{f(s^{-1})}.\]
Since the left Haar measure is a ``faithful semi-finite normal weight'' on $L^1(G)$, by the Gelfand-Naimark-Segal construction, we have a bounded $*$-homomorphism $\lambda:L^1(G)\to B(L^2(G))$ and a semi-cyclic map $L^1(G)\cap L^2(G)\to L^2(G)$.
Then, $L^1(G)$ has a dense left Hilbert algebra $A$, for example, $C_c(G)$ or $L^1(G)\cap L^2(G)\cap L^2(G)^*$.
Note that $L^1(G)$ may not admit any full left Hilbert algebra as a $*$-subalgebra.
On $A$, we have $S$, $F$, $\Delta$, and $J$ given by
\[S\xi(s)=\Delta(s^{-1})\bar{\xi(s^{-1})},\qquad F\xi(s)=\bar{\xi(s^{-1})},\]
\[J\xi(s)=\Delta(s)^{-\frac12}\bar{\xi(s^{-1})},\qquad\Delta\xi(s)=\Delta(s)\xi(s),\]
and they have the following norm formulas
\[\|S\xi\|_2=\|\Delta^{\frac12}\xi\|_2,\quad\|F\xi\|_2=\|\Delta^{-\frac12}\xi\|_2,\quad\|S\xi\|_1=\|\xi\|_1,\quad\|F\xi\|_1=\|\Delta^{-1}\xi\|_1.\]
We can do the same thing for the algebra $M(G)=C_0(G)^*$ of measures.

Since $A$ is norm dense in $L^1(G)$ and weakly$^*$ dense in $M(G)$, and since $\lambda:L^1(G)\to B(L^2(G))$ is bounded and $\lambda:M(G)\to B(L^2(G))$ is continuous between weak$^*$ topologies because $(s\mapsto\omega(\lambda_s))\in C_0(G)$ for $\omega\in B(L^2(G))_*$, we can define
\[C_r^*(G):=\bar{\lambda(A)}^{\|\cdot\|}=\bar{\lambda(L^1(G))}^{\|\cdot\|},\qquad L(G):=\bar{\lambda(A)}^{\sigma w}=\bar{\lambda(M(G))}^{\sigma w}.\]
The corresponding faithful semi-finite normal weight $\psi$ on $L(G)$ is called the \emph{Plancherel weight}, satisfying
\[\lambda(L^1(G)\cap L^2(G)\cap L^2(G)^*\cap A(G))\subset\fm,\qquad\lambda(L^1(G)\cap L^2(G)\cap L^2(G)^*)\subset\fa,\qquad\lambda(L^1(G)\cap L^2(G))\subset\fn,\]
which is strict.


\iffalse
The left involution $S$ is an isometric anti-linear automorphism of $L^1(G)=L^1(G,ds)$, but the right involution $F$ defines an isometric anti-linear isomorphism between $L^1(G,ds)$ and $L^1(G,ds^{-1})$.
To sum up,
\[\begin{array}{lrl}
\xi\in H &\Leftrightarrow& \|\xi\|_2<\infty,\\
\xi\in D &\Leftrightarrow& \|S\xi\|_2+\|\xi\|_2<\infty,\\
\xi\in A &\Rightarrow& \|\lambda(\xi)\|+\|S\xi\|_2+\|\xi\|_2<\infty,\\
\xi\in B' &\Leftrightarrow& \|\rho(\xi)\|+\|\xi\|_2<\infty,\\
\xi\in D' &\Leftrightarrow& \|F\xi\|_2+\|\xi\|_2<\infty,\\
\xi\in A' &\Leftrightarrow& \|\rho(\xi)\|+\|F\xi\|_2+\|\xi\|_2<\infty,\\
\xi\in B &\Leftrightarrow& \|\lambda(\xi)\|+\|\xi\|_2<\infty,\\
\xi\in A'' &\Leftrightarrow& \|\lambda(\xi)\|+\|S\xi\|_2+\|\xi\|_2<\infty.
\end{array}\]
\fi

\end{ex}

\begin{ex}[Locally compact abelian groups]
If $G$ is a locally compact abelian group, then $A=\cF^{-1}(L^2(\hat G)\cap L^\infty(\hat G))\subset L^2(G)$ is a full Hilbert algebra, where $\cF:L^2(G)\to L^2(\hat G)$ is the Fourier transform, such that $B=A$ and $D=H=L^2(G)$.

For a locally compact abelian group $G$, the corresponding faithful semi-finite normal weight can be identified to a Haar measure on the Pontryagin dual $\hat G$, called the \emph{Plancherel measure}, which is suitably normalized via the Fourier transform $\cF:L^1(G)\to C_0(\hat G)$.
The construction is a kind of pushforward process of a weight on $L^1(G)$ to $C_0(G)$.
The arrows on the below diagram are injective bounded $*$-homomorphisms:
\[\begin{tikzcd}
L^1(G) \rar{\lambda}\dar[swap]{\cF} & B(L^2(G)) \dar[<->]{\Ad\cF} \\
C_0(\hat G) \rar{m} & B(L^2(\hat G)).
\end{tikzcd}\]
\end{ex}

\begin{ex}[Measure spaces]
If $(X,\mu)$ is a $\sigma$-finite measure space, then $L^2(X)\cap L^\infty(X)$ is a full Hilbert algebra.
\end{ex}


\begin{ex}[Cyclic separating vector]
Let $M$ be a countably decomposable von Neumann algebra and $\omega$ be a faithful normal state.
It has
\[\fm=\fa=\fn=M,\]
and we may assume $M\subset B(H)$ by the GNS construction for $\omega$, which admits the cyclic separating vector $\Omega\in H$ such that $\omega(x)=\<x\Omega,\Omega\>$.
Then, $A:=M\Omega\subset H$ has the following full left Hilbert algebra structure:
\[\<x\Omega,y\Omega\>\text{ is defined as it is},\qquad (x\Omega)(y\Omega):=xy\Omega,\qquad (x\Omega)^*:=x^*\Omega.\]
There is no specific description of $J$ and $\Delta$ in general, but it is known that
\[D=\{c\Omega:c\in C(H)\text{ affiliated with $M$ such that }\Omega\in\dom c\cap\dom c^*\}.\]
\end{ex}


\begin{ex}[Trace of $B(H)$]
Let $\tau:=\Tr$ be the usual trace of $B(H)$.
Recall that thee space $L^2(H)$ of Hilbert-Schmidt operators is an ideal of $B(H)$.
It is a faithful semi-finite normal weight with
\[\fm=L^1(H),\qquad\fa=L^2(H),\qquad\fn=L^2(H).\]
The semi-cyclic representation is given by
\[\pi(x)\xi=x\xi,\qquad\Lambda(x)=x,\qquad H=L^2(H).\]
The left Hilbert algebra is given by
\[S\xi=\xi^*,\qquad J\xi=\xi^*,\qquad\Delta=\id_{L^2(H)}.\]



\end{ex}







\newpage
\section{December 20}

\subsection{Pettis integral}

\begin{defn}[Properties of dual pairs]
Let $(X,F)$ be a dual pair.
For example, if $X$ is a topological vector space and $F$ is a linear subspace of $X^*$, then $(X,F)$ is a dual pair if and only if $F$ is weakly$^*$ dense in $X^*$.
Conversely, every dual pair $(X,F)$ can be understood as $(X,X^*)$ by endowing with the weak topology $\sigma(X,F)$ on $X$.
Then, we say $(X,F)$ has the \emph{Krein property} if the closed balanced convex hull of a compact subset of $X$ is compact in the topology $\sigma(X,F)$, and say $(X,F)$ has the \emph{Goldstine property} if $X$ is $\beta(X,F_\beta)$-closed in the strong bidual $(F_\beta)^*_\beta$.
\end{defn}
\begin{rmk*}
Let $X$ a Banach space.
The weak dual pair $(X,X^*)$ satisfies the Krein property by the Krein-\v Smulian theorem, and the Goldstine property by the closedness of $X$ in $X^{**}$.
If there is a predual $X_*$ of $X$, then the weak$^*$ dual pair $(X,X_*)$ satisfies the Krein property by the fact that the closed convex hull of a bounded set is bounded, and the Golstine property because the norm topology and $\beta(X,(X_*)_\beta)$ coincide by the Goldstine theorem.
In particular, a dual pair $(X,F)$ with $F\subset X^*$ has the Goldstine property if and only if the closed unit ball $F_1=F\cap X^*_1$ is weakly$^*$ dense in the closed ball $X^*_1$.
\end{rmk*}

\begin{prop}[Well-definedness of Pettis integral]
Let $x:\Omega\to X$ be a $\sigma(X,F)$-bounded $\sigma(X,F)$-measurable function, where $(\Omega,\mu)$ is a localizable measure space and $(X,F)$ is a dual pair.
Then, it determines a linear operator $F\to L^\infty(\mu)$ by definition.
By the transpose and restriction, we have a linear operator $\phi_x:L^1(\mu)\to F^\#$, which satisfies
\[\<\phi_x(f),x^*\>:=\int_\Omega f(s)\<x(s),x^*\>\,d\mu(s),\qquad f\in L^1(\mu),\ x^*\in F.\]
We usually write as
\[\phi_x(f)=\int_\Omega f(s)x(s)\,d\mu(s).\]
\begin{parts}
\item $\phi_x(L^1(\mu))\subset(F_\beta)^*$ and $\phi_x$ is always weak-$\sigma((F_\beta)^*,F)$-continuous.
\item Suppose $(X,F)$ has the Krein property.
If $x$ is $\sigma(X,F)$-compactly valued, then $\phi_x(L^1(\mu))\subset X$.
\item Suppose $(X,F)$ has the Krein and Goldstine property.
Suppose $\Omega$ is a locally compact Hausdorff space with a Radon measure $\mu$.
If $x$ is $\sigma(X,F)$-continuous, then $\phi_x(L^1(\mu))\subset X$.
\item Suppose we have $\phi_x(L^1(\mu))\subset X$. Let $Y$ be another topological vector space and $G$ is a weakly$^*$ dense subspace of $Y^*$. If $T:X\to Y$ is a $\sigma(X,F)$-$\sigma(Y,G)$-continuous linear operator, then $T\phi_x=\phi_{T\circ x}$. In other words,
\[T\int_\Omega f(s)x(s)\,d\mu(s)=\int_\Omega f(s)Tx(s)\,d\mu(s),\qquad f\in L^1(\mu).\]
\item Suppose we have $\phi_x(L^1(\mu))\subset X$, $(X,F)$ has the Goldstine property, and $X$ is a Banach space. Then,
\[\|\int f(s)x(s)\,d\mu(s)\|\le\int\|f(s)x(s)\|\,d\mu(s),\qquad f\in L^1(\mu).\]
\end{parts}
\end{prop}
\begin{pf}
(a)
Let $B^*\subset F$ be a $\beta(F,X_\sigma)$-bounded set.
For $x^*\in F$ we have an inequality
\[|\<\phi_x(f),x^*\>|\le\int_\Omega |f(s)\<x(s),x^*\>|\,d\mu(s)\le\|f\|_{L^1}\sup_{y\in x(\Omega)}|\<y,x^*\>|,\]
and a bound
\[\sup_{x_*\in B^*}\sup_{y\in x(\Omega)}|\<y,x^*\>|<\infty\]
due to the $\sigma(X,F)$-boundedness of $x(\Omega)$, so $\phi_x(f)\in(F_\beta)^*$.
If $f_\alpha\in L^1(\mu)$ converges weakly to zero, then
\[\<\phi_x(f_\alpha),x^*\>=\int_\Omega f(s)\<x(s),x^*\>\,d\mu(s)\to0,\qquad x^*\in F\]
because $x$ is $\sigma(X,F)$-integrable so that $(s\mapsto\<x(s),x^*\>)\in L^\infty(\mu)$, so the continuity of $\phi_x$.

(b)
Fix $p\in L^\infty(\mu)$ and let $C$ be the $\sigma(X,F)$-closed balanced convex hull of $x(\Omega)\subset X$.
Then $C$ is $\sigma(X,F)$-compact by the Krein property.
Since for every $x^*\in F$ we have
\[|\<\phi_x(f),x^*\>|\le\int_\Omega|f(s)\<x(s),x^*\>|\,d\mu(s)\le\|f\|_{L^1}\sup_{y\in x(\Omega)}|\<y,x^*\>|\le\|f\|_{L^1}\sup_{y\in C}|\<y,x^*\>|,\]
the linear functional $\phi_x(f)$ on $F$ is continuous with respect to the Mackey topology $\tau(F,X)$, which is a dual topology so that $\phi_x(f)$ can be naturally identified with a vector in $(F_\tau)^*=X$.

(c)
Fix $f\in L^1(\mu)$.
By the tightness of $\mu$, there is a sequence of compact sets $K_n\subset\Omega$ such that $\int_{\Omega\setminus K_n}|f(s)|\,d\mu(s)<n^{-1}$.
Since for each $x^*\in F$ we have
\[|\<\phi_x(f)-\phi_{x|_{K_n}}(f),x^*\>|\le\int_{\Omega\setminus K_n}|f(s)|\,d\mu(s)\cdot\sup_{s\in\Omega}|\<x(s),x^*\>|<n^{-1}\sup_{y\in x(\Omega)}|\<y,x^*\>|\]
so that
\[\sup_{x^*\in B^*}|\<\phi_x(f)-\phi_{x|_{K_n}}(f),x^*\>|\le n^{-1}\sup_{x_*\in B^*}\sup_{y\in x(\Omega)}|\<y,x^*\>|\to0,\qquad n\to\infty,\]
which means that $\phi_{x|_{K_n}}(f)$ converges to $\phi_x(f)$ in $\beta((F_\beta)^*,F_\beta)$.
Since $\phi_{x|_{K_n}}(f)\in X$ by the part (b) and $X$ is closed in $\beta((F_\beta)^*,F_\beta)$ by the Goldstine property, we have $\phi_x(f)\in X$.

(d)
By the continuity of $T$, the adjoint $T^*:G\to F$ is well-defined.
The measurability of $T$ and the existence of the adjoint $T^*$ imply that the composition $T\circ x:\Omega\to Y$ is $\sigma(Y,G)$-bounded and $\sigma(Y,G)$-measurable, so the operator $\phi_{T\circ x}:L^1(\mu)\to G^\#$ is well-defined.
Then,
\begin{align*}
\<T\phi_x(f),y^*\>&=\<\phi_x(f),T^*y^*\>=\int_\Omega f(s)\<x(x),T^*y^*\>\,d\mu(s)\\
&=\int_\Omega f(s)\<Tx(s),y^*\>\,d\mu(s)=\<\phi_{T\circ x}(f),y^*\>,\qquad f\in L^1(\mu),\ y^*\in G.
\end{align*}
In particular, $\phi_{T\circ x}:L^1(\mu)\to Y$.

(e)
By the Goldstine property,
\begin{align*}
\|\int f(s)x(s)\,d\mu(s)\|
&=\sup_{x^*\in F_1}|\int f(s)x(s)\,d\mu(s)|
\le\sup_{x^*\in F_1}\int|f(s)x(s)|\,d\mu(s)\\
&\le\int\sup_{x^*\in F_1}|f(s)x(s)|\,d\mu(s)
\le\int\|f(s)x(s)\|\,d\mu(s).\qedhere\\
\end{align*}
\end{pf}





\subsection{Group actions on dual pairs}


\textbf{Assumption.}
From now on, we always let $G$ be a locally compact group, and let $(X,F)$ be a dual pair such that both $(X,F)$ and $(F,X)$ satisfy the Krein and Goldstine property.
Let $\End(X)$ be the topological algebra of $\sigma(X,F)$-$\sigma(X,F)$-continuous linear operators on $X$ endowed with the point-$\sigma(X,F)$ topology, and $\GL(X)$ be the group of invertible elements of $\End(X)$.

By an \emph{action} of $G$ on $(X,F)$, we mean a continuous and bounded group homomorphism $\alpha:G\to\GL(X)$ in the sense that $G\to X:s\mapsto\alpha_s(x)$ is continuous and bounded with respect to $\sigma(X,F)$ for each $x\in X$.
Furthermore, we always add the following assumptions for each case:
\begin{enumerate}[(i)]
\item For a Banach space $V$, we have $\alpha:G\to\Isom(V)$ and $(X,F)=(V,V^*)$.
\item For a C$^*$-algebra $A$, we have $\alpha:G\to\Aut(A)$ and $(X,F)=(A,A^*)$.
\item For a von Neumann algebra $M$, we have $\alpha:G\to\Aut(M)$ and $(X,F)=(M,M_*)$.
\end{enumerate}


\begin{prop}[Dual actions]
Let $\alpha:G\to\GL(X)$ be an action of $G$ on $(X,F)$.
Then, there is an action $\alpha^*:G\to\GL(F)$ on $(F,X)$ such that
\[\<x,\alpha^*_s(x^*)\>:=\<\alpha_s(x),x^*\>,\qquad x\in X,\ x^*\in F.\]
\end{prop}
\begin{pf}
Clear.
\end{pf}

\begin{prop}[Averaging]
Let $\alpha:G\to\GL(X)$ be an action of $G$ on $(X,F)$.
Then, there is a (faithful non-degenerate continuous?) homomorphism $\alpha:M(G)\to\End(X)$ using the same notation, defined by
\[\alpha(\mu)x:=\int_G\alpha_s(x)\,d\mu(s),\qquad\mu\in M(G),\ x\in X,\]
which is justified by the Pettis integral.
\end{prop}
\begin{pf}
(Well-definedness)
Since $(F,X)$ also satisfies the Krein and Goldstine properties and the dual action $\alpha^*$ can be checked to be bounded, the Pettis integral
\[\int\alpha_s^*(x^*)\,d\mu(s)\]
is well-defined in $F$.
Therefore, if a net $x_i$ converges to zero in $\sigma(X,F)$, then for each $x^*\in F$ we have
\[\<\int\alpha_s(x_i)\,d\mu(s),x^*\>
=\int\<\alpha_s(x_i),x^*\>\,d\mu(s)
=\int\<x_i,\alpha_s^*(x^*)\>\,d\mu(s)
=\<x_i,\int\alpha_s^*(x^*)\,d\mu(s)\>\to0,\]
so $\pi_\alpha(\mu)\in\End(X)$.

(Homomorphism)
For each $x\in X$, since $G\to X:s\mapsto\alpha_s(x)$ is bounded and continuous with respect to $\sigma(X,F)$, by (c) of the previous proposition, we can define the Pettis integral
\[\pi_\alpha(\mu)x:=\phi_{s\mapsto\alpha_s(x)}(1)=\int_G\alpha_s(x)\,d\mu(s),\qquad x\in X,\ \mu\in M(G).\]
For $\mu,\nu\in M(G)$,
\begin{align*}
\pi_\alpha(\mu*\nu)x
&=\iint\alpha_{st}(x)\,d\mu(s)\,d\nu(t)=\iint\alpha_s(\alpha_t(x))\,d\nu(t)\,d\mu(s)\\
&=\int\alpha_s\Bigl(\int\alpha_t(x)\,d\nu(t)\Bigr)\,d\mu(s)=\pi_\alpha(\mu)\pi_\alpha(\nu)x,\qquad x\in M.\qedhere
\end{align*}




\end{pf}
\begin{rmk*}
Let $\alpha:G\to\End(V)$ be an action on a Banach space $V$.
Recall that $\alpha_s$ is an isometry for each $s\in G$ by additional assumption.
Consider the set $S:=\{x\in V:\lim_{s\to e}\|\alpha_s(x)-x\|=0\}$.
It is a strongly closed convex set, which is weakly closed by the Hahn-Banach separation.
Define
\[x_\alpha:=\int_Ge_\alpha(t)\alpha_t(x)\,dt,\]
where $e_\alpha$ is an approximate identity of $L^1(G)$ in $C_c(G)^+$ such that $\|e_\alpha\|_{L^1(G)}=1$ and $\supp e_\alpha\to\{e\}$.
Since $x_\alpha\to x$ weakly in $V$ because the weak continuity of the action implies
\[|\<x_\alpha-x,x^*\>|=|\int_Ge_\alpha(t)\<\alpha_t(x)-x,x^*\>\,dt|\le\sup_{t\in\supp e_\alpha}|\<\alpha_t(x)-x,x^*\>|\xrightarrow{\alpha}0,\]
and $x_\alpha\in S$ for each index $\alpha$ because the uniform continuity of $e_\alpha$ implies
\[\|\alpha_s(x_\alpha)-x_\alpha\|=\|\int_G(e_\alpha(s^{-1}t)-e_\alpha(t))\alpha_t(x)\,dt\|\le\|x\|\int_G|e_\alpha(s^{-1}t)-e_\alpha(t)|\,dt\xrightarrow{s\to e}0,\]
so we have $S=V$.
It means that the function $G\to V:s\mapsto\alpha_s(x)$ is strongly measurable for each $x\in V$, which allows to use the Bochner integral to justify the above integral.
However, for von Neumann algebras, the $\sigma$-weak continuity cannot imply the strong measurability in general, so we need to develop the Pettis integral.
\end{rmk*}




\subsection{One-parameter groups}

An action of $G=\R$ is somtimes called a \emph{one-parameter group}, or a \emph{flow}.


\begin{prop}
exponential growth, infinitesimal generator, smooth elements
\end{prop}

\begin{prop}
Hille-Yosida, Lumer-Phillips
\end{prop}



\begin{prop}[Smoothing operator]
Let $\alpha:\R\to\GL(X)$ be a flow on $(X,F)$.
(I guess the boundedness is not necessary)
For each $n>0$, define a linear operator $R_n:X\to X$ such that
\[R_n(x):=\sqrt{\frac n\pi}\int_\R e^{-ns^2}\alpha_s(x)\,ds,\qquad x\in X.\]
\begin{parts}
\item $R_n(x)\to x$ in $\sigma(X,F)$.
\item $R_n(x)\to x$ strongly in $V_1$ if $(X,F)=(V,V^*)$ for a Banach space $V$.
\item $R_n(x)\to x$ $\sigma$-strongly$^*$ if $(X,F)=(M,M_*)$ for a von Neumann algebra $M$.
\end{parts}
\end{prop}
\begin{pf}
Relatively obvious.
\end{pf}


\begin{prop}[Analytic continuation]
Let $\alpha:\R\to\GL(X)$ be a flow on $(X,F)$.
We have a family of densely defined closed operators $\{\alpha_z:z\in\C\}$ on $X$ which extends the original $\alpha$, such that
\begin{enumerate}[(i)]
\item $\alpha_z\alpha_t=\alpha_{z+s}=\alpha_s\alpha_z$ and $\alpha_z\alpha_w\subset\alpha_{z+w}$ for $s\in\R$ and $z,w\in\C$,
\item $\alpha_z^{-1}=\alpha_{-z}$,
\item dual action?
\item $\dom\alpha_z\subset\dom\alpha_w$ if $\Im z\ge\Im w\ge0$,
\item $\bigcap_{z\in\C}\dom\alpha_z$ is dense in $X$.
\end{enumerate}
\end{prop}
\begin{pf}
Define $\alpha_z:\{R_n(x):n\in\N,\ x\in X\}\to X$ for $z\in\C$ such that
\[\alpha_z(R_n(x)):=\sqrt{\frac n\pi}\int_\R e^{-n(s-z)^2}\alpha_s(x)\,ds.\]
It satisfies some properties:
\begin{parts}
\item It extends the original $\{\alpha_s:s\in\R\}$.
\item For fixed $x\in X_0$, $z\mapsto\alpha_z(x)$ is $\sigma(X,F)$-entire.
\item $X_0$ is $\sigma(X,F)$-dense in $E$, so $\alpha_z$ is densely defined for each $z\in\C$.
\item $\alpha_z$ is closable for each $z\in\C$.
\end{parts}
(a) is clear by coordinate change, and (b) follows from the Fubini and the Morera after taking arbitrary elements of $E^*$.
(c) is by an approximate identity $e_n$ of $L^1(\R)$ has $x=\lim_{n\to\infty}\int_\R e_n(s)\alpha_s(x)\,ds$.
For (d), we have the adjoint $(\alpha_z)_0^*\supset(\alpha_{-\bar z})_0$, which is densely defined.
Now we have a family of closed densely defined operators $\{\alpha_z:z\in\C\}$ on $E$ such that $\alpha_z\alpha_w\subset\alpha_{z+w}$ for all $z,w\in\C$.
\end{pf}

\begin{defn}[Entire elements]
The set of entire elements is $\bigcup_{z\in\C}\dom\alpha_z=\bigcup_{n\in\Z}\dom\alpha_{ni}$, which is dense.
If $X$ is a von Neumann algebra, then it is a $*$-subalgebra of $M$.

\end{defn}





\subsection{Tomita-Takesaki commutation theorem}

In this section, we let $A$ be a left Hilbert algebra.
We will use the following notations freely:
\[H,M,S,\lambda,F,\rho,B,D,A',B',D',\Delta,J.\]
Also note that
\[\begin{tikzcd}
\fm:=\fn^*\fn \rar[phantom,"\subset"] & \fa:=\fn^*\cap\fn \rar[phantom,"\subset"]\dar[shift left]{\Lambda} & \fn \rar[phantom,"\subset"]\dar[shift left]{\Lambda} & M\\
& A \rar[phantom,"\subset"]\uar[shift left]{\lambda} & B \rar[phantom,"\subset"]\uar[shift left]{\lambda} & H.
\end{tikzcd}\]

The goal of this section is to prove that there exists the following commutative ``cube'' diagram:
\[\begin{tikzcd}[column sep={3.5em,between origins},row sep={2.5em,between origins}]
&A\ar{rr}{\Delta^{it}}&&A\ar{dd}{\lambda}\\
A'\ar[<->]{ur}{J}\ar[swap]{rr}{\Delta^{it}}\ar[swap]{dd}{\rho}&&A'\ar[<->,swap]{ur}{J}\ar{dd}{\rho}&\\
&&&\lambda(A)\\
\rho(A')\ar[swap]{rr}{\Ad\Delta^{it}}&&\rho(A')\ar[<->,swap]{ur}{\Ad J}&.
\end{tikzcd}\]


\begin{lem*}
For every $t\in\R$, the unitary operator $\Delta^{it}$ commutes with $J$, $S$, and $F$.
\end{lem*}
\begin{pf}
It is enough to show $\Delta^{it}J=J\Delta^{it}$.
By the relation $J\Delta J=\Delta^{-1}$, the anti-linearity of $J$, and the uniqueness of the bounded Borel functional calculus, we have the commutation.
More precisely, if we let $f(s):=e^{it\log s}$ on $(0,\infty)$, then
\[\Delta^{-it}=f(\Delta^{-1})=f(J\Delta J)=J\bar{f(\Delta)}J=J(\Delta^{it})^*J=J\Delta^{-it}J.\]
(Here we omit the detailed proof of $f(J\Delta J)=J\bar{f(\Delta^{-1})}J$.)
\end{pf}


\begin{lem*}
$J:D'\to D$ and $\Delta^{it}:D\to D$.
\end{lem*}
\begin{pf}
We have $J:D'\to D$ since $\eta\in D'$ implies that $SJ\eta=JF\eta$ is well-defined in $H$.
We have $\Delta^{it}:D\to D$ for real $t$ since $\xi\in D$ implies that $S\Delta^{it}\xi=\Delta^{it}S\xi$ is well-defined in $H$ because $S\xi\in D$.
\end{pf}



We need two critical lemmas.

\begin{lem}[Fourier inversion of sech]
Let $\alpha:\R\to\Isom(X)$ be flows on $(X,F)$.
Then, we have a Pettis integral
\[\int_\R\frac{e^{-ist}}{e^{\pi t}+e^{-\pi t}}\alpha_t(x)\,dt
=(e^{-\frac s2}\alpha_{-\frac i2}+e^{\frac s2}\alpha_{\frac i2})^{-1}x,\qquad s\in\R,\ x\in\dom\alpha_{-\frac i2}\cap\dom\alpha_{\frac i2}.\]
\end{lem}
\begin{rmk*}
Consider
\[f(t):=\frac1{e^{\frac t2}+e^{-\frac t2}},\qquad\hat f(t)=\frac{\sqrt{2\pi}}{e^{\pi t}+e^{-\pi t}},\] and write $\alpha_t=e^{t\delta}$ formally, then the equation in the lemma can be rewritten as the Fourier inversion
\[\frac1{\sqrt{2\pi}}\int_\R e^{it(-i\delta-s)}\hat f(t)\,dt=f(-i\delta-s),\qquad s\in\R.\]
However, this Fourier calculus in general setting using an unbounded holomorphic functional calculus for unbounded operators acting on Banach spaces is impossible, because even for a fairly normal example (e.g.~$\sigma_t=\Ad u_t$, $u_t$ is given by the translation on $L^2(\R)$) we have a counterexample having the entire spectrum of the analytic generator $\sigma(\sigma_{-i})=\C$.

Note also that this theorem does not consist of any operator algebras.
\end{rmk*}
\begin{pf}
We use the special fact that the function $\hat f(t):=\sqrt{2\pi}(e^{\pi t}+e^{-\pi t})^{-1}$ has imaginary period $i$.
Fix $s\in\R$ and $x\in\dom\alpha_{-\frac i2}\cap\dom\alpha_{\frac i2}$.
Define a $\sigma(X,F)$-meromorphic vector function $g:\C\setminus i\Z\to X$ such that
\[g(z):=-i\frac{\sqrt{2\pi}}{e^{\pi z}-e^{-\pi z}}e^{-isz}\alpha_z(x).\]
It satisfies relations
\[g(t-\frac i2)=e^{-\frac s2}\alpha_{-\frac i2}(e^{-ist}\hat f(t)\alpha_t(x)),\qquad
g(t+\frac i2)=-e^{\frac s2}\alpha_{\frac i2}(e^{-ist}\hat f(t)\alpha_t(x))\]
and enjoys an estimate
\[\sup_{|r|\le\frac12}\|g(t+ir)\|\le\sup_{|r|\le\frac12}\sqrt{2\pi}\frac{e^{sr}}{|e^{\pi(t+ir)}-e^{-\pi(t+ir)}|}\|x\|=O(e^{-\pi|t|}),\qquad|t|\to\infty.\]
Then, by the residue theorem
\begin{align*}
\sqrt{2\pi}x=2\pi\lim_{z\to0}zg(z)
&=\int_{-\infty}^\infty g(t-\frac i2)\,dt-\int_{-\infty}^\infty g(t+\frac i2)\,dt\\
&=\int_{-\infty}^\infty(e^{-\frac s2}\alpha_{-\frac i2}+e^{\frac s2}\alpha_{\frac i2})e^{-ist}\hat f(t)\alpha_t(x)\,dt,
\end{align*}
where the limit is in $\sigma(X,F)$.
Extend for $x\in X$ using the boundedness.

Using the dual action, the analytic generators can get out of the integral.
It is because the domain of the analytic generator of the dual isometric action is $\sigma(F,X)$-dense, and is preserved by the analytic generator.

\end{pf}





\begin{lem}
Let $A$ be a left Hilbert algebra.
For $s\in\R$, we have $(e^{-s}+\Delta)^{-1}:A'\to A\cap D'$.
In particular, $A\cap D'$ is dense in $H$.
\end{lem}
\begin{pf}
Let $\eta\in A'$ and $\xi:=(e^{-s}+\Delta)^{-1}\eta$.
Then, $\Delta\xi=\eta-e^{-s}\xi\in H$ implies $\xi\in\dom\Delta\subset\dom\Delta^{\frac12}=D$, and $F\xi=e^s(F\eta-S\xi)\in H$ implies $\xi\in D'$.
The only non-trivial fact is $\xi\in B$.
Since $\xi\in D$, by the polar decomposition, we have
\[\lambda(\xi)=vh=kv,\qquad h:=|\lambda(\xi)|,\quad k:=|\lambda(\xi)^*|.\]
Let $f\in C_c((0,\infty))^+$.
Since
\begin{align*}
\<f(h)S\xi,\zeta\>
&=\<S\xi,f(h)\zeta\>
=\<Ff(h)\zeta,\xi\>
=\<Fv^*\grave f(k)\lambda(\xi)\zeta,\xi\>
=\<F\lambda(v^*\grave f(k)\xi)\zeta,\xi\>\\
&=\<F\rho(\zeta)v^*\grave f(k)\xi,\xi\>
=\<\rho(v^*\grave f(k)\xi)^*F\zeta,\xi\>
=\<F\zeta,\rho(v^*\grave f(k)\xi)\xi\>\\
&=\<F\zeta,\lambda(\xi)v^*\grave f(k)\xi\>
=\<F\zeta,f(k)\xi\>
=\<Sf(k)\xi,\zeta\>,\qquad\xi\in D,\ \zeta\in D',
\end{align*}
we have
\begin{align*}
\|f(k)\eta\|^2
&=\|f(k)(e^{-s}+\Delta)\xi\|^2\\
&=e^{-2s}\|f(k)\xi\|^2+\|f(k)\Delta\xi\|^2+2e^{-s}\Re\<f(k)\xi,f(k)\Delta\xi\>\\
&\ge2e^{-s}\|f(k)\xi\|\|f(k)\Delta\xi\|+2e^{-s}\Re\<f(k)\xi,f(k)\Delta\xi\>\\
&\ge4e^{-s}\Re\<f(k)\xi,f(k)\Delta\xi\>\\
&=4e^{-s}\Re\<f(k)^2\xi,FS\xi\>\\
&=4e^{-s}\Re\<Sf(k)^2\xi,S\xi\>\\
&=4e^{-s}\Re\<f(h)^2S\xi,S\xi\>\\
&=4e^{-s}\|f(h)S\xi\|^2,
\end{align*}
and
\begin{align*}
\|f(k)\eta\|^2
&=\|\grave f(k)k\eta\|^2
=\|\grave f(k)v\lambda(\xi)^*\eta\|^2
=\|v\grave f(h)\rho(\eta)S\xi\|^2\\
&=\|\rho(\eta)v\grave f(h)S\xi\|^2
\le\|\rho(\eta)\|^2\|\grave f(h)S\xi\|^2,\qquad\eta\in A',\ f\in C_c((0,\infty))^+,
\end{align*}
which imply that $c:=\frac12e^{\frac s2}\|\rho(\eta)\|$ satisfies
\[\|f(h)S\xi\|\le c\|\grave f(h)S\xi\|.\]

For arbitrary $\e>0$, by considering a net in $C_c((0,\infty))^+$ which increasingly converges to $1_{(c+\e,\infty)}$ and defining $p_\e:=1_{[0,c+\e]}(h)$, we have $\grave f\le(c+\e)^{-1}f$ and that
\[\|(1-p_\e)S\xi\|\le\frac c{c+\e}\|(1-p_\e)S\xi\|,\]
which implies $p_\e S\xi=S\xi$ for all $\e>0$, so $p_0S\xi=S\xi$.
Then,
\[\|\lambda(\xi)^*\zeta\|=\|p_0\lambda(\xi)^*\zeta\|=\|1_{[0,c]}(h)hv^*\zeta\|\le c\|v^*\zeta\|\le c\|\zeta\|,\qquad\zeta\in A'.\]
Therefore, $S\xi\in B$, which implies $S\xi\in A$ and $\xi\in A$.

For the density of $A\cap D'$, we approximate $\zeta\in\dom\Delta$.
Define a sequence $\eta_n\in A'$ such that $\eta_n\to(1+\Delta)\zeta$.
Then, since $(1+\Delta)^{-1}$ is bounded, we have $(1+\Delta)^{-1}\eta_n\to\zeta\in\bar{A\cap D'}$.
Since $\dom\Delta$ is dense in $H$, we are done.
\end{pf}


\begin{thm}[Tomita-Takesaki commutation theorem]
Let $A$ be a left Hilbert algebra.
Then, for every $t\in\R$, the following diagram commutes:
\[\begin{tikzcd}[column sep={3.5em,between origins},row sep={2.5em,between origins}]
&D\ar{rr}{u_t=\Delta^{it}}&&D\ar{dd}{\lambda}\\
A'\ar{ur}{J}\ar[swap]{dd}{\rho}&&&\\
&&&C(H).\\
\rho(A')\ar[swap]{rr}{\sigma_t=\Ad\Delta^{it}}&&B(H)\ar[swap]{ur}{\Ad J}&
\end{tikzcd}\]
\end{thm}
\begin{pf}
Fix $\eta\in A'$ and define
\begin{align*}
\xi:&=(e^{-\frac s2}u_{\frac i2}+e^{\frac s2}u_{-\frac i2})^{-1}J\eta
=(e^{-\frac s2}\Delta^{-\frac12}+e^{\frac s2}\Delta^{\frac12})^{-1}J\eta\\
&=(e^{-\frac s2}\Delta^{-\frac12}+e^{\frac s2}\Delta^{\frac12})^{-1}\Delta^{-\frac12}F\eta
=e^{-\frac s2}(e^{-s}+\Delta)^{-1}F\eta
\in A\cap D'.
\end{align*}
By the computations
\begin{align*}
\<\rho(F\xi)\zeta_1,\zeta_2\>
&=\<\lambda(\zeta_1)F\xi,\zeta_2\>
=\<F\xi,\lambda(\zeta_1)^*\zeta_2\>
=\<S\lambda(\zeta_1)^*\zeta_2,\xi\>\\
&=\<\lambda(\zeta_2)^*\zeta_1,\xi\>
=\<\rho(\zeta_1)S\zeta_2,\xi\>
=\<S\zeta_2,\rho(\zeta_1)^*\xi\>\\
&=\<S\zeta_2,\lambda(\xi)F\zeta_1\>
=\<F\lambda(\xi)F\zeta_1,\zeta_2\>,\qquad\qquad\zeta_1\in A\cap D',\ \zeta_2\in A,\\
\<\rho(S\xi)\zeta_1,\zeta_2\>
&=\<\lambda(\zeta_1)S\xi,\zeta_2\>
=\<S\xi,\lambda(\zeta_1)^*\zeta_2\>
=\<S\xi,\rho(\zeta_2)S\zeta_1\>\\
&=\<\rho(\zeta_2)^*S\xi,S\zeta_1\>
=\<\lambda(\xi)^*F\zeta_2,S\zeta_1\>
=\<F\zeta_2,\lambda(\xi)S\zeta_1\>\\
&=\<S\lambda(\xi)S\zeta_1,\zeta_2\>,\qquad\qquad\qquad\qquad\qquad\qquad\zeta_1\in A\cap D',\ \zeta_2\in D',
\end{align*}
the domains of $\rho(F\xi)$ and $\rho(S\xi)$ contain $A\cap D'$ and we have
\begin{align*}
\rho(\eta)
&=\rho(J(e^{-\frac s2}\Delta^{-\frac12}+e^{\frac s2}\Delta^{\frac12})\xi)\\
&=e^{-\frac s2}\rho(F\xi)+e^{\frac s2}\rho(S\xi)\\
&=e^{-\frac s2}F\lambda(\xi)F+e^{\frac s2}S\lambda(\xi)S\\
&=e^{-\frac s2}\Delta^{\frac12}J\lambda(\xi)J\Delta^{-\frac12}+e^{\frac s2}\Delta^{-\frac12}J\lambda(\xi)J\Delta^{\frac12}\\
&=(e^{-\frac s2}\sigma_{-\frac i2}+e^{\frac s2}\sigma_{\frac i2})(\Ad J)\lambda(\xi)
\end{align*}
as sesquilinear forms on $A\cap D'$.
The conditions for $\xi$ and $\zeta_1$ to belong to $A\cap D'$ are necessary in the above computation.
By the density of $A\cap D'$ in $H$, we have the bounded operators
\[(\Ad J)(e^{-\frac s2}\sigma_{-\frac i2}+e^{\frac s2}\sigma_{\frac i2})^{-1}\rho(\eta)=\lambda((e^{-\frac s2}u_{\frac i2}+e^{\frac s2}u_{-\frac i2})^{-1}J\eta).\]


Then, we get the equation of bounded linear operators
\[(\Ad J)\Bigl(\int\frac{e^{-ist}}{e^{\pi t}+e^{-\pi t}}\sigma_t(\rho(\eta))\,dt\Bigr)
=\lambda\Bigl(\int\frac{e^{ist}}{e^{\pi t}+e^{-\pi t}}u_t(J\eta)\,dt\Bigr),\qquad s\in\R,\ \eta\in A',\]
changing the variable using that the hyperbolic secant functions is even.
For every $\zeta\in B'$, since $\Ad J:B(H)\to B(H)$, $\cdot\zeta:B(H)\to H$, and $\rho(\zeta):H\to H$ are all continuous between weak$^*$ topologies, we have
\begin{align*}
\int\frac{e^{ist}}{e^{\pi t}+e^{-\pi t}}(\Ad J)\sigma_t(\rho(\eta))\zeta\,dt
&=(\Ad J)\Bigl(\int\frac{e^{-ist}}{e^{\pi t}+e^{-\pi t}}\sigma_t(\rho(\eta))\,dt\Bigr)\zeta,
\end{align*}
which is equal to
\begin{align*}
\lambda\Bigl(\int\frac{e^{ist}}{e^{\pi t}+e^{-\pi t}}u_t(J\eta)\,dt\Bigr)\zeta
&=\rho(\zeta)\int\frac{e^{ist}}{e^{\pi t}+e^{-\pi t}}u_t(J\eta)\,dt\\
&=\int\frac{e^{ist}}{e^{\pi t}+e^{-\pi t}}\rho(\zeta)u_t(J\eta)\,dt\\
&=\int\frac{e^{ist}}{e^{\pi t}+e^{-\pi t}}\lambda(u_t(J\eta))\zeta\,dt.
\end{align*}
Then, by taking arbitrary bounded linear functionals of $H$ on the above integral, and by the injectivity of the Fourier transform, we finally obtain $\Ad_J\circ\,\sigma_t\circ\rho=\lambda\circ u_t\circ J$ on $A'$.
\end{pf}

\begin{cor}
Let $A$ be a full left Hilbert algebra.
Then, for $t\in\R$, the following diagram is well-defined and commutes:
\[\begin{tikzcd}[column sep={3.5em,between origins},row sep={2.5em,between origins}]
&A\ar{rr}{\Delta^{it}}&&A\ar{dd}{\lambda}\\
A'\ar[<->]{ur}{J}\ar[swap]{rr}{\Delta^{it}}\ar[swap]{dd}{\rho}&&A'\ar[<->,swap]{ur}{J}\ar{dd}{\rho}&\\
&&&\lambda(A)\\
\rho(A')\ar[swap]{rr}{\Ad_{\Delta^{it}}}&&\rho(A')\ar[<->,swap]{ur}{\Ad_J}&.
\end{tikzcd}\]
In particular, we have
\[JA=A',\qquad\Delta^{it}A=A,\qquad JMJ=M',\qquad\Delta^{it}M\Delta^{-it}=M,\]
and $J$ is an anti-homomorphism, $\Delta^{it}$ is a $*$-homomorphism
\end{cor}
\begin{pf}

\end{pf}



\begin{ex}[Flow on a von Neumann algebra]
\[R_n(\xi):=\sqrt{\frac n\pi}\int e^{-ns^2}u_s(\xi)\,ds,\qquad R_n(x):=\sqrt{\frac n\pi}\int e^{-ns^2}\sigma_s(x)\,ds.\]

$\Delta^{\frac12}R_n(\xi)=R_n(\Delta^{\frac12}\xi)$ since $s\mapsto\Delta^{\frac12+is}\xi$ is bounded and continuous weakly.

For $\xi\in A$, by the Tomita-Takesaki commutation, we have $\lambda(u_s\xi)=\sigma_s\lambda(\xi)$, and $R_n(\lambda(\xi))=\lambda(R_n(\xi))$ since $s\mapsto\lambda(u_s(\xi))=\sigma_s(\lambda(\xi))$ is bounded and continuous $\sigma$-weakly.


\[R_n(x):=\sqrt{\frac n\pi}\int e^{-ns^2}\sigma_s(x)\,ds.\]

A \emph{Tomita algebra} is a left Hilbert algebra $A_0$ such that every element of $A_0$ is entire with respect to the associated modular automorphism group.
\end{ex}








\newpage
\section{January 17}

\subsection{Cocycle conjugacy}



\begin{defn}[Cocycle of an action]
Let $\alpha:G\to\Aut(M)$ be an action on a von Neumann algebra $M$.
The unitary group $U(M)$ is a topological $G$-group with the strong operator topology, on which the action is given by $\alpha$.
A \emph{continuous 1-cocycle} of the $G$-group $(U(M),\alpha)$, or just simply an \emph{$\alpha$-cocycle}, is a continuous map $v:G\to U(M)$ satisfying the cocycle condition $v_{st}=v_s\alpha_s(v_t)$, and we denote by $Z_\alpha^1(G,U(M))$ the set of all $\alpha$-cocycles.
The \emph{cocycle perturbation} of a $\alpha$-cocycle $u$ of $\alpha$ is another action $\alpha^v:\R\to\Aut(M)$ defined by $\alpha_t^v:=(\Ad v_t)\circ\alpha_t$ for $t\in\R$.
The cocycle condition is the equivalent relation for $\alpha^v$ to be a group homomorphism.
It is not hard to see that $v^*$ is an $\alpha^v$-cocycle if $v$ is an $\alpha$-cocycle.
\end{defn}



\begin{thm}[Connes cocycle derivative]
Let $\f$ and $\psi$ be faithful semi-finite normal weights on a von Neumann algebra $M$.
\begin{parts}
\item The representations $\pi_\f$ and $\pi_\psi$ are unitarily equivalent.
In particular, every normal state is a vector state in a semi-cyclic reprsentation of a faithful semi-finite normal weight.
\item The modular automorphism groups $\sigma_t^\f$ and $\sigma_t^\psi$ are cocycle conjugate.
In particular, there is a canonical continuous group homomorphism $\sigma:\R\to\operatorname{Out}(M)$ for $M$.
\end{parts}
\end{thm}
\begin{pf}
\AtBeginEnvironment{bmatrix}{\setlength{\arraycolsep}{2pt}}
Consider the \emph{balanced weight} $\f\oplus\psi$ on $M\otimes M_2(\C)=M_2(M)$.
We can easily prove it is faithful semi-finite and normal.
We investigate the semi-cyclic representation and the left Hilbert algebra structure corresponding to $\f\oplus\psi$.
First, we have
\[\fn_{\f\oplus\psi}=\mat{\fn_\f&\fn_\psi\\\fn_\f&\fn_\psi},\qquad
\fa_{\f\oplus\psi}=\mat{\fa_\f&\fn_\f^*\cap\fn_\psi\\\fn_\psi^*\cap\fn_\f&\fa_\psi},\qquad
\fm_{\f\oplus\psi}=\mat{\fm_\f&\fn_\f^*\fn_\psi\\\fn_\psi^*\fn_\f&\fm_\psi}.
\]
The semi-cyclic representation of $\f\oplus\psi$ can be realized on the identification with the direct sum
\[H_{\f\oplus\psi}=H_\f\oplus H_\f\oplus H_\psi\oplus H_\psi\]
such that $\Lambda_{\f\oplus\psi}:\fn_{\f\oplus\psi}\to H_{\f\oplus\psi}$ and $\pi_{\f\oplus\psi}:M_2(M)\to B(H_{\f\oplus\psi})$ given by
\[\Lambda_{\f\oplus\psi}\mat{x_{11}&x_{12}\\x_{21}&x_{22}}=\mat[b]{\Lambda_\f(x_{11})\\\Lambda_\f(x_{21})\\\Lambda_\psi(x_{12})\\\Lambda_\psi(x_{22})},\quad
\pi_{\f\oplus\psi}\mat{x_{11}&x_{12}\\x_{21}&x_{22}}=\mat[b]{
\pi_\f(x_{11})&\pi_\f(x_{12})&0&0\\
\pi_\f(x_{21})&\pi_\f(x_{22})&0&0\\
0&0&\pi_\psi(x_{11})&\pi_\psi(x_{12})\\
0&0&\pi_\psi(x_{21})&\pi_\psi(x_{22})}.\]
The Hilbert algebra structure $S_{\f\oplus\psi},\ \Delta_{\f\oplus\psi},\ J_{\f\oplus\psi}:A_{\f\oplus\psi}\to H_{\f\oplus\psi}$ on $A_{\f\oplus\psi}=\Lambda_{\f\oplus\psi}(\fa_{\f\oplus\psi})$ are computed as
\[S_{\f\oplus\psi}=\mat[b]{
S_\f&0&0&0\\
0&0&S_{\f,\psi}&0\\
0&S_{\psi,\f}&0&0\\
0&0&0&S_\psi},\quad
J_{\f\oplus\psi}=\mat[b]{
J_\f&0&0&0\\
0&0&J_{\f,\psi}&0\\
0&J_{\psi,\f}&0&0\\
0&0&0&J_\psi},\quad
\Delta_{\f\oplus\psi}=\mat[b]{
\Delta_\f&0&0&0\\
0&\Delta_{\f,\f}&0&0\\
0&0&\Delta_{\psi,\psi}&0\\
0&0&0&\Delta_\psi}.\]
We have the relations
\[S_\f=J_\f\Delta_\f^{\frac12}=\Delta_\f^{-\frac12}J_\f,\qquad S_{\f,\psi}=J_{\f,\psi}\Delta_{\psi,\psi}^{\frac12}=\Delta_{\f,\f}^{-\frac12}J_{\f,\psi}.\]
The self-adjoint operator $\Delta_{\psi,\psi}=|S_{\f,\psi}|$ is called the \emph{spatial derivative} of $\f$ with respect to the opposite weight $\psi'$ on $M'$, and it is frequently denoted by $d\f/d\psi'$.



(a)
Since
\[\pi_{\f\oplus\psi}\mat{x&0\\0&x}=\mat[b]{
\pi_\f(x)&0&0&0\\
0&\pi_\f(x)&0&0\\
0&0&\pi_\psi(x)&0\\
0&0&0&\pi_\psi(x)},\quad
J\pi_{\f\oplus\psi}\mat{0&1\\0&0}J=\mat[b]{
\ \ 0\ \ &\ \ 0\ \ &J_\f J_{\f,\psi}&0\\
0&0&0&J_{\f,\psi}J_\psi\\
0&0&0&0\\
0&0&0&0}\]
commute by the Tomita-Takesaki commutation theorem, we obtain
\[\mat[b]{\pi_\f(x)J_\f J_{\f,\psi}&0\\0&\pi_\f(x)J_{\f,\psi}J_\psi}=\mat[b]{J_\f J_{\f,\psi}\pi_\psi(x)&0\\0&J_{\f,\psi}J_\psi\pi_\psi(x)},\]
so if we define $u_{\f,\psi}:=J_\f J_{\f,\psi}=J_{\f,\psi}J_\psi:H_\psi\to H_\f$, then it is unitary such that $\pi_\f(x)=u_{\f,\psi}\pi_\psi(x)u_{\f,\psi}^*$ for all $x\in M$.
Be cautious that they are unitarily equivalent as representations, not as semi-cyclic representations, for $\Lambda_\psi(x)\ne u\Lambda_\f(x)$ unless $\f=\psi$ in general.

(b)
Define $\sigma_t^{\f,\psi}$ and $\sigma_t^{\psi,\f}$ such that
\[\sigma_t^{\f\oplus\psi}\mat{x_{11}&x_{12}\\x_{21}&x_{22}}=:\mat{\sigma_t^\f(x_{11})&\sigma_t^{\f,\psi}(x_{12})\\\sigma_t^{\psi,\f}(x_{21})&\sigma_t^\psi(x_{22})}.\]
From the relation $\pi_{\f\oplus\psi}(\sigma_t^{\f\oplus\psi}([x_{ij}]))=\Delta_{\f\oplus\psi}^{it}\pi_{\f\oplus\psi}([x_{ij}])\Delta_{\f\oplus\psi}^{-it}$, we can deduce
\[\pi_\f(\sigma_t^{\f,\psi}(x))=\Delta_\f^{it}\pi_\f(x)\Delta_{\f,\f}^{-it},\qquad\pi_\psi(\sigma_t^{\f,\psi}(x))=\Delta_{\psi,\psi}^{it}\pi_\psi(x)\Delta_\psi^{-it}.\]
Define a map $v:\R\to U(M)$ such that $v_t:=\sigma_t^{\f,\psi}(1)$ for $t\in\R$, which is well-defined and continuous.
Then,
\[\sigma_t^\f(x)=\sigma_t^{\f\oplus\psi}\mat{x&0\\0&0}=\sigma_t^{\f\oplus\psi}\left(\mat{0&1\\0&0}\mat{0&0\\0&x}\mat{0&0\\1&0}\right)=v_t\sigma_t^\psi(x)v_t^*.\]
Since $\sigma^\f$ and $\sigma^\psi$ are group homomorphisms, $v$ satisfies the cocycle condition, and $\sigma^\f$ and $\sigma^\psi$ are cocycle conjugate by $v$.
The $\sigma^\psi$-cocycle $v$ is called the \emph{cocycle derivative} of $\f$ with respect to $\psi$, introduced by Connes, and denoted by $(D\f:D\psi)$.
\end{pf}

\begin{prop}[Converse of cocycle derivative theorem]
Let $\psi$ be a faithful semi-finite normal weight on a von Neumann algebra $M$.
For a $\sigma^\psi$-cocycle $v:\R\to U(M)$, there is a faithful semi-finite normal weight $\f$ such that $v=(D\f:D\psi)$.
\end{prop}
\begin{pf}
Difficult.
Omitted.
\end{pf}


\subsection{Kubo-Martin-Schwinger weights}


\begin{defn}[Kubo-Martin-Schwinger weights]
Let $M$ be a von Neumann algebra.
Let $\alpha$ be a flow on $M$, and $\f$ be a faithful semi-finite normal weight on $M$.
For $x,y\in M$, their \emph{two-point function} at inverse temperature $\beta\in\R$ is a bounded continuous function $f:\Im^{-1}([\beta,0]\cup[0,\beta])\to\C$ which is holomorphic on its interior such that
\[f(t)=\f(y\alpha_t(x)),\qquad f(t+i\beta)=\f(\alpha_t(x)y),\qquad t\in\R.\]
If $\f$ is invariant under $\alpha$ and every pair $x,y\in\fa$ admits a two-point function at $\beta$, then we say $\f$ is a \emph{Kubo-Martin-Schwinger weight} or \emph{KMS weight} for $\alpha$ at $\beta$.
From now on, we always assume $\beta=-1$.
\end{defn}

\begin{rmk*}
If $\f$ is a state, then for $\f$ to be a KMS state the invariance condition is superfluous: since $\fm=M$, we can put $y=1$ to show the invariance using the Liouville theorem and the Schwarz reflection principle.
\end{rmk*}


\begin{lem}[Existence of two-point functions]
Let $M$ be a von Neumann algebra and $x,y\in M$.
For $\sigma$ the associated modular automorphism group to a faithful semi-finite normal weight $\f$ on $M$, the followings hold.
\begin{parts}
\item If $x,y\in\fa$, then they admit a two-point function.
\item If $x$ is a multiplier of $\fm$ and $y\in\fa_0^*\fa_0$, then they admit an entire two-point function.
\item If $x$ is entire for $\sigma$ and $y\in\fm$, then they admit an entire two-point function.
\end{parts}
Here, we say $x$ is a \emph{multiplier} of the subalgebra $\fm$ of $M$ if $x\fm\cup\fm x\subset\fm$.
\end{lem}
\begin{pf}
We may assume $x,y\ge0$.
In this case, $\Delta^{\frac12}\Lambda(y)=JS\Lambda(y)=J\Lambda(y)$ for $y\in\fn^+$.
Note also that we can show the inner product $z\mapsto\<\xi(z),\eta(z)\>$ of weakly holomorphic and weakly anti-holomorphic functions $\xi,\eta:\Omega\subset\C\to H$ is holomorphic with the derivative $z\mapsto\<\partial\xi(z),\eta(z)\>+\<\xi(z),\bar\partial\eta(z)\>$ by the direct estimate of the difference quotient, using the strong boundedness of the weakly holomorphic functions.

(a)
Define
\[f(z):=\<\Delta^{i\frac z2}\Lambda(x),\Delta^{-i\frac{\bar z}2}\Lambda(y)\>.\]
Since $\Lambda(x),\Lambda(y)\in A\subset\dom\Delta^{\frac12}$, the function $f$ is bounded and continuous on $\Im^{-1}([-1,0])$, and holomorphic on its interior.
Also, for $t\in\R$
\begin{align*}
\f(y\sigma_t(x))
&=\<\Delta^{it}\Lambda(x),\Lambda(y)\>
=\<\Delta^{i\frac t2}\Lambda(x),\Delta^{-i\frac t2}\Lambda(y)\>
=f(t),\\
\f(\sigma_t(x)y)
&=\<\Lambda(y),\Delta^{it}\Lambda(x)\>
=\<\Lambda(y),J\Delta^{it}J\Lambda(x)\>
=\<\Delta^{it}J\Lambda(x),J\Lambda(y)\>\\
&=\<\Delta^{it}\Delta^{\frac12}\Lambda(x),\Delta^{\frac12}\Lambda(y)\>
=\<\Delta^{i\frac{t-i}2}\Lambda(x),\Delta^{-i\frac{t+i}2}\Lambda(y)\>
=f(t-i).
\end{align*}

(b)
We may assume $y^{\frac12}\in\fa_0$ by the linear span.
Define
\[f(z):=\<x\Delta^{-i(z+i)}\Lambda(y^{\frac12}),\Delta^{-i\bar z}\Lambda(y^{\frac12})\>.\]
Since $\Lambda(y^{\frac12})\in A_0\subset\dom\Delta$, the function $f$ is bounded and continuous on $\Im^{-1}([-1,0])$, and holomorphic on its interior, and in fact it is entire.
Then, for $t\in\R$,
\begin{align*}
\f(y\sigma_t(x))
&=\<\Lambda(y^{\frac12}\sigma_t(x)),\Lambda(y^{\frac12})\>
=\<J\Lambda(y^{\frac12}),J\Lambda(y^{\frac12}\sigma_t(x))\>\\
&=\<\Delta^{\frac12}\Lambda(y^{\frac12}),\Delta^{\frac12}\Lambda(\sigma_t(x)y^{\frac12})\>
=\<\Delta\Lambda(y^{\frac12}),\sigma_t(x)\Lambda(y^{\frac12})\>\\
&=\<\Delta\Lambda(y^{\frac12}),\Delta^{it}x\Delta^{-it}\Lambda(y^{\frac12})\>
=\<x\Delta^{-it+1}\Lambda(y^{\frac12}),\Delta^{-it}\Lambda(y^{\frac12})\>
=f(t),\\
\f(\sigma_t(x)y)
&=\<\Lambda(y^{\frac12}),\Lambda(y^{\frac12}\sigma_t(x))\>
=\<J\Lambda(y^{\frac12}\sigma_t(x)),J\Lambda(y^{\frac12})\>\\
&=\<\Delta^{\frac12}\Lambda(\sigma_t(x)y^{\frac12}),\Delta^{\frac12}\Lambda(y^{\frac12})\>
=\<\sigma_t(x)\Lambda(y^{\frac12}),\Delta\Lambda(y^{\frac12})\>\\
&=\<\Delta^{it}x\Delta^{-it}\Lambda(y^{\frac12}),\Delta\Lambda(y^{\frac12})\>
=\<x\Delta^{-it}\Lambda(y^{\frac12}),\Delta^{it+1}\Lambda(y^{\frac12})\>
=f(t-i).
\end{align*}

(c)
To see $y\sigma_t(x)$ and $\sigma_t(x)y$ belong to $\fm$, since $\sigma_t$ sends an entire element to an entire element, we need to check an entire elment is a multiplier of $\fm$.
As a first step, we claim that $x$ is a multiplier of $\fa$, that is, $x\fa\cup\fa x\subset\fa$.
By symmetry, it suffices to show $xy^{\frac12}\in\fa$.
Since $\fn$ is a left ideal of $M$, $xy^{\frac12}\in\fn$, which implies $\Lambda(xy^{\frac12})\in B$.
Consider
\[\xi(t):=\Delta^{it}\Lambda(xy^{\frac12})=\Delta^{it}x\Lambda(y^{\frac12})=\sigma_t(x)\Delta^{it}\Lambda(y^{\frac12}).\]
Since $\Lambda(y^{\frac12})\in D=\dom\Delta^{\frac12}$, the function $f$ is holomorphically extended to the strip $\Im^{-1}([-\frac12,0])$.
It means that $\xi(0)=\Lambda(xy^{\frac12})\in\dom\Delta^{\frac12}$, so $\Lambda(xy^{\frac12})\in D$.
Thus we have $\Lambda(xy^{\frac12})\in A$ and $xy^{\frac12}\in\fa$.
For the original claim $x\fm\cup\fm x\subset\fm$, since $y^{\frac12}\in\fa$ and $xy^{\frac12}\in\fa$ as above, we have $xy=(xy^{\frac12})y^{\frac12}\in\fa^2=\fm$.
The linear span and symmetry show $x\fm\cup\fm x\subset\fm$.

Define
\[f(z):=\<\sigma_{z+\frac i2}(x)\Delta^{\frac12}\Lambda(y^{\frac12}),\Delta^{\frac12}\Lambda(y^{\frac12})\>.\]
Since $x$ is entire, the function $f$ is entire and bounded on the strip $\Im^{-1}([-1,0])$.
Then, for $t\in\R$,
\begin{align*}
\f(y\sigma_t(x))
&=\<\Delta^{\frac12}\Lambda(y^{\frac12}),\Delta^{\frac12}\sigma_t(x)\Lambda(y^\frac12)\>
=\<\Delta^{\frac12}\Lambda(y^{\frac12}),\sigma_{t-\frac i2}(x)\Delta^{\frac12}\Lambda(y^\frac12)\>\\
&=\<\sigma_{t+\frac i2}(x)\Delta^{\frac12}\Lambda(y^{\frac12}),\Delta^{\frac12}\Lambda(y^\frac12)\>=f(t),\\
\f(\sigma_t(x)y)
&=\<\Delta^{\frac12}\sigma_t(x)\Lambda(y^{\frac12}),\Delta^{\frac12}\Lambda(y^\frac12)\>
=\<\sigma_{t-\frac i2}\Delta^{\frac12}\Lambda(y^{\frac12}),\Delta^{\frac12}\Lambda(y^\frac12)\>=f(t-i).\qedhere
\end{align*}
\end{pf}

\begin{prop}[Characterization of modular automorphism group]
Let $\f$ be a faithful semi-finite normal weight on a von Neumann algebra $M$.
An action $\alpha:\R\to\Aut(M)$ is the modular automorphism group of $\f$ if and only if $\f$ is a KMS weight for $\alpha$.
\end{prop}
\begin{pf}
($\Rightarrow$)
Let $\alpha=\sigma$, where $\sigma=\Ad\Delta^{it}$ is the modular automorphism group.
By the Tomita-Takesaki commutation theorem, we have $\sigma_t(\fa)=\fa$ and $\Lambda(\sigma_t(x))=\Delta^{it}\Lambda(x)$ for all $t\in\R$.
Then, the invariance follows as
\begin{align*}
\f(\sigma_t(y^*x))&=\f(\sigma_t(y)^*\sigma_t(x))=\<\Lambda(\sigma_t(x)),\Lambda(\sigma_t(y))\>\\
&=\<\Delta^{it}\Lambda(x),\Delta^{it}\Lambda(y)\>=\<\Lambda(x),\Lambda(y)\>=\f(y^*x),\qquad x,y\in\fn.
\end{align*}
The part (a) in the previous lemma proves that the existence of the two-point function, so $\f$ is a KMS weight.

($\Leftarrow$)
Suppose $\f$ satisfies the KMS condition for $\alpha$.
Since $\f$ is invariant under $\alpha$, for each $t\in\R$ the map $B\to B:\Lambda(x)\mapsto\Lambda(\alpha_t(x))$ where $x\in\fn$ is unitarily extended to define a stronlgy continuous one-parameter group $u:\R\to U(H)$ such that $\Lambda(\alpha_t(x))=u_t\Lambda(x)$.
Using the Stone theorem, take the self-adjoint operator $h$ on $H$ such that $u_t=h^{it}$.
We claim that $h=\Delta$.

Since $\alpha$ preserves $*$-structure of $M$, we have $u_tS\xi=Su_t\xi$ for $\xi\in A$.
Because $A$ is dense in $D$ in the sense that every $\xi\in D$ has a sequence $\xi_n\in A$ such that $\xi_n\to\xi$ and $\Delta^{\frac12}\xi_n\to\Delta^{\frac12}\xi$, we have $u_tS\xi=Su_t\xi$ for $\xi\in D$.
Then, we have $u_t:D\to D$ since
\[\<u_t\xi,\Delta^{\frac12}\eta\>=\<J\eta,Su_t\xi\>=\<J\eta,u_tS\xi\>=\<Ju_tS\xi,\eta\>,\qquad\xi\in D,\ \eta\in A_0,\]
and $u_t:\dom\Delta\to\dom\Delta$ since $S\dom\Delta\subset\dom\Delta^{\frac12}=D$ and
\[\<u_t\xi,\Delta\eta\>=\<F\Delta\eta,Su_t\xi\>=\<S\eta,u_tS\xi\>=\<Su_tS\xi,\eta\>,\qquad\xi\in\dom\Delta,\ \eta\in A_0.\]
Also, $\Delta$ and $u_t$ commute by
\[\<u_t\Delta\xi,\eta\>=\<Su_t^*\eta,S\xi\>=\<u_t^*S\eta,S\xi\>=\<S\eta,Su_t\xi\>=\<\Delta u_t\xi,\eta\>,\qquad\xi\in\dom\Delta,\ \eta\in A_0.\]

*** not yet done...



The proof becomes much easier if we assume $\f$ is a state.
\end{pf}

\subsection{Centralizers and commuting weights}

\begin{prop}[Centralizers]
Let $\f$ be a faithful semi-finite normal weight on a von Neumann algebra $M$.
For $x\in M$, the followings are all equivalent:
\begin{parts}
\item $\sigma_t(x)=x$ for all $t\in\R$,
\item $x$ is a multiplier of $\fm$ and $\f(xy)=\f(yx)$ for all $y\in\fm$.
\end{parts}
The set of all $x$ satisfying one of the above conditions is called the \emph{centralizer} or the \emph{fixed point algebra} of $\f$, and denoted by $M^\f$.
\end{prop}
\begin{pf}
(a)$\Rightarrow$(b)
Let $\sigma_t(x)=x$ for all $t\in\R$ and $y\in\fm$, then since the constant function is entire, we have $x$ is entire.
By (c) of the previous lemma, observing the two-point function $f$ is constant on the real line so that it is entirely constant by the identity principle, we can check that the KMS condition gives
\[\f(yx)=\f(y\sigma_t(x))=f(t)=f(t-i)=\f(\sigma_t(x)y)=\f(xy).\]

(b)$\Rightarrow$(a)
Let $x\fm\cup\fm x\subset\fm$ and $y\in(\fa_0^*\fa_0)^+$.
Then, $\sigma_{-t}(y)\in\fa_0^*\fa_0$ for all $t\in\R$.
By (b) of the previous lemma, we have a two-point function such that
\[f(t)=\f(y\sigma_t(x))=\f(\sigma_{-t}(y)x)=\f(x\sigma_{-t}(y))=\f(\sigma_t(x)y)=f(t-i),\qquad t\in\R.\]
By the Liouville theorem $f$ is constant, so we have
\[0=\f(y\sigma_t(x))-\f(yx)=\<(\sigma_t(x)-x)\Lambda(y^{\frac12}),\Lambda(y^{\frac12})\>.\]
Since $\Lambda(\fa_0^*\fa_0)=A_0^2$ is dense in $H$, we have $\sigma_t(x)=x$.
\end{pf}





\begin{prop}
Let $\f$ be a faithful semi-finite normal weight on a von Neumann algebra $M$.
Let $h$ be a non-negative self-adjoint operator affiliated with the centralizer $M^\f$.
Then,
\[\f_h(x):=\lim_{\e\to0}\f(h_\e^{\frac12}xh_\e^{\frac12}),\qquad x\in M^+\]
is a semi-finite normal weight, where $h_\e=h(1+\e h)^{-1}$.
The weight $\f$ is faithful if and only if $h$ is non-singular.
\end{prop}
\begin{pf}
We first claim that $\f_{h+k}(x)=\f_h(x)+\f_k(x)$ for $x\in M^+$ if $h$ and $k$ are bounded.
Take $u$ and $v$ in $M^\f$ such that
\[u^*u+v^*v=s(h+k),\qquad h^{\frac12}=u(h+k)^{\frac12},\qquad k^{\frac12}=v(h+k)^{\frac12}.\]
Let $x\in M^+$ and define $y:=(h+k)^{\frac12}x(h+k)^{\frac12}$.
If $x$ satisfies $\f_{h+k}(x)<\infty$ so that $y\in\fm$ and $uy,vy\in\fm$, then we have
\begin{align*}
\f_h(x)+\f_k(x)
&=\f(h^{\frac12}xh^{\frac12}+k^{\frac12}xk^{\frac12})
=\f(uyu^*+vyv^*)\\
&=\f(u^*uy+v^*vy)=\f(s(h+k)y)=\f(y)=\f_{h+k}(x).
\end{align*}
If $x$ satisfies $\f_h(x)+\f_k(x)<\infty$ so that $uu^*h^{\frac12}xh^{\frac12},vv^*k^{\frac12}xk^{\frac12}\in\fm$, then from the inequality
\[|t^{\frac12}-t(t+\e)^{-\frac12}|\le|\e t^{\frac12}(t+\e)|\le\frac{\e^{\frac12}}2,\qquad t\ge0,\ \e>0,\]
we have
\begin{align*}
\f(y)
&=\lim_{\e\to0}\f((h+k+\e)^{-\frac12}(h+k)x(h+k)(h+k+\e)^{-\frac12})\\
&\le2\lim_{\e\to0}\f((h+k+\e)^{-\frac12}(hxh+kxk)(h+k+\e)^{-\frac12})\\
&=2\lim_{\e\to0}\f((h+k+\e)^{-\frac12}(h+k)(u^*uxu^*u+v^*vxv^*v)(h+k)(h+k+\e)^{-\frac12})\\
&=2\f((h+k)^{\frac12}(u^*uxu^*u+v^*vxv^*v)(h+k)^{\frac12})\\
&=2\f(u^*h^{\frac12}xh^{\frac12}u+v^*k^{\frac12}xk^{\frac12}v)
=2\f(uu^*h^{\frac12}xh^{\frac12}+vv^*k^{\frac12}xk^{\frac12})<\infty.
\end{align*}
It implies $y\in\fm$, we can reduce it to the case $\f_{h+k}(x)<\infty$, so the claim follows.

In particular, the additivity of $h\mapsto\f_h$ proves the convergence of $\f(h_\e^{\frac12}xh_\e^{\frac12})$ as $\e\to0$.
The normality follows easily from the normality of each $\f_{h_\e}$.
For the semi-finiteness, if we let $e_n$ be the spectral projection of $h$ corresponding to the interval $[0,n]$, then $\bigcup_{n=1}^\infty e_n\fm_\f e_n$ is a subset of $\fm_{\f_h}$ which is $\sigma$-weakly dense in $M$.
\end{pf}




\begin{prop}[Commuting weights]
Let $\f$ and $\psi$ be faithful semi-finite normal weights on a von Neumann algebra $M$.
TFAE
\begin{parts}
\item $\psi=\f_h$ for a positive non-singular self-adjoint operator $h$ affiliated with $M^\f$.
\item $\psi=\psi\circ\sigma_t^\f$.
\end{parts}
How about strong commutation of modular automorphism groups?
\end{prop}
\begin{pf}
(a)$\Rightarrow$(b)
Let $h$ be a positive non-singular self-adjoint operator affiliated with $M^\f$.
We calim that the modular automorphism group of $\psi:=\f_h$ is given by
\[\sigma_t^{\psi}(x)=h^{it}\sigma_t^\f(x)h^{-it}.\]
Since $h^{it}\in M^\f$, the invariance will follow.

First suppose $h\in M^{\f+}$ invertible.
Since $h$ is entire, $\fm_\f=\fm_\psi$.
It is enough to show that $t\mapsto h^{it}\sigma_t^\f(x)h^{-it}$ satisfies the KMS condition.
We will construct $f$.
Take a sequence 

For the general unbounded case


(b)$\Rightarrow$(a)
\[\sigma_s^\f((D\psi:D\f)_t)=(D\psi\circ\sigma_{-s}^\f:D\f\circ\sigma_{-s}^\f)_t=(D\psi:D\f)_t.\]
Stone's theorem, construct $h$.

\end{pf}

\begin{prop}[Spatial derivatives]
\end{prop}


\begin{thm}[Semi-finiteness]
Existence of trace vs inner modular automorphism
\end{thm}
\begin{pf}

\end{pf}



\iffalse

VIII.3.5.
\begin{thm*}[KMS characterization of modular automorphism group]
Let $\f$ be a faithful semi-finite normal weight on a von Neumann algebra $M$.
Suppose $x:\R\to\fa$ is a $\sigma$-wealy continuous map that satisfies that
\begin{enumerate}[(i)]
\item $\sup_{t\in\R}\f(x_t^*x_t+x_tx_t^*)<\infty$,
\item for each $y\in\fa$ there is a bounded continuous function $f:\Im^{-1}([-1,0])\to\C$ holomorphic on its interior such that
\[f_{x,y}(t)=\f(yx_t),\qquad f_{x,y}(t-i)=\f(x_ty).\]
Then, $x_t=\sigma_t(x_0)$.
\end{enumerate}
\end{thm*}
\begin{pf}
For each $y\in\fa$, if we let $\eta:=\Lambda(y)$ and $\xi_t:=\Lambda(x_t)$, then there is $F_\eta\in\cA(\Im^{-1}([-1,0]))$ such that
\[F_\eta(t)=\f(yx_t)=\<\Delta^{\frac12}\eta,J\xi_t\>,\qquad F_\eta(t-i)=\f(x_ty)=\<\eta,J\Delta^{\frac12}\xi_t\>.\]


The Phragmen-Lindel\"of theorem, 

$D$ is a Hilbert space since $\Delta^{\frac12}$ is closed.
By the Riesz representation theorem, there is $\zeta_z\in D$ such that
\[F_\eta(z)=\<\eta,\zeta_z\>_D=\<\eta,\zeta_z\>+\<\Delta^{\frac12}\eta,\Delta^{\frac12}\zeta_z\>,\qquad\eta\in D.\]
Then $\zeta_z$ is anti-holomorphic(?)
Let $t\in\R$.
Then,
\begin{align*}
\<\eta,\zeta_t\>_D&=F_\eta(t)=\<\Delta^{\frac12}\eta,J\xi_t\>=\<\eta,(\Delta^{-\frac12}+\Delta^{\frac12})^{-1}J\xi_t\>_D,\\
\<\eta,\zeta_{t-i}\>_D&=F_\eta(t-i)=\<\eta,J\Delta^{\frac12}\xi_t\>=\<\eta,(1+\Delta)^{-1}J\Delta^{\frac12}\xi_t\>_D,
\end{align*}
so we get by the density of $A$ in $D$ that
\[\zeta_t=(\Delta^{-\frac12}+\Delta^{\frac12})^{-1}J\xi_t,\qquad\zeta_{t-i}=(1+\Delta)^{-1}J\Delta^{\frac12}\xi_t,\]
and in particular we have $\Delta\zeta_{t-i}=\zeta_t$.
Define
\[G_\eta(z):=\<\Delta^{i\frac z2}\eta,\Delta^{-i\frac{\bar z}2}\zeta_z\>,\]
which is holomorphic in the open strip $\Im^{-1}((-1,0))$
\[G_\eta(t-i)=\<\Delta^{it}\eta,\Delta\zeta_{t-i}\>=\<\Delta^{it}\eta,\zeta_t\>=G_\eta(t),\qquad t\in\R.\]
Using the Morera and Fubini, $G_\eta$ can be extended to a bounded entire function, and the Liouville theorem is applied to deduce $G_\eta(z)$ is a constant function.


\end{pf}
\fi











\newpage
\section{March 8}


\subsection{Standard form}

\begin{defn}[Standard form]
Let $M$ be a von Neumann algebra.
A \emph{standard form} of $M$ is a faithful unital normal representation $\pi:M\to B(H)$ together with an anti-linear isometric involution $J:H\to H$ and a self-dual cone $P\subset H$ such that
\begin{enumerate}[(i)]
\item $J\pi(M)J=\pi(M)'$,
\item $J\pi(z)J=\pi(z^*)$ for $z\in Z(M)$,
\item $J\xi=\xi$ for $\xi\in P$,
\item $\pi(x)J\pi(x)JP\in P$ for $x\in M$.
\end{enumerate}
It is customary to let $M\subset B(H)$ and omit to write the symbol $\pi$, and in this case we say $M$ is represented in a standard form $(M,H,J,P)$.
\end{defn}

\begin{rmk*}
Let $P$ be a subset of a complex vector space $H$.
The \emph{dual cone} of $P$ is the set $P^\circ:=\{\xi\in H:\<\xi,\eta\>\ge0,\ \xi\in P\}$.
Note that $P^\circ$ is a closed convex cone.
We say a set $P$ is a \emph{self-dual cone} if $P=P^\circ$.
\end{rmk*}

\begin{prop}[Existence]
Let $A$ be a left Hilbert algebra.
Then, $(M,H,J,P)$ is a standard form, where
\[P:=\bar{\{\xi J\xi:\xi\in A_0\}}\]
and $A_0$ is the maximal Tomita algebra.
\end{prop}
\begin{pf}
We first show $P$ is a self-dual cone.
Define
\[P_0^\flat:=\{\eta\eta^\flat:\eta\in A_0\},\qquad P_0:=\{\zeta J\zeta:\zeta\in A_0:\zeta\in A_0\},\qquad P_0^\sharp:=\{\xi\xi^\sharp:\xi\in A_0\},\]
and $P^\flat:=\bar{P_0^\flat}$, $P:=\bar{P_0}$, $P^\sharp:=\bar{P_0^\sharp}$.

We claim $P^\flat$ and $P^\sharp$ are dual.
We have $P^\sharp\subset(P^\flat)^\circ$ since
\[\<\xi\xi^\sharp,\eta\eta^\flat\>=\<\lambda(\xi)\xi^\sharp,\rho(\eta^\flat)\eta\>=\<\rho(\eta)\xi^\sharp,\lambda(\xi^\sharp)\eta\>=\|\xi^\sharp\eta\|^2\ge0,\qquad\xi,\eta\in A_0.\]
Now suppose $\xi\in(P^\flat)^\circ$.
Because $\<\xi,\eta\eta^\flat\>\ge0$ is real, we have
\[\<S\xi,\eta\eta^\flat\>=\<F(\eta\eta^\flat),\xi\>=\<\eta\eta^\flat,\xi\>=\<\xi,\eta\eta^\flat\>,\qquad\eta\in A_0,\]
and $A_0^2$ is dense in $H$, so $\xi\in D=\dom\Delta$ and $\xi=S\xi$.
Then, $\lambda(\xi)$ is a non-negative densely defined symmetric operator since
\[\<\lambda(\xi)\eta,\eta\>=\<\rho(\eta)\xi,\eta\>=\<\xi,\eta\eta^\flat\>\ge0,\qquad\eta\in A_0.\]
By the Friedrichs extension(a non-negative densely defined symmetric operator admits a canonical non-negative self-adjoint extension), we can take a positive self-adjoint extension $x$ affiliated with $M$.
If we let $e_n=1_{[n^{-1},n]}(x)$, then since
\[\lambda(e_n\xi)=e_n\lambda(e_n\xi)=\lambda(e_n\xi)e_n=\lambda(\xi)e_n=xe_n\]
implies $\Lambda(e_nx)=e_n\xi$, we have
\[\Lambda(e_nx^{\frac12})\Lambda(e_nx^{\frac12})^\sharp=\Lambda(e_nx)=e_n\xi\to\xi\]
by $\xi\in\bar{x\dom x}$.
From $\Lambda(e_nx^{\frac12})\in A$, and $A^2$ can be approximated by $A_0^2$ using $R_n$, we have $\xi\in P^\sharp$, hence the duality between $P^\sharp$ and $P^\flat$.

Here, since $J\Delta^{\frac14}J=\Delta^{-\frac14}$, we get $\Delta^{\frac14}S\Delta^{-\frac14}=J=\Delta^{-\frac14}F\Delta^{\frac14}$ and $\Delta^{\frac14}P_0^\sharp=P_0=\Delta^{-\frac14}P_0^\flat$.
If $\xi\in P$, then
\[\<\xi,\eta\>=\<\Delta^{\frac14}\xi,\Delta^{-\frac14}\eta\>\ge0,\qquad\eta\in P,\]
so $\xi\in P^\circ$.
Conversely, let $\xi\in P^\circ$.
Then, we do not know if $\xi\in D$, but by introducing $R_n(\xi)\in D$, and the inequality $\<\xi,\eta\>\ge0$ implies
\[\<\Delta^{\frac14}R_n(\xi),\Delta^{-\frac14}\eta\>=\sqrt{\frac n\pi}\int_\R e^{-n^2t^2}\<\Delta^{\frac14}\Delta^{it}\xi,\Delta^{-\frac14}\eta\>\,dt=\sqrt{\frac n\pi}\int_\R e^{-n^2t^2}\<\Delta^{it}\xi,\eta\>\,dt\ge0\]
for all $\eta\in P$ by $\Delta^{it}P=P$ (Tomita-Takesaki commutation).
Thus, $\Delta^{\frac14}R_n(\xi)\in P^\flat\subset\bar{\Delta^{\frac14}P}$ and the limit shows $\xi\in P$.

Now we check the four conditions.
The first and third condition is clear.
For the second condition, taking a central $z\in Z(M)\cap\fa$ which can be compensated by approximation to the whole $Z(M)$,
\[\lambda(Sz\xi)=\lambda(z\xi)^*=(z\lambda(\xi))^*=z^*\lambda(\xi)^*=\lambda(z^*S\xi)\]
implies $Sz=z^*S$.
Every $u\in U(Z(M))$ satisfies $J\Delta^{\frac12}=uJuu^*\Delta^{\frac12}u$, and by the uniqueness of the polar decomposition, $J=uJu$.
Then, by linearly extend the identity $Ju=u^*J$, we have $Jz=z^*J$ and $JzJ=z^*$ for all $z\in Z(M)$.
To show the forth condition, approximate $\lambda(\xi)\in\fa$ to arbitrary $x\in M$ using the Kaplansky density keeping a net bounded, in the equation
\[\lambda(\xi)J\lambda(\xi)J(\eta\eta^*)=\lambda(\xi)\rho(\xi^*)(\eta\eta^*)=\xi\eta(\xi\eta)^*\in P,\qquad\eta\in A_0\]
to obtain $xJxJP\in P$.
\end{pf}

\begin{lem}
Let $M$ be a von Neumann algebra on a Hilbert space $H$.
The \emph{central support projection} of $x\in M$ is the smallest central projection $z(p)$ that fixes $x$ from left and right, and the \emph{cyclic projection} of $\xi\in H$ is the smallest projection $e(\xi)$ in $M$ that fixes $\xi$.
\begin{parts}
\item $z(p)H=MpH$.
\item $e(\xi)H=\bar{M'\xi}$.
\item $e(\xi)=s(\omega_\xi)$.
\end{parts}
\end{lem}
\begin{pf}
(a)
Let $z\in B(H)$ be the projection onto $MxH$.
We clearly have $p\le z$.
Since every element of $Mz$ or $M'z$ is fixed by the left action of $z$, we have that $z$ is central and $z(p)\le z$.
We also have
\[zH=\bar{MpH}=\bar{Mz(p)pH}=\bar{z(p)MpH}\subset z(p)H,\]
so $z\le z(p)$.
Therefore, $z=z(p)$.

(b)
Let $eH:=\bar{M'\xi}$.
Since $\xi\in\bar{M'\xi}=eH$, we have $e\xi=\xi$ and $e(\xi)\le e$ by definition of $e(\xi)$.
Conversely we have
\[eH=\bar{M'\xi}=\bar{M'e(\xi)\xi}=\bar{e(\xi)M'\xi}\subset e(\xi)H,\]
so $e\le e(\xi)$.
Therefore, $e=e(\xi)$.

(c)
The support projection $s(\omega)$ of a normal state $\omega$ is the smallest projection $p$ such that $\omega(1-p)=0$.
Since $\omega_\xi(1-e(\xi))=\<(1-e(\xi))\xi,\xi\>=0$, we have $s(\omega_\xi)\le e(\xi)$, and conversely since
\[0=\omega_\xi(1-s(\omega_\xi))=\omega_\xi((1-s(\omega_\xi))^2)=\|(1-s(\omega_\xi))\xi\|^2,\]
we have $s(\omega_\xi)\xi=\xi$ and $e(\xi)\le s(\omega_\xi)$.
\end{pf}

\begin{lem}
Let $(M,H,J,P)$ be a standard form.
Let $p\in M$ be a projection and let $q:=pJpJ$.
Then, $q\ne0$ if and only if $p\ne0$, and $(qMq,qH,qJq,qP)$ is a standard form.
\end{lem}
\begin{pf}
If $qxq=0$, then since $J$ commutes with central projections so that
\[p=JJpJJ\le Jz(JpJ)J=z(JpJ),\]
and we have $pxp=0$ from
\[pxpH=pxppH\subset pxpM(JpJ)H=pxp(JpJ)MH=qxqMH=0,\]
so $pMp\to qMq:pxp\mapsto qxq=pxp(JpJ)$ is a $*$-isomorphism and $q\ne0$ if and only $p\ne0$.

Note that $qJ=JqJ=Jq$.
The Hilbert space $qH$ is invariant under $J$, so $qJq$ is an anti-linear isometric involution on $qH$.
Because $qP=pJpJP\subset P$, we have $qP\subset(qP)^\circ$, and conversely if $q\xi\in(qP)^\circ$, then $\<q\xi,\eta\>=\<q\xi,q\eta\>\ge0,\qquad\eta\in P$ implies $q\xi\in P$ and $q\xi\in qP$, hence $qP$ is a self-dual cone.

(i)
Then, $qM'q$
\[(qJq)(qMq)(qJq)=qJMJq=qM'q=?=(qMq)'\]
in $B(qHq)$.

(ii)
For $qzq\in Z(qMq)=?=qZ(M)q$ for $z\in Z(M)$,
\[(qJq)(qzq)(qJq)=qJzJq=qz^*q=(qzq)^*.\]

(iii) and (iv) are clear from $qP\subset P$.
\end{pf}

\begin{lem}
Let $(M,H,J,P)$ be a standard form.
Then, $P$ spans $H$.
\end{lem}
\begin{pf}
We only show the closed linear span of $P$ is $H$.
If $\xi\in H\setminus\bar{\spn P}$, then $\<\pm\xi,\eta\>=0\ge0$ for all $\eta\in P$ implies that $\pm\xi\in P$ by the self-duality.
Then, $\<\xi,-\xi\>\ge0$ implies $\xi=0$.
We are done.
\end{pf}

\begin{prop}[Uniqueness]
Let $(M,H,J,P)$ and $(\tilde M,\tilde H,\tilde J,\tilde P)$ be standard forms.
If $\pi:M\to\tilde M$ is a $*$-isomorphism, then there exists a unique unitary $u:H\to\tilde H$ such that $\pi=\Ad u$, $\tilde J=uJu^*$, and $\tilde P=uP$.
\end{prop}
\begin{pf}
(Uniqueness)
If $u':H\to \tilde H$ also satisfies the condition and let $v:=u^*u'$.
The first condition implies $x=vxv^*$ for $x\in M$, we have $v\in M'$.
The second condition implies $v=JvJ\in M$ and $v\in Z(M)$ so that $v=JvJ=v^*$.
Apply the spectral decomposition to write $v=p-q$ for orthogonal central projections $p$ and $q$ such that $p+q=1$.
Then, $JqJ=q^*$ implies $qP=qJqJP\subset P$.
For any $\xi\in qP$, we have $p\xi=0$ by orthogonality, and the third condition $vP=P$ deduces that
\[0\le\<v\xi,\xi\>=-\|q\xi\|.\]
Thus, $qP=q(qP)=0$, and because $P$ spans $H$, we have $q=0$.
Therefore, $v=1$.

(Existence)
First we assume $M$ is countably decomposable.
We first claim there is a cyclic separating vector in $P$.
Let $\{\xi_i\}_{i\in I}$ be a maximal family of non-zero vectors in $P$ such that the cyclic projections $e(\xi_i)$ are mutually orthogonal.
Suppose $p:=1-\sum_ie(\xi_i)\ne0$ and let $q:=pJpJ$.
Then, $(qMq,qH,qJq,qP)$ is a standard form.
Take $\xi\in qP$ so that $p\xi=\xi$ and $pM'\xi=M'p\xi=M'\xi$ implies $e(\xi)\le p$.
It constradicts to the maximality of the family $\{\xi_i\}$, we have $p=0$.
Since $M$ is countably decomposable, the index set $I$ is countable, and we may assume $\sum_i\|\xi_i\|^2<\infty$.
Let $\xi:=\sum_i\xi_i\in P$.
The orthogonality of $e(\xi_i)$ is saying that $M'\xi_i$ are mutually orthogonal, and the commutation $JMJ=M'$ implies that $M\xi_i$ are all mutually orthogonal, hence $\omega_\xi=\sum_i\omega_{\xi_i}$ by expansion of the sum.
Then,
\[e(\xi)=s(\omega_\xi)=\sum_is(\omega_{\xi_i})=\sum_ie(\xi_i)=1,\]
and $\xi$ is separating.
For $J\xi=\xi$ and $JMJ=M'$, $\xi$ is also cyclic.

Now with any fixed cyclic separating vector $\xi$ taken as above, we construct a left Hilbert algebra $M\xi$ in $H$ and the associated standard form $(M,H,J_\xi,P_\xi)$.
Note that $M\xi$ is a core of $F_\xi J$ and $JS_\xi$.
Since
\[(JS_\xi)^*x\xi=F_\xi Jx\xi=F_\xi(JxJ)\xi=(JxJ)^*\xi=Jx^*J\xi=Jx^*\xi=JS_\xi x\xi,\qquad x\in M\]
and
\[\<JS_\xi x\xi,x\xi\>=\<Jx\xi,S_\xi x\xi\>=\<JxJ\xi,x^*\xi\>=\<xJxJ\xi,\xi\>\ge0,\qquad x\in M\]
from $xJxJP\subset P$, $JS_\xi$ is a non-negative self-adjoint operator.
By the uniqueness of the polar decomposition for $S_\xi$, we have $J=J_\xi$.
By definition of $P$ and $P_\xi$, we also have $P=P_\xi$.
Therefore, for a countably decomposable von Neumann algebra on a fixed Hilbert space the standard form is uniquely determined as sets if it admits, so is the unitary $u:H\to\tilde H$.

In general case, take an increasing net $p_i$ of countably decomposable projections of $M$ whose supremum is the unit.
Define $q_i:=p_iJp_iJ$.
Consider countably decomposable von Neumann algebras $q_iMq_i$ and projections $r_i\in\tilde M$ defined such that $\pi(q_iMq_i)=r_i\tilde Mr_i$.
We can see that $q_i$ and $r_i$ also increase to the units.
For each $i$, we have a unitary $u_i:q_iH\to r_i\tilde H$ which preserves the unique standard forms on $(q_iMq_i,q_iH)$ and $(r_i\tilde M r_i,r_i\tilde H)$.
By the uniqueness of the unitaries, if $q_i\le q_j$ then $u_i\subset u_j$, so the limit $u=\bar{\bigcup_iu_i}$ is a well-defined unitary satisfying the desired conditions.
\end{pf}

\begin{cor}[Normal state correpondence]
Let $(M,H,J,P)$ be a standard form.
Then, $P\to M_*^+:\xi\mapsto\omega_\xi$ is a homeomosphism.
\end{cor}
\begin{pf}
We first show the surjectivity.
Take a normal state $\omega$.
Let $p:=s(\omega)$ be the support projection and let $q:=pJpJ$.
Since $pMp\to qMq$ is a $*$-isomorphism, we can induce $\omega$ to define a faithful normal state $\omega'$ on $qMq$ such that $\omega'(qxq)=\omega(x)$ for $x\in M$.
Since the GNS representation of a faithful normal state defines a standard form, applying the uniqueness theorem, we can take $q\xi\in qP$ such that $\omega'(qxq)=\<qxq\xi,\xi\>$.
Then, $q\xi$ is the vector we want.

The vector state map is a continuous embedding by the following inequality:
\[\|\xi-\eta\|^2\le\|\omega_\xi-\omega_\eta\|\le\|\xi-\eta\|\|\xi+\eta\|,\qquad\xi,\eta\in P.\qedhere\]
\end{pf}

\begin{cor}[Unitary implementation]
Let $(M,H,J,P)$ be a standard form.
Let $U$ be the set of all untaries $u\in U(H)$ such that $uMu^*=M'$, $uJu^*=J$, and $uP=P$.
Then, $U\to\Aut(M):u\mapsto\Ad u$ is a topological isomorphism.
\end{cor}
\begin{pf}
The uniqueness theorem proves the bijectivity.
The continuity and inverse continuity comes from the one-to-one correspondence $P\to M_*^+$ by taking transpose.
\end{pf}



\subsection{Crossed product}

\begin{defn}[Convolution algebra of action]
Let $(M,G,\alpha)$ be a W$^*$-dynamical system.
Let $C_c(G,M)$ be the $*$-algebra of $\sigma$-strongly$^*$ continuous functions such that
\[f*g(s):=\int_G\alpha_t(f(st))g(t^{-1})\,dt,\qquad f^\sharp(s):=\Delta(s)^{-1}\alpha_{s^{-1}}(f(s^{-1})^*).\]
If we fix a faithful semi-finite normal weight $\f$ on $M$, then we can define a left Hilbert algebra structure on $C_c(G,M)$.
\end{defn}


\begin{defn}[Covariant representations]
Let $(M,G,\alpha)$ be a W$^*$-dynamical system.
A \emph{covariant representation} of $(M,G,\alpha)$ is an equivariant $*$-homomorphism
\[\pi:(M,G,\alpha)\to(B(H),G,\Ad u),\]
where $u:G\to U(H)$ is a strongly continuous representation.

Under a fixed normal covariant representation $(\pi,u)$ of $(M,G,\alpha)$ on $H$ in which the action $\alpha$ is implemented as unitaries, we introduce a $*$-homomorphism
\[\tilde\pi:C_c(G,M)\to B(H)\]
defined by the $\sigma$-weak Pettis integral
\[\tilde\pi(f):=\int u_s\pi(f(s))\,ds,\qquad f\in C_c(G,M).\]
This integral, sometimes called the \emph{integral form} of $f$, justified by Pettis integral.
\end{defn}


\begin{rmk*}
If we omit the symbol $\pi$ by letting it as an inclusion, we can write the product of $f,g\in C_c(G,M)$ more intuitively as
\begin{align*}
\Bigl(\int u_tf(t)\,dt\Bigr)\Bigl(\int u_sg(s)\,ds\Bigr)
&=\iint u_tf(t)u_sg(s)\,ds\,dt=\iint u_{ts}\alpha_{s^{-1}}(f(t))g(s)\,ds\,dt\\
&=\iint u_s\alpha_{s^{-1}t}(f(t))g(t^{-1}s)\,ds\,dt=\int u_s\int\alpha_t(f(st))g(s)\,dt\,ds
\end{align*}
\end{rmk*}

\begin{rmk*}
The existence of covariant representation is always guaranteed by a standard form.
Consider an equivariant $*$-homomorphism $\pi:(A,G,\alpha)\to(B(H),G,\beta)$ to a W$^*$-dynamical system.
For $(B(L^2(H)),L^2(H),*,L^2(H)^+)$ is the standard form of $B(H)$ with respect to the usual trace, the left multiplication defines an injective unital normal equivariant $*$-homomorphism $\iota:(B(H),G,\beta)\to(B(L^2(H)),G,\Ad u)$, where $u:G\to U(L^2(G,H))$ is a strongly continuous representation taken by the unitary implementation.
Thus, the closure of $\pi(A)$ in $B(H)$ can be seen to be in $B(L^2(H))$ up to isomorphism.
It means that covariant representations of an algebraic $*$-dynamical system $(A,G,\alpha)$ are sufficient when considering C$^*$ or W$^*$ closures of $*$-algebras in various representations.
\end{rmk*}


\begin{prop}[Regular representation]
Let $(M,G,\alpha)$ be a W$^*$-dynamical system.
Let $(\pi,u)$ be a covariant representation on $H$.
There is a commutative diagram of equivariant $*$-homomorphisms
\[\begin{tikzcd}
(M,G,\alpha) \rar{\pi\otimes1} \ar[swap]{dr}{\pi_\alpha} & (B(H\otimes L^2(G)),G,\Ad u\otimes\Ad\lambda) \dar{\Ad w^*}\\
&(B(H\otimes L^2(G)),G,\id\otimes\Ad\lambda),
\end{tikzcd}\]
where $\pi_\alpha$ and $w$ are defined such that
\[(\pi_\alpha(x)\xi)(s):=\pi(\alpha_{s^{-1}}(x))\xi(s),\qquad w\xi(s):=u_s\xi(s),\]
for $x\in M$, $s\in G$, and $\xi\in L^2(G,H)=H\otimes L^2(G)$.
\end{prop}

\begin{defn}
Let $(M,G,\alpha)$ be a W$^*$-dynamical system.
Consider the covariant representation $(\pi_\alpha,1\otimes\lambda)$ of $(M,G,\alpha)$ and the corresponding integral form $\tilde\pi_\alpha:C_c(G,M)\to B(L^2(G,H))$.
The \emph{crossed product} of $M$ by $G$ is the von Neumann algebra $M\rtimes_\alpha G:=\tilde\pi(C_c(G,M))''$ on $L^2(G,H)$.
\end{defn}
\begin{prop}
Let $(M,G,\alpha)$ be a W$^*$-dynamical system.
\begin{parts}
%\item For a cocycle $v:\R\to U(M)$, if we define $u\in U(L^2(G,H))$ such that $u\xi(x):=v_s^{-1}\xi(s)$, then $(\Ad u)M\rtimes_\alpha G=M\rtimes_{\alpha^v}G$.
\item $M\rtimes_\alpha G=\{\ \pi_\alpha(x)\ ,1\otimes\lambda_s:x\in M,\ s\in G\}''$.
\item $M\rtimes_\alpha G\cong\{\pi(x)\otimes1,u_s\otimes\lambda_s:x\in M,\ s\in G\}''$.
\item $M\rtimes_\alpha G\cong(M\bar\otimes B(L^2(G)))^{\alpha\otimes\Ad\rho}$.
\end{parts}
\end{prop}
\begin{pf}
The equivariance of $\pi_\alpha$ follows from
\begin{align*}
(\Ad\lambda_t(\pi_\alpha(x))\xi)(s)
&=(\lambda_t\pi_\alpha(x)\lambda_{t^{-1}}\xi)(s)\\
&=(\pi_\alpha(x)\lambda_{t^{-1}}\xi)(t^{-1}s)\\
&=\pi(\alpha_{s^{-1}t}(x))\lambda_{t^{-1}}\xi(t^{-1}s)\\
&=\pi(\alpha_{s^{-1}}(\alpha_t(x)))\xi(s)\\
&=(\pi_\alpha(\alpha_t(x))\xi)(s),\qquad s,t\in G,\ x\in M,\ \xi\in H.
\end{align*}

\end{pf}

\begin{ex}[Semi-direct product of locally compact groups]
Let $G$ and $H$ be locally compact groups.
Let $\alpha:G\to\Aut(H)$ be a continuous action and let $\alpha:G\to\Aut(W^*_r(H))$ be the corresponding action.
Is it true that $W^*_r(H)\rtimes_\alpha G\cong W^*_r(H\rtimes_\alpha G)$?
How about C$^*$-cases and full algebra cases?
($C^*(H)\rtimes_\alpha G\cong C^*(H\rtimes_\alpha G)$ is okay.)

Group homomorphisms to group operator algebras?
Closed groups for operator algebras?
\end{ex}




\subsection{Takesaki duality}


Heisenberg-Weyl commutation relation

\[\begin{tikzcd}
\{(x\otimes1\otimes1),(u_s\otimes\lambda_s\otimes1),(1\otimes\mu_p\otimes\lambda_p)\}''\subset B(H\otimes L^2(G)\otimes L^2(\hat G)) \dar{\id\otimes\id\otimes\Ad\cF}\\
\{(x\otimes1\otimes1),(u_s\otimes\lambda_s\otimes1),(1\otimes\mu_p\otimes\mu_p)\}''\subset B(H\otimes L^2(G)\otimes L^2(G)) \dar{\id\otimes\Ad W^*}\\
\{(x\otimes1\otimes1),(u_s\otimes\lambda_s\otimes1),(1\otimes\mu_p\otimes1)\}''\subset B(H\otimes L^2(G)\otimes L^2(G))\\
\{(x\otimes1),(u_s\otimes\lambda_s),(1\otimes\mu_p)\}''\subset B(H\otimes L^2(G)) \dar{\id\otimes\Ad w^*}\uar{\id\otimes\id\otimes1}\\
\{\ \pi_\alpha(x)\ ,(1\otimes\lambda_s),(1\otimes\mu_p)\}''\subset B(H\otimes L^2(G))
\end{tikzcd}\]




\subsection{Spectral analysis}

For $f\in L^1(G)$, its Fourier transform is defined as a field of operators $\hat f:=\{\hat f(p)\}_{p\in\hat G}$ such that $\hat f(p):=\int_Gf(s)p_s^{-1}\,ds\in B(H_p)$.



\[I(\alpha):=\{f\in L^1(G):\int_Gf(s)\alpha_s^{-1}\,ds=0\},\qquad\sigma(\alpha):=\{p\in\hat G:\hat f(p)=0,\ f\in I(\alpha)\}\]

\[I_\alpha(x):=\{f\in L^1(G):\int_Gf(s)\alpha_s^{-1}(x)\,ds=0\},\qquad\sigma_\alpha(x):=\{p\in\hat G:\hat f(p)=0,\ f\in I_\alpha(x)\}\]

\[X^\alpha(E):=\{x\in X:\sigma_\alpha(x)\subset E\},\qquad X_0^\alpha(U):=\bar{\{\int_Gf(s)\alpha_s^{-1}(x)\,ds:\supp f\subset U,\ x\in X\}}^{\sigma(X,F)}\]

In some sense, here we want to decompose $\alpha$ in terms of irreducible representations of $G$.

For a unitary action $u:\R\to U(H)$, the Arveson spectrum of $u$ is the spectrum of its imaginary infinitesimal generator.

Connes spectrum and centrally ergodic actions





\subsection{Structure theory of type III von Neumann algebras}

\begin{thm}
A von Neumann algebra $M$ is of type III if and only if there is a flow $(N,\R,\theta)$ on a semi-finite von Neumann algebra with a faithful semi-finite trace $\tau$ such that
\begin{enumerate}[(i)]
\item $\theta$ scales the trace: $\tau\circ\theta_s=e^{-s}\tau$ for $s\in\R$,
\item $(C_N,\R,\theta)$ has no subsystem isomorphic to $(L^\infty(\R),\R,\lambda)$.
\item $M\cong N\rtimes_\theta\R$.
\end{enumerate}
\end{thm}

$(N,\R,\theta)$ non-commutative flow of weights,
$(C_N,\R,\theta)$ flow of weights.

$M$ is a factor if and only if $\theta$ is centrally ergodic.




\emph{modular spectrum} $S(M)$
\emph{modular period group} $T(M)$


\begin{thm}
If $M$ is a factor of type III$_{0<\lambda<1}$, then $M\cong N_0\rtimes_\theta\Z$ for a factor $N_0$ of type II$_\infty$ with some $\theta\in\Aut(N)$ of modulus $\lambda^{-1}$.
\end{thm}
discrete decomposition, discrete core


Mackey's induced representation?







\subsection{Operator-valued weights}


\begin{defn}
Let $M$ and $N$ be von Neumann algebras.
The \emph{extended positive cone} of $M$ is the set $\hat M^+$ of lower semi-continuous additive homogeneous functions $M_*^+\to[0,\infty]$.

An \emph{operator-valued weight} from 
\end{defn}
Every normal weight $M^+\to[0,\infty]$ has a natural extension $\hat M^+\to[0,\infty]$.









\newpage
\section*{Appendix}

\begin{prop}
For positive elements
\[x(1+\e x)^{-1}\]
\begin{parts}
\item operator monotone,
\item $\sigma$-strongly continuous on a closed subset of its domain due to the boundedness of $f_\e$,
\item $f_\e(x)\to x$ in norm as $\e\to0$.
\end{parts}
\end{prop}



\begin{prop}
Let $X$ be a Fr\'echet space.
\begin{parts}
\item A weakly continuous semi-group $\alpha:[0,\infty)\to\End(X)$ is strongly continuous.
\item A weakly holomorphic function $f:\Omega\to X$ is strongly holomorphic.
\end{parts}
\end{prop}
\begin{pf}
(a)
On a compact interval, by applying the uniform boundedness principle twice, we can show $s\mapsto\alpha_s$ is uniformly bounded.
In detail, $s\mapsto\<\alpha_s(x),x^*\>\in\C$ is bounded for each $x\in X$ and $x^*\in X^*$, so $s\mapsto\alpha_s(x)\in X$ is bounded and $s\mapsto\alpha_s$ is bounded.
If $\alpha$ is already bounded, then the above argument is not required.

Let $S$ be the set of all $x\in X$ such that $\alpha_s(x)\to x$ strongly as $s\to0$.
By the boundedness of the action, $S$ is a closed convex subset of $X$, and weakly closed by the Hahn-Banach separation.
By smoothing, we can show $S$ is weakly dense in $X$, so we get $S=X$.

(b)
Suppose $f:B(0,1+\e)\to X$ is weakly holomorphic such that $f(0)=0$.
By the Pettis integral in the weak topology of $X$, the integral
\[\frac1{2\pi i}\int_{|w|=1}\frac{f(w)}{w^2}\,dw\]
exists in $X$ and we have
\[\frac{f(z)}z=\frac1{2\pi i}\int_{|w|=1}\frac1{w-z}\frac{f(w)}w\,dw,\qquad z\in B(0,1).\]
Therefore, since the set $\{f(w):|w|=1\}$ is weakly bounded and hence strongly bounded by the uniform boundedness principle, we have for $z\in B(0,\frac12)$
\[\Bigl\|\frac{f(z)}z-\frac1{2\pi i}\int_{|w|=1}\frac{f(w)}{w^2}\,dw\Bigr\|=\Bigl\|\frac z{2\pi i}\int_{|w|=1}\frac{f(w)}{w^2(w-z)}\,dw\Bigr\|\le2|z|\sup_{|w|=1}\|f(w)\|\to0,\qquad z\to0,\]
which means $f$ is strongly holomorphic at zero.

\end{pf}

\begin{prop}
smooth vector, analytic vector, entire vector
\end{prop}

\begin{prop}
strong commutation
\end{prop}

\end{document}
