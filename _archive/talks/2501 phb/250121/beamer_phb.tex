\documentclass{../../../slide}


\title{Positive Hahn-Banach separation theorems in operator algebras}
\author{Ikhan Choi}
\institute{The University of Tokyo}
\date{Tokyo, January 2025}

% 20 minutes

\begin{document}


\begin{frame}[plain]
\titlepage
\end{frame}



\section{Introduction and preliminaries}
\contents


\begin{frame}{Positive Hahn-Banach separation theorems in operator algebras}
In $E$ an ordered vector space, $F\subset E^+$ is called \emph{hereditary} if $0\le x\le y\in F$ implies $x\in F$.
\begin{thm}[Haagerup '75, C. '25]
Let $M$ be a von Neumann algebra, and let $A$ be a C$^*$-algebra.
\begin{enumerate}[(1)]
\item If $F$ is a $\sigma$-weakly closed convex hereditary subset of $M^+$, then for any $x\in M^+\setminus F$ there exists $\omega\in M_*^+$ such that $\omega(x)>1$ and $\omega(x')\le1$ for all $x'\in F$.
\item If $F_*$ is a norm closed convex hereditary subset of $M_*^+$, then for any $\omega\in M_*^+\setminus F_*$ there exists $x\in M^+$ such that $\omega(x)>1$ and $\omega'(x)\le1$ for all $\omega'\in F_*$.
\item If $F$ is a norm closed convex hereditary subset of $A^+$, then for any $a\in A^+\setminus F$\qquad there exists $\omega\in A^{*+}$ such that $\omega(a)>1$ and $\omega(a')\le1$ for all $a'\in F$.
\item If $F^*$ is a weakly$^*$ closed convex hereditary subset of $A^{*+}$, then for any $\omega\in A^{*+}\setminus F^*$ there exists $a\in A^+$ such that $\omega(a)>1$ and $\omega'(a)\le1$ for all $\omega'\in F^*$.
\end{enumerate}
\end{thm}
\pause
Haagerup proved (1)\sim(3) in his master's thesis \cite{MR380438}, and asked if (4) holds.
The part (1) plays a major role in the proof of some equivalence conditions for normal weights on a von Neumann algebra.
The difficulty is (3)$<$(2)$\approx$(1)$<$(4).
I proved (1) and (2) in different ways, and solved (4).
\end{frame}



\begin{frame}{Suppression by the one-parameter family of functional calculi}
Since we can write $F^{r+r+}=(F-E^+)^{rr+}=(\overline{F-E^+})^+$ by the usual real bipolar theorem, where $r$ denotes the real polar, each statement is equivalent to $(\overline{F-E^+})^+\subset F$.
\pause
\begin{defn}
For $\delta>0$, we define $f_\delta:(-\delta^{-1},\infty)\to\mathbb{R}$ such that $f_\delta(t):=\dfrac t{\delta t+1}$ for $t>-\delta^{-1}$.
\end{defn}
It has many interesting properties such as operator monotonicity, semi-group property, increasing strong convergence to the identity, etc.
\pause
For example in (1), fixing $\delta$,
\[\begin{array}{ccccccc}
F-M^+ & \ni & x_i & \le & y_i & \in & F \\
&& \downarrow && \downarrow &&\\
0 & \le & x & \le & ? & \in & F
\end{array}
\quad\pause\leadsto\quad
\begin{array}{ccccccc}
F-M^+ & \ni & f_\delta(x_i) & \le & f_\delta(y_i) & \in & F \\
&& \downarrow && \downarrow &&\\
0 & \le & f_\delta(x) & \le & y_\delta & \in & F
\end{array}\]
Haagerup used the $\sigma$-strong topology to have $f_\delta(x_i)\to f_\delta(x)$ in the proof of $(1)$.
\pause
Since $A^*$ has no analogue of the $\sigma$-strong topology, we use an inequality like $t-\varepsilon\le f_\delta(t)\le t$ on a suitable interval, to approximate $x$ by elements majorized by $y_i$.
\[\begin{array}{ccccccccc}
F-M^+ & \ni & x_i-\varepsilon & \le & f_\delta(x_i) & \le & f_\delta(y_i) & \in & F \\
&& \downarrow && && \downarrow &&\\
0 & \le & x && \le && y_\delta & \in & F
\end{array}\]
\end{frame}



\begin{frame}{Bounded commutant Radon-Nikodym derivatives}
To take functional calculi on linear functionals:
\pause
\begin{defn}
Let $M$ be a von Neumann algebra, and let $\psi\in M_*^+$.
Consider the Gelfand-Naimark-Segal representation $\pi:M\to B(H)$ associated to $\psi$ with the canonical cyclic vector $\Omega\in H$.
Then, we have a positive bounded linear map $\theta:\pi(M)'\to M_*$ defined such that
\[\theta(h)(x):=\langle h\pi(x)\Omega,\Omega\rangle,\qquad h\in\pi(M)',\ x\in M.\]
It has the image
\[\operatorname{im}\theta=\{\omega\in M_*:\text{there is $C>0$ such that $|\omega(x)|\le C\psi(x)$ for all $x\in M^+$}\}.\]
We will call $\theta^{-1}(\omega)$ the \emph{commutant Radon-Nikodym derivative} of $\omega$ with respect to $\psi$.
\end{defn}
\pause
For example in (2), when $\omega_n\in F_*-M_*^+$ converges to $\omega\in M_*^+$ in norm, we can find a suitable $\psi\in M_*^+$ such that
\[\begin{array}{ccccccc}
F_*-M_*^+ & \ni & \theta(f_\delta(\theta^{-1}(\omega_n)) & \le & \theta(f_\delta(\theta^{-1}(\varphi_n)) & \in & F_* \\
&& \downarrow && \downarrow &&\\
0 & \le & \theta(f_\delta(\theta^{-1}(\omega)) & \le & \varphi_\delta & \in & F_*
\end{array}\]
\end{frame}

\section{Proof sketches}
\contents

\begin{frame}{Proof of (1)}
We prove (1) in a different way to motivate the proof methods of (4).
Recall that we need to prove $(\overline{F-M^+})^+\subset F$.
To use the Krein-\v Smulian theorem, we define a subset $G$ satisfying $F-M^+\subset G$ and $G^+\subset F$ and $\overline G\subset G$.
\pause
Haagerup originally defined and proved the $\sigma$-strong closedness of
\[G:=\{x\in M^{sa}:f_\delta(x)\in F-M^+\text{ for all }\delta<\|x_-\|^{-1}\}.\]
\pause
Instead, to avoid the use of $\sigma$-strong topology, we define
\[G:=\left\{x\in M^{sa}:\begin{tabular}{c}
for any $\varepsilon>0$, there is a net $y_\delta\in F$\\
indexed on $0<\delta\le(1+\|x\|)^{-1}$ such that\\
$\|y_\delta\|\le\delta^{-1}$ and $f_\delta(x)\le y_\delta+\varepsilon\delta^{\frac12}$
\end{tabular}\right\}.\]
\pause
\begin{itemize}
\item $F-M^+\subset G$: Easy.
\item $G^+\subset F$: Relatively easy. Fix $\delta'>0$ and obtain $(1+\delta'\|x\|)^{-1}f_\delta(x)\in F$ by limiting
\[0\le(1+\delta'\|x\|)^{-1}f_\delta(x)\le f_{\delta'}(f_\delta(x))\le f_{\delta'}(y_\delta+\delta^{\frac12})\le f_{\delta'}(y_\delta)+\delta^{\frac12}.\]
\item $\overline G\subset G$: If $x_i\in G$ is bounded and $x_i\to x$ $\sigma$-weakly, then we can construct $y_\delta\in F$ such that $y_{i,\delta}\to y_\delta$ for $\delta\le\delta_0$ and $y_\delta:=f_{\delta-\delta_0}(y_{\delta_0})$ for $\delta>\delta_0$ for small $\delta_0>0$.
Long computations.
The convexity follows from $F-M^+\subset G$ and $\overline G\subset G$, so the Krein-\v Smulian theorem completes the proof.
\end{itemize}
\end{frame}

\begin{frame}{Proof of (4)}
For (4), we define
\pause
\[G^*:=\left\{\omega\in A^{*sa}:\begin{tabular}{c}
for any $\varepsilon>0$, there are nets $\psi_\delta\in A^{*+}$ and $\varphi_\delta\in F^*$\\
indexed on $0<\delta\le(1+4\|\omega\|)^{-6}$ such that\\
the following five conditions are satisfied:\\
 $|\omega(a)|\le\delta^{-\frac16}\psi_\delta(a)$ for all $a\in A^+$, $\|\psi_\delta\|\le1$, $\|\varphi_\delta\|\le\delta^{-1}$,\\
$\omega_\delta\le\varphi_\delta+\varepsilon\delta^{\frac12}\psi_\delta$, and $\omega_\delta\to\omega$ weakly$^*$ in $A^*$ as $\delta\to0$
\end{tabular}\right\},\]
where $\omega_\delta:=\theta_\delta(f_\delta(\theta_\delta^{-1}(\omega)))$, and here $\theta_\delta$ is associated to $\psi_\delta$.
\pause
\begin{itemize}
\item $F^*-A^{*+}\subset G^*$: Take $\psi_\delta:=(1+\|\omega\|)^{-1}([\omega]+(1+\|\varphi\|)^{-1}\varphi)$ and $\varphi_\delta:=\theta(f_\delta(\theta(\varphi)))$.
\item $G^{*+}\subset F^*$: Take the Radon-Nikodym for $\omega+\delta\varphi_\delta+\psi_\delta$ and do the same thing as (1).
\item $\overline{G^*}\subset G^*$: ... we can prove in a similar way to (1) ... but Looong computations ...
\end{itemize}
\end{frame}


\begin{frame}{Questions}
\begin{itemize}
\item Simpler proof? (in conversation with N. Ozawa)
\item Weight theory on C$^*$-algebras?
\item Convex hereditary subsets instead of convex balanced subsets?
\item Non-commutative $L^p$ spaces?
\end{itemize}
\end{frame}

\references


\end{document}