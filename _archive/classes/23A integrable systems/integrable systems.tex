\documentclass{../../../small}
\usepackage{../../../ikhanchoi}

\DeclareMathOperator{\Int}{Int}

\begin{document}
\title{Integrable Systems}
\author{Ikhan Choi\\Lectured by Ralph Willox\\University of Tokyo, Autumn 2023}
\maketitle
\tableofcontents

\newpage
\section{Day 1: October 9}

Every linear system will be considered to be solvable.

Two mainstreams: 1+1-dimensional, 2+1-dimensional (Russia, Japan)

Sato theory

Hirota: relation to discrete systems

\subsection*{KdV equation}
Consider $u=u(x,t)\in C^\infty(\R^2)$ which is Schwarz in the sense that $\lim_{x\to\pm\infty}u_{px,qt}=0$ for all $p,q$.
The KdV equation is
\[u_t+u_{3x}+6uu_x=0.\]
One of the form of the discrete KdV equation(dKdv) by Hirota is
\[\frac1{u_{n+1}^{t+1}}-\frac1{u_n^t}=\delta(u_n^{t+1}-u_{n+1}^t)\]
for $u:\Z^2\to\C$ and $\delta\in\C\setminus\{0\}$.

There is a continuous limit(?) deduces KdV from dKdV.
Using the Darboux transformation(?) we can construct dKdV from KdV, preserving some additive formulas(such as sine and cosine?), but it seems to be impossible to reconstruct by numberical schemes.

There is an ultradiscrete limit which derives udKdV equation.
($\delta=e^{-1/\e}$ and $u=e^{U/\e}$ with $\e\to0$.
\[\e\log(\frac1{u_{j+1}^{t+1}}+\delta u_{j+1}^t)=\e\log(\delta u_j^{t+1}+\frac1{u_j^t})\]
\[\max[-U_{j+1}^{t+1},U_{j+1}^t-1]=\max[U_j^{t+1}-1,-U_j^t].\]
Toda equation
\[(\Theta_j)_{xy}=e^{-\Theta_{j+1}}-2e^{-2\Theta_j}+e^{-\Theta_{j-1}}.\]
)
The multiplication and addition are transformed to the addition and $\max$(or $\min$) respectively.
One of the form of the ultradiscrete KdV equation(udKdV) is
\[U_n^{t+1}=\min\left[1-U_n^t,\sum_{k=-\infty}^{n-1}(U_k^t-U_k^{t+1})\right].\]
If $U_n^0\in\{0,1\}$ for all $n$, then it is the Box and Ball system.

\subsection*{Properties that must be satisfied by integrable systems}

\begin{enumerate}
\item There are special solutions descrbing interactions of $\forall N$-soliton.
\item There exists a corresponding Hirota bilinear form.
\item There exists a Lax pair. (for udKdV, there are four equations for the Lax ``pair'', so it is not a Lax pair rigorously)
\item There are infinitely many invariants. (for udKdV there are some combinatoral arguments with Young tableaux) (for dKdV it is an open problem. For example, $\prod_ju_j^t$ is conserved, but only three? or four quantities are known.)
\item There are infinitely many symmetries. (Nobody has shown that there is no conserved quantity which does not come from a symmetry in the sense of Noether's theorem)
\item There is a Hamiltonian structure: the continuous KdV only have this property. (dKdV and udKdV admit Poisson structures)
\end{enumerate}

\subsection*{$N$-soliton of KdV}

Let $u(x,t)=U(x-ct)=U(\zeta)\not\equiv0$ with $c>0$ as an ansatz.
Then, $u_t=-cU'$ and $u_x=U'$ so that the KdV becomes
\[-cU'=U'''+6UU'.\]
This process generating an ODE is called a reduction.
We can integrate the equation to get
\[-cU=U''+3U^2+a\]
and
\[-\frac c2U^2=\frac12(U')^2+U^3+aU+b.\]
Thus we have a general solution of $U$ with the Weierstrass $\wp$ such that $(\wp')^2=4\wp^3-g_2\wp-g_3$ that
\[u=\frac c6-2\wp(\zeta),\]
where 
\[g_2=a+\frac{c^2}12,\qquad g_3=-(\frac{c^3}{216}+\frac{ac}{12}+\frac bc).\]
The general means that every transmitting solution has this form.
The Weierstrass $\sigma$ function satisfies $\wp(\zeta)=-\frac d{d\zeta^2}\log\sigma(\zeta)$.
Then we have
\[u(x,t)=2\partial_\zeta^2\log[\sigma(\zeta)e^{\frac c{24}\zeta^2}].\]
The function $\sigma(\zeta)e^{\frac c{24}\zeta^2}$ is a prototype of $\tau$ functions of Hirota and Sato.
If $a=b=0$, then the pole of $\sigma$ disappears in the sense that
\[u(x,t)=2\partial_x^2\log[1+e^{\theta(x,t)}],\qquad\theta(x,t)=k(x-k^2t+\delta),\quad k,\delta\in\R.\]
Here $c=k^2>0$.
This solution is the 1-soliton parametrized by $k$ and $\delta$.

\subsection*{Hirota bilinear form}
If we define $\tau:=1+e^\theta$ so that $u=2\partial_x^2\log\tau$, then
\[u_t+u_{3x}+uu_x=(2\log\tau)_{xxt}+(u_{2x}+3u^2)_x=2\left[\frac{\tau_{xt}\tau-\tau_x\tau_t+\tau_{4x}\tau-4\tau_{3x}\tau_x+3\tau_{2x}^2}{\tau^2}\right]_x.\]
One of the miracles is the denominator is $\tau^2$, not $\tau^3$.
For the case $\tau=1+e^{\theta(x,t)}$ we have the numerator zero:
\[\tau_{xt}\tau-\tau_x\tau_t+\tau_{4x}\tau-4\tau_{3x}\tau_x+3\tau_{2x}^2=0.\]

Define a differential operator $\cD$ such that
\[\cD_x^p\cD_t^qf\cdot g:=\partial_\e^p\partial_\eta^qf(x+\e,t+\eta)g(x-\e,t-\eta)|_{\e=\eta=0}.\]
For example,
\begin{gather*}
\cD_x^pf\cdot g=\sum_{n=0}^p{p\choose n}(-1)^{p-n}f_{nx}g_{(p-n)x},\\
\cD_x\cD_tf\cdot g=f_{xt}g-f_xg_t-f_tg_x+fg_{xt},\\
\cD_x^4f\cdot g=f_{4x}g-4f_{3x}g_x+6f_{2x}g_{2x}-4f_xg_{3x}+fg_{4x}.
\end{gather*}
In fact, $(\cD_x\cD_t+\cD_x^4)\tau\cdot\tau=0$.

We have some properties
\begin{parts}
\item $\cD_x^p\cD_t^qf\cdot g=(-1)^{p+q}\cD_x^p\cD_t^qg\cdot f$
\item $\cD_x^p\cD_t^q(\lambda_1f+\lambda_2f_2)\cdot(\mu_1g_1+\mu_2g_2)=\cdots$ (bilinearity)
\item $\cD_x^p\cD_t^qe^{k_1x+\omega_1t}\cdot e^{k_2x\omega_2k}=(k_1-k_2)^p(\omega_1-\omega_2)^qe^{(k_1+k_2)x+(\omega_1+\omega_2)t}$.
\end{parts}
We can compute
\[(\cD_x\cD_t+\cD_x^4)(1+e^\theta)\cdot(1+e^\theta)=-2k^4e^\theta+2k^4e^\theta=0.\]

Let $u=2\partial_x^2\log\tau_2$, where
\[\tau_2:=1+e^{\theta_1}+e^{\theta_2}+A_{12}e^{\theta_1+\theta_2},\]
$\theta_j=k_j(x-k_j^2t+\delta_j)$, $A_{12}=((k_1-k_2)/(k_1+k_2))^2\notin\{0,\infty\}$.
Let $k_1>k_2>0$.
After interaction, the 1-solitons $u\sim2\partial_x^2\log(1+e^{\theta_1})$ and $u\sim2\partial_x^2\log(1+e^{\theta_2})$ is slightly translated with $\delta_1'=\delta_1+\frac1{k_1}\log A_{12}$ and $\delta_2'=\delta_2-\frac1{k_2}\log A_{12}$.
The fastest particle propagates because $\delta_1'<\delta_1$.

For 3-soliton,
\[\tau_3=1+e^{\theta_1}+e^{\theta_2}+e^{\theta_3}+A_{12}e^{\theta_1+\theta_2}+A_{13}e^{\theta_1+\theta_3}+A_{23}e^{\theta_2+\theta_3}+A_{12}A_{13}A_{23}e^{\theta_1+\theta_2+\theta_3}\]
with $A_{ij}=((k_i-k_j)/(k_i+k_j))^2$.

Then, $\delta_1\to\delta_1'=\delta_1+\frac1{k_1}(\log A_{12}+\log A_{13})$.
The three-body problem is chaotic and not integrable, so it is not surprising that integrable systems do not include three-body interactions.

The modified KdV(mKdV) equation is
\[V_t+V_{3x}-6V^2V_x=0.\]
If we put $u=\pm V_x-V^2$, then
\[u_t+u_{3x}+6uu_x=(\pm\partial_x-2V)(V_t+V_{3x}-6V^2V_x).\]
It implies that if we have a solution $V$ of mKdV, then we can construct a pair of solutions of KdV.
It is called the Miura transformation.
The inverse transformation is also abtained by solving the ODE: if we let $V=\partial_x\log\psi$, then
\[V_x-V^2=\psi^{-1}\psi_{xx}.\]
The Miura transformation is very close to the Lax pair.

The nonlinear Schr\"odinger equation, which describes photon soliton in optical LAN, is
\[i\psi_t+\psi_{2x}+\sigma|\psi|^2\psi=0.\]
The sine-Gordon equation is
\[\omega_{xt}=\sin\omega,\]
whose discrete version is virtually equivalent to the one of mKdV.

\newpage
\section{Day 2: October 16}
\subsection*{Lax pair}
For a function $u$, define operators
\[L(u):=\partial_x^2+u,\qquad M(u):=-4\partial_x^3-3(u\partial_x+\partial_xu)=-4\partial_x^3-6u\partial_x-3u_x.\]
If $u$ satisfies the continuous KdV $u_t+u_{3x}+6uu_x=0$, then we have as operators
\[L_t-[M,L]=u_t+u_{3x}+6uu_x=0,\]
where $L_t$ is just the multiplication operator by $u_t$ as the differential operator $\partial_x^2$ vanishes by $\partial_t$.

For any function $u$ and $\lambda\in\C$, consider the following ``Lax pair equation''
\begin{align*}
&L(u)\psi=\lambda^2\psi\\
&\psi_t=-M(u)\psi.
\end{align*}
It is important to note that this equation is linear.
If a non-trivial (for all $t$) $\psi$ satisfies the above equation for some $\lambda$, then the computation
\[((\lambda^2-u)\psi)_t=(-4M(u)\psi)_{xx}\quad\Leftrightarrow\quad(u_t+u_{3x}+6uu_x)\psi=0\]
implies $u$ satisfies KdV.
In other words, by solving the ``Lax pair equation'', we obtain the solutions of KdV equation.

A reference: Peter Lax (1968) ``Integrals of nonlinear equations of evolution and solitary waves'' Comm Pure Applied Math 21, 467-490.

We can see some similar structure to $L_t=[M,L]$ in seversal situations, especially in physics.
Recall the Hamilton equation $\dot f=\{f,H\}$ in the Hamiltonian mechanics.
Also in quantum mechanics if we let $U(t):=e^{-\frac i\hbar Ht}$, then
\[A_t=(A(t))_t=(U(t)^*AU(t))_t|_{t=0}=-\frac i\hbar[A,H]\]

\[*\qquad*\qquad*\]
\smallskip

Every transpose of operator will respect the \emph{real} inner product
\[\<A,B\>:=\int_\R A(x)B(x)\,dx.\]

Let $M(t)$ be such that $M(t)+M(t)^t=0$ and let $L(0)$ be some operator, like $\partial_x^2+u_0$.
If we define $U(t)$ such that $U(0)=\id$ and $U_t(t)=M(t)U(t)$ so that
\[U(t)=\id+\int_0^tM(u(s))U(s)\,ds,\]
whence $(U^tU)_t=(UU^t)_t=0$ so that the unitarity of $U(0)$ implies the unitarity of $U(t)$.
($U(t)$ commutes with $\partial_x^2$?)
For this $U(t)$ defined by $M(t)$, if we define $L(t):=U(t)L(0)U(t)^t$, then we automatically have $L_t(t)=[M(t),L(t)]$.

For example, let
\[M_1(t)\equiv-\partial_x,\qquad L(0):=\partial_x^2+u_0.\]
Then, $L(t)=\partial_x^2+u(t)$, where $u(t)$ is the solution of $u(0)=u_0$ and
\[0=L_t(t)-[M_1(t),L(t)]=u_t+u_x.\]

For another example, for any function $u(t)$, let
\[M_3(t):=-4[\partial_x^3+\frac34(u(t)\partial_x+\partial_xu(t))],\qquad L(0):=\partial_x^2+u_0.\]
Then, similarly, if we consider $U(t)$ we obtain $L(t)=\partial_x^2+u(t)$, where $u$ satisfies the KdV equation because
\[0=L_t(t)-[M_3(t),L(t)]=u_t+u_{3x}+6uu_x.\]

Under the condition that $M(t)$ is skew symmetric, once we have a non-trivial solution for the first ``Lax pair equation'' $L(0)\psi_0=\lambda^2\psi_0$ only at $t=0$, then we can deduce the solution $\psi(t)$ of $\psi_t(t)=M(t)\psi(t)$ is always non-trivial for all $t>0$ because $\psi(t)$ is given by the unitary transformation of $\psi_0$.
It implies that the existence of $\psi_0$ gives the existence of $u(t)$!

An exmplae of this kind of linearization: the logistic map $x_{n+1}=4x_n(1-x_n)$ with $[0,1]\to[0,1]:x_n\mapsto x_{n+1}$ is a solvable chaos in which the solution is given by $x_n=\sin^2\theta_n$, where $\theta_{n+1}=2\theta_n$.
The $\theta_{n+1}=2\theta_n$ is linear chaos.

\[*\qquad*\qquad*\]
\smallskip

Let $M_1(t):\equiv-\partial_x$, $M_3(t):=-4\partial_x^3-6u\partial_x-3u_x$, and $L(0)=\partial_x^2+u_0$.
We want to define $U_{13}(t_1,t_3)$ such that
\[U_{13}(0,0)=\id,\qquad (U_{13})_{t_1}=M_1(t)U_{13}(t),\qquad (U_{13})_{t_3}=M_3(t)U_{13}(t).\]
For $U_{13}$ to exist, we need to satisfy the consistency
\[(M_1)_{t_3}-(M_3)_{t_1}=[M_3,M_1],\]
implying
\[2(u_{t_1}+u_x)\partial_x+(u_{t_1}+u_x)_x=0.\]

KdV hierarchy
$L:=\partial_x^2+u(x,t_1,t_3,t_5,\cdots)$,
\[L_{t_j}=M_jL-LM_j\quad\Leftrightarrow\quad u_{t_j}+u_{jx}+\cdots=0,\]
\[u_{t_1}+u_x=0,\]
\[u_{t_3}+u_{3x}+6uu_x=0,\]
\[u_{t_5}+u_{5x}+(10uu_{2x}+5u_x^2+10u^3)_x=0,\cdots\]



\newpage
\section{Day 3: October 23}

Every $u$ is supposed to be in Schwartz class.
If $u$ is a solution of the KdV
\[u_t=u_{3x}+6uu_x,\]
then
\[\int u\,dx,\qquad\int\frac12u^2\,dx,\qquad\int(u^3-\frac12u_x^2)\,dx\]
are invariants as time flows, called the mass, momentum, and energy, respectively.
In fact, there are infinitely many invariants.
They follow not from the symmetry of KdV equation, but from the symmetry of action.

The symmetry of KdV: If $u(x,t)$ is a solution of KdV, then
\begin{enumerate}
\item so is $u(x+\e,t)$,
\item so is $u(x,t+\e)$,
\item so is $\lambda^2u(\lambda x,\lambda^3t)$, (scaling)
\item so is $u(x-6\e t,t)+\e\notin\cS$. (Galilean boost)
\end{enumerate}
For example, there is no corresponding invariant to the scaling symmetry of KdV, but Galilean boost is related to an invariant.
These (local) symmetries come from the one-parameter transformations
\begin{enumerate}
\item $X=x+\e$, $T=t$, $U=u$,
\item $X=x$, $T=t+\e$, $U=u$,
\item $X=e^{-\e}x$, $T=e^{-3\e}t$, $U=e^{2\e}u$,
\item $X=x+6\e t$, $T=t$, $U=u+\e$.
\end{enumerate}
They are invertible in sufficiently small $\e$.
For $u$ satisfying KdV, we want for
\[U((X,T)^{-1}(x,t;\e),u((X,T)^{-1}(x,t;\e));\e)\]
to satisfy KdV.
Differentiating along $\e$ at $\e=0$, define
\begin{align*}
\sigma:&=\frac{\partial U}{\partial(x,t)}\frac{\partial(X,T)^{-1}}{\partial\e}+\frac{\partial U}{\partial u}\frac{\partial u}{\partial (x,t)}\frac{\partial(X,T)^{-1}}{\partial\e}\\
&=(U_x,U_t)\cdot\frac{\partial(X,T)^{-1}}{\partial\e}+U_u(u_x,u_t)\cdot\frac{\partial(X,T)^{-1}}{\partial\e}\\
&=(U_x+U_uu_x,\ U_t+U_uu_t)\cdot(-X_\e,-T_\e)\ (?)\\
&=-(U_x+U_uu_x)X_\e-(U_t+U_uu_t)T_\e
\end{align*}

Let $K(u):=-(u_{3x}+6uu_x)$, that is, $K$ is a differential polynomial.
Then, for multi-indices $J=(j_t,j_x)$, by considering Fr\'echet derivative $K'$ of $K$, we have
\begin{align*}
u_t+\e\sigma_t+o(\e)
&=K(u)+\e K'(u)[\sigma]+o(\e)\\
&=K(u)+\e\sum_J\frac{\partial K}{\partial u_J}\sigma_J+o(\e)\\
&=K(u)+\e(\sigma_{3x}+6u\sigma_x+6u_x\sigma)+o(\e).
\end{align*}
We say $\sigma=\sigma(t,u)$ is a \emph{symmetry} of $K$ if it satisfies
\[\sigma(t,u)_t=K'(u)[\sigma(t,u)]\quad\Leftrightarrow\quad\frac{\partial\sigma}{\partial t}(t,u)+\sigma'(t,u)[u_t]=K'(u)[\sigma(t,u)]\quad\Leftrightarrow\quad\frac{\partial\sigma}{\partial t}=K'[\sigma]-\sigma'[K].\]
We use the notation $\sigma'=\partial\sigma/\partial u$.

Here is the fifth symmetry
\[\sigma(u)=u_{5x}+10uu_{3x}+20u_xu_{2x}+30u^2u_x.\]
It satisfies $\partial\sigma/\partial t=0$ and $\sigma'[K]=K'[\sigma]$.


\bigskip
Let $\rho$ be a differential polynomial with respect only to $x$.
If we let $I(u):=\int\rho(u)\,dx$, then
\[\frac{dI}{dt}=\int(\frac{\partial\rho}{\partial t}+\rho'(u)[u_t])\,dx=\frac{\partial I}{\partial t}+\int\rho'(u)[u_t]\,dx.\]
If we try to compute the gradient of $I$ as in the calculus of variations,
\[I'(u)[v]=\<\rho'(u)^T[1],v\>_{L^2(\R_x)}=:\<(\grad I)(u),v\>_{L^2(\R_x)}.\]
We can apply the integration by parts to compute the transpose.
For examples,
\begin{enumerate}
\item If $\rho(u)=u$, then $\grad I=1$.
\item If $\rho(u)=u^2$, then $\grad I=2u$.
\item If $\rho(u)=u^3-\frac12u_x^2$, then $\grad I=u_{2x}+3u^2$.
\end{enumerate}

Olver

\end{document}