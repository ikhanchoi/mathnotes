\documentclass[a4paper]{amsart}

\usepackage[T1]{fontenc}
\usepackage[bitstream-charter,cal]{mathdesign}
\linespread{1.15}

\newcommand{\e}{\varepsilon}

\theoremstyle{plain}
\newtheorem{thm}{Theorem}[section]
\newtheorem{lem}[thm]{Lemma}
\newtheorem{cor}[thm]{Corollary}
\theoremstyle{definition}
\newtheorem{defn}[thm]{Definition}


\title{Positive Hahn-Banach separations in operator algebras}
\author{Ikhan Choi}
\address{}
\subjclass[2020]{}

\begin{document}

\begin{abstract}

\end{abstract}

\maketitle

\section{Introduction}






\begin{itemize}
\item definition and properties of $f_\e(t):=(1+\e t)^{-1}t$
\item commutant Radon-Nikodym, relation between $\{\omega'\in M_*^+:\omega'\le\omega\}$ and $\{h\in\pi(M)'^+:h\le1\}$, order preserving linear map
\item Mazur lemma
\end{itemize}


\begin{defn}[Hereditary subsets]
Let $E$ be a partially ordered real vector space.
We say a subset $F$ of the positive cone $E^+$ is \emph{hereditary} if $0\le x\le y$ in $E$ and $y\in F$ imply $x\in F$, or equivalently $F=(F-E^+)^+$, where $F-E^+$ is the set of all positive elements of $E$ bounded above by an element of $F$.
A $*$-subalgebra $B$ of a $*$-algbera $A$ is called \emph{hereditary} if the positive cone $B^+$ is a hereditary subset of $A^+$.
We define the \emph{positive polar} of $F$ as the positive part of the real polar
\[F^{r+}:=\{x^*\in(E^*)^+:\sup_{x\in F}x^*(x)\le1\}.\]
\end{defn}
An example that is a non-hereditary closed convex subset of a C$^*$-algebra is $\mathbb{C}1$ in any unital C$^*$-algebra.


\begin{defn}[Lower dominated sequences]
Let $E$ be a partially ordered real vector space.
A sequence $x_n\in E$ is called \emph{lower dominated} if there is $x\in E$ such that $x\le x_n$ for all $n$.
If $E$ is the self-adjoint part of the predual of a von Neumann algebra where the Jordan decomposition holds, then we can change the definition such that $x\in -E^+$.
\end{defn}

\section{Positive Hahn-Banach separation theorems}

We start with the positive Hahn-Banach separation for von Neumann algebras and their preduals, and will close this section with the same theorem for C$^*$-algebras and their duals.


\begin{thm}[Positive Hahn-Banach separation for von Neumann algebras]\label{positive hahn-banach w*}
Let $M$ be a von Neumann algebra.
\begin{enumerate}
\item If $F$ is a $\sigma$-weakly closed convex hereditary subset of $M^+$, then $F=F^{r+r+}$. In particular, if $x'\in M^+\setminus F$, then there is $\omega\in M_*^+$ such that $\omega(x')>1$ and $\omega(x)\le1$ for $x\in F$.
\item If $F_*$ is a norm closed convex hereditary subset of $M_*^+$, then $F_*=F_*^{r+r+}$. In particular, if $\omega'\in M_*^+\setminus F_*$, then there is $x\in M^+$ such that $\omega'(x)>1$ and $\omega(x)\le1$ for $\omega\in F_*$.
\end{enumerate}
\end{thm}
\begin{proof}
(1)
Since the positive polar is represented as the real polar
\[F^{r+}=F^r\cap M_*^+=F^r\cap(-M^+)^r=(F\cup-M^+)^r=(F-M^+)^r,\]
the positive bipolar can be written as $F^{r+r+}=(F-M^+)^{rr+}=(\overline{F-M^+})^+$ by the usual real bipolar theorem, where the closure is for the $\sigma$-weak topology.
Because $F=(F-M^+)^+\subset(\overline{F-M^+})^+$, it suffices to prove the opposite inclusion $(\overline{F-M^+})^+\subset F$.

Let $x\in(\overline{F-M^+})^+$.
Take a net $x_i\in F-M^+$ such that $x_i\to x$ $\sigma$-strongly, and take a net $y_i\in F$ such that $x_i\le y_i$ for each $i$.
Suppose we may assume that the net $x_i$ is bounded.
For sufficiently small $\e$ so that the bounded net $x_i$ has the spectra in $[-(2\e)^{-1},\infty)$, we have $f_\e(x_i)\to f_\e(x)$ $\sigma$-strongly, and hence $\sigma$-weakly.
On the other hand, by the hereditarity and the $\sigma$-weak compactness of $F$, we may asumme that the bounded net $f_\e(y_i)\in F$ converges $\sigma$-weakly to a point of $F$ by taking a subnet.
Then, we have $f_\e(x)\in F-M^+$ by
\[0\le f_\e(x)=\lim_if_\e(x_i)\le\lim_if_\e(y_i)\in F,\]
thus we have $x\in F$ since $f_\e(x)\uparrow x$ as $\e\to0$.
What remains is to prove the existence of a bounded net $x_i\in F-M^+$ such that $x_i\to x$ $\sigma$-strongly.

Define a convex set
\[G:=\left\{x\in\overline{F-M^+}:\begin{tabular}{c}there is a sequence $x_m\in F-M^+$\\such that $-2x\le x_m\uparrow x$ $\sigma$-weakly\end{tabular}\right\}\subset M^{sa},\]
where $x_m$ denotes a sequence.
In fact, it has no critical issue on allowing $x_m$ to be uncountably indexed.
Since we clearly have $F-M^+\subset G$ and every non-decreasing net with supremum is bounded and $\sigma$-strongly convergent, it suffices to show that $G$, or equivalently its intersection with the closed unit ball by the Krein-Sm\v ulian theorem, is $\sigma$-strongly closed.
Let $x_i\in G$ be a net such that $\sup_i\|x_i\|\le1$ and $x_i\to x$ $\sigma$-strongly.
For each $i$, take a sequence $x_{im}\in F-M^+$ such that $-2x_i\le x_{im}\uparrow x_i$ $\sigma$-strongly as $m\to\infty$, and also take $y_{im}\in F$ such that $x_{im}\le y_{im}$.
Since $\|x_{im}\|\le2\|x_i\|\le2$ is bounded, it implies that there is a bounded net $x_j$ in $F-M^+$ such that $x_j\to x$ $\sigma$-strongly, and we can choose arbitrarily small $\e>0$ such that $\sigma(x_j)\subset[-(2\e)^{-1},\infty)$ for all $j$.
Since $f_\e(x_j)$ converges to $f_\e(x)$ $\sigma$-strongly and $f_\e(y_j)$ is a bounded net for each $\e>0$ so that we may assume that the net $f_\e(y_j)$ is $\sigma$-weakly covergent by taking a subnet, we have $f_\e(x)\in F-M^+$ by
\[f_\e(x)=\lim_jf_\e(x_j)\le\lim_jf_\e(y_j)\in F,\]
where the limits are in the $\sigma$-weak sense.
By taking $\e$ as any decreasingly convergent sequence to zero, we have $x\in G$, hence the closedness of $G$.


(2)
It is enough to prove $(\overline{F_*-M_*^+})^+\subset F_*$, where the closure is for the weak topology or equivalently in norm by the convexity of $F_*-M_*^+$, so we begin our proof by fixing $\omega\in(\overline{F_*-M_*^+})^+$.
For a sequence $\omega_n\in F_*-M_*^+$ such that $\omega_n\to\omega$ in norm of $M_*$, we can take $\varphi_n\in F_*$ such that $\omega_n\le\varphi_n$ for all $n$.
By modifying $\omega_n$ into $\omega_n-(\omega_n-\omega)_+\in F_*-M_*^+$ and taking a rapidly convergent subsequence, we may assume $\omega_n\le\omega$ and $\|\omega-\omega_n\|\le2^{-n}$ for all $n$.
If we consider the Gelfand-Naimark-Segal representation $\pi:M\to B(H)$ associated to a positive normal linear functional \[\widehat\omega:=\sum_n(\omega-\omega_n)+\omega+\sum_n2^{-n}\left(\frac{[\omega_n]}{1+\|\omega_n\|}+\frac{\varphi_n}{1+\|\varphi_n\|}\right)\]
on $M$ with the canonical cyclic vector $\Omega$, we can construct commutant Radon-Nikodym derivatives $h,h_n,k_n\in\pi(M)'$ of $\omega,\omega_n,\varphi_n$ with respect to $\widehat\omega$ respectively.
Since $-1\le h_n\le h$ is bounded, $h_n\to h$ in the weak operator topology of $\pi(M)'$.
By the Mazur lemma, we can take a net $h_i$ by convex combinations of $h_n$ such that $h_i\to h$ strongly in $\pi(M)'$, and the corresponding linear functionals $\omega_i$ and $\varphi_i$ satisfy $\omega_i\le\varphi_i$ with $\varphi_i\in F_*$ by the convexity of $F_*$ so that $\omega_i\in F_*-M_*^+$.
The net $h_i$ can be taken to be a sequence in fact because $\pi(M)'$ is $\sigma$-finite by the existence of the separating vector $\Omega$, but it is not necessary in here.
For each $i$ and $0<\e<1$, define
\[h_\e:=f_\e(h),\quad h_{i,\e}:=f_\e(h_i),\quad k_{i,\e}:=f_\e(k_i)\]
in $\pi(M)'$, where the functional calculi are well-defined because $-1\le h_i$ and $0\le h,k_i$ for all $i$, and define $k_\e$ as the $\sigma$-weak limit of the bounded net $k_{i,\e}$, which may be assumed to be $\sigma$-weakly convergent.
Define $\omega_\e,\omega_{i,\e},\varphi_{i,\e},\varphi_\e$ as the corresponding normal linear functionals on $M$ to $h_\e,h_{i,\e},k_{i,\e},k_\e$.
Note that $\varphi_i\in F_*$.
The hereditarity of $F_*$ and $0\le\varphi_{i,\e}\le\varphi_i$ imply $\varphi_{i,\e}\in F_*$, and the weak closedness of $F_*$ and the weak convergence $\varphi_{i,\e}\to\varphi_\e$ in $M_*$ imply $\varphi_\e\in F^*$.
From $\omega_i\le\varphi_i$, we can deduce $0\le\omega_\e\le\varphi_\e$ by considering the operator monotonicity $f_\e$ and taking the weak limit on $i$.
Thus again, the hereditarity of $F_*$ implies $\omega_\e\in F^*$, and the weak closedness of $F_*$ and the weak convergence $\omega_\e\to\omega$ in $M_*$ imply $\omega\in F^*$.
\end{proof}

Now we prepare some lemmas for the positive Hahn-Banach separation theorem for C$^*$-algebras.


\begin{lem}\label{lower dominated sequence}
Let $A$ be a C$^*$-algebra, and let $G^*$ be a norm closed and downward closed convex subset of $A^{*sa}$.
If an element of $A^{*sa}$ is approximated weakly$^*$ by a lower dominated sequence of $G^*$, then it is approximated in norm by a sequence of $G^*$.
\end{lem}
\begin{proof}
Let $\omega_n\in F^*-A^{*+}$, $\varphi_n\in F^*$, $\widehat\omega_0\in A^{*+}$ be such that $\omega_n|_B\to\omega|_B$ weakly$^*$ in $B^*$ and $-\widehat\omega_0\le\omega_n\le\varphi_n$ for all $n$.
Consider the Gelfan-Naimark-Segal representation $\pi:A\to B(H)$ of
\[\widehat\omega:=\widehat\omega_0+[\omega]+\sum_n2^{-n}\left(\frac{[\omega_n]}{1+\|\omega_n\|}+\frac{\varphi_n}{1+\|\varphi_n\|}\right)\]
with the canonical cyclic vector $\Omega\in H$.
Let $\theta^*:\pi(A')\to A^*:y\mapsto(a\mapsto\langle y\pi(a)\Omega,\Omega\rangle)$.
Define the commutant Radon-Nikodym derivatives $h,h_n,k_n\in\pi(A)'$ of $\omega,\omega_n,\varphi_n$ with respect to $\widehat\omega$.
Note that $-1\le h\le1$, $-1\le h_n$, and $0\le k_n$, and the functional calculus $h_{n,\e}:=f_\e(h_n)$ and $k_{n,\e}:=f_\e(k_n)$ are well-defined in $\pi(A)'$ if $0<\e<1$.

\newpage

$\theta^*()\subset F^*-A^{*+}$

Let $h_{1n}:=h_n$ and $m\ge2$.
Define
\[S_{(m-1)n}:=\operatorname{conv}\{h_{(m-1)n,m^{-1}},h_{(m-1)(n+1),m^{-1}},\cdots\},\quad T_{(m-1)n}:=\operatorname{conv}\{h_{(m-1)n},h_{(m-1)(n+1)},\cdots\}.\]
If we choose any element $h_{m\infty,m^{-1}}\in\bigcap_n\overline{S_{(m-1)n}}^w$ from a non-empty set, then for each $n$ we can take $y_n\in S_{(m-1)n}$ such that
\[\|(y_n-h_{m\infty,m^{-1}})\Omega\|<\frac232^{-n},\qquad\|(y_n-h_{m\infty,m^{-1}})h\Omega\|<\frac232^{-n},\]
and there is $h_{mn}\in(T_{(m-1)n}-\pi(A)'^+)\cap\pi(A)'_{\ge-1}$ such that $h_{mn,m^{-1}}=y$, so $\theta^*(h_{mn})\in F^*-A^{*+}$ because the operator concavity of $f_{m^{-1}}$ implies
\[S_{(m-1)n}\subset f_{m^{-1}}((T_{(m-1)n}-\pi(A)'^+)\cap\pi(A)'_{\ge-1})\]
and we have
\[\theta^*(T_{mn}-\pi(A)'^+)\subset F^*-A^{*+}.\]
Note that $h_{mn}\in(T_{(m-1)n}-\pi(A)'^+)\cap\pi(A)'_{\ge-1}$.

like $T_{mn}\subset T_{(m-1)n}-\e_{mn}\pi(A)'^+$ to get $T_{mn}\subset T_{2n}-(\sum_m\e_{mn})\pi(A)'^+$?
we want $\lim_n\sum_m\e_{mn}=0$.
summability of $t_m^2/(t_{m-1}^{-1}-t_m^{-1}+t_m)$?

Since
\[\|(h_{(m-1)n,(m-1)^{-1}}-h_{(m-1)n',(m-1)^{-1}})\Omega\|<2^{-n},\qquad n'\ge n,\]
if we write
\[h_{mn,m^{-1}}=\sum_{k\ge n}\lambda_kh_{(m-1)k,m^{-1}}=\sum_{k\ge n}\lambda_kf_{-(m(m-1))^{-1}}(h_{(m-1)k,(m-1)^{-1}})\ge f_{-(m(m-1))^{-1}}(\sum_{k\ge n}\lambda_kh_{(m-1)k,(m-1)^{-1}})\]and
\[f_{(m-1)^{-1}}(h_{mn})=f_{(m(m-1))^{-1}}(h_{mn,m})\ge\sum_{k\ge n}\lambda_kh_{(m-1)k,(m-1)^{-1}}\]
\[h_{nn}\ge f_{(n-1)^{-1}}(\sum_{k\ge n}\lambda_kh_{(n-1)k,(n-1)^{-1}})\]

\[h_{nn}-h=(h_{nn}-h_{nn,m^{-1}})+(h_{nn,m^{-1}}-h_{,m^{-1}})-(h_{,m^{-1}}-h)\]







The bounded sequences $h_{n,\e}$ and $k_{n,\e}$ have weakly convergent subnets in $\pi(A)'$, and denote their limits by $h_\e$ and $k_\e$ respectively.
Be cautious that $h_\e':=f_\e(h)$ is not equal to $h_\e$ in general.
By the operator concavity of the function $f_\e$ and the $\sigma$-finiteness of $\pi(A)'$, the Mazur lemma retakes sequences $\omega_n\in F^*-A^{*+}$ and $\varphi_n\in F^*$ such that $\omega_n\le\varphi_n$ for all $n$ and $h_{n,\e}\to h_\e$ and $k_{n,\e}\to k_\e$ strongly.
We may assume
\[\|(h_{n,\e}-h_\e)\Omega\|<n^{-1},\qquad\|(h_{n,\e}-h_\e)h\Omega\|<n^{-1}\]
for all $n$ uniformly on $\e$, which will be used later.
Note also that we have the identity
\[(1+\e h)(h_\e'-h_{n,\e})(1+\e h)=(h-h_n)+\e(h-h_n)(1+\e h_n)^{-1}(h-h_n).\]


Denote by $\omega_{n,\e},\omega_\e,\omega_\e',\varphi_{n,\e},\varphi_\e$ the linear functionals in $A^{*sa}$ corresponded to operators in the commutant $h_{n,\e},h_\e,h_\e',k_{n,\e},k_\e\in\pi(A)'$.
It follows clearly that $\omega_{n,\e}\to\omega_\e$ and $\varphi_{n,\e}\to\varphi_\e$ as $n\to\infty$, and $\omega_\e'\uparrow\omega$ as $\e\to0$, all weakly in $A^*$.
If we prove $\omega_\e'|_B-\omega_\e|_B\to0$ weakly in $B^*$ as $\e\to0$, then since $\omega_{n,\e}\le\varphi_{n,\e}\in F^*$ implies $\omega_\e\le\varphi_\e\in F^*$, we obtain the weak convergence $\omega_\e|_B\to\omega|_B$ in $B^*$ as $\e\to0$ with $\omega_\e\in F^*-A^{*+}$.
A desired sequence by applying the Mazur lemma on $\omega_\e$ within $\e<(1-\delta)^{-1}\delta$, where $\delta<1$ is assumed.

Thus, what remains is to prove $\omega_\e'|_B-\omega_\e|_B\to0$ weakly in $B^*$ as $\e\to0$.
Consider the normal extension $\pi^{**}:A^{**}\to B(H)$ of the representation $\pi$.
For $x\in A^{**}$ with $\|x\|\le1$, the one-parameter family $(h_\e'-h_\e)\pi^{**}(x)\Omega$ of vectors is uniformly bounded on $0<\e\le\frac12$ by the uniform boundedness principle, because for each $\xi\in H$, fixing any $n$, say $n=1$, we have
\begin{align*}
&|\langle(h_\e'-h_\e)\pi^{**}(x)\Omega,\xi\rangle|\\
&\quad\le|\langle(h_\e'-h_{1,\e})\pi^{**}(x)\Omega,\xi\rangle|+|\langle(h_{1,\e}-h_\e)\pi^{**}(x)\Omega,\xi\rangle|\\
&\quad\le|\langle(1+\e h)^{-1}(h-h_1)(1+\e h)^{-1}\pi^{**}(x)\Omega,\xi\rangle|\\
&\qquad+\e|\langle(1+\e h)^{-1}(h-h_1)(1+\e h_1)^{-1}(h-h_1)(1+\e h)^{-1}\pi^{**}(x)\Omega,\xi\rangle|\\
&\qquad+\|(h_{1,\e}-h_\e)\Omega\|\|\pi^{**}(x^*)\xi\|\\
&\quad\le4\|h-h_1\|\|\Omega\|\|\xi\|+4\|h-h_1\|^2\|\Omega\|\|\xi\|+\|\xi\|,
\end{align*}
which is uniformly bounded on $\e$.
If we let $p\in B(H)$ the orthogonal projection on the closed linear subspace $\overline{\pi(B)\Omega}$, then for $y\in B^{**}$ with $\|y\|\le1$, we further have $p(h_\e'-h_\e)\pi^{**}(y)\Omega\to0$ weakly in $B(pH)$ as $\e\to0$, which can be shown as follows.
By the boundedness of $(h_\e'-h_\e)\pi^{**}(y)\Omega$, it is enough to choose $\pi(c)\Omega$ with $c\in B$ satisfying $\|c\|\le1$ for the test vector. 
As $\|(h_\e'-h_\e)\pi(c)\Omega\|$ is uniformly bounded on $\e$ because $c\in A^{**}$, we can also prove $\|(h_\e'-h_\e)h\pi(c)\Omega\|$ is uniformly bounded in the same manner but using $\|(h_{1,\e}-h_\e)h\Omega\|<1$ instead of $\|(h_{1,\e}-h_\e)\Omega\|<1$. 
Choose their common bound $C>0$.
For an arbitrarily fixed $\delta>0$, take $b\in B$ such that $\|(\pi^{**}(y)-\pi(b))\Omega\|<\delta C^{-1}$ and $\|b\|\le1$ by the Kaplansky density, and fix $n$ such that $|(\omega-\omega_n)(c^*b)|<\delta$ and $n>\frac94\|\Omega\|\delta^{-1}$.
Then,
\begin{align*}
&|\langle(h_\e'-h_\e)\pi^{**}(y)\Omega,\pi(c)\Omega\rangle|\\
&\quad<|\langle(h_\e'-h_\e)\pi(b)\Omega,\pi(c)\Omega\rangle|+\delta\\
&\quad<|\langle(h_\e'-h_\e)(1+\e h)\pi(b)\Omega,(1+\e h)\pi(c)\Omega\rangle|+O(\e)+\delta\\
&\quad<|\langle(h_\e'-h_{n,\e})(1+\e h)\pi(b)\Omega,(1+\e h)\pi(c)\Omega\rangle|+\delta+O(\e)+\delta\\
&\quad\le|(\omega-\omega_n)(c^*b)|+\e|\langle(1+\e h_n)^{-1}(h-h_n)\pi(b)\Omega,(h-h_n)\pi(c)\Omega\rangle|+\delta+O(\e)+\delta\\
&\quad<\delta+\e(1-\e)^{-1}\|(h-h_n)\|^2\|\Omega\|^2+\delta+O(\e)+\delta,
\end{align*}
where the asymptotic notation $O(\e)$ can be exactly computed as $C(2\e+\e^2)\|\Omega\|$, so we have
\[\limsup_{\e\to0}|\langle(h_\e'-h_\e)\pi^{**}(y)\Omega,\pi(c)\Omega\rangle|\le3\delta.\]
Since $\delta>0$ was taken arbitrarily, we finally have $p(h_\e'-h_\e)\pi^{**}(y)\Omega\to0$ weakly in $pH$, which implies $\omega_\e'|_B-\omega_\e|_B\to0$ weakly in $B^*$ as $\e\to0$.
\end{proof}


The following lemma is a modification of the Krein-\v Smulian theorem, and it can be proved in a similar way to the proof of the original theorem.

\begin{lem}\label{krein-smulian}
Let $A$ be a C$^*$-algebra, and $C^*_n$ be a non-decreasing sequence of weakly$^*$-closed convex subsets of $A^{*sa}$, whose union $C_\infty^*$ contains $A^{*+}$.
If a norm closed convex subset $G^*$ of $A^{*sa}$ has the property that $G^*\cap C^*_n$ is weakly$^*$ closed for each $n$, then $G^*\cap C_\infty^*$ is relatively weakly$^*$ closed in $C_\infty^*$.
\end{lem}
\begin{proof}
Fix an element $\omega_0$ of $C_\infty^*\setminus G^*$.
It is enough to construct an element $a$ of $A^{sa}$ separating a norm open ball centered at $\omega_0$ from $G^*$.
Since $G^*$ is norm closed, there exists $r>0$ such that $G^*\cap B(\omega_0,r)=\varnothing$.
By replacing $G^*$ to $r^{-1}(G^*-\omega_0)$ and $C_n^*$ to $r^{-1}(C_n^*-\omega_0)$, we may assume $G^*\cap B(0,1)=\varnothing$, and the claim follows if we prove there is $a\in A^{sa}$ separating $B(0,1)$ and $G^*$.
The condition $A^{*+}\subset C_\infty^*$ becomes $A^{*+}-\omega_0\subset C_\infty^*$.
Letting the index $n$ start from one, we may also replace $C_n^*$ to $n(C_n^*\cap B(0,1))$ since its union is still $C_\infty^*$.
Note that $C_n^*$ is bounded for each $n$, and we can easily see that $G^*\cap C_1^*=\varnothing$ and $n^{-1}C_n^*\subset(n+1)^{-1}C_{n+1}^*$.

Note that for any Banach space $X$, if $F$ is a bounded subset of $X$, then by endowing with the discrete topology on $F$, we have a natural bounded linear operator $\ell^1(F)\to X$ by completeness of $X$, with its dual $X^*\to\ell^\infty(F)$.
We will construct a bounded subset $F$ of $A^{sa}$ such that the subset $G^*\cap C_\infty^*$ of $A^{*sa}$ induces a subset of the smaller subspace $c_0(F)$ of $\ell^\infty(F)$ via the restriction map $A^{*sa}\to\ell^\infty(F)$, and also such that it satisfies $G^*\cap C_\infty^*\cap F^\circ=\varnothing$, where $F^\circ:=\{\omega\in A^{*sa}:\sup_{a\in F}|\omega(a)|\le1\}$ denotes the absolute polar of $F$.
If such a set $F\subset X$ exists, then the image of $G^*\cap C_\infty^*$ in $c_0(F)$ is a convex set disjoint to the closed unit ball of $c_0(F)$ by the condition $G^*\cap C_\infty^*\cap F^\circ=\varnothing$.
Therefore, there exists a separating linear functional $l\in\ell^1(F)$ by the Hahn-Banach separation, and it induces a linear functional separating $G^*$ and the unit ball of $A^{*sa}$.
Then, we are done.

Let $F_0:=\{0\}\subset A^{sa}$.
As an induction hypothesis on $n$, suppose for each $0\le k\le n-1$ we already have a finite subset $F_k$ of $(C_k^*)^\circ$ such that
\[G^*\cap C^*_n\cap\left(\bigcup_{k=0}^{n-1}F_k\right)^\circ=\varnothing.\]
If every finite subset $F_n$ of $(C_n^*)^\circ$ satisfies
\[G^*\cap C_{n+1}^*\cap\left(\bigcup_{k=0}^{n-1}F_k\right)^\circ\cap F_n^\circ\ne\varnothing,\]
then since they are weakly$^*$ compact, the finite intersection property leads a contradiction because the intersection of all absolute polars $F_n^\circ$ of finite subsets $F_n$ of $(C_n^*)^\circ$ is $C_n^*$, which is the polar of all union of finite subsets $F_n$ of $(C_n^*)^\circ$ by the bipolar theorem.
Thus, we can take a finite subset $F_n$ of $(C_n^*)^\circ$ such that
\[G^*\cap C_{n+1}^*\cap\left(\bigcup_{k=0}^nF_k\right)^\circ=\varnothing.\]
Let $F:=\bigcup_{k=0}^\infty F_k$.
Then, we have $G^*\cap C_\infty^*\cap F^\circ=\varnothing$, and every element of $C_\infty^*$ is restricted to $F$ to define an element of $c_0(F)$ because for each $\omega\in C_n^*$ and $k\ge0$ we have 
\[\omega(F_{n+k})\subset\omega((C_{n+k}^*)^\circ)\subset\frac n{n+k}\omega((C_n^*)^\circ)\subset[-\frac n{n+k},\frac n{n+k}].\]
Finally, for any $\omega\in A^{*sa}$, if we enumerate $F$ as a sequence $f_m$, then
\[|\omega(f_m)|\le|(\omega_+-\omega_0)(f_m)|+|(\omega_--\omega_0)(f_m)|\to0,\]
so the uniform boundedness principle concludes that $F$ is bounded.
Therefore, the set $F$ satisfies the properties we desired.
\end{proof}





\begin{thm}[Positive Hahn-Banach separation for C$^*$-algebras]
Let $A$ be a C$^*$-algebra.
\begin{enumerate}
\item If $F$ is a norm closed convex hereditary subset of $A^+$, then $F=F^{r+r+}$. In particular, if $a'\in A^+\setminus F$, then there is $\omega\in A^{*+}$ such that $\omega(a')>1$ and $\omega(a)\le1$ for $a\in F$.
\item If $F^*$ is a weakly$^*$ closed convex hereditary subset of $A^{*+}$, then $F^*=(F^*)^{r+r+}$. In particular, if $\omega'\in A^{*+}\setminus F^*$, then there is $a\in A^+$ such that $\omega'(a)>1$ and $\omega(a)\le1$ for $\omega\in F^*$.
\end{enumerate}
\end{thm}
\begin{proof}
(1)
We directly prove the separation without invoking the arguments of positive bipolars.
Denote by $F^{**}$ the $\sigma$-weak closure of $F$ in the universal von Neumann algebra $A^{**}$.
We first show that $F^{**}$ is hereditary subset of $A^{**+}$.
Suppose $0\le x\le y$ in $A^{**}$ and $y\in F^{**}$.
Then, there is $z\in A^{**}$ such that $x^{\frac12}=zy^{\frac12}$.
Take bounded nets $b_i$ in $F$ and $c_i$ in $A$ such that $b_i\to y$ and $c_i\to z$ $\sigma$-strongly$^*$ in $A^{**}$ using the Kaplansky density.
We may assume the indices of these two nets are same.
Since both the multiplication and the involution of a von Neumann algebra on bounded parts are continuous in the $\sigma$-strong$^*$ topology, and since the square root on a positive bounded interval is a strongly continuous function, we have the $\sigma$-strong$^*$ limit
\[x=y^{\frac12}z^*zy^{\frac12}=\lim_ib_i^{\frac12}c_i^*c_ib_i^{\frac12},\]
so we obtain $x\in F^{**}$ from $b_i^{\frac12}c_i^*c_ib_i^{\frac12}\in F$.
Thus, $F^{**}$ is hereditary in $A^{**+}$.

Let $a\in A^+\setminus F$.
Observe that we have $a\in A^{**+}\setminus F^{**}$ because if $a\in F^{**}$, then we have a net $a_i$ in $F$ such that $a_i\to a$ $\sigma$-weakly in $A^{**}$, meaning that $a_i\to a$ weakly in $A$ and by the weak closedness of $F$ in $A$ we get a contradiction $a\in F^{**}\cap A=F$.
By Theorem \ref{positive hahn-banach w*}, there is $\omega\in A^{*+}$ such that $\omega(a)>1$ and $\omega\le1$ on $F\subset F^{**}$, so it completes the proof.

(2)
As same as above, our goal is to prove $(\overline{F^*-A^{*+}})^+\subset F^*$, where the closure is always for the weak$^*$ topology through the proof.
We first prove it when $A$ is commutative.
On a commutative C$^*$-algebra, the rectifier function $\mathbb{R}\to\mathbb{R}:t\mapsto\max\{0,t\}$ is operator monotone.
Define
\[G^*:=\left\{\omega\in\overline{F^*-A^{*+}}:\begin{tabular}{c}there is a bounded net $\omega_j\in F^*-A^{*+}$\\such that $\omega_j\to\omega$ weakly$^*$ in $A^*$\end{tabular}\right\}.\]
We can easily check $F^*-A^{*+}\subset G^*$ by considering constant sequences.
Take a bounded net $\omega_i\in G^*$ in the spirit of the Krein-\v Smulian theorem such that $\omega_i\to\omega$ weakly$^*$ in $A^*$.
Then, for each $i$ we have a bounded net $\omega_{ij}\in F^*-A^{*+}$ and a net $\varphi_{ij}\in F^*$ such that $\omega_{ij}\le\varphi_{ij}$ for all $j$ and $\omega_{ij}\to\omega_i$ weakly$^*$ in $A^*$ by definition of $G^*$.
Since $0\le\omega_{ij+}\le\varphi_{ij}\in F^*$ implies $\omega_{ij+}\in F^*$ and since it is bounded for each $i$ so that we may assume $\omega_{ij+}\to\omega_i'$ weakly$^*$ in $A^*$, we have $\omega_i\le\omega_i'\in F^*$ by the weak$^*$ closedness of $F^*$.
Since $0\le\omega_{i+}\le\omega_i'\in F^*$ implies $\omega_{i+}\in F^*$ and since it is bounded so that we may assume $\omega_{i+}\to\omega'$ weakly$^*$ in $A^*$, we have $\omega\le\omega'\in F^*$.
It implies that $\omega\in F^*-A^{*+}\subset G^*$ and $G^*$ is weakly$^*$ closed, so $G^*=\overline{F^*-A^{*+}}$.
Therefore, if $\omega\in(\overline{F^*-A^{*+}})^+$, then there is a bounded net $\omega_i\in F^*-A^{*+}$ and a net $\varphi_i\in F^*$ such that $\omega_i\le\varphi_i$ for all $i$ and $\omega_i\to\omega$ weakly$^*$ in $A^*$, so since $0\le\omega_{i+}\le\varphi_i\in F^*$ implies $\omega_{i+}\in F^*$, and since it is bounded so that we may assume $\omega_{i+}\to\omega'$ weakly$^*$ in $A^*$, we have $\omega\le\omega'\in F^*$, which gives $\omega\in F^*$.
This completes the proof of $(\overline{F^*-A^{*+}})^+\subset F^*$ provided that $A$ is commutative.
\[-\]

Now we consider a general C$^*$-algebra $A$.
For a separable C$^*$-subalgebra $B$ of $A$, define
\[F_B^*:=\overline{\{\omega_B\in B^{*+}:\text{there is $\omega\in F^*-A^{*+}$ such that $\omega|_B=\omega_B$}\}}^{\|\cdot\|},\]
which is clearly a norm closed and convex, and we can see that it is hereditary in $B^{*+}$ by the positive Hahn-Banach extension.
We claim $(\overline{F_B^*-B^{*+}})^+\subset F_B^*$.
As a corollary, $F_B^*$ becomes weakly$^*$ closed, and if $A$ is separable itself, then the proof of the theorem follows by letting $B=A$.
Note that the separability of $B$ makes the weak$^*$ topology on any bounded part of $B^{*sa}$ metrizable.
Consider the following convex set
\[G_B^*:=\left\{\omega_B\in\overline{F_B^*-B^{*+}}:\begin{tabular}{c}there is a lower dominated sequence $\omega_n\in F^*-A^{*+}$\\such that $\omega_n|_B\to\omega_B$ weakly$^*$ in $B^*$\end{tabular}\right\}.\]
\[G_B^*:=\overline{F_B-B^{*+}}^{\|\cdot\|}\]
We can see that $F_B^*-B^{*+}\subset G_B^*$ by suitably taking the Hahn-Banach extension for the constant sequence of each element of $F^*-A^{*+}$, and it implies $(\overline{F_B^*-B^{*+}})^+\subset(\overline{G_B^*})^+$.
We also have $G_B^{*+}\subset F_B^*$ by Theorem \ref{positive hahn-banach w*} (2), so if we prove $G_B^*$ is weakly$^*$ closed, then the claim $(\overline{F_B^*-B^{*+}})^+\subset F_B^*$ follows.

First we prove $G_B^*$ is norm closed in $B^*$.
Let $\omega_{B,n}\in G_B^*$ be a sequence such that $\omega_{B,n}\le\omega_B$ and $\|\omega_{B,n}-\omega_B\|<2^{-n}$.
There is also a lower dominated sequence $\omega_{nm}\in F^*-A^{*+}$ for each $n$ such that $\omega_{nm}|_B\le\omega_{B,n}$ and $\|\omega_{B,n}-\omega_{nm}|_B\|<2^{-n-m}$ for all $m$ by definition of $G_B^*$ and by Lemma \ref{lower dominated sequence}.
Then,
\[-(\omega_B)_--\sum_n(\omega_B-\omega_{B,n})-\sum_{n,m}(\omega_{B,n}-\omega_{nm}|_B)\le\omega_{nn}|_B.\]


To prove $G_B^*$ is weakly$^*$ closed, we can take a sequence $\omega_{B,n}\in G_B^*$ such that $\omega_{B,n}\to\omega_B$ weakly$^*$ in $B^*$ by the Krein-\v Smulian theorem and the separability of $B$.
Since $G_B^*$ is norm closed and $\omega_B$ belongs to the relative weak$^*$ closure of $G_B^*\cap C_\infty^*$ in $C_\infty^*$, where
\[C_n^*:=\{\omega_B'\in B^{*sa}:-\sum_{k\le n}(\omega_{B,k})_--(\omega_B)_-\le\omega_B'\},\qquad C_\infty^*:=\bigcup_nC_n^*,\]
so if we only check $G^*_B\cap C_n^*$ is weakly$^*$ closed in $B^*$ for each $n$, then we obtain $\omega_B\in G^*_B$ by Lemma \ref{krein-smulian}, which implies the weak$^*$ closedness of $G_B^*$.
Because every sequence in $C_n^*$ is lower dominated, now it suffices to show $\omega_B\in G_B^*$ when it is the weak$^*$ limit of a lower dominated sequence $\omega_{B,n}\in G_B^*$.
By Lemma \ref{lower dominated sequence}, we may assume $\omega_{B,n}\to\omega_B$ in norm.
There is also for each $n$ a sequence $\omega_{nm}\in F^*-A^{*+}$ such that $\omega_{nm}|_B\to\omega_{B,n}$ in norm.
Observing there was the same situation in the proof of the norm closedness of $G_B^*$, we can conclude $\omega_B\in G_B^*$ in the same way, hence $G_B^*$ is weakly$^*$ closed.
\[-\]

So far, we just proved $(\overline{F_B^*-B^{*+}})^+\subset F_B^*$.
Now let $\omega\in(\overline{F^*-A^{*+}})^+$.
Take a net $\omega_i\in F^*-A^{*+}$ and $\varphi_i\in F^*$ such that $\omega_i\to\omega$ weakly$^*$ in $A^*$ and $\omega_i\le\varphi_i$ for each $i$.
For each separable C$^*$-subalgebra $B$ of $A$, we have $\varphi_i|_B\in F^*_{B}$ and $\omega_i|_B\in F^*_B-B^{*+}$ with the weak$^*$ convergence $\omega_i|_B\to\omega|_B$ in $B^*$, thus we have $\omega|_B\in(\overline{F_B^*-B^{*+}})^+=F_B^*$ because $B$ is separable.
If we consider the increasing net of all separable C$^*$-subalgebras $(B_j)_{j\in J}$ of $A$, then we have $\omega|_{B_j}\in F_{B_j}^*$ so that there is a net $\omega_{(j,\e)}\in F^*-A^{*+}$ based on the product directed set $\{(j,\e):j\in J,\ \e>0\}$ such that $\|\omega_{(j,\e)}|_{B_j}-\omega|_{B_j}\|<\e$ for each $(j,\e)$.

With this net, as an intermediate step, we prove that $\omega$ belongs to the $\sigma(A^*,A_0^{**})$-closure of $F^*-A^{*+}$, where $A_0^{**}$ denotes the set of all elements of $A^{**}$ whose left or right support projection is $\sigma$-finite.
Let $x\in A_0^{**+}$ with $\|x\|\le1$, and let $p$ be the support projection of $x$.
Then, $p$ is $\sigma$-finite and we can take a bounded sequence $a_n\in A^+$ such that $\|a_n\|\le1$ for all $n$ and $a_np\to x$ $\sigma$-strongly in $A^{**}$ by the Kaplansky density theorem and the Mazur lemma, which is sequential because $A^{**}p$ is $\sigma$-weakly closed and its bounded part is $\sigma$-strongly metrizable.
Let $p_n$ be the support projection of $a_n$ for each $n$.
Since $a_m^{1/k}\to p_m$ and $x^{1/k}\to p$ $\sigma$-strongly as $k\to\infty$ for each $m$, we have $p_m\to p$ $\sigma$-strongly as $m\to\infty$.
If we choose $j_0$ such that $a_n\in B_{j_0}$ for all $n$, then for any $j\succ j_0$, by taking limits $k\to\infty$, $m\to\infty$, and $n\to\infty$ in order on the inequality
\begin{align*}
|(\omega_{(j,\e)}-\omega)(x)|
&\le|(\omega_{(j,\e)}-\omega)(x-a_np)|+|(\omega_{(j,\e)}-\omega)(a_n(p-p_m))|\\
&\quad+|(\omega_{(j,\e)}-\omega)(a_n(p_m-a_m^{1/k}))|+|(\omega_{(j,\e)}-\omega)(a_na_m^{1/k})|,
\end{align*}
since the final term is uniformly estimated up to $\e$ because $a_na_m^{1/k}\in B_j$ is uniformly bounded by one, we obtain $\lim_{(j,\e)}(\omega_{(j,\e)}-\omega)(x)=0$.
This proves that $\omega$ is contained in the $\sigma(A^*,A_0^{**})$-closure of $F^*-A^{*+}$.



Suppose now $\omega\notin F^*$.
Then, there exists $x\in A^{**+}$ such that $\omega(x^2)>1$ and $\omega'(x^2)\le1$ for all $\omega'\in F^*$ by Theorem \ref{positive hahn-banach w*} (2).
Let $\{p_i\}_{i\in I}$ be a maximal orthogonal family of $\sigma$-finite projections of the von Neumann algebra $A^{**}$ whose sum is the support projection of $x$.
If we consider order-preserving bounded linear maps $\Gamma:c_0(I)\to A^{**}$ and $\Gamma^*:A^*\to\ell^1(I)$ given by
\[\Gamma((c_i)_{i\in I}):=\sum_ic_ixp_ix,\qquad
\Gamma^*(\omega'):=(\omega'(xp_ix))_{i\in I},\]
then these maps are in dual, and $\Gamma$ is extended to the linear map $\Gamma^{**}:\ell^\infty(I)\to A^{**}$ continuous with respect to weak$^*$ topologies.
Observing that the left and right support projections of an arbitrary element of a von Neumann algebra are Murray-von Neumann equivalent, we can see $A_0^{**}$ is an algebraic ideal of $A^{**}$, and we have $\Gamma(c_0(I))\subset A_0^{**}$ due to the fact that each element of $c_0(I)$ has at most countably many non-zero components.
Since $\omega$ is an element of the $\sigma(A^*,A_0^{**})$-closure of $F^*-A^{*+}$, we have $\Gamma^*(\omega)\in \overline{\Gamma^*(F^*-A^{*+})}$, where the closure is taken in the weak$^*$ topology of $\ell^1(I)$.
Since the set $\overline{\Gamma^*(F^*-A^{*+})}$ is contained in the weak$^*$ closure of $\overline{\Gamma^*(F^*)}-\ell^1(I)^+$ by $\Gamma^*(F^*-A^{*+})\subset\Gamma^*(F^*)-\ell^1(I)^+$, whose positive part is $\overline{\Gamma^*(F^*)}$ because $c_0(I)$ is a commutative C$^*$-algebra.
Therefore, we have $\Gamma^*(\omega)\in\overline{\Gamma^*(F^*)}$.
For any $\delta>0$, if we choose $c\in c_0(I)^+$ such that $c\le1$ and $|\langle1_{\ell^\infty(I)}-c,\Gamma^*(\omega)\rangle|<\delta$ using the Kaplansky density, and choose $\omega'\in F^*$ such that $|\langle c,\Gamma^*(\omega)-\Gamma^*(\omega')\rangle|<\delta$, then we have a contradiction
\begin{align*}
1<\omega(x^2)&=\langle1_{\ell^\infty(I)},\Gamma^*(\omega)\rangle\approx_\delta\langle c,\Gamma^*(\omega)\rangle\\
&\approx_\delta\langle c,\Gamma^*(\omega')\rangle\le\langle1_{\ell^\infty(I)},\Gamma^*(\omega')\rangle=\omega'(x^2)\le1,
\end{align*}
where the relation symbol $\approx_\delta$ means that the difference converges to zero as $\delta\to0$, so finally we have $\omega\in F^*$.
\end{proof}




\section{Applications to weight theory}


The positive Hahn-Banach separation theorem implies a generalization of the Combes theorem on subadditive normal weights.
\begin{cor}
Let $M$ be a von Neumann algebra.
Then, there is a one-to-one correspondence
\[\begin{array}{ccc}
\left\{\emph{\begin{tabular}{c}subadditive normal\\weights of $M$\end{tabular}}\right\}&\leftrightarrow&\left\{\emph{\begin{tabular}{c}hereditary closed\\convex subsets of $M_*^+$\end{tabular}}\right\}\\[10pt]
\varphi&\mapsto&\{\omega\in M_*^+:\omega\le\varphi\}
\end{array}\]
\end{cor}



\end{document}