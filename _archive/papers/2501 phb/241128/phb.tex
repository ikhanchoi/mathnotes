\documentclass[a4paper]{amsart}

\usepackage[T1]{fontenc}
\usepackage[bitstream-charter,cal]{mathdesign}
\linespread{1.15}

\newcommand{\e}{\varepsilon}

\theoremstyle{plain}
\newtheorem{thm}{Theorem}[section]
\newtheorem{lem}[thm]{Lemma}
\newtheorem{cor}[thm]{Corollary}
\theoremstyle{definition}
\newtheorem{defn}[thm]{Definition}


\title{Positive Hahn-Banach separations in operator algebras}
\author{Ikhan Choi}
\address{}
\subjclass[2020]{}

\begin{document}

\begin{abstract}

\end{abstract}

\maketitle

\section{Introduction}






\begin{itemize}
\item definition and properties of $f_\e(t):=(1+\e t)^{-1}t$
\item relation between $\{\omega'\in M_*^+:\omega'\le\omega\}$ and $\{h\in\pi(M)'^+:h\le1\}$
\end{itemize}




\section{}



\begin{defn}[Hereditary subsets]
Let $E$ be a partially ordered real vector space.
We say a subset $F$ of the positive cone $E^+$ is \emph{hereditary} if $0\le x\le y$ in $E$ and $y\in F$ imply $x\in F$, or equivalently $F=(F-E^+)^+$, where $F-E^+$ is the set of all positive elements of $E$ bounded above by an element of $F$.
A $*$-subalgebra $B$ of a $*$-algbera $A$ is called \emph{hereditary} if and only if the positive cone $B^+$ is a hereditary subset of $A^+$.
We define the \emph{positive polar} of $F$ as the positive part of the real polar
\[F^{\circ+}:=\{x^*\in(E^*)^+:\sup_{x\in F}x^*(x)\le1\}.\]
\end{defn}
An example that is a non-hereditary closed convex subset of a C$^*$-algebra is $\mathbb{C}1$ in any unital C$^*$-algebra.


\begin{thm}[Positive Hahn-Banach separation for von Neumann algebras]
Let $M$ be a von Neumann algebra.
\begin{enumerate}
\item If $F$ is a $\sigma$-weakly closed convex hereditary subset of $M^+$, then $F=F^{\circ+\circ+}$. In particular, if $x\in M^+\setminus F$, then there is $\omega\in M_*^+$ such that $\omega(x)>1$ and $\omega\le1$ on $F$.
\item If $F_*$ is a weakly closed convex hereditary subset of $M_*^+$, then $F_*=F_*^{\circ+\circ+}$. In particular, if $\omega\in M_*^+\setminus F_*$, then there is $x\in M^+$ such that $\omega(x)>1$ and $x\le1$ on $F_*$.
\end{enumerate}
\end{thm}
\begin{proof}
(1)
Since the positive polar is represented as the real polar
\[F^{\circ+}=F^\circ\cap M_*^+=F^\circ\cap(-M^+)^\circ=(F\cup-M^+)^\circ=(F-M^+)^\circ,\]
the positive bipolar can be written as $F^{\circ+\circ+}=(F-M^+)^{\circ\circ+}=\overline{F-M^+}^+$ by the usual bipolar theorem.
Because $F=(F-M^+)^+\subset(\overline{F-M^+})^+$, it suffices to prove the opposite inclusion $(\overline{F-M^+})^+\subset F$.

Let $x\in\overline{F-M^+}^+$.
Take a net $x_i\in F-M^+$ such that $x_i\to x$ $\sigma$-strongly, and take a net $y_i\in F$ such that $x_i\le y_i$ for each $i$.
Suppose we may assume that the net $x_i$ is bounded.
Define strongly continuous functions $f_\e:[-(2\e)^{-1},\infty)\to\mathbb{R}:z\mapsto z(1+\e z)^{-1}$ parametrized by $\e>0$.
Then, for sufficiently small $\e$ so that the bounded net $x_i$ has the spectra in $[-(2\e)^{-1},\infty)$, we have $f_\e(x_i)\to f_\e(x)$ $\sigma$-strongly, and hence $\sigma$-weakly.
On the other hand, by the hereditarity and the $\sigma$-weak compactness of $F$, we may asumme that the bounded net $f_\e(y_i)\in F$ converges $\sigma$-weakly to a point of $F$ by taking a subnet.
Then, we have $f_\e(x)\in F-M^+$ by
\[0\le f_\e(x)=\lim_if_\e(x_i)\le\lim_if_\e(y_i)\in F,\]
thus we have $x\in F$ since $f_\e(x)\uparrow x$ as $\e\to0$.
What remains is to prove the existence of a bounded net $x_i\in F-M^+$ such that $x_i\to x$ $\sigma$-strongly.

Define a convex set
\[G:=\{x\in\overline{F-M^+}:\exists\,x_m\in F-M^+,\ -2x\le x_m\uparrow x\}\subset M^{sa},\]
where $x_m$ denotes a sequence.
In fact, it has no critical issue for allowing $x_m$ to be uncountably indexed.
Since we clearly have $F-M^+\subset G$ and every non-decreasing net with supremum is bounded and $\sigma$-strongly convergent, it suffices to show that $G$, or equivalently its intersection with the closed unit ball by the Krein-Sm\v ulian theorem, is $\sigma$-strongly closed.
Let $x_i\in G$ be a net such that $\sup_i\|x_i\|\le1$ and $x_i\to x$ $\sigma$-strongly.
For each $i$, take a sequence $x_{im}\in F-M^+$ such that $-2x_i\le x_{im}\uparrow x_i$ $\sigma$-strongly as $m\to\infty$, and also take $y_{im}\in F$ such that $x_{im}\le y_{im}$.
Since $\|x_{im}\|\le2\|x_i\|\le2$ is bounded, it implies that there is a bounded net $x_j$ in $F-M^+$ such that $x_j\to x$ $\sigma$-strongly, and we can choose arbitrarily small $\e>0$ such that $\sigma(x_j)\subset[-(2\e)^{-1},\infty)$ for all $j$.
Since $f_\e(x_j)$ converges to $f_\e(x)$ $\sigma$-strongly and $f_\e(y_j)$ is a bounded net for each $\e>0$ so that we may assume that the net $f_\e(y_j)$ is $\sigma$-weakly covergent by taking a subnet, we have $f_\e(x)\in F-M^+$ by
\[f_\e(x)=\lim_jf_\e(x_j)\le\lim_jf_\e(y_j)\in F,\]
where the limits are in the $\sigma$-weak sense.
By taking $\e$ as any decreasingly convergent sequence to zero, we have $x\in G$, hence the closedness of $G$.


(2)
It suffices to prove $(\overline{F_*-M_*^+})^+\subset F_*$, so we begin our proof by fixing $\omega\in(\overline{F_*-M_*^+})^+$.
Since the norm closure and the weak closure of the convex set $F_*-M_*$ coincide, we have a sequence $\omega_n\in F_*-M_*^+$ such that $\omega_n\to\omega$ in norm of $M_*$, and we can take $\varphi_n\in F_*$ such that $\omega_n\le\varphi_n$ for all $n$.
By modifying $\omega_n$ into $\omega_n-(\omega_n-\omega)_+\in F_*-M_*^+$ and taking a rapidly convergent subsequence, we may assume $\omega_n\le\omega$ and $\|\omega-\omega_n\|\le2^{-n}$ for all $n$.
If we consider the Gelfand-Naimark-Segal representation $\pi:M\to B(H)$ associated to a positive normal linear functional $\omega+\sum_n(\omega-\omega_n)$ on $M$ with the canonical cyclic vector $\Omega$, we can construct commutant Radon-Nikodym derivatives $h,h_n,k_n\in\pi(M)'$ of $\omega,\omega_n,\varphi_n$ respectively.
Since $-1\le h_n\le h$ is bounded, $h_n\to h$ in the weak operator topology of $\pi(M)'$.
By the Mazur lemma, we can take a net $h_i$ by convex combinations of $h_n$ such that $h_i\to h$ strongly in $\pi(M)'$, and the corresponding linear functionals $\omega_i$ and $\varphi_i$ satisfy $\omega_i\le\varphi_i$ with $\varphi_i\in F_*$ by the convexity of $F_*$ so that $\omega_i\in F_*-M_*^+$.
The net $h_i$ can be taken to be a sequence in fact because $\pi(M)'$ is $\sigma$-finite, but it is not necessary.
For each $i$ and $0<\e<1$, define
\[h_\e:=f_\e(h),\quad h_{i,\e}:=f_\e(h_i),\quad k_{i,\e}:=f_\e(k_i)\]
in $\pi(M)'$, where the functional calculus $f_\e(h_i)$ can be defined because $h_i\ge-1$ for all $i$, and define $k_\e$ as the $\sigma$-weak limit of the bounded net $f_\e(k_i)$, which may be assumed to be $\sigma$-weakly convergent.
Define $\omega_\e,\omega_{i,\e},\varphi_{i,\e}$, and $\varphi_\e$ as the corresponding normal linear functionals on $M$ to $h_\e,h_{i,\e},k_{i,\e}$, and $k_\e$.
Note that $\varphi_i\in F_*$.
The hereditarity of $F_*$ and $0\le\varphi_{i,\e}\le\varphi_i$ imply $\varphi_{i,\e}\in F_*$, and the weak closedness of $F_*$ and the weak convergence $\varphi_{i,\e}\to\varphi_\e$ in $M_*$ imply $\varphi_\e\in F^*$.
From $\omega_i\le\varphi_i$, we can deduce $0\le\omega_\e\le\varphi_\e$ by considering the operator monotonicity $f_\e$ and taking the weak limit on $i$.
Thus again, the hereditarity of $F_*$ implies $\omega_\e\in F^*$, and the weak closedness of $F_*$ and the weak convergence $\omega_\e\to\omega$ in $M_*$ imply $\omega\in F^*$.
\end{proof}



\section{}


\begin{lem}
Let $A$ be a C$^*$-algebra, and let $F^*$ be a weakly$^*$ closed convex hereditary subset of $A^{*+}$.
If $\omega_i\in F^*-A^{*+}$ is a net such that $\omega_i\to\omega$ weakly$^*$ in $A^*$, and if there is $\widetilde\omega\in A^{*+}$ such that $-\widetilde\omega\le\omega_i$ for all $i$, then for any $\delta>0$ there is a sequence $\omega_n\in F^*-A^{*+}$ such that $\omega-\delta\widetilde\omega\le\omega_n\uparrow\omega$ weakly in $A^*$.
\end{lem}
\begin{proof}
Let $\varphi_i\in F^*$ be a net such that $\omega_i\le\varphi_i$ for all $i$.
Consider the Gelfan-Naimark-Segal representation $\pi:A\to B(H)$ of $\widetilde\omega$ with the canonical cyclic vector $\Omega\in H$.
Since $0\le\widetilde\omega+\omega$, the Friedrichs extension theorem defines a positive self-adjoint operator $\widetilde h$ affiliated with $\pi(A)'$ such that $\pi(A)\Omega\subset\operatorname{dom}\widetilde h$ and $(\widetilde\omega+\omega)(a)=\langle\widetilde h\pi(a)\Omega,\Omega\rangle$ for all $a\in A$, and we can define the commutant Radon-Nikodym derivative of $\omega$ with respect to $\widetilde\omega$ by $h:=\widetilde h-1$, which satisfies $\omega(a)=\langle h\pi(a)\Omega,\Omega\rangle$ for all $a\in A$.
Similarly, we can define the commutant Radon-Nikodym derivatives $h_i,k_i$ of $\omega_i,\varphi_i$ with respect to $\widetilde\omega$, by the Friedrichs extension theorem.

Fix $0<\e<1$.
Since $-1\le h_i$ and $0\le k_i$, the functional calculus $h_{i,\e}:=f_\e(h_i)$ and $k_{i,\e}:=f_\e(k_i)$ are well-defined.
Taking a subnet, we may assume $h_{i,\e}$ and $k_{i,\e}$ are $\sigma$-weakly convergent in $\pi(A)'$, and denote by $h_\e$ and $k_\e$ the limits.
By the operator concavity of $f_\e$, we may assume $h_{i,\e}\to h_\e$ and $k_{i,\e}\to k_\e$ $\sigma$-stronlgy.
(It is true, but I will write more details later)
Since
\[f_\e(h)-f_\e(h_i)=\e^{-1}((1+\e h_i)^{-1}-(1+\e h)^{-1})=(1+\e h_i)^{-1}(h-h_i)(1+\e h)^{-1},\]
and since $(1+\e h)\pi(A)\Omega$ is dense in $H$ because $\pi(A)\Omega$ is a core(?) of $h$ and $1+\e h$ is surjective, we have $f_\e(h)=h_\e$ from
\begin{align*}
&\|(f_\e(h)-h_\e)(1+\e h)\pi(a)\Omega\|\\
&\quad\le\|(f_\e(h)-f_\e(h_i))(1+\e h)\pi(a)\Omega\|
+\|(f_\e(h_i)-h_\e)(1+\e h)\pi(a)\Omega\|\\
&\quad\le\|(1+\e h_i)^{-1}\|\|(h-h_i)\pi(a)\Omega\|
+\|(h_{i,\e}-h_\e)(1+\e h)\pi(a)\Omega\|\\
&\quad\le(1-\e)^{-1}\|(h-h_i)\pi(a)\Omega\|
+\|(h_{i,\e}-h_\e)(1+\e h)\pi(a)\Omega\|\\
\end{align*}

\begin{align*}
&\langle(1+\e h_i)^{-1}(h-h_i)\pi(a)\Omega,\pi(b)\Omega\rangle\\
&=\langle(1-\e h_i(1+\e h_i)^{-1})(h-h_i)\pi(a)\Omega,\pi(b)\Omega\rangle\\
&=\langle(h-h_i)\pi(a)\Omega,\pi(b)\Omega\rangle+\langle\e h_i(1+\e h_i)^{-1}(h-h_i)\pi(a)\Omega,\pi(b)\Omega\rangle\\
\end{align*}

When we denote by $\omega_{i,\e},\omega_i,\varphi_{i,\e},\varphi_i$ the linear functionals in $A^{*sa}$ corresponded to $h_{i,\e},h_i,\varphi_{i,\e},\varphi_i$, it follows clearly that $\omega_{i,\e}\to\omega_\e$ and $\varphi_{i,\e}\to\varphi_\e$ weakly$^*$ in $A^*$.
The inequality $\omega_{i,\e}\le\varphi_{i,\e}\in F^*$ implies $\omega_\e\le\varphi_\e\in F^*$.
Considering the normal extension $\pi^{**}:A^{**}\to B(H)$ of the representation $\pi$, we have $\omega_\e\uparrow\omega$ weakly in $A^*$ as $\e\to0$.
Then, we obtain a desired sequence $\omega_n$ by taking $\e$ to be a decreasing sequence that converges to zero and less than $\delta\|h\|^{-1}(\|h\|-\delta)^{-1}$ so that $t-\delta\le f_\e(t)$ on $|t|\le\|h\|$, where $\delta<\|h\|$ is assumed without loss of generality.
\end{proof}


The following lemma is a slight generalization of the Krein-\v Smulian theorem, and it can be proved as similar as the original theorem.
\begin{lem}
Let $A$ be a C$^*$-algebra, and $C^*_n$ be a non-decreasing sequence of weakly$^*$-closed convex subsets of $A^{*sa}$, whose union $C_\infty^*$ contains $A^{*+}$.
If a norm closed convex subset $G^*$ of $A^{*sa}$ has the property that $G^*\cap C^*_n$ is weakly$^*$ closed for each $n$, then $G^*\cap C_\infty^*$ is relatively weakly$^*$ closed in $C_\infty^*$.
\end{lem}
\begin{proof}
Fix an element $\omega_0$ of $C_\infty^*\setminus G^*$.
It is enough to construct an element $a$ of $A^{sa}$ separating a norm open ball centered at $\omega_0$ from $G^*$.
Since $G^*$ is norm closed, there exists $r>0$ such that $G^*\cap B(\omega_0,r)=\varnothing$.
By replacing $G^*$ to $r^{-1}(G^*-\omega_0)$ and $C_n^*$ to $r^{-1}(C_n^*-\omega_0)$, we may assume $G^*\cap B(0,1)=\varnothing$, and the claim follows if we prove there is $a\in A^{sa}$ separating $B(0,1)$ and $G^*$.
The condition $A^{*+}\subset C_\infty^*$ becomes $A^{*+}-\omega_0\subset C_\infty^*$.
Letting the index $n$ start from one, we may also replace $C_n^*$ to $n(C_n^*\cap B(0,1))$ since its union is still $C_\infty^*$.
Note that $C_n^*$ is bounded for each $n$, and we can easily see that $G^*\cap C_1^*=\varnothing$ and $n^{-1}C_n^*\subset(n+1)^{-1}C_{n+1}^*$.

Note that for any Banach space $X$, if $F$ is a bounded subset of $X$, then by endowing with the discrete topology on $F$, we have a natural bounded linear operator $\ell^1(F)\to X$ with its dual $X^*\to\ell^\infty(F)$.
We will construct a bounded subset $F$ of $X$ such that the subset $G^*\cap C_\infty^*$ of $X^*$ induces a subset of the smaller subspace $c_0(F)$ of $\ell^\infty(F)$ via the map $X^*\to\ell^\infty(F)$, and also such that it satisfies $G^*\cap C_\infty^*\cap F^\circ=\varnothing$, where $F^\circ:=\{x^*\in X^*:\sup_{x\in F}|x^*(x)|\le1\}$ denotes the complex polar of $F$.
If such a set $F\subset X$ exists, then the image of $G^*\cap C_\infty^*$ in $c_0(F)$ is a convex set disjoint to the closed unit ball $B_{c_0(F)}$ by the condition $G^*\cap C_\infty^*\cap F^\circ=\varnothing$.
Therefore, there exists a separating linear functional $l\in B_{\ell^1(F)}$ by the Hahn-Banach separation, and it induces a linear functional separating $G^*$ and $B_{X^*}$.
Then, we are done.

Let $F_0:=\{0\}\subset X$.
As an induction hypothesis on $n$, suppose for each $0\le k\le n-1$ we already have a finite subset $F_k$ of $(C_k^*)^\circ$ such that
\[G^*\cap C^*_n\cap\left(\bigcup_{k=0}^{n-1}F_k\right)^\circ=\varnothing.\]
If every finite subset $F_n$ of $(C_n^*)^\circ$ satisfies
\[G^*\cap C_{n+1}^*\cap\left(\bigcup_{k=0}^{n-1}F_k\right)^\circ\cap F_n^\circ\ne\varnothing,\]
then since the intersection $G^*\cap C_{n+1}^*$ is weakly$^*$ compact, the finite intersection property leads a contradiction because the intersection of all complex polars $F_n^\circ$ of finite subsets $F_n$ of $(C_n^*)^\circ$ is $C_n^*$, which is the polar of all union of finite subsets $F_n$ of $(C_n^*)^\circ$ by the bipolar theorem.
Thus, we can take a finite subset $F_n$ of $(C_n^*)^\circ$ such that
\[G^*\cap C_{n+1}^*\cap\left(\bigcup_{k=0}^nF_k\right)^\circ=\varnothing.\]
Let $F:=\bigcup_{k=0}^\infty F_k$.
Then, we have $G^*\cap C_\infty^*\cap F^\circ=\varnothing$, and every element of $C_\infty^*$ is restricted to $F$ to define an element of $c_0(F)$ because for each $\omega\in C_n^*$ and $k\ge0$ we have 
\[\omega(F_{n+k})\subset\omega((C_{n+k}^*)^\circ)\subset\frac n{n+k}\omega((C_n^*)^\circ)\subset[-\frac n{n+k},\frac n{n+k}].\]
Finally, for any $\omega\in A^{*sa}$, if we enumerate $F$ as a sequence $f_m$, then
\[|\omega(f_m)|\le|(\omega_+-\omega_0)(f_m)|+|(\omega_--\omega_0)(f_m)|\to0,\]
so the uniform boundedness principle concludes that $F$ is bounded.
Therefore, the set $F$ satisfies the properties we desired.
\end{proof}


\begin{lem}
Let $M$ be a von Neumann algebra, and let $\mathfrak{M}$ be a $\sigma$-weakly dense hereditary $*$-subalgebra of $M$.
If $\omega_i\in M_*^{sa}$ is a dominated net such that $\omega_i\to\omega\in M_*^{sa}$ pointwisely on $\mathfrak{M}$, then $\omega_i\to\omega$ weakly in $M_*$.
\end{lem}
\begin{proof}
Let $\widetilde\omega\in M_*^+$ be a dominating functional of $\omega_i$ such that $-\widetilde\omega\le\omega_i\le\widetilde\omega$ for all $i$.
If $e_j\in\mathfrak{M}^+$ is a net such that $\|e_j\|\le1$ and $e_j\to1$ $\sigma$-strongly in $M$ taken by the Kaplansky density theorem, then we have $e_jxe_j\in\mathfrak{M}$ by the hereditarity of $\mathfrak{M}$ and we can check $e_jxe_j\to x$ $\sigma$-strongly for each $x\in A^{**}$.
For any $x\in A^{**+}$ and $\e>0$, since the absolute value function is a strongly continuous function, we can fix $j$ such that $\widetilde\omega(|x-e_jxe_j|)<\e$, so the convergence $\omega_i\to\omega$ on $\mathfrak{M}$ is enhanced to the weak convergence in $M_*$ by
\begin{align*}
(\omega-\omega_i)(x)
&=(\omega-\omega_i)((x-e_jxe_j)_+)-(\omega-\omega_i)((x-e_jxe_j)_-)+(\omega-\omega_i)(e_jxe_j)\\
&\le(\omega+\widetilde\omega)((x-e_jxe_j)_+)+(\omega+\widetilde\omega)((x-e_jxe_j)_-)+(\omega-\omega_i)(e_jxe_j)\\
&\le2\widetilde\omega(|x-e_jxe_j|)+(\omega-\omega_i)(e_jxe_j)\to2\e+0.\qedhere
\end{align*}
\end{proof}


The main difficulty in dominating an approximating net $\omega_i$ of $\omega$ in the weak$^*$ closure is that we cannot modify $\omega_i$ to $\omega_i-(\omega_i-\omega)_+$ because $(\omega_i-\omega)_+$ may not coverge to zero weakly$^*$.


\begin{thm}[Positive Hahn-Banach separation for C$^*$-algebras]
Let $A$ be a C$^*$-algebra.
\begin{enumerate}
\item If $F$ is a weakly closed convex hereditary subset of $A^+$, then $F=F^{\circ+\circ+}$. In particular, if $a\in A^+\setminus F$, then there is $\omega\in A^{*+}$ such that $\omega(a)>1$ and $\omega\le1$ on $F$.
\item If $F^*$ is a weakly$^*$ closed convex hereditary subset of $A^{*+}$, then $F^*=(F^*)^{\circ+\circ+}$. In particular, if $\omega\in A^{*+}\setminus F^*$, then there is $a\in A^+$ such that $\omega(a)>1$ and $a\le1$ on $F^*$.
\end{enumerate}
\end{thm}
\begin{proof}
(1)
We directly prove the separation without invoking the arguments of positive bipolars.
Denote by $F^{**}$ the $\sigma$-weak closure of $F$ in the universal von Neumann algebra $A^{**}$.
We first show that $F^{**}$ is hereditary subset of $A^{**+}$.
Suppose $0\le x\le y$ in $A^{**}$ and $y\in F^{**}$.
Then, there is $z\in A^{**}$ such that $x^{\frac12}=zy^{\frac12}$.
Take bounded nets $b_i$ in $F$ and $c_i$ in $A$ such that $b_i\to y$ and $c_i\to z$ $\sigma$-strongly$^*$ in $A^{**}$ using the Kaplansky density.
We may assume the indices of these two nets are same.
Since both the multiplication and the involution of a von Neumann algebra on bounded parts are continuous in the $\sigma$-strong$^*$ topology, and since the square root on a positive bounded interval is a strongly continuous function, we have the $\sigma$-strong$^*$ limit
\[x=y^{\frac12}z^*zy^{\frac12}=\lim_ib_i^{\frac12}c_i^*c_ib_i^{\frac12},\]
so we obtain $x\in F^{**}$ from $b_i^{\frac12}c_i^*c_ib_i^{\frac12}\in F$.
Thus, $F^{**}$ is hereditary in $A^{**+}$.

Let $a\in A^+\setminus F$.
Observe that we have $a\in A^{**+}\setminus F^{**}$ because if $a\in F^{**}$, then we have a net $a_i$ in $F$ such that $a_i\to a$ $\sigma$-weakly in $A^{**}$, meaning that $a_i\to a$ weakly in $A$ and by the weak closedness of $F$ in $A$ we get a contradiction $a\in F^{**}\cap A=F$.
By Theorem, there is $\omega\in A^{*+}$ such that $\omega(a)>1$ and $\omega\le1$ on $F\subset F^{**}$, so it completes the proof.

(2)
As same as above, our goal is to prove $(\overline{F^*-A^{*+}})^+\subset F^*$, so take $\omega\in(\overline{F^*-A^{*+}})^+$.
We first prove it when $A$ is separable, which makes the weak$^*$ topology on any bounded part of $A^{*sa}$ metrizable.
Consider the following convex set
\[G^*:=\left\{\omega\in\overline{F^*-A^{*+}}:\begin{tabular}{c}there is a sequence $\omega_n\in F^*-A^{*+}$ and $\widetilde\omega\in A^{*+}$ such that\\$-\widetilde\omega\le\omega_n\to\omega$ weakly$^*$ in $A^*$\end{tabular}\right\}.\]
In the spirit of the Krein-\v Smulian theorem, let $\omega_n$ be a bounded sequence in $G^*$ such that $\omega_n\to\omega$ weakly$^*$ in $A^*$, and claim $\omega\in G^*$.
Since $\omega$ belongs to the relative weak$^*$ closure of $G^*\cap C_\infty^*$ in $C_\infty^*$, where
\[C_n^*:=\{\omega'\in A^{*sa}:-\sum_{k\le n}\omega_k-\omega_-\le\omega'\},\quad C_\infty^*:=\bigcup_nC_n^*,\]
if we prove $G^*$ is norm closed and $G^*\cap C_n^*$ is weakly$^*$ closed for each $n$, then we obtain $\omega\in G^*$ by Lemma, which completes the proof.

Since the limit of a norm convergent sequence in $G^*$ can be approximated by a dominated sequence in $G^*$ as in the proof of Theorem, and every sequence in $C_n^*$ is dominated, it is enough to show $\omega\in G^*$ when it is the weak$^*$ limit of a dominated sequence $\omega_n$ in $G^*$.
Since $A$ is $\sigma$-unital, when we denote by $e$ a strictly positive element of $A$, we can take a sequence $\omega_{nm}\in F^*-A^{*+}$ such that $\omega_{nm}(e)\uparrow\omega_n(e)$, which implies the weak$^*$ convergence $\omega_{nm}\uparrow\omega_n$ for each $n$ by boundedness.
Associated to $\omega$, $\omega_n$, $\omega_{nm}$, and $\varphi_{nm}$, the commutant Radon-Nikodym derivatives $h$, $h_n$, $h_{nm}$, and $k_{nm}$ are defined.
Note that $h_n$ is a bounded sequence, and $h_{nm}$ are bounded increasing sequences for each $n$, and $k_{nm}$ are self-adjoint operators constructed by Friedrichs extension with $h_{nm}\le k_{nm}$ for every $n$ and $m$.
The boundedness implies that $h_n\to h$ as $n\to\infty$ and $h_{nm}\uparrow h_n$ as $m\to\infty$ for each $n$ in the weak operator topology.
By applying the Mazur lemma, we can take a convergent diagonal sequence such that $h_{nn}\to h$ in the strong operator topology, after taking rapidly convergent subsequence from $h_{nm}$ for each $n$, which can be done because the existence of a cyclic vector implies that the commutant is a $\sigma$-finite von Neumann algebra and the strong operator topology is metrizable.
Then, $h_{nn}$ is bounded by the uniform boundedness principle, we can take $f_\e$ to show $\omega\in F^*-A^{*+}$.



Now we consider a general C$^*$-algebra $A$.
For a C$^*$-subalgebra $B$ of $A$, we define a set
\[F_B^*:=\{\omega\in B^{*+}:\text{there is $\varphi\in F^*$ such that $\omega\le\varphi$ on $B^+$}\}.\]
It is clearly a convex hereditary subset of $B^{*+}$.
If $\omega_i\in F_B^*$ is a net such that $\omega_i\to\omega$ weakly$^*$ in $B^*$ and $\omega_i\le\varphi_i\in F^*$ on $B^+$, then there are positive extensions $\widetilde\omega_i\in A^{*+}$ of $\omega_i$ such that $\widetilde\omega_i\le\varphi_i$ on $B^+$.

If $\widetilde\omega_i$ is a positive norm preserving extension of $\omega_i$, then $\widetilde\omega_i\le\varphi_i$? no.


$\omega_n\to\omega$ in norm and $\omega_n\le\omega$.
Let 


Considering $k_{i,\e}\to k_\e$ so that $\varphi_\e\in F^*$.
We need weak$^*$ convergence $\varphi_{i,\e}\to\varphi_\e$.

$\widetilde\omega_{i,\e}\le\varphi_{i,\e}$

$\widetilde\omega_\e\le\varphi_\e$

$\varphi_i=\omega_i^\sim+(\varphi_i-\omega_i)^\sim$ on $B$

$\|\omega_i^\sim+(\varphi_i-\omega_i)^\sim\|\le\|\omega_i^\sim\|+\|(\varphi_i-\omega_i)^\sim\|=\|\omega_i\|+\|\varphi_i|_B-\omega_i\|\le\|\varphi_i|_B\|$


Let $\omega\in(\overline{F^*-A^{*+}})^+$, where the closure is taken in the weak$^*$ topology.
Take a net $\omega_i\in F^*-A^{*+}$ and $\varphi_i\in F^*$ such that $\omega_i\to\omega$ weakly$^*$ in $A^*$ and $\omega_i\le\varphi_i$ for each $i$.
If we denote by $\omega_B,\omega_{i,B},\varphi_{i,B}$ the restrictions of $\omega,\omega_i,\varphi_i$ on $B$, then we have $\varphi_{i,B}\in F^*_{B}$ and $\omega_{i,B}\in F^*_B-B^{*+}$, with the weak$^*$ convergence $\omega_{i,B}\to\omega_B$ in $B^*$, thus we have $\omega_B\in(\overline{F_B^*-B^{*+}})^+=F_B^*$ because $B$ is separable.
If we consider the non-decreasing net of all separable C$^*$-subalgebras $B_j$ of $A$, then the restriction $\omega_{B_j}$ of $\omega$ on $B_j$ belongs to the set $F_{B_j}^*$ as we have seen just now, so there is $\varphi_j\in F^*$ such that $\omega\le\varphi_j$ on $B_j^+$ for each $j$.
Here we let $\psi$ be a faithful semi-finite normal weight on $A^{**}$, and let $\pi:A^{**}\to B(H)$ be the Gelfand-Naimark-Segal representation associated to $\psi$, together with the left $A^{**}$-liner map $\Lambda:\mathfrak{N}_\psi\to H$ of dense range such that $\psi(x^*x)=\|\Lambda(x)\|^2$ for all $x\in\mathfrak{N}_\psi$.
Note that because the weight $\psi$ is faithful and semi-finite, $\Lambda$ is injective and $\sigma$-weakly densely defined, meaning that $\mathfrak{M}_\psi$ is a hereditary $\sigma$-weakly dense $*$-subalgebra of $A^{**}$.
Construct the commutant Radon-Nikodym derivatives $h,k_j$ of $\omega,\varphi_j$ with respect to $\psi$.
Here $k_j$ is a positive self-adjoint operator defined by the Friedrichs extension such that $\operatorname{ran}\Lambda\subset\operatorname{dom}k_j$ for all $j$.
Taking a subnet, we may assume that there is $k_\e\in\pi(A)'^+$ satisfying $f_\e(k_j)\to k_\e$ $\sigma$-weakly.
Because of the operator concavity of $f_\e$ (more detail), we can take a net $\varphi_l\in F^*$ such that $f_\e(k_l)\to k_\e$ $\sigma$-strongly, where $k_l$ are again the commutant Radon-Nikodym derivatives of $\varphi_l$ defined by the Friedrichs extension.
Since  is a strongly continuous function, we have $(f_\e(k_l)-k_\e)\to0$ $\sigma$-strongly, so if we define $\varphi_{l,\e}\in F^*-A^{*+}$ and $\varphi_\e\in A^{*+}$ such that
\[\varphi_{l,\e}(x^*x):=\langle(f_\e(k_l)-(f_\e(k_l)-k_\e)_+)\Lambda(x),\Lambda(x)\rangle,\quad\varphi_\e(x^*x):=\langle k_\e\Lambda(x),\Lambda(x)\rangle\]
for each $x\in\mathfrak{N}_\psi$, then we have $\varphi_{l,\e}\to\varphi_\e$ pointwisely on $\mathfrak{M}_\psi$ and $\varphi_{l,\e}\le\varphi_\e$ for all $l$.

How to dominate $\varphi_{l,\e}$ from below?

By Lemma we have $\varphi_{l,\e}\to\varphi_\e$ weakly in $A^*$, so Theorem implies that $\varphi_\e\in(\overline{F^*-A^{*+}}^w)^+=F^*$.
If we define $\omega_\e\in A^{*+}$ and $\varphi_{j,\e}\in F^*$ by
\[\omega_\e(x^*x):=\langle f_\e(h)\Lambda(x),\Lambda(x)\rangle,\quad\varphi_{j,\e}(x^*x):=\langle f_\e(k_j)\Lambda(x),\Lambda(x)\rangle\]
for each $x\in\mathfrak{N}_\psi$, then since $\omega\le\varphi_j$ on $B_j^+$ implies $\omega_\e\le\varphi_{j,\e}$ on $B_j^+$, the weak$^*$ limit $\omega_\e\le\lim_j\varphi_{j,\e}=\varphi_\e$ deduces $\omega_\e\in F^*-A^{*+}$.
Since $\omega_\e\to\omega$ pointwisely on $\mathfrak{M}_\psi$ and $0\le\omega_\e\le\omega$ for all $0<\e$, we have $\omega\in(\overline{F^*-A^{*+}}^w)^+=F^*$ by Lemma and Theorem.
\end{proof}





\section{}

\begin{cor}
Let $M$ be a von Neumann algebra.
Then, there is a one-to-one correspondence
\[\begin{array}{ccc}
\left\{\emph{\begin{tabular}{c}subadditive normal\\weights of $M$\end{tabular}}\right\}&\leftrightarrow&\left\{\emph{\begin{tabular}{c}hereditary closed\\convex subsets of $M_*^+$\end{tabular}}\right\}\\[10pt]
\varphi&\mapsto&\{\omega\in M_*^+:\omega\le\varphi\}
\end{array}\]
\end{cor}



\end{document}