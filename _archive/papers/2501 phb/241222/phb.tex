\documentclass[a4paper]{amsart}

\usepackage[T1]{fontenc}
\usepackage[bitstream-charter,cal]{mathdesign}
\linespread{1.15}

\newcommand{\e}{\varepsilon}

\theoremstyle{plain}
\newtheorem{thm}{Theorem}[section]
\newtheorem{lem}[thm]{Lemma}
\newtheorem{cor}[thm]{Corollary}
\theoremstyle{definition}
\newtheorem{defn}[thm]{Definition}


\title{Positive Hahn-Banach separations in operator algebras}
\author{Ikhan Choi}
\address{}
\subjclass[2020]{}

\begin{document}

\begin{abstract}

\end{abstract}

\maketitle

\section{Introduction}






\begin{itemize}
\item definition and properties of $f_\e(t):=(1+\e t)^{-1}t$
\item commutant Radon-Nikodym, relation between $\{\omega'\in M_*^+:\omega'\le\omega\}$ and $\{h\in\pi(M)'^+:h\le1\}$, order preserving linear map
\item Mazur lemma
\end{itemize}


\begin{defn}[Hereditary subsets]
Let $E$ be a partially ordered real vector space.
We say a subset $F$ of the positive cone $E^+$ is \emph{hereditary} if $0\le x\le y$ in $E$ and $y\in F$ imply $x\in F$, or equivalently $F=(F-E^+)^+$, where $F-E^+$ is the set of all positive elements of $E$ bounded above by an element of $F$.
A $*$-subalgebra $B$ of a $*$-algbera $A$ is called \emph{hereditary} if the positive cone $B^+$ is a hereditary subset of $A^+$.
We define the \emph{positive polar} of $F$ as the positive part of the real polar
\[F^{r+}:=\{x^*\in(E^*)^+:\sup_{x\in F}x^*(x)\le1\}.\]
\end{defn}
An example that is a non-hereditary closed convex subset of a C$^*$-algebra is $\mathbb{C}1$ in any unital C$^*$-algebra.


\begin{defn}[Lower dominated sequences]
Let $E$ be a partially ordered real vector space.
A sequence $x_n\in E$ is called \emph{lower dominated} if there is $x\in E$ such that $x\le x_n$ for all $n$.
If $E$ is the self-adjoint part of the predual of a von Neumann algebra where the Jordan decomposition holds, then we can change the definition such that $x\in -E^+$.
\end{defn}

\section{Positive Hahn-Banach separation theorems}

We start with the positive Hahn-Banach separation for von Neumann algebras and their preduals, and will close this section with the same theorem for C$^*$-algebras and their duals.


\begin{thm}[Positive Hahn-Banach separation for von Neumann algebras]\label{positive hahn-banach w*}
Let $M$ be a von Neumann algebra.
\begin{enumerate}
\item If $F$ is a $\sigma$-weakly closed convex hereditary subset of $M^+$, then $F=F^{r+r+}$. In particular, if $x'\in M^+\setminus F$, then there is $\omega\in M_*^+$ such that $\omega(x')>1$ and $\omega(x)\le1$ for $x\in F$.
\item If $F_*$ is a norm closed convex hereditary subset of $M_*^+$, then $F_*=F_*^{r+r+}$. In particular, if $\omega'\in M_*^+\setminus F_*$, then there is $x\in M^+$ such that $\omega'(x)>1$ and $\omega(x)\le1$ for $\omega\in F_*$.
\end{enumerate}
\end{thm}
\begin{proof}
(1)
Since the positive polar is represented as the real polar
\[F^{r+}=F^r\cap M_*^+=F^r\cap(-M^+)^r=(F\cup-M^+)^r=(F-M^+)^r,\]
the positive bipolar can be written as $F^{r+r+}=(F-M^+)^{rr+}=(\overline{F-M^+})^+$ by the usual real bipolar theorem, where the closure is for the $\sigma$-weak topology.
Because $F=(F-M^+)^+\subset(\overline{F-M^+})^+$, it suffices to prove the opposite inclusion $(\overline{F-M^+})^+\subset F$.

Let $x\in(\overline{F-M^+})^+$.
Take a net $x_i\in F-M^+$ such that $x_i\to x$ $\sigma$-strongly, and take a net $y_i\in F$ such that $x_i\le y_i$ for each $i$.
Suppose we may assume that the net $x_i$ is bounded.
For sufficiently small $\e$ so that the bounded net $x_i$ has the spectra in $[-(2\e)^{-1},\infty)$, we have $f_\e(x_i)\to f_\e(x)$ $\sigma$-strongly, and hence $\sigma$-weakly.
On the other hand, by the hereditarity and the $\sigma$-weak compactness of $F$, we may asumme that the bounded net $f_\e(y_i)\in F$ converges $\sigma$-weakly to a point of $F$ by taking a subnet.
Then, we have $f_\e(x)\in F-M^+$ by
\[0\le f_\e(x)=\lim_if_\e(x_i)\le\lim_if_\e(y_i)\in F,\]
thus we have $x\in F$ since $f_\e(x)\uparrow x$ as $\e\to0$.
What remains is to prove the existence of a bounded net $x_i\in F-M^+$ such that $x_i\to x$ $\sigma$-strongly.

Define a convex set
\[G:=\left\{x\in\overline{F-M^+}:\begin{tabular}{c}there is a sequence $x_m\in F-M^+$\\such that $-2x\le x_m\uparrow x$ $\sigma$-weakly\end{tabular}\right\}\subset M^{sa},\]
where $x_m$ denotes a sequence.
In fact, it has no critical issue on allowing $x_m$ to be uncountably indexed.
Since we clearly have $F-M^+\subset G$ and every non-decreasing net with supremum is bounded and $\sigma$-strongly convergent, it suffices to show that $G$, or equivalently its intersection with the closed unit ball by the Krein-Sm\v ulian theorem, is $\sigma$-strongly closed.
Let $x_i\in G$ be a net such that $\sup_i\|x_i\|\le1$ and $x_i\to x$ $\sigma$-strongly.
For each $i$, take a sequence $x_{im}\in F-M^+$ such that $-2x_i\le x_{im}\uparrow x_i$ $\sigma$-strongly as $m\to\infty$, and also take $y_{im}\in F$ such that $x_{im}\le y_{im}$.
Since $\|x_{im}\|\le2\|x_i\|\le2$ is bounded, it implies that there is a bounded net $x_j$ in $F-M^+$ such that $x_j\to x$ $\sigma$-strongly, and we can choose arbitrarily small $\e>0$ such that $\sigma(x_j)\subset[-(2\e)^{-1},\infty)$ for all $j$.
Since $f_\e(x_j)$ converges to $f_\e(x)$ $\sigma$-strongly and $f_\e(y_j)$ is a bounded net for each $\e>0$ so that we may assume that the net $f_\e(y_j)$ is $\sigma$-weakly covergent by taking a subnet, we have $f_\e(x)\in F-M^+$ by
\[f_\e(x)=\lim_jf_\e(x_j)\le\lim_jf_\e(y_j)\in F,\]
where the limits are in the $\sigma$-weak sense.
By taking $\e$ as any decreasingly convergent sequence to zero, we have $x\in G$, hence the closedness of $G$.


(2)
It is enough to prove $(\overline{F_*-M_*^+})^+\subset F_*$, where the closure is for the weak topology or equivalently in norm by the convexity of $F_*-M_*^+$, so we begin our proof by fixing $\omega\in(\overline{F_*-M_*^+})^+$.
For a sequence $\omega_n\in F_*-M_*^+$ such that $\omega_n\to\omega$ in norm of $M_*$, we can take $\varphi_n\in F_*$ such that $\omega_n\le\varphi_n$ for all $n$.
By modifying $\omega_n$ into $\omega_n-(\omega_n-\omega)_+\in F_*-M_*^+$ and taking a rapidly convergent subsequence, we may assume $\omega_n\le\omega$ and $\|\omega-\omega_n\|\le2^{-n}$ for all $n$.
If we consider the Gelfand-Naimark-Segal representation $\pi:M\to B(H)$ associated to a positive normal linear functional \[\widehat\omega:=\sum_n(\omega-\omega_n)+\omega+\sum_n2^{-n}\left(\frac{[\omega_n]}{1+\|\omega_n\|}+\frac{\varphi_n}{1+\|\varphi_n\|}\right)\]
on $M$ with the canonical cyclic vector $\Omega$, we can construct commutant Radon-Nikodym derivatives $h,h_n,k_n\in\pi(M)'$ of $\omega,\omega_n,\varphi_n$ with respect to $\widehat\omega$ respectively.
Since $-1\le h_n\le h$ is bounded, $h_n\to h$ in the weak operator topology of $\pi(M)'$.
By the Mazur lemma, we can take a net $h_i$ by convex combinations of $h_n$ such that $h_i\to h$ strongly in $\pi(M)'$, and the corresponding linear functionals $\omega_i$ and $\varphi_i$ satisfy $\omega_i\le\varphi_i$ with $\varphi_i\in F_*$ by the convexity of $F_*$ so that $\omega_i\in F_*-M_*^+$.
The net $h_i$ can be taken to be a sequence in fact because $\pi(M)'$ is $\sigma$-finite by the existence of the separating vector $\Omega$, but it is not necessary in here.
For each $i$ and $0<\e<1$, define
\[h_\e:=f_\e(h),\quad h_{i,\e}:=f_\e(h_i),\quad k_{i,\e}:=f_\e(k_i)\]
in $\pi(M)'$, where the functional calculi are well-defined because $-1\le h_i$ and $0\le h,k_i$ for all $i$, and define $k_\e$ as the $\sigma$-weak limit of the bounded net $k_{i,\e}$, which may be assumed to be $\sigma$-weakly convergent.
Define $\omega_\e,\omega_{i,\e},\varphi_{i,\e},\varphi_\e$ as the corresponding normal linear functionals on $M$ to $h_\e,h_{i,\e},k_{i,\e},k_\e$.
Note that $\varphi_i\in F_*$.
The hereditarity of $F_*$ and $0\le\varphi_{i,\e}\le\varphi_i$ imply $\varphi_{i,\e}\in F_*$, and the weak closedness of $F_*$ and the weak convergence $\varphi_{i,\e}\to\varphi_\e$ in $M_*$ imply $\varphi_\e\in F^*$.
From $\omega_i\le\varphi_i$, we can deduce $0\le\omega_\e\le\varphi_\e$ by considering the operator monotonicity $f_\e$ and taking the weak limit on $i$.
Thus again, the hereditarity of $F_*$ implies $\omega_\e\in F^*$, and the weak closedness of $F_*$ and the weak convergence $\omega_\e\to\omega$ in $M_*$ imply $\omega\in F^*$.
\end{proof}

Now we prepare some lemmas for the positive Hahn-Banach separation theorem for C$^*$-algebras.


\begin{lem}\label{lower dominated sequence}
Let $A$ be a separable C$^*$-algebra, and let $G^*$ be a norm closed and downward closed convex subset of $A^{*sa}$.
If an element of $A^{*sa}$ is approximated weakly$^*$ by a lower dominated sequence of $G^*$, then it is approximated in norm by a sequence of $G^*$.
\end{lem}
\begin{proof}
Let $\omega_n\in G^*$ and $\widehat\omega_0\in A^{*+}$ be such that $\omega_n\to\omega$ weakly$^*$ in $A^*$ and $-\widehat\omega_0\le\omega_n$ for all $n$.
Consider the Gelfan-Naimark-Segal representation $\pi:A\to B(H)$ of
\[\widehat\omega:=\widehat\omega_0+[\omega]+\sum_n2^{-n}\frac{[\omega_n]}{1+\|\omega_n\|}.\]
%Let $\theta^*:\pi(A')\to A^*:y\mapsto(a\mapsto\langle y\pi(a)\Omega,\Omega\rangle)$.
Define the commutant Radon-Nikodym derivatives $h,h_n\in\pi(A)'$ of $\omega,\omega_n$ with respect to $\widehat\omega$.
Note that we have $-1\le h\le1$ and $-1\le h_n$, and the commutant $\pi(A)'$ is $\sigma$-finite so that the strong operator topology on a bounded part is metrizable.




Fix $m\ge2$.
Let
\[S_{mn}:=\operatorname{conv}\{(f_{m^{-1}}((h_k-m^3)_++m^3),(h_k-m^3)_-)\in\pi(A)'\times\pi(A)':k\ge n\}\]
be the convex hull of a sequence consisting of pairs of suitably truncated operators.
Since $S_{mn}$ is a non-increasing sequence of bounded subsets of $\pi(A)'\times\pi(A)'$, the intersection $\bigcap_n\overline{S_{mn}}^w$ is non-empty by the finite intersection property of compact subsets in the weak operator topology, where the closure is taken in either the weak or strong operator topology.
Then, it follows from the strong continuity of $f_{-m^{-1}}$ on the interval $[-1-m^{-1},m-m^{-1}]$ that there is a strongly convergent sequence $h_{mn}\in\pi(A)'$ such that for each $n$ there exists a finitely supported positive real sequence $(\lambda_k)_{k\ge n}$ such that $\sum_{k\ge n}\lambda_k=1$ and
\[h_{mn}=f_{-m^{-1}}\left(-m^{-1}+\sum_{k\ge n}\lambda_kf_{m^{-1}}((h_k-m^3)_++m^3)\right)+\sum_{k\ge n}\lambda_k(h_k-m^3)_-.\]
Let $k_{mn}:=\sum_{k\ge n}\lambda_kf_{m^{-1}}((h_k-m^3)_++m^3)$, and let $k_{m\infty}$ and $h_{m\infty}$ be the strong limits of $k_{mn}$ and $h_{mn}$ respectively as $n\to\infty$.
By the operator concavity of the function $f_{m^{-1}}$, we have
\begin{align*}
h_{mn}
&\le f_{-m^{-1}}\left(\sum_{k\ge n}\lambda_kf_{m^{-1}}((h_k-m^3)_++m^3)\right)+\sum_{k\ge n}\lambda_k(h_k-m^3)_-\\
&\le\sum_{k\ge n}\lambda_k((h_k-m^3)_++m^3)+\sum_{k\ge n}\lambda_k(h_k-m^3)_-=\sum_{k\ge n}\lambda_kh_k.
\end{align*}


On the other hand,
\begin{align*}
h_{mn}
&=f_{-m^{-1}}\left(-\frac{m}{1+m^{-1}t}+\sum_{k\ge n}\lambda_kf_{m^{-1}}((h_k-t)_++t)\right)+\sum_{k\ge n}\lambda_k(h_k-t)_-\\
&\ge f_{-m^{-1}}\left(-\frac{m}{1+m^{-1}t}-(m-f_{m^{-1}}(t))+f_{m^{-1}}\left(\sum_{k\ge n}\lambda_k(h_k-t)_++t\right)\right)+\sum_{k\ge n}\lambda_k(h_k-t)_-\\
&=f_{-m^{-1}}\left(-\frac{2m}{1+m^{-1}t}+f_{m^{-1}}\left(\sum_{k\ge n}\lambda_k(h_k-t)_++t\right)\right)+\sum_{k\ge n}\lambda_k(h_k-t)_-\\
&\ge\sum_{k\ge n}\lambda_kh_k-\delta\frac1{(1-m^{-1}x)^2}\\
\end{align*}
where $\delta=\frac{2m}{1+m^{-1}t}$, because
\[f_{-m^{-1}}(-\delta+x)-f_{-m^{-1}}(x)=\frac{-\delta}{(1-m^{-1}x)(1-m^{-1}(-\delta+x))}\ge-\frac\delta{(1-m^{-1}x)^2},\]
where
\[x=f_{m^{-1}}\left(\sum_{k\ge n}\lambda_k(h_k-t)_++t\right)=m\left(1-\frac1{1+m^{-1}\left(\sum_{k\ge n}\lambda_k(h_k-t)_++t\right)}\right).\]
We have
\[\frac\delta{(1-m^{-1}x)^2}=\delta\left(1+m^{-1}\left(\sum_{k\ge n}\lambda_k(h_k-t)_++t\right)\right)^2,\]
and since
\[\sum_{k\ge n}\lambda_k(h_k-t)_++t=\sum_{k\ge n}\lambda_kh_k+\sum_{k\ge n}\lambda_k(h_k-t)_-,\]
so
\begin{align*}
\frac\delta{(1-m^{-1}x)^2}&=\delta\left(1+m^{-1}\left(\sum_{k\ge n}\lambda_kh_k+\sum_{k\ge n}\lambda_k(h_k-t)_-\right)\right)^2\\
&\le\delta\left(1+m^{-1}\left(\sum_{k\ge n}\lambda_kh_k+(1+t)\right)\right)^2\\
&=\delta\left(1+m^{-1}+m^{-1}t+m^{-1}\left(\sum_{k\ge n}\lambda_kh_k\right)\right)^2\\
\end{align*}

\[\frac1{1-m^{-1}x+\frac2{1+m^{-1}t}}\le(\frac1{1+m^{-1}(\sum_{k\ge n}\lambda_kh_k+(1+t))}+\frac2{1+m^{-1}t})^{-1}\]


Let $\omega_{mn}:=\theta^*(h_{mn})$ be the corresponding functionals for all $n\in\mathbb{N}\cup\{\infty\}$, then $\omega_{mn}\to\omega_{m\infty}$ weakly in $A^*$ by the strong convergence $h_{mn}\to h_{m\infty}$,

\[\sum_{k\ge n}\lambda_kh_k-m^{-1}\le h_{mn}\le\sum_{k\ge n}\lambda_kh_k.\]

 and we have
\[\sum_{k\ge n}\lambda_k\omega_k-m^{-1}\le\omega_{mn}\le\sum_{k\ge n}\lambda_k\omega_k.\]
Since the right term belongs to $G^*$, we have $\omega_{mn}\in G^*$ by the downward closedness of $G^*$, and $\omega_{m\infty}\in G^*$ by the weak closedness of $G^*$.
Then, because the weak$^*$ convergence $\omega_n\to\omega$ in $A^*$ gives the inequality $\omega-m^{-1}\le\omega_{m\infty}\le\omega$, the squeeze implies $\omega_{m\infty}\to\omega$ weakly in $A^*$.
Therefore, we can obtain our desired sequence by applying the Mazur lemma on the sequence $\omega_{m\infty}$.
\end{proof}


The following lemma is a modification of the Krein-\v Smulian theorem, and it can be proved in a similar way to the proof of the original theorem.

\begin{lem}\label{krein-smulian}
Let $A$ be a C$^*$-algebra, and $C^*_n$ be a non-decreasing sequence of weakly$^*$-closed convex subsets of $A^{*sa}$, whose union $C_\infty^*$ contains $A^{*+}$.
If a norm closed convex subset $G^*$ of $A^{*sa}$ has the property that $G^*\cap C^*_n$ is weakly$^*$ closed for each $n$, then $G^*\cap C_\infty^*$ is relatively weakly$^*$ closed in $C_\infty^*$.
\end{lem}
\begin{proof}
Fix an element $\omega_0$ of $C_\infty^*\setminus G^*$.
It is enough to construct an element $a$ of $A^{sa}$ separating a norm open ball centered at $\omega_0$ from $G^*$.
Since $G^*$ is norm closed, there exists $r>0$ such that $G^*\cap B(\omega_0,r)=\varnothing$.
By replacing $G^*$ to $r^{-1}(G^*-\omega_0)$ and $C_n^*$ to $r^{-1}(C_n^*-\omega_0)$, we may assume $G^*\cap B(0,1)=\varnothing$, and the claim follows if we prove there is $a\in A^{sa}$ separating $B(0,1)$ and $G^*$.
The condition $A^{*+}\subset C_\infty^*$ becomes $A^{*+}-\omega_0\subset C_\infty^*$.
Letting the index $n$ start from one, we may also replace $C_n^*$ to $n(C_n^*\cap B(0,1))$ since its union is still $C_\infty^*$.
Note that $C_n^*$ is bounded for each $n$, and we can easily see that $G^*\cap C_1^*=\varnothing$ and $n^{-1}C_n^*\subset(n+1)^{-1}C_{n+1}^*$.

Note that for any Banach space $X$, if $F$ is a bounded subset of $X$, then by endowing with the discrete topology on $F$, we have a natural bounded linear operator $\ell^1(F)\to X$ by completeness of $X$, with its dual $X^*\to\ell^\infty(F)$.
We will construct a bounded subset $F$ of $A^{sa}$ such that the subset $G^*\cap C_\infty^*$ of $A^{*sa}$ induces a subset of the smaller subspace $c_0(F)$ of $\ell^\infty(F)$ via the restriction map $A^{*sa}\to\ell^\infty(F)$, and also such that it satisfies $G^*\cap C_\infty^*\cap F^\circ=\varnothing$, where $F^\circ:=\{\omega\in A^{*sa}:\sup_{a\in F}|\omega(a)|\le1\}$ denotes the absolute polar of $F$.
If such a set $F\subset X$ exists, then the image of $G^*\cap C_\infty^*$ in $c_0(F)$ is a convex set disjoint to the closed unit ball of $c_0(F)$ by the condition $G^*\cap C_\infty^*\cap F^\circ=\varnothing$.
Therefore, there exists a separating linear functional $l\in\ell^1(F)$ by the Hahn-Banach separation, and it induces a linear functional separating $G^*$ and the unit ball of $A^{*sa}$.
Then, we are done.

Let $F_0:=\{0\}\subset A^{sa}$.
As an induction hypothesis on $n$, suppose for each $0\le k\le n-1$ we already have a finite subset $F_k$ of $(C_k^*)^\circ$ such that
\[G^*\cap C^*_n\cap\left(\bigcup_{k=0}^{n-1}F_k\right)^\circ=\varnothing.\]
If every finite subset $F_n$ of $(C_n^*)^\circ$ satisfies
\[G^*\cap C_{n+1}^*\cap\left(\bigcup_{k=0}^{n-1}F_k\right)^\circ\cap F_n^\circ\ne\varnothing,\]
then since they are weakly$^*$ compact, the finite intersection property leads a contradiction because the intersection of all absolute polars $F_n^\circ$ of finite subsets $F_n$ of $(C_n^*)^\circ$ is $C_n^*$, which is the polar of all union of finite subsets $F_n$ of $(C_n^*)^\circ$ by the bipolar theorem.
Thus, we can take a finite subset $F_n$ of $(C_n^*)^\circ$ such that
\[G^*\cap C_{n+1}^*\cap\left(\bigcup_{k=0}^nF_k\right)^\circ=\varnothing.\]
Let $F:=\bigcup_{k=0}^\infty F_k$.
Then, we have $G^*\cap C_\infty^*\cap F^\circ=\varnothing$, and every element of $C_\infty^*$ is restricted to $F$ to define an element of $c_0(F)$ because for each $\omega\in C_n^*$ and $k\ge0$ we have 
\[\omega(F_{n+k})\subset\omega((C_{n+k}^*)^\circ)\subset\frac n{n+k}\omega((C_n^*)^\circ)\subset[-\frac n{n+k},\frac n{n+k}].\]
Finally, for any $\omega\in A^{*sa}$, if we enumerate $F$ as a sequence $f_m$, then
\[|\omega(f_m)|\le|(\omega_+-\omega_0)(f_m)|+|(\omega_--\omega_0)(f_m)|\to0,\]
so the uniform boundedness principle concludes that $F$ is bounded.
Therefore, the set $F$ satisfies the properties we desired.
\end{proof}





\begin{thm}[Positive Hahn-Banach separation for C$^*$-algebras]
Let $A$ be a C$^*$-algebra.
\begin{enumerate}
\item If $F$ is a norm closed convex hereditary subset of $A^+$, then $F=F^{r+r+}$. In particular, if $a'\in A^+\setminus F$, then there is $\omega\in A^{*+}$ such that $\omega(a')>1$ and $\omega(a)\le1$ for $a\in F$.
\item If $F^*$ is a weakly$^*$ closed convex hereditary subset of $A^{*+}$, then $F^*=(F^*)^{r+r+}$. In particular, if $\omega'\in A^{*+}\setminus F^*$, then there is $a\in A^+$ such that $\omega'(a)>1$ and $\omega(a)\le1$ for $\omega\in F^*$.
\end{enumerate}
\end{thm}
\begin{proof}
(1)
We directly prove the separation without invoking the arguments of positive bipolars.
Denote by $F^{**}$ the $\sigma$-weak closure of $F$ in the universal von Neumann algebra $A^{**}$.
We first show that $F^{**}$ is hereditary subset of $A^{**+}$.
Suppose $0\le x\le y$ in $A^{**}$ and $y\in F^{**}$.
Then, there is $z\in A^{**}$ such that $x^{\frac12}=zy^{\frac12}$.
Take bounded nets $b_i$ in $F$ and $c_i$ in $A$ such that $b_i\to y$ and $c_i\to z$ $\sigma$-strongly$^*$ in $A^{**}$ using the Kaplansky density.
We may assume the indices of these two nets are same.
Since both the multiplication and the involution of a von Neumann algebra on bounded parts are continuous in the $\sigma$-strong$^*$ topology, and since the square root on a positive bounded interval is a strongly continuous function, we have the $\sigma$-strong$^*$ limit
\[x=y^{\frac12}z^*zy^{\frac12}=\lim_ib_i^{\frac12}c_i^*c_ib_i^{\frac12},\]
so we obtain $x\in F^{**}$ from $b_i^{\frac12}c_i^*c_ib_i^{\frac12}\in F$.
Thus, $F^{**}$ is hereditary in $A^{**+}$.

Let $a\in A^+\setminus F$.
Observe that we have $a\in A^{**+}\setminus F^{**}$ because if $a\in F^{**}$, then we have a net $a_i$ in $F$ such that $a_i\to a$ $\sigma$-weakly in $A^{**}$, meaning that $a_i\to a$ weakly in $A$ and by the weak closedness of $F$ in $A$ we get a contradiction $a\in F^{**}\cap A=F$.
By Theorem \ref{positive hahn-banach w*}, there is $\omega\in A^{*+}$ such that $\omega(a)>1$ and $\omega\le1$ on $F\subset F^{**}$, so it completes the proof.

(2)
As same as above, our goal is to prove $(\overline{F^*-A^{*+}})^+\subset F^*$, where the closure notation will always be used for the weak$^*$ topology throughout the whole proof.
We first prove it when $A$ is commutative.
On a commutative C$^*$-algebra, the rectifier function $\mathbb{R}\to\mathbb{R}:t\mapsto\max\{0,t\}$ plays the role of an operator monotone function in the sense that if $\omega_1\le\omega_2$ are functionals in $A^{*sa}$ then we have $\omega_{1+}\le\omega_{2+}$ for the Jordan decompositions.
To make use of the Krein-\v Smulian theorem, define
\[G^*:=\left\{\omega\in\overline{F^*-A^{*+}}:\begin{tabular}{c}there is a bounded net $\omega_j\in F^*-A^{*+}$\\such that $\omega_j\to\omega$ weakly$^*$ in $A^*$\end{tabular}\right\}.\]
We can easily check $F^*-A^{*+}\subset G^*$ by considering constant sequences.
To show $G^*$ is weakly$^*$ closed, take a bounded net $\omega_i\in G^*$ in the spirit of the Krein-\v Smulian theorem such that $\omega_i\to\omega$ weakly$^*$ in $A^*$.
Then, for each $i$ we have a bounded net $\omega_{ij}\in F^*-A^{*+}$ and a net $\varphi_{ij}\in F^*$ such that $\omega_{ij}\le\varphi_{ij}$ for all $j$ and $\omega_{ij}\to\omega_i$ weakly$^*$ in $A^*$ by definition of $G^*$.
Since $\omega_{ij}\le\varphi_{ij}$ implies $0\le\omega_{ij+}\le\varphi_{ij}\in F^*$ and $\omega_{ij+}\in F^*$ by the hereditarity of $F^*$, and since the net $\omega_{ij+}$ is bounded for each $i$ so that we may assume $\omega_{ij+}\to\omega_i'$ weakly$^*$ in $A^*$, we have $\omega_i'\in F^*$ by the weak$^*$ closedness of $F^*$.
Since $\omega_i\le\omega_i'$ implies $0\le\omega_{i+}\le\omega_i'\in F^*$ and $\omega_{i+}\in F^*$ by the hereditarity of $F^*$, and since the net $\omega_{i+}$ is bounded so that we may assume $\omega_{i+}\to\omega'$ weakly$^*$ in $A^*$, we have $\omega'\in F^*$.
Then, $0\le\omega\le\omega'$ implies that $\omega\in F^*-A^{*+}\subset G^*$ and $G^*$ is weakly$^*$ closed, so $G^*=\overline{F^*-A^{*+}}$.
Now if we take $\omega\in(\overline{F^*-A^{*+}})^+$, then there is a bounded net $\omega_i\in F^*-A^{*+}$ and a net $\varphi_i\in F^*$ such that $\omega_i\le\varphi_i$ for all $i$ and $\omega_i\to\omega$ weakly$^*$ in $A^*$ by definition of $G^*$, so since $0\le\omega_{i+}\le\varphi_i\in F^*$ implies $\omega_{i+}\in F^*$, and since the net $\omega_{i+}$ is bounded so that we may assume $\omega_{i+}\to\omega'$ weakly$^*$ in $A^*$, we have $\omega\le\omega'\in F^*$, which gives $\omega\in F^*$ by the hereditarity of $F^*$.
This completes the proof of $(\overline{F^*-A^{*+}})^+\subset F^*$ provided that $A$ is commutative.

Now we consider a general C$^*$-algebra $A$.
For any separable C$^*$-subalgebra $B$ of $A$, define
\[F_B^*:=\overline{\{\omega|_B\in B^{*+}:\omega\in F^*-A^{*+}\}}^{\|\cdot\|},\]
which is clearly a norm closed and convex, and we can see that it is hereditary in $B^{*+}$ by the positive Hahn-Banach extension.
We first claim $(\overline{F_B^*-B^{*+}})^+\subset F_B^*$.
As a remark, we take a note that the claim implies that $F_B^*$ is weakly$^*$ closed, and if $A$ is separable itself, then the proof of the theorem follows by letting $B=A$.
Note that the separability of $B$ makes the weak$^*$ topology on any bounded part of $B^{*sa}$ metrizable.
Consider
\[G_B^*:=\overline{F_B-B^{*+}}^{\|\cdot\|}.\]
By Theorem \ref{positive hahn-banach w*} (2), if we prove $G_B^*$ is weakly$^*$ closed, then the claim $(\overline{F_B^*-B^{*+}})^+\subset F_B^*$ easily follows.
To this end, we take a sequence $\omega_{B,n}\in G_B^*$ such that $\omega_{B,n}\to\omega_B$ weakly$^*$ in $B^*$ to use the Krein-\v Smulian theorem and the separability of $B$.
Since $G_B^*$ is norm closed and $\omega_B$ belongs to the relative weak$^*$ closure of $G_B^*\cap C_\infty^*$ in $C_\infty^*$, where
\[C_n^*:=\{\omega_B'\in B^{*sa}:-\sum_{k\le n}(\omega_{B,k})_--(\omega_B)_-\le\omega_B'\},\qquad C_\infty^*:=\bigcup_nC_n^*,\]
so if we only check $G^*_B\cap C_n^*$ is weakly$^*$ closed in $B^*$ for each $n$, then we obtain $\omega_B\in G^*_B$ by Lemma \ref{krein-smulian}, which implies the weak$^*$ closedness of $G_B^*$.
Because every sequence in $C_n^*$ is lower dominated, now it is enough to show $\omega_B\in G_B^*$ when it is the weak$^*$ limit of a lower dominated sequence $\omega_{B,n}\in G_B^*$.
By Lemma \ref{lower dominated sequence}, we may assume $\omega_{B,n}\to\omega_B$ in norm.
There is also for each $n$ a sequence $\omega_{nm}\in F^*-A^{*+}$ such that $\omega_{nm}|_B\to\omega_{B,n}$ in norm by definition of $G_B^*$ and $F_B^*$.
Therefore, we can see $\omega_B\in G_B^*$ by taking a suitable diagonal sequence, so the claim $(\overline{F_B^*-B^{*+}})^+\subset F_B^*$ follows.

Now let $\omega\in(\overline{F^*-A^{*+}})^+$.
Take a net $\omega_i\in F^*-A^{*+}$ and $\varphi_i\in F^*$ such that $\omega_i\to\omega$ weakly$^*$ in $A^*$ and $\omega_i\le\varphi_i$ for each $i$.
For each separable C$^*$-subalgebra $B$ of $A$, we have $\varphi_i|_B\in F^*_{B}$ and $\omega_i|_B\in F^*_B-B^{*+}$ with the weak$^*$ convergence $\omega_i|_B\to\omega|_B$ in $B^*$, thus we have $\omega|_B\in(\overline{F_B^*-B^{*+}})^+=F_B^*$ because $B$ is separable.
If we consider the increasing net of all separable C$^*$-subalgebras $(B_j)_{j\in J}$ of $A$, then we have $\omega|_{B_j}\in F_{B_j}^*$ so that there is a net $\omega_{(j,\e)}\in F^*-A^{*+}$ based on the product directed set $\{(j,\e):j\in J,\ \e>0\}$ such that $\|\omega_{(j,\e)}|_{B_j}-\omega|_{B_j}\|<\e$ for each $(j,\e)$.
With this net, as an intermediate step, we will prove that $\omega$ belongs to the $\sigma(A^*,A_0^{**})$-closure of $F^*-A^{*+}$, where $A_0^{**}$ denotes the set of all elements of $A^{**}$ whose left or right support projection is $\sigma$-finite.
Let $x\in A_0^{**+}$ with $\|x\|\le1$, and let $p$ be the support projection of $x$.
Take a net $a_i\in A^+$ such that $\|a_i\|\le1$ for all $i$ and $a_i\to x$ $\sigma$-strongly using the Kaplansky density theorem, which implies $a_ip\to x$ $\sigma$-strongly.
Since $p$ is $\sigma$-finite so that on the $\sigma$-weakly closed left ideal $A^{**}p$ of $A^{**}$ its bounded part is $\sigma$-strongly metrizable, we can take a bounded sequence $a_n\in A^+$ such that $\|a_n\|\le1$ for all $n$ and $a_np\to x$ $\sigma$-strongly in $A^{**}$.

If we choose $j_0$ such that $a_n\in B_{j_0}$ for all $n$, then for any $j\succ j_0$, by taking limits $l\to\infty$, $m\to\infty$, and $n\to\infty$ in order on the inequality
\begin{align*}
|(\omega_{(j,\e)}-\omega)(x)|
&\le|(\omega_{(j,\e)}-\omega)(x-a_np)|+|(\omega_{(j,\e)}-\omega)(a_n(p-x^{1/m}))|\\
&\quad+|(\omega_{(j,\e)}-\omega)(a_n(x^{1/m}-a_l^{1/m}))|+|(\omega_{(j,\e)}-\omega)(a_na_l^{1/m})|,
\end{align*}
where the last term is uniformly estimated by $\e$ because $a_na_l^{1/m}\in B_j$ is uniformly bounded by one, we obtain $\lim_{(j,\e)}(\omega_{(j,\e)}-\omega)(x)=0$.
This proves that $\omega$ is contained in the $\sigma(A^*,A_0^{**})$-closure of $F^*-A^{*+}$.



Suppose now $\omega\notin F^*$.
Then, there exists $x\in A^{**+}$ such that $\omega(x^2)>1$ and $\omega'(x^2)\le1$ for all $\omega'\in F^*$ by Theorem \ref{positive hahn-banach w*} (2).
Let $\{p_i\}_{i\in I}$ be a maximal orthogonal family of $\sigma$-finite projections of the von Neumann algebra $A^{**}$ whose sum is the support projection of $x$.
If we consider order-preserving bounded linear maps $\Gamma:c_0(I)\to A^{**}$ and $\Gamma^*:A^*\to\ell^1(I)$ given by
\[\Gamma((c_i)_{i\in I}):=\sum_ic_ixp_ix,\qquad
\Gamma^*(\omega'):=(\omega'(xp_ix))_{i\in I},\]
then these maps are in dual, and $\Gamma$ is extended to the linear map $\Gamma^{**}:\ell^\infty(I)\to A^{**}$ continuous with respect to weak$^*$ topologies.
Observing that the left and right support projections of an arbitrary element of a von Neumann algebra are Murray-von Neumann equivalent, we can see $A_0^{**}$ is an algebraic ideal of $A^{**}$, and we have $\Gamma(c_0(I))\subset A_0^{**}$ due to the fact that each element of $c_0(I)$ has at most countably many non-zero components.
Since $\omega$ is an element of the $\sigma(A^*,A_0^{**})$-closure of $F^*-A^{*+}$, we have $\Gamma^*(\omega)\in \overline{\Gamma^*(F^*-A^{*+})}$, where the closure is taken in the weak$^*$ topology of $\ell^1(I)$.
Since the set $\overline{\Gamma^*(F^*-A^{*+})}$ is contained in the weak$^*$ closure of $\overline{\Gamma^*(F^*)}-\ell^1(I)^+$ by $\Gamma^*(F^*-A^{*+})\subset\Gamma^*(F^*)-\ell^1(I)^+$, whose positive part is $\overline{\Gamma^*(F^*)}$ because $c_0(I)$ is a commutative C$^*$-algebra.
Therefore, we have $\Gamma^*(\omega)\in\overline{\Gamma^*(F^*)}$.
For any $\delta>0$, if we choose $c\in c_0(I)^+$ such that $c\le1$ and $|\langle1_{\ell^\infty(I)}-c,\Gamma^*(\omega)\rangle|<\delta$ using the Kaplansky density, and choose $\omega'\in F^*$ such that $|\langle c,\Gamma^*(\omega)-\Gamma^*(\omega')\rangle|<\delta$, then we have a contradiction
\begin{align*}
1<\omega(x^2)&=\langle1_{\ell^\infty(I)},\Gamma^*(\omega)\rangle\approx_\delta\langle c,\Gamma^*(\omega)\rangle\\
&\approx_\delta\langle c,\Gamma^*(\omega')\rangle\le\langle1_{\ell^\infty(I)},\Gamma^*(\omega')\rangle=\omega'(x^2)\le1,
\end{align*}
where the relation symbol $\approx_\delta$ means that the difference converges to zero as $\delta\to0$, so finally we have $\omega\in F^*$.
\end{proof}




\section{Applications to weight theory}


The positive Hahn-Banach separation theorem implies a generalization of the Combes theorem on subadditive normal weights.
\begin{cor}
Let $M$ be a von Neumann algebra.
Then, there is a one-to-one correspondence
\[\begin{array}{ccc}
\left\{\emph{\begin{tabular}{c}subadditive normal\\weights of $M$\end{tabular}}\right\}&\leftrightarrow&\left\{\emph{\begin{tabular}{c}hereditary closed\\convex subsets of $M_*^+$\end{tabular}}\right\}\\[10pt]
\varphi&\mapsto&\{\omega\in M_*^+:\omega\le\varphi\}
\end{array}\]
\end{cor}



\end{document}