\documentclass[a4paper]{amsart}

\usepackage[T1]{fontenc}
\usepackage[bitstream-charter,cal]{mathdesign}
\linespread{1.15}

\newcommand{\e}{\varepsilon}

\theoremstyle{plain}
\newtheorem{thm}{Theorem}[section]
\newtheorem{lem}[thm]{Lemma}
\newtheorem{cor}[thm]{Corollary}
\theoremstyle{definition}
\newtheorem{defn}[thm]{Definition}


\title{Positive Hahn-Banach separations in operator algebras}
\author{Ikhan Choi}
\address{}
\subjclass[2020]{}

\begin{document}

\begin{abstract}

\end{abstract}

\maketitle

\section{Introduction}






\begin{itemize}
\item definition and properties of $f_\e(t):=(1+\e t)^{-1}t$
\item commutant Radon-Nikodym, relation between $\{\omega'\in M_*^+:\omega'\le\omega\}$ and $\{h\in\pi(M)'^+:h\le1\}$, order preserving linear map
\item Mazur lemma
\end{itemize}


\begin{defn}[Hereditary subsets]
Let $E$ be a partially ordered real vector space.
We say a subset $F$ of the positive cone $E^+$ is \emph{hereditary} if $0\le x\le y$ in $E$ and $y\in F$ imply $x\in F$, or equivalently $F=(F-E^+)^+$, where $F-E^+$ is the set of all positive elements of $E$ bounded above by an element of $F$.
A $*$-subalgebra $B$ of a $*$-algbera $A$ is called \emph{hereditary} if the positive cone $B^+$ is a hereditary subset of $A^+$.
We define the \emph{positive polar} of $F$ as the positive part of the real polar
\[F^{r+}:=\{x^*\in(E^*)^+:\sup_{x\in F}x^*(x)\le1\}.\]
\end{defn}
An example that is a non-hereditary closed convex subset of a C$^*$-algebra is $\mathbb{C}1$ in any unital C$^*$-algebra.


\begin{defn}[Lower dominated sequences]
Let $E$ be a partially ordered real vector space.
A sequence $x_n\in E$ is called \emph{lower dominated} if there is $x\in E$ such that $x\le x_n$ for all $n$.
If $E$ is the self-adjoint part of the predual of a von Neumann algebra where the Jordan decomposition holds, then we can change the definition such that $x\in -E^+$.
\end{defn}

\section{Positive Hahn-Banach separation theorems}

Now we start with the positive Hahn-Banach separation for von Neumann algebras, and will close this section with the same theorem for C$^*$-algebras.


\begin{thm}[Positive Hahn-Banach separation for von Neumann algebras]\label{positive hahn-banach w*}
Let $M$ be a von Neumann algebra.
\begin{enumerate}
\item If $F$ is a $\sigma$-weakly closed convex hereditary subset of $M^+$, then $F=F^{r+r+}$. In particular, if $x\in M^+\setminus F$, then there is $\omega\in M_*^+$ such that $\omega(x)>1$ and $\omega\le1$ on $F$.
\item If $F_*$ is a norm closed convex hereditary subset of $M_*^+$, then $F_*=F_*^{r+r+}$. In particular, if $\omega\in M_*^+\setminus F_*$, then there is $x\in M^+$ such that $\omega(x)>1$ and $x\le1$ on $F_*$.
\end{enumerate}
\end{thm}
\begin{proof}
(1)
Since the positive polar is represented as the real polar
\[F^{r+}=F^r\cap M_*^+=F^r\cap(-M^+)^r=(F\cup-M^+)^r=(F-M^+)^r,\]
the positive bipolar can be written as $F^{r+r+}=(F-M^+)^{rr+}=(\overline{F-M^+})^+$ by the usual real bipolar theorem, where the closure is for the $\sigma$-weak topology.
Because $F=(F-M^+)^+\subset(\overline{F-M^+})^+$, it suffices to prove the opposite inclusion $(\overline{F-M^+})^+\subset F$.

Let $x\in(\overline{F-M^+})^+$.
Take a net $x_i\in F-M^+$ such that $x_i\to x$ $\sigma$-strongly, and take a net $y_i\in F$ such that $x_i\le y_i$ for each $i$.
Suppose we may assume that the net $x_i$ is bounded.
For sufficiently small $\e$ so that the bounded net $x_i$ has the spectra in $[-(2\e)^{-1},\infty)$, we have $f_\e(x_i)\to f_\e(x)$ $\sigma$-strongly, and hence $\sigma$-weakly.
On the other hand, by the hereditarity and the $\sigma$-weak compactness of $F$, we may asumme that the bounded net $f_\e(y_i)\in F$ converges $\sigma$-weakly to a point of $F$ by taking a subnet.
Then, we have $f_\e(x)\in F-M^+$ by
\[0\le f_\e(x)=\lim_if_\e(x_i)\le\lim_if_\e(y_i)\in F,\]
thus we have $x\in F$ since $f_\e(x)\uparrow x$ as $\e\to0$.
What remains is to prove the existence of a bounded net $x_i\in F-M^+$ such that $x_i\to x$ $\sigma$-strongly.

Define a convex set
\[G:=\left\{x\in\overline{F-M^+}:\begin{tabular}{c}there is a sequence $x_m\in F-M^+$\\such that $-2x\le x_m\uparrow x$ $\sigma$-weakly\end{tabular}\right\}\subset M^{sa},\]
where $x_m$ denotes a sequence.
In fact, it has no critical issue on allowing $x_m$ to be uncountably indexed.
Since we clearly have $F-M^+\subset G$ and every non-decreasing net with supremum is bounded and $\sigma$-strongly convergent, it suffices to show that $G$, or equivalently its intersection with the closed unit ball by the Krein-Sm\v ulian theorem, is $\sigma$-strongly closed.
Let $x_i\in G$ be a net such that $\sup_i\|x_i\|\le1$ and $x_i\to x$ $\sigma$-strongly.
For each $i$, take a sequence $x_{im}\in F-M^+$ such that $-2x_i\le x_{im}\uparrow x_i$ $\sigma$-strongly as $m\to\infty$, and also take $y_{im}\in F$ such that $x_{im}\le y_{im}$.
Since $\|x_{im}\|\le2\|x_i\|\le2$ is bounded, it implies that there is a bounded net $x_j$ in $F-M^+$ such that $x_j\to x$ $\sigma$-strongly, and we can choose arbitrarily small $\e>0$ such that $\sigma(x_j)\subset[-(2\e)^{-1},\infty)$ for all $j$.
Since $f_\e(x_j)$ converges to $f_\e(x)$ $\sigma$-strongly and $f_\e(y_j)$ is a bounded net for each $\e>0$ so that we may assume that the net $f_\e(y_j)$ is $\sigma$-weakly covergent by taking a subnet, we have $f_\e(x)\in F-M^+$ by
\[f_\e(x)=\lim_jf_\e(x_j)\le\lim_jf_\e(y_j)\in F,\]
where the limits are in the $\sigma$-weak sense.
By taking $\e$ as any decreasingly convergent sequence to zero, we have $x\in G$, hence the closedness of $G$.


(2)
It is enough to prove $(\overline{F_*-M_*^+})^+\subset F_*$, where the closure is for the weak topology or equivalently in norm by the convexity of $F_*-M_*^+$, so we begin our proof by fixing $\omega\in(\overline{F_*-M_*^+})^+$.
For a sequence $\omega_n\in F_*-M_*^+$ such that $\omega_n\to\omega$ in norm of $M_*$, we can take $\varphi_n\in F_*$ such that $\omega_n\le\varphi_n$ for all $n$.
By modifying $\omega_n$ into $\omega_n-(\omega_n-\omega)_+\in F_*-M_*^+$ and taking a rapidly convergent subsequence, we may assume $\omega_n\le\omega$ and $\|\omega-\omega_n\|\le2^{-n}$ for all $n$.
If we consider the Gelfand-Naimark-Segal representation $\pi:M\to B(H)$ associated to a positive normal linear functional \[\widetilde\omega:=\sum_n(\omega-\omega_n)+\omega+\sum_n2^{-n}\left(\frac{[\omega_n]}{1+\|\omega_n\|}+\frac{\varphi_n}{1+\|\varphi_n\|}\right)\]
on $M$ with the canonical cyclic vector $\Omega$, we can construct commutant Radon-Nikodym derivatives $h,h_n,k_n\in\pi(M)'$ of $\omega,\omega_n,\varphi_n$ with respect to $\widetilde\omega$ respectively.
Since $-1\le h_n\le h$ is bounded, $h_n\to h$ in the weak operator topology of $\pi(M)'$.
By the Mazur lemma, we can take a net $h_i$ by convex combinations of $h_n$ such that $h_i\to h$ strongly in $\pi(M)'$, and the corresponding linear functionals $\omega_i$ and $\varphi_i$ satisfy $\omega_i\le\varphi_i$ with $\varphi_i\in F_*$ by the convexity of $F_*$ so that $\omega_i\in F_*-M_*^+$.
The net $h_i$ can be taken to be a sequence in fact because $\pi(M)'$ is $\sigma$-finite by the existence of the separating vector $\Omega$, but it is not necessary in here.
For each $i$ and $0<\e<1$, define
\[h_\e:=f_\e(h),\quad h_{i,\e}:=f_\e(h_i),\quad k_{i,\e}:=f_\e(k_i)\]
in $\pi(M)'$, where the functional calculi are well-defined because $-1\le h_i$ and $0\le h,k_i$ for all $i$, and define $k_\e$ as the $\sigma$-weak limit of the bounded net $k_{i,\e}$, which may be assumed to be $\sigma$-weakly convergent.
Define $\omega_\e,\omega_{i,\e},\varphi_{i,\e},\varphi_\e$ as the corresponding normal linear functionals on $M$ to $h_\e,h_{i,\e},k_{i,\e},k_\e$.
Note that $\varphi_i\in F_*$.
The hereditarity of $F_*$ and $0\le\varphi_{i,\e}\le\varphi_i$ imply $\varphi_{i,\e}\in F_*$, and the weak closedness of $F_*$ and the weak convergence $\varphi_{i,\e}\to\varphi_\e$ in $M_*$ imply $\varphi_\e\in F^*$.
From $\omega_i\le\varphi_i$, we can deduce $0\le\omega_\e\le\varphi_\e$ by considering the operator monotonicity $f_\e$ and taking the weak limit on $i$.
Thus again, the hereditarity of $F_*$ implies $\omega_\e\in F^*$, and the weak closedness of $F_*$ and the weak convergence $\omega_\e\to\omega$ in $M_*$ imply $\omega\in F^*$.
\end{proof}

Now we prepare some lemmas for the positive Hahn-Banach separation theorem for C$^*$-algebras.


\begin{lem}\label{lower dominated sequence}
Let $A$ be a C$^*$-algebra, and let $F^*$ be a weakly$^*$ closed convex hereditary subset of $A^{*+}$.
If $\omega\in A^{*sa}$ is approximated weakly$^*$ by a lower dominated sequence of $F^*-A^{*+}$, then it is approximated in norm by a sequence of $F^*-A^{*+}$.
\end{lem}
\begin{proof}
Let $\omega_n\in F^*-A^{*+}$, $\varphi_n\in F^*$, $\widetilde\omega_0\in A^{*+}$ be such that $\omega_n\to\omega$ weakly$^*$ in $A^*$ and $-\widetilde\omega_0\le\omega_n\le\varphi_n$ for all $n$.
Consider the Gelfan-Naimark-Segal representation $\pi:A\to B(H)$ of
\[\widetilde\omega:=\widetilde\omega_0+[\omega]+\sum_n2^{-n}\left(\frac{[\omega_n]}{1+\|\omega_n\|}+\frac{\varphi_n}{1+\|\varphi_n\|}\right)\]
with the canonical cyclic vector $\Omega\in H$.
Define the commutant Radon-Nikodym derivatives $h,h_n,k_n\in\pi(A)'$ of $\omega,\omega_n,\varphi_n$ with respect to $\widetilde\omega$.


Consider the range $0<\e\le\frac12$ for $\e$.
Since $-1\le h_n,h$ and $0\le k_n$, the functional calculus $h_{n,\e}:=f_\e(h_n)$ and $k_{n,\e}:=f_\e(k_n)$ are well-defined in $\pi(A)'$.
The bounded sequences $h_{n,\e}$ and $k_{n,\e}$ have weakly convergent subnets in $\pi(A)'$, and denote their limits by $h_\e$ and $k_\e$ respectively.
Be cautious that $h_\e':=f_\e(h)$ may not be equal to $h_\e$.
By the operator concavity of the function $f_\e$ and the $\sigma$-finiteness of $\pi(A)'$, the Mazur lemma retakes sequences $\omega_n\in F^*-A^{*+}$ and $\varphi_n\in F^*$ such that $\omega_n\le\varphi_n$ for all $n$ and $h_{n,\e}\to h_\e$ and $k_{n,\e}\to k_\e$ strongly.
We may assume
\[\|(h_{n,\e}-h_\e)\Omega\|<n^{-1},\qquad\|(h_{n,\e}-h_\e)h\Omega\|<n^{-1}\]
for all $n$ uniformly on $\e$, which will be used later.
Note also that we have the identity
\[(1+\e h)(h_\e'-h_{n,\e})(1+\e h)=(h-h_n)+\e(h-h_n)(1+\e h_n)^{-1}(h-h_n).\]


Denote by $\omega_{n,\e},\omega_\e,\omega_\e',\varphi_{n,\e},\varphi_\e$ the linear functionals in $A^{*sa}$ corresponded to operators in the commutant $h_{n,\e},h_\e,h_\e',k_{n,\e},k_\e\in\pi(A)'$.
It follows clearly that $\omega_{n,\e}\to\omega_\e$ and $\varphi_{n,\e}\to\varphi_\e$ as $n\to\infty$, and $\omega_\e'\uparrow\omega$ as $\e\to0$, weakly in $A^*$.
If we prove $\omega_\e'-\omega_\e\to0$ weakly in $A^*$ as $\e\to0$, then since $\omega_{n,\e}\le\varphi_{n,\e}\in F^*$ implies $\omega_\e\le\varphi_\e\in F^*$, we obtain the weak convergence $\omega_\e\to\omega$ in $A$ as $\e\to0$ with $\omega_\e\in F^*-A^{*+}$.
A desired sequence by applying the Mazur lemma on $\omega_\e$ after taking $\e$ to be a decreasing sequence that converges to zero.

Thus, what remains is to prove $\omega_\e'-\omega_\e\to0$ weakly in $A^*$ as $\e\to0$.
Fix $x\in A^{**}$ with $\|x\|\le1$.
The one-parameter family $(h_\e'-h_\e)\pi(x)\Omega$ of vectors is uniformly bounded on $0<\e\le\frac12$ by the uniform boundedness principle because for each $\eta\in H$, fixing any $n$, say $n=1$, we have
\begin{align*}
&|\langle(h_\e'-h_\e)\pi(x)\Omega,\eta\rangle|\\
&\quad\le|\langle(h_\e'-h_{1,\e})\pi(x)\Omega,\eta\rangle|+|\langle(h_{1,\e}-h_\e)\pi(x)\Omega,\eta\rangle|\\
&\quad\le|\langle(1+\e h)^{-1}(h-h_1)(1+\e h)^{-1}\pi(x)\Omega,\eta\rangle|\\
&\qquad+\e|\langle(1+\e h)^{-1}(h-h_1)(1+\e h_1)^{-1}(h-h_1)(1+\e h)^{-1}\pi(x)\Omega,\eta\rangle|\\
&\qquad+\|(h_{1,\e}-h_\e)\Omega\|\|\pi(x^*)\eta\|\\
&\quad\le4\|h-h_1\|\|\Omega\|\|\eta\|+4\|h-h_1\|^2\|\Omega\|\|\eta\|+\|\eta\|,
\end{align*}
which is uniformly bounded on $\e$.
We further have $(h_\e'-h_\e)\pi(x)\Omega\to0$ weakly in $H$ as $\e\to0$, which can be shown as follows.
By the boundedness of $(h_\e'-h_\e)\pi(x)\Omega$, it is enough to choose $\pi(b)\Omega$ with $b\in A$ satisfying $\|b\|\le1$ for the test vector. 
As $\|(h_\e'-h_\e)\pi(b)\Omega\|$ is uniformly bounded on $\e$ because $b\in A^{**}$, we can also prove $\|(h_\e'-h_\e)h\pi(b)\Omega\|$ is uniformly bounded in the same manner but using $\|(h_{1,\e}-h_\e)h\Omega\|<1$ instead of $\|(h_{1,\e}-h_\e)\Omega\|<1$. 
Choose their common bound $C>0$.
For an arbitrarily fixed $\delta>0$, take $a\in A$ such that $\|(\pi(x)-\pi(a))\Omega\|<\delta C^{-1}$ and $\|a\|\le1$ by the Kaplansky density, and fix $n$ such that $|(\omega-\omega_n)(b^*a)|<\delta$ and $n>\frac94\|\Omega\|\delta^{-1}$.
Then,
\begin{align*}
&|\langle(h_\e'-h_\e)\pi(x)\Omega,\pi(b)\Omega\rangle|\\
&\quad<|\langle(h_\e'-h_\e)\pi(a)\Omega,\pi(b)\Omega\rangle|+\delta\\
&\quad<|\langle(h_\e'-h_\e)(1+\e h)\pi(a)\Omega,(1+\e h)\pi(b)\Omega\rangle|+O(\e)+\delta\\
&\quad<|\langle(h_\e'-h_{n,\e})(1+\e h)\pi(a)\Omega,(1+\e h)\pi(b)\Omega\rangle|+\delta+O(\e)+\delta\\
&\quad\le|(\omega-\omega_n)(b^*a)|+\e|\langle(1+\e h_n)^{-1}(h-h_n)\pi(a_\e)\Omega,(h-h_n)\pi(b)\Omega\rangle|+\delta+O(\e)+\delta\\
&\quad<\delta+\e(1-\e)^{-1}\|(h-h_n)\|^2\|\Omega\|^2+\delta+O(\e)+\delta,
\end{align*}
where $O(\e)$ can be computed as $C(2\e+\e^2)\|\Omega\|$, so we have
\[\limsup_{\e\to0}|\langle(h_\e'-h_\e)\pi(x)\Omega,(1+\e h)\pi(b)\Omega\rangle|\le3\delta.\]
Since $\delta>0$ was taken arbitrarily, we finally have $(h_\e'-h_\e)\pi(x)\Omega\to0$ weakly in $H$, which implies $\omega_\e'-\omega_\e\to0$ weakly in $A^*$ as $\e\to0$.
\end{proof}


The following lemma is a modification of the Krein-\v Smulian theorem, and it can be proved in a similar way to the proof of the original theorem.

\begin{lem}\label{krein-smulian}
Let $A$ be a C$^*$-algebra, and $C^*_n$ be a non-decreasing sequence of weakly$^*$-closed convex subsets of $A^{*sa}$, whose union $C_\infty^*$ contains $A^{*+}$.
If a norm closed convex subset $G^*$ of $A^{*sa}$ has the property that $G^*\cap C^*_n$ is weakly$^*$ closed for each $n$, then $G^*\cap C_\infty^*$ is relatively weakly$^*$ closed in $C_\infty^*$.
\end{lem}
\begin{proof}
Fix an element $\omega_0$ of $C_\infty^*\setminus G^*$.
It is enough to construct an element $a$ of $A^{sa}$ separating a norm open ball centered at $\omega_0$ from $G^*$.
Since $G^*$ is norm closed, there exists $r>0$ such that $G^*\cap B(\omega_0,r)=\varnothing$.
By replacing $G^*$ to $r^{-1}(G^*-\omega_0)$ and $C_n^*$ to $r^{-1}(C_n^*-\omega_0)$, we may assume $G^*\cap B(0,1)=\varnothing$, and the claim follows if we prove there is $a\in A^{sa}$ separating $B(0,1)$ and $G^*$.
The condition $A^{*+}\subset C_\infty^*$ becomes $A^{*+}-\omega_0\subset C_\infty^*$.
Letting the index $n$ start from one, we may also replace $C_n^*$ to $n(C_n^*\cap B(0,1))$ since its union is still $C_\infty^*$.
Note that $C_n^*$ is bounded for each $n$, and we can easily see that $G^*\cap C_1^*=\varnothing$ and $n^{-1}C_n^*\subset(n+1)^{-1}C_{n+1}^*$.

Note that for any Banach space $X$, if $F$ is a bounded subset of $X$, then by endowing with the discrete topology on $F$, we have a natural bounded linear operator $\ell^1(F)\to X$ by completeness of $X$, with its dual $X^*\to\ell^\infty(F)$.
We will construct a bounded subset $F$ of $A^{sa}$ such that the subset $G^*\cap C_\infty^*$ of $A^{*sa}$ induces a subset of the smaller subspace $c_0(F)$ of $\ell^\infty(F)$ via the restriction map $A^{*sa}\to\ell^\infty(F)$, and also such that it satisfies $G^*\cap C_\infty^*\cap F^\circ=\varnothing$, where $F^\circ:=\{\omega\in A^{*sa}:\sup_{a\in F}|\omega(a)|\le1\}$ denotes the absolute polar of $F$.
If such a set $F\subset X$ exists, then the image of $G^*\cap C_\infty^*$ in $c_0(F)$ is a convex set disjoint to the closed unit ball of $c_0(F)$ by the condition $G^*\cap C_\infty^*\cap F^\circ=\varnothing$.
Therefore, there exists a separating linear functional $l\in\ell^1(F)$ by the Hahn-Banach separation, and it induces a linear functional separating $G^*$ and the unit ball of $A^{*sa}$.
Then, we are done.

Let $F_0:=\{0\}\subset A^{sa}$.
As an induction hypothesis on $n$, suppose for each $0\le k\le n-1$ we already have a finite subset $F_k$ of $(C_k^*)^\circ$ such that
\[G^*\cap C^*_n\cap\left(\bigcup_{k=0}^{n-1}F_k\right)^\circ=\varnothing.\]
If every finite subset $F_n$ of $(C_n^*)^\circ$ satisfies
\[G^*\cap C_{n+1}^*\cap\left(\bigcup_{k=0}^{n-1}F_k\right)^\circ\cap F_n^\circ\ne\varnothing,\]
then since they are weakly$^*$ compact, the finite intersection property leads a contradiction because the intersection of all absolute polars $F_n^\circ$ of finite subsets $F_n$ of $(C_n^*)^\circ$ is $C_n^*$, which is the polar of all union of finite subsets $F_n$ of $(C_n^*)^\circ$ by the bipolar theorem.
Thus, we can take a finite subset $F_n$ of $(C_n^*)^\circ$ such that
\[G^*\cap C_{n+1}^*\cap\left(\bigcup_{k=0}^nF_k\right)^\circ=\varnothing.\]
Let $F:=\bigcup_{k=0}^\infty F_k$.
Then, we have $G^*\cap C_\infty^*\cap F^\circ=\varnothing$, and every element of $C_\infty^*$ is restricted to $F$ to define an element of $c_0(F)$ because for each $\omega\in C_n^*$ and $k\ge0$ we have 
\[\omega(F_{n+k})\subset\omega((C_{n+k}^*)^\circ)\subset\frac n{n+k}\omega((C_n^*)^\circ)\subset[-\frac n{n+k},\frac n{n+k}].\]
Finally, for any $\omega\in A^{*sa}$, if we enumerate $F$ as a sequence $f_m$, then
\[|\omega(f_m)|\le|(\omega_+-\omega_0)(f_m)|+|(\omega_--\omega_0)(f_m)|\to0,\]
so the uniform boundedness principle concludes that $F$ is bounded.
Therefore, the set $F$ satisfies the properties we desired.
\end{proof}





\begin{thm}[Positive Hahn-Banach separation for C$^*$-algebras]
Let $A$ be a C$^*$-algebra.
\begin{enumerate}
\item If $F$ is a norm closed convex hereditary subset of $A^+$, then $F=F^{r+r+}$. In particular, if $a\in A^+\setminus F$, then there is $\omega\in A^{*+}$ such that $\omega(a)>1$ and $\omega\le1$ on $F$.
\item If $F^*$ is a weakly$^*$ closed convex hereditary subset of $A^{*+}$, then $F^*=(F^*)^{r+r+}$. In particular, if $\omega\in A^{*+}\setminus F^*$, then there is $a\in A^+$ such that $\omega(a)>1$ and $a\le1$ on $F^*$.
\end{enumerate}
\end{thm}
\begin{proof}
(1)
We directly prove the separation without invoking the arguments of positive bipolars.
Denote by $F^{**}$ the $\sigma$-weak closure of $F$ in the universal von Neumann algebra $A^{**}$.
We first show that $F^{**}$ is hereditary subset of $A^{**+}$.
Suppose $0\le x\le y$ in $A^{**}$ and $y\in F^{**}$.
Then, there is $z\in A^{**}$ such that $x^{\frac12}=zy^{\frac12}$.
Take bounded nets $b_i$ in $F$ and $c_i$ in $A$ such that $b_i\to y$ and $c_i\to z$ $\sigma$-strongly$^*$ in $A^{**}$ using the Kaplansky density.
We may assume the indices of these two nets are same.
Since both the multiplication and the involution of a von Neumann algebra on bounded parts are continuous in the $\sigma$-strong$^*$ topology, and since the square root on a positive bounded interval is a strongly continuous function, we have the $\sigma$-strong$^*$ limit
\[x=y^{\frac12}z^*zy^{\frac12}=\lim_ib_i^{\frac12}c_i^*c_ib_i^{\frac12},\]
so we obtain $x\in F^{**}$ from $b_i^{\frac12}c_i^*c_ib_i^{\frac12}\in F$.
Thus, $F^{**}$ is hereditary in $A^{**+}$.

Let $a\in A^+\setminus F$.
Observe that we have $a\in A^{**+}\setminus F^{**}$ because if $a\in F^{**}$, then we have a net $a_i$ in $F$ such that $a_i\to a$ $\sigma$-weakly in $A^{**}$, meaning that $a_i\to a$ weakly in $A$ and by the weak closedness of $F$ in $A$ we get a contradiction $a\in F^{**}\cap A=F$.
By Theorem \ref{positive hahn-banach w*}, there is $\omega\in A^{*+}$ such that $\omega(a)>1$ and $\omega\le1$ on $F\subset F^{**}$, so it completes the proof.

(2)
As same as above, our goal is to prove $(\overline{F^*-A^{*+}})^+\subset F^*$, so take $\omega\in(\overline{F^*-A^{*+}})^+$, where the closure is for the weak$^*$ topology.
We first prove it when $A$ is separable, which makes the weak$^*$ topology on any bounded part of $A^{*sa}$ metrizable.
Consider the following convex set
\[G^*:=\left\{\omega\in\overline{F^*-A^{*+}}:\begin{tabular}{c}there is a lower dominated sequence $\omega_n\in F^*-A^{*+}$\\such that $\omega_n\to\omega$ weakly$^*$ in $A^*$\end{tabular}\right\}.\]
We can easily see that $F^*-A^{*+}\subset G^*$, and we claim $G^*$ is the weak$^*$ closure.
If the claim is true, then we have $G^*=\overline{F^*-A^{*+}}$, and it follows that $\omega\in F^*$ by Lemma \ref{lower dominated sequence} and Theorem \ref{positive hahn-banach w*} (2), so we are done.
To prove $G^*$ is weakly$^*$ closed, we can take a sequence $\omega_n\in G^*$ such that $\omega_n\to\omega$ weakly$^*$ in $A^*$ by the Krein-\v Smulian theorem, and we will prove $\omega\in G^*$.
Because $\omega$ belongs to the relative weak$^*$ closure of $G^*\cap C_\infty^*$ in $C_\infty^*$, where
\[C_n^*:=\{\omega'\in A^{*sa}:-\sum_{k\le n}\omega_{k-}-\omega_-\le\omega'\},\qquad C_\infty^*:=\bigcup_nC_n^*,\]
so if we prove that $G^*$ is norm closed and $G^*\cap C_n^*$ is weakly$^*$ closed for each $n$, then we obtain $\omega\in G^*$ by Lemma \ref{krein-smulian}, and it completes the proof.
Since the limit of a norm convergent sequence in $G^*$ can be approximated by a lower dominated sequence in $G^*$ as in the proof of Theorem \ref{positive hahn-banach w*} (2), and since every sequence in $C_n^*$ is lower dominated, now it suffices to show $\omega\in G^*$ when it is the weak$^*$ limit of a lower dominated sequence $\omega_n\in G^*$.
Take $\widetilde\omega\in A^{*+}$ such that $-\widetilde\omega\le\omega_n$ for all $n$.
Since $\omega_n$ is weakly$^*$ approximated by a lower dominated sequence in $F^*-A^{*+}$ by definition of $G^*$, applying Lemma \ref{lower dominated sequence} for each $n$, we can find a sequence $\omega_{nm}\in F^*-A^{*+}$ such that $\omega_{nm}\to\omega_n$ in norm of $A^*$ as $m\to\infty$.
After modifying $\omega_{nm}$ into $\omega_{nm}-(\omega_{nm}-\omega_n)_+\in F^*-A^{*+}$ to assume $\omega_{nm}\le\omega_n$, if we take a subsequence to have $\|\omega_n-\omega_{nm}\|<2^{-(n+m)}$, then $\widetilde\omega_n:=\sum_m(\omega_n-\omega_{nm})$ satisfies $-\widetilde\omega_n\le\omega_{nm}-\omega_n$ for all $n$ and $m$, and $\|\widetilde\omega_n\|\le2^{-n}$.
Then, for the diagonal sequence $\omega_{nn}\in F^*-A^{*+}$, we have $\omega_{nn}\to\omega$ weakly$^*$ by
\[|(\omega_{nn}-\omega)(a)|\le|(\omega_{nn}-\omega_n)(a)|+|(\omega_n-\omega)(a)|\le2^{-2n}\|a\|+|(\omega_n-\omega)(a)|\to0\]
as $n\to\infty$ for each $a\in A$, and it is lower dominated by $-\widetilde\omega-\sum_n\widetilde\omega_n$, therefore we get the claim $\omega\in G^*$.



Now we consider a general C$^*$-algebra $A$.
For a separable C$^*$-subalgebra $B$ of $A$, we define a set
\[F_B^*:=\{\omega\in B^{*+}:\text{there is $\varphi\in F^*$ such that $\omega\le\varphi$ on $B^+$}\}.\]
It is clearly a convex hereditary subset of $B^{*+}$, and to prove the weak$^*$ closedness via the Krein-\v Smulian theorem, take a sequence $\omega_{B,n}\in F_B^*$ such that $\omega_{B,n}\to\omega_B$ weakly$^*$ in $B^*$.
Let $\varphi_n\in F^*$ be a sequence such that $\omega_{B,n}(b)\le\varphi_n(b)$ on $b\in B^+$, and let $\omega_n\in A^{*+}$ be the extension of $\omega_{B,n}$ for each $n$.
Consider the Gelfand-Naimark-Segal representation $\pi:A\to B(H)$ associated to the positive linear functional
\[\widetilde\omega:=\sum_n2^{-n}\left(\frac{\omega_n}{1+\|\omega_n\|}+\frac{\varphi_n}{1+\|\varphi_n\|}\right),\]
with the canonical cyclic vector $\Omega$.
Let $p\in B(H)$ be the orthogonal projection onto the closed linear subspace $\overline{\pi(B)\Omega}\subset H$.
Then, $\omega_n(b)\le\varphi_n(b)$ on $b\in B^+$ implies $ph_np\le pk_np$, so it follows that $ph_{n,\e}p\le pk_{n,\e}p$. 
Taking weakly convergent subnets of $h_{n,\e}$ and $k_{n,\e}$, we may define the limits $h_\e$ and $k_\e$, and $ph_\e p\le pk_\e p$ implies that the corresponding functionals have the relations $\omega_\e(b)\le\varphi_\e(b)$ for all $b\in B^+$.
We clearly have $\varphi_\e\in F^*$, so $\omega_{B,\e}=\omega_\e|_B\in F_B^*$(?).
Since $\omega_{B,\e}\uparrow\omega_B$ weakly in $B^*$, we only need to prove the convex set $F_B^*$ is norm closed.
Take a sequence $\omega_{B,n}\in F_B^*$ again, but at this time such that $\omega_{B,n}\to\omega_B$ in norm of $B^*$, together with $\varphi_n\in F^*$ such that $\omega_{B,n}(b)\le\varphi_n(b)$ on $b\in B^+$....


Let $\omega\in(\overline{F^*-A^{*+}})^+$, where the closure is taken in the weak$^*$ topology.
Our goal is to show $\omega\in F^*$.
Take a net $\omega_i\in F^*-A^{*+}$ and $\varphi_i\in F^*$ such that $\omega_i\to\omega$ weakly$^*$ in $A^*$ and $\omega_i\le\varphi_i$ for each $i$.
We have $\varphi_i|_B\in F^*_{B}$ and $\omega_i|_B\in F^*_B-B^{*+}$, with the weak$^*$ convergence $\omega_i|_B\to\omega_B$ in $B^*$, thus we have $\omega|_B\in(\overline{F_B^*-B^{*+}})^+=F_B^*$ because $B$ is separable.
If we consider the non-decreasing net of all separable C$^*$-subalgebras $B_j$ of $A$, then the restriction $\omega|_{B_j}$ of $\omega$ on $B_j$ belongs to the set $F_{B_j}^*$, so there is a net $\varphi_j\in F^*$ such that $\omega(b)\le\varphi_j(b)$ on $b\in B_j^+$ for each $j$, and by the Hahn-Banach extension, we obtain a net $\omega_j\in F^*-A^{*+}$ such that $\omega_j(b)=\omega(b)$ on $b\in B_j^+$ for all $j$.

Let $A_0^{**}$ be the set of all elements of $A^{**}$ whose left or right support projection is $\sigma$-finite.
It is known that $A_0^{**}$ is an algebraic ideal of $A^{**}$.
Let $x\in A_0^{**+}$.

$(\omega_j-\omega)(x^2)\to0$.

Thus, $\omega$ belongs to the $\sigma(A^*,A_0^{**})$-closure of $F^*-A^{*+}$.


Suppose $\omega$ does not belong to the weak closure of $F^*-A^{*+}$.
Then, there is $x\in A^{**+}$ such that $\omega(x^2)>1$ and $x^2\le1$ on $F^*-A^{*+}$ by Theorem \ref{positive hahn-banach w*} (2).
Let $\{p_i\}_{i\in I}$ be a maximal orthogonal family of $\sigma$-finite projections of the von Neumann algebra $A^{**}$.
Consider the bounded linear maps
\begin{align*}
\Gamma&:c_0(I)\to A:(\lambda_i)_{i\in I}\mapsto\sum_i\lambda_ixp_ix,\\
\Gamma^*&:A^*\to\ell^1(I):\omega'\mapsto(\omega'(xp_ix))_{i\in I},\\
\Gamma^{**}&:\ell^\infty(I)\to A^{**}:(\lambda_i)_{i\in I}\mapsto\sum_i\lambda_ixp_ix,
\end{align*}

$\Gamma$ is a topological embedding so that $\Gamma^*$ is surjective?

$\Gamma^*(F^*)$ weakly$^*$ closed convex hereditary?

$\Gamma^*(F^*-A^{*+})=\Gamma^*(F^*)-\ell^1(I)^+$?

We can check $\Gamma^*(\omega_j)\to\Gamma^*(\omega)$ weakly$^*$ so that $\Gamma^*(\omega)\in\Gamma^*(F^*)$.
We can take a net $\omega_k\in F^*-A^{*+}$ such that $\Gamma^*(\omega_k)\to\Gamma^*(\omega)$ weakly, and it follows a contradiction
\[1\ge\omega_k(x^2)=\langle1,\Gamma^*(\omega_k)\rangle\to\langle1,\Gamma^*(\omega)\rangle=\omega(x^2)>1.\]
Therefore, $\omega$ is a positive functional contained in the weak closure of $F^*-A^{*+}$, so $\omega\in F^*$ by Theorem \ref{positive hahn-banach w*} (2).




\[-\]


If we prove the closability of positive quadratic forms...

If we prove $F_B^*$ is weakly$^*$ closed, then there is a net $\varphi_{j,\e}\in F^*$ such that $\omega(b)\le\varphi_j(b)$ for $b\in B_j^+$, and hence $\omega(y)\le\varphi_j(y)$ for $y\in B_j^{**+}$.
Assuming $k_{j,\e}\to k_\e$ weakly in $\pi(A)'$, we have $\varphi_{j,\e}(x)\to\varphi_\e(x)$ for $x\in\mathfrak{M}$.

If $\varphi_{j,\e}(x)\to\varphi_\e(x)$ for $x\in\mathfrak{M}$ implies $\varphi_{j,\e}(x)\to\varphi_\e(x)$ for all $x\in A^{**}$, then $\varphi_\e\in F^*$.

If $\mathfrak{M}\cap B_j^{**}$ is $\sigma$-weakly dense in $B_j^{**}$, then $\omega_\e(y)\le\varphi_{j,\e}(y)$ for $y\in B_j^{**+}$.
Then, we have $\omega_\e(x)\le\varphi_\e(x)$ for all $x\in A^{**+}$.
Then, $\varphi_\e\in F^*$ implies $\omega_\e\in F^*$.
Then, $\omega\in F^*$.

\[-\]


If $\widetilde\omega_i$ is a positive norm preserving extension of $\omega_i$, then $\widetilde\omega_i\le\varphi_i$? no.


$\omega_n\to\omega$ in norm and $\omega_n\le\omega$.
Let 


Considering $k_{i,\e}\to k_\e$ so that $\varphi_\e\in F^*$.
We need weak$^*$ convergence $\varphi_{i,\e}\to\varphi_\e$.

$\widetilde\omega_{i,\e}\le\varphi_{i,\e}$

$\widetilde\omega_\e\le\varphi_\e$

$\varphi_i=\omega_i^\sim+(\varphi_i-\omega_i)^\sim$ on $B$

$\|\omega_i^\sim+(\varphi_i-\omega_i)^\sim\|\le\|\omega_i^\sim\|+\|(\varphi_i-\omega_i)^\sim\|=\|\omega_i\|+\|\varphi_i|_B-\omega_i\|\le\|\varphi_i|_B\|$


Here we let $\psi$ be a faithful semi-finite normal weight on $A^{**}$, and let $\pi:A^{**}\to B(H)$ be the Gelfand-Naimark-Segal representation associated to $\psi$, together with the left $A^{**}$-liner map $\Lambda:\mathfrak{N}_\psi\to H$ of dense range such that $\psi(x^*x)=\|\Lambda(x)\|^2$ for all $x\in\mathfrak{N}_\psi$.
Note that because the weight $\psi$ is faithful and semi-finite, $\Lambda$ is injective and $\sigma$-weakly densely defined, meaning that $\mathfrak{M}_\psi$ is a hereditary $\sigma$-weakly dense $*$-subalgebra of $A^{**}$.
Construct the commutant Radon-Nikodym derivatives $h,k_j$ of $\omega,\varphi_j$ with respect to $\psi$.
Here $k_j$ is a positive self-adjoint operator defined by the Friedrichs extension such that $\operatorname{ran}\Lambda\subset\operatorname{dom}k_j$ for all $j$.
Taking a subnet, we may assume that there is $k_\e\in\pi(A)'^+$ satisfying $f_\e(k_j)\to k_\e$ $\sigma$-weakly.
Because of the operator concavity of $f_\e$ (more detail), we can take a net $\varphi_l\in F^*$ such that $f_\e(k_l)\to k_\e$ $\sigma$-strongly, where $k_l$ are again the commutant Radon-Nikodym derivatives of $\varphi_l$ defined by the Friedrichs extension.
Since  is a strongly continuous function, we have $(f_\e(k_l)-k_\e)\to0$ $\sigma$-strongly, so if we define $\varphi_{l,\e}\in F^*-A^{*+}$ and $\varphi_\e\in A^{*+}$ such that
\[\varphi_{l,\e}(x^*x):=\langle(f_\e(k_l)-(f_\e(k_l)-k_\e)_+)\Lambda(x),\Lambda(x)\rangle,\quad\varphi_\e(x^*x):=\langle k_\e\Lambda(x),\Lambda(x)\rangle\]
for each $x\in\mathfrak{N}_\psi$, then we have $\varphi_{l,\e}\to\varphi_\e$ pointwisely on $\mathfrak{M}_\psi$ and $\varphi_{l,\e}\le\varphi_\e$ for all $l$.

How to dominate $\varphi_{l,\e}$ from below?


By Lemma \ref{lower dominated sequence}, we have $\varphi_{l,\e}\to\varphi_\e$ weakly in $A^*$, so Theorem \ref{positive hahn-banach w*} (2) implies that $\varphi_\e\in(\overline{F^*-A^{*+}}^w)^+=F^*$.
If we define $\omega_\e\in A^{*+}$ and $\varphi_{j,\e}\in F^*$ by
\[\omega_\e(x^*x):=\langle f_\e(h)\Lambda(x),\Lambda(x)\rangle,\quad\varphi_{j,\e}(x^*x):=\langle f_\e(k_j)\Lambda(x),\Lambda(x)\rangle\]
for each $x\in\mathfrak{N}_\psi$, then since $\omega\le\varphi_j$ on $B_j^+$ implies $\omega_\e\le\varphi_{j,\e}$ on $B_j^+$, the weak$^*$ limit $\omega_\e\le\lim_j\varphi_{j,\e}=\varphi_\e$ deduces $\omega_\e\in F^*-A^{*+}$.
Since $\omega_\e\to\omega$ pointwisely on $\mathfrak{M}_\psi$ and $0\le\omega_\e\le\omega$ for all $0<\e$, we have $\omega\in F^*$ by Lemma \ref{lower dominated sequence} and Theorem \ref{positive hahn-banach w*} (2).

\end{proof}




\section{Applications to weight theory}

\begin{cor}
Let $M$ be a von Neumann algebra.
Then, there is a one-to-one correspondence
\[\begin{array}{ccc}
\left\{\emph{\begin{tabular}{c}subadditive normal\\weights of $M$\end{tabular}}\right\}&\leftrightarrow&\left\{\emph{\begin{tabular}{c}hereditary closed\\convex subsets of $M_*^+$\end{tabular}}\right\}\\[10pt]
\varphi&\mapsto&\{\omega\in M_*^+:\omega\le\varphi\}
\end{array}\]
\end{cor}



\end{document}