\documentclass{kms-j}

%%% Start of the area for technical editor.
\newcommand{\publname}{J.~Korean Math.~Soc.}
%\newcommand{\doiname}{http://dx.doi.org/10.4134/JKMS}
\issueinfo{}% volume number
  {}%        % issue number
  {}%        % month
  {}%     % year
\pagespan{1}{}
%\received{Received January 5, 2005}
%\received{Received August 25, 2005;\enspace Revised October 20, 2005}
\copyrightinfo{}%              % copyright year
  {Korean Mathematical Society}% copyright holder
%%% End of the area for technical editor.

\usepackage{graphicx}
\allowdisplaybreaks

\theoremstyle{plain}
\newtheorem{theorem}{Theorem}[section]
\newtheorem{proposition}[theorem]{Proposition}
\newtheorem{lemma}[theorem]{Lemma}
\newtheorem{corollary}[theorem]{Corollary}

\theoremstyle{definition}
\newtheorem*{definition}{Definition}

\theoremstyle{remark}
\newtheorem{remark}[theorem]{Remark}

\begin{document}

\title[Cutting]
{Cutting}

\author[author]{author}
\address{address}
\email{email}

\subjclass{subclass}
\keywords{keywords}

\begin{abstract}
abstrsct
\end{abstract}

\maketitle

\section{Introduction}

\section{Definition}
\begin{definition}
Let $O$ be the pole of the plane $U$ that has polar coordinate $(r,\theta)$ and $M$ be a surface that is parameterized by an isometry $\phi : U\backslash \{ O\} \to U \times {\mathbb R}$ such that $\phi(P)\to(0,\theta,z)$ as $P\to O$ for a positive real number $z$.
Then $M$ is developable surface that forms generalized cone from Lemma *.
Suppose that $\gamma$ is Jordan curve lying on $U\backslash \{ O\}$.
If $z$-coordinate of any point of the image $\phi(\gamma)$ vanishes, the curve $\gamma$ is called {\it cutting} with {\it folding} $\phi$.
\end{definition}

\begin{theorem}
If a Jordan curve $\gamma$ lying on $U\backslash \{ O\}$ is cutting, then the pole $O$ belongs to interior of $\gamma$.
Moreover, there exists a bijective function between the points on $\gamma$ and angular coordinates of each point.
\end{theorem}

\begin{proof}
\end{proof}

\begin{definition}
Let a mapping $\tau :U\to U$ be {\it folding transformation} of folding $\phi$ if $\tau\circ\phi ^{-1}$ is a projection such that $\tau\circ\phi ^{-1}(r,\theta,z)=(r,\theta)$.
\end{definition}











\section{}
\begin{theorem}\label{pExists}
For certain positive real number $Z$, if a function $q(z,\theta)$ satisfies the followings:

{\rm (1)} $q(z,\theta)$ is continuous and bounded for $z$ and $\theta$;

{\rm (2)} $q(z,\theta)$ is strictly increasing for $z$;

{\rm (3)} $q(z,\theta)$ is periodic with period $2\pi$ for $\theta$;

{\rm (4)} there is an interval $[\alpha,\beta)\subset [0,2\pi)$ s.t. $q(z,\theta)$ is monotonic over $\theta\in[\alpha,\beta)$;

{\rm (5)} $q(0,\theta)=1$ for all $\theta$ in the domain
\\for all real number $z$ in $(0,Z)$ and $\theta$, then there exists a real number $z_0\in(0,Z)$ such that for all $z\in(0,z_0)$ there is a function $p(z,\theta)$ such that:

{\rm (6)} $|p(z,\theta)|=q(z,\theta)$;

{\rm (7)} $p(z,\theta)$ is piecewise continuous for $\theta$;

{\rm (8)} $p(z,\theta)$ has simple folding curve for $\theta$;

{\rm (9)} $\int_0^{2\pi} p(z,\theta)\,d\theta=2\pi$

\end{theorem}

\begin{proof}
Since $q=1$ where $z=0$ and $q(z,\theta)$ is strictly increasing for $z$, a real number $\phi$ is positive which is defined such that
$$\phi=\int_0 ^{2\pi} q(Z,\theta)\,d\theta-2\pi>0.$$
Let $Q(z;a,b)$ be a function such that
$$Q(z;a,b)=\int_0^{2\pi} q(z,\theta)\,d\theta-2\int_{\frac{2a+b}3}^{\frac{a+2b}3} q(Z,\theta)\,d\theta.$$
That $q(Z,\theta)$ is bounded implies
$$\int_{\frac{2a+b}3}^{\frac{a+2b}3} q(Z,\theta)\,d\theta\le\frac{b-a}3\sup\{q(Z,\theta):\theta\in{\mathbb R}\}$$
for all interval $[a,b)$, therefore if we take an interval $[a_0,b_0)\subset[\alpha,\beta)$ whose length is less than $3\phi/2\sup\{q(Z,\theta)\}$, we have
\begin{eqnarray}\label{Q2pi}
Q(Z;a_0,b_0)>2\pi.
\end{eqnarray}
$Q(0;a_0,b_0)=2\pi-\frac23(b-a)<2\pi$ and the inequality (\ref{Q2pi}) imply that there exists $z_0\in(0,Z)$ which makes $Q(z_0;a,b)$ be supposed to be $2\pi$ according to IVT.
It means that there is an interval $[a_0,b_0)$ such that $q(z,\theta)$ is monotonic over $\theta\in[a_0,b_0)$ and there exists $z_0\in(0,Z)$ such that $Q(z_0;a,b)=2\pi$

To prove the proposition for all $z<z_0$, consider a positive real number $t$ less than $(b-a)/2$.
$Q(z;a_0+t,b_0-t)$ is continuous for $t$ and
$$Q(z;a_0,b_0)<Q(z_0;a_0,b_0)=2\pi$$
$$Q(z;a_0+t,b_0-t)=0<2\pi$$
imply there exists $t\in(0,(b-a)/2)$ such that $Q(z;a,b)=2\pi$ letting $[a,b)=[a_0+t,b_0-t)$. $q(z,\theta)$ is obviously monotonic over $\theta\in[a,b)$.

Let $p(z,\theta)$ be a function such that
$$p(z,\theta)=
\begin{cases}
-q(z,\theta), & {\text{if }}\theta\in[a,b) \\
q(z,\theta), & {\text{if }}\theta\not\in[a,b)
\end{cases}
$$
where $a,b$ is the numbers determined above.
Then we can prove the function $p(z,\theta)$ satisfy the conditions.
\end{proof}

In Theorem \ref{pExists}, let trivial supremum of $z$ defined the supremum of $Z$, nontrivial supremum of $z$ defined the supremum of $z_0$.

\begin{thebibliography}{99}

\bibitem{Bl} B. Blackadar, {\it $K$-Theory for Operator Algebras}, Math. Sci. Res. Inst. Publ. {\bf 5}, Springer-Verlag, New York, 1986.

\end{thebibliography}

\end{document}

