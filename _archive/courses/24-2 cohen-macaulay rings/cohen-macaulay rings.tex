\documentclass{../../../small}
\usepackage{../../../ikhanchoi}

\renewcommand{\fd}{\operatorname{fd}}
\renewcommand{\pd}{\operatorname{pd}}
\newcommand{\gldim}{\operatorname{gldim}}
\newcommand{\depth}{\operatorname{depth}}
\newcommand{\Ass}{\operatorname{Ass}}
\newcommand{\Ann}{\operatorname{Ann}}
\newcommand{\hgt}{\operatorname{ht}}

\begin{document}

\title{Cohen-Macaulay Rings}
\author{Ikhan Choi\\Lectured by \\University of Tokyo, Spring 2024}
\maketitle

\newpage
\section{Day 1: October 2}

\begin{thm*}[0.1]
Let $A$ be a noetherian local ring.
If $A$ is regular, then $A_\fp$ is regular for $\fp\in\Spec A$.
\end{thm*}

\begin{thm*}[0.2, Auslander-Buchsbaum-Serre]
Let $A$ be a noetherian local ring.
Then, $A$ is regular if and only if it has finite global dimension.
\end{thm*}

\begin{pf}[Proof of (0.2) $\Rightarrow$ (0.1)]
Let $N$ be a $A_\fp$-module and let $g:=\gldim A$ so that there is a projective resolution of $N$ in $\Mod_A$ of length at most $g$.
Tensoring with $A_\fp$, we have a projective resolution
\[0\to P_g\to\cdots\to P_0\to N\otimes_AA_\fp\cong N\to0,\]
of $N$ in $\Mod_{A_\fp}$, so we have $\gldim A_\fp\le g$, and we are done.
\end{pf}

\begin{thm*}[0.3, Matlis duality]
Let $A$ be a complete noetherian local ring.
Then, there is a one-to-one correspondence between noetherian $A$-modules and artinian $A$-modules, which maps a noetherian module $M$ to $\Hom_A(M,E)$ and conversely an artinian module $N$ to $\Hom_A(N,E)$, where $E$ is the injective hull of $A/\fm$.
\end{thm*}

\begin{thm*}[0.4, Local duality]
Let $A$ be a Cohen-Macaulay complete noetherian local ring.
Let $\omega_A$ be the canonical $A$-module, and $M$ be a finitely generated $A$-module.
Then, $\Hom_A(-,E):H_m^i(M)\leftrightarrow\Ext_A^{d-i}(M,\omega_A)$ is an isomorphism.
\end{thm*}
\begin{rmk*}
The canonical module $\omega_A$ is finitely generated, and $H_m^i(M)$ is artinian.
\end{rmk*}

\section*{Review}

\begin{rmk*}
If $A$ is noetherian and $M$ is a finitely generated $A$-module, then there is a free resolution of $M$ consisting of finitely generated free modules. (kernel is finitely generated by noetherianness)
\end{rmk*}

Recall that $\Tor^A_i(M,N)$ is defined by the homology groups a chain complex obtained by taking $-\otimes_AN$ on a flat resolution of $M$.
Note $\Tor^A_0(M,N)=(F_0\otimes_AN)/(F_1\otimes_AN)=M\otimes_AN$ because tensoring preserves cokernels.

Non-trivial facts: (1) does not depend on the choice of flat resolutions (2) symmetry (3) long exact sequence (4) preserves possibly infinite direct sums and filtered colimits in each variable (5) projective resolutions are also possible instead of flat resolutions.

\begin{ex*}
Since the Tor functor preserves direct sums, we can compute the Tor group for every finitely generated abelian groups if we compute for cyclic groups.
Let $A:=\Z$ and $M:=\Z/d\Z$ for $d\in\Z_{>0}$.
For $N\in\Mod_A$, by considering $0\to\Z\xrightarrow{\times d}\Z\to\Z/d\Z\to0$, we have
\[\Tor_i^\Z(M,N)\cong\begin{cases}N/dN&\text{ if }i=0,\\\{n\in N:dn=0\}&\text{ if }i=1,\\0&\text{ if }i\ge2.\end{cases}\]
The name of Tor comes from $\{n\in N:dn=0\}$.
This argument is same for integral domains $A$.
\end{ex*}

Recall that $\Ext_A^i(M,N)$ is defined by the cohomology groups a cochain complex obtained by either taking $\Hom_A(-,N)$ on a projective resolution of $M$ or taking $\Hom_A(M,-)$ on an injective resolution of $N$.
Note that $\Ext_A^0(M,N)=\Hom_A(M,N)$.

\begin{ex*}
If $A$ is noetherian and $M$ is a finitely generated $A$-module, then $M$ is flat if and only if $M$ is projective.
If $A$ is a Dedekind domain, then an $A$-module $M$ is torsion-free if and only if $M$ is flat, and every ideal is projective(=flat).
\end{ex*}



\newpage
\section{Day 1: October 9}

flat dimension

\begin{prop*}[2.4]
Let $A$ be a ring and $M$ an $A$-module.
TFAE:
\begin{enumerate}[(i)]
\item $\fd_A(M)\le d$,
\item $\Tor_i^A(M,N)=0$ for all $i\ge d+1$ and all $A$-modules $N$.
\item $\Tor_{d+1}^A(M,N)=0$ for all $A$-modules $N$.
\item If
\[0\to E\to F_{d-1}\to\cdots\to F_0\to M\to0\]
is exact and $F_i$ flat, then $E$ flat.
\end{enumerate}
\end{prop*}
\begin{pf}
(iv)$\Rightarrow$(i)$\Rightarrow$(ii)$\Rightarrow$(iii) are clear.

(iii)$\Rightarrow$(iv)
Take a flat resolution of $E$ as
\[\cdots\to F_d\to E.\]
Then we can construct a flat resolution of $M$ by concatenation
\[\cdots\to F_d\to F_{d-1}\to\cdots\to F_0\to M\to0.\]
Then, for any $A$-module $N$ by tensoring $N$ and taking homology, we can compute
\[0=\Tor_{d+1}^A(M,N)=\frac{\ker(F_{d+1}\otimes N\to F_d\otimes N)}{\im(F_{d+2}\otimes N\to F_{d+1}\otimes N)}=\Tor_1^A(E,N).\]
By (2.5), $E$ is flat.
\end{pf}

\begin{lem*}[2.5]
Let $A$ be a ring and $M$ an $A$-module.
TFAE:
\begin{enumerate}[(i)]
\item $M$ is flat.
\item $\Tor_i^A(M,N)=0$ for all $i\ge1$ and all $A$-modules $N$.
\item $\Tor_i^A(M,A/I)=0$ for all $i\ge1$ and all ideals $I$ of $A$.
\end{enumerate}
\end{lem*}
\begin{pf}
By (2.2), (i) is equivalent to $\fd_A(M)=0$, which is equivalent to (ii) and it implies (iii).
For (iii)$\Rightarrow$(i), since
\[0=\Tor_1^A(A/I,M)\to I\otimes M\to A\otimes M\to A/I\otimes M\to0\]
is exact, $M$ is flat.
\end{pf}


\begin{prop*}[2.6]
Let $A$ be a ring.
Then,
\[\sup_{M\in\Mod_A}\fd_A(M)=\sup_{I\in\Mod_A}\fd_A(A/I)=\sup_{M,N\in\Mod_A}\{d\in\Z:\Tor_d^A(M,N)\ne0\}.\]
\end{prop*}
\begin{pf}
(2)\le(1) is clear, and (1)=(3) by (2.4).
For (1)\le(2), take $M\in\Mod_A$ and tensor on the exact sequence
\[0\to E\to F_{d-1}\to\cdots\to F_0\to M\to0,\]
we have $0=\Tor_{d+1}^A(A/I,M)=\Tor_1^A(A/I,E)$ so that $E$ is flat.
\end{pf}

\begin{defn*}[2.7]
Let $A$ be a ring and $M$ be an $A$-module.
Then, the \emph{projective dimension} $\pd_A(M)$ is the smallest $n\in\Z_{\ge0}$ such that there is a projective resolution
\[0\to P_n\to\cdots\to P_0\to M\to0.\]
If no such a projective resolution of $M$, then we write $\pd_A(M)=\infty$.
\end{defn*}

\begin{prop*}[2.8]
Let $A$ be a ring.
Then,
\[\sup_{M\in\Mod_A}\pd_A(M)=\sup_{I\in\Mod_A}\pd_A(A/I)=\sup_{M,N\in\Mod_A}\{d\in\Z:\Ext^d_A(M,N)\ne0\}.\]
\end{prop*}

\begin{prop*}[2.9]
Let $A$ be a noetherian ring and $M$ be a finitely generated $A$-module.
\begin{parts}
\item $M$ is flat if and only if $M$ is projective.
\item $\fd_A(M)=\pd_A(M)$.
\end{parts}
\end{prop*}
\begin{pf}
Sketch.
(a)
We may assume $A$ is local, which is the well-known Kaplansky theorem on projective modules.

(b)
We basically have $\fd_A(M)\le\pd_A(M)$.
By (a), we are done..?
\end{pf}

\begin{thm*}[2.10]
If $A$ is a noetherian ring, then $\sup_M\fd_A(M)=\sup_M\pd_A(M)$.
\end{thm*}
\begin{pf}
Since we may assume $M$ is a quotient of $A$, which is finitely generated.
By (2.9), we are done.
\end{pf}

\begin{ex*}[2.11]
For a field $K$, $\gldim K=\sup_I\fd_K(K/I)=0$, since $I=0$ or $I=K$.
\end{ex*}
\begin{ex*}[2.12]
For an ideal $I$ of $\Z$, $\fd_\Z(\Z/I)=0$ if $I\in\{0,\Z\}$ and $\fd_\Z(\Z/I)=1$ otherwise.
Thus $\gldim\Z=1$.
\end{ex*}

\begin{thm*}[2.13, The first change of rings]
Let $A$ be a ring, $x\in A$, and $N$ a $A/xA$-module.
Assume $x$ is a non-zero divisor of $A$, $N\ne0$, and $\pd_{A/xA}(N)<\infty$.
Then, $\pd_A(N)=1+\pd_{A/xA}(N)$.
\end{thm*}
\begin{pf}
Let $F$ be a free $A/xA$-module with a short exact sequence of $A/xA$-modules
\[0\to N'\to F\to N\to0.\]
For any $A/xA$-module $L$, if we let $p:=\pd_{A/xA}(N)$, then taking hom functor and considering long exact sequence in $\Mod_{A/xA}$ we have
\[0=\Ext_{A/xA}^i(F,L)\to\Ext_{A/xA}^i(N',L)\to\Ext_{A/xA}^{i+1}(N,L)=0,\qquad i\ge p.\]
We can see $\pd_{A/xA}(N')=p-1$.
By induction hypothesis, $\pd_A(N')=1+\pd_{A/xA}(N')=p$, so it is enough to show $\pd_A(N)=1+\pd_A(N')$.

For any $A$-module $K$, we have $\Ext_A^i(F,K)=0$ for $i\ge2$ by considering the exact sequence $0\to A\to A\to A/xA\to0$.
Since $\pd_A(N')=p$, we have $\Ext_A^i(F,K)=0$ for $i\ge p+2$.
So, for $p\ge2$ we obtain the proof.
Assume $p\le1$ and $\pd_A(N)\le2$.
If $\pd_A(N)=2$, then $\Ext^1(N',K)\to\Ext^2(N,K)\to0$ with $\Ext^2(N,K)\ne0$, we have $\pd_A(N')=1$.
If $\pd_A(N)=1$, then on $0\to P_1\to P_0\to N\to0$, tensoring $A/xA$, we have
\[0\to N=\{n\in N:xn=0\}=\Tor_1^A(N,A/xA)\to\bar P_1\to\bar P_0\to N\otimes_AA/xA=N\to0.\]
The projectivity is preserved by tensor, and by (2.4) we finally have $N$ is projective.
\end{pf}

\begin{ex*}[2.14]
Let $A=\C[x_1,\cdots,x_n]$ and $\fm=(x_1,\cdots,x_n)$.
Then, $\pd_A(A/\fm)=n$ by induction.
\end{ex*}

\begin{exe*}[2.15]
Let $A=\C[x_1,\cdots,x_n]$ and $\fm=(x_1,\cdots,x_n)$.
Then, $\pd_A(\fm)=?$
\end{exe*}

\begin{rmk*}[2.16]
Let $(A,\fm)$ be a regular local ring.
Then, it is known that $\dim A=\pd_A(A/\fm)=\gldim A$.
\end{rmk*}



\newpage
\section{Day 3: October 16}
\subsection*{Auslander-Buchsbaum-Serre theorem}

\begin{thm*}[Auslander-Buchsbaum-Serre]
Let $A$ be a noetherian local ring.
Then, $A$ is regular if and only if it has finite globale dimension.
\end{thm*}

(3.1)
Let $A$ be a noetherian local ring.
Let $\fm$ be the unique maximal ideal and $k:=A/\fm$.
Let $M$ be an arbitrary finitely generated $A$-module.
Suppose $x_1,\cdots,x_{r_0}$ form a minimal local generating system in the sense that $M=Ax_1+\cdots+Ax_{r_0}$ and $\dim_k(M/\fm M)=r_0$.
Then,
\[F_{r_0}:=A^{\oplus r_0}\to M\to0\]
is exact.
For $K:=\ker(F_0\to M)=Ay_1+\cdots+Ay_{r_1}$ minimally generated by $y_i$, we have an exact sequence
\[F_{r_1}:=A^{\oplus r_1}\to F_{r_0}\to M\to0.\]
By repeating this, we can easily check that the following is a \emph{minimal free resolution}:
\[\cdots\to F_2\to F_1\to F_0\to M\to0.\]

\begin{defn*}[3.2]
Let $A$ be a noetherian local ring with the maximal ideal $\fm$ and the residue field $k:=A/\fm$.
Let $M$ be a finitely generated $A$-module.
An exact sequence
\[\cdots\to F_2\xrightarrow{d_2} F_1\xrightarrow{d_1} F_0\xrightarrow{\e} M\to0\]
is called a \emph{minimal free resolution} if
\begin{enumerate}[(i)]
\item $F_i$ is a finitely generated free $A$-module,
\item $\bar d_i:\bar F_i\to\bar F_{i-1}$ is zero,
\item $\bar\e:\bar F_0\to\bar M$ is an isomorphism,
\end{enumerate}
where $\bar F_i:=F_i\otimes_Ak$ and $\bar M:=M\otimes_Ak$.
\end{defn*}

\begin{rmk*}[3.3]
Minimal free resolution is unique.
\end{rmk*}

\begin{ex*}[3.4]
Let $A:=\Z_{(2)}$ and $M:=\Z/2\Z$.
Then, $0\to A\xrightarrow{\times2}A\to M\to0$ is the minimal free resolution of $M$.

Let $A:=\Z/4\Z$, $\fm=2\Z/4\Z$, and $M:=\Z/2\Z$.
Then, $\cdots\to A\xrightarrow{\times2}A\xrightarrow{\times2}A\to M\to0$ is the minimal free resolution.
\end{ex*}

\begin{thm*}[3.5]
Let $A$ be a noetherian local ring with the maximal ideal $\fm$ and the residue field $k:=A/\fm$.
Let $M$ be a finitely generated $A$-module, and $F_i$ is the minimal free resolution.
\begin{parts}
\item $\dim_k\Tor_i^A(M,k)=\rk F_i$.
\item $\pd_A(M)=\sup\{i:\Tor_i^A(M,k)\ne0\}\le\pd_A(k)$. In particular, $\gldim A=\pd_A(k)$.
\end{parts}
\end{thm*}
\begin{pf}
(a)
By tensoring $k$ on the minimal free resolution we can compute $\Tor_i^A(M,k)\cong\bar F_i\cong k^{\oplus\rk F_i}$.

(b)
Recall $\pd_A(M)=\sup_N(\sup\{i:\Tor_i^A(M,N)\ne0\})$, where $N$ runs through all finitely generated $A$-modules.
We want to show
\[\pd_A(M)\ge\sup\{i:\Tor_i^A(M,k)\ne0\}\le\pd_A(k).\]
($\le$) follows from that $\pd_A(M)$ is the infimum of the length of projective resolutions which is $\le\sup\{i:\Tor_i^A(M,k)\ne0\}$.
($\ge$) follows from ($\le$).
\end{pf}

\begin{ex*}[3.6]
$\gldim(\Z/4\Z)=\pd_{\Z/4\Z}(\Z/2\Z)$ by (3.5), $\pd_{\Z/4\Z}(\Z/2\Z)=\infty$ by (3.4).
\end{ex*}

\begin{thm*}[3.7]
Let $A$ be a noetherian local ring.
If it is regular, then its global dimension is equal to the Krull dimension.
\end{thm*}
\begin{pf}
Let $d:=\dim A$.
We have $\fm=Ax_1+\cdots+Ax_d$.
By (3.5),
\[\gldim A=\pd_A(k)=\pd_{A/x_1A}(k)+1=\gldim A/x_1A+1=(d-1)+1=d.\qedhere\]
\end{pf}

\begin{thm*}[3.8]
Let $A$ be a noetherian local ring.
If it has finite global dimension, then it is regular.
\end{thm*}
\begin{pf}
Let $g:=\gldim A<\infty$.
Define the \emph{embedding dimension} $e:=\dim_k\fm/\fm^2<\infty$.
As a remark, $0\le\dim A\le e$, and $\dim A=e$ is equivalent to the regularity of $A$.
If $e=0$, then the claim trivially follows.
Assume $e\ge1$.
Then, $g\ge1$ because $g=0$ implies the flatness(freeness) of $A/\fm$, telling us $\fm=0$, which implies $A$ is a field and $e=0$.

We claim
\begin{enumerate}[(i)]
\item there is $x\in\fm\setminus\fm^2$ a non-zero divisor of $A$,
\item $\fm/x\fm\twoheadrightarrow \fm/xA$ splits as $A$-modules.
\end{enumerate}
Assume these claims for now.
For $B:=A/xA$, we consider an inductive hypothesis that $B$ is a noetherian local ring satisfying $\dim_k\fm_B/\fm_B^2<\dim_k\fm/\fm^2$ by $\fm/\fm^2\to\fm_B/\fm_B^2:x\mapsto0$.
Then, it is enough to show $\pd_B(k)<\infty$, because it implies $\gldim B<\infty$ and the regularity of $B$ by inductive hypothesis, which proves the regularity of $A$ with $\dim A=\dim B+1$ by the generator lift of $\fm_B$ and $x$.
The proof of $\pd_B(k)<\infty$ is along the following steps:
\begin{parts}
\item $\pd_B(\fm/x\fm)<\infty$,
\item $\pd_B(\fm_B)<\infty$,
\item $\pd_B(k)<\infty$.
\end{parts}
For (a), by the second change of rings,
\[\pd_B(\fm/x\fm)=\pd_{A/xA}(\fm/x\fm)=\pd_A(\fm)\le\gldim A<\infty.\]
For (b), by the splitting claim,
\[\pd_B(\fm_B)=\pd_B(\fm/xA)\le\pd_A(\fm/x\fm)<\infty.\]
For (c), since we have an exact sequence $0\to\fm_B\to B\to k\to0$, $\Tor_{0\ll N}^B(\fm,)=0$, $\Tor_{0<i}^B(B,\forall)$, so $\Tor_{0\ll N}^B(k,\forall)=0$ and $\pd_B(k)<\infty$......?


Now we prove the claim.

(i)' We prove that there is a non-zero-divisor $y\in\fm$ of $A$.
Suppose every element of $\fm$ is a zero-divisor of $A$.
Then, $\fm\subset\fp$ for some associated prime ideal with $\fp=\operatorname{Ann}(a)$ for some $a\in A$.
Then, $a\ne0$ and $a\fm=0$.
Consider the minimal free resolution
\[0\to F_g(\ne0)\to\cdots\to F_0\to k\to0.\]
By tensoring, we have $F_g\subset\fm F_{g-1}$.
Then, $0\ne aF_g\subset a\fm F_{g-1}=0$ is a contradiction, so we have a non-zero-divisor $y\in\fm$ of $A$.

(i)
$\fm\ne0$ implies $\fm\ne\fm^2$ by the Nakayama lemma.
By (i)', (prime avoidance?) $\fm\not\subset \fm^2\cup(\bigcup_{\fp\in\operatorname{Ass}A}\fp)$.

(ii)
We can take minimal generators $x=x_1,\cdots,x_e$ of $\fm$.
It means that $\bar x_1,\cdots,\bar x_e$ is a basis of $\fm/\fm^2$.
Let $I:=Ax_2+\cdots+Ax_e$.
For $\zeta\in I\cap xA$ with $\zeta=xy=x_2z_2+\cdots x_ez_e$ for some $y,z_2,\cdots,z_e\in A$, by taking modulo $\fm^2$, we have $\bar x\bar y=\bar x_2\bar z_2+\cdots\bar x_e\bar z_e$ in $\fm/\fm^2$, where $\bar x,\bar x_i\in\fm/\fm^2$ and $\bar y,\bar z\in A/\fm$, and $\bar x=\bar x_1,\bar x_2,\cdots\bar x_e$ is a $A/\fm$-linear basis of $\fm/\fm^2$ so that $\bar y=0$.
Thus we have $I\cap xA\subset x\fm$, so the splitting from the identity
\[\fm/xA=(xA+I)/xA\cong I/(xA\cap I)\twoheadrightarrow(I+\fm)/x\fm\hookrightarrow\fm/x\fm\twoheadrightarrow\fm/xA.\qedhere\]
\end{pf}

\begin{thm*}[3.9, second change of rings]
Let $A$ be a noetherian local ring.
Let $M$ be a finitely generated $A$-module.
If $x\in\fm$ is a non-zero-divisor of $A$ and $M$, then $\pd_A(M)=\pd_{A/xA}(M/xM)$.
\end{thm*}
\begin{pf}
Tensoring $A/xA$ on the minimal free resolution of $M$, we have
\[\cdots\xrightarrow{\bar d_2}\bar F_1\xrightarrow{\bar d_1}\bar F_0\xrightarrow{\bar \e}M/xM\to0.\]
To prove it is a minimal free resolution, the only non-trivial part is the exactness, which is by (3.10).
\end{pf}

\begin{lem*}[3.10]
Let $A$ be a ring.
Let $x\in A$ be a non-zero-divisor of $A$.
\begin{parts}
\item If $\cdots\to L_2\to L_1\to0$ is exact and $x$ is a non-zero-divisor of $L_i$, then $\cdots\to\bar L_2\to\bar L_1\to0$ is exact.
\item If $0\to N_1\to N_2\to N_3\to0$ is exact and $x$ is a non-zero-divisor of $N_3$, then $0\to N_1/xN_1\to N_2/xN_2\to N_3/xN_3\to0$ is exact.
\end{parts}
\end{lem*}
\begin{pf}
(b)$\Rightarrow$(a)
Relatively clear by thinking $0\to\ker\to L_3\to\im\to0$ with $L^4\twoheadrightarrow\ker$ and $\im\hookrightarrow L_2$.

(b)
\[\Tor_1^A(N_3,A/xA)\to N_1\otimes A/xA\to N_2\otimes A/xA\to N_3\otimes A/xA\to0,\]
where
\[\Tor_1^A(N_3,A/xA)=\{y\in N_3:xy=0\}=\ker(\times x:N_3\to N_3)=0.\qedhere\]
\end{pf}





\newpage
\section{Day 4: October 23}


\subsection*{Depth}

\begin{defn*}[4.1]
Let $A$ be a noetherian local ring.
Let $M$ be a finitely generated $A$-module.
We say $x\in A$ is \emph{$M$-regular} if $x$ is not a zero-divisor of $M$ such that $M/xM\ne0$, i.e.~$M\xrightarrow{x}M$ is a proper embedding.
For $M\ne0$, by the Nakayama lemma, $x$ is $M$-regular if and only if $x\in\fm$....?

\end{defn*}
\begin{exe*}[4.2]
Let $A$ be a noetherian local ring.
Let $M$ be a finitely generated $A$-module.
If $x_1,\cdots,x_n\in A$ is an $M$-regular sequence, then every permutation is also an $M$-regular sequence.
\end{exe*}
\begin{prop*}[4.3]
Let $A$ be a noetherian local ring.
Let $L\to M\to N$ be an exact sequence of $A$-modules.
Assume $N$ is finitely generated and $x_i$ is a $N$-regular sequence.
Then, ....?
\end{prop*}

\begin{ex*}[4.4]
Let $A$ be a regular local ring.
Let $x_i$ be an RSP, i.e. $d=\dom A$ and $\fm=(x_1,\cdots,x_d)$.
Then, $x_i$ is a $A$-regular sequence.
\end{ex*}
\begin{ex*}[4.5]
Let $A:=\C[x,y](x,y)/(xy,y^2)$.
Consider the minimal primary decomposition $(xy,y^2)=(y)\cap(x,y)^2$.
Let $\fm:=(x,y)^2$.
Then, $\Ass A=\{yA,\fm\}$, so the set of zero-divisors of $A$ is $yA\cup\fm=\fm$.
It means that there is no $A$-regular element.
\end{ex*}

\begin{rmk*}[4.6]
Let $A$ be a noetherian ring and $M$ be a finitely generated $A$-module.
\[\Ass M:=\{\fp\in\Spec A:\exists m\in M,\ \fp=\Ann(m)\}\]
is always finite, and
\[\Ass M\subset\supp M:=\{\fp\in\Spec A:M_\fp\ne0\}.\]
The condition $\exists m\in M,\ \fp=\Ann(m)$ is equivalent to the existence of embedding $A/\fp\to M$.
It is known that the set of zero-divisors of $M$ is the union of all associated prime ideals of $M$.
If every element of an ideal $I$ is a zero-divisor of $M$, then there is an associated prime ideal $\fp$ such that $I\subset\fp$ by prime avoidance.
\end{rmk*}

\begin{lem*}[4.7]
Let $A$ be a noetherian ring and $M,N$ be finitely generated $A$-modules.
Then, TFAE:
\begin{parts}
\item there is $a\in\Ann N$ which is not a zero-divisor of $M$,
\item $\Hom_A(N,M)=0$.
\end{parts}
\end{lem*}
\begin{pf}
(a)$\Rightarrow$(b)
Let $a\in A$ such that $N\xrightarrow{a}N$ is zero and $M\xrightarrow{a}M$ is injective.
Then, for $\f\in\Hom_A(N,M)$ we have $\f(n)=0$ by diagram chasing.

(b)$\Rightarrow$(a)
Suppose every element of $\Ann N$ is a zero-divisor of $M$.
Find an associated prime $\fp$ of $M$ containing $\Ann N$.
We will construct $A_\fp$-linear maps $N_\fp\twoheadrightarrow k(\fp)$ and $k(\fp)\hookrightarrow M_\fp$.
Then we are done since $k(\fp)\ne0$.
By the Nakayama lemma, $N_\fp/\fp N_\fp$ is a non-zero finite dimensional linear space over $k(\fp)$, so we have surjective $N_\fp\to N_\fp/\fp N_\fp\to k(\fp)$ which is $A_\fp$-linear.
Find $z\in M$ such that $\fp=\Ann(z)$.
Then, $A/\fp\hookrightarrow M$, and the localization functor $-\otimes_AA_\fp$ is exact so that we have $k(\fp)\hookrightarrow M_\fp$.
\end{pf}

\begin{prop*}[4.8]
Let $A$ be a noetherian ring and $M,N$ be finitely generated $A$-modules.
Assume $(x_i)_{i=1}^n$ is an $M$-regular sequence in $\Ann N$.
\begin{parts}
\item $\Ext_A^n(N,M)\cong\Hom_A(N,M/(x_1,\cdots,x_n)M)$.
\item $\Ext_A^i(N,M)=0$ for $i<n$.
\end{parts}
\end{prop*}
\begin{pf}
(a)
Induction on $n$.
\[0\to\Ext^{n-1}(N,M/x_1M)\to\Ext^n(N,M)\xrightarrow{\f}\Ext^n(N,M)\]
The map $\f$ is induced from $x_1:M\to M$ but also from $x_1:N\to N$.
Since $x_1\in\Ann N$, $\f=0$.
Therefore, the inductive hypothesis implies
\[\Ext_A^n(N,M)\cong\Ext_A^{n-1}(N,M/x_1M)\cong\Hom_A(N,(M/x_1M)/(x_2,\cdots,x_n)(M/x_1M))\cong\Hom_A(N,M/(x_1,\cdots,x_n)M).\]

(b)
By (a), $\Ext_A^i(N,M)\cong\Hom_A(N,M/(x_1,\cdots,x_i)M)=0$ by (4.7) (take $a=x_{i+1}$).

\end{pf}


\begin{thm*}[4.9]
Let $A$ be a noetherian local ring.
Let $M$ be a non-zero finitely generated $A$-module.
Then,
\[\max\{n:\exists\text{ $M$-regular sequence }x_1,\cdots,x_n\}=\min\{i:\Ext_A^i(k,M)\ne0\}.\]
\end{thm*}
\begin{rmk*}
$\depth_A0:=-\infty$...? $+\infty$...?
\end{rmk*}
\begin{pf}
Take $x_1,\cdots,x_n$ a maximal $M$-regular sequence by (4.10).
They are contained in $\fm=\Ann(k)$.
By (4.8)(b), $\Ext_A^i(k,M)=0$ for $i<n$.
Enough to show $\Ext_A^n(k,M)\ne0$.
Note $\Ext_A^n(k,M)\cong\Hom_A(k,M/(x_1,\cdots,x_n)M)$.
Every element of $\fm$ is a zero-divisor of $M/(x_1,\cdots,x_n)M$ by maximality, by (4.7), we are done.
\end{pf}

\begin{thm*}[4.10]
Let $A$ be a noetherian local ring, and $M$ be a finitely generated $A$-module.
Let $x_1,\cdots,x_n$ be an $M$-regular sequence.
Then,
\begin{parts}
\item $\dim(M/(x_1,\cdots,x_n)M)=\dim M-n$.
\item $\depth(M/(x_1,\cdots,x_n)M)=\depth(M)-n$.
\item $\dim M\ge\depth M$.
\end{parts}
($\dim M:=\dim\supp M$.)
\end{thm*}
\begin{pf}
(a) dimension theory.

(b)
If we choose $y_1,\cdots,y_m$ an $M/(x_1,\cdots,x_n)M$-regular sequence, then $x_i,y_j$ is a maximal $M$-regular sequence.
By (4.9),
\[\depth M=n+m=n+\depth(M/(x_1,\cdots,x_n)M).\]

(c)
\[\dim M=n+\dim(M/(x_1,\cdots,x_n)M)\ge n=\depth M.\]
\end{pf}

\begin{cor*}[4.11]
Let $A$ be a noetherian local ring.
If $A$ is regular, then $A$ is Cohen-Macaulay.
\end{cor*}
\begin{pf}
RSP $x_1,\cdots,x_{\dim A}$.
Then, $\depth A=\dim A$.
\end{pf}

\begin{thm*}[4.12, Auslander-Buchsbaum formula]
Let $A$ be a noetherian local ring and $M$ a finitely generated $A$-module of finite projective dimension.
Then, $\pd_A(M)+\depth M=\depth A$.
\end{thm*}

\begin{ex*}[4.13]

\end{ex*}



\newpage
\section{Day 5: November 6}


\subsection*{Cohen-Macaulay rings}

\begin{defn*}[5.1]
Let $A$ be a noetherian local ring.
Then, $A$ is called Cohen-Macaulay if $\depth A=\dim A$.
Recall we always have $\depth A\le\dim A$.
\end{defn*}

\begin{prop*}[5.2]
Let $A$ be a noetherian local ring.
If $A$ is regular, then $A$ is CM.
\end{prop*}
\begin{pf}
We have $\fm=Ax_1+\cdots+Ax_d$, where $d=\dim A$.
Enough to show $x_1,\cdots,x_d$ is a $A$-sequence, i.e.~$d\le\depth A$.
Since $x_1\ne0$ and $A$ is an integral domain(?), $x_1$ is $A$-regular.
$A/x_1A$ is regular.
$\bar x_2,\cdots,\bar x_d$ is an $A$-sequence, by induction on dimension of $A$, so $x_1,\cdots,x_d$ is an $A$-sequence.
\end{pf}

\begin{prop*}[5.3]
Let $A$ be a noetherian local ring.
\begin{enumerate}[(i)]
\item If $\dim A=0$, then $A$ is CM.
\item If $\dim A=1$ and $A$ is an integral domain, then $A$ is CM.
\end{enumerate}
\end{prop*}
\begin{pf}
(i)
$0\le\depth A\le\dim A=0$.

(ii)
Take non-zero $0\in\fm$.
Then, $A$ is regular, so $1\le\depth A\le\dim A=1$.
\end{pf}


\begin{ex*}[5.4]
$\Z/4\Z$ and $\C[x,y]_{(x,y)}/(y^2-x^3)$ are non-regular CM rings.
\end{ex*}

\begin{prop*}[5.5]
Let $A$ be a noetherian local ring.
Then, $A$ is CM iff $A/(x_1,\cdots,x_n)A$ is CM, where $x_i\in\fm$ is an $A$-sequence.
\end{prop*}

\begin{ex*}[5.6]
$\C[x_1,\cdots,x_n]_{(x_1,\cdots,x_n)}/(f)$ is CM for any non-zero $f\in(x_1,\cdots,x_n)\C[x_1,\cdots,x_n]_{(x_1,\cdots,x_n)}$.
\end{ex*}

\begin{thm*}[5.7]
Let $A$ be a noetherian local ring.
For associated prime $\fp$ of $A$, $\depth A\le\dim A/\fp$.
\end{thm*}
\begin{pf}
Induction on $\depth A$.
Assume $\depth A>0$ and fix $\fp\in\Ass A$.
Take $x\in\fm$ which is not a zero-divisor such that $x\notin\fp$ since $\fp$ only contains zero-divisors.
Let $B:=A/xA$.
Then, $\depth B=\depth A-1$ by definition of depth, and $\dim B/\fp B=\dim A/\fp-1$ by the Krull principal ideal theorem.
By induction, $\depth\le\dim B/\fq$ for all associated primes $\fq\in\Ass B$.
Now it is enough to show the existence of $\fq\in\Ass B$ such that $\fp B\subset\fq$, since it follows that
\[\dim A/\fp-\depth A=\dim B/\fp B-\depth B\ge\dim B/\fq-\depth B\ge0.\]

Since $\fp$ is an associated prime of $A$, there is $y\in A$ such that $\fp=\Ann(y)=\ker(A\xrightarrow{y}A)$.
Tensoring an exact sequence $0\to\fp\to A\xrightarrow{y} A\to0$ with $B$, we obtain a complex $\fp\otimes_AB\to B\xrightarrow{\bar y}B$.
We can show $B\xrightarrow{\bar y}B$ is non-zero as follows.
If we assume $y\in xA$ so that $y=xy'$ then $\Ann(y)=\Ann(y')$ since $x$ is not a zero-divisor, and it gives rise to an increasing chain of principal ideals with same annihilator, which stops by the noetherian condition.
At the stopping stage we have $y'A=yA$, so that $y'A=yA=xy'A\subset\fm(y'A)$, implying $y=0$ by the Nakayama lemma.
Then, $y=x^k\tilde y$ with $\tilde y\notin xA$ proves that $\bar y:B\to B$ is non-zero.

Since $\fp B$ is the image of $\fp\otimes_AB\to B$, if we define $\fq$ as a maximal element of $\{\Ann(\bar z):\bar z\in B\}$, then it is prime and $\fp B\subset\Ann(\bar y)\subset \fq$.
\end{pf}

\begin{cor*}[5.8]
Let $A$ be a CM local ring.
Then, there is no embedded prime, and $\dim A=\dim A/\fp$ for all minimal primes $\fp$.
\end{cor*}
\begin{pf}
Let $\fp\in\Ass A$.
From the CM we have $\dim A=\depth A$, from (5.7) we have $\depth A\le\dim A/\fp$, and clearly $\dim A/\fp\le\dim A$.
\end{pf}

\begin{ex*}[5.9]
$\C[x,y,z]_{(x,y,z)}/(xz,yz)$ is not CM by (5.8) because there are two minimal primes having different dimensions of quotients.
\end{ex*}

\begin{ex*}[5.10]
$\C[x,y]_{(x,y)}/(xy,y^2)$ is not CM by (5.8) because it has an embedded prime.
\end{ex*}

\begin{prop*}[5.11]
Let $A$ be a CM local ring.
Then, the localization $A_\fp$ is CM for any prime $\fp$.
\end{prop*}
\begin{pf}
Fix $\fp\in\Spec A$.
Induction on $\depth A_\fp$.
Suppose $\depth A_\fp=0$.
Since depth zero is equivalent to $\bigcup_{\fq\in\Ass A_\fp}\fq=ZD(A_\fp)=\fp A_\fp$, so $\fp A\fp\in\Ass(A_\fp)$.
Note that there is bijection between $\Ass(A_\fq)$ and $\{\fq\in\Ass A:\fq\subset\fp\}$.
It means that $\fp$ is minimal because $A$ is CM and by (5.8), $\dim A_\fp=0$.

Suppose $\depth A_\fp>0$.
Then, every associated prime $\fq\in\Ass(A_\fp)$ has $\fp\not\subset\fq$ since if not, then $\fq$ is minimal from that $A$ is CM we have $\fp=fq$ so that $\depth A_\fp\le\dim A_\fp=0$.
The prime avoidance we have $\fp\not\subset\bigcup_{\fq\in\Ass A}\fq=ZD(A)$.
Take a non-zeor-divisor $x\in\fp$.
Then, $A/xA$ is CM, $A_\fp/xA_\fp$ is CM, so $A_\fp$ is CM.
\end{pf}

\begin{prop*}[5.12]
Let $A$ be a noetherian local ring and $M$ be a finitely generated $A$-module.
Then, $\depth_AM=\depth_{\hat A}\hat M$, where the hat notation denotes the $\fm$-adic completion.
\end{prop*}
\begin{pf}
Since $\hat A$ is flat over $A$, by (5.13), the statement clearly follows from
\[\Ext_A^i(k,M)\otimes_A\hat A\cong\Ext_{\hat A}^i(k,\hat M).\qedhere\]
\end{pf}

\begin{exe*}[5.13]
Let $\f:A\to B$ be a flat ring homomorphism of noetherian rings.
Let $M$ and $N$ be finitely generated $A$-modules.
For $i\in\Z_{\ge0}$,
\[\Ext_A^i(M,N)\otimes_AB\cong\Ext_B^i(M\otimes_AB,N\otimes_AB).\]
\end{exe*}

\begin{cor*}[5.14]
Let $A$ be a noetherian local ring.
If $A$ is CM, then $\hat A$ is CM.
\end{cor*}
\begin{pf}
We have $\dim A=\dim\hat A$.
We have $\depth A=\depth\hat A$ by (5.13).
\end{pf}

\subsection*{Serre criterion for normality}
\begin{thm*}[5.15]
Let $A$ be a noetherian integral domain.
Then, $A$ is integrally closed iff $A$ is R$_1$ and S$_2$, where
\begin{enumerate}[(i)]
\item R$_1$: $A_\fp$ is regular(DVR) for all prime $\fp$ of height one.
\item S$_2$: $\min\{2,\dim A_\fp\}\le\depth A_\fp$ for all prime $\fp$.
\end{enumerate}
\end{thm*}
\begin{rmk*}[5.16]\,
\begin{parts}
\item If $A$ is CM, then S$_2$ automatically holds.
\item If $A$ ir regular, then R$_1$ automatically holds.
\end{parts}
\end{rmk*}

\begin{pf}[Sketch]
Our strategy is to show $A$ is integrally closed iff $A=\bigcap_{\operatorname{hight}\fp=1}A_\fp$ iff $A$ satisfies R$_1$ and S$_2$.

We show if a CM ring $A$ has $\dim A=2$ and R$_1$, then $A=\bigcap_{\operatorname{hight}\fp=1}A_\fp$.

If we use local cohomology, then clearly follows from $H_m^i(A)=0$ for $i<\dim A$ and
\[0\to H_\fm^0(A)\to A\to\bigcap_{\operatorname{hight}\fp=1}A_\fp\to H^1_\fm(A)...\qedhere\]
\end{pf}



\begin{ex*}[5.17]
For non-zero $f\in\C[x,y,z]$, let $A:=\C[x,y,z]/(f)$ be an integral domain.
Then, since $S_2$ is OK, so $A$ is integrally closed iff $A$ has $R_1$.
By the Jacobian criterion, it is equivalent to $\#\{(a,b,c)\in\C^3:f=f_x=f_y=f_z=0\}<\infty$.
\end{ex*}

\begin{rmk*}[5.18]
We can construct a non-CM integral domain $A$ as follows: gluing $(1,0)$ and $(-1,0)$ on the plane $\Spec\C[x,y]$.
Since $A$ has $R_1$, $A$ is integrally closed iff $A$ is CM.
More precisely,
\[\begin{tikzcd}
A\rar\dar\pullback & \C\dar\\
\C[x,y]\rar[->>]{\pi} &\dfrac{\C[x,y]}{(x-1,y)}\times\dfrac{\C[x,y]}{(x+1,y)}
\end{tikzcd}\]

\end{rmk*}



\section{Day 6: November 20}


\subsection*{}
injective modules, injective dimension

\begin{prop*}[6.4]
Let $M$ be an abelian group.
Then, $M$ is injective iff $M$ is divisible, i.e.~for every $x\in M$ and $n\in\Z\setminus\{0\}$, there is $y\in M$ such that $x=ny$.
\end{prop*}
\begin{pf}
If $M$ is divisible, then $M\leftarrow n\Z\hookrightarrow\Z$ gives $\Z\to M$.

If $M$ is injective, then $M\leftarrow \Z\hookrightarrow\frac1n\Z$ gives $\frac1n\Z\to M$.
\end{pf}
\begin{ex*}[6.5]
$\Q$ and $\Q/\Z$ are injective $\Z$-modules.
\end{ex*}

\begin{lem*}[6.6]
Let $A$ be a noetherian ring, and $M$ be an $A$-module.
Fix $i\ge0$.
If $\Ext^i(A/\fp,M)=0$ for all prime $\fp$ of $A$, then $\Ext^i(N,M)=0$ for all finitely generated $N$.
\end{lem*}
\begin{pf}
We can consider a chain $\{0\}=N_s\subset\cdot\subset N_0:=N$ such that $N_k/N_{k+1}\cong A/\fp_k$ for $\fp_k\in\Spec A$.
Consider the exact sequence $0\to N_{k+1}\to N_k\to A/\fp_k$.
\end{pf}

\begin{prop*}[6.7, a key lemma]
Let $A$ be a noetherian local ring.
Let $M$ be a finitely generated $A$-module.
Let $\fp\subsetneq\fm$ be a prime.
Fix $i\in\Z_{\ge0}$.
Then, $\Ext_A^{i+1}(A/\fq,M)=0$ for all prime $\fq$ such that $\fp\subsetneq\fq$, then $\Ext_A^i(A/\fp,M)=0$.
\end{prop*}
\begin{pf}
Fix $x\in\fm/\fp$.
Then, $x$ is a non-zero-divisor of $A/\fp$ so that $0\to A/\fp\to A/\fp\to A/(\fp+Ax)\to0$ is exact.
Take $\Hom_A(-,M)$ to get
\[\Ext_A^i(A/\fp,M)\to\Ext_A^i(A/\fp,M)\to\Ext_A^{i+1}(A/(\fp+Ax),M)=0,\]
so $\Ext_A^i(A/\fp,M)=x\Ext_A^i(A/\fp,M)\subset\fm\Ext_A^i(A/\fp,M)$.
By the Nakayama lemma, $\Ext_A^i(A/\fp,M)=0$.
\end{pf}

\begin{thm*}[6.8]
Let $A$ be a noetherian local ring.
Let $M$ be a finitely generated $A$-module.
Then, $\id_A(M)=\sup\{i:\Ext_A^i(k,M)\ne0\}$.
\end{thm*}
\begin{pf}
Let $s$ be the right hand-side.
It is easy to see $\id_A(M)\ge s$.

Suppose $\Ext_A^{\ge s+1}(A/\fm,M)=0$.
Then, $\Ext_A^{\ge s}(A/\fp,M)=0$ for $\fp$ with $\hgt\fp=\hgt\fm-1$ by (6.7), and $\Ext_A^{\ge s+1}(A/\cdot,M)=0$ for all $\hgt\cdot\ge\hgt\fm-1$, and by repeation $\Ext_A^{\ge s+1}(A/\fp,M)=0$ for all primes $\fp$.
By (6.9), $\id_A(M)\le s$.
\end{pf}

\begin{lem*}[6.9]
Let $A$ be a noetherian ring, $M$ a finitely generated $A$-module.
Fix $n\in\Z_{\ge0}$. TFAE
\begin{parts}
\item $\id_A(M)\le n$,
\item $\Ext_A^{n+1}(A/\fp,M)=0$ for all $\fp\in\Spec A$,
\item $\Ext_A^r(N,M)=0$ for all finitely generated $N$ and $r\ge n+1$.
\end{parts}
\end{lem*}
\begin{pf}
Similar to projective and flat dimensions.
\end{pf}

\subsection*{Gorestein rings}



\begin{thm*}[6.10]
Let $A$ be a noetherian local ring.
TFAE
\begin{parts}
\item $\id_A(A)<\infty$.
\item $\id_A(A)=\dim A=\depth A$.
\item $\Ext_A^i(k,A)=0$ iff $i\ne\dim A$.
\end{parts}
The noetherian local ring $A$ is called \emph{Gorenstein} if one of the above holds.
\end{thm*}
\begin{pf}
Since (b)$\Rightarrow$(a) and (c)$\Rightarrow$(a), we assume (a).
Recall that $\id_A(A)=\sup\{i:\Ext_A^i(k,A)\ne0\}$ by (6.8) and $\depth(A)=\min\{i:\Ext_A^i(k,A)\ne0\}$.
Also we have $\depth A\le\dim A\le\id_A(A)$.
$\Ext_A^i(k,A)=0$ if $i>\id_A(A)$ or $i<\depth(A)$.

It follows from (6.12).
\end{pf}
\begin{exe*}[6.11]
Find a zero-dimensional non-Gorenstein noetherian local ring.
\end{exe*}

\begin{thm*}[6.12]
Let $A$ be a noetherian local ring.
\begin{parts}
\item $\dim A\le\id_A(A)$.
\item If $\id_A(A)<\infty$, then $\depth(A)=\id_A(A)$.
\end{parts}
\end{thm*}

\begin{pf}
(a)
Enough to show $\Ext_A^{\dim A}(k,A)\ne0$ using induction on $\dim A$.
If $\dim A=0$, then $\Hom_A(k,A)\ne0$.
Assume $d:=\dim A>0$.
Fix $\fp\subsetneq\fm$ such that $\hgt\fp=\hgt\fm-1=d-1$.
$\Ext_{A_\fp}^{d-1}(k(\fp),A_\fp)\ne0$ by induction hypothesis, which is same with $\Ext_A^{d-1}(A/\fp,A)_\fp$, so $\Ext_A^{d-1}(A/\fp,A)\ne0$, and $\Ext_A^d(A/\fm,A)\ne0$ by (6.7).


(b)
We claim $\id_A(A)\le\depth A$.
Let $B:=A/(x_1,\cdots,x_{\dim A})$, where $x_1,\cdots,x_d$ is a $A$-regular sequence.
Since $\id(B)=\id(A)-\depth A<\infty$ and $\depth B=0$ by (6.13), enough to show $\id(B)=0$.
Taking $\Hom_B(-B)$ on an exact sequence $0\to k\to B\to C\to0$, we have an exact sequence $\Ext_B^i(B,B)\to\Ext_B^i(k,B)\to\Ext_B^{i+1}(C,B)=0$, where $i:=\id(B)$, by (6.7).
Thus, $\Ext_B^i(B,B)\ne0$, so $i=0$.
\end{pf}

\begin{prop*}[6.13]
Let $A$ be a noetherian local ring, $M$ a finitely generated $A$-module.
Let $x\in\fm$ be a non-zero-divisor of $A$ and $M$.
Then, $\id_A(M)=\id_{A/xA}(M/xM)+1$.
\end{prop*}
\begin{pf}
Omitted.
\end{pf}


\end{document}