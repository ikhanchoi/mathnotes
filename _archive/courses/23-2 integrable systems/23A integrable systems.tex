\documentclass{../../../small}
\usepackage{../../../ikhanchoi}

\newcommand{\Gr}{\operatorname{Gr}}
\newcommand{\Det}{\operatorname{Det}}

\begin{document}
\title{Integrable Systems}
\author{Ikhan Choi\\Lectured by Ralph Willox\\University of Tokyo, Autumn 2023}
\maketitle


\section{Symmetric polynomials}
Let $x=(x_i)_{i=1}^n$ be some auxiliary variables for some $n$.
The \emph{power sum symmetric polynomial} is defined by
\[p_k(x):=\sum_ix_i^k.\]
We define \emph{flow variables}
\[t=(t_1,t_2,\cdots),\qquad t_k:=k^{-1}p_k\]
The \emph{complete homogeneous symmetric polynomial} is
\[h_k(x):=\sum_{1\le i_1\le\cdots\le i_k\le n}x_{i_1}\cdots x_{i_k}.\]
For Schur polynomial $s_\lambda$, there are various definitions, where $\lambda$ is a Young diagram for partition of $m$.

Since every symmetric function is generated by power sum symmetric functions $p_k$, we can represent $h_k$ and $s_\lambda$ in terms of $t$.
Furthermore, $h_k$ has generating function representation
\[\sum_{k=0}^\infty h_k(t)z^k=\exp{\sum_{k=1}^\infty t_kz^k}.\]
For example,
\begin{gather*}
h_1(t)=t_1,\qquad h_2(t)=t_2+\frac12t_1,\qquad h_3(t)=t_3+t_1t_2+\frac16t_1^3\\
s_{(1,1)}(t)=\frac12 t_1^2-t_2,\qquad s_{(2)}(t)=t_2+\frac12t_1^2.
\end{gather*}
From now on, we will forget any of information for the variables $x_i$, and $t_k$ will be the most fundamental variables.


Let $V$ be a vector space over $\C$ with a fixed basis $\{e_i\}$.
A basis of $\wedge^mV$ can be indexed by subset of $\{e_i\}$ of cardinality $m$.
For such a subset $l$, we will write $l=(l_1,\cdots,l_m)$ with $l_1\le\cdots\le l_m$.
This kind of $m$-tuple is a \emph{Maya diagram}.
For a Maya diagram $l$, we can associate a Young diagram $\lambda=(\lambda_1,\cdots,\lambda_m)$ such that $l_j=\lambda_{m-j+1}+j-1$.
Zeros in the Young diagram will be omitted.
For each Young diagram $\lambda$, we will use the notation
\[e_\lambda:=e_{l_1}\wedge\cdots\wedge e_{l_m},\]
where $(l_1,\cdots,l_m)$ is the corresponding Maya digram of the Young diagram $\lambda$.
Then, $\{e_\lambda\}$ forms a basis of $\wedge^mV$.

\section{Pl\"ucker coordinates}

For a positive integer $m$, the \emph{Grassmann variety} $\Gr_m(V)$ is the algebraic variety of $m$-dimensional subspaces of $V$.
The \emph{Pl\"ucker embedding}
\[\psi:\Gr_m(V)\to\P(\Lambda^mV):w=\spn\{w_j\}_{j=1}^m\mapsto\wedge^mw=\spn\{\wedge_{j=1}^mw_j\}\]
shows that the Grassmann variety is projective.

The \emph{tautological vector bundle} $T$ over the grassmann variety $\Gr_m(V)$ is defined as a subbundle of the trivial bundle $\Gr_m(V)\times V\to\Gr_m(V)$ such that
\[T:=\{(w,v)\in\Gr_m(V)\times V:v\in w\}.\]
The rank of the tautological bundle $T$ is $m$.
The \emph{determinant line bundle} $\Det$ over the grasmann variety $\Gr_m(V)$ is the top exterior power of the tautological vector bundle
\[\Det:=\wedge^mT.\]

The tautological line bundle of the projective space $\P(\wedge^mV)$ is identified with $\cO_{\P(\wedge^V)}(-1)$.
The identity
\[\Det_w=\wedge^mw=\psi(w)=\cO_{\P(\wedge^V)}(-1)_{\psi(w)}\]
on each fiber at $w$ and $\psi(w)$ defines a bundle isomorphism $\Det\to\cO_{\P(\wedge^mV)}(-1)$, so we have the isomorphic line bundles
\[\Det\cong\psi^*\cO_{\P(\wedge^mV)}(-1).\]
Taking the inverses, we have
\[\Det^*\cong\psi^*\cO_{\P(\wedge^mV)}(1).\]

The line bundle $\cO_{\P(\wedge^mV)}(1)$ admits global sections spanned by the homogeneous polynomial of degree one, which are identified to the coordinate functions $e_\lambda^*:\wedge^mV\to\C$ defined such that $e_\lambda^*(e_{\lambda'})=\delta_{\lambda,\lambda'}$, where $\lambda$ and $\lambda'$ are Young diagrams.
For each Young diagram $\lambda$, define the \emph{Pl\"ucker coordinate} $\pi_\lambda:=\psi^*e_\lambda^*$ as a global section of the dual determinant line bundle $\Det^*$.

Since $\Gr_m(V)$ is projective via the Pl\"ucker embedding $\psi$, there is a homogeneous ideal $I$ such that the image of $\psi$ has the homogeneous coordinate ring $\C[e_\lambda^*]_\lambda/I$.
The \emph{Pl\"ucker relations} are special generators of $I$, which are quadratic homogeneous polynomials.

\section{Tau functions}

Suppose $V=\C^\infty$.
Consider an abelian group action $\gamma$ of $\C^\infty$ on the Grassmann variety $\Gr_m(V)$ defined such that
\[\gamma(t):=\exp{\sum_{k=1}^\infty t_k\Lambda^k},\qquad t=(t_1,t_2,\cdots)\in\C^\infty,\]
where $\Lambda:V\to V$ is a linear map satisfying $\Lambda e_i:=e_{i+1}$, which is called the \emph{shift matrix}.


Fix $w\in\Gr_m(V)$.
The \emph{$\tau$-function} associated with the abelian group action and the initial point $w$ is the function $\tau:\C^\infty\to\C$ defined by the first Pl\"ucker coordinate of the curve $\gamma(t)w$, i.e.
\[\tau(t):=\pi_{(0)}(\gamma(t)w).\]
We can also define
\[\tau_\lambda(t):=\pi_\lambda(\gamma(t)w).\]
(This is the $\tau$-function given in the problem.)
Then, we have the Schur function expansion
\[\tau(t)=\sum_\lambda\pi_\lambda(w)s_\lambda(t),\]
and
\[\tau_\lambda(t)=s_\lambda(\tilde\partial_t)\tau(t),\qquad\tilde\partial_t=(k^{-1}\partial_{t_k})_{k=1}^\infty.\]



\section{KP equation}

Let $x=t_1$, $y=t_2$, and $t=t_3$.
The equation
\[\tau_{(0)}\tau_{(2,2)}-\tau_{(1)}\tau_{(2,1)}+\tau_{(2)}\tau_{(1,1)}=0\]
is deduced from a Pl\"ucker relation
\[(e_0^*\wedge e_1^*)(e_2^*\wedge e_3^*)-(e_0^*\wedge e_2^*)(e_1^*\wedge e_3^*)+(e_0^*\wedge e_3^*)(e_1^*\wedge e_2^*)=0.\]
Since
\begin{align*}
\tau_{(0)}&=\tau\\
\tau_{(1)}&=\tau_x\\
\tau_{(1,1)}&=\frac{\tau_{2x}-\tau_y}2\\
\tau_{(2)}&=\frac{\tau_{2x}+\tau_y}2\\
\tau_{(2,1)}&=\frac{\tau_{3x}-\tau_{xy}}2\\
\tau_{(2,2)}&=\frac{\tau_{2x,y}+\tau_{2y}-2\tau_{tx}}2,
\end{align*}

Let
\[u:=\partial_x^2\log\tau,\qquad v:=\partial_x\partial_y\log\tau.\]


\end{document}