\documentclass{../../../small}
\usepackage{../../../ikhanchoi}

\newcommand{\Et}{\mathrm{\acute Et}}

\begin{document}

\title{\'Etale cohomology}
\author{Ikhan Choi\\Lectured by Shane Kelly\\University of Tokyo, Spring 2024}
\maketitle

\newpage
\section{Day 1: April 11}
\subsection{Counting points}

Let $X$ be a smooth projective variety over $\F_q$.
How many elements are in $X/\F_{q^n}=\hom_{\F_q}(\Spec\F_{q^n},X)$?
Equivalently, if $f_i$ is a finite sequence of homogeneous polynomials in $\F_q[t_0,\cdots,t_p]$ defining $X$, how many solutions do $f_i$ have in $\F_{q^n}$?
We are interested in the sequence of the numbers of points $\#X(\F_{q^n})$ indexed by $n$, and a power series is associated as
\[Z(X,t):=\exp\left(\sum_{n=1}^\infty\#X(\F_{q^n})\frac{t^n}n\right)=\prod_{x\in X_{(0)}}(1-t^{\deg(x)})^{-1},\]
where $X_{(0)}$ is the set of closed points of the scheme $X$, and $\deg(x)$ is the degree $[\kappa(x):\F_q]$ of the residue field.
It is called the \emph{zeta function}.

\begin{rmk*}
The exponential function is recognized as a formal power series
\[\exp(T)=\sum_{m=0}^\infty\frac{T^m}{m!}\in\C[[T]].\]
The series $Z(X,t)$ is defined for any variety(not necessarily smooth or projective).
If $W\to X$ is a closed immersion and $U:=X\setminus W$, then $Z(X,t)=Z(W,t)Z(U,t)$.
\end{rmk*}
\begin{exe*}
Prove the identity in the definition of $Z(X,t)$.
\end{exe*}

Our question is to calculate $Z(X,t)$.

\begin{ex*}
Let $X=\A^d$.
We have $X(\F_{q^n})=\F_{q^n}^d$, so $\#X(F_{q^n})=q^{dn}$, thus
\[Z(\A^d,t)=(1-q^dt)^{-1}\in\Q(T)\subset\C((T)).\]
\end{ex*}
\begin{ex*}
Let $X=\P^d$.
We have a sequence of closed immersions $\P^0\subset\cdots\subset\P^d$ such that $\P^i/\P^{i-1}\cong\A^i$.
So we get
\[Z(\P^d,t)=\prod_{i=0}^dZ(\A^i,t)=(1-t)^{-1}(1-qt)^{-1}\cdots(1-q^dt)^{-1}.\]
\end{ex*}
\begin{ex*}
Let $X$ be a elliptic curve.
This has a multiplication map $X\times X\to X$, and we can iterate this and combine with the diagonal map to get a map $X\to X^m\to X:x\mapsto mx$.
Define the Tate module
\[T_\ell X:=\varprojlim_{n}\ker(X(\bar{\F_q})\xrightarrow{\ell^n}X(\bar{\F_q}))\in\Z_\ell\Mod.\]
Consider the action of the Frobenius morphism $\mathrm{Frob}:X\to X$ on $T_\ell X$ and induce
\[\mathrm{Frob}_\ell:T_\ell X\otimes_{\Z^\ell}\Q^\ell\to T_\ell X\otimes_{\Z^\ell}\Q^\ell.\]
One can calculate, by counting fixed points of the Frobenius map and to make it to a matrix with base change, $\#X(\F_{q^n})=\deg(1-\mathrm{Frob}^n)=\det(1-\mathrm{Frob}_\ell^n)=1-\alpha^n-\beta^n+q^n$, where $\alpha,\beta\in\C$ are complex conjugates with absolute value $\sqrt q$.
Using this one can show that
\[Z(X,t)=\frac{(1-\alpha t)(1-\beta t)}{(1-t)(1-qt)}.\]
For more details, see the chapter 5 of Silverman's book ``The arithmetic of elliptic curves''.
\end{ex*}

\begin{ex*}
Let $X$ be a smooth projective curve of genus $g$.
The zeta function can be written in terms of linear systems of divisors of line bundles using Riemann-Roch theorem for curves.
One can calculate
\[Z(X,t)=\frac{f(t)}{(1-t)(1-qt)},\]
where $f(t)\in\Z[t]$ has degree $2g$.
For more details, see Raskin's book ``The weil conjecture for curves''.
\end{ex*}

\begin{ex*}
Let $X$ be a smooth hypersurface defined by an equation of the form $\sum_{i=0}^ra_ix_i^{n_i}$.
Then,
\[Z(X,t)=\frac1{(1-t)\cdots(1-q^{r-1}t)}\prod_\alpha(1-C(\alpha)t^{\mu(\alpha)})^{(-1)^n/\mu(\alpha)},\]
where $\alpha\in(\F_q^\times)^{n+1}$, $\mu(\alpha)\in\N$, $C(\alpha)\in\C$ with $|C(\alpha)|=q^{\frac{(r-1)\mu(\alpha)}2}$.
For more details, see Weil's ``Number of solutions of equations in finite fields''.
\end{ex*}

\begin{thm*}[Weil conjecture]
Suppose $X$ is geometrically connected smooth projective variety over $\F_q$.
Then, the zeta function of $X$ satisfies the following properties:
\begin{enumerate}[(i)]
\item (Rationality) $Z(X,t)\in\Q(t)$,
\item (Functional equation) There is an integer $e$ such that $Z(X,q^{-m}t^{-1})=\pm q^{\frac{en}2}t^eZ(X,t)$,
\item (Riemann hypothesis) We can write
\[Z(X,t)=\frac{P_1(t)\cdots P_{2n-1}(t)}{P_0(t)\cdots P_{2n}(2t)},\]
where $P_i(t)\in\Z[t]$ and all roots of $P_i(t)$ have absolute value $q^{-i/2}$.
Moreover, $P_0(t)=1-t$ and $P_{2n}(t)=1-q^nt$.
\item (Betti numbers) Suppose there is a number field $K/\Q$ with ring of integers $\cO_K$, and homogeneous polynomials $f_1,\cdots,f_c\in\cO_K[t_0,\cdots,t_d]$ such that $\F_q\cong\cO_K/\fq$ and $X$ is defined by $\bar f_1,\cdots,\bar f_c\in\F_q[t_0,\cdots,t_d]$ (defining homogeneous polynomials can be lifted to $\cO_K$-valued polynomials).
Suppose further that for some embedding $K\subset\C$ the variety $X(\C)$ is still smooth.
Consider $X(\C)\subset\P^d(\C)$ as a smooth complex manifold.
Then, $\deg P_i(t)=\dim_\Q H^i_{\mathrm{Sing}}(X(\C),\Q)$.
\end{enumerate}
\end{thm*}


\subsection{Counting points with cohomology}
Suppose $M$ is an $n$-dimensional compact real manifold.
We have associated vector spaces $H^i(M,\Q)$ with $i\ge0$.
Heuristically, $H^i(M,\Q)$ counts $i$-dimensional holes in $M$.
\begin{ex*}
Let $M:=\C^m\setminus\{0\}$.
Then,
\[H^i(M,\Q)=\begin{cases}
\Q&i\in\{0,2m-1\},\\
0&\text{otherwise}.	
\end{cases}\]
\end{ex*}
\begin{ex*}
Let $M=\Sigma_g$ be a sphere with $g$ hands, or $M=X(\C)$ for $X$ smooth projective curve of genus $g$.
Then,
\[H^i(M,\Q)=\begin{cases}
\Q&i=0,\\
\Q^{2n}&i=1,\\
\Q&i=2,\\
0&i>2.
\end{cases}\]
\end{ex*}
\begin{ex*}
Let $M=\P^d(\C)$.
Then,
\[H^i(M,\Q)=\begin{cases}
\Q&i\in\{0,2,\cdots,2n\},\\
0&\text{otherwise}.	
\end{cases}\]
\end{ex*}

In general, the cohomology groups of connected compact real manifolds satisfy
\begin{enumerate}[(i)]
\item (Finiteness) $H^i(M,\Q)$ is finite-dimensional over $\Q$ for all $i$, and moreover if $M=X(\C)$ is a smooth projective variety of real dimension $d$, then $H^i(M,\Q)=0$ for $i>2d$,
\item (Functoriality) For any continuous map $\phi:M\to N$ there are linear transformations $H^i(\phi):H^i(N,\Q)\to H^i(M,\Q)$ compatible with compositions,
\item (Poincar\'e duality) There is a canonical isomorphism $H^{\dim M}(M,\Q)\cong\Q$ and a natural non-degenerate bilinear form between
\[H^i(M,\Q)\times H^{\dim M-i}(M,\Q)\to H^{\dim M}(M,\Q)\cong\Q.\]
\item (Lefschetz trace formula) Suppose $\phi:M\to M$ is a continuous map with only simple isolated fixed points.
Then, the number of fixed points is equal to an alternating sum
\[\sum_{i=0}^{\dim M}(-1)^i\tr(H^i(\phi)).\]
\end{enumerate}

Now suppose we had cohomology groups defined for algebraic varieties over $\F_q$, satisfying the analogues of the above properties.
Since $X(\F_{q^n})$ is the set of fixed points of
\[\mathrm{Frob}^n:X(\bar\F_q)\to X(\bar\F_q):(a_0:\cdots:a_d)\mapsto(a_0^{q^n}:\cdots:a_d^{q^n}),\]
an appropriate version of the Lefschetz trace formula would give
\[\#X(\F_{q^n})=\sum_{i=0}^{2\dim X}(-1)^i\tr H^i(\mathrm{Frob}^m).\]
Inserting this sum description of $Z(X,t)$, we have
\[Z(X,t)=\exp\sum_{n=1}^\infty\left(\sum_{i=0}^{2\dim X}(-1)^i\tr H^i(\mathrm{Frob}^n))\frac{t^n}n\right)=\prod_{i=0}^{2\dim X}\exp\left(\sum_{n=1}^\infty\tr H^i(\mathrm{Frob}^n)\frac{t^n}n\right).\]
Combine this with $\det(1-A)^{-1}=\exp\sum_{n=1}^\infty\tr A^n/n$, we get
\[Z(X,t)=\prod_{i=0}^{2\dim X}\det(1-tH^i(\mathrm{Frob}))^{(-1)^{i+1}}.\]
That is, we obtain the first conclusion of the Weil conjecture, the rationality.
Moreover, an appropriate version of Poincar\'e duality would imply the second conlusion of the Weil conjecture, the functional equation.
If the new cohomology groups are compatible with singular ones in an appropriate way, we get Berry numbers.
Finally, this description suggests that the polynomials $P_i(t)$ are $\dim(1-tH^i(\mathrm{Frob}))$, and so the second part of the Riemann hypothesis can be formulated as: the eigenvalues of $H^i(\mathrm{Frob})$ have absolute value $q^{-i/2}$.

How can we define a cohomology theory for algebraic varieties over $\F_q$?

In part I, we define and study
\[H_{\mathrm{\acute et}}^i(X,\Q_\ell):=(\varprojlim_nH_{\acute et}^i(X,\Z/\ell^n))\otimes_{\Z_\ell}\Q_\ell.\]
In part II, we define and study
\[H_{\mathrm{pro\acute et}}^i(X,\Q_\ell).\]


\newpage
\section{Day 2: April 18}

Last week, we motivated the \'etale cohomology in the context of the Weil conjecture.
We want a cohomology theory for varieties over $\F_q$ which behaves like $H_{\mathrm{Sing}}^\bullet(X(\C),\Q)$ for smooth varities $X$ over $\C$, where the topology of $X$ comes from $\C^n\cong\R^{2n}$.
For ``nice'' topological spaces $M$, the cohomology $H_{\mathrm{Sing}}^1(M,\Q)$ can be defined in terms of local homeomorphisms.

\subsection{Flatness}
\begin{defn*}
Let $A$ be a ring.
An $A$-module $M$ is \emph{flat} if tensor functor $M\otimes_A-$ preserves monomorphisms in $A$-$\mathrm{Mod}$.
An $A$-algebra $B$ is \emph{flat} when it is flat as an $A$-module.
A homomorphism $A\to B$ is \emph{flat} if $B$ is flat over $A$.
\end{defn*}
\begin{exe*}
Show that if $k$ is a field, then all $k$-modules are flat.
\end{exe*}
\begin{exe*}
Show that if $A$ is a ring, and $S\subset A$ is a multiplicatively closed subset, then the localization $A[S^{-1}]$ is a flat $A$-algebra.
\end{exe*}
\begin{ex*}
An $A$-module $M$ is flat if and only if $M_\fp$ is flat for every prime ideal $\fp\subset A$.
\end{ex*}
\begin{ex*}
For any sequence $f_1,\cdots,f_n\in A$, the product of the localizations $\prod_{i=1}^nA[f_i^{-1}]$ is flat.
\end{ex*}
\begin{ex*}
Neither of $k[x,y/x]$ and $k[x/y,y]$ are flat over $k[x,y]$.

Intuition: flat modules mean that the dimension of fibers should be locally constant.
They represent blow-ups over $\A^2$, where the origin has a fiber of different dimension compared to neighborhoods.
As a more general principle, flatness is like a local uniformity condition.
\end{ex*}

\begin{exe*}
Let $A$ be a ring and $I$ an ideal of $A$.
Show that if $A/I$ is a flat $A$-algebra, then $I=I^2$.
Show that if $I$ is finitely generated and $I=I^2$, then $A/I$ is flat.
Give an exmample of an idela $I$ in $A$ such that $I=I^2$ but $A/I$ is not flat.

Intuition: $A\to A/I$ is like a closed immersion $\Spec A/I\to\Spec A$.
$\Spec A[f^{-1}]\to\Spec A$ is an open immersion (with multiple fibers if it is producted).
\end{exe*}

\begin{exe*}
Show that if $A\to B$ and $B\to C$ are flat ring homomorphisms, then $A\to C$ is flat.
Show that if $A\to B$ is flat, then $D\to D\otimes_AB$ is flat for $A\to D$ is any homomorphism.
\end{exe*}

\begin{defn*}
Let $A$ be a ring.
An $A$-module $M$ is \emph{faithfully flat} if it is flat and a morphism $\phi$ of $A$-modules is a monomorphism if $\id_M\otimes_A\phi$ is a monomorphism.
\end{defn*}

\begin{ex*}
A homomorphism $A\to B$ is faithfully faithful if and only if it is flat and $\Spec B\to\Spec A$ is surjective.
\end{ex*}
\begin{exe*}
Let $M$ be a flat $A$-module.
Show that $M$ is faithfully flat if and only if for any $A$-module $N$ such that $M\otimes_AN\cong0$, we have $N\cong0$.
Show that if $M$ faithfully flat, then given any morphism of $A$-modules $N\to N'$ such that $M\otimes_AN\to M\otimes_AN'$ is surjective, the morphism $N\to N'$ is also surjective.
Deduce that if $M$ is faithfully flat, then a sequence of $A$-modules is exact if it is exact after applying $M\otimes_A-$.
\end{exe*}

\begin{thm*}[Milne, I.2.17]
Suppose $f:A\to B$ is a faithfully flat homomorphism.
Then,
\[0\to A\to B\xrightarrow{d}B^{\otimes_A2}\xrightarrow{d}B^{\otimes_A3}\to\cdots\]
is an exact sequence of $A$-modules, where $d:=\sum_{i=0}^r(-1)^ie_i$ for
\[e_i:b_0\otimes\cdots\otimes b_{r-1}\mapsto b_0\otimes\cdots\otimes b_{i-1}\otimes1\otimes b_i\otimes\cdots\otimes b_{r-1}.\]
\end{thm*}

\begin{exe*}
Show that $df=0$ and $d^2=0$.
\end{exe*}

\begin{pf}
First suppose there is $r:B\to A$ with $rf=\id$.
Define
\[s:B^{\otimes_A(r+2)}\to B^{\otimes_A(r+1)}:b_0\otimes\cdots\otimes b_{r+1}\mapsto r(b_0)b_1\otimes\cdots\otimes b_{r+1}\]
and check that $sd+fr=\id$ and $sd+ds=\id$.
In other words, $s$ forms a chain complex homotopy between zero and the identity.
Consequently, the sequence is exact.

Now consider some $A$-algebra $A'$.
Let $B':=A'\otimes_AB$ and $f'=A'\otimes_Af$.
Since $A'\otimes_A(B\otimes_A\cdots\otimes_AB)=(A'\otimes_AB)\otimes_{A'}\cdots\otimes_{A'}(A'\otimes_AB)$, applying $A'\otimes_A-$ to the sequence for $f$ gives the sequence for $f'$.
So if we can find a faithfully flat $A\to A'$ such that $A'\to B'$ has a retraction $B'\to A'$, we are done.
Take $A'=B$ with $r:B\otimes_AB\to B:(b_1\otimes b)\mapsto b_1b_2$.
\end{pf}


\subsection{Unramified morphisms}
Recall that if $\fp$ is a prime ideal of $A$, then one defines $k(\fp):=A_\fp/\fp A_\fp$.

\begin{defn*}
A morphism of rings $\phi:A\to B$ is \emph{unramified} at a prime $\fq\subset B$ if $k(\fp)\to k(\fq)$ is a finite separable field extension(hence $\fp=\phi^{-1}\fq$ and $\fq B_\fq=\phi(\fp)B_\fq$.
It is \emph{unramified} if it is finitely presented and unramified at every prime $\fp\in B$.
\end{defn*}

\begin{ex*}
$k[x]\to k[x,y]/(x^2-y^2)$ and $k[x]\to k[x]:x\mapsto x^2$ are unramified except at $\fq=(x,y)$ and $\fq=(x)$, respectively.
\end{ex*}

\begin{rmk*}
A finitely presented morphism $A\to B$ is unramified if and only if $B\otimes_AB\to B$ is flat.
\end{rmk*}

\begin{exe*}
Let $A\to B\to C$ and $A\to D$ be ring homomorphisms.
Show that if $A\to B$ and $B\to C$ is unramified, then $A\to C$ is unramified.
Show that blabla
\end{exe*}

\subsection{\'Etale morphisms}
\begin{defn*}
A morphism of finite presentation is \emph{\'etale} if it is flat and unramified.
\end{defn*}
\begin{rmk*}
A morphism is \'etale if and only if it is smooth and relative dimension zero.
\end{rmk*}
\begin{ex*}
Let $k$ be a field, and $k\to\A$ a finitely presented $k$-algebra.
Then, $A$ is \'etale if and only if $A\cong L_1\times\cdots\times L_n$ for some finite separable field extensions $L_i/k$
\end{ex*}

\begin{exe*}
Show that if $A\to B$ and $B\to C$ is \'etale, then so is $A\to C$.
Show that if $A\to B$ is \'etale, so is $D\to D\otimes_AB$.
\end{exe*}
\begin{ex*}
Suppose $Y\to X$ is a morphism of smooth affine complex varieties, and say $Y=\Spec B$ and $X=\Spec A$.
Then, $A\to B$ is \'etale if and only if $Y(\C)\to X(\C)$ is a local homeomorphism between the analytic topologies.
\end{ex*}

\subsection{Hensel rings}
\begin{defn*}
A local ring $A$ with maximal ideal $\fm$ is \emph{henselian} if for every \'etale morphism $\phi:A\to B$ and for every prime $\fq\subset B$ such that $\phi^{-1}\fq=\fm$ and $k(\fq)=k(\fm)$, there is a ring homomorphism $\sigma:B\to A$ such that $\sigma^{-1}\fm=\fq$ and $\sigma\phi=\id$.

It can be considered as a kind of the inverse function thorem
\[\begin{tikzcd}
\Spec k(\fq) \rar\dar[equal] & \Spec B\dar\\
\Spec k(\fm) \rar & \Spec A\uar[bend left, dashed].
\end{tikzcd}\]
\end{defn*}

\begin{ex*}
A field is henselian.
Any complete local ring is henselian, for example, $k[[t]]$.
\end{ex*}
\begin{prop*}
For every local ring $A$, there exists a universal local morphism to a local henselian ring, that is, there exists a local morphism $A\to A^h$ to a local henselian ring $A^h$, such that for any other local homomorphism $A\to B$ to a local henselian ring $B$ there exists a unique factorization $A^h\to B$ of $A\to B$.
\end{prop*}
\begin{pf}
Consider the category of factorizations $A\xrightarrow{\phi}B\to A/\fm$ such that $\phi$ is \'etale.
Morphisms are naturally defined as commutative diagrams.
This category has an initial object $A=A\to A/\fm$ and pushouts constructed by tensor products, so this is a filtered category.
Define $A^h$ to be the colimit of $B$ with respect to this filtered category.
Then, we can check $A^h$ is a local ring with the same residue field as $A$, $A^h$ is henselian, and it has the universal property.

(Compare this procedure with the construction of an algebraic closure)
\end{pf}

\begin{rmk*}
If $A$ is henselian local ring, then since its completion $\hat A=\varinjlim_n A/\fm^n$ is henselian, so there is a unique canonical factorization $A\to A^h\to \hat A$, and in particular, both morphisms are monomorphisms.
\end{rmk*}
\begin{rmk*}
For normal integral $A$, $A^h\subset\mathrm{Frac}(A)^s$, the separable closure of the field of fractions.
\end{rmk*}

\begin{defn*}
A local henselian ring is \emph{strictly local} if its residue fields $A/\fm$ is separably closed.
\end{defn*}

\begin{exe*}
Show if $A$ is strictly local, then every faithfully flat \'etale morphism $A\to B$ admits a retraction $B\to A$.
\end{exe*}

\begin{prop*}
If $A$ is a local ring, for $\phi:k(\fm)\to k(\fm)^s$, there is a universal local homeomorphism of local rings $A\to A^{sh}$ such that $A^{sh}$ is strictly local, and $\phi$ is equal to the map $A/\fm\to A^{sh}/\fm^{sh}$.
\end{prop*}
\begin{exe*}
Let $\phi:A\to B$ be a finite \'etale morphism.
Then, for every $\fp\subset A$ we have $A_\fp^{sh}\otimes_AB\cong\prod_{i=1}A_\fp^{sh}$.
\end{exe*}


\newpage
\section{Day 3: April 25}

\subsection{Sheaves}


\begin{defn*}
A \emph{Grothendieck topology} on a category $\cC$ is the data of: for every objects $U\in\cC$, we are given a collection of families of moprhisms $\{U_i\to U\}_i$.
Each distinghuished family for $U$ is called the \emph{covering} of $U$, and a Grothendieck topology assigns a collection of coverings of $U$ for each object $U$.
This data is required to satisfiy
\begin{enumerate}
\item $\{U\xrightarrow{\id}U\}$ is a covering of each object $U$,
\item If $\{U_i\to U\}$ is a covering of $U$ and $V\to U$ is any morphisms, then the pullback $V\times_UU_i$ exists, and $\{V\times_UU_i\to V\}$ is a covering of $V$,
\item If $\{U_i\to U\}$ is a covering of $U$ and for every $i$ the family $\{U_{ij}\to U_i\}_{j\in J_i}$ is a covering of $U_i$, then $\{U_{ij}\to U_i\to U\}_{i\in I,\ j\in J_i}$ is a covering of $U$.
\end{enumerate}
A category equipped with a topology is called a \emph{site}.
\end{defn*}

\begin{exe*}
Let $X$ be a topological space.
Show that the category $\cO(X)$ of open sets is canonically a site defined as follows:
A family $\{U_i\to U\}_i$ is a covering if $\bigcup U_i=U$.
\end{exe*}
\begin{exe*}
Let $X$ be a topological space.
Define $LH(X)$ to be the category such that objects are local homeomorphisms from a topological space $Y$ to $X$ and morphisms are commutative triangles (if we restrict the locally homeomorphisms to finitely presented ones, then it is the category of \'etale bundles).
Define coverings $\{f_i:Y_i\to Y\}_i$ such that $\bigcup f_i(Y_i)=Y$.
Show that it defines a site.
\end{exe*}
\begin{exe*}
Recall that a morphism of schemes $Y\to X$ is \'etale if it is locally of finite presentation and for every $y\in Y$, $\cO_{X,f(x)}\to\cO(Y,y)$ is \'etale.
Define $\mathrm{\acute Et}(X)$ be the category of \'etale morphisms to $X$.
Show that $\mathrm{\acute Et}(X)$ is a site.
\end{exe*}
\begin{exe*}
Let $A$ be a ring.
The coverings on $\mathrm{\acute Et}(X)$ are as follows: $\{B\to B_i\}_i$ such that there is finite $J$ such that $B\to\prod_jB_j$ is faithfully flat.
\end{exe*}

\begin{defn*}
Let $\cC$ be a site.
A presheaf $\cF$ on $\cC$ is a \emph{sheaf} if for every covering $\{U_i\to U\}_{i\in I}$ we have
\[\cF(U)=\mathrm{eq}\left(\prod_{i\in I}\cF(U_i)\rightrightarrows\prod_{i,j\in I}\cF(U_i\times_UU_j)\right)=\{(s_i)_{i\in I}\in\prod\cF(U_i):\cF(\pr_1)(s_i)=\cF(\pr_2)(s_j)\}.\]
\end{defn*}
\begin{rmk*}
If $A$ is a ring, write $\Et(A)=\Et(\Spec(A))$, $F(B)=F(\Spec B)$.
\end{rmk*}
\begin{ex*}
Let $A$ be a ring.
The followins are sheaves on $\mathrm{\acute Et}(A)$:
\begin{enumerate}[1)]
\item $\cO:B\to(B,+)$,
\item $\cO^\times:B\to (B^\times,\times)$,
\item $\mu_n:B\to\{b\in B:b^n=1\}$,
\item $\Omega^n:B\to\Omega_B^n$.
\end{enumerate}
\end{ex*}

\begin{rmk*}
If a presheaf $\cF$ takes values in abelian groups, then the sheaf condition is equivalent to asking that
\[0\to\cF(U)\to\prod_i\cF(U_i)\xrightarrow{\cF(p_1)-\cF(p_2)}\prod_{i,j}\cF(U_i\times_UU_k)\]
is exact.
\end{rmk*}

\begin{exe*}
Let $\Spec(L)\to\Spec(L')$ be a morphism in $\Et(k)$ for some field $k$ such that $L/L'$ is Galois with group $G=\Gal(L/L')$.
Recalle that $L\otimes_L'L\cong\prod_{g\in G}L$ and the two morphisms $L\rightrightarrows L\otimes_L'\to L:a\mapsto(a\otimes1),(1\otimes a)$ corresponds to $L\rightrightarrows\prod_{f\in G}L:a\mapsto(a,a,a,a,a,)$.
Show that if $\cF$ is a sheaf of $\Et(k)$, then $\cF(\prod_GL)\cong\prod_G\cF(L)$ and $\cF(L')=\cF(L)(=\{s\in\cF(L):\cF(g)(s)=s,\ g\in G\})$.
Deduce that if $\cF_1\to\cF_2$ is a morhpism of \'etale sheaves such that $\cF_1(L)\cong\cF_2(L)$ for every Galois $L/k$, then $\cF_1\cong\cF_2$.
\end{exe*}

\begin{thm*}[Milne, II.1.9]
Suppose that $k$ is a field with $G=\Gal(k^{sp}/k)$.
Then, there is a canonical equivalence of categories $G$-Set and Sh(\'Et($k$)).
\end{thm*}


\subsection{Sheafification}

\begin{defn*}
A presheaf $\cF$ on a site is called \emph{separated} if $\cF(U)\to\prod_i\cF(U_i)$ is injective for every covering.
Every sheaf is separated.
\end{defn*}
\begin{exe*}
Suppose $U$ is an object of a site $\cC$ and a presheaf $\cF$ of abelian groups.
Define $\cF^s(U):=\cF(U)/\bigcup_{\text{coverings}}\ker(\cF(U)\to\prod\cF(U_i))$.
Show that $\cF^s$ is a separated presheaf, and if $\cF\to\cG$ is a morphism with $\cG$ a separated presheaf, then there is a unique factorization $\cF\to\cF^s\dashrightarrow\cG$.
\end{exe*}

\begin{prop*}
Let $\cC$ be a site.
For every presheaf $\cF$, there is a universal morphsim towards a sheaf.
That is, a sheaf $\cF^a$, a morphism $\cF\to\cF^a$, such that for every $\cF\to\cG$, with $\cG$ sheaf, there exists a unique factorization $\cF\to\cF^a\dashrightarrow\cG$.
\end{prop*}
\begin{pf}
By the previous exercise, it suffices to consider the case $\cF$ is separated.
For $U\in\cC$ define
\[\check H^0(U,\cF):=\colim_{\text{coverings}}\mathrm{eq}\left(\prod_I\cF(U_i)\rightrightarrows\prod_{I\times I}\cF(U_i\times_UU_j)\right).\]
Then, $\cF(U)\to\check H^0(U,\cF)$ defines a morphism of presheaves.
We get
\[\begin{tikzcd}
\cF \rar\dar & \check H^0(\cF) \dar\ar[dotted]{dl}\\
\cG \rar[equal] & \check H^0(\cG).
\end{tikzcd}\]
So it is enough to show $\check H^0(\cF)$ is a sheaf.
For simplicity, suppose $\cF$ is a presheaf of abelian groups.
Let $\{V\to U\}$ be a covering with one element.
We want to show
\[0\to \check H^0(U,\cF)\to\check H^0(V,\cF)\to\check H^0(V\times_UV,\cF)\]
is exact.

\end{pf}

\begin{cor*}
Let $\cC$ be a small site.
Then, the category $\mathrm{Sh}(\cC,\mathrm{Ab})$ is abelian.
\end{cor*}

\begin{rmk*}
$\mathrm{Sh}(\cC,\mathrm{Ab})\to\mathrm{PSh}(\cC,\mathrm{Ab})$ is a right adjoint, so it preserves limits.
It also preserves colimits.
\end{rmk*}

\subsection{Stalks}
\begin{defn*}
A \emph{geometric point} of a scheme is a morphism $\bar x\to X$ such that $\bar x=\Spec\Omega$ with $\Omega$ a separably closed field.
\end{defn*}
\begin{defn*}
Take $\cF\in\mathrm{PSh}(\Et(X))$.
The \emph{stalk} of $\cF$ at a geometric point $\bar x\to X$ is 
\[\cF_{\bar x}=\varinjlim_{\bar x\to Y\to X}\cF(Y).\]
\end{defn*}


...
\begin{prop*}
Suppose $\cF$ is a sheaf of abelian groups on $\Et(X)$ for a scheme $X$, and $Y\in\Et(X)$, and $s\in\cF(Y)$.
Then, $s=0$ iff $s_{\bar x}=0$ for all $\bar x\to Y$, where $s_{\bar x}$ is the image of $s$ under $\cF(Y)\to\cF_{\bar x}$.
\end{prop*}

\newpage
\section{Day 4: May 2}

\subsection{Intro}
Let $\cA$ be an abelian site.
For example, $\cA=\mathrm{Sh}_{\acute et}(X,\mathrm{Ab})$, $\mathrm{Ab}$, $R\mathrm{Mod}$, etc.

Suppose $\Phi:\cA\to\cB$ is a finite limit preserving additive functor between abelian site.
For example, $\Phi=\Gamma(X,-):\mathrm{Sh}_{\acute et}(X,\mathrm{Ab})\to\mathrm{Ab}$, or $\Phi=\mathrm{Hom_R}(M,-):R\mathrm{Mod}\to R\mathrm{Mod}$.
We want to extend the exact sequence
\[0\to\Phi(A)\to\Phi(B)\to\Phi(C)\]
to the right.


\subsection{Chain complexes}
There is a unique colimit preserving functor $\cA\times\mathrm{Ab}\to\cA:(M,A)\mapsto M\otimes A$.

We may consider $\cA$ as a subcategory of $\mathrm{Ch}(\cA)$, which again an abelian site.
The bilimits, bikernels, direct sums, etc are all computed componentwise.

In $\mathrm{Ch}(\mathrm{Ab})$, we have
\[S^n:=[\cdots0\to\to0\to\Z\to0\to\cdots],\quad D^{n+1}:=[\cdots\to0\to\Z\to\Z\to0\to\cdots]\]
($-n-2,-n-1,-n,-n+1$).
We have an exact sequcne $0\to S^n\to D^{n+1}\to S^{n+1}\to0$.

Extend $\otimes$ to $\mathrm{Ch}(\cA)\times\mathrm{Ch}(\mathrm{Ab})\to\mathrm{Ch}(\cA)$, where $C^\bullet\otimes A^\otimes)^n:=\bigoplus_{i+j=n}C^i\otimes A^j$ and $d(f\otimes g)=df\otimes g+(-1)^{|f|}f\otimes dg$.
For $C^\bullet\in\mathrm{Ch}(\cA)$ and $j\in\Z$, $C^\bullet[j]:=C^\bullet\otimes S^j$.
For $f:A^\bullet\to B^\bullet$ in $\mathrm{Ch}(\cA)$, the \emph{mapping cone} is the pushout
\[\begin{tikzcd}
A^\bullet\otimes S^0 \rar\dar & A^\bullet\otimes D^1 \dar\\
B^\bullet\otimes S^0 \rar & \mathrm{Cone}(f).
\end{tikzcd}\]
For $A^\bullet,B^\bullet\in\mathrm{Ch}(\cA)$,
\[\mathrm{Map}(A^\bullet,B^\bullet)^n:=\prod_{i\in\Z}\mathrm{Hom}_\cA(A^i,B^{i-n})\]
and $df:=d_Bf-(-1)^nfd_A$ defines a complex $\mathrm{Map}(A^\bullet,B^\bullet)\in\mathrm{Ch}(\mathrm{Ab})$.


\subsection{Quasi-isomorphisms}
We can define cohomology functor $H:\mathrm{Ch}(\cA)\to\mathrm{grAb}$.

If $R$ is a field or $\Z$, then every chain complex over $R$ is quasi-isomorphic to a chain complex with zero morphisms.
If $R$ is $\Z/4\Z$ or $\Q[x,y]$, then there is a counterexample.


\subsection{Fibrant replacement}
A chain complex $Q^\bullet$ is called \emph{fibrant} if for every monomorphic quasi-isomorphism $A^\bullet\to B^\bullet$ and every morphism $A^\bullet\to Q^\bullet$, there exists a factorization $A^\bullet\to B^\bullet\dashrightarrow Q^\bullet$.

Suppose $\Phi:\cA\to\cB$ is a limit preserving functor between abelian site.
The \emph{derived functors} of $\Phi$ are the cohomology $R^n\Phi(A):=H^n(\Phi(Q_A^\bullet))$, where $A\hookrightarrow Q_A^\bullet$ is any monomorphic quasi-isomorphism towards a fibrant $Q_A^\bullet$.
In particular, given $F\in\mathrm{Sh}(X,\mathrm{Ab})$,

Claim:
\begin{enumerate}
\item $R^n\Phi(A)$ exist and are functorial in $A\in\cA$,
\item $R^n\Phi(A)$ are independent of the choice of $Q_A$ (up to canonical natural isomorphism),
\item $R^n\Phi(A)$ send short exact sequences to long exact sequences.
\end{enumerate}

\subsection{Functorial fibrant replacement}
\begin{thm*}
Let $\cA$ be an abelian site.
There exists a functor $Q:\mathrm{Ch}(\cA)\to\mathrm{Ch}(\cA)$ and a natural isomorphism $\id\Rightarrow Q$ such that for each $C^\bullet\in\mathrm{Ch}(\cA)$, $QC^\bullet$ is fibrant and $C^\bullet\to QC^\bullet$ is a monomorphic quasi-isomorphism.
\end{thm*}



We can write $\cA\cong\mathrm{Ind}_\kappa(\cA^\kappa)$ for some small $\cA^\kappa$ and regular cardial $\kappa$.
For example, if $\cA=\mathrm{Ab}$, and $\kappa=\aleph_0$, then $\mathrm{Ab}^\kappa$ only includes finitely generated abelian groups.

small object argument




\newpage
\section{Day 5: May 9}

\subsection{The homotopy category}
Define
\[\Delta^1:=\{\cdots\to0\to\Z\xrightarrow{f}\Z\oplus\Z\to0\to\cdots\},\qquad\Delta^0:=\Z=\{\cdots\to0\to\Z\to0\to\cdots\},\]
where $f:\Delta^1_{-1}\to\Delta^1_0:n\mapsto(n,n)$.
In $\mathrm{Ch}(\mathrm{Ab})$ we have
\[\begin{tikzcd}
\Z \dar[hook]{\iota_0} & \\
\Delta^1 \rar{\pi} & \Z \ar[equal]{lu}\ar[equal]{ld}, \\
\Z \uar[hook]{\iota_1} &
\end{tikzcd}\]
where $\iota_0:n\mapsto(n,0)$, $\iota_1:n\mapsto(0,-n)$, $\pi:(m,n)\mapsto m-n$, the face maps and degeneracy maps.
It is an anlogy of
\[\begin{tikzcd}
\{0\}\dar[hook] & \\
[0,1] \rar & \{*\} \ar[equal]{lu}\ar[equal]{ld}. \\
\{1\} \uar[hook] &
\end{tikzcd}\]
For $f,g:A^\bullet\rightrightarrows B^\bullet$ in $\mathrm{Ch}(\cA)$, a \emph{chain homotopy} from $f$ to $g$ is a morphism $h:A^\bullet\otimes\Delta^1\to B^\bullet$ such that $h(1\otimes\iota_0)=f$ and $h(1\otimes\iota_1)=g$.


$\mathrm{Ch}(\mathrm{Ab})$ is like something simplicial, $\mathrm{Ch}(\cA)$ is like the category of topological spaces, the algebraic homotopy category $K(\cA)$ is like the homotopy category of topological spaces.
Quasi-isomorphisms and fibrant complexes are the analogies of weak equivalences and CW complexes.
The fibrant replacement $A^\bullet\mapsto QA^\bullet$ is the analogy of the geometric realization $X\mapsto|\mathrm{Sing}_*X|$.


\subsection{Derived category}

Given a morphism $f$ in $\mathrm{Ch}(\cA)$, $f$ is a quasi-isomoprhism if $f$ is an isomorphism in $K(\cA)$, but not is the case for the converse.
We want to find a small category than $K(\cA)$ in which the isomorphisms are exactly given by quasi-isomorphisms.

An object $Q^\bullet$ of $K(\cA)$ is called \emph{quasi-isomorphism local} if for every quasi-isomorphism $f:A^\bullet\to B^\bullet$ the induced morphism $\Hom(B^\bullet,Q^\bullet)\to\Hom(A^\bullet,Q^\bullet)$ is an isomorphism in $K(\cA)$.
The full subcategory of quasi-isomorphism local objects is denoted by $\cD(\cA)\subset K(\cA)$, and called the \emph{derived category} of $\cA$.
There are many other ways of defining $\cD(\cA)$.


\begin{prop*}
Every fibrant complex is quasi-isomoprhism local.
\end{prop*}

Reduce the case to monics, monic quasi-isomorphisms, and to the final case.


\begin{cor*}
Any choice of fibrant replacement $Q:\mathrm{Ch}(\cA)\to\mathrm{Ch}(\cA)$ induces a left adjoint $L:K(\cA)\to\cD(\cA)$ to inclusion $K(\cA)\leftarrow\cD(\cA):\iota$.
In particular, the derived functor $R^n\Phi$ are independent of the cohice of $Q$.
\end{cor*}



\subsection{Identifying the zeroth derived functor}
\begin{prop*}
If $A\in\cA$ and $\Phi:\cA\to\cB$ is additive, then $R^n\Phi(\cA)=0$ for $n<0$.
If $\Phi$ preserves finite limits, then $R^0\Phi(A)=\Phi(A)$.
\end{prop*}


\subsection{Exact sequences}


\begin{prop*}
If $0\to A\to B\to C\to0$ is exact in $\cA$ and $\Phi:\cA\to\cB$ is additive, then $R\Phi A\to R\Phi B\to R\Phi C$ is a cofiber sequence in $K(\cB)$.
If $\Phi$ preserves finite limits, we get a long exact sequence $0\to\Phi A\to\Phi B\to\Phi C\to R^1\Phi A\to\cdots$.
\end{prop*}



A \emph{differential graded category} is a category $\cC$ enriched over the monoidal category $\Ch(\Ab)$.
It means that there is a functor $\mathrm{Map}:\cC^\op\times\cC\to\Ch(\Ab)$ such that

An \emph{infinite categorical differential graded category} is an infinite category associated to $\cD(\cA)$ with the $\mathrm{Map}$ as mapping complexes...? objects are fibrant?




\bigskip

Differential graded:

Abelian categories?
The category of cochain complexes forms a differential graded category.
Its nerve is an infinite category, but not of infinite groupoids.

The category of cochain complexes defines its derived category.

The nerve of the simplicial category of Kan complexes?

derived functor of $\Phi$: compute the cohomology after seding through $\Phi$ (with fibrant replacement)



\end{document}