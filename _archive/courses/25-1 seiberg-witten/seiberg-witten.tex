\documentclass{../../../small}
\usepackage{../../../ikhanchoi}

\begin{document}
\title{Seiberg-Witten Theory}
\author{Ikhan Choi\\Lectured by Konno Hokuto\\University of Tokyo, Spring 2025}
\maketitle
\tableofcontents

\newpage
\section{Day 1: April 7}

cancelled: 4/28, 5/19

References
\begin{itemize}
\item Morgan ``The Seiberg-Witten equations and applications to the topology of smooth 4-manifolds''
\item Sasahira ``Saibaagu-wittenhouteishiki'' (Japanese)
\item Matsumoto ``Yojigentayoutai I,II''
\end{itemize}

Plan
\begin{itemize}
\item Introduction/results from SW theory
\item SW equations
\item SW invariant (definition/properties)
\item Finite-dimensional approximation of the SW equation
\item Bauer-Furuta invariant, Donaldson's diagonalization theorem
\end{itemize}


\subsection*{Introduction}

SW theory $\subset$ Gauge theory.
We do not regard physical aspects.

Gauge theory: Using some non-linear PDE, study the topology of smooth 4-manifolds.

Two big methods:
\begin{enumerate}
\item 
By ``counting'' solutions to the PDE, construct an invariant of smooth 4-manifolds.
Distinguish smooth 4-manifolds (detect exotic 4-manifolds, and two smooth manifolds are called \emph{exotic} if they are homeomorphic but not diffeomorphic)
\item
By studying the ``solution space''(moduli space) for the PDE, give a constraint on the topology of smooth 4-manifolds. (Donaldson)
Detect non-smoothable topological 4-manifolds, such manifolds are known to exist for dimension $\ge4$.
\end{enumerate}

More about 1.
There is a fact that a compact topological manifold of dimension $\ne4$ admits at most finitely many smooth structures.
(cf. Kirby-Siebenmann theory in dim $>4$.)
It is still difficult to compute the number in general.
On the other hand, there are many compact 4-manifolds that admit infinitely many smooth structures.
It is shown by gauge theoretic invariants such as Donaldson, SW, Bauer-Furuta.
For example, Kodaira showed the K3 surface is unique up to diffeomorphism (of course uncountably many up to isomorphism).
If $K3:=\{[z_0:\cdots:z_3]\in\CP^3:z_0^4+\cdots+z_3^4=0\}$, then it is known to admit infinitely many smooth structures.
(cf. log transform, (elliptic fibration$\to$new elliptic fibration, this is closed in complex geometry, and Kronheimer showed $SO(3)$-Donaldson invariant), knot-surgery)

More about 2.
Donaldson's diagonalization theorem and Furuta 10/8 inequality.


It is an open problem that asks if every smoothable topological 4-manifold has infinitely many smooth structures.
Bauer-Furuta invariant gives a hint, but it is negative to be solved in this way...
In experience true, knot theoretically?
Gauge theory is effective when the intersection form is far from trivial.
Example that we do not know if it is exotic: $S^4$, $\CP^2$, $S^2\times S^2$, $\CP^2\#\bar\CP^2$.

\subsection{Intersection form}

Let $X$ be an oriented closed topological 4-manifold.
Then, $Q_X:(H^2(X,\Z)/\Tor)^{\otimes2}\to\Z:\alpha\otimes\beta\mapsto\<\alpha\smile\beta,[X]\>$ is called the \emph{intersection form}, and it is clearly homotopy invariant.
As a remark, by the Poincar\'e duality we can identify $H^2(X,\Z)$ with $H_2(X,\Z)$, and $Q_X(\alpha,\beta)$ means $\#(\Sigma\cap\Sigma')$ counting with sign, where $\alpha=[\Sigma]$ and $\beta=[\Sigma']$ with $\Sigma$ and $\Sigma'$ oriented closed embedded surfaces in $X$ (existence of such representative may not be clear for topological cases, but relatively clear for smooth cases.).

For example if $X=S^4$, then $H^2(S^4)=0$ and $Q_{S^4}=*$ with size zero.

For example if $X=\CP^2$, then $H_2(\CP^2)=\Z[\CP^1]$ and $Q_{\CP^2}=(1)$ with size one.

For example if $X=S^2\times S^2$, then $H_2(S^2\times S^2)=\Z[S^2\times*]\oplus\Z[*\times S^2]$ and $Q_{S^2\times S^2}=\left(\mat[small]{0&1\\1&0}\right)$.
We can check self-intersection vanishes and off-diagonal part has a single intersection. (perturbing using trivial normal bundle?)

In general, $Q_X$ is symmetric.
If we consider the same device for 2-dimensional, it becomes anti-symmetric.
It is also bilinear over $\Z$ and non-degenerate.

\begin{exe}
The intersection form is non-degenerate.
Use the Poincar\'e duality.
\end{exe}

There is a converse question.
When is a non-degenerate symmetric bilinear form over $\Z$ the intersection form of a topological 4-manifold?

\begin{thm}[Freedman, 1982]
Let $Q$ be a non-degenerate symmetric bilinear form over $\Z$.
Then, there is a topological simply connected closed orientable 4-manifold $X$ such that $Q_X\cong Q$ over $\Z$.
Moreover, such $X$ are at most two up to homeomorphism, and determined by the Kirby-Siebenmann invariant $KS(X)\in H^4(X,\Z/2\Z)\cong\Z/2\Z$.
\end{thm}

(proved by wild topology such as approximation with Cantor sets, and there would be almost no people understanding the proof.)

This theorem says that the intersection form is strong enough in topological cases.
Also it is known that the Kirby-Siebenmann invariant of a smoothable simply connected closed orientable 4-manifold vanishes.

In general, there are many definite quadratic form $Q$, i.e.~$Q(\alpha,\alpha)\ge0$ for all $\alpha$ or $Q(\alpha,\alpha)\ge0$ for all $\alpha$, when fixed the rank.
If the rank is $32$, then there are many more than $10^7$.


\begin{thm}[Donaldson's diagonalization theorem, 1983]
If $X$ is a smooth closed orientable 4-manifold (originally assumed to be simply connected, but it is revealed that it is not required) with $Q_X$ is negative-definite, then $Q_X$ is isomorphic to $\diag(-1,\cdots,-1)$ over $\Z$.
\end{thm}

In particular, there should be many many non-smoothable topological 4-manifolds by the diagonalization theorem and Freedman's thoeorem.


High-dimensional $>4$, exotic structures can be detected by characteristic classes, which is a dominating method.
Simply connected 4-manifolds, it is hard to say that there is another method to detect such phenomena...?

With boundary: Khovanov homology...


More about intersection forms:
\begin{defn}
Let $Q:L\otimes L\to\Z$ be a symmetric non-degenerate bilinear form over $\Z$, where $L$ is a lattice, i.e.~finitely generated free abelian group.
Then,
\[b^+(Q):=\text{max dim of positive-definite subspace of $L\otimes_\Z\R$ with respect to $Q$},\]
\[b^-(Q):=\text{max dim of negative-definite subspace of $L\otimes_\Z\R$ with respect to $Q$},\]
and the signature
\[\sigma(Q):=b^+(Q)-b^-(Q).\]
Note that the rank of $L$ is $b^+(Q)+b^-(Q)$.

Also, $Q$ is called \emph{even} if $Q(\alpha,\alpha)\cong0$ modulo $2$ for all $\alpha$, and \emph{odd} if it not even.

When $X$ is a closed orientable topological 4-manifold, then we use the following notations:
\[b^\pm(X):=b^\pm(Q_X),\qquad\sigma(X):=\sigma(Q_X),\]
and parity and definiteness of $X$ is defined by the parity and definiteness of $Q_X$.
\end{defn}

\begin{ex}
\[\begin{array}{c|cccc}
&b^+&b^-&\sigma&\text{parity}\\\hline
S^4&0&0&0&\text{even}\\
\CP^2&1&0&1&\text{odd}\\
S^2\times S^2&1&1&0&\text{even}\\
K3&3&19&-16&\text{even}
\end{array}\]
\end{ex}

\begin{exe}
$Q_{K3}=3\left(\mat[small]{0&1\\1&0}\right)\oplus2(-E_8)$, where $-E_8$ is an even, rank 8, negative definite form, given by
\[-E_8:=\mat{
-2&&&1&&&&\\
&-2&1&&&&&\\
&1&-2&1&&&&\\
1&&1&-2&1&&&\\
&&&1&-2&1&&\\
&&&&1&-2&1&\\
&&&&&1&-2&1\\
&&&&&&1&-2
}.\]
\end{exe}
See the following theorem in the course in arithmetic written by Serre.
\begin{thm}
Let $(Q,L)$ be a symmetric non-degenerate bilinear form over $\Z$.
Suppose $Q$ is indefinite, i.e.~not definite.
Then, over $\Z$,
\[Q\cong\begin{cases}
b^+(Q)(1)\oplus b^-(Q)(-1)&\text{ if $Q$ odd}\\
m\left(\mat[small]{0&1\\1&0}\right)\oplus n(-E_8)&\text{ if $Q$ even and $\sigma(Q)\le0$},
\end{cases}\]
for some $m\ge1$ and $n\ge0$.
\end{thm}

Other constraints on the intersection form of a smooth 4-manifold.
\begin{thm}[Rokhlin, 1952, before gauge theory]
Let $X$ be a smooth closed spin 4-manifold.
It is known that $X$ admits a spin structure iff $X$ is even, when $X$ is simply connected.
Then, $16$ divides $\sigma(X)$.
\end{thm}

In general, $8$ divides $\sigma(Q)$ if $Q$ is even.
Since $\sigma(-E_8)=-8$, so we can discover a non-smoothable manifold.

A conjecture by Y.~Matsumoto, 1982: For $X$ be a smooth closed spin 4-manifold, do we have $b_2(X)\ge\frac{11}8|\sigma(X)|$? Note that $b_i(X)$ is the rank of $H_i(X,\Z)$.
The equality is attained by $K3^{\#n}$ for $n\ge1$.
\begin{exe}
$Q_{X_1\# X_2}=Q_{X_1}\oplus Q_{X_2}$.
Use the Mayer-Vietoris.
\end{exe}

If this $11/8$-conjecture is solved, then we can obtain a complete classification of simply connected closed smooth 4-manifolds up to homeomorphism.

\begin{exe}
If this $11/8$-conjecture is solved, then simply connected closed smooth 4-manifolds are exhausted by $S^2\times S^2$, $\CP^2$, $\bar\CP^2$, $K3$, $-K3$, up to homeomorphism.
\end{exe}

There are no known examples whose smooth structures are classified.

\begin{thm}[Furuta, $10/8$-inequality, 2001]
Let $X$ be a smooth closed spin indefinite 4-manifold, then
\[b_2(X)\ge\frac{10}8|\sigma(X)|+2,\]
and the equality is attained by K3.
\end{thm}







\newpage
\section{Day 2: April 21}


Correction: non-degenerate $\to$ unimodular.
\subsection{SW invariant}

Input of SW invariant: closed oriented smooth 4-manifold $X^4$ with $b^+(X)\ge2$.


\begin{rmk}
There is a version for $b^+(X)=1$.
There is no reasonable definition of the SW invariant for $b^+(X)=0$.
\end{rmk}

Output is a $\Z$-valued invariant $SW(X,\fs)\in\Z$, where $\fs$ is a spin$^c$ structure on $X$.
The sign of this invariant is determined by an orientation of $H^1(X,\R)\oplus^+(X)$ called a \emph{homology orientation}.
Here $H^+(X)$ is a maximal dimensional positive-definite subspace of $H^2(X,\R)$ with respect to $Q_X$, meaning that $Q_X|_{H^+(X)}$ is a positive definite and $\dim H^+(X)=b^+(X)$.


\begin{ex}
Let $X=\CP^2\#2\bar\CP^2$.
Then, $H_2(X)=\Z H\oplus\Z E_1\oplus\Z E_2\cong\Z^3$, where $H$ is hyperplane class, and $E_1$ and $E_2$ are exceptional curve classes.
The smooth structure underlied on the connected sum with $\bar\CP^2$ corresponds to the blow-up at a point.
In $H_2(X,\R)$, the equation $Q_X\equiv0$ defines a subset of a real vector space $H_2(X,\R)$ which can be called the light cone, and $H^+(X)$ is a line in the `interior' of the light cone.
In this example, the set of choices of $H^+(X)$ is homeomorphic to the open unit disk.
\end{ex}

In general, we have a following exercise, which is not easy to prove at a first glance.

\begin{lem}
The moduli space of linear subspaces $H\subset H^2(X,\R)$ such that $b^+(X)=\dim H$ and $Q_X|_H$ is positive definite is contractible.
\end{lem}

Thanks to this lemma, if you pick one choice of $H^+(X)$, and fix an orientation on it, then it induces an orientation on another choice of $H^+(X)$ in a canonical way.
Every choice of $H^+(X)$ is unique up to homotopy.

Let $\cO$ be a homology orientation on $X$.
Precisely, the SW invariant is an invariant of triple: $SW(X,\fd,\cO)\in\Z$ with property that $SW(X,\fs,-\cO)=-SW(X,\fs,\cO)$.
Usually $\cO$ is dropped from our notation.

\subsection{Spin and Spin$^c$ sturctures}


\begin{lem}
\[\pi_1\SO(n)=\begin{cases}
\Z/2\Z&\text{ if }n\ge3,\\
\Z&\text{ if }n=2.
\end{cases}\]
\end{lem}
\begin{pf}
Exercise.
Use the fibration $\SO(n)\to\SO(n+1)\to S^n$.
\end{pf}

We define the spin group $\Spin(n)$ as the unique double cover of $\SO(n)$ for $n\ge3$.

\begin{defn}
Let $X$ be a oriented smooth manifold of dimension $n$ equipped with a metric $g$.
A \emph{spin structure} $\fs$ on $X$ is a pair $(P,\Psi)$, where $P$ is a principal $\Spin(n)$-bundle over $X$ and $\Psi$ is an isomorphism of principal $\SO(n)$-bundles
\[\Psi:P\times_{\Spin(n)}\SO(n)\xrightarrow{\sim}\mathrm{Fr}(X).\]
We should not forget $\Psi$, a compatibility with the data of tangent bundle.
In other words, a spin structure is a choice of a lift of the frame bundle map $X\to B\SO(n)$ along $B\Spin(n)\to B\SO(n)$. Of course it is up to homotopy because classifying spaces are defined up to homotopy equivalence, which corresponds to isomorphism classes of principal $\Spin(n)$-bundles.
\end{defn}

\begin{rmk}
To consider an isomorphism class of a spin($^c$) structure, one does not need to pick a metric, because $\GL^+(n,\R)$ is homotopy equivalent to $\SO(n)$.
However, a choice of a metric is needed to define the Dirac operator. 
\end{rmk}

\begin{rmk}
A 4-manifold may or may not admit a spin structure.
For examples, $S^2\times S^2$ and $K3$ admit spin structures, but $\CP^2$ does not.
\end{rmk}

We have another exercise.
\begin{lem}
Let $X^4$ be a closed oriented smooth 4-manifold.
If $X$ admits a spin structure, then $Q_X$ is even.
The converse holds if $\pi_1(X)=1$.
\end{lem}


\begin{defn}
\[\Spin^c(n):=\frac{\Spin(n)\times\rU(1)}{\Z/2\Z},\]
where $\Z/2\Z$ acts on $\Spin(n)$ as the covering transformation and on $\rU(1)$ faithfully as the scalar multiplication.
The alphabet $c$ stands for ``complex''.
We have a canonical surjection $\Spin^c(n)\to\SO(n)$.
\end{defn}

\begin{defn}
Let $X$ be a oriented smooth $n$-manifold equipped with a metric.
A \emph{spin$^c$ structure} $\fs$ on $X$ is a pair $(P,\Psi)$, where $P$ is a principal $\Spin^c(n)$-bundle and $\Psi$ is an isomorphism of principal $\SO(n)$-bundles
\[\Psi:P\times_{\Spin^c(n)}\SO(n)\xrightarrow{\sim}\mathrm{Fr}(X).\]
\end{defn}

There is a fact that every oriented 4-manifold admits a spin$^c$ structure.
(It is open that every topological 4-manifold has a CW complex structure.)
We usually do not use this fact because almost every argument begin with fixing a concretely given spin$^c$ structure.
Since we have a natural $\Spin(n)\to\Spin^c(n)$, a spin structure induces a spin$^c$ structure.


We have another exercise.
\begin{lem}
Let $X$ be an oriented smooth manifold.
If $X$ admits a spin structure, then there is a non-canonical bijection between the set of isomorphism classes of spin structures on $X$ and $H^1(X,\Z/2\Z)$.
If $X$ admits a spin$^c$ structure, then there is a non-canonical bijection between the set of isomorphism classes of spin$^c$ structures on $X$ and $H^2(X,\Z)$.
\end{lem}

If $\pi_1(X)=1$, then spin structure is unique up to isomorphism.

orientability iff $w_1=0$, spin iff $w_2=0$, spin$^c$ iff $w_2$ is lifted to $H^2(X,\Z)$.

\subsection{SW equations}
Let $X$ be an oriented smooth 4-manifold with a metric $g$ and a spin$^c$ structure $\fs$.
We can write down the SW equation
\[\left\{\begin{alignedat}{1}
F_A^+&=\sigma(\Phi)\\
D_A\Phi&=0
\end{alignedat}\right.\]
where $(A,\Phi)$ is unknown functions.
Here $A$ is a $\rU(1)$-connection on the determinant line bundle for $\fs$, and $\Phi$ is a positive spinor for $\fs$.
Roughly, $SW(X,\fs)$ is the signed count of
\[\frac{\#\{\text{solutions $(A,\Phi)$}\}}{\text{gauge symmetry}}.\]
It is typically an oriented compact 0-dimensional manifold.

$b^+\ge2$, the singularity with codimension 1?


\begin{lem}
\[\Spin(4)\cong\Sp(1)\times\Sp(1),\qquad\Sp(1)=S(\H).\]
\end{lem}
\begin{pf}
$\Sp(1)\times\Sp(1)\to\SO(\H):(q_+,q_-)\mapsto(q\mapsto q_-q_+)$ is a double cover.
\end{pf}

Since $\Sp(1)\cong\SU(2):a+jb\mapsto\mat{a&b\\-\bar b&\bar a}$ canonically, we have four representations of
\[\Spin^c(4)=\frac{\SU(2)\times\SU(2)\times\rU(1)}{\Z/2\Z}\]
given by
\[\Delta_+:\Spin^c(4)\to\rU(2):[q_+,q_-,\lambda]\mapsto\lambda q_+,\]
\[\Delta_-:\Spin^c(4)\to\rU(2):[q_+,q_-,\lambda]\mapsto\lambda q_-,\]
\[\det:\Spin^c(4)\to\rU(1):[q_+,q_-,\lambda]\mapsto\lambda^2.\]
These representations induces positive spinor bundle $S^+$, negative spincor bundle $S^-$, and the determinant line bundle $L$.

\end{document}