\documentclass{../../../small}
\usepackage{../../../ikhanchoi}

\DeclareMathOperator{\Gr}{Gr}

\begin{document}
\title{Fiber Bundles}
\author{Ikhan Choi\\Lectured by Takuya Sakasai\\University of Tokyo, Spring 2023}
\maketitle
\tableofcontents

\newpage
\section{Day 1: April 10}

References:
Steenrod, \emph{The topology of fiber bundles}, and Tamaki, \emph{Fiber bundles and homotopy} (Japanese)

\subsection*{1. Introduction}

An ultimate goal of topology is to classify topological spaces, up to homeomorphism.
If you want to show two spaces are homeomorphic, we should construct a homeomorphism: \emph{Shokuninwaza} (wild knot, Casson handle).
If you want to show two spaces are not homeomorphic, then we can investigate topological \emph{properties}, and as their quantitative comparison, we can investigate topological \emph{invariants}
Some examples include
\begin{itemize}
\item the number of connected componenets,
\item the Euler characteristic,
\item homology groups,
\item homotopy groups,
\item the minimal number of open contractible sets to cover the spaces (Lusternik-Schnirelmann category, topological complexity),
\item Gelfand-Naimark theorem: $C(X)\cong C(Y)$ implies $X\cong Y$ if they are compact Hausdorff.
\end{itemize}


We will restrict objects to study.
For example, metric spaces, manifolds, CW-complexes.
As the assumptions change, invariants may have different appearances.
For a manifold $X$,
\[\chi(X)=\sum_{q=0}^\infty(-1)^q\rk_\Z H_q(X)=\sum_{q=0}^\infty(-1)^qb_q(X).\]
For a CW-complex $X$,
\[\chi(X)=\sum_{q=0}^\infty(-1)^q(\text{the number of $q$-cells}).\]


\smallskip\hrule\bigskip

Let $M$ be an connected closed $n$-dimensional manifold.
Some classification results are as follows(up to both homeomorphisms and diffeomorphisms, because $d\le2$):
\begin{itemize}
\item ($n=0$) $M\cong *$, and $\chi(*)=1$.
\item ($n=1$) $M\cong S^1$, and $\chi(S^1)=0$.
\item ($n=2$)
	\begin{itemize}
	\item If $M$ is orientable, then $M\cong\Sigma_g$ for $g\ge0$, and $\chi(\Sigma_g)=2-2g$.\\
	$\Sigma_0\cong S^2$, $\Sigma_1\cong T^2$.
	\item If $M$ is not orientable, then $M\cong(\RP^2)^{\# h}$ for $h\ge1$, and $\chi((\RP^2)^{\#h})=2-h$.\\
	$\RP^2(\cong\text{M\"obius strip}\cup D^2)$, $K=\RP^2\#\RP^2$
	\end{itemize}
\end{itemize}

\textbf{Problem 1.} Show $\RP^2\#T^2\cong\RP^2\#K$.
\bigskip

Here are some facts about triangulability:
\begin{itemize}
\item Cairns(1935), Whitehead (1940): every $C^1$-manifold is triangulable (unique as a PL-manifold).
\item Rado(1925, $n=2$), Moise(1952, $n=3$): for $n\le3$, every $C^0$-manifold is triangulable (unique as a PL-manifold).
\item Kirby-Siebermann(1966, $n\ge5$): for $n\ge4$, there is a non-triangulable PL-manifold.
\item Donaldson, Freedman, Casson: for $n=4$, there is a non-triangulable manifold as a topological space.
\item Manolescu(2013): for $n\ge5$, there is a non-triangulable manifold as a topological space.
\end{itemize}

Orientability?
For a connected closed surface $S$, it is orientable iff $H_2(S)\cong\Z$, not orientable iff $H_2(S)\cong0$. The generator of $H_2(S)$ is called the fundamental class.
Orientability asks if the tubular neighborhood of every simple closed curve is homeomorphic to an anulus.
It is described by the first Stiefel-Whitney class:
\[w_1(S)\in H^1(S;\Z/2\Z)\cong\Hom(H^1(S),\Z/2\Z)\cong\Hom(\pi_1(S),\Z/2\Z).\]

\smallskip\hrule

\subsection*{Euler characteristic of manifolds}

\subsubsection*{(0) Odd-dimensional manifolds}

\begin{thm*}
For an odd-dimensional closed connected manifold, $\chi(M^{2n+1})=0$.
\end{thm*}
\begin{pf}
If orientable, then $b_0(M)=1$, $b_3(M)=1$, $b_1(M)=b_2(M)$ by the Poincar\'e duality.
If not, a double cover is orientable, and $\chi(\tilde M)=2\chi(M)$.
\end{pf}


\subsubsection*{(1) Gauss-Bonnet theorem}
\begin{thm*}[Gauss-Bonnet]
If a smooth manifold $M^n$ embeds into $\R^{n+1}$ (hypersurface), then it is orientable and the Euler characteristic is given by
\[\chi(M)=\frac2{\vol(S^n)}\int_MK\,d\vol_M.\]
\end{thm*}





\section{Day 2: April 17}

We have a cohomological interpretation.
In the Chern-Weil theory, we have a generalized version of the Gauss-Bonnet theorem for a general compact manifold using the theory of connections.
We can interpret $2\vol(S^n)^{-1}K\cdot d\vol_M$ as a differential form which provides with the Euler characteristic.
In the context of the de Rham theorem, we will eventually call the equivalence class of this differential form as the \emph{Euler class}.

\subsubsection*{(2) Poincar\'e-Hopf theorem}

Let $M^n$ be a orientable connected smooth closed manifold.
Let $X$ be a smooth vector field on $M$ such that there are only finitely many zeros $\{p_1,\cdots,p_m\}$.
For each $p_j$, define the index $\Ind(X,p_j)$ as follows: 
seeing $X$ as a vector field on $\f_j(U_j)$ for a chart $(U_j,\f_j)$ not containing zeros of $X$ but $p_j$ and mapping $p_j$ to zero in $\R^n$, we define $\Ind(X,p_j)=\deg f_j$, where $f_j:S_\e(\approx S^{n-1})\to S^{n-1}:x\mapsto X_x/\|X_x\|$.

\begin{ex*}
Let $n=2$.
We have indices $1, 1, 1, -1, 0, 2$ for
\[X_1(x,y)=(x,y),\quad X_2(x,y)=(-x,-y),\quad X_3(x,y)=(-y,x),\]
\[X_4(x,y)=(-x,y),\quad X_5(x,y)=\sqrt{x^2+y^2}(1,1),\quad X_6(x,y)=(x^2-y^2,2xy).\]
\end{ex*}

\begin{thm*}[Poincar\'e-Hopf]
\[\sum_{j=1}^m\Ind(X,p_j)=\chi(M).\]
\end{thm*}

We have a cohomological interpretation.
Let $c=\sum_{j=1}^m\Ind(X,p_j)p_j$ be a singular 0-cycle on $M$.
Then, the Poincar\'e-Hopf theorem states that we have
\[\begin{array}{ccc}
H_0(M)&\xrightarrow{\sim}&\Z\\
p_j&\mapsto&1\\
c&\mapsto&\chi(M).
\end{array}\]
By the Poincar\'e duality, we can identify the homology class $[c]$ with a de Rham cohomology class, and the above map is just an integration map.

The cycle $c$ tells us the information of intersections of $X$ and zero section(of the tangent bundle).
If $TM$ is trivial, then the zero section does not self-intersection(?) so that $c=0$.
The Euler characteristic measures the twist of a bundle, and the characteristic class generalizes this wakugumi.


\subsection*{2. Fiber bundles}

From now we will only consider paracompact Hausdorff spaces.
Recall that a space is paracompact iff for every open cover there is a locally finite refinement.
\begin{ex*}
Open sets of $\R^n$, metric spaces, CW-complexes, countable inductive limit of compact spaces are paracompact.
\end{ex*}
\begin{thm}
For any open cover of a paracompact Hausdorff space $X$, there is a partition of unity subordinate to it.
\end{thm}

\textbf{Problem 2.} Prove the above theorem.

\begin{defn}
Let $B$ be connected(for simplicity).
A map $E\to B$ is called a fiber bundle with fiber $F$, or just a $F$-bundle, if it is locally trivial: every point $x\in B$ has an open neighborhood $U_x$ such that there is a homeomorphism $\f:p^{-1}(U_x)\to U_x\times F$ with $p=\pr_{U_x}\circ\f$.

For each $y\in B$ $E_y:=p^{-1}(y)$ is homeomorphic to $F$, and is called the fiber at $y$.
Also, $E$ and $B$ are called the total space and the base space.
We somtimes write as $\xi=(F\to E\xrightarrow{p}B)$.
\end{defn}
\begin{ex*}\,
\begin{parts}
\item We say $\pr_1:B\times F\to B$ is the product or bundle.
\item $p:\R\to S^1=\R/\Z:t\mapsto[t]$ is a $\Z$-bundle. In general, a fiber bundle with a discrete fiber is called a covering space.
\item $p_1:S^n\to\RP^n=S^n/(x\sim-x)$ is a $\Z/2\Z$-bundle.
\item $p:S^{2n+1}\to\CP^n=S^{2n+1}/(z\sim uz)$ for $u\in S^1$ is a $S^1$-bundle. (a generalization of Hopf bundles)
\item Let $M^n$ be a smooth manifold. Then, the tangent and the contangent bundles are $\R^n$-bundles.
\end{parts}
\end{ex*}

\textbf{Problem 3.} Show that $p:S^{2n+1}\to\CP^n$ is a $S^1$-bundle by checking concretely its local triviality.

\begin{defn}
If $F,E,B$ are $C^r$, $p:E\to B$ is $C^r$, and the local trivialzation is $C^r$, then we say the fiber bundle is $C^r$.
\end{defn}
\begin{defn}
For $\xi_1=(F\to E_1\xrightarrow{p_1}B_1)$, $\xi_2=(F\to E_2\xrightarrow{p_2}B_2)$, a bundle map $\Phi=(\tilde f,f):\xi_1\to\xi_2$ is a pair of maps $\tilde f:E_1\to E_2$ and $f:B_1\to B_2$ such that $f\circ p_1=p_2\tilde f$ and the restriction $\tilde f:p_1^{-1}(b)\to p_2^{-1}(f(b))$ is a homeomorphism for every $b\in B$.

If both $f$ and $\tilde f$ are homeomorphisms, then $\Phi$ is called a bundle isomorphism.
If a bundle is isomorphic to a product bundle, then it is called to be trivial.
\end{defn}

\textbf{Problem 4} For a bundle map $\Phi$, is $\tilde f$ homeomorphic if $f$ is homeomorphic? (If we are doing in the category of smooth manifolds, then the inverse function theorem may be helpful.)


\newpage
\section{Day 3: April 24}


\subsubsection*{Transition maps and structure groups}
Let $\xi=(F\to E\xrightarrow{p}B)$ be an $F$-bundle.
We have an open cover $\{U_\alpha\}$ such that for each $\alpha$ we have a local trivialization $p^{-1}(U_\alpha)\xrightarrow{\sim}U_\alpha\times F$.
For $U_\alpha\cap U_\beta\ne\varnothing$, we have a map
\[\f_\alpha\circ\f_\beta^{-1}:(U_\alpha\cap U_\beta)\times F\to(U_\alpha\cap U_\beta)\times F,\]
by which we can define $\tilde g_{\alpha\beta}:(U_\alpha\cap U_\beta)\times F\to F$ such that $\f_\alpha\circ\f_\beta^{-1}(b,f)=:(b,\tilde g_{\alpha\beta}(b,f))$.
The map $\tilde g_{\alpha\beta}$ is continuous, and we have for each $b$ a homeomorphism
\[g_{\alpha\beta}(b):F\to F:f\mapsto\tilde g(b,f),\]
that is, $g_{\alpha\beta}:U_\alpha\cap U_\beta\to\Homeo(F)$.
If we endow the compact-open topology on $\Homeo(F)$, then $g_{\alpha\beta}$ is continuous.

From definition, $g_{\alpha\beta}(b)\circ g_{\beta\alpha}(b)=\id_F$ for $b\in U_\alpha\cap U_\beta\ne\varnothing$, and $g_{\alpha\beta}(b)\circ g_{\beta\gamma}(b)=g_{\alpha\gamma}(b)$ for $b\in U_\alpha\cap U_\beta\cap U_\gamma\ne\varnothing$
(Note that the second relation implies the first.).
The second condition is called the cocycle condition.
The maps $\{g_{\alpha\beta}\}$ are called transition maps.

\setcounter{section}{2}
\setcounter{thm}{4}
\begin{thm}
Let $\{U_\alpha\}$ be an open cover of a connected space $B$.
Suppose we have a collection of continuous maps
\[\{g_{\alpha\beta}:U_\alpha\cap U_\beta\to\Homeo(F)\}_{(\alpha,\beta):U_\alpha\cap U_\beta\ne\varnothing}\] satisfying the cocycle condition.

($\spadesuit$) Suppose also that $F$ is locally compact, or there exists a topological transformation group $G$(i.e. $G$ is a topological group such that the group action $G\times F\to F$ is continuous) with
\[\bigcup_{\alpha,\beta}g_{\alpha\beta}(U_\alpha\cap U_\beta)\subset G\subset\Homeo(F).\]

Then, there exists a unique $F$- bundle $(F\to E\xrightarrow{p}B$ such that it is locally trivializable over $\{U_\alpha\}$ and $\{g_{\alpha\beta}\}$ is the transition maps of the bundle.
\end{thm}

The viewpoint of the above theorem is more likely to be the physicist's way of defining manifolds in the sense that they sometimes deifne a manifold as a collection of open subsets of a Euclidean space and transition maps between them.

The condition ($\spadesuit$) gaurantees for the second map in
\[\tilde g_{\alpha\beta}:(U_\alpha\cap U_\beta)\times F\to(U_\alpha\cap U_\beta)\times\Homeo(F)\times F\to(U_\alpha\cap U_\beta)\times F\]
\[(b,f)\mapsto(b,g_{\alpha\beta}(b),f)\mapsto(b,g_{\alpha\beta}(f))\]
to be continuous.

\begin{pf}(Sketch)
Define
\[\tilde E:=\coprod U_\alpha\times F\]
and $E:=\tilde E/\sim$, where the equivalence relation $\sim$ is generated by: for each $(b_1,f_1)\in U_\alpha\times F$ and $(b_2,f_2)\in U_\beta\times F$ we have $(b_1,f_1)\sim(b_2,f_2)$ iff $b_1=b_2$ and $f_1=g_{\alpha\beta}(b_2)(f_2)$.
Let $\pi:\tilde E\to E$ be the canonical projection.
Define also
\[\f_\alpha:p^{-1}(U_\alpha)\to U_\alpha\times F:[(b,f)\in U_\alpha,F]\mapsto (b,f),\] which are homeomorphisms by the assumption ($\spadesuit$), satisfying $\pr_1\circ\f_\alpha=p$.
\end{pf}

For the second condition in (\spadesuit), $G$ is called a structure group of the $F$-bundle.
From now on, whenever we consider a fiber bundle along with a structure group $G$, we assume it includes the data of local trivialization.

\begin{rmk*}
We will always think of $G$ for bundle maps between fiber bundles with structure group $G$.
We will frequently consider the maximal transition data and compatible(i.e. satisfying the cocycle condition) local trivializations.
\end{rmk*}
\begin{ex*}\,
\begin{enumerate}
\item Let $F=V\cong\R^n$ be a real vector space, and $G\in\{\GL(V),\SL(V)\}$ or $G\in\{\rO(V),\SO(V)\}$ with a fixed inner product on $V$. These fiber bundles are called real vector bundles.
\item Let $F=V\cong\C^n$ be a complex vector space, and $G\in\{\GL_\C(V)\}$ or $G\in\{\rU(V)\}$ with a fixed inner product on $V$. These fiber bundles are called complex vector bundles.
\item $F=G$ be a Lie group. Then, $G$-bundle with structure group $G$ is called a principal bundle.
\item Let $F$ be a nice smooth manifold and $G=\Diff^{C^\infty}(F)$ be the group of smooth diffeomorphisms together with the Fr\'echet topology. Then, we have smooth $F$-bundles.
\end{enumerate}
\end{ex*}

\begin{defn}
Let $G$ be a structure group and $B$ be a topological space.
If an $F$-bundle $\xi=(F\to E\to B,G)$ and an $F'$-bundle $\xi=(F'\to E'\to B,G)$ has the same transition data, then they are called associated bundles.
\end{defn}


\begin{ex*}
Let $F=\R^n$ be a real vector space with the standard inner product.
Let $G=O(n)$.
With $S^{n-1}\subset F$, the sphere bundle inside a real vector bundle is an associated bundle of the original real vector bundle.
In particular for $n=2$ and $G=\SO(2)$, then the circle bundle can be recognized as a principal $SO(2)$-bundle associated to a real plane bundle, and if we see the plane bundle as a complex line bundle, then it corresponds to a pricipal $U(1)$-bundle.
\end{ex*}

\begin{prop}
Let $G$ be a topological group and $\xi=(G\to E\to B,G)$ be a principal $G$-bundle.
Then, there is a natural right action of $G$ on $E$ which is free and the orbit space $E/G$ is homeomorphic to $B$(transitively act on each fiber).
\end{prop}
\begin{pf}
Let $u\in E$ and $\f_\alpha$ a local trivialization containing $u$ such that
\[\f_\alpha:p^{-1}(U_\alpha)\to U_\alpha\times G:u\mapsto(p(u),h).\]
We can check the well-definedness of $ug=\f_\alpha^{-1}(p(u),hg)$ by
\[\f_\beta(ug)=\f_\beta\circ\f_\alpha^{-1}(p(u),hg)=(p(u),g_{\beta\alpha}(p(u))(hg))=(p(u),h'g).\]

The right action of $G$ on $G$ is continuous, free, and transitive.
The right action of $G$ on $E$ is continuous and free, and $\bar p:E/G\to B$ is continuous and bijective.
\end{pf}

\textbf{Problem 5.} Show that $\bar{p}^{-1}$ is also continuous.

\begin{rmk*}
A principal $G$-bundle may also be defined as follows: a $G$-bundle such that (1) there is a continuous free right action of $G$ on $E$ which is (2) fiber-preserving and fiberwise transitive, and (3) we can choose \emph{$G$-equivariant} local trivialization such that $\f_\alpha(u)=(p(u),h)$ implies $\f_\alpha(ug)=(p(u),hg)$.
\end{rmk*}

\newpage
\setcounter{section}{3}
\section{Day 4: May 1}
\setcounter{section}{2}
\setcounter{thm}{7}

Let $G$ be a topological group.
A pricipal $G$-bundle $(G\to E\to B,G)$ has a continuou free action of $G$ on $E$.

\begin{rmk*}
For two principal $G$-bundles, $(\tilde f,f)$ is a bundle map if and only if $\tilde f$ is $G$-equivariant.
\end{rmk*}

\begin{defn}
Let $\xi=(F\to E\xrightarrow{p}B)$ be a fiber bundle.
A continuous map $s:B\to E$ such that $p\circ s=\id_B$ is called a section or a cross section.

An important question asks if there is a section globally defined on the whole $B$.
\end{defn}

\begin{prop}
Let $\xi=(G\to E\to B,G)$ be a principal $G$-bundle.
Then, $\xi$ is trivial if and only if it admits a global section.
\end{prop}

\begin{pf}
($\Rightarrow$)
Clear.

($\Leftarrow$)
Let $s:B\to E$ be a global section.
Define
\[\Phi:B\times G\to E:(b,g)\mapsto s(b)g.\]
Then, it is an $G$-equivariant isomorphism.
\end{pf}


Let $X$ be a right $G$-space which is free.
Then, is $X/G$ a principal $G$ bundle?
We have two problems:
\begin{parts}
\item Is the inverse image($=$orbit) of each point of $X/G$ homeomorphic to $G$?
No, the dynamics $\T^2\curvearrowright\R$ with irrational slope.
\item Does it satisfy the local triviality?
No, the translation $\R\leftarrow\Q$.
\end{parts}

\begin{prop}
Let $X$ be a right $G$-space which is free.
The quotient map $\pi:X\to X/G$ defines a principal $G$-bundle if and only if $X\curvearrowright G$ strongly freely(i.e. $X\times X\to G:(x,xg)\mapsto g$ is continuous) and there is a local section for some $y\in X/G$.
\end{prop}
\begin{pf}
($\Rightarrow$) Clear.

($\Leftarrow$)
\[\pi^{-1}(U)\to U\times G:s(x)g\mapsto(x,g)\]
is continuous by the strongly free action.
It defines local trivializations.
\end{pf}

\begin{thm}[Gleason, 1950]
Let $M$ be a smooth manifold and $G$ a compact Lie group which gives a free right smooth action on $M$.
Then, $M/G$ is a smooth manifold such that $M\to M/G$ is a principal $G$-bundle.
\end{thm}
(Compactness of $G$ implies the properness of the action, and smoothness implies the local triviality)
\begin{cor}[Samelson, 1941]
Let $H$ be a compact Lie subgroup of a Lie group $G$.
Then, $G\to G/H$ is a principal $H$-bundle.
In fact, it is sufficient for $H$ to be a closed subgroup of $G$, even if it is not compact.
\end{cor}
\begin{ex*}\,
\begin{parts}
\item
With an action $S^{2n+1}\curvearrowright S^1$ such that $(z_0,\cdots,z_n)w=(z_1w,\cdots,z_nw)$, we have an $S^1$-bundle
\[S^{2n+1}\to\CP^n:(z_0,\cdots,z_n)\mapsto[z_0:\cdots:z_n].\]
It is a general Hopf bundle.
\item
For $k\le n$, the Stiefel variety is
\[V_k(\R^n):=\{M\in M_{n,k}(\R):\rk M=k\}.\]
Also define
\[V_k^O(\R^n):=\{M\in V_k(\R^n):\text{column vectors of $M$ are orthonormal}\}\]
and the Grassmannian manifold
\[G_k(\R^n):=\{\text{$k$-dimensional subspaces of $\R^n$}\}.\]
Stiefel varieties can be realized as principal bundles on Grassmannian manifolds.

With an action $V_k(\R^n)\curvearrowright\GL(k,\R)$ such that $(v_1,\cdots,v_k)X=(v_1X,\cdots,v_kX)$, we have $G_k(\R^n)\cong V_k(\R^n)/\GL(k,\R)$ and $G_k(\R^n)\cong V_k^O(\R^n)/\rO(k)$.
Then, $(\rO(k)\to V_k^O(\R^n)\to G_k(\R^n))$ and $(\GL(k,\R)\to V_k(\R^n)\to G_k(\R^n))$ are principal bundles.
\item
As a complex version of (b), we have principal bundles $(\rU(k)\to V_k^U(\C^n)\to G_k(\C^n))$ and $(\GL(k,\C)\to V_k(\C^n)\to G_k(\C^n))$.
\end{parts}
\end{ex*}


\begin{thm}
Let $M$ be smooth manifold and suppose we have a transitive smooth left action of a Lie group $G$ on $M$.
Let $H$ be \emph{the} isotropy group.
Then, $G/H\to M$ defines a diffeomorphism and $(H\to G\to M)$ is a principal bundle.
Such $M$ is called a homogeneous space.
\end{thm}

\begin{ex*}
With an action $\SO(n)\curvearrowleft S^{n-1}$, since the isotropy group is isomorphic to $\SO(n-1)$, we have a principal bundle $SO(n-1)\to SO(n)\to S^n$.

We can also see the examples above(Grassmann and Steifel manifolds) as principal bundles on homogeneous spaces with a diffeomorphsim $\rO(n-k)\setminus\rO(n)\to V_k^O(\R^n):[A]\mapsto(Ae_1,\cdots,Ae_k)$ and $\rO(n)/\rO(n-k)\times\rO(k)\cong G_k(\R^n)$: principal $\rO(k)$-bundle
\begin{cd}
\rO(k)\ar{r} & V_k^O(\R^n) \ar{d}\ar{r}{\sim}& \rO(n)/\rO(n-k)\\
& G_k(\R^n) \ar{r}{\sim} & (\rO(n)/\rO(n-k))/\rO(k).
\end{cd}
We also have a complex version.


\end{ex*}



\newpage
\setcounter{section}{4}
\section{Day 5: May 8}


\subsubsection*{Principal bundles and associated bundles}

Let $G$ be a topological group and $\xi=(G\to E\xrightarrow{p}B,G)$ be a principal $G$-bundle.
Let $\{U_\alpha\}$ be an open cover of $B$.
Let $G$ effectively act on $F$ from left as a transformation group, i.e. there is an injective group homomorphism $\sigma:G\to\Homeo(F)$ such that the action $G\times F\to F$ is continuous.
Define
\[E\times_GF:=E\times F/(eh,f)\sim(e,\sigma(h)f)\]
and
\[\pi:E\times_GF\to B:[e,f]\mapsto p(e).\]
This map is well-defined and continuous so that $\eta=(F\to E\times_GF\xrightarrow{\pi}B,G)$ is a fiber bundle with structure group $G$ and fiber $F$.

In fact, if $\{g_{\alpha\beta}\}$ is the transition maps of $\xi$, then the transition maps of $\eta$ are given by $\{\sigma\circ g_{\alpha\beta}\}$.

Conversely, let $\tilde\eta=(F\to\tilde E\to B,G)$ be a fiber bundle with structure group $G$ and fiber $F$.
If we construct principal $G$-bundle $\xi$ with the transition data $\{g_{\alpha\beta}\}$ of $\tilde\eta$, then $\eta$ and $\tilde\eta$ are isomorphic.

\begin{rmk*}
If $\sigma:G\to\Homeo(F)$ is not injective, then $\eta=(F\to E\times_GF\to B$ is a $G/\ker\sigma$-bundle with fiber $F$.
It can be seen as a generalized version of assoicated bundles.
\end{rmk*}
\begin{ex*}
Let $M^n$ be a smooth manifold and $p:TM\to M$ be the tangent bundle with structure group $\GL(n,\R)$.
For each $x\in M$, consider
\[F_x:=\{[v_1,\cdots,v_n]:\text{ordered bases of }T_xM\}\]
and $FM:=\bigcup_{x\in M}F_x\curvearrowright\GL(n,\R)$.
We call $FM\to M$ the tangent frame bundle.
\end{ex*}

\begin{thm*}[2.14]
\[\begin{array}{ccc}
\left\{\begin{tabular}{c}isomorphism classes of\\real vector bundles of rank $n$ on $B$\end{tabular}\right\}
&\xrightarrow{\sim}&
\left\{\begin{tabular}{c}isomorphism classes of\\principal $\GL(n,\R)$-bundles on $B$\end{tabular}\right\}.
\end{array}\]
\end{thm*}
\begin{pf}
Transition maps.
\end{pf}

\begin{ex*}
The tautological vector bundle $\gamma_k$ is defined as $\R^k\to E_k\xrightarrow{\pr_1}G_k(\R^n)$, where
\[E_k:=\{(W,p)\in G_k(\R^n)\times\R^n:p\in W\}.\]
This is the vector bundle associated to the canonical principal $\GL(n,\R)$-bundle on $G_k(\R^n)$.
\end{ex*}

\subsubsection*{Reduction of structure groups}
\begin{defn*}[2.15]
Let $H$ be a closed subgroup $G$.
We say the structure group of a $G$-bundle $\xi$ with fiber $F$ can be reduced to $H$ if $\xi$ is isomorphic to a $H$-bundle with fiber $F$.
In other words, we have a collection of $H$-valued transition maps on an appropriately taken open cover on the base space.
\end{defn*}

\begin{ex*}\,
\begin{parts}
\item Let $H:=\Homeo^+(F)\subset G:=\Homeo(F)$.
A bundle with fiber $F$ is orientable if and only if the structure group can be reduced to $H$.
\item Let $H:=\rO(n)\subset G:=\GL(n,\R)$.
A vector bundle of rank $n$ has a Euclidean metric(it is a Riemannian metric if smooth) if and only if the structure group of the associated principal $G$-bundle can be reduced to $H$.

($\Rightarrow$)
Suppose a vector bundle $\xi$ has a collection of $\rO(n)$-valued transition maps on a sufficiently refiend open cover, and the local trivialization is written by $\f_\alpha:p^{-1}(U_\alpha)\to U_\alpha\times\R^n$.
Then, for $x,y\in E_b$ and $b\in U_\alpha\subset B$, the symmetric bilinear form
\[(x,y)_b:=(\pr_2\circ\f_\alpha(x),\pr_2\circ\f_\alpha(y))_{\R^n}\]
is a well-defined inner product.

($\Leftarrow$)
Suppose a Euclidean metric on a vector bundle $\xi$ of rank $n$ is given.
Since $p^{-1}(U_\alpha)\to U_\alpha$ is trivial, we can take sections $(s_i)_{i=1}^n$ on $U_\alpha$ which are linearly independent at each point of $U_\alpha$.
Using the given Euclidean metric, we can apply the Gram-Schmidt algorithm to get another set of sections $(e_i)_{i=1}^n$ which form an orthonormal basis at each point of $U_\alpha$.
With these sections we can construct new local trivializations, having $\rO(n)$-valued transition maps.

(Another remark)
Since every vector bundle over a paracompact space $B$ admits a Euclidean metric, the structure group of every principal $\GL(n,\R)$-bundle can be reduced to $\rO(n)$.

\item
For a complex version, a complex vector bundle of rank $n$ admits a Hermitian metric if and only if the structure group $\GL(n,\C)$ can be reduced to $\rU(n)$.
Similarly, the reduction is always possible if $B$ is paracompact.
\end{parts}
\end{ex*}


\newpage
\section{Day 6: May 15}

\subsection*{3. Classification of principal bundles}

\subsubsection*{Pullback bundles}

Let $\xi=(F\to E\to B)$ and $f:A\to B$ a map.
Then, the pullback bundle of $\xi$ by $f$ is $(F\to f^*E\to A)$, where $f^*E:=\{(a,e)\in A\times F:f(a)=p(e)\}$.
For example, if $\iota:A\hookrightarrow B$ is an inclusion, then the restriction is defined by the pullback $\xi|_A:=\iota^*\xi=(F\to p^{-1}(A)\to A)$.
For $i:\mathrm{Gt}_k(\R^n)\hookrightarrow\mathrm{Gr}_k(\R^n)$ with $k<n'<n$, if we define $E_{n,k}:=\{(W,p)\in\mathrm{Gr}_k(\R^n)\times\R^n\mid p\in W\}$, then $i^*E_{n,k}=E_{n',k}$.
If $\eta$ is a pricnipal $G$-bundle and $F$ is a left $G$-space, then $f^*(\eta\times_G F)=f^*\eta\times_GF$.

\begin{lem*}[3.1]
Let $\xi=(F\to E_1\xrightarrow{p_1}B_1,G)$ and $\eta=(F\to E_2\xrightarrow{p_2}B_2,G)$.
For a map $f:B_1\to B_2$, then a bundle map $(\tilde f,f):\xi\to\eta$ exists if and only if $\xi\cong f^*\eta$ over $B_1$.
\end{lem*}
\begin{pf}
($\Rightarrow$)\[E_1\to f^*E:x\mapsto(p_1(x),\tilde f(x)).\]

($\Leftarrow$)
\begin{cd}[column sep=5pt]
E_1\ar{rr}\ar{dr}&&f^*E_2\ar{dl}\ar{rr}&\qquad&E_2\ar{d}\\
&B_1\ar{rrr}{f}&&&B_2
\end{cd}
\end{pf}
\begin{exe*}[6]\,
\begin{enumerate}[(1)]
\item Check that $f^*\eta$ over $B_1$ is an $F$-bundle which have $G$ as the structure group.
\item Fill the gap of the above proof of Lemma 3.1.
\end{enumerate}
\end{exe*}
\begin{thm*}[3.2]
Let $\xi=(F\to E\to B,G)$ be a bundle and $A$ be a paracompact space.
If there is a homotopy $F:A\times I\to B:(x,t)\mapsto f_t(x)$, then $f_0^*\xi\cong f_1^*\xi$ over $B_1$.
\end{thm*}
\begin{pf}
Consider the maps $\e_i:A\to A\times\{i\}$ for $i\in\{0,1\}$.
Since $\e_1=r\circ \e_0$, we have
\[\e_1^*\xi=(r\e_0)^*\xi=r^*\e_0^*\xi=\e_0^*\xi,\]
by Step 3(see below).
\end{pf}
\begin{cor*}
Let $A,B$ be paracompact homotopy equivalent spaces.
Then, there is a one-to-one correspondence between the sets of equivalence classes of bundles with fiber $F$ and structure group $G$ over $A$ and $B$ respectively.
\end{cor*}

In the following lemmas, let $\xi=(F\to E\to A\times I,G)$ and $A$ a paracompact space.
\begin{lem*}[Step 1 of 3.2]
If $\xi|_{A\times[a,b]}$ and $\xi|_{A\times[b,c]}$ are trivial, then $\xi|_{A\times[a,c]}$ is also trivial, where $0\le a<b<c\le1$.
\end{lem*}
\begin{pf}
Let $\f_1:E_{A\times[a,b]}\to(A\times[a,b])\times F$ and $\f_2:E_{A\times[b,c]}\to(A\times[b,c])\times F$ be trivializations.
The transition $g_{21}:A\to G\subset\Homeo(F)$ is continuous, where
\[(A\times\{b\})\times F\xrightarrow{\f_1^{-1}}E|_{A\times\{b\}}\xrightarrow{\f_2}(A\times\{b\})\times F\]
maps $((x,b),v)$ to $((x,b),g_{21}(x)(v))$.
Then, the map $w:(A\times[a,c])\times F\to E|_{A\times[a,c]}$ given by
\[w((x,t),v):=\begin{cases}\f_1^{-1}((x,t),v)&,t\le b\\\f_2^{-1}((x,t),g_{21}(x)(v))&,t\ge b\end{cases}\]
is well-defined, hence a bundle isomorphism.
\end{pf}

\begin{lem*}[Step 2 of 3.2]
Each point $a\in A$ has an open neighborhood $U$ of $a$ such that $\xi|_{U\times I}$ is trivial.
\end{lem*}
\begin{pf}
If we cover $\{a\}\times I$ with $U_t\times(a_t,b_t)$ for each $t\in I$, then by compactness we can define an open $U:=\bigcap U_t$ by choosing finite numbers of $t$.
By Step 1, $\xi$ is trivial on $U\times I$.
\end{pf}

\begin{lem*}[Step 3 of 3.2]
Suppose $A$ is paracompact.
If we define $r:A\times I\to A\times I$ by $r(a,t)=(a,1)$, then $\xi\cong r^*\xi$.
\end{lem*}
\begin{pf}
Step 2 implies that we have an open cover $\{V_\alpha\times I\}_\alpha$ of $A\times I$ such that $\xi$ is trivial on $V_\alpha\times I$.
Since $A$ is paracompact, we may assume $V_\alpha$ is locally finite.
Take a partition of unity $w_\alpha$ which subordinates to $V_\alpha$.
Define
\[\cU_\alpha(x):=\frac{w_\alpha(x)}{\max_\gamma w_\gamma(x)}.\]

Let $\f_\alpha:p^{-1}(V_\alpha\times I)\to(V_\alpha\times I)\times F$ be a local trivialization.
Define $(\tilde f_\alpha,f_\alpha):\xi|_{V_\alpha\times I}\to\xi|_{V_\alpha\times I}$ as follows:
$f_\alpha:V_\alpha\times I\to V_\alpha\times I$ and $\tilde f_\alpha:p^{-1}(V_\alpha\times I)\to p^{-1}(V_\alpha\times I)$ satisfy
\[f_\alpha(a,t):=(a,\max(\cU_\alpha(a),t)),\qquad \tilde f_\alpha(\f_\alpha^{-1}((a,t),g)):=\f_\alpha^{-1}(a,\max\{\cU_\alpha(a),t\},g).\]
Since $\supp\cU_\alpha\subset V_\alpha$, $\tilde f_\alpha$ is well-defined.

Assign any linear order on the index set $S$ containing $\alpha$.
With $(\tilde f_\alpha,f_\alpha)$, define $(\tilde f,f):\xi\to\xi$ as follows.
For each $a\in A$, there is an open neighborhood $V(a)$ such that $V(\alpha)\cap V_\alpha\ne\varnothing$ for at most finite number of $\alpha$ because $V_\alpha$ is locally finite.
Give the indices $\alpha$ their names $\alpha(1)<\cdots<\alpha(n)$ in order.
Then, $r:A\times I\to A\times I$ has the equality
\[r(a,t)=f_{\alpha(n)}\circ\cdots\circ f_{\alpha(1)}(a,t).\]
Similarly we can define $\tilde r:E\to E$ such that
\[\tilde r(x):=\tilde f_{\alpha(n)}\circ\cdots\circ\tilde f_{\alpha(1)}(x).\]
By Lemma 3.1, we have $\xi\cong r^*\xi$.
\end{pf}


\begin{thm*}[3.4]
Let $G$ be a topological group.
Then, there is a principal $G$-bundle $\xi_G:=(G\to EG\to BG,G)$ such that there is a one-to-one correspondence between $[B,BG]$ and the equivalence classes of principal $G$-bundles, via $[f]\mapsto[f^*\xi_G]$.
The space $BG$ is called the \emph{classifying space} and $\xi_G$ is called the \emph{universal principal $G$-bundle}.
\end{thm*}


\newpage
\section{Day 7: May 22}

\subsection*{Milnor's construction}

Let $EG$ be the countable \emph{join}, i.e.
\begin{align*}
EG=G\circ\cdots\circ G:=\Bigl\{&((t_0,t_1,\cdots),(g_0,g_1,\cdots))\in[0,1]^\infty\times G^\infty:\\
&\quad t_i=0\text{ but finitely many },\sum_{i=0}^\infty t_i=1\Bigr\}/\sim,
\end{align*}
where the equivalence relation is given by $((t_0,t_1,\cdots),(g_0,g_1,\cdots))\sim((t'_0,t'_1,\cdots),(g'_0,g'_1,\cdots))$ if and only if $t_n=t'_n$ for all $n$ and $g_n=g'_n$ for all $n$ satisfying $t_n=t'_n\ne0$.
We will write the element $((t_0,t_1,\cdots),(g_0,g_1,\cdots))$ as $(t_0g_0,t_1g_1,\cdots)$.
Endow the coarsest topology on $EG$ such that the maps $t_i:EG\to[0,1]$ and $g_i:t_i^{-1}(0,1]\to G$ are continuous.
Consider the right action $EG\curvearrowright G$ given by
\[(t_0g_0,t_1g_1,\cdots)\cdot g:=(t_0g_0g,t_1g_1g,\cdots).\]
As the orbit space we define $BG:=EG/G$ with the quotient topology.

\begin{lem*}[3.5]
$p:EG\to BG$ is a principal $G$-bundle.
\end{lem*}
\begin{pf}
We will check the conditions in Proposition 2.10.
\begin{itemize}
\item We can see the action $EG\times G\to EG$ is continuous and free.
\item We can directly check $EG\times EG\to G:(x,xg)\mapsto g$ is continuous.
\item Let $V_i:=p(t_i^{-1}(0,1])$ be an open set. Then, $s_i:V_i\to EG:p(t_0g_0,t_1g_1,\cdots)\mapsto(t_0g_0,t_1g_1,\cdots)$ is a local section.
\end{itemize}
\end{pf}


\begin{lem*}[3.6]
Let $B$ be a paracompact space and $\xi=(G\to E\to B,G)$ be a principal $G$-bundle.
Then, there is a \emph{countable} partition of unity $\{v_n\}_{n=0}^\infty$ on $B$ such that $\xi|_{v_n^{-1}(0,1]}$ is trivial, $\{v_n^{-1}(0,1]\}_{n=0}^\infty$ is an open cover of $B$, and $\{v_n\}_{n=0}^\infty$ is locally finite.
\end{lem*}
\begin{pf}
By the paracompactness, we have a possibly uncountable partition of unity $\{w_t\}_{t\in T}$ satisfying the three conditions in the statement.
We can see that $S(b):=\{t\in T:w_t(b)>0\}$ is finite.
For each finite set $S\in T$,
\[\cU_S:B\to[0,1]:b\mapsto\max\{0,\min_{\substack{s\in S\\t\in T\setminus S}}\{w_s(b)-w_t(b)\}\}\]
is a continuous function, and
\[V(S):=\cU_S^{-1}((0,1])=\{b:w_s(b)-w_t(b)\text{ for }s\in S,\ t\in T\setminus S\}\]
is an open set such that $V(S)\subset\bigcup_{s\in S}w_s^{-1}(0,1]$.
If, in addition, $S,S'\subset T$ satisfy $|S|=|S'|<\infty$, then $S=S'$ or $V(S)\cap V(S')=\varnothing$.
Here we can define a disjoint union $V_m:=\bigcup_{|S|=m}V(S)$ and a finite sum $u_m(b):=\sum_{|S|=m}\cU_S(b)$ so that we have $V_m=u_m^{-1}((0,1])$ for each $m$.
Thus, $\{V_m\}_{m=0}^\infty$ is an open cover of $B$.
Then, $\xi|_{V_m}$ is trivial since $\xi|_{V(S)}$ is trivial for each $S$, and $v_m(b):=u_m(b)/\sum_{n\ge0}u_n(b)$ provides our partition of unity.
\end{pf}


\begin{pf}[Proof of Theorem 3.4]
We want to show $[B,BG]\to\mathrm{Prin}_G(B)$ is bijective.

(Surjectivity)
By Lemma 3.6, we have a countable partition of unity $v_m$ that subordinates to an open cover $V_m$, and local trivializations $\f_m:\pi^{-1}(V_m)\to V_m\times F$.
With these, if we define
\[\tilde f:E\to EG:z\mapsto(v_0(\pi(z))\pr_2(\f_0(z)),v_1(\pi(z))\pr_2(\f_1(z)),\cdots),\]
where we set $v_m(\pi(z))=0$ and $\pr_2(\f_m(z))=e$ for $z\notin\pi^{-1}(V_m)$,
then it induces a well-defined right $G$-equivariant $f:B\to BG$.
Also $\tilde f$ is continuous according to the topology we have set on $EG$, so is $f$(\textbf{Exercise 7}).
Since $(\tilde f,f):\xi\to \xi_G$ is now a bundle map, $\xi\cong f^*\xi_G$ over $B$.

(Injectivity)
We first claim that $\id_{BG}$ is homotopic to a continuous map $F_0^{odd}:BG\to BG$, which will be defined later, and satisfies that $F_0^{odd}(BG)=p(\{t_0g_0,t_1g_1,\cdots)$.
Define
\[\tilde F^{odd}:EG\times[0,1]\to EG:((t_0g_0,\cdots),s)\mapsto(t'_0g'_0,\cdots),\]
where $t'_i$ and $g'_i$ are determined as follows: if $s\in[1-(\frac12)^n,1-(\frac12)^{n-1}]=:I_n$ for $n\ge0$, then we have
\[t'_i:=\begin{cases}t_i&,0\le i\le n-1,\\\alpha_n(s)t_{n+j}&,i=n+2j\ (j\ge0),\\(1-\alpha_n(s))t_{n+j}&,i=n+2j+1\ (j\ge0),\end{cases}\qquad g'_i:=\begin{cases}g_i&,0\le i\le n-1,\\g_{n+j}&,i\in\{n+2j,n+2j+1\}\ (j\ge0),\end{cases}\]
where $\alpha_n:I\to I$ is a continuous function such that $\alpha_n(0)=0$, $\alpha_n(1)=1$, $\alpha_n$ is constant outside $I_n$ and is linear on $I_n$, and if $s=1$, then $t'_i:=t_i$ and $g'_i=g_i$.
Then, $\tilde F^{odd}$ is continuous.
Since $\tilde F^{odd}$ is equivariant, it induces $F^{odd}:BG\times[0,1]\to BG$, which is a homotopy between $\id_{BG}$ and $F_0^{odd}:=F^{odd}(-,0)$.

Now we prove the injectivity: if $f_0^*\xi\sim f_1^*\xi$, then $f_0\sim f_1$.
Consider the pullback maps $\tilde f_0$ and $\tilde f_1$ of $\xi_G$.
If $t_i$ and $g_i$ are given such that
\[\tilde F_0^{even}\circ\tilde f_0=(t_0(x)g_0(x),0,t_2(x)g_2(x),0,\cdots),\quad\tilde F_0^{odd}\circ\tilde f_1=(0,t_1(x)g_1(x),0,t_3(x)g_3(x),\cdots),\]
then
\[\tilde F(x,s):=(st_0(x)g_0(x),(1-s)t_1(x)g_1(x),st_2(x)g_2(x),(1-s)t_3(x)g_3(x),\cdots)\]
is right $G$-equivariant, which induces a homotopy $F:BG\times[0,1]\to BG$ between $F_0^{even}\circ f_0$ and $F_0^{odd}\circ f_1$.
It implies
\[f_0\sim F_0^{even}\circ f_0\sim F_0^{odd}\circ f_1\sim f_1.\qedhere\]
\end{pf}

\begin{ex*}
If $G=S^1=\SO(2)=\rU(1)$, the Hopf bundle can be described by the pullback
\[\begin{tikzcd}
S^{2n+1} \ar{r}{\tilde f_n}\ar{d} & ES^1 \ar{d}\\
\CP^n \ar{r}{f_n} & BS^1
\end{tikzcd}\]
where $\tilde f_n(z_0,\cdots,z_n)\mapsto(|z_0|^2\frac{z_0}{|z_0|},|z_1|^2\frac{z_1}{|z_1|},\cdots)$.
By taking limit, we have a commutative diagram
\[\begin{tikzcd}
S^\infty \ar{r}{\tilde f}\ar{d} & ES^1 \ar{d}\\
\CP^\infty \ar{r}{f} & BS^1
\end{tikzcd}\]
and $\tilde f$ has the inverse $(t_0g_0,t_1g_1,\cdots)\mapsto(\sqrt{t_0}g_0,\sqrt{t_1}g_1,\cdots)$.
Therefore, $\xi_{S^1}=(S^1\to S^\infty\to\CP^\infty,S^1)$.
\end{ex*}


\newpage
\section{Day 8: June 5}
The total space $EG$ is always contractable.

Note that the base space $BG$ need not be paracompact, but in most cases $BG$ is paracompact, e.g.~compact Lie groups and discrete groups.

\begin{prop*}[3.7]
Let $B_1,B_2$ be paracompact spaces and $\xi_1,\xi_2$ be the universal principal $G$-bundles of $B_1,B_2$, respectively.
Then, there is a homotopy equivalence $f:B_1\to B_2$ such that $f^*\xi_2=\xi_1$.
In particular, $BG$ is uniquely defined for a homotopy equivalence of paracompact spaces.
\end{prop*}
\begin{pf}
By the universality, we have maps $f_1:B_1\to B_2$ and $f_2:B_2\to B_1$ such that $f_1^*\xi_2=\xi_1$ and $f_2^*\xi_1=\xi_2$.
Then, $\id^*\xi_1=\xi_1=(f_2\circ f_1)^*\xi$ implies $f_2\circ f_1\simeq\id$ by Theorem 3.4.
\end{pf}

\begin{thm*}[3.8]
Let $\xi=(G\to E\to B,G)$ be a principal $G$-bundle on a paracompact space $B$.
If $E$ is contractible, then $\xi$ is universal principal $G$-bundle.
\end{thm*}
\begin{pf}
For a paracompact $X$, we want to show the correspondence $[X,B]\to\mathrm{Prin}_G(X):[f]\mapsto f^*\xi$ is bijective.

(Surjectivity)
Let $\eta=(G\to E_1\to X,G)$ be a principal $G$-bundle on $X$.
Define a left action $G\curvearrowleft E$ such that $ge:=eg^{-1}$ using the principal right action on $E$.
Let $(\eta,\xi)=(E\to E_1\times_G E\to X)$ be the associated bundle of $\eta$.
Since $E$ is contractible, it admits a global section by Theorem 3.9, which implies the existence of a bundle map $(\tilde f,f):\eta\to\xi$.
Thus, $f^*\xi\cong\eta$.

\noindent\textbf{Problem 8.} For two principal $G$-bundles $\xi_i=(G\to E_i\to B_i,G)$ with $i=1,2$, show that there is a bundle map $(\tilde f,f):\xi_1\to\xi_2$ if and only if the associated bundle $(\xi_1,\xi_2):=(E_2\to E_1\times_G E_2\to B_1)$ has a global section.

(Injectivity)
Suppose $f_0,f_1:X\to B$ satisfies $f_0^*\xi\cong f_1^*\xi$.
Then, there eixst bundle maps $(\tilde f_i,f_i):f_i^*\xi\to\xi$ and a bundle isomorphism $(\tilde h,\id_X):f_0^*\xi\to f_1^*\xi$.
Define $\xi:=(G\to f_0^*E\times[0,1]\xrightarrow{\pi\times\id}X\times[0,1],G)$, where $\pi:f_0^*\xi\to X$.
Also define a partial bundle map $s:\zeta|_{X\times([0,\frac12)\cup(\frac12,1])}\to\xi$ such that
\[s(z,t):=\begin{cases}
\tilde f_0(z),&t<\frac12,\\
\tilde f_1(\tilde h(z)),& t>\frac12
\end{cases}.\]
We can see $s$ as a section of the associated bundle $(\zeta,\xi)$ on $X\times([0,\frac12)\cup(\frac12,1])$.
By Theorem 3.9, there exists a section $s'$ on $X\times[0,1]$ such that $s'|_{X\times\{0,1\}}=s|_{X\times\{0,1\}}$.
The bundle map $S:\zeta\to\xi$ corresponding to $s'$ induces a map $X\times [0,1]\to B$, and it is a homotopy between $f_0$ and $f_1$.
\end{pf}

\begin{thm*}[3.9]
Let $A$ be a closed subset of a paracompact Hausdorff space $B$, and $C$ be a contractible space.
Let $\zeta=(C\to E\to B)$ be a $C$-bundle.
For a section $s:N\to E$ on an open neighborhood of $N$ of $A$, there exists a section $S:B\to E$ such that $S|_A=s|_A$.
In particular, if we take $A=N=\varnothing$, then $S:B\to E$ always exists.
\end{thm*}
\begin{pf}
Uploaded on ITC-LMS.
\end{pf}

\begin{ex*}
Recall that $\xi=(\rO(k)\to V_k^O(\R^n)\to\mathrm{Gr}_k(\R^n),\rO(k))$ is a principal $\rO(k)$-bundle, where
\[V_k^O(\R^n)=\{M\in M_{n\times k}(\R)\mid\rk(M)=k,\text{orthonormal columns}\}\cong\rO(n)/\rO(n-k).\]
Here, we refer to a well-known fact: $\mathrm{Gr}_k(\R^n)$ has a natural cell decomposition(Schubert) so that $\Gr_k(\R^n)\to\Gr_k(\R^{n+1})$ induced by the embedding $\R^n\to\R^{n+1}$ is a cellular map.

Let $\xi_{\rO(k)}=(\rO(k)\to V_k^O(\R^\infty)\to\Gr_k(\R^\infty),\rO(k))$.
(space of $k$-dimensional subspaces of a separable infinite dimensional Hilbert space..?)
\end{ex*}
\begin{thm*}[3.10]
$\xi_{\rO(k)}$ is a universal principal $\rO(k)$-bundle.
\end{thm*}
\begin{pf}
By Theorem 3.8, enough to show $V_k^O(\R^\infty)$ is contractible.

Define unilateral $k$-shift $f:\R^\infty\to\R^\infty$, where $\R^\infty=\bigcup_{n\ge1}\R^n$.
With the inner product on $\R^\infty$, we can check
\[h_t(v):=\mathrm{Schmidt}[(1-t)v_1+tf(v_1),\cdots,(1-t)v_k+tf(v_k)]\]
is well-defined homotopy such that $h_0=\id$ and $h_1(V_k^O(\R^\infty))=V_k^O(\{0\}^k\times\R^\infty)$.
We also have
\[h'_t:V_k^O(\{0\}^k\times\R^\infty)\to V_k^O(\R^\infty)\]
such that
\[h'_t(w_1,\cdots,w_k)\mapsto\mathrm{Schmidt}[te_1+(1-t)w_1,\cdots,te_k+(1-t)w_k],\]
and by connecting them, we are done.
\end{pf}
With the same reasoning,
\begin{gather*}
\xi_{\GL(k,\R)}=(\GL(k,\R)\to V_k(\R^\infty)\to\Gr_k(\R^\infty),\GL(k,\R),\\
\xi_{\rU(k)}=(\rU(k)\to V_k^U(\C^\infty)\to\Gr_k(\C^\infty),\rU(k),\\
\xi_{\GL(k,\C)}=(\GL(k,\C)\to V_k(\C^\infty)\to\Gr_k(\C^\infty),\GL(k,\C)
\end{gather*}
are all universal.

\begin{ex*}
Let $G$ be a compact Lie group with closed embedding $G\hookrightarrow\rU(k)$.
Then, the bundle
\[G\to V_k^U(\C^\infty)\to V_k^U(\C^\infty)/G\]
has a paracompact base space, hence it is universal since $V_k^U(\C^\infty)$ is contractible.
If we write $EG=V_k^U(\C^\infty)$ and $BG=V_k^U(\C^\infty)/G$, then $U(k)/G\to BG\to BU(k)$ is a bundle.
\end{ex*}

In differential geoemtry, we sometimes stop to limit and take a sufficiently large but finite $n$ when, for example, considering $V_k^U(\C^\infty)/G=\lim_\to V_k^U(\C^n)/G$.

Let $G$ act $F$ faithfully.

\begin{defn*}[3.11]
For $\xi=(F\to E\to B,G)$ with $B$ paracompact,
if $c(\xi_1)=f^*(c(\xi_2))\in H^*(B;R)$ for every bundle map $(\tilde f,f):\xi_1\to\xi_2$, then $c$ is called a \emph{characteristic class} of bundles with structure group $G$ and fiber $F$ with coefficient $R$.
\end{defn*}
From definition, $\xi_1\cong\xi_2$ implies $c(\xi_1)=c(\xi_2)$, and for trivial $\xi$, $c(\xi)=0$.

\begin{prop*}[3.12]
Let $G$ be a topological group such that we can take paracompact $BG$.
Then,
\[\left\{\begin{tabular}{c}
characteristic classes\\
of $G,F,R$
\end{tabular}\right\}=H^*(BG,R).\]
\end{prop*}
Let $a\in H^*(BG,R)$.
For $\xi=(F\to E\to B,G)$, if we define $a(\xi):=f^*(a)$ for $f:B\to BG$, then $a(\xi)$ is a characteristic class.




\newpage
\section{Day 9: June 12}

\subsection*{4. Characteristic classes of vector bundles}

Let $X$ be a paracompact space.
Then,
\[\left\{\begin{tabular}{c}
isomorphism classes of\\
rank $n$ $\mathbb{K}$-vector bundles on $X$
\end{tabular}\right\}
\leftrightarrow[X,B\GL(n,\mathbb{K})].\]
Note that $BGL(n,\mathbb{K})=\Gr_n(\mathbb{K}^\infty)$.

\begin{ex*}
Let $n=1$, $\K=\C$.
Since $G=\GL(1,\C)$ and we can reduct it to $\rU(1)$, which, in fact, has the following homeomorphism:
\[BG=B\GL(1,\C)\cong B\rU(1)\cong\CP^\infty.\]
We will see later about these isomorphisms.
\[H^*(\CP^\infty)=\begin{cases}
	\Z,&*\text{ even }\\ 0,&*\text{ odd }
\end{cases},\]
and we have a rign isomorphism $H^*(\CP^\infty)\cong\Z[c_1]$, where $c_1\in H^2(\CP^\infty)$ is the element given by the cocycle $e^2\mapsto-1$.
Here,
\[\CP^\infty=\bigcup_{n\ge1}\CP^n=e_0\cup e^2\cup e^3\cup\cdots\]
represents the CW structure.
\end{ex*}

\begin{thm*}[4.1]
Let $X$ be a CW complex.
If we see $\mathrm{Vect}_\C^1(X)$ as an abelian group with tensor product, and if we denote by $\gamma_1$ the tautological bundle, then we have group isomorphisms
\[\begin{tikzcd}
\mathrm{Vect}_\C^1(X) \ar{r}{=} & {[X,B\rU(1)(=\CP^\infty)]} \ar{r}{\cong} & H^2(X) \\
f^*\gamma_1 \ar[|->]{r} & f \ar[|->]{r} & f^*(c_1)=:c_1(\xi).
\end{tikzcd}\]
\end{thm*}

The element $c_1\in H^2(\CP^\infty)$ is called the \emph{first Chern class}, and also called the \emph{Euler class} for oriented real plane bundles.
We have
\[\mathrm{Vect}_\C^1(\CP^\infty)=[\CP^\infty,\CP^\infty]\cong H^2(\CP^\infty)\cong\Z,\]
\[\gamma_1\mapsto\id\mapsto c_1\mapsto -1.\]



\bigskip
\subsubsection*{Basics of homotopy theory}
Let $(X,x_0)$ be a based space.
\[\pi_n(X,x_0):=[(S^n,b),(X,x_0)]=[(I^n,\partial I^n),(X,x_0)].\]
The homotopy group $\pi_n(X,x_0)$ is indeed given a group structure from the componentwise composition along $(I_n,\partial I_n)$.
It is abelian for $n\ge2$.
By the obstruction theory, we have for $f:(S^n,b)\to(X,x_0)$ that $[f]=0\in\pi_n(X,x_0)$ if and only if it has an extension $D^{n+1}\to X$.

\bigskip
\subsubsection*{Eilenberg-MacLane complex}
\begin{defn*}[4.2]
Let $G$ be a group and $n\in\Z$ with $n\ge1$, such that $G$ is abelian if $n\ge2$.
Then, a (homotopy class of a) topological space $X$ which satisfies $\pi_n(X)=G$ and $\pi_i(X)=0$ for $i\ne n$ is called the \emph{Eilenberg-MacLane space} and denoted by $K(G,n)$.
\end{defn*}
\begin{prop*}[4.3]\,
\begin{parts}
\item For every group $G$, there is $K(G,1)$ space which is a CW complex.
\item For every abelian group $G$ and $n\ge2$, there is $K(G,n)$ space which is a CW complex.
\item The space $K(G,n)$ is unique up to homotopy.
\end{parts}
\end{prop*}
\begin{ex*}
\[K(\{1\},1)=\{*\}=\text{any contractible spaces},\quad K(\Z,1)=S^1,\quad K(\Z^n,1)=T^n,\quad K(F_n,1)=\bigvee_{n\in\N}S^1.\]
There is a long exact sequence
\begin{cd}[row sep=5pt, column sep=5pt]
\,&\Z&\R&S^1\\
\pi_3&0\ar{r}&0\ar{r}&0\ar{dll}\\
\pi_2&0\ar{r}&0\ar{r}&0\ar{dll}\\
\pi_1&0\ar{r}&0\ar{r}&\Z\ar{dll}\\
\pi_0&\Z&&\\
\end{cd}
Since the universal cover of $\bigvee_\N S^1$, given by the Cayley graph, is contractible so that we have $\pi_i(\bigvee_\N S^1)=0$ for $i\ge2$.
\[K(\Z,2)=\CP^\infty,\quad K(\Z/2\Z,1)=\RP^\infty.\]
\[
\begin{tikzcd}[row sep=5pt, column sep=5pt]
\,&S^1&S^\infty&\CP^\infty\\
\pi_3&0\ar{r}&0\ar{r}&0\ar{dll}\\
\pi_2&0\ar{r}&0\ar{r}&\Z\ar{dll}\\
\pi_1&\Z\ar{r}&0\ar{r}&0\ar{dll}\\
\pi_0&0&&\\
\end{tikzcd}
\qquad\qquad
\begin{tikzcd}[row sep=5pt, column sep=5pt]
\,&S^0&S^\infty&\RP^\infty\\
\pi_3&0\ar{r}&0\ar{r}&0\ar{dll}\\
\pi_2&0\ar{r}&0\ar{r}&0\ar{dll}\\
\pi_1&0\ar{r}&0\ar{r}&\Z/2\Z\ar{dll}\\
\pi_0&\Z/2\Z&&\\
\end{tikzcd}
\]
\end{ex*}

\begin{pf}[Partial(?) proof of Proposition 4.3]
For every $B$ which is $K(G,1)$, we have $\xi=(G\to\tilde B\to B,G)$ for a discrete $G$ and the universal cover $B$.
The homotopy long exact sequence implies $\pi_i(\tilde B)=\{1\}$ for all $i\ge1$, then Whitehead theorem says that $\tilde B$ is contractible.
Therefore, $\xi$ is a universal principal $G$-bundle, in particular,
\[K(G,1)=BG.\]
Since $BG$ is unique up to homotopy, so is $K(G,1)$.

For the existence, we have two ways.
For the first method, we can slightly modify the Milnor construction of $BG$.
For the second method, if we wright $G=\<g_\alpha\mid r_\beta\>_{\alpha\in\cA,\beta\in\cB}$, then we can attach 2-cells on $X_1:=K(F_\cA,1)$ to construct $X_2:=\bigvee_{\alpha\in\cA}S^1\cup(\text{2-cells})$ such that each attaching corresponds to $r_\beta$.
Then, $X$ is a CW complex that satisfies $\pi_1(X)=G$.
Put 3-cells and 4-cells and so on.
\end{pf}
\begin{ex*}
How to construct $K(\Z/2\Z,1)$ in the second method:
Note that $\Z/2\Z=\<x\mid x^2\>$ and $X_1=S^1$.
Then, we can show $X_n=\RP^n=e^0\cup\cdots\cup e^n$ using $\Z/2\Z\to S^n\to\RP^n$.
\end{ex*}

\noindent\textbf{Problem.}
Investigate the proof of Proposition 4.3.


\begin{thm*}[4.4]
Let $G$ be an abelian group.
For every CW complex $X$, we have a bijection
\[[X,K(G,n)]\xrightarrow{\sim} H^n(X,G):f\mapsto f^*(\iota).\]
Here,
\[H^n(K(G,n),G)\cong\Hom(H_n(K(G,n)),G)=\Hom(\pi_n(K(G,n)),G)=\Hom(G,G),\]
where the first isomorphism follows from the Hurewicz theorem.
Fundamental class of $K(G,n)$.
\end{thm*}

Now we will show
\[c_1(\xi_1\otimes\xi_2)=c_1(\xi_1)+c_1(\xi_2).\]
Suppose $g_i:X\to \CP^\infty$ are classfying maps of $\xi_i$, i.e. $g_i^*\gamma_1=\xi_i$.
Let $p_i:\CP^\infty\times\CP^\infty\to\CP^\infty$ be canonical projections.
Then, $p_1^*\gamma_1\otimes p_2^*\gamma_1$ is a line bundle on $\CP^\infty\times\CP^\infty$, let $h:\CP^\infty\times\CP^\infty\to\CP^\infty$ be the classifying map, i.e. $h^*\gamma\cong p_1^*\gamma_1\otimes p_2^*\gamma_1$.
Consider
\[g:X\xrightarrow{\Delta}X\times X\xrightarrow{g_1\times g_2}\CP^\infty\times\CP^\infty\xrightarrow{h}\CP^\infty.\]
Then,
\begin{align*}
g^*\gamma_1
&=\Delta^*(g_1\times g_2)^*h^*\gamma_1\\
&\cong\Delta^*(g_1\times g_2)^*(p_1^*\gamma_1\otimes p_2^*\gamma_1)\\
&\cong\Delta^*(g_1\times g_2)^*p_1^*\gamma_1\otimes\Delta^*(g_1\times g_2)^*p_2^*\gamma_1\\
&=g_1^*\gamma_1\otimes g_2^*\gamma_1\\
&\cong\xi_1\otimes\xi_2.
\end{align*}
Therefore,
\[c_1(\xi_1\otimes\xi_2)=c_1(g^*\gamma_1)=g^*(c_1(\gamma_1))=\Delta^*(g_1\times g_2)^*h^*c_1(\gamma_1).\]
Consider an isomorphism
\[H^2(\CP^\infty\times\CP^\infty)\xrightarrow{i_1^*\oplus i_2^*}H^2(\CP^\infty\times\{p\})\oplus H^2(\{p\}\times\CP^\infty).\]
such that $(i_1^*\oplus i_2^*)(p_1^*a+p_2^*b)=(a,b)$.
Then, we have
\begin{align*}
i_1^*h^*c_1(\gamma_1)
&=c_1(i_1^*h^*\gamma_1)=c_1(i_1^*(p_1^*\gamma\otimes p_2^*\gamma1))\\
&=c_1(i_1^*p_1^*\gamma_1\otimes i_1^*p_2^*\gamma_1)\\
&=c_1(\gamma_1\otimes\C)=c_1(\gamma_1).
\end{align*}
Similarly, $i_2^*h^*c_1(\gamma_1)\cong c_1(\gamma_1)$.
Therefore, $h^*c_1(\gamma_1)=p_1^*c_1(\gamma_1)+p_2^*c_1(\gamma_1)$, and
\begin{align*}
c(\xi_1\otimes\xi_2)
&=\Delta^*(g_1\times g_2)^*h^*c_1(\gamma_1)\\
&=\Delta^*(g_1\times g_2)^*(p_1^*c_1(\gamma_1)+p_2^*c_1(\gamma_1))\\
&=g_1^*c_1(\gamma_1)+g_2^*c_1(\gamma_1)\\
&=c_1(g_1^*\gamma_1)+c_1(g_2^*\gamma_1)\\
&=c_1(\xi_1)+c_1(\xi_2).
\end{align*}



\newpage
\section{Day 10: June 19}

\begin{thm*}[4.5]
Let $X$ be a CW complex.
If we see $\mathrm{Vect}_\R^1(X)$ as an abelian group with tensor product, and if we denote by $w_1$ the tautological bundle, then we have group isomorphisms
\[\begin{tikzcd}
\mathrm{Vect}_\R^1(X) \ar{r}{=} & {[X,B\rO(1)]} \ar{r}{\cong} & H^1(X,\Z/2\Z) \\
\xi \ar[|->]{r} & f \ar[|->]{r} & f^*(w_1)=:w_1(\xi).
\end{tikzcd}\]
\end{thm*}
Here we note that $B\rO(1)=\RP^\infty=K(\Z/2\Z,1)$.
If $w_1(\xi)\ne0$, then from
\[w_1(\xi)\in H_1(X,\Z/2\Z)=\Hom(H_1(X),\Z/2\Z)=\Hom(\pi_1(X),\Z/2\Z),\]
there is a loop $\gamma$ on $X$ such that $w_1(\gamma^*\xi)\ne0$.
It means that on the loop $\gamma$ the line bundle is given by the M\"obius strip.


\subsection*{Thom isomorphism}

The Lie groups $\SO(n)\subset\SL(n,\R)\subset\GL^+(n,\R)$ are all homotopy equivalent for $n\ge1$.
Let $G$ be any of them.
Let $\xi=(\R^n\to E\to B,G)$ be a vector bundle with paracompact $B$.
Suppose $\xi$ is oriented, i.e.~there is a (maximal) collection of local trivializations $\{(U_\alpha,\f_\alpha)\}_\alpha$ with $A_x\in G$ that is globally determined such that whenever $U_\alpha\cap U_\beta\ne\varnothing$ we have
\[\f_\beta\circ\f_\alpha^{-1}:(U_\alpha\cap U_\beta)\times\R^n\to(U_\alpha\cap U_\beta)\times\R^n:(x,v)\mapsto(x,A_xv).\]

An \emph{orientation} on a $n$-dimensional real vector space $V$ is defined as an element of
\[\{\text{ordered bases of $V$}\}/\GL^+(n,\R)=((\bigwedge^n V)\setminus\{0\})/\R_{>0}.\]
Write $V_0:=V\setminus\{0\}$.

Let $\sigma:\Delta^n\hookrightarrow V$ be an orientation preserving embedding such that $\frac1{n+1}\sum_{i=0}^ne_i\mapsto0\in V$, where the orientation on $\Delta^n\subset\R^{n+1}$ is given by $\<e_1-e_0,\cdots,e_n-e_{n-1}\>$. 
Note that
\[H_n(V,V_0)\cong H_{n-1}(V_0)\cong H_{n-1}(S^{n-1})\cong\Z.\]
Then $[\sigma]\in H_n(V,V_0)$ is a generator.
Take a dual $U_V\in H^n(V,V_0)\cong\Z$ such that $\<U_V,[\sigma]\>=1$.
In other words, $H^n(V,V_0)$ determines an orientation of $V$.

Let $\xi=(\R^n\to E\to B,\GL^+(n,\R)$.
Let $\R^n$ be a vector space oriented by $\<e_1,\cdots,e_n\>$.
Then, orientations on $E_x=p^{-1}(x)$ are induced for each $x\in B$ via arbitrarily taken local trivializations.
The orientation does not depend on the choice of local trivializations in the structure group preserves the orientation.

\begin{thm*}[4.6, Thom's isomorphism theorem]
Let $B$ be a paracompact, arc-connected space.
There is a unique $t\in H^n(E,E_0)$, called the \emph{Thom class}, such that
\[H^n(E,E_0)\to H^n(E_x,E_{x,0}):t\mapsto U_x.\]
Also, we have an isomorphism, called the \emph{Thom isomorphism},
\[H^j(E)\to H^{j+n}(E,E_0):[\sigma]\mapsto[\sigma]\smile t\]
for each $j$.
The above theorem holds for PID coefficients.
\end{thm*}

See also the Leray-Hirsch theorem.
We prove the Thom isomorphism theorem in the following substeps:
\begin{parts}
\item If 
\end{parts}

\begin{pf}
Suppose first that $E=B\times\R^n$.
By the (weak) K\"unneth theorem,
\[H^*(B)\otimes H^*(\R^n,\R^n_0)\xrightarrow{\times}H^*(B\times\R^n,B\times\R^n_0)\]
is a ring isomorphism
Since $H^i(\R^n,\R^n_0)=0$ for $i\ne n$ and $H^n(\R^n,\R^n_0)=\Z$, the above cross product means that
\[H^j(B)\otimes H^n(\R^n,\R^n_0)\xrightarrow{\times}H^{j+n}(B\times\R^n,B\times\R^n_0)\]
is an isormorphism between abelian groups.
We can take a generator $e^n\in H^n(\R^n,\R^n_0)$ inductively by the following isomorphism:
\[H^{n-1}(\R^{n-1},\R^{n-1}_0)\otimes H^1(\R,\R_0)\xrightarrow{\times}H^n(\R^n,\R^n_0):e^{n-1}\otimes e_1\mapsto e^n.\]
(Consider $H^*(X,A)\otimes H^*(Y,B)\xrightarrow{\times}H^*(X\times Y,(X\times B)\cup(A\times Y))$)
Then, every morphism of the following diagram is an isomorphism:
\[\begin{tikzcd}
\Z \ar{r}\ar{d} & H^0(B) \ar{r}{\times e^n}& H^n(B\times\R^n,B\times\R_0^n)\\
\Z \ar{r} & H^0(\{x_0\}) \ar{r}{\times e^n}& H^n(\{x_0\}\times\R^n,\{x_0\}\times\R_0^n).\end{tikzcd}\]
Therefore $t=1\times e^n$ is unique element satisfying the theorem.
The condition is indeed satisfied by checking the following diagram of isomorphisms commutes:
\[\begin{tikzcd}
H^j(B)\otimes H^0(\R^n)\ar{r} & H^j(B\times\R^n) \ar{r}{\smile t} & H^{j+n}(B\times\R^n,B\times\R_0^n)\\
H^j(B)\ar{u}\ar{urr}{\times e^n}&\,&\,.\\
\end{tikzcd}\]

Now consider $B$ is decomposed into $B=B_1\cup B_2$, where the theorem holds on $B_1$ and $B_2$.





\end{pf}




\begin{defn*}[4.7]
Let $\xi=(\R^n\to E\to B,\GL^+(n,\R))$ be an oriented real vector bundle.
We define the \emph{Euler class} $e(\xi)\in H^n(B)$ such that \[H^n(E,E_0)\to H^n(E)\xrightarrow{s^*,\cong}H^n(B):t\mapsto e(\xi).\]
For the case $R=\Z/2\Z$, then the orientation is not necessary to be given on $\xi$, and then we can define the $\Z/2\Z$-coefficient Euler class.
\end{defn*}

\begin{thm*}[4.8]
The Euler class is a characteristic class for oriented real vector bundles.
\end{thm*}
\begin{pf}
The naturality of $H^n(E,E_0)\to H^n(B)$.
The pullback of the Thom class, preseves fibers.
\end{pf}




\newpage
\section{Day 11: June 26}

\begin{prop*}[4.10]
Let $\xi=(\R^m\to E\to X)$, $\xi':(\R^n\to E'\to Y)$ be oriented real vector bundles and $t_\xi$, $t_{\xi'}$ be their Thom classes.
Define
\[\xi\times\xi':=(\R^{m+n}\to E\times E'\to X\times Y).\]
\begin{parts}
\item With the orientation naturally induced on $\xi\times\xi'$, we have $t_{\xi\times\xi'}=t_\xi\times t_{\xi'}\in H^{m+n}(E\times E',(E\times E')_0)$.
\item $e(\xi\times\xi')=e(\xi)\times e(\xi')\in H^{m+n}(X\times Y)$.
\item If $X=Y$, then the Euler class of the Whitney sum(pullback of $\xi\times\xi'$ along the diagonal map) is $e(\xi\oplus\xi')=e(\xi)\smile e(\xi')\in H^{m+n}(X)$.
\end{parts}
\end{prop*}
\begin{pf}
If we show (a), then the rest easily follows.
If we restrict to each fiber $(E\times E')_{(x,y)}=E_x\times E'_y$, then
\[H^{m+n}(E_x\times E'_y,(E_x\times E_y)_0)\cong H^m(E_x,E_{x,0})\otimes H^n(E'_y,E'_{y,0}),\]
where $u_{(x,y)}$ maps from $u_x\otimes u_y$.
\end{pf}

\begin{cor*}[4.11]
If $\xi=(\R^n\to E\to B,\GL^+(n,\R)$ admits a nowhere vanishing section, then $e(\xi)=0$.
\end{cor*}
\begin{pf}
The existence of nowhere vanishing section implies that $\xi$ admits a trivial line subbundle.
Then, $e(\xi)=e(\R\otimes\xi/\R)=e(\R)\smile e(\xi/\R)=0\smile e(\xi/\R)=0$.
\end{pf}

\begin{exe*}[11]
Let $\eta=(\R^{2n+1}\to E\to B,\GL^+(2n+1,\R))$.
Show that $2e(\eta)=0\in H^{2n+1}(B)$.	
\end{exe*}

\subsubsection*{Gysin sequence}
Let $\xi=(\R^n\to E\to B,\GL^+(n,\R))$.
Then, there is a long exact sequence of cohomology:
\[\cdots\to H^{q-1}(E_0)\to H^q(E,E_0)\to H^q(E)\to H^q(E_0)\to H^{q+1}(E,E_0)\to\cdots.\]
Note that there are isomorphisms
\[H^{q-n}(B)\xrightarrow{p^*,\sim}H^{q-n}(E)\xrightarrow{\cdot\smile t,\sim}H^q(E,E_0)\quad\text{ and }\quad H^q(B)\xrightarrow{p^*,\sim}H^q(E),\]
so the long exact sequence becomes
\[\cdots\to H^{q-1}(E_0)\xrightarrow{\cdot\smile e(\xi)} H^{q-n}(B)\to H^q(B)\to H^q(E_0)\to H^{q-n+1}(B)\to\cdots.\]
This sequence is called the \emph{Gysin sequence}.

\begin{ex*}\,
\begin{parts}
\item
Consdier the cohomology ring $H^*(\RP^n)=H^*(\RP^n,\Z/2\Z)$.
Let $\gamma_1=(\R\to E\to\RP^\infty)$ be the universal real line bundle.
Note that $\RP^\infty=K(\Z/2\Z,1)=B\Z/2\Z$ and $E=S^\infty\times_{\Z/2\Z}\R$ because the associated principal bundle is given by $\Z/2\Z\to S^\infty\to\RP^\infty$.
From the Gysin sequence
\[H^{q-1}(E_0)\to H^{q-1}(\RP^\infty)\xrightarrow{\cdot\smile e(\gamma_1)}H^q(\RP^\infty)\to H^q(E_0)\]
with $E_0\simeq S^\infty\simeq\{*\}$, we have an isomorphism
\[\Z/2\Z\cong H^{q-1}(\RP^\infty)\xrightarrow{\cdot\smile e(\gamma_1)}H^q(\RP^\infty)\cong\Z/2\Z\]
for $q\ge2$.
Therefore $e(\gamma_1)$ is nonzero in $H^1(\RP^\infty)$, so it coincides with the Stiefel-Whitney class $w_1$.
Also we have
\[H^{-1}(\RP^\infty)\to H^0(\RP^\infty)\to H^0(E_0)\to H^0(\RP^\infty)\xrightarrow{\cdot\smile w_1}H^1(\RP^\infty)\to0,\]
which gives
\[0\to\Z/2\Z\to\Z/2\Z\to\Z/2\Z\to\Z/2\Z\to0.\]
Hence we have a ring isomorphism $H^*(\RP^\infty,\Z/2\Z)\cong\Z/2\Z[w_1]$ to a polynomial ring.

\item
Consider the cohomology ring $H^*(\CP^\infty)=H^*(\CP^\infty,\Z)$.
A complex vector bundle has a natural orientation as a real vector bundle(\textbf{Exercise 12}; If $(z_j)_j$ is an ordered $\C$-basis, then $(z_j,iz_j)_j$ is an ordered $\R$-basis.).
Consider the universal line bundle $\gamma_1=(\C\to S^\infty\times_{\rU(1)}\C\to\CP^\infty)$.
Similarly as (a), we have an isomorphism
\[\cdot\smile e(\gamma_1):H^{q-2}(\CP^\infty)\to H^q(\CP^\infty)\]
for $q\ge2$.
Combining with $H^0(\CP^\infty)\cong\Z$ and $H^1(\CP^\infty)=0$, we have $H^*(\CP^\infty)\cong\Z[e(\gamma_1)]$, where $e(\gamma_1)\in H^2(\CP^\infty)\cong\Z$.
\end{parts}
\end{ex*}

\begin{thm*}[4.13]
Suppose $\CP^\infty=e^0\cup e^2\cup e^4\cup\cdots$ be the canonical cell decomposition.
If we give an orientation on $e^2$ via $e^2\cong\CP^1$, then $\<e(\gamma_1),e^2\>=-1$.
In other words,
\[e(\gamma_1)=c_1(\gamma_1)=-1\in H^2(\CP^\infty)\cong\Z.\]
\end{thm*}
\begin{pf}
Let $i:\CP^1\hookrightarrow\CP^\infty$ be the embedding as the 2-skeleton.
Since $\<i^*(e(\gamma_1)),e^2\>=\<e(i^*\gamma_1),e^2\>$, we are enough only to consider the case on $\CP^1$.
Precisely, we have $i^*\gamma_1=(\C\to S^3\times_{\rU(1)}\xrightarrow{p}\CP^1)$.
For convenience, fix a section $\tilde s:\CP^1\to S^3\times_{\rU(1)}\C$ given by
\[\tilde s([\lambda:\mu]):=\left(\left(\frac{\lambda}{\sqrt{|\lambda|^2+|\mu|^2}},\frac{\mu}{\sqrt{|\lambda|^2+|\mu|^2}}\right),\frac{\bar\lambda}{\sqrt{|\lambda|^2+|\mu|^2}}\right).\]

Identify the 2-cell $e^2$ as the image of $\C\mapsto\CP^1:z\mapsto[z:1]$.
We have a local trivialization on $e^2$:
\[p^{-1}(e^2)\to e^2\times\C:[(z_1,z_2),w]\mapsto([z_1:z_2],\frac{z_2w}{|z_2|}\]
such that
\[[(z_1\alpha,z_2\alpha),\alpha^{-1}w]=[(z_1,z_2),w]\quad\text{ and }\quad([z_1\alpha:z_2\alpha],\frac{z_2\alpha\alpha^{-1}w}{|z_2\alpha|}\]
with the inverse
\[([z_1:z_2],w)\mapsto[\frac{|z_2|}{z_2\sqrt{|z_1|^2+|z_2|^2}}(z_1,z_2),w].\]
Then, see the picture...
%\[\begin{tikzcd}

%\end{tikzcd}\]
\end{pf}


\begin{thm*}[4.14, Chern class and Stiefel-Whitney class]\,
\begin{parts}
\item We have ring isomorphisms
\[H^*(B\rU(n))\cong H^*(B\GL(n,\C))\cong H^*(\mathrm{Gr}_n(\C^\infty))\cong\Z[c_1,c_2,\cdots,c_n]\]
with the polynomial ring, where $c_i\in H^{2i}(\mathrm{Gr}_n(\C^\infty))$.
\item We have ring isomorphisms
\[H^*(B\rO(n),\Z/2\Z)\cong H^*(B\GL(n,\R),\Z/2\Z)\cong H^*(\mathrm{Gr}_n(\R^\infty),\Z/2\Z)\cong\Z/2\Z[w_1,w_2,\cdots,w_n]\]
with the polynomial ring, where $w_i\in H^{i}(\mathrm{Gr}_n(\R^\infty),\Z/2\Z)$.
\end{parts}
\end{thm*}
\begin{pf}
(a)
Consider the universal principal $\rU(1)$-bundle
\[\xi_{\rU(1)}=(\rU(n)\to V_n^U(\C^\infty)\to\Gr_n(\C^\infty),\rU(n)).\]
Then, with an embedding $A\mapsto\mat{1&0\\0&A}:\rU(n-1)\to\rU(n)$,
\[(\rU(n-1)\to V_n^U(\C^\infty)\to V_n^U(\C^\infty)/\rU(n-1),\rU(n-1))\]
is a universal principal $\rU(n-1)$-bundle.
Let $\iota:B\rU(n-1)=V_n^U(\C^\infty)/\rU(n-1)\to\Gr_n(\C^\infty)=B\rU(n)$.
Now we have a bundle
\[(\rU(n)/\rU(n-1)\to B\rU(n-1)\xrightarrow{\iota}B\rU(n),\rU(n))\]
with fiber $S^{2n-1}$.
For the universal $n$-dimensional complex vector bundle
\[\gamma_n=(\C^n\to E=V_n^U(\C^\infty)\times_{\rU(1)}\C^n\to\Gr_n(\C^\infty)),\]
we have
\begin{align*}
E_0
&=V_n^U(\C^\infty)\times_{\rU(n)}(\C^n\setminus\{0\})\\
&\sim V_n^U(\C^\infty)\times_{\rU(n)}S^{2n-1}\\
&\cong V_n^U(\C^\infty)/\rU(n-1)\\
&=B\rU(n-1).
\end{align*}
Here the notation $\cong$ means the bundle isomorphism with fibers $S^{2n-1}$ over $B\rU(n)$.
The Gysin sequnce with respect to $\gamma_n$ gives then
\[H^{q-1}(B\rU(n-1))\to H^{q-2n}(B\rU(n))\xrightarrow{\cdot\smile e(\gamma_n)}H^q(B\rU(n))\xrightarrow{\iota^*}H^q(B\rU(n-1))\to H^{q-2n+1}(B\rU(n))\]
For $n=1$, we have $H^*(B\rU(1))=H^*(\CP^\infty)=\Z[c_1]$.
As an induction hypothesis, assume $H^*(B\rU(n-1))=\Z[c_1,\cdots,c_{n-1}]$.

If $q-2n+1\le1$ with $q=2k$, then since
\[\iota^*:H^q(B\rU(n))\to H^q(B\rU(n-1)):c_k\mapsto c_k\]
is an isomoprhism, we have that $H^*(B\rU(n))\to H^q(B\rU(n-1))$ is surjective ring isomorphism.
Therefore, we have a short exact sequence
\[0\to H^{*-2n}(B\rU(n))\xrightarrow{\cdot\smile e(\gamma_n)}H^q(B\rU(n))\xrightarrow{\iota^*}H^q(B\rU(n-1))\to0.\]
Hence if we let $c_n=e(\gamma_n)$, then
\[0\to H^{*-2n}(B\rU(n))\xrightarrow{\cdot\smile c_n}H^*(B\rU(n))\xrightarrow{\iota^*}H^*(B\rU(n-1))\to0\]
is exact so that $H^*(B\rU(n))\cong\Z[c_1,\cdots,c_n]$.
\end{pf}
\begin{defn*}[4.15]
The element $c_i\in H^{2i}(B\rU(n))$ that has appeared in the proof of 4.14 is called the \emph{$i$th Chern class}.
Similarly, $w_i\in H^i(B\rO(n),\Z/2\Z)$ is called the \emph{$i$th Stiefel-Whitney class}.
In general, if $\xi=(\C^n\to E\to B)$ is an $n$-dimensional vector bundle and $f:B\to B\rU(n)$ is the classifying map of $\xi$, then we define the Chern class as $c_k(\xi)=f^*c_k$ if $1\le k\le n$, then $0$ if $k\ge n+1$.
\end{defn*}

\end{document}