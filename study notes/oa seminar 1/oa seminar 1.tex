\documentclass{../../small}
\usepackage{../../ikhanchoi}

\begin{document}
\title{Operator Algebra Seminar Note I}
\author{Ikhan Choi}
\maketitle
\tableofcontents


\section*{Acknowledgement}
This note has been written based on the first-year graduate seminar presented at the University of Tokyo in the 2023 Spring semester.
Each seminar was delivered for 105 minutes. % without referring to any notes or references.
The main reference for this seminar was Brown-Ozawa, and detailed gaps were filled with the aids of other books such as Takesaki, Murphy, and Paulsen whenever required.
% I gratefully acknowledge advice of Prof. Yasuyuki Kawahigashi and support of my colleagues Futaba Sato and Yusuke Suzuki.



\newpage
\section{April 14}

\subsection{Completely positive maps}

\begin{defn}
Let $\cA$ and $\cB$ be C$^*$-algebras.
A linear map $\f:\cA\to\cB$ is said to be \emph{completely positive} (c.p.) if the inflation $\f_n:M_n(\cA)\to M_n(\cB):[a_{ij}]\mapsto[\f(a_{ij})]$ is positive for each $n\ge1$.
\end{defn}

\begin{rmk}
For the positivity in matrix algebras, the following equivalent statements are useful.
\begin{parts}
\item $[a_{ij}]\in M_n(\cA)$ is positive.
\item $[a_{ij}]=[b_{ij}]^*[b_{ij}]=[b_{ji}^*][b_{ij}]=[\sum_kb_{ki}^*b_{kj}]$ for some $[b_{ij}]\in M_n(\cA)$.
\item $\sum_{i,j}\<\pi(a_{ij})\xi_j,\xi_i\>_H\ge0$ for $[\xi_i]\in H^n$, for a faithful representation $\pi:\cA\to B(H)$.
\item $\sum_{i,j}\<\pi(a_{ij})\xi_j,\xi_i\>_H\ge0$ for $[\xi_i]\in H^n$, for every representation $\pi:\cA\to B(H)$.
\end{parts}
\end{rmk}

\begin{ex}\,
\begin{parts}
\item A $*$-homomorphism is c.p.
\item A state is c.p.
\item A conjugation $B(\hat H)\to B(H):a\mapsto V^*aV$ is c.p. for every bounded linear $V:H\to\hat H$.
\item The transpose $M_2(\C)\to M_2(\C)$ is not c.p.
\item The convex combination, composition, restriction of c.p. maps is c.p.
\end{parts}
\end{ex}
\begin{pf}
(a)
A $*$-homomorphism is positive, and its inflations are all $*$-homomorphisms.

(b)
Let $\rho:\cA\to\C$ be a state.
If $[a_{ij}]=[\sum_kb_{ki}^*b_{kj}]\in M_n(\cA)_+$, then we have for $[x_i]\in\ell_2^n$ that
\[\sum_{i,j}\<\rho(a_{ij})x_j,x_i\>_\C=\sum_{i,j}\bar{x_i}\rho(a_{ij})x_j=\rho(\sum_{i,j,k}\bar{x_i}b_{ki}^*b_{kj}x_j)=\sum_k\rho((\sum_ib_{ki}x_i)^*(\sum_jb_{kj}x_j))\ge0.\]

(c)
If $[a_{ij}]=[\sum_kb_{ki}^*b_{kj}]\in M_n(B(\hat H))_+$, then we have for $[\xi_i]\in H^n$ that
\[\sum_{i,j}\<V^*a_{ij}V\xi_j,\xi_i\>=\sum_{i,j,k}\<b_{kj}V\xi_j,b_{ki}V\xi_i\>=\sum_k\<\sum_jb_{kj}V\xi_j,\sum_ib_{ki}V\xi_i\>\ge0.\]

(d)
We have a counterexample for $M_2(M_2(\C))\to M_2(M_2(\C))$:
\[\mat{1&0&0&1\\0&0&0&0\\0&0&0&0\\1&0&0&1}\mapsto\mat{1&0&0&0\\0&0&1&0\\0&1&0&0\\0&0&0&1}.\]
The former has an eigenvalues $\{2,0\}$, and the latter has $\{\pm1\}$.

(e) Clear.
\end{pf}

\begin{thm}[Stinespring dilation]
Let $\cA$ be a unital C$^*$-algebra and $\f:\cA\to B(H)$ be a c.p.~map.
Then, there is a representation $\pi:\cA\to B(\hat H)$ and a bounded linear operator $V:H\to\hat H$ such that the following diagram commutes:
\begin{cd}
\cA \ar{r}{\f} \ar[swap]{d}{\pi} & B(H) \\
B(\hat H) \ar[swap]{ur}{V^*\cdot V} &
\end{cd}
\end{thm}
\begin{pf}
Define a sesquilinear form on the algebraic tensor product $\cA\odot H$ as
\[\<\sum_ja_j\otimes\xi_j,\sum_ib_i\otimes\eta_i\>:=\sum_{i,j}\<\f(b_i^*a_j)\xi_j,\eta_i\>.\]
It is positive since
\[\sum_{i,j}\<a_i^*a_j\xi_j,\xi_i\>=\sum_{i,j}\<a_j\xi_j,a_i\xi_i\>=\|\sum_ia_i\xi_i\|^2\ge0\]
implies
\[\<\sum_ja_j\otimes\xi_j,\sum_ia_i\otimes\xi_i\>=\sum_{i,j}\<\f(a_i^*a_j)\xi_j,\xi_i\>\ge0.\]
Taking quotient by the left kernel $N$ and completion, we obtain a hilbert space $\hat H:=(\cA\odot H/N)^-$.

Define $\pi:\cA\to B(\hat H)$ such that
\[\pi(a)(b\otimes\xi+N):=ab\otimes+N,\]
and define $V:H\to\hat H$ such that
\[V\xi:=1_\cA\otimes\xi+N.\]
Then for any $\xi,\eta\in H$,
\[\<V^*\pi(a)V\xi,\eta\>=\<\pi(a)(1_\cA\otimes\xi+N),1_\cA\otimes\xi+N\>=\<a_\cA\otimes\xi+N,1_\cA\otimes\xi+N\>=\<\f(a)\xi,\eta\>.\qedhere\]
\end{pf}

\begin{rmk}\,
\begin{parts}
\item If $\f$ is unital, then $V$ is an isometry since $V^*V=V^*\pi(1)V=\f(1)=1$.
\item If $\f$ is unital and $H=\C$, then it is just the GNS-construction with the cyclic vector $V1_\C$.
\item If $\f:\cA\to\cB$ is c.p., then by embedding $\cB$ into $B(H)$ and applying the Stinespring dilation,
\[\|\f(a)\|=\|V^*\pi(a)V\|\le\|V\|\|a\|\|V\|=\|a\|\|V^*V\|=\|a\|\|\f(1)\|\]
implies $\|\f\|\le\|\f(1)\|$, hence $\|\f\|=\|\f(1)\|$.
\item It has a physical meaning: a unital completely positive map is called quantum channel or quantum operation in quantum information theory. They are interpreted as an evolution in open quantum system, and taking $\hat H$ means introducing a closed ambient system in which unitary evolution occurs.
\end{parts}
\end{rmk}

\begin{thm}[Completely positive maps for matrix algebras]
Let $\cA$ be a C$^*$-algebra.
Let $e_i\in\ell_2^n$ be standard orthonormal basis and let $e_{ij}=e_i\otimes e_j=|e_i\>\<e_j|\in M_n(\C)$ be unit matrix elements.
\begin{parts}
\item
There is a 1-1 correspondence
\[\mathrm{CP}(M_n(\C),\cA)\to M_n(\cA)_+:\psi\mapsto[\psi(e_{ij})].\]
\item
Let $\cA$ be unital.
There is a 1-1 correspondence
\[\mathrm{CP}(\cA,M_n(\C))\to M_n(\cA)^*_+:\f\mapsto(\hat\f:[a_{ij}]\mapsto\sum_{i,j}\<\f(a_{ij})e_j,e_i\>).\]
\end{parts}
\end{thm}
\begin{pf}
(a)
Fix $\cA\to B(H)$ a faithful representation and just write $\cA\subset B(H)$.

Suppose $\psi:M_n(\C)\to\cA$ is a c.p.~map.
Identify $M_n(\C)=B(\ell_2^n)$.
Since $[e_{ij}]\in M_n(B(\ell_2^n))_+$ is positive because
\[\sum_{i,j}\<e_{ij}\xi_j,\xi_i\>=\sum_{i,j}\<e_j,\xi_j\>\<\xi_i,e_i\>=|\sum_i\<e_i,\xi_i\>|^2\ge0,\qquad\forall[\xi_i]\in(\ell_2^n)^n,\]
it follows that $[\psi(e_{ij})]\in M_n(\cA)_+$ by the complete positivity of $\psi$.

Conversely, let $[\psi(e_{ij})]=[\sum_kb_{ki}^*b_{kj}]\in M_n(B(H))_+$,
For $T=[t_{ij}]\in M_n(\C)$ and $\xi,\eta\in H$, write
\begin{align*}
\<\psi(T)\xi,\eta\>
&=t_{ij}\<\psi(e_{ij})\xi,\eta\>\\
&=t_{ij}\<b_{kj}\xi,b_{ki}\eta\>\\
&=t_{ij}\delta_{kl}\<b_{lj}\xi,b_{ki}\eta\>\\
&=\<Te_j,e_i\>\<e_l,e_k\>\<b_{lj}\xi,b_{ki}\eta\>\\
&=\<(T\otimes1\otimes1)(e_j\otimes e_l\otimes(b_{lj}\xi)),(e_i\otimes e_k\otimes(b_{ki}\eta))\>.
\end{align*}
The summation symols are omitted in each row.
Then, if we define
\[V:H\to\ell_2^n\otimes\ell_2^n\otimes H:\xi\mapsto\sum_{i,k}e_i\otimes e_k\otimes(b_{ki}\eta),\]
we have an expression
\[\<\psi(T)\xi,\eta\>=\<V^*(T\otimes1\otimes1)V\xi,\eta\>,\]
which implies that $\psi$ is c.p.~because $T\mapsto T\otimes1_{\ell_2^n}\otimes1_H$ is a $*$-homomorphism.

(b)
Suppose $\f:\cA\to M_n(\C)$ is a c.p.~map.
Then, $\hat\f$ is positive since $[a_{ij}]\in M_n(\cA)_+$ implies
\[\hat\f([a_{ij}])=\sum_{i,j}\<\f(a_{ij})e_j,e_i\>\ge0.\]

Conversely, let $\hat\f\in M_n(\cA)^*_+$.
By the GNS-construction, we have a cyclic representation $\pi:M_n(\cA)\to B(H)$ with a cyclic vector $\psi\in H$ such that
\[\hat\f([a_{ij}])=\<\pi([a_{ij}])\psi,\psi\>.\]
For $\xi=\sum_j\xi_je_j,\eta=\sum_i\eta_ie_i\in\ell_2^n$, write
\begin{align*}
\<\f(a)\xi,\eta\>
&=\sum_{i,j}\<\f(a)\xi_je_j,\eta_ie_i\>
=\sum_{i,j}\<\f(\bar{\eta_i}a\xi_j)e_j,e_i\>\\
&=\hat\f([\bar{\eta_i}a\xi_j])
=\<\pi([\bar{\eta_i}a\xi_j])\psi,\psi\>
=\<\pi([\delta_{ij}\eta_i1_\cA]^*[a][\delta_{ij}\xi_j1_\cA])\psi,\psi\>\\
&=\<\pi([a])\pi([\delta_{ij}\xi_j1_\cA])\psi,\pi([\delta_{ij}\eta_i1_\cA])\psi\>.
\end{align*}
If we define
\[V:\ell_2^n\to H:\xi\mapsto\pi([\delta_{ij}\xi_j1_\cA])\psi,\]
then
\[\<\f(a)\xi,\eta\>=\<V^*\pi([a])V\xi,\eta\>,\]
so $\f$ is c.p.~since $\cA\to M_n(\cA):a\mapsto[a]$ is a $*$-homomorphism.
\end{pf}

\begin{thm}[Arveson extension]
Let $\cB\subset\cA$ be C$^*$-algebras such that $1_\cA\in\cB$.
Then, every c.p.~map $\f:\cB\to B(H)$ has an norm-preserving c.p.~extension $\tilde\f:\cA\to B(H)$, i.e. $\|\tilde\f\|=\|\f\|$.
\end{thm}
\begin{pf}
Let $p_\alpha$ be the net of projections of finite rank $n_\alpha$ in $B(H)$ with the image $V_\alpha$, which strongly converges to $\id_H$.
Fix $\alpha$ temporarily and let $\f_\alpha:=p_\alpha\f|_{V_\alpha}:\cB\to B(V_\alpha)$.
Choosing an any orthonormal basis of each $V_\alpha$, we can rewrite as $\f_\alpha:\cB\to M_{n_\alpha}(\C)$.
By the above theorem, we have the associated linear functional $\hat\f_\alpha\in M_{n_\alpha}(\cB)$.
Then, the Hahn-Banach extension provides an extension $(\hat\f_\alpha)^\sim\in M_{n_\alpha}(\cA)$, and we can define $\tilde\f_\alpha:\cA\to M_{n_\alpha}(\C)$ as the associated completely positive map.
Via the identification $B(V_\alpha)=M_{n_\alpha}(\C)$ we used to write $\f_\alpha:\cB\to M_{n_\alpha}(\C)$, we have $\tilde\f_\alpha:\cA\to B(V_\alpha)$.
We can check $\tilde\f_\alpha$ actually extends $\f_\alpha$, i.e. $\tilde\f_\alpha(b)=\f_\alpha(b)$ for $b\in\cB$, by putting $[b\delta_{ik}\delta_{jl}]_{i,j}\in M_{n_\alpha}(\cB)$ and comparing matrix components for each $k,l$.

Since $\|\tilde\f_\alpha\|=\|\tilde\f_\alpha(1)\|=\|\f_\alpha(1)\|=\|\f_\alpha\|\le\|\f\|$, the net $\tilde\f_\alpha$ is bounded in $B(\cA,B(H))$.
The norm-closed unit ball is compact in the point-$\sigma$-weak topology $\sigma(B(\cA,B(H)),\cA\odot L^1(H))$ because it is coarser than the weak$^*$ topology $\sigma(B(\cA,B(H)),\cA\hat\otimes_\pi L^1(H))$.
By taking a convergent subnet, we have a limit point $\tilde\f:\cA\to B(H)$.
It is easily seen to be completely positive and extend $\f$, and satisfies $\|\f\|=\|\f(1)\|=\|\tilde\f(1)\|=\|\tilde\f\|$.
\end{pf}



\subsection{Enveloping von Neumann algebras}

\begin{thm}[Sherman-Takeda]
Let $\cA$ be a C$^*$-algebra and $\pi:\cA\to B(H)$ a faithful representation.
Here we can obtain an linear map $\tilde\pi:\cA^{**}\to\pi(\cA)''$ by taking bitranspose for $\pi:\cA\to(\pi(\cA)'',\sigma w)$.
\begin{parts}
\item $\tilde\pi$ is an isometric isomorhpism (w.r.t. norms), and is an homeomorphism (w.r.t. weak$^*$-topologies)
\item $\cA^{**}$ enjoys a universal property in the sense that for every $*$-homomorphism $\f:\cA\to\cM$ to a von Neumann algebra $\cM$, there exists a unique $\sigma$-weakly continuous extension $\tilde\f:\cA^{**}\to\cM$ of $\f$.
\end{parts}
We will always see the bidual $\cA^{**}$ as a von Neumann algebra.
\end{thm}
\begin{pf}
(a)
Consider
\[\pi:\cA\to(\pi(\cA)'',\sigma w),\qquad\pi^*:\pi(\cA)''_*\to\cA^*,\qquad\tilde\pi:=\pi^{**}:\cA^{**}\to\pi(\cA)'',\]
where $\pi(\cA)''_*$ denotes the set of $\sigma$-weakly continuous(=normal) linear functionals on $\pi(\cA)''$.
Note that $\pi$ is isometric and has dense range.
It implies that $\pi^*$ is surjective and injective.
In fact, $\pi^*$ is isometric because for $l\in\pi(\cA)''_*$ we have by the density that
\[\|\pi^*(l)\|=\sup_{\substack{\|a\|=1\\a\in\cA}}|l(\pi(a))|=\sup_{\substack{\|b\|=1\\b\in\pi(\cA)''}}|l(b)|=\|l\|.\]
Then, the claim for $\pi^{**}$ is now clear.

(b)
We can define $\tilde\f$ as the bitranspose of $\f:\cA\to(\cM,\sigma w)$ as in the part (a), and it is a unique extension because $\cA$ is $\sigma$-weakly dense in $\cA^{**}$.
\end{pf}


\begin{thm}[Tomiyama]
Let $\cB\subset\cA$ be C$^*$-algebras.
Let $\f:\cA\to\cB$ be a \emph{conditional expectation}, i.e. a contractive idempotent linear map.
\begin{parts}
\item $\f$ is $B$-bimodule map.
\item $\f$ is completely positive.
\end{parts}
\end{thm}
\begin{pf}
Since each conclusion of (a) and (b) still holds for restriction, we may assume $\cA$ and $\cB$ are von Neumann algebras by thinking of the bitranspose $\f^{**}:\cA^{**}\to\cB^{**}$.

(a)
Since the linear span of projections is $\sigma$-weakly dense in a von Neumann algebra, we are enough to show $p\f(a)=\f(pa)$ and $\f(ap)=\f(a)p$ for any projection $p\in\cB$.

Let $p\in\cB$ be a projection and let $a\in\cA$.
Note that we have
\[p\f(a)=pp\f(a)=p\f(p\f(a))\]
and
\[(a-pa)^*(p\f(a-pa))=(p\f(a-pa))^*(a-pa)=0.\]
Then,
\begin{align*}
(1+t)^2\|p\f(a-pa)\|^2
&=\|p\f(a-pa)+tp\f(a-pa)\|^2\\
&=\|p\f((a-pa)+tp\f(a-pa))\|^2\\
&\le\|(a-pa)+tp\f(a-pa)\|^2\\
&=\|a-pa\|^2+t^2\|p\f(a-pa)\|^2
\end{align*}
implies $p\f(a-pa)=0$ by letting $t\to\infty$.
Putting $1_\cB-p$ and $1_\cB$ instead of $p$, we obtain $(1_\cB-p)\f(a-1_\cB a+pa)=0$ and $\f(a-1_\cB a)=0$, so
\[p\f(a)=p\f(pa)=\f(pa).\]
Similarly, we can show $\f(a-ap)p=0$ and $\f(ap)(1-p)=0$, we are done.

(b)
Let $[a_{ij}]\in M_n(\cA)_+$.
Let $\pi:\cB\to B(H)$ be a cyclic representation with a cyclic vector $\psi$.
Then, $[\xi_i]\in H^n$ can be replaced to $[\pi(b_i)\psi]$, so we can check the positivity of inflations $\f_n$ as
\[\sum_{i,j}\<\pi(\f(a_{ij}))\pi(b_j)\psi,\pi(b_i)\psi\>=\<\pi(\f(\sum_{i,j}b_i^*a_{ij}b_j))\psi,\psi\>\ge0,\]
because it follows $\sum_{i,j}b_i^*a_{ij}b_j\ge0$ by the positivity of $a_{ij}$ from
\[\<\pi_\cA(\sum_{i,j}b_i^*a_{ij}b_j)\xi,\xi\>=\sum_{i,j}\<\pi_\cA(a_{ij})\pi_\cA(b_j)\xi,\pi_\cA(b_i)\xi\>\ge0,\]
where $\pi_\cA$ is any representation of $\cA$.
\end{pf}

\begin{thm}[Sakai]
Suppose $\cA$ is a C$^*$-algebra which admits a predual $F$.
\begin{parts}
\item There is an injective $*$-homomorphism $\pi:\cA\to\cA^{**}$ with weakly$^*$ closed image.
\item $\pi$ is a topological embedding w.r.t. $\sigma(\cA,F)$ and $\sigma(\cA^{**},\cA^*)$.
\item The predual $F$ is unique in $\cA^*$.
\end{parts}
In particular, there is a faithful representation $\cA\to B(H)$ whose image is ($\sigma$-)weakly closed.
\end{thm}
\begin{pf}
(a)
By taking the adjoint for the inclusion $i:F\hookrightarrow\cA^*$, we have a conditional expectation $\e:\cA^{**}\twoheadrightarrow\cA$.
Its kernel is a $\cA$-bimodule, and by the $\sigma$-weak density of $\cA$ in $\cA^{**}$ and the continuity of $\e$ between weak$^*$ topologies, so it is in fact a $\cA^{**}$-bimodule, which means it is a $\sigma$-weakly closed ideal of $\cA^{**}$.
Thus we have a central projection $z\in\cA^{**}$ such that $\ker\e=(1-z)\cA^{**}$.

Define $\pi:\cA\to\cA^{**}$ such that $\pi(a):=za$.
It is clearly a $*$-homomorphism.
The injectivity follows from $a=\e(a)=\e(za)$ for $a\in\cA$.
The image is weakly$^*$ closed because $\e(x-\e(x))=0$ implies $z(x-\e(x))=0$ for $x\in\cA^{**}$ so that $z\cA^{**}=z\cA$.

(b)
Since $\<a,f\>=\<\e(za),f\>=\<za,f\>$ for $a\in\cA$ and $f\in F$, in which the second equality holds by the definition of $\e$, it is enough to show $\sigma(z\cA,\cA^*)=\sigma(z\cA,F)$.

For $l\in\cA^*$, we claim there exists $f$ such that $\<za,l\>=\<za,f\>$.
Define $\tilde l\in\cA^*$ such that $\<x,\tilde l\>:=\<zx,l\>$ for $x\in\cA^{**}$.
Then, $\<zx,l\>=\<z^2x,l\>=\<zx,\tilde l\>$ for $x\in\cA^{**}$.
Suppose $\tilde l\notin F$.
Because $F$ is closed in $\cA^*$, there is $x\in\cA^{**}$ such that $\<x,\tilde l\>\ne0$ and $\<x,f\>=0$ for all $f\in F$ by the Hahn-Banach extension.
Then, $0=\<x,f\>=\<x,i(f)\>=\<\e(x),f\>$ implies $\e(x)=0$ so that $zx=0$, which leads a contradiction $\<x,\tilde l\>=\<zx,l\>=0$, so we have $\tilde l\in F$.

(c)
If closed subspaces $F_1$ and $F_2$ of $\cA^*$ are preduals of $\cA$, then $\sigma(\cA,F_1)=\sigma(\cA,F_2)$ by the part (b).
If $l\in F_1$, which is obviously continuous on $\sigma(\cA,F_1)$, and the continuity in $\sigma(\cA,F_2)$ implies that $l$ is contained in a linear span of some finitely many elements of $F_2$, hence $F_1\subset F_2$.
\end{pf}



\newpage
\section{May 12}

\subsection{Nuclear maps}

\begin{defn}
A linear map $\theta:\cA\to\cB$ between C$^*$-algebras is called \emph{nuclear} if it is a limit of finite-rank contractive c.p.(c.c.p.)~maps in the point-norm topology.
Equivalently, by the following lemma, there is a net of pairs of c.c.p.~maps $\f_\alpha:\cA\to M_{n_\alpha}(\C)$ and $\psi_\alpha:M_{n_\alpha}(\C)\to\cB$ such that $\|\theta(a)-\psi_\alpha\circ\f_\alpha(a)\|\to0$ for each $a\in\cA$.

If $\cB$ is a con Neumann algebra, $\theta$ is called \emph{weakly nuclear} if it is a limit of finite-rank c.c.p. maps in the point-$\sigma$-weak topology.
\end{defn}

\begin{lem}
A c.c.p.~map $\theta:\cA\to\cB$ between C$^*$-algebras is of finite-rank iff there are c.c.p.~maps $\f:\cA\to M_n(\C)$ and $\psi:M_n(\C)\to\cB$ for some $n$ such that $\theta=\psi\circ\f$.
In Brown-Ozawa, a finite-rank c.c.p.~map is called a factorable map.
\end{lem}
\begin{pf}
($\Leftarrow$) Clear.
($\Rightarrow$)
By the structure theorem of finite-dimensional C$^*$-algebras, we have $\im\theta\cong\bigoplus_{i=1}^mM_{n_i}(\C)$,
so for $n=\sum_{i=1}^mn_i$ there is a unital embedding $\im\theta\hookrightarrow M_n(\C)$ and conditional expectation $M_n(\C)\to\im\theta:T\mapsto\sum_{i=1}^mP_iTP_i$, where $P_i$ denotes the projection on the image of $M_{n_i}(\C)$.
Now we are done.
(In fact, such a conditional expectation also exists for unital subalgebras beetween von Neumann algebras.)
\end{pf}


\begin{prop}[Local property]
Let $\theta:\cA\to\cB$ be a linear map between C$^*$-algebras.
If the restriction of $\theta$ on any finite-dimensional subspace of $\cA$ is nuclear, then $\theta$ is nuclear.
\end{prop}
\begin{pf}

\end{pf}


\begin{prop}[Weak approximations]
Let $\cA$ and $\cB$ be C$^*$-algebras, and $\cM\subset B(H)$ a von Neumann algebra.
\begin{parts}
\item $\theta:\cA\to\cB$ is nuclear if there is a net $\cA\xrightarrow{\f_\alpha}M_{n_\alpha}(\C)\xrightarrow{\psi_\alpha}\cB$ such that
\[\lim_\alpha\<\theta(a)-\psi_\alpha\circ\f_\alpha(a),l\>=0\qquad a\in\cA,\ l\in\cB^*.\]
\item $\theta:\cA\to\cM$ is weakly nuclear if there is a net $\cA\xrightarrow{\f_\alpha}M_{n_\alpha}(\C)\xrightarrow{\psi_\alpha}\cM$ such that
\[\lim_\alpha\<(\theta(a)-\psi_\alpha\circ\f_\alpha(a))\xi,\xi\>=0\qquad a\in\cA,\ \xi\in H.\]
\end{parts}
\end{prop}

\begin{pf}
(a)
By applying the Hahn-Banach extension for each $a\in\cA$, we can show the closures of a convex set is same with respect to the point-norm topology and the point-$\sigma(\cB,\cB^*)$-topology.
Thus it suffices to show that the set of finite-rank c.c.p.~maps is convex.

Let $\cA\xrightarrow{\psi_i}M_{n_i}(\C)\xrightarrow{\f_i}\cB$ be c.c.p.~maps for $i\in\{0,1\}$.
Then, we have a diagram
\begin{cd}
\cA \ar{rrr}{(1-t)\psi_0\circ\f_0+t\psi_1\circ\f_1}\ar{d}&&&\cB\\
\cA\oplus\cA\ar[swap]{r}{\f_0\oplus\f_1}&M_{n_0}(\C)\oplus M_{n_1}(\C)\ar[swap]{rr}{((1-t)\psi_0)\oplus(t\psi_1)}&&\cB\oplus\cB\ar{u}
\end{cd}
which is commutative, so we are done.

(b)
Fix $a\in\cA$.
Note that are net is bounded.
Since the unit ball is compact in $\sigma$-weak topology and hence in the weak operator topology, we are enough to verify the convergence of $\psi_\alpha\circ\f_\alpha$ in the weak operator topology.
Using the polarization identity, the claim holds.
\end{pf}


nonunital technicalities



\subsection{Examples of nuclear C$^*$-algebras}



C$^*$-subalgebra of a nuclear C$^*$-algebra may not be nuclear.
C$^*$-subalgebra of a exact C$^*$-algebra is exact.
injective limit of nuclear C$^*$-algebras is nuclear.
$M_n(\cA)$ is nuclear if $\cA$ is nuclear.

\begin{thm}[Effros-Lance]
If $\cA^{**}$ is semidiscrete, then $\cA$ is nuclear.
(The converse also holds)
\end{thm}
\begin{pf}
Since the set of finite-rank c.c.p.~maps is convex, and since the closures of a convex set are same in the norm and weak topologies on a Banach space, 
\end{pf}

\begin{thm}
An abelian C$^*$-algebra is nuclear.
\end{thm}



\newpage
\section{May 22}

\end{document}




熊谷 山崎 318 数理物理
倉橋 新井 421 基礎論
小松 木田 418 에르고딕
近藤 伊藤 402 シュレディンガー
佐藤 317 PDE Hamilton-Jacobi
佐藤 318 量子群
高橋 新井 421
田渕 会田

友永 井山 357 表現論
永井 伊藤 代数幾何
中浦 新井 570 モデル
中村 宮本 402
星屋 伊藤 557 pで
宮内 小林 406
Liu 志甫 326 数論幾何
渡辺 志甫 557 数論幾何
会沢 松井 261

原田 山崎 457



21
