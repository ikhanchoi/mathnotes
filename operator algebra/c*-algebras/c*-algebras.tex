\documentclass{../../large}
\usepackage{../../ikhanchoi}


\begin{document}
\title{C$^*$-Algebras}
\author{Ikhan Choi}
\maketitle
\tableofcontents

\part{C$^*$-algebras}
\chapter{Basic concepts}

\section{Multiplier algebra}

\begin{prb}[Multiplier algebra]
Let $\cA$ be a C$^*$-algebra.
A \emph{double centralizer} of $\cA$ is a pair $(L,R)$ of bounded linear maps on $\cA$ such that $aL(b)=R(a)b$ for all $a,b\in\cA$.
The \emph{multiplier algebra} $M(\cA)$ of $\cA$ is defined to be the set of all double centralizers of $\cA$.
\end{prb}

\begin{prb}[Essential ideals]
\begin{parts}
\item Hilbert C$^*$-module description
\end{parts}
\end{prb}

\begin{prb}[Examples of multiplier algebras]
\begin{parts}
\item $M(K(H))\cong B(H)$.
\item $M(C_0(\Omega))\cong C_b(\Omega)$.
\end{parts}
\end{prb}
\begin{pf}
(a)

(b)
First we claim $C_0(\Omega)$ is an essential ideal of $C_b(\Omega)$.
Since $C_b(\Omega)\cong C(\beta\Omega)$, and since closed ideals of $C(\beta\Omega)$ are corresponded to open subsets of $\beta\Omega$, $C_0(\Omega)\cap J$ is not trivial for every closed ideal $J$ of $C_b(\Omega)$.

Now we have an injective $^*$-homomorphism $C_b(\Omega)\to M(C_0(\Omega))$, for which we want to show the surjectivity.
Let $g\in M(C_0(\Omega))^+$.
\end{pf}

\begin{prb}[Strict topology]
\end{prb}



\begin{prb}[Hereditary C$^*$-subalgebra]
state extension, representation extension
\end{prb}




\section*{Exercises}
\begin{prb}
Let $\cB$ be a hereditary C$^*$-subalgebra of a C$^*$-algebra $\cA$.
Let $a\in\cA^+$.
If for any $\e>0$ there is $b\in\cB^+$ such that $a-\e\le b$, then $a\in B^+$.
\end{prb}
\begin{pf}
To catch the idea, suppose $\cA$ is abelian.
We want to approximate $a$ by the elements of $\cB$ in norm.
To do this, for each $\e>0$, we want to construct $b'\in\cB^+$ such that $a-\e\le b'\le a+\e$ using $b$.
Taking $b'=\min\{a,b\}$ is impossible in non-abelian case, but we can put $b'=\frac a{b+\e}b$.
For a simpler proof, $b'=(\frac{\sqrt{ab}}{\sqrt b+\sqrt\e})^2$ is a better choice.

Define
\[b':=\frac{\sqrt b}{\sqrt b+\sqrt\e}a\frac{\sqrt b}{\sqrt b+\sqrt\e}.\]
Then,
\[\|\sqrt a-\sqrt a\frac{\sqrt b}{\sqrt b+\sqrt\e}\|^2=\|\frac{\sqrt\e}{\sqrt b+\sqrt\e}a\frac{\sqrt\e}{\sqrt b+\sqrt\e}\|\le\e\]
implies
\[\lim_{\e\to0}b'=\lim_{\e\to0}\frac{\sqrt b}{\sqrt b+\sqrt\e}\sqrt a\cdot\sqrt a\frac{\sqrt b}{\sqrt b+\sqrt\e}=\sqrt a\cdot\sqrt a=a.\]
\end{pf}


\chapter{Completely positive maps}

\section{Operator systems and spaces}

\section{Dilation theorems}

\section{Extension theorems}

Arveson
Trick




\chapter{Tensor products}

\section{Minimal tensor product}
spatiality
Takesaki theorem

\section{Maximal tensor product}

universal property
restriction theorem
c.c.p.~tensor product

\section{Nuclear C$^*$-algebras}
finite dimensional, abelian, some constructions




a separable C$^*$-algebra is nuclear if and only if every factor representation is hyperfinite.





\part{Approximation properties}

\section{Finite dimensional approximation}
nuclear and exact C$^*$-algebras
\section{Voiculescu theorem}
\section{Quasidiagonal C$^*$-algebras}


\part{Constructions}




\part{Operator K-theory}
\chapter{Brown-Douglas-Fillmore theory}
\begin{prb}[Haagerup property]
\end{prb}

Baum-Connes conjecture
Non-commutative geometry
Elliott theorem




\section{Approximately finite algebras}
Elliott conjecture: amenable simple separable C$^*$-algerbas are classified by K-theory.








\end{document}