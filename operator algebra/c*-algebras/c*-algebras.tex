\documentclass{../../large}
\usepackage{../../ikhanchoi}

\DeclareMathOperator{\Mor}{Mor}

\begin{document}
\title{C$^*$-Algebras}
\author{Ikhan Choi}
\maketitle
\tableofcontents

\part{C$^*$-algebras}
\chapter{Basic concepts}



\begin{prb}[Hereditary C$^*$-subalgebra]
state extension, representation extension(not ideal?)
\end{prb}

conditional expectation


\section*{Exercises}
\begin{prb}
Let $B$ be a hereditary C$^*$-subalgebra of a C$^*$-algebra $A$.
Let $a\in A_+$.
If for any $\e>0$ there is $b\in B_+$ such that $a-\e\le b$, then $a\in B_+$.
\end{prb}
\begin{pf}
To catch the idea, suppose $A$ is abelian.
We want to approximate $a$ by the elements of $B$ in norm.
To do this, for each $\e>0$, we want to construct $b'\in B_+$ such that $a-\e\le b'\le a+\e$ using $b$.
Taking $b'=\min\{a,b\}$ is impossible in non-abelian case, but we can put $b'=\frac a{b+\e}b$.
For a simpler proof, $b'=(\frac{\sqrt{ab}}{\sqrt b+\sqrt\e})^2$ is a better choice.

Define
\[b':=\frac{\sqrt b}{\sqrt b+\sqrt\e}a\frac{\sqrt b}{\sqrt b+\sqrt\e}.\]
Then,
\[\|\sqrt a-\sqrt a\frac{\sqrt b}{\sqrt b+\sqrt\e}\|^2=\|\frac{\sqrt\e}{\sqrt b+\sqrt\e}a\frac{\sqrt\e}{\sqrt b+\sqrt\e}\|\le\e\]
implies
\[\lim_{\e\to0}b'=\lim_{\e\to0}\frac{\sqrt b}{\sqrt b+\sqrt\e}\sqrt a\cdot\sqrt a\frac{\sqrt b}{\sqrt b+\sqrt\e}=\sqrt a\cdot\sqrt a=a.\]
\end{pf}





\chapter{Completely positive maps}
\section{Operator systems and spaces}
\section{Dilation theorems}
\section{Extension theorems}
Arveson
Trick




\chapter{Tensor products}
\section{Minimal tensor product}
spatiality
Takesaki theorem

\section{Maximal tensor product}
universal property
restriction theorem
c.c.p.~tensor product

\section{Nuclear and exact C$^*$-algebras}
finite dimensional, abelian, some constructions




a separable C$^*$-algebra is nuclear if and only if every factor representation is hyperfinite.



\section{Voiculescu theorem}
\section{Quasidiagonal C$^*$-algebras}
\section{AF-embeddability}



\chapter{Hilbert C$^*$-modules}


right $A$ convention: to make it commute with the action by adjointable operators.

examples
A itself, direct sum, tensoring with hilbert space, localization

C$^*$-correspondence

\section{Multiplier algebras}
\begin{prb}[Multiplier algebra]
Let $A$ be a C$^*$-algebra.
A \emph{double centralizer} of $A$ is a pair $(L,R)$ of bounded linear maps on $A$ such that $aL(b)=R(a)b$ for all $a,b\in A$.
The \emph{multiplier algebra} $M(A)$ of $A$ is defined to be the set of all double centralizers of $A$.
There is another characterization $M(A):=L_A(A)$, the set of adjointable operators to itself.
\end{prb}
\begin{prb}[Cohen factorization theorem]
\end{prb}
\begin{prb}[Strict topology]
\end{prb}
\begin{prb}[Essential ideals]
\begin{parts}
\item Hilbert C$^*$-module description
\end{parts}
\end{prb}


\begin{prb}[Examples of multiplier algebras]
\begin{parts}
\item $M(K(H))\cong B(H)$.
\item $M(C_0(\Omega))\cong C_b(\Omega)$.
\end{parts}
\end{prb}
\begin{pf}
(a)

(b)
First we claim $C_0(\Omega)$ is an essential ideal of $C_b(\Omega)$.
Since $C_b(\Omega)\cong C(\beta\Omega)$, and since closed ideals of $C(\beta\Omega)$ are corresponded to open subsets of $\beta\Omega$, $C_0(\Omega)\cap J$ is not trivial for every closed ideal $J$ of $C_b(\Omega)$.

Now we have an injective $*$-homomorphism $C_b(\Omega)\to M(C_0(\Omega))$, for which we want to show the surjectivity.
Let $g\in M(C_0(\Omega))_+$.
\end{pf}



Induced representations and Morita equivalence
KK-theory
C$^*$-algebraic quantum groups

JenTho KK


\chapter{Operator K-theory}
\section{Construction of K-theory}

\begin{prb}[Homotopy of $*$-homomorphisms]
Let $A,B$ be C$^*$-algebras.
Two $*$-homomorphisms in $\Mor(A,B)$ are said to be \emph{homotopic} if they are connected by a path in $\Mor(A,B)$ that is continuous with the point-norm topology.
\begin{parts}
\item For pointed compact Hausdorff spaces $(X,x_0),(Y,y_0)$, two pointed maps $\f_0,\f_1:X\to Y$ are homotopic if and only if $\f_0^*,\f_1^*:C_0(Y\setminus\{y_0\})\to C_0(X\setminus\{x_0\})$ are homotopic.
\end{parts}
\end{prb}
\begin{pf}
(a)
Suppose $\f_0$ and $\f_1$ are connected by a homotopy $\f_t$.
Fixing $g\in C_0(Y)$ and $t_0\in I$, we want to show
\[\lim_{t\to t_0}\sup_{x\in X}|g(\f_t(x))-g(\f_{t_0}(x))|=0.\]
Since the function $g$ is uniformly continuous, with respect to an arbitrarily chosen uniformity on $Y$, so that there is an entourage $E\subset Y\times Y$ such that $(y,y')\in E\circ E$ implies $|g(y)-g(y')|<\e$.
Using compactness we have a finite sequence $(y_i)_{i=1}^n\subset Y$ such that for every $y$ there is $y_i$ satisfying $(y,y')\in E$.
Then, $\f^{-1}(E[y_i])$ is a finite open cover of $X\times I$, so we have $\delta$ such that $|t-t_0|<\delta$ implies for any $x\in X$ the existence of $i$ satisfying $(\f_t(x),y_i)\in E$ and $(\f_{t_0}(x),y_i)\in E$, which deduces the desired inequality.

Conversely, suppose $\f_0^*$ and $\f_1^*$ are connected by a homotopy $\f_t^*$.
By taking dual, we can induce $\f_t:X\to Y$ such that $g(\f_t(x))=(\f_t^*g)(x)$ for each $g\in C(Y)$ from $\f_t^*$ via the embedding $X\to M(X)$ by Dirac measures.
Let $V$ be an open neighborhood of $\f_{t_0}(x_0)$ and take $g\in C(Y)$ such that $g(\f_{t_0}(x_0))=1$ and $g(y)=0$ for $y\notin V$.
Now we have an open neighborhood $U$ of $x_0$ such that $x\in U$ implies $|(\f_{t_0}^*g)(x)-(\f_{t_0}^*g)(x_0)|<\frac12$.
Also we have $\delta>0$ such that $|t-t_0|<\delta$ implies $\|\f_t^*g-\f_{t_0}^*g\|<\frac12$.
Therefore, $(x,t)\in U\times(t_0-\delta,t_0+\delta)$ implies $g(\f_t(x))>0$, hence $\f_t(x)\in V$, which means $X\times I\to Y:(x,t)\mapsto\f_t(x)$ is continuous.
\end{pf}

We have $\tilde K^n(X,x_0)=K_n(C_0(X\setminus\{x_0\}))$ for a pointed compact Hausdorff space $X$.
Now then since the inclusion $\{x_0\}\to X$ induces the section so that
\[0\to K_0(C_0(X\setminus\{x_0\}))\to K_0(C(X))\to K_0(\{x_0\})\to0\]
splits, we have
\[K^0(X)=\tilde K^0(X,x_0)\oplus\Z=K_0(C_0(X\setminus\{x_0\}))\oplus K_0(\{x_0\})=K_0(C(X))\]
for a compact connected Hausdorff space $X$.
The additivity of $K_0$ and $K^0$ removes the connectedness condition.

\[K_0(\C)=\Z,\quad K_0(C_0(\R))=0,\quad K_1(C_0(\R))=K_0(C_0(\R^2))=\Z\]
\[K^0(*)=\Z,\quad K^0(S^1)=\Z,\quad K^1(S^1)=K^0(S^2)=\Z[x]/(x-1)^2\]




\section{Brown-Douglas-Fillmore theory}
\begin{prb}[Haagerup property]
\end{prb}

Baum-Connes conjecture
Non-commutative geometry
Elliott theorem




\section{Approximately finite algebras}
Elliott conjecture: amenable simple separable C$^*$-algerbas are classified by K-theory.
Brattelli diagram








\end{document}