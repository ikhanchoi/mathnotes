\documentclass{../../large}
\usepackage{../../ikhanchoi}


\begin{document}
\title{Von Neumann Algebras}
\author{Ikhan Choi}
\maketitle
\tableofcontents




\part{}

\chapter{Factor classifications}


\section{Factors and traces}


Every trace of factor is faithful

\begin{prb}
Normal states is a state in which the monotone convergence theorem holds.
Precisely, a state $\rho$ is \emph{normal} if a monotone net $a_\alpha$ strongly converges to $a$ then $\rho(a_\alpha)\to\rho(a)$.
\end{prb}


\section{}
\begin{prb}[Semi-finite traces]
Let $M$ be a von Neumann algebra and $\tau$ is a trace.
For a trace $\tau$
\begin{parts}
\item $\tau$ is semi-finite if and only if $x\in M^+$ has a net $x_\alpha\in L^1(M,\tau)^+$ such that $x_\alpha\uparrow x$ strongly.
\item Let $\tau$ be normal and faithful. Then, $\tau$ is semi-finite if and only if
\[\tau(x)=\sup\{\,\tau(y):y\le x,\ y\in L^1(M,\tau)^+\,\}\quad\text{ for }\quad x\in M^+.\]
\end{parts}
\end{prb}

\section{}


Direct integral of factors.

Type I factors.
It possess a minimal projection.
It is isomorphic to the whole $B(H)$ for some Hilbert space.
Therefore, it is classified by the cardinality of $H$.

Type II factors.
No minimal projection, but there are non-zero finite projections so that every projection can be ``halved'' by two Murray-von Neumann equivalent projections.

In type II$_1$ factors, the identity is a finite projection
Also, Murray and von Neumann showed there is a unique finite tracial state and the set of traces of projections is $[0,1]$.
Free probability theory attacks the free groups factors, which are type II$_1$.

In type II$_\infty$ factors
There is a unique semifinite tracial state up to rescaling and the set of traces of projections is $[0,\infty]$.

In type III factors no non-zero finite projections exists.
Classified the $\lambda\in[0,1]$ appeared in its Connes spectrum, they are denoted by III$_\lambda$.
Tomita-Takesaki theory.
It is represented as the crossed product of a type II$_\infty$ factor and $\R$.

Amenability, equivalently hyperfiniteness is a very nice condition in von Neumann algebra theory.
Group-measure space construction can construct them.
There are unique hyperfinite type II$_1$ and II$_\infty$ factors, and their property is well-known.
Fundamental groups of type II factors, discrete group theory, Kazhdan's property (T) are used.


Tensor product facctors such as Araki-Woods factors and Powers factors.

\section{Hyperfinite factors}

weight, trace, state.

finite trace$=$tracial state.

\begin{prb}[Uniformly hyperfinite algebras]
Let $\cA$ be a uniformly hyperfinite algebra.
\begin{parts}
\item Every matrix algebra admits a unique finite trace.
\item Every UHF algebra admits a unique finite trace.
\item Every hyperfinite 
\end{parts}
\end{prb}

\begin{prb}[Classification of UHF algebras]

\end{prb}




\chapter{Weight theory}





\chapter{}
\section{Connes' bicentralizer problem}



















\part{Subfactor theory}






The way how quantum systems are decomposed.
Quite combinatorial!
And has Galois analogy.

\begin{prb}[Jones index theorem]
A \emph{subfactor} of a factor $M$ is a factor $N$ containing $1_M$.
\end{prb}

Tensor categories and topological invariants of 3-folds.
Ergodic flows.



\end{document}