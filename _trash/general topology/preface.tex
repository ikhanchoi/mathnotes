%!TeX root=general topology.tex


\chapter*{Preface}

% essentially penetrating short definition
% relation to the other fields
%


One way to state the definition for general topology is the abstract study of topologies and topological spaces.
The word topology is used in two different contexts: analytic sense and geometric sense.
When we are talking the stories of doughnuts and coffee mugs, they are in fact involved in topology of geometric sense, which is also referred to as a branch of mathematics that studies constinuous structures of spaces such as manifolds or CW complexes.
In analysis, the topology is mentioned greatly unrelatedly to the doughnuts, but it refers to the minimal structure that is required in order to define concepts of limit and continuity.
More precisely, once a structure called ``topology'' is settled on a set, then we can expand basic analytic theories about limit and continuity.
Normed spaces are the first examples which possess a particularly nice topology.
With the topologies, we can describe formally whether a sequence converges or a function is continuous.
This book is interested in the latter issues as noted in the title of the book.

According to the usage of topologies, similarly as mentioned, there are two large branches of general topology; both contribute to build nice frameworks for the wide regions of mathematics.
One is for algebraic topology and studies the category of convenient spaces in which well-known constructions and computational tools are available, and the other is for abstract analysis.
In general topology focused on analysis, we are more concerned with the implications among individual topologies and special properties of them, rather than the global shapes of topological spaces.
For real analysis or functional analysis, general topology provides with extremely important viewpoints for recognizing the various convergence modes of functional sequences.
An interesting feature of general topology is that the basic topology in analysis is a preliminary of the abstract study of the spaces used in algebraic topology, hence everyone starts to learn it from analysis.

\iffalse
For generalization, we firstly come up with the following question: exactly what aspects of norms could have made one define the notion of convergence?
This leads the concept of neighborhood system, and unltimately, topology.
We may, furthermore, compare different topologies.
For example, the convergence changes even for a same sequence when judged in different topologies so that we can argue in what sense a sequence converges.
We can ask if a convergence implies another convergence.
Also, as a generalization of metric, we want to ask the essential difference between metric and topology.
The uniformity will answer the question.

Also, we are concerned with the properties of a given topology.
Some convergences that occur within various applications such as pointwise convergence of functional sequences with uncountable domain cannot be formulated with neither norms nor metrics.
They are in fact non-metrizable topologies.
\fi




The purpose of this book is to grasp a big picture and learn basic languages in order to establish frameworks for the next study of modern analysis such as harmonic analysis or functional analysis following after calculus topics, in a quite abstract viewpoints.
In particular, we mainly focus on finding admissible answers for the following questions:
\begin{itemize}
\item Why are topologies defined in that way? Is that a suitably optimized definition?
\item What can metric spaces or normed spaces do more than topological spaces?
\item What properties are needed to take and use sequences for describing topologies instead of general nets without any anxiety? 
\item What does the definition of compactness mean? What roles do they do in practical analysis?
\item What are the purposes of introduction of the compactness related concepts such as sequentially compact, $\sigma$-compact, or relatively compact spaces?
\item Why do locally compact Hausdorff spaces so frequently appear?
\item Why is the uniform convergence natural in a continuous function space?
\item What is the hidden meaning of complicated theorems of like Arzela-Ascoli or Stone-Weierstrass?
\end{itemize}
For the first in this book, the basic topological structures including metrics, topologies, and uniformities are introduced in Chapter 1.
Although many texts do not cover uniform spaces, they are greatly useful in studying nonmetrizable topologies.
In Chapter 2, we learn about continuity of functions and maps.
Continuous maps functionally connects two different topological spaces and allow us to compare them.
Homeomorphisms and some connectivity will be also covered.
Chapter 3 is dedicated to the deeper study of convergence of sequences or nets.
In Chapter 4, 5, and 6, we learn compact spaces, separability axioms, and continuous function spaces.

In this book, we are going to assume the reader is already familiar to the theory of normed spaces and elementary foundations of calculus including the epsilon-delta definitions.
For instance, we can require the reader to know what the uniform convergence is and that it can be regarded as just a convergence in the properly defined norm on a space of functions.

This book would not be a good choice for a standard course text relative to the other existing great books, because it is written to be helpful in self-teaching.
It has been tried to put convincing explanations at every newly defined concept and to cram supplementary stories that are not necessary, but they might not be really satisfied.
Nevertheless, I will be very satisfied only if just one of readers could enjoy math with this book.
