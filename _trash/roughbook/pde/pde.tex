\documentclass[a4paper]{article}
\usepackage{../../../ikhanchoi}
\usepackage[margin=3cm]{geometry}
\usepackage[T1]{fontenc}
\usepackage[bitstream-charter,cal]{mathdesign}
\linespread{1.15}

\def\AC{\mathrm{AC}}

\begin{document}
\tableofcontents

\newpage
\section{The Bartle-Graves theorem}
Let $E$ be a Banach space and $N$ a closed subspace.
For $\e>0$, there is a continuous homogeneous map $\rho:E/N\to E$ such that $\pi\rho(y)=y$ and $\|\rho(y)\|\le(1+\e)\|y\|$ for all $y\in E/N$.
\begin{pf}
We want to construct a continuous map $\psi:S_{E/N}\to E$ with $\|\psi(y)\|\le1+\e$ for all $y\in S_{E/N}$.
If then, $\rho$ can be made from $\psi$.

For each $y_0\in S_{E/N}$, choose $x_0\in\pi^{-1}(y_0)\cap B_{1+\e}$.
There is a neighborhood $V_{y_0}\subset S_{E/N}$ of $y_0$ such that $y\in V_{y_0}$ implies $x_0$ belongs to $(\pi^{-1}(y)\cap B_{1+\e})+U_{2^{-1}}$, which is convex.
With a locally finite subcover $V_{y_\alpha}$ and a partition of unity $\eta_\alpha(y)$, define $\psi_1(y)=\sum_\alpha\eta_\alpha(y)x_\alpha$.
Then, $\psi_1(y)\in(\pi^{-1}(y)\cap B_{1+\e})+U_{2^{-1}}$.

For $i\le2$, choose for each $y_0$ the element $x_0$ in $\pi^{-1}(y_0)\cap B_{1+\e}\cap(\psi_{i-1}(y_0)+U_{2^{-{i-1}}})$.
Then, we obtain
\[\psi_i(y)\in\Bigl(\pi^{-1}(y)\cap B_{1+\e}\cap(\psi_{i-1}(y_0)+U_{2^{-{i-1}}})\Bigr)+U_{2^{-i}}.\]
Therefore, $\|\psi_i(y)-\psi_{i-1}(y)\|<2^{-{i-2}}$, so it converges uniformly to $\psi$ such that $\psi(y)\in\pi^{-1}(y)\cap B_{1+\e}$.
\end{pf}


\newpage

\section{Nets of measurable functions}
\begin{prb}
\begin{parts}
\item 
\end{parts}
\end{prb}
If $f_\alpha$ is continuous, then $f$ is lower semi-continuous.
We use the inner regularity of the measure on the open set $f^{-1}(j2^{-n},\infty)$.
% https://math.stackexchange.com/questions/3656902/a-monotone-convergence-theorem-for-nets-in-reed-and-simon-vol-i

\newpage

\section{Potential from a source}

\begin{thm*}
Let $d\ge3$.
A distribution $u\in\cD'(\R^d)$ is a harmonic function on $\R^d\setminus\{0\}$ and vanishes at infinity if and only if there is a distribution $\rho\in\cD'(\R^d)$ such that $u=\Phi*\rho$ and $\supp(\rho)\subset\{0\}$, where $\Phi$ denotes the fundamental solution of Laplace's equation.
\end{thm*}
\begin{pf}
($\Rightarrow$)
Define a distribution $\rho$ by
\[\<\rho,\f\>:=-\<u,\Delta\f\>\]
for $\f\in C_c^\infty(\R^d)$.
In other words, $\rho=-\Delta u$ in distributional sense.
Then, $\rho$ has the support contained in $\{0\}$ because if $\f\in C_c^\infty(\R^d\setminus\{0\})$ then
\[\<\rho,\f\>=-\<u,\Delta \f\>=-\int u(x)\Delta\f(x)\,dx=-\int\Delta u(x)\f(x)\,dx=0.\]
Therefore, we only need to verify $u=\Phi*\rho$ to complete the proof.

Let $\f\in C_c^\infty(\R^d)$.
Be cautious that the argument
\[\<\Phi*\rho,\f\>=\<\rho,\Phi*\f\>=-\<u,\Delta(\Phi*\f)\>=\<u,\f\>\]
fails to provide a proof because the function $\Phi*\rho$ is not compactly supported so that we cannot deduce $\<\rho,\Phi*\f\>=-\<u,\Delta(\Phi*\f)\>$, and here we use the condition that $u$ vanishes at infinity to justify the equality.
Define a cutoff function $\chi\in C_c^\infty(\R^d)$ such that
\[\chi(x)=\begin{cases}1&\text{ if }|x|\le\frac54\\0&\text{ if }|x|\ge\frac74\end{cases}.\]
If we denote $\chi_r(x):=\chi(\frac xr)$, then we have
\[\<\rho,(\Phi\chi_r)*\f\>=-\<u,\Delta((\Phi\chi_r)*\f)\>\]
by the definition of $\rho$.
We have the limit of the left-hand side
\[\lim_{r\to\infty}\<\rho,(\Phi\chi_r)*\f\>=\<\rho,\Phi*\f\>\]
because
\begin{align*}
\supp((\Phi(1-\chi_r)*\f)&\subset\supp(\Phi(1-\chi_r))+\supp(\f)\\
&\subset\R^d\setminus B(0,2R)+\cl B(0,R)=\R^d\setminus B(0,R)
\end{align*}
for all $r>2R$ so that the supports of $\Phi(1-\chi_r)*\f$ and $\rho$ are disjoint, where we define $R:=\sup_{x\in\supp(\f)}|x|$.
However, the right-hand limit
\[-\lim_{r\to\infty}\<u,\Delta((\Phi\chi_r)*\f)\>=-\<u,\Delta(\Phi*\f)\>\]
is not a trivial result.

Assuming $\chi(x)=\chi(-x)$ without loss of generality, we have
\[\<u,\Delta(\Phi(1-\chi_r)*\f)\>=\<u*\Delta(\Phi(1-\chi_r)),\f\>.\]
Because
\[\Delta_y\Bigl[\Phi(x-y)\bigl(1-\chi(\tfrac{x-y}r)\bigr)\Bigr]=0\]
for $|y|<R$ and $x\in\supp(\f)$ if $r>2R$, we can write
\[\<u*\Delta(\Phi(1-\chi_r)),\f\>
=\int\f(x)\int u(y)\Delta_y\Bigl[\Phi(x-y)\bigl(1-\chi(\tfrac{x-y}r)\bigr)\Bigr]\,dy\,dx.\]
We compute
\begin{align*}
\Delta_y\Bigl[\Phi(x-y)\bigl(1&-\chi(\tfrac{x-y}r)\bigr)\Bigr]
=2\nabla\Phi(x-y)\cdot\frac1r\nabla\chi(\tfrac{x-y}r)-\Phi(x-y)\frac1{r^2}\Delta\chi(\tfrac{x-y}r)\\
&=-\frac2{\omega_d}\frac{x-y}{|x-y|^d}\cdot\frac1r\nabla\chi(\tfrac{x-y}r)
-\frac1{(d-2)\omega_d}\frac1{|x-y|^{d-2}}\frac1{r^2}\Delta\chi(\tfrac{x-y}r).
\end{align*}
Then, since $\frac54r\le|x-y|\le\frac74r$ if $\nabla\chi(\tfrac{x-y}r)\ne0$ and $\Delta\chi(\tfrac{x-y}r)\ne0$, we obtain
\[\Bigl|\Delta_y\Bigl[\Phi(x-y)\bigl(1-\chi(\tfrac{x-y}r)\bigr)\Bigr]\Bigr|\le C\frac1{r^d}\psi(\frac{x-y}r)\]
for some constant $C>0$, where
\[\psi(y):=|\nabla\chi(y)|+|\Delta\chi(y)|.\]
For each $x\in\supp(\f)$, since we have $\frac54r\le|x-y|\le\frac74r$ implies $r\le|y|\le2r$ if $r>4R$, it follows that
\begin{align*}
|\int u(y)\Delta_y\bigl[\Phi(x-y)\bigl(1-\chi(\tfrac{x-y}r)\bigr)\bigr]\,dy|
&\le C\int|u(y)\frac1{r^d}\psi(\frac{x-y}r)|\,dy\\
&\le C\max_{r\le|y|\le2r}u(y)
\end{align*}
converges to zero as $r\to\infty$.
By the bounded convergence theorem, we can deduce
\[\lim_{r\to\infty}\int\f(x)\int u(y)\Delta_y\Bigl[\Phi(x-y)\bigl(1-\chi(\tfrac{x-y}r)\bigr)\Bigr]\,dy\,dx=0,\]
so we are done.

($\Leftarrow$)
Let $\f\in C_c^\infty(\R^d\setminus\{0\})$.
Since
\[\<\Phi*\rho,\Delta\f\>=\<\rho,\Phi*(\Delta\f)\>=\<\rho,\f\>=0,\]
the distribution $\Phi*\rho$ on $\R^d\setminus\{0\}$ is weakly harmonic, and by Weyl's lemma for distributions, it is a smooth harmonic function on $\R^d\setminus\{0\}$.

Since $\rho$ is supported at zero, we have a positive integer $k$ and constants $a_\alpha$ such that
\[|\<\rho,\f\>|\le\sum_{|a|\le k}|a_\alpha D^\alpha\f(0)|\]
for $\f\in C^\infty(\R^d)$.
Then, for non-zero $x\in\R^d$, by taking a cutoff function $\chi\in C_c^\infty(\R^d)$ such that
\[\chi(y)=\begin{cases}1&\text{ if }|y-x|\le\frac13|x|\\0&\text{ if }|y|\le\frac13|x|\end{cases},\]
we have
\[|\Phi*\rho(x)|=|(\Phi\chi)*\rho(x)|=|\<\rho(x-y),\Phi(y)\chi(y)\>_y|\le\sum_{|a|\le k}|a_\alpha D^\alpha\Phi(x)|=O(r^{2-d})\]
as $r\to\infty$.
Therefore, $\Phi*\rho$ vanishes at infinity.
\end{pf}

\begin{lem*}
Let $\rho$ be a distribution on $\R^d$ such that $\supp(\rho)\subset\{0\}$.
Then, there is a constant coefficient partial differential operator $P(D)$ such that $\rho=P(D)\delta$.
\end{lem*}

\begin{cor*}
Let $d\ge3$.
If a distribution $u\in\cD'(\R^d)$ is a harmonic function on $\R^d\setminus\{0\}$ and vanishes at infinity, then there are an integer $k\ge0$ and constants $a_\alpha$ such that
\[u(x)=\sum_{|a|\le k}a_\alpha D^\alpha\Phi(x)\]
for $x\ne0$, where $\Phi$ denotes the fundamental solution of Laplace's equation.
\end{cor*}




\newpage
\section{Unified error analysis}

\subsection{Approximation of Banach spaces}
We follow closely Temam for the abstract error analysis.
The word ``approximation'' in here can be replaced into ``discretization''.

\begin{defn}[Approximation]
Let $X$ be a Banach space.
An \emph{approximation} of $X$ is an indexed family $X_h$ of finite-dimensional normed spaces, with a \emph{prolongation operator} $p_h\in B(X_h,X)$ and a \emph{restriction operator} $r_h:X\to X_h)$.
The operator $p_hr_h:X\to X$ is called the \emph{truncation operator}.
\begin{cd}
X\dar[->>,swap]{r_h} \\ X_h\uar[bend right,swap]{p_h}
\end{cd}
\end{defn}

\begin{defn}[Errors]
Let $X_h$ be an approximation of a Banach space $X$.
For $x\in X$ and $x_h\in X_h$, the quantities $E(x_h,x):=\|p_hx_h-x\|$ and $DE(x_h,x):=\|x_h-r_hx\|$ are called the \emph{error} and the \emph{discrete error} between $x$ and $x_h$.
The quantity $TE(x):=\|x-p_hr_hx\|$ is called the \emph{truncation error}.
\end{defn}

\begin{defn}[Stable and convergent approximations]
We say an approximation $X_h$ is
\begin{parts}
\item \emph{stable} if $\|p_h\|+\|r_h\|\lesssim1$,
\item \emph{convergent} if $\|p_hr_hx-x\|\to0$ for each $x\in X$.
\end{parts}
\end{defn}

\begin{lem}
Let $X_h$ be an approximation of a Banach space $X$.
If $X_h$ is stable and convergent, then for each net $x_h\in X_h$ the discrete convergence implies the strong convergence.
\end{lem}
\begin{pf}
We have for each $x\in X$ that
\[DE=\|r_h\|\cdot E\quad\text{ and }\quad E=\|p_h\|\cdot DE+TE.\qedhere\]
\end{pf}

\begin{lem}
Let $X_h$ be an approximation of a Banach space $X$.
If $\|p_hx\|\sim\|x\|$, then the stability of $X_h$ follows from the convergence of $X_h$.
\end{lem}
\begin{pf}
It is by the uniform boundedness principle:
\[\|r_hx\|\lesssim\|p_hr_hx-x\|+\|x\|.\]
\end{pf}
In most cases we have $\|p_hx\|=\|x\|$, so for an approximation it is enough to verify the truncation error converges to zero.



\subsection{Approxiamation of problems}

A \emph{well-posed problem} is an operator $L:\cX\to\cY$ such that there is a continuous operator $L^{-1}:Y\to X$ satisfying $LL^{-1}=\id_Y$, where $X\subset\cX$ and $Y\subset\cY$ are embeddings.
Say, consider the spaces $\cX$ and $\cY$ as space of distributions.
We will always assume $L:X\to Y$ is a right invertible(i.e. well-posed) linear operator between Banach spaces.

\begin{defn}[Approximation]
Let $L$ be a well-posed linear problem.
An \emph{approximation} of $L$ is an indexed family $L_h\in L(X_h,Y_h)$ of invertible linear operators, where $X_h$ and $Y_h$ are stable and convergent approximations of $X$ and $Y$.
\end{defn}
We also do not need to assume in fact the stability of $r_h$.
The approximation $X_h$ of $X$ is where we should take subtly, and the art of numerical analysis begins with the choice of $X_h$.
The following diagram does not commute, but \emph{approximately} commute.
\begin{cd}
X \dar[dashed,bend right,swap]{r_h}\rar{L} & Y \dar{r_h} \\
X_h \uar[dashed,bend right,swap]{p_h}\rar{L_h} & Y_h
\end{cd}

\begin{defn}
Let $L_h$ be an approximation of a well-posed linear problem $L$.
We say $L_h$ is
\begin{parts}
\item \emph{consistent} if $CE=\|r_hLx-L_hr_hx\|\to0$ for each $x$,
\item \emph{stable} if $\|L_h^{-1}\|\lesssim1$,
\item \emph{convergent} if $DE=\|L_h^{-1}r_hLx-r_hx\|\to0$ for each $x$.
\end{parts}
\end{defn}

\begin{thm}[Lax equivalence]
Let $L_h$ be an approximation of a well-posed linear problem $L$.
If $L_h$ is consistent, then it is stable if and only if it is convergent.
\end{thm}
\begin{pf}
($\Rightarrow$)
It is clear from
\[DE=\|x_h-r_hx\|\le\|L_h^{-1}\|\|r_hLx-L_hr_hx\|=\|L_h^{-1}\|\cdot CE.\]

($\Leftarrow$)
If we show for the net of operators $p_hL_h^{-1}r_h:Y\to X$ that $p_hL_h^{-1}r_hy$ is bounded in $X$ for each $y\in Y$, then by the uniform boundedness principle the operators $p_hL_h^{-1}r_h$ is uniformly bounded, and we obtain the stability from
\[\|L_h^{-1}\|=\|r_hp_hL_h^{-1}r_hp_h\|\le\|r_h\|\|p_hL_h^{-1}r_h\|\|p_h\|.\]

Since $L$ is surjective by the well-posedness, there is $x\in X$ such that $Lx=y$.
With this $x$ we have
\[\|p_hL_h^{-1}r_hy-x\|\le\|p_h\|\cdot DE+TE\to0,\]
so we are done. 

\end{pf}




\subsection{Numerical analyses}
For a numerical approximation, we can consider three analyses:
\begin{enumerate}
	\item Consistency analysis,
	\item Statbility analysis,
	\item Error analysis.
\end{enumerate}
Note that we have $DE\le\|L_h^{-1}\|\cdot CE$.
If we have the estimate for the rate of the consistency error from the consistency analysis, and also if we have the bound of $\|L_h^{-1}\|$ in the stability analysis, we can easily obtian an \emph{error estimate}.
In this regard, the main difficulty is the former two.


\subsubsection*{Consistency analysis}
Usually the Taylor's theorem is used in finite difference schemes.


\subsubsection*{Stability analysis}
For the bound of $\|L_h^{-1}\|$, we have to make a \emph{stability estimate}
\[\|x_h\|\lesssim\|L_hx_h\|.\]

We have some notes about uniqueness and existence: the injectivity of $L_h^{-1}$ clearly follows from the above estimate, and the surjectivity is deduced thanks to the finite-dimensional nature of $X_h$ and $Y_h$ when their dimensions coincide.

\subsubsection*{Error analysis}
In the Ritz-Galerkin approximation the discrete solution operator $p_hL_h^{-1}r_hL$ can be directly shown to be an orthogonal projection called the \emph{Ritz projection}, which deduces an \emph{a priori} convergence result before justifying proving consistency and stability.




\subsection{Applications}
\begin{ex}
Consider
\[\left\{\begin{alignedat}{2}
u'(x)-u(x)&=f(x) &\qquad&\text{ in }x\in(0,1),\\
u(0)&=c. &&
\end{alignedat}\right.\]
Let $X:=C^1([0,1])$, $Y:=C([0,1])\times\R$, and $Au(x):=(u'(x)-u(x),u(0))$.
Then it is well-posed since there is $E:Y\to X$ defined by
\[E(f,c)(x):=c+\int_0^xe^{-y}f(y)\,dy\]
satisfies
\end{ex}

\begin{ex}
Consider
\[\left\{\begin{alignedat}{2}
-\Delta u(x)&=f(x) &\qquad&\text{ in }x\in(0,1)^2,\\
u(x)&=0 &&\text{ on }x\in\partial(0,1)^2.
\end{alignedat}\right.\]
Let $X=$, $Y=$, $Au$
\end{ex}

\begin{ex}
Consider
\[\left\{\begin{alignedat}{2}
\partial u(t,x)&=\Delta u(t,x) &\qquad&\text{ in }(t,x)\in(0,\infty)\times(0,1),\\
u(0,x)&=f(x) &&\text{ on }x\in[0,1],\\
u(t,0)&=0 &&\text{ on }t\in[0,\infty),\\
u(t,1)&=0 &&\text{ on }t\in[0,\infty),\\
\end{alignedat}\right.\]
Let $X=$, $Y=$, $Au$
\end{ex}


$u_j^n$, $t=t_0+nk$, $x=x_0+jh$




\newpage
\section{Kinetic theory}
\subsection{Velocity averaging lemmas}
The velocity averaging lemma is used to get regularity of averaged quantity when boundary condition is not given.
\begin{thm}[Velocity averaging]
Let $L$ be a free transport operator $\pd_t+v\cdot\nabla_x$ on $\R_t\times\R_x^n\times\R_v^n$.
Then,
\[\|\int u\f\,dv\|_{H_{t,x}^{1/2}}\lesssim_\f\|u\|_{L_{t,x,v}^2}^{1/2}\|Lu\|_{L_{t,x,v}^2}^{1/2}\]
for $\f\in C_c^\infty(\R_v^n)$,
\end{thm}
\begin{pf}
Let $m(t,x)=\int u\f\,dv$.
By Fourier transform with respect to $t$ and $x$, we have
\[\hat u(\tau,\xi,v)=\frac1i\,\frac{\hat{Lu}(\tau,\xi,v)}{\tau+v\cdot\xi}\]
and
\[\hat m(\tau,\xi)=\int\hat u(\tau,\xi,v)\f(v)\,dv.\]
Fixing $\tau,\xi$, decompose the integral and use H\"older's inequality to get
\begin{align*}
|\hat m(\tau,\xi)|
&\le\int_{|\tau+v\cdot\xi|<\alpha}|\hat u\f|\,dv+\int_{|\tau+v\cdot\xi|\ge\alpha}\frac{|\hat{Lu}\f|}{|\tau+v\cdot\xi|}\,dv\\
&\le\|\hat u\|_{L_v^2}^{1/2}\ (\int_{|\tau+v\cdot\xi|<\alpha}|\f|^2\,dv)^{1/2}+\|\hat{Lu}\|_{L_v^2}^{1/2}\ (\int_{|\tau+v\cdot\xi|\ge\alpha}\frac{|\f|^2}{|\tau+v\cdot\xi|^2}\,dv)^{1/2},
\end{align*}
where $\alpha>0$ is an arbitrary constant that will be determined later.
Let
\[I_s(\tau,\xi,\alpha):=\int_{|\tau+v\cdot\xi|<\alpha}|\f|^2\,dv,\qquad
I_n(\tau,\xi,\alpha):=\int_{|\tau+v\cdot\xi|\ge\alpha}\frac{|\f|^2}{|\tau+v\cdot\xi|}\,dv.\]
We are going to estimate the integrals as
\[I_s\lesssim\frac{\alpha}{\sqrt{\tau^2+|\xi|^2}},\qquad
I_n\lesssim\frac1{\alpha\sqrt{\tau^2+|\xi|^2}}.\]

Define coordinates $(v_1,v_2)$ on $\R_v$ as follows:
\[v_1:=\frac{\tau+v\cdot\xi}{|\xi|}\ \in\R\ ,\qquad v_2:=v-\frac{v\cdot\xi}{|\xi|^2}\xi\ \in\ker(\xi^T)\cong\R^{n-1}.\]
Note that
\[|v|^2=(v_1-\frac\tau{|\xi|})^2+|v_2|^2\qquad\text{and}\qquad\int\,dv=\iint\,dv_2\,dv_1.\]

For the first integral, suppose that $\f$ is supported on a ball $|v|\le R$.
If $\frac{|\tau|-\alpha}{|\xi|}>R$, then the region of integration vanishes so that $I_s=0$.
If $|\tau|\le\alpha+R|\xi|$, then
\begin{align*}
I_s&\lesssim\int_{|v_1|<\frac\alpha{|\xi|}}\int_{|v_2|^2\le R^2-(v_1-\frac\tau{|\xi|})^2}\,dv_2\,dv_1\\
&\lesssim\int_{|v_1|<\frac\alpha{|\xi|},\ |v_1|\le R}\int_{|v_2|\le R}\,dv_2\,dv_1\\
&\lesssim\min\{\frac{2\alpha}{|\xi|},R\}\cdot R^{n-1}\\
&\simeq\frac1{\sqrt{1+(\frac{|\xi|}\alpha)^2}}\\
&\lesssim\frac\alpha{\sqrt{\tau^2+|\xi|^2}}.
\end{align*}

For the second integral, suppose that $\f$ is supported on $|v|<R$ so that $|v_1-\frac\tau{|\xi|}|,|v_2|<R$.
Then,
\begin{align*}
I_n&\lesssim\int_{|v_1|\ge\frac\alpha{|\xi|},\ |v_1-\frac\tau{|\xi|}|<R}\int_{|v_2|<R}\frac1{v_1^2|\xi|^2}\,dv_2\,dv_1\\
&\simeq\int_{\max\{\frac\alpha{|\xi|},\frac{|\tau|}{|\xi|}-R\}\le v_1<\frac{|\tau|}{|\xi|}+R}\frac1{v_1^2|\xi|^2}\,dv_1\\
&\simeq\frac1{|\xi|^2}(\frac1{\max\{\frac\alpha{|\xi|},\frac{|\tau|}{|\xi|}-R\}}-\frac1{\frac{|\tau|}{|\xi|}+R}).
\end{align*}
If $\frac{|\tau|}{|\xi|}-R>\frac\alpha{|\xi|}$, then
\[I_n\lesssim\frac{2R}{\tau^2-(R|\xi|)^2}<\frac{2R}{\alpha(|\tau|+R|\xi|)}\simeq\frac1{\alpha\sqrt{\tau^2+|\xi|^2}}.\]
If $|\tau|\le\alpha+R|\xi|$, then
\[I_n\lesssim\frac1{|\xi|}\frac{(|\tau|+R|\xi|)-\alpha}{\alpha(|\tau|+R|\xi|)}\le\frac{2R}{\alpha(|\tau|+R|\xi|)}\simeq\frac1{\alpha\sqrt{\tau^2+|\xi|^2}}.\]

To sum up, we have
\[|\hat m(\tau,\xi)|\lesssim\frac1{(\tau^2+|\xi|^2)^{1/4}}(\sqrt\alpha\cdot\|\hat u\|_{L_v^2}^{1/2}+\frac1{\sqrt\alpha}\cdot\|\hat{Lu}\|_{L_v^2}^{1/2}).\]
Letting $\alpha=\sqrt{\|\hat{Lu}\|_{L_v^2}/\|\hat u\|_{L_v^2}}$ and squaring,
\[(\tau^2+|\xi|^2)^{1/2}|\hat m(\tau,\xi)|^2\lesssim\|\hat u\|_{L_v^2}^{1/2}\|\hat{Lu}\|_{L_v^2}^{1/2}.\]
Therefore, the integration on $\R_\tau\times\R_\xi^n$ and Plancheral's theorem gives
\[\|m\|_{H_{t,x}^{1/2}}\lesssim_\f\|u\|_{L_{t,x,v}^2}^{1/2}\|Lu\|_{L_{t,x,v}^2}^{1/2}.\]
\end{pf}


\begin{cor}
Let $\cF$ be a family of functions on $\R_t\times\R_x^n\times\R_v^n$.
If $\cF$ and $L\cF$ are bounded in $L_{t,x,v}^2$, then $\int\cF\f\,dv$ is bounded in $H_{t,x}^{1/2}$.
\end{cor}

\begin{thm}
Let $\cF$ be a family of functions on $I_t\times\R_x^n\times\R_v^n$.
If $\cF$ is weakly relatively compact and $L\cF$ is bounded in $L_{t,x,v}^1$, then $\int\cF\f\,dv$ is relatively compact in $L_{t,x}^1$.
\end{thm}




















\section{Sturm-Liouville theory}
\subsection{Self-adjointness}
Let $I=[a,b]$ and
\begin{gather*}
L=-\frac1{w(x)}\left[\dd{x}\left(p(x)\dd{x}\right)+q(x)\right],\\
0\le p(x)\in C^\infty(I),\quad q(x)\in C^\infty(I),\quad 0<w(x)\in C^\infty(I).
\end{gather*}
We expect $L$ to be self-adjoint.
In this regard, our interest is ellimination of the difference term
\[\<f,Lg\>-\<Lf,g\>=p(f'g-fg')|_a^b.\]


\begin{center}
\renewcommand{\arraystretch}{2.5}
\begin{tabular}{l|l|c|l}
\hline
Name & Operator & Domain & B.C. \\[5pt]
\hline
Helmholtz & $\displaystyle L=-\dd[2]{x}$ & $[a,b]$ & Periodic \\
Helmholtz & $\displaystyle L=-\dd[2]{x}$ & $[a,b]$ & Separated Robin \\[5pt]
\hline
Legendre & $\displaystyle L=-\dd{x}\left((1-x^2)\dd{x}\right)$ & $[-1,1]$ & None \\
A. Legendre & $\displaystyle L=-\left[\dd{x}\left((1-x^2)\dd{x}\right)-\frac{m^2}{1-x^2}\right]$ & $[-1,1]$ & Dirichlet \\
Hermite & $\displaystyle L=-e^{x^2}\left[\dd{x}\left(e^{-x^2}\dd{x}\right)\right]$ & $(-\infty,\infty)$ & Polynomial growth\\
Laguerre
\\[5pt]
\hline
\end{tabular}
\end{center}


\subsection{Regular Sturm-Liouville problem}
We mean \emph{regular Sturm-Liouville problems} by the case that $p$ does not vanish on the boundary of $I$ that we should cancel $f'g-fg'|_a^b$.
View the Sturm-Liouville operator $L$ as a non-densely defined operator on the space $C^\infty(I)$ with inner product $\<f,g\>=\int_Ifgw$ with domain
\[V=\{\,u\in C^\infty(I):\alpha_0u(a)+\alpha_1u'(a)=0,\ \beta_0u(b)+\beta_1u'(b)=0\,\},\]
the subspace for the \emph{separated} Robin boundary condition.
\begin{prop}
The operator $L:V\to C^\infty(I)$ is self-adjoint when $C^\infty(I)$ has the inner product $\<f,g\>=\int_Ifgw$.
\end{prop}
We are interested in the eigenvalue problem of $L:V\to C^\infty(I)$ on $V$.
Fortunately, if we choose a constant $z\in\C\setminus\R$, then $(L-z)^{-1}:C^\infty(I)\to V$ is well-defined.
\begin{prop}
If $z$ is not an eigenvalue of $L$, then $L-z:V\to C^\infty(I)$ is bijective.
\end{prop}
\begin{pf}
The injectivity follows from the definition of eigenvalues.
We may assume that $L$ is injective by translation $q\mapsto q-\lambda$.

Suppose $f\in C^\infty(I)$.
The surjectivity is equivalent to the existence of a second order inhomogeneous boundary problem:
\begin{gather*}
-pu''-p'u'-qu=fw,\\
\alpha_0u(a)+\alpha_1u'(a)=0,\quad\beta_0u(b)+\beta_1u'(b)=0.
\end{gather*}
Let $u_a$, $u_b$ be the unique solutions of the corresponding homogeneous equation with initial conditions
\[u_a(a)=-\alpha_1,\quad u'_a(a)=\alpha_0,\quad u_b(b)=-\beta_1,\quad u'_b(b)=\beta_0.\]
Then we can define $L^{-1}:C^\infty([0,1])\to D(L)$ by
\[L^{-1}f(x):=u_a(x)\int_x^b\frac{u_b}{W[u_a,u_b]}\frac f{(-p)}w\,+\,u_b(x)\int_a^x\frac{u_a}{W[u_a,u_b]}\frac f{(-p)}w,\]
where $W[u_a,u_b]:=u_au_b'-u_bu_a'$ denotes the Wronskian.
This formula is derived from variation of parameters: we can compute $c_a$ and $c_b$ from the fact that
\[\begin{pmatrix}0\\\frac f{(-p)}w\end{pmatrix}=\begin{pmatrix}u_a&u_b\\u_a'&u_b'\end{pmatrix}\begin{pmatrix}c_a'\\c_b'\end{pmatrix}\impl L(c_au_a+c_bu_b)=f.\]
Then, we can check that
\[L^{-1}Lu=u\]
for $u\in D(L)$ by computation, which implies $L$ is surjective.
\end{pf}


\subsection{Legendre's equation}
The Legendre equation is
\[(1-x^2)u''-2xu'+l(l+1)u=0,\quad\text{ on }[-1,1].\]
The Sturm-Liouville operator is
\[L=-\dd{x}\left((1-x^2)\dd{x}\right).\]
Since $p(\pm1)=0$, the operator $L:C^\infty([-1,1])\to C^\infty([-1,1])$ is self-adjoint on the whole domain.


Its eigenvalues and corresponding eigenspaces are
\begin{center}\renewcommand{\arraystretch}{1.2}
\begin{tabular}{c|c|l}
\hline
    & Eigenvalue & Eigenbasis \\
$l$ & $l(l+1)$   & \\
\hline
0   & 0          & $P_0(x)=1$ \\
1   & 2          & $P_1(x)=x$ \\
2   & 6          & $P_2(x)=\frac32x^2-\frac12$ \\
3   & 12         & $P_3(x)=\frac52x^3-\frac32x$ \\
4   & 20         & $P_4(x)=\frac{35}8x^4-\frac{15}4x^2+\frac38$\\
\hline
\end{tabular}
\end{center}
If we admit
\[Q_0(x)=\frac12\log\frac{1+x}{1-x},\quad Q_1(x)=1-\frac12x\log\frac{1+x}{1-x},\ \cdots\ \in L^2(-1,1)\setminus C^\infty([-1,1])\]
as eigenvectors of $L$, then the self-adjointness fails on the extended domain.
For example,
\begin{align*}
\<Q_0,Lf\>-\<LQ_0,f\>
&=\left.p(x)\bigl(Q_0'(x)f(x)-Q_0(x)f'(x)\bigr)\right|_{-1}^1\\
&=f(1)-f(-1)
\end{align*}
does not vanish in general even for $f\in C^\infty([-1,1])$.

\subsection{Bessel's equation}
The Bessel equation is
\[x^2u''+xu'+(k^2x^2-\nu^2)u=0,\quad\text{ on }(0,\infty).\]
The Sturm-Liouville operator is
\[-\frac1x\left[\dd{x}\left(x\dd{x}\right)-\nu^2\frac1x\right].\]


















\section{Peetre's theorem}


\begin{lem}
Suppose a linear operator $L:C_c^\infty(M)\to C_c^\infty(M)$ satisfies
\[\supp(Lu)\subset\supp(u)\quad\text{for}\quad u\in C_c^\infty(X).\]
For each point $x\in M$, there is a bounded neighborhood $U$ together with a nonnegative integer $m$ such that 
\[\|Lu\|_{C^0}\lesssim\|u\|_{C^m}\]
for $u\in C_c^\infty(U\setminus\{x\})$.
\end{lem}
\begin{pf}
Suppose not.
There is a point $x$ at which the inequality fails; for every bounded neighborhood $U$ and for every nonnegative $m$, we can find $u\in C_c^\infty(U\setminus\{x\})$ such that
\[\|Lu\|_{C^0}\ge C\|u\|_{C^m},\]
for arbitrarily large $C$.
We want to construct a function $u\in C_c^\infty(U)$ such that $Lu$ has a singularity at $x$.

(Induction step)
Take a bounded neighborhood $U_m$ of $x$ such that
\[U_m\subset U\setminus\bigcup_{i=0}^{m-1}\bar U_i.\]
There is $u_m\in C_c^\infty(U_m\setminus\{x\})$ such that
\[\|Lu_m\|_{C^0}>4^m\|u_m\|_{C^m}.\]

Note that
\[\supp(u_i)\cap\supp(u_j)=\varnothing\quad\text{for}\quad i\ne j.\]
Define
\[u:=\sum_{i\ge0}2^{-i}\frac{u_i}{\|u_i\|_{C^i}}.\]
We have that $u\in C_c^\infty(U)$ since the series converges in the inductive topology of the LF space $C_c^\infty(U)$: it converges absolutely with respect to the seminorms $\|\cdot\|_{C^m}$ for all $m$:
\begin{align*}
\sum_{i\ge0}\|2^{-i}\frac{u_i}{\|u_i\|_{C^i}}\|_{C^m}
&=\sum_{0\le i<m}2^{-i}\frac{\|u_i\|_{C^m}}{\|u_i\|_{C^i}}+\sum_{i\ge m}2^{-i}\frac{\|u_i\|_{C^m}}{\|u_i\|_{C^i}}\\
&\le\sum_{0\le i<m}2^{-i}\frac{\|u_i\|_{C^m}}{\|u_i\|_{C^i}}+\sum_{i\ge m}2^{-i}\\
&<\infty.
\end{align*}
Also, since the supports of each term are disjoint and $L$ is locally defined, we have
\[Lu=\sum_{i\ge0}2^{-i}\frac{Lu_i}{\|u_i\|_{C^i}}.\]
Thus,
\[\|Lu\|_{C^0}=\sup_{i\ge0}2^{-i}\frac{\|Lu_i\|_{C^0}}{\|u_i\|_{C^i}}>\sup_{i\ge0}2^{-i}\cdot4^i=\infty,\]
which leads a contradiction.

\end{pf}













\section{Characteristic curve}
Algorithm:
\begin{parts}
\item Establish the associated vector field by substituting $u\mapsto y$.
\item Find the integral curve.
\item Eliminate the auxiliary variables to get an algebraic equation.
\item Verify the computed solution is in fact the real solution.
\end{parts}
\begin{prop}
Suppose that there exists a smooth solution $u:\Omega\to\R_y$ of an initial value problem

\[\left\{\begin{alignedat}{2}
u_t+u^2u_x&=0, && (t,x)\in\Omega\subset\R_{t\ge0}\times\R_x,\\
u(0,x)&=x, && \text{at}\ x\in\R,
\end{alignedat}\right.\]
and let $M$ be the embedded surface defined by $y=u(t,x)$.

Let $\gamma:I\to\Omega\times\R_y$ be an integral curve of the vector field
\[\pd{t}+y^2\pd{x}\]
such that $\gamma(0)\in M$.
Then, $\gamma(\theta)\in M$ for all $\theta\in I$.
\end{prop}
\begin{pf}
We may assume $\gamma$ is maximal.
Define $\tilde\gamma:\tilde I\to M$ as the maximal integral curve of the vector field
\[\tilde X=\pd{t}+u^2\pd{x}\in\Gamma(TM)\]
such that $\tilde\gamma(0)=\gamma(0)$.
Since $X$ and $\tilde X$ coincide on $M$, the curve $\tilde\gamma$ is also an integral curve of $X$ with $\tilde\gamma(0)=\gamma(0)$.
By the uniqueness of the integral curve, we get $\tilde I\subset I$ and $\gamma(\theta)=\tilde\gamma(\theta)$ for all $\theta\in\tilde I$.

Since $M$ is closed in $E$, the open interval $\tilde I=\gamma^{-1}(M)$ is closed in $I$, hence $\tilde I=I$ by the connectedness of $I$.
\end{pf}
\begin{defn}
The projection of the integral curve $\gamma$ onto $\Omega$ is called a \emph{characteristic}.
\end{defn}
This proposition implies that we might be able to describe the points on the surface $M$ explicitly by finding the integral curves of the vector field $X$.
Once we find a necessary condition of the form of algebraic equation, we can demostrate the computed hypothetical solution by explicitly checking if it satisfies the original PDE.

Since $X$ does not depend on $u$, we can solve the ODE: let $\gamma(\theta)=(t(\theta),x(\theta),y(\theta))$ be the integral curve of $X$ such that $\gamma(0)=(0,\xi,\xi)$.
Then, the system of ODEs
\begin{alignat*}{2}
\dd{t}{\theta}&=1,   &\qquad t(0)&=0,\\
\dd{x}{\theta}&=y(\theta)^2, & x(0)&=\xi,\\
\dd{y}{\theta}&=0,   & y(0)&=\xi
\end{alignat*}
is solved as
\[t(\theta)=\theta,\qquad y(\theta)=\xi,\qquad x(\theta)=\xi^2\theta+\xi.\]
Therefore,
\[u(t,x)=\frac{-1+\sqrt{1+4tx}}{2t}.\]
From this formula, we would be able to determine the suitable domain $\Omega$ as
\[\Omega=\{\,(t,x):tx>-\tfrac14\,\}.\]


\subsection{Wave equation}

\begin{align*}
&u_{tt}-c^2u_{xx}=0 \quad\text{for}\quad t,x>0, \\
&u(0,x)=g(x),\qquad u(0,x)=h(x),\qquad u_x(t,0)=\alpha(t).
\end{align*}

Define $v:=u_t-cu_x$.
Then we have
\[\left\{\begin{alignedat}{2}
v_t+cv_x &= 0 && t,x>0,\\
v(0,x) &= h(x)-cg'(x). &&
\end{alignedat}\right.\]
By method of characteristic,
\[v(t,x)=h(x-ct)-cg'(x-ct).\]

Then, we can solve two system
\[\left\{\begin{alignedat}{2}
u_t-cu_x &= v, && x>ct>0,\\
u(0,x) &= g(x), &&
\end{alignedat}\right.\]
and
\[\left\{\begin{alignedat}{2}
u_t-cu_x &= v, && ct>x>0,\\
u_x(t,0) &= \alpha(t), &&
\end{alignedat}\right.\]

For the first system, introducing parameter $\xi>0$,
\begin{gather*}
\dd{t}{\theta}=1,\qquad\dd{x}{\theta}=-c,\qquad\dd{y}{\theta}=-v(t,x),\\
t(0)=0,\qquad x(0)=\xi,\qquad y(0)=g(\xi)
\end{gather*}
is solved as
\[t(\theta)=\theta,\qquad x(\theta)=-c\theta+\xi,\qquad y(\theta)=g(\xi)+\int_0^\theta-v(\theta',\xi-c\theta')\,d\theta',\]
hence for $x>ct>0$,

\begin{align*}
u(t,x)&=g(\xi)-\int_0^\theta v(s,\xi-cs)\,ds\\
&=g(x+ct)\\
&=\frac{3g(x+ct)-g(x-ct)}2-\int_0^th(x+c(t-2s))\,ds
\end{align*}



\clearpage
\subsection{Burgers' equation}

Consider the inviscid Burgers' equation
\[u_t+uu_x=0.\]
\begin{parts}
\item Suppose $u(0,x)=\tanh(x)$. For what values of $t>0$ does the solution of the quasi-linear PDE remain smooth and single valued? Given an approximation sketch of the characteristics in the $tx$-plane.
\item Suppose $u(0,x)=-\tanh(x)$. For what values of $t>0$ does the solution of the quasilinear PDE remain smooth and single valued? Given an approximation sketch of the characteristics in the $tx$-plane.
\item Suppose
\[u(0,x)=\begin{cases}0,&x<0\\x,&0\le x<1,\\1,&1\le x\end{cases}.\]
Sketch the characteristics. Solve the Cauchy problem. Hint: solve the problem in each region separately and ``paste'' the solution together.
\end{parts}


















\newpage


\section{Statements in functional analysis and general topology}
Function analysis:
\begin{itemize}
\item Suppose a densely defined operator $T$ induces a Hilbert space structure on its domain. If the inclusion is bounded, then $T$ has the bounded inverse. If the inclusion is compact, then $T$ has the compact inverse.
\item A closed subspace of an incomplete inner product space may not have orthogonal complement: setting $L^2$ inner product on $C([0,1])$, define $\phi(f)=\int_0^{\frac12}f$.
\item Every seperable Banach space is linearly isomorphic and homeomorphic. But there are two non-isomorphic Banach spaces.
\item open mapping theorem -> continuous embedding is really an embedding.
\item $D(\Omega)$ is defined by a \emph{countable stict} inductive limit of $D_K(\Omega)$, $K\subset\Omega$ compact. Hence it is not metrizable by the Baire category theorem. (Here strict means that whenever $\alpha<\beta$ the induced topology by $\cT_\beta$ coincides with $\cT_\alpha$)
\item A net $(\phi_d)_d$ in $D(\Omega)$ converges if and only if there is a compact $K$ such that $\phi_d\in D_K(\Omega)$ for all $d$ and $\phi_d$ converges uniformly.
\item Th integration with a locally integrable function is a distribution. This kind of distribution is called \emph{regular}. The nonregular distribution such as $\delta$ is called \emph{singular}.
\item $D'$ is equipped with the weak$^*$ topology.
\item $\pd{x}\colon D'\to D'$ is continuous. They commute (Schwarz theorem holds).
\item $D\to S\to L^p$ are continuous (immersion) but not imply closed subspaces (embedding).
\end{itemize}
General topology:
\begin{itemize}
\item $H\subset\C$ and $H\subset\hat\C$ have distinct Cauchy structures which give a same topology. In addition, the latter is precompact while the former is not.
\end{itemize}






\section{Ultrafilter}
\begin{defn}
An \emph{ultrafilter} is a synonym for maximal filter.
If we sat $\cU$ is an \emph{ultrafilter on a set} $A$, then it means $\cU$ is a maximal filter as a directed subset of $\cP(A)$.
\end{defn}
existence of ultrafilter.
\begin{thm}
Let $\cU$ be an ultrafilter on a set $A$ and $X$ be a compact space.
For a function $f:A\to X$, the limit $\cU$-$\lim f$ always exists.
\end{thm}

\begin{thm}
Let $X=\prod_{\alpha\in\cA}X_\alpha$ be a product space of compact spaces $X_\alpha$.
A net $f:\cD\to X$ has a convergent subnet.
\end{thm}
\begin{pf}[1]
Use Tychonoff.
Compactness and net compactness are equivalent.
\end{pf}
\begin{pf}[2]
It is a proof without Tychonoff.
Let $\cU$ be a ultrafilter on a set $\cD$ contatining all $\uparrow d$.
Define a directed set $\cE=\{(d,U)\in\cD\times\cU:d\in U\}$ as $(d,U)\succ(d',U')$ for $U\subset U'$.
Let $f:\cE\to X$ be a subnet of $f:\cD\to X$ defined by $f_{(d,U)}=f_d$.

By the previous theorem, $\cU$-$\lim\pi_\alpha f_d\in X_\alpha$ exsits for each $\alpha$.
Define $f\in X$ such that $\pi_\alpha f=\cU$-$\lim\pi_\alpha f_d$.
Let $G=\prod_\alpha G_\alpha\subset X$ be any open neighborhood of $f$.
Then, $\pi_\alpha f\in G_\alpha$ and we have $G_\alpha=X_\alpha$ except finite.
For $\alpha$, we can take $U_\alpha:=\{d:\pi_\alpha f_d\in G_\alpha\}\in\cU$ by definition of convergence with ultrafilter
Since $U_\alpha=\cD$ except finites, we can take an upper bound $U_0\in\cU$ of $\{U_\alpha\}_\alpha$.
Then, by taking any $d_0\in U_0$, we have $f_{(d,U)}\in G$ for every $(d,U)\succ(d_0,U_0)$.
This means $f=\lim_\cE f_{(d,U)}$, so we can say $\lim_\cE f_{(d,U)}$ exists.
\end{pf}







\section{Selected analysis problems}

\begin{prb}
The following series diverges: \[\sum_{n=1}^\infty\frac1{n^{1+|\sin n|}}.\]
\end{prb}
\begin{sol}
Let $A_k:=[1,2^k]\cap\{x:|\sin x|<\frac1k\}$.
Divide the unit circle $\R/2\pi\Z$ by $7k$ uniform arcs.
There are at least $2^k/7k$ integers that are not exceed $2^k$ and are in a same arc.
Let $S$ be the integers and $x_0$ be the smallest element.
Since, $|x-x_0|\pmod{2\pi}<\frac{2\pi}{7k}$ for $x\in S$,
\[|\sin(x-x_0)|<|x-x_0|\pmod{2\pi}<\frac{2\pi}{7k}<\frac1k.\]
Also, $1\le x-x_0\le x\le2^k$, $x-x_0\in A_k$.
\[|A_k|\ge\frac{2^k}{7k}.\]
Therefore,
\begin{align*}
\sum_{n=1}^\infty\frac1{n^{1+|\sin n|}}
&\ge\sum_{n\in A_N}\frac1{n^{1+|\sin n|}}\\
&\ge\sum_{k=1}^N(|A_k|-|A_{k-1}|)\frac1{2^{k+1}}\\
&=\sum_{k=1}^N\frac{|A_k|}{2^{k+1}}-\sum_{k=1}^{N-1}\frac{|A_k|}{2^{k+2}}\\
&=\frac{|A_N|}{2^{N+1}}+\sum_{k=1}^{N-1}\frac{|A_k|}{2^{k+2}}\\
&>\sum_{k=1}^N\frac{2^k}{2^{k+2}}\frac1{7k}\\
&=\frac1{28}\sum_{k=1}^N\frac1k\\
&\to\infty.
\end{align*}
\end{sol}

\clearpage
\begin{prb}
If $|xf'(x)|\le M$ and $\frac1x\int_0^xf(y)\,dy\to L$, then $f(x)\to L$ as $x\to\infty$.
\end{prb}
\begin{sol}
It is a kind of Tauberian theorems.
Since for each fixed $\e>0$ we have
\begin{align*}
|f(x)-\frac1{\e x}\int_{(1-\e)x}^xf(y)\,dy|
&\le\frac1{\e x}\int_{(1-\e)x}^x|f(x)-f(y)|\,dy\\
&\le\frac M{\e x}\int_{(1-\e)x}^x\frac{x-y}y\,dy\\
&=M(\tfrac1\e\log\tfrac1{1-\e}-1)=O(\e)
\end{align*}
by the mean value theorem and 
\[\frac1{\e x}\int_{(1-\e)x}^xf(y)\,dy
=\frac1{\e x}\int_0^xf(y)\,dy-\frac1{\e x}\int_0^{(1-\e)x}f(y)\,dy\to\frac1\e L-\frac{1-\e}\e L=L\]
as $x\to\infty$, we get
\[\limsup_{x\to\infty}|f(x)-L|=O(\e),\]
so we are done.
\end{sol}

\clearpage
\begin{prb}
Let $f_n:[0,1]\to[0,1]$ be a sequence of functions such that $|f_n(x)-f_n(y)|\le|x-y|$ whenever $|x-y|\ge\frac1n$ for each $n\ge1$.
Then, it has a uniformly convergent subsequence.
\end{prb}
\begin{sol}
By the Bolzano-Weierstrass theorem and the diagonal argument for subsequence extraction, we may assume that $f_n$ converges to a function $f:\Q\cap[0,1]\to[0,1]$ pointwisely.

Let $n\ge4$.
Then, for $x\in[0,1]$ there is $z\in[0,1]$ such that $|x-z|=\frac2n$ so that
\[|f_n(x)-f_n(z)|\le|x-z|=\frac2n.\]
Whenever $y\in[0,1]$ satisfies $|x-y|\le\frac1n$, then we have $|y-z|\ge|x-z|-|x-y|\ge\frac1n$, so we get
\[|f_n(y)-f_n(z)|\le|y-z|\le|y-x|+|x-z|\le\frac3n.\]
Combining the two inequalities, we obtain
\begin{align}|x-y|\le\frac1n\impl|f_n(x)-f_n(y)|\le\frac5n\end{align}
for $n\ge4$.

Let $\e>0$ and suppose $|x-y|\le\frac\e5$.
For every $n\ge\max\{\frac{10}\e,4\}$, since $|x-y|\le\frac1n$ implies by the inequality (1) that
\[|f_n(x)-f_n(y)|\le\frac5n\le\frac\e2,\]
and since $|x-y|>\frac1n$ implies by the condition in the problem that
\[|f_n(x)-f_n(y)|\le|x-y|\le\frac\e5<\frac\e2,\]
we have
\begin{align}|x-y|\le\frac\e5\impl|f_n(x)-f_n(y)|\le\frac\e2\end{align}
for all $n\ge\max\{\frac{10}\e,4\}$.

For $\e>0$, take $\delta:=\e/5$ and fix $x$ and $y$ in $\Q\cap[0,1]$ satisfying $|x-y|<\delta$.
Then, we have
\begin{align*}
|f(x)-f(y)|
&\le|f(x)-f_n(x)|+|f_n(x)-f_n(y)|+|f_n(y)-f(y)|\\
&\le|f(x)-f_n(x)|+\frac\e2+|f_n(y)-f(y)|
\end{align*}
for all $n\ge\max\{\frac{10}\e,4\}$, and by limiting $n\to\infty$,
\[|f(x)-f(y)|\le0+\frac\e2+0<\e.\]
Therefore, $f$ is uniformly continuous on $\Q\cap[0,1]$ so that it has a unique continuous extension on the whole $[0,1]$.
Let it denoted by the same notation $f$.

Finally, we are going to show $f_n\to f$ uniformly on $[0,1]$.
By the uniform continuity of $f$, for each $\e>0$ we have $\delta>0$ such that
\begin{align}|x-y|<\delta\impl|f(x)-f(y)|<\frac\e2.\end{align}
Take a finite subset $F\in\Q\cap[0,1]$, such that for every $x$ there is $y$ satisfying $|x-y|<\min\{\frac\e5,\delta\}$.
Then, by (2) and (3), we have an inequality
\begin{align*}
|f_n(x)-f(x)|&\le|f_n(x)-f_n(y)|+|f_n(y)-f(y)|+|f(y)-f(x)|\\
&<\frac\e2+\max_{z\in F}|f_n(z)-f(z)|+\frac\e2
\end{align*}
for all $n\ge\max\{\frac{10}\e,4\}$.
Therefore, by taking supremum for $x$ and limiting $n\to\infty$ on it we have
\[\limsup_{n\to\infty}\|f_n-f\|\le\e,\]
so we are done because $\e$ is arbitrary.
\end{sol}




\section{Physics problem}
\subsection{Resonance}
Let $m,b,k,A,\omega_d$ be positive real constants.
Consider an underdamped oscillator with sinusoidal diving force described as
\[mx''+bx'+kx=A\sin\omega_dt,\quad x(0)=x_0,\ x'(0)=0.\]
There are some observations:
\begin{parts}
\item The underdamping condition means $b^2-4mk<0$ so that the roots of characteristic equation are imaginary.
\item The positivity of $m,b$ implies the real part of solution that will be denoted by $-\beta=-\frac b{2m}$ is negative; it shows exponential decay of solutions.
\item Introducing the natural frequency $\omega_n=\sqrt{k/m}$, we can rewrite the equation as
\[x''+2\zeta\omega_n x'+\omega_n^2x=A\sin\omega t.\]
\item The complementary solution is computed as
\[x_c(t)=x_0\ e^{-\beta t}\cos\sqrt{\beta^2-\omega_n^2}t,\]
and it can be verified that this solution is asymptotically stable, i.e.
\[\lim_{t\to\infty}x_c(t)=0.\]
\item The condition $\beta>\omega_n$ is equivalent to that the oscillator is underdamped.
\item Let $m,k$ be fixed. Then, the solution $x_c$ decays most fastly when $b$ satisfied $b^2=4mk$, equivalently, $\beta=\omega_n$.
\item When $\omega_d=\omega_n$ such that the amplitude of particular solution diverges.
\end{parts}



\end{document}

최고차항의 계수가 1인 사차함수 f(x)는 다음 조건을 만족시킨다.
[각각의 양의 실수 s에 대하여, |f(t)|≤g(t)를 만족시키는 실수 t가 두 개 존재하고 그 차가 s가 되게 하는 이차 이하의 다항함수 g(x)가 유일하게 존재한다.]
또한 위의 조건에서 s=1,3에 대응하는 함수 g(x)를 각각 g₁(x), g₃(x)라 할 때, 방정식 g₁(x)+g₃(x-1/2)=0을 만족시키는 실수 x가 단 한 개 존재한다. 함수 f(x)가 최솟값 -p/q을 가질 때, p+q를 구하시오. (단, p와 q는 서로 소인 자연수이다.)

ans: 43

\iffalse
\begin{defn}
A quantum field is defined as an operator valued tempered distribution satisfying:
\end{defn}

이차양자화: 임베드 L(H) -> L(F(H))
생성소멸: f를 left tensor prod하는 작용소로 바꾸기
		H의 원소를 L(F(H))


일차양자화에서 R이 L(H)로 바뀐 것처럼,
이차양자화도   R이 L(H)로 바뀐 것.


근데 다체론에서 H 대신 F를 쓰고 싶은데, L(H) 에서 L(F(H))로 확장시키는 방법이 있어서 L(F)로 바꿔도 됨.


양자장은 H에 작용해서 입자를 만들거나 없애는 작용소이며, 물리량과 조화롭게 연관됨

Wightman quantum field theory:
quantum field = operator valued tempered distribution
F carries a unitary representation of restricted orthochronous Poincare group

이 양자장조차 미분가능해 어머

RQFT satisfies: invariance of c, Poincare covariance


time-translation covariant quantum field에 대해서 hamiltonian과의 bracket이 미분과 같다고 식을 세우면 이게 양자장의 방정식.




H는 해공간
L(H)는 미분연산자공간

물리과 3학년 버전 파동함수: H = C valued L2 space		L(H) = C coeff pdd valued fields (물리량)
디랙입자에 대한 고전스피너: H = DA irrep valued L2 space	L(H) = DA coeff pdd valued fields?? (물리량)

M이 국소적으로 민코프스키 메트릭을 가지면 그 접공간은 STA irrep
spinor로서의 DA irrep은 four vector가 아냐 (이 STA irrep과 다른 거야, 접공간이 아냐)


디랙방정식의 해 in H 를 classical Dirac field, free Dirac spinor라고 함

STA = 실수 16차원    DA = 복소 16차원


creation operator 하나로 된 field에 vaccum state를 넣으면 state vector가 나온다.
\fi
