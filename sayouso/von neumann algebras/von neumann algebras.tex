\documentclass{../../large}
\usepackage{../../ikhanchoi}


\begin{document}
\title{Von Neumann Algebras}
\author{Ikhan Choi}
\maketitle
\tableofcontents

\iffalse
injectivity
Connes embeddability
property Gamma
Connes' bicentralizer problem
Shlyakhtenko semicircular system
group stability
bimodule
noncommutative probability
\fi


\part{Weights}


\chapter{Projections}




\section{Countability}


\begin{prb}[Countable decomposability]
Let $M$ be a von Neumann algebra.
A projection $p\in M$ is called \emph{countably decomposable} if mutually orthogonal nonzero projections majorized by $p$ are at most countable, and we say $M$ is \emph{countably decomposable} if the identity is.
The followings are all equivalent.
\begin{parts}
\item $M$ is countably decomposable.
\item $M$ admits a faithful normal state.
\item $M$ admits a faithful normal non-degenerate representation with a cyclic and separating vector.
\item The unit ball of $M$ is metrizable in the $\sigma$-strong topology.
\end{parts}
\end{prb}
\begin{pf}
\end{pf}

\begin{prb}[Faithful normal states]
A vector state is separating iff it is faithful.
Cyclic and separating vectors
If $M\subset B(H)$ admits a separating vector, then every normal state is a vector state. (T:V.1.12, J:7.1.4?)
\end{prb}



\begin{prb}[Separable predual]
Let $M$ be a von Neumann algebra.
The followings are all equivalent.
\begin{parts}
\item $M$ has the separable predual.
\item $M$ admits a faithful normal non-degenerate representation on a separable Hilbert space.
\item $M$ is countably decomposable and countably generated.
\item The unit ball of $M$ is metrizable in the $\sigma$-weak topology.
\end{parts}
\end{prb}
\begin{pf}
\end{pf}


\section{Commutative von Neumann algebras}

\begin{prb}
Noncommutative $L^p$ spaces for a general weight?
\begin{parts}
\item For $1\le p<\infty$, $C_0(X)\to L^p(X,\mu)$ is a bounded linear maps of dense range.
\item $L^\infty(X,\mu)$ is a m.a.s.a.~of $B(L^2(X,\mu))$.
\end{parts}
\end{prb}
\begin{pf}
We will show bounded linear maps $L^\infty(X,\mu)'\to M(X)$ and $L^\infty(X,\mu)\to M(X)$ have the same image.
Let $y\in L^\infty(X,\mu)'$ and define $\mu_y\in M(X)$ by
\[\mu_y(a):=\<\pi_\mu(a)y\psi_\mu,\psi_\mu\>.\]
We claim that $\mu_y$ factors through $L^1(X,\mu)$.
\end{pf}

Monotone convergence theorem states that a measure on a countably decomposable(?) enhanced measurable space $X$ uniquely defines a `countably' normal weight on the space of all measurable functions.
Note that a `countably' normal weight is normal on a countably decomposable von Neumann algebra.



\begin{prb}[Maximal commutative subalgebras]
A commutative von Neumann algebra $M$ is m.a.s.a.~if and only if it admits a cyclic vector.
In this case, $M$ is spatially isomorphic to some $L^\infty$(if separable?).
\end{prb}
\begin{pf}
\end{pf}

separable commutative von Neumann algebra is generated by one self-adjoint element.





\begin{prb}
The set of projections is a complete orthomodular lattice.
If $M$ is commutative, then the set of projections is a complete boolean algebra.
\end{prb}

\begin{itemize}
\item commutative ring - distributive lattice - coherent locale
\item clean ring+$\alpha$ - boolean algebra - stone space
\item - complete boolean algebra - stonean space
\item commutative von Neumann algebra - localizable boolean algebra - hyperstonean space
\end{itemize}


A \emph{frame} is a partially ordered set $F$ that admits a finite meets and arbitrary joins, and for any $a\in F$ the map $F\to F:x\mapsto x\wedge a$ preserves suprema.
A \emph{locale} is an object of the opposite category of frames.
An element of a locale is called \emph{open}.


A locale is called \emph{coherent} if the set of compact opens is closed under finite meets and every open is the join of compact opens, i.e.~generates opens.
It is known that a coherent locale is spatial.
\begin{enumerate}[(i)]
\item $X$ is a coherent space.
\item $X$ is a (compact) sober space such that the set of compact open subsets is closed under finite intersections and forms a base.
\item $X$ is homeomorphic to the underlying space of an affine scheme.
\end{enumerate}
A morphism of $\mathrm{CohLoc}$ is a compact open preserving local morphism.
A morphism of $\mathrm{DistLat}$ is just a lattice morphism.
We can consider the compact open functor $\mathrm{CohLoc}\to\mathrm{DistLat}^{\mathrm{op}}$ and the ideal functor $\mathrm{DistLat}^{\mathrm{op}}\to\mathrm{CohLoc}$.
They form a categorical equivalence between the category of coherent locales and the opposite category of distributive lattices with lattice morphisms (i.e.~preserving finite meets and joins).

A locale is called \emph{Stone} if it is a coherent locale in which every open is the join of all subopens of it.
\begin{enumerate}[(i)]
\item $X$ is a Stone space.
\item $X$ is totally disconnected and compact Hausdorff.
\item $X$ is a compact zero-dimesional sober space.
\item $X$ is a compact zero-dimesional Hausdorff space.
\item $X$ is coherent and Hausdorff.
\end{enumerate}
A morphism of $\mathrm{StoneLoc}$ is a compact open (clopen) preserving locale morphism.
A morphism of $\mathrm{BoolLat}$ is just a lattice morphism.





A locale is called \emph{Stonean} if it is a Stone locale in which the (unique) complement of any element is clopen.
A morphism of $\mathrm{StoneanLoc}$ is an open locale morphism.
A morphism of $\mathrm{CpltBoolLat}$ is a continuous lattice morphism.

A locale is called \emph{Hyperstonean} if...
A boolean lattice is called \emph{localizable} if it is complete, and the identity is approximated by elements admitting a faithful continuous valuation on their compression.
The category $\mathrm{LBAlg}$ admits small products, and the products are preserved by the forgetful functor $\mathrm{LBAlg}\to\mathrm{BAlg}$.



\[*\]

\begin{prb}[Boolean algebra]
A \emph{boolean ring} is a ring in which every element is idempotent, which is automatically commutative.
A \emph{boolean algebra} is a unital boolean ring.
A \emph{boolean lattice} is a complemented distributive lattice.
\begin{parts}
\item There is a one-to-one correspondence between boolean rings and boolean lattices.
\item The category of boolean algebras with unital homomorphisms and the category of Stone spaces with continuous maps are equivalent.
\item The category of complete boolean algebras with order continuous unital homomorphisms and the category of Stonean spaces with open continuous maps are equivalent.
In the Stonean space, the join and meet is realized as the closure of union and the interior of intersection, respectively.
\end{parts}
\end{prb}


\begin{prb}[Measurable algebras]
For a boolean algebra, existences of sequential suprema and sequential infima are equivalent.
A boolean algebra is called a \emph{measurable algebra} if it is order $\sigma$-complete.
\begin{parts}
\item (Loomis-Sikorski representation) Every measurable algebra $\cL$ is realized as $\cM/\cM\cap\cN$ from a enhanced measurable space $(X,\cM,\cN)$.
\item (Dedekind completion) Every boolean algbera $\cL$
\end{parts}
\end{prb}
\begin{pf}
(a)
Let $X$ be the Stone space of $\cL$, $\cM$ the set of clopen subsets, and $\cN$ the set of meager sets.
Then, $\cM$ is a $\sigma$-algebra on $X$ and $\cN$ is a $\sigma$-ideal of $X$.

(b)
complete extension of order continuous homomorphisms and universal property.
regular open algebra of $X$.
\end{pf}



\begin{prb}[Measure algebras]
A \emph{measure} on a measurable algebra $\cL$ is a completely additive monotone function $\cL\to[0,\infty]$.
A \emph{measure algebra} is a measurable algebra together with a faithful measure.

Let $(X,M,\mu)$ be a measure space, which is not necessarily faithful.
There is a canonically associated measure algebra $(\cM/\cM\cap\cN,\mu)$, which is faithful, where $\cN:=\mu^{-1}(0)$.

\end{prb}


\begin{prb}[Localizable measure algebras]

For a measure space $(X,\cM,\mu)$, the completion always does not change the measure algebra, and the complete locally determined version
\[\tilde\cM:=\{E\subset X:E\cap A\in\cM\triangle\cN,\ \mu(A)<\infty\},\qquad\tilde\mu(E):=\sup\{\mu(E\cap A):\mu(A)<\infty\}\]
does not change the measure algebra when the measure space is localizble.


\begin{parts}
\item Every localizable measure algebra is obtained from a compact decomposable measure space.
\item A $\sigma$-finite measure space is compact decomposable.
\end{parts}
\end{prb}









\begin{itemize}
\item $\mathrm{HSTop}$: hyperstonean spaces with open continuous maps,
\item $\mathrm{HSLoc}$: hyperstonean locales with open localic maps,
\item $\mathrm{LBAlg}$: localizable boolean lattices with continuous lattice homomorphisms,
\item $\mathrm{CW^*Alg}$: commutative W$^*$ algebras with normal $*$-homomorphisms.
\end{itemize}

\[\begin{tikzcd}
\mathrm{HSTop} \rar[shift left]{top}&
\mathrm{HSLoc} \lar[shift left]{sp}\rar[shift left]{clopen}&
\mathrm{LBAlg}^\op=\mathrm{MLoc} \lar[shift left]{ideal}\rar[shift left]{L^\infty}&
\mathrm{CW^*Alg}^\op \lar[shift left]{proj}
\end{tikzcd}\]


\begin{prb}[]
\begin{parts}
\item Construction of projection lattice functor.
\item Construction of $L^\infty$ functor.
\item Equivalence.
\end{parts}
\end{prb}
\begin{pf}
(b)
Let $L$ be a measurable locale.
For $\F\in\{\R,\C\}$, define $L^\infty(L,\F)$ to be the set of all bounded localic maps $x:L\to\F$, which are given by the opposite of lattice homomorphism $x^{-1}:\mathrm{top}(\F)\to L$ which preserves finite meets and arbitrary joins, and factors an open ball of $\F$.
We can define a normed $*$-algebra structure on $L^\infty(L,\F)$ such that
\begin{gather*}
(x+y)^{-1}(U):=\bigvee_{U_x+U_y\subset U}(x^{-1}(U_x)\wedge y^{-1}(U_y)),\qquad(xy)^{-1}(U):=\bigvee_{U_xU_y\subset U}(x^{-1}(U_x)\wedge y^{-1}(U_y)),\\
(x^*)^{-1}(U):=x^{-1}(\{\bar z:z\in U\}),\qquad\|x\|=\inf\{\sup_{z\in U}|z|:x^{-1}(U)=1\in L\}.
\end{gather*}
Using the axioms of locales, for example that the meet with a single element preserves arbitrary joins, we can manually check that $L^\infty(L,\F)$ is a commutative normed $*$-algebra, and in particular the C$^*$-identity when $\F=\C$.
Furthermore, since $L^\infty(L,\C)$ is the complexification of $L^\infty(L,\R)$, if we prove $L^\infty(L,\R)$ has a predual, then the completeness with respect to norm follows automatically, so $L^\infty(L,\C)$ becomes a C$^*$-algebra with a predual, i.e.~a von Neumann algebra.

Define $L^1(L,\R)$ the real linear span of continuous valuations on $L$, equipped with the variation norm.
Recall that a continuous valuation is a monotone function $v:L\to[0,\infty)$ such that $v(0)=0$ and $v(p)+v(q)=v(p\vee q)+v(p\wedge q)$, which preserves directed suprema.
Note that $L^\infty(L,\R)$....
\end{pf}




$\sigma$-field is a unital $\sigma$-ring.
$\sigma$-ideal is an ideal of a $\sigma$-ring which is a $\sigma$-ring.
$\sigma$-ideal is sometimes called the measure class because it corresponds to an equivalence class of measures up to absolute continuity.


\begin{prb}[Enhanced measurable spaces]
An \emph{enhanced measurable space} is a measurable space $(X,M)$ together with a $\sigma$-ideal $N$ of $M$.
A morphism between enhanced measurable spaces is a partial function $f:X_1\to X_2$ on a conegligible set such that $f^*$ induces a ring homomorphism $M_2/N_2\to M_1/N_1$.
\begin{parts}
\item Maharam's theoem: every enhanced measurable space is isomorphic to the disjoint union of $\{0,1\}^I$, where $I$ is an aribitrary cardinality...?
\item A $\sigma$-finite enhanced measurable space is isomorphic to a enhenced measurable space induced from a standard probability space...?
\item For $\sigma$-finite enhanced measurable spaces, a $*$-homomorphism $L^\infty(X_2)\to L^\infty(X_1)$ induces a morphism $X_1\to X_2$...?
\end{parts}
\end{prb}


Premaps: 

Strict maps: an a.e.~equivalence class of premaps. For each strict map with non-empty codomain, there is a everywhere defined representative.


Quotients on morphisms:
\[\mathrm{PreEMS}\to\mathrm{StrictEMS}\to\mathrm{EMS}.\]
Fully faithful functors:
\[\mathrm{REMS}\to\mathrm{CDEMS}\to\mathrm{DEMS}\to\mathrm{LEMS}, \mathrm{DEMS}\to\mathrm{LDEMS}.\]
The functor $\mathrm{LEMS}\to\mathrm{LBAlg}:(X,M,N)\mapsto M/N$ is a well-defined essentially surjective functor, which is fully faithful on the full subcategory $\mathrm{CDEMS}$.

We say a enhanced measurable space is \emph{decomposable} or \emph{strictly localizable} if it is isomorphic to the small coproduct of countably decomposable enhanced measurable spaces.

$\mathrm{DEMS}$ is a full subcategory of $\mathrm{PreEMS}$, but not of $\mathrm{EMS}$, and we embed it to $\mathrm{LDEMS}$










\begin{prb}[Maharam classification]
atomic parts, $\{0,1\}^\N\cong\R$, $\{0,1\}^\R\cong?$.
simply count the (infinite) number of summands for each possible cardinality the index set.
\end{prb}

Every commutative von neumann algebra can be realized as $L^\infty$ of the disjoint union, or equivalently, the direct product of $L^\infty$, of countably decomposable enhanced measurable spaces.
Every countably decomposable commutative von Neumann algebra is the tensor product of $\ell^\infty$'s.



atomless ergodic measurable spaces are classified by infinite cardinals.

\section{Tensor products}

$L^2(X,\mu,H)=L^2(X,\mu)\otimes H$
vector or operator-valued integrals


\section{Direct integrals}

\begin{prb}[Effros Borel structure]
\end{prb}

\begin{prb}[Decomposition of states]
\end{prb}




\section{Types}


finite, infinite, purely infinite, properly infinite, abelian projections



Type I factors.
It possess a minimal projection.
It is isomorphic to the whole $B(H)$ for some Hilbert space.
Therefore, it is classified by the cardinality of $H$.

Type II factors.
No minimal projection, but there are non-zero finite projections so that every projection can be ``halved'' by two Murray-von Neumann equivalent projections.

In type II$_1$ factors, the identity is a finite projection
Also, Murray and von Neumann showed there is a unique finite tracial state and the set of traces of projections is $[0,1]$.
Examples of II$_1$ factors include crossed product, tensor product, free product, ultraproduct.
Free probability theory attacks the free groups factors, which are type II$_1$.

In type II$_\infty$ factors.
There is a unique semifinite tracial state up to rescaling and the set of traces of projections is $[0,\infty]$.

In type III factors no non-zero finite projections exists.
Classified the $\lambda\in[0,1]$ appeared in its Connes spectrum, they are denoted by III$_\lambda$.
Tomita-Takesaki theory.
It is represented as the crossed product of a type II$_\infty$ factor and $\R$.

\begin{itemize}
\item Type III$0<\lambda<1$ factor: unique $N\rtimes_\alpha\Z$, $N$ II$_\infty$ factor,
\item Type III$_1$ factor: unique $N\rtimes_\alpha\R$, $N$ II$_\infty$ factor,
\item Type III$_0$ factor: one-to-one correpondence with nontransitive ergodic flows.
\end{itemize}

Amenability, equivalently hyperfiniteness is a very nice condition in von Neumann algebra theory.
Group-measure space construction can construct them.
There are unique hyperfinite type II$_1$ and II$_\infty$ factors, and their property is well-known.
Fundamental groups of type II factors, discrete group theory, Kazhdan's property (T) are used.

Tensor product factors such as Araki-Woods factors and Powers factors.








cyclic group actions implies the classification of injective factors.

\begin{itemize}
\item cyclic groups: Connes (II, III$<1$), Haagerup (III$_1$),
\item finite groups: Jones (II$_1$)
\item discrete amenable groups: Ocneanu (II$_1$), 
\item property T:
\item one-parameter:
\item compact abelian: Takesaki duality?
\end{itemize}

Type I:
Every automorphism of type I factor is inner.
Cocycle conjugacy classes of actions of $\Gamma$ on the injective type I factor $B(\ell^2)$ is correponded to $H^2(\Gamma,\T)$.

approximately inner automorphisms
centrally trivial automorphisms
pointwise inner automorphisms

minimal action












\chapter{Weights}



\section{Semi-cyclic representations}


\begin{prb}[Ideals associated to weights]
left ideal, definition ideal
\end{prb}


\begin{prb}[Semi-cyclic representations]
Let $A$ be a C$^*$-algebra.
A \emph{semi-cyclic representation} is a representation $\pi:A\to B(H)$ together with a linear map $\psi:\fN\to H$ from a left ideal $\fN$ of $A$ into $H$ with dense range, such that $\pi(x)\psi(y)=\psi(xy)$ for $x\in A$ and $y\in\fN$.

For a semi-cyclic representation, if we denote $\fM:=\fN^*\fN$, then we have a bilinear form
\[\Theta:\fM\times\pi(A)'\to\C:(y^*x,z)\mapsto\<z\psi(x),\psi(y)\>.\]
With this, we can construct a linear map $\theta:\fM\to(\pi(A)')_*$ and its transpose $\theta^*:\pi(A)'\to\fM^\#$.

Consider a weight $\f$.
\begin{parts}
\item A (it might require some condition here if $A$ is not W$^*$) weight on $A$ defines a semi-cyclic representation and vice versa?
\item If $A=M$ is a von Neumann algebra, then we can let $\theta_*:\pi(M)'\to M_*$ to have $\theta^{**}=\theta$.
\item $\theta^*$ is bijective onto the space of linear functionals on $\fM$ absolutely continuous with respect to $\f$. (bounded Radon-Nikodym)
\end{parts}
\end{prb}



\begin{prb}[Normal weights]
Let $M$ be a von Neumann algebra.
Let $\omega$ be a weight of $M$.
\begin{parts}
\item $\omega$ is normal.
\item $\omega$ is $\sigma$-weakly lower semi-continuous.
\item $\omega$ is the supremum of a set of normal positive linear functionals.
\end{parts}
\end{prb}
\begin{pf}
(c)$\Rightarrow$(b)$\Rightarrow$(a) are clear.

(a)$\Rightarrow$(b)


Suppose first $M$ is countably decomposable so that $B$ is metrizable.


\end{pf}


\section{Hilbert algebras}






\chapter{Traces}

\section{}


\begin{prb}[Semi-finite and tracial von Neumann algebras]
Let $M$ be a von Neumann algebra.
We say $M$ is \emph{semi-finite} if it admits a faithful semi-finite normal trace, and \emph{tracial} if it admits a faithful normal tracial state.
\begin{parts}
\item regular representation and antilinear isometric involution $J$. $L(G)=\rho(G)'$
\item $M$ is semi-finite if and only if type III does not occur in the direct sum.

\item A factor $M$ has at most one tracial state, which is normal and faithful.
\item A factor is tracial if and only if it is type II$_1$.
\end{parts}
\end{prb}


\begin{prb}[Semi-finite traces]
Let $M$ be a von Neumann algebra and $\tau$ is a trace.
For a trace $\tau$
\begin{parts}
\item $\tau$ is semi-finite if and only if $x\in M^+$ has a net $x_\alpha\in L^1(M,\tau)^+$ such that $x_\alpha\uparrow x$ strongly.
\item Let $\tau$ be normal and faithful. Then, $\tau$ is semi-finite if and only if
\[\tau(x)=\sup\{\,\tau(y):y\le x,\ y\in L^1(M,\tau)^+\,\}\quad\text{ for }\quad x\in M^+.\]
\end{parts}
\end{prb}

\begin{prb}[Uniformly hyperfinite algebras]
Let $A$ be a uniformly hyperfinite algebra.
\begin{parts}
\item Every matrix algebra admits a unique tracial state.
\item Every UHF algebra admits a unique tracial state.
\item Every hyperfinite 
\end{parts}
\end{prb}


measurable operators,
unbounded operators affilated with $M$,
noncommutative $L^p$ spaces for semi-finite con Neumann algebras,
noncommutative $L^p$ space for general von Neumann algebras: by Haagerup(crossed product), and by Kosaki-Terp(complex interpolation).

On semi-finite von Neumann algebras, measurable operators are affiliated.
On a finite von Neumann algebras, affiliated operators are measurable.


\begin{itemize}
\item density of $C(X)$ in $L^p(X,\mu)$
\item H\"older inequality
\item Radon-Nikodym
\item Riesz representation
\item Fubini
\item maximality of $L^\infty$ in $B(L^2)$
\end{itemize}



\section{Noncommutative integration}










\part{Actions}
\chapter{}
cyclic, discrete, abelian, flow

kahzdahn property T, compact

Rokhlin property

(kazhdan T, some properties like pointwise inner, etc.)





\chapter{Modular theory}






\section{Modular automorphisms}

\begin{rmk*}
Let $H$ be a self-adjoint operator of bounded from below.
Consider the modular operator as an analytic generator $\Delta=e^{-\beta H}$ of the modular automorphism $\sigma_t:=\Delta^{-i\frac t{\hbar\beta}}\cdot\Delta^{i\frac t{\hbar\beta}}=\Ad u_t$.
Then, $\Delta$ is an invertible trace-class operator.
The unitary operator
\[u_t^*=\Delta^{i\frac t{\hbar\beta}}=e^{-i\frac t\hbar H}\]
is called the \emph{propagator}.


The one-parameter automorphism $\sigma_t$ has the infinitesimal generator $i\frac1\hbar\ad_H=i\frac1\hbar[H,-]$.

\[\sigma_t:=\Ad_{u_t}=\Delta^{-i\frac t{\hbar\beta}}\cdot\Delta^{i\frac t{\hbar\beta}}=e^{i\frac t\hbar H}\cdot e^{-i\frac t\hbar H}=e^{i\frac t\hbar\ad_H}\]

\[\begin{tikzcd}[column sep={3.5em,between origins},row sep={2.5em,between origins}]
&A\ar{rr}{u_t=\Delta^{it}}&&A\ar{dd}{\lambda}\\
A'\ar[<->]{ur}{J}\ar[swap]{rr}{\Delta^{it}}\ar[swap]{dd}{\rho}&&A'\ar[<->,swap]{ur}{J}\ar{dd}{\rho}&\\
&&&\lambda(A)\\
\rho(A')\ar[swap]{rr}{\sigma_t=\Ad\Delta^{it}}&&\rho(A')\ar[<->,swap]{ur}{\Ad J}&.
\end{tikzcd}\]
\end{rmk*}


\begin{prb}[Unitary group]
\begin{parts}
\item $U(H)$ is strongly$^*$ complete.
\item $U(H)$ is not strongly complete.
\item $U(H)$ is weakly relatively compact.
\end{parts}
\end{prb}


Let $A$ be a C$^*$-algebra.
Then, $\bar{U(A)\cap B(1,r)}^{s*}=U(A'')\cap B(1,r)$.
In particular, $U(A)$ is strongly$^*$ dense in $U(A'')$.
(Kaplansky?)





\section*{Exercises}
\begin{prb}[Lower semi-continuous weights]
Let $\f$ be a weight on a C$^*$-algebra $A$.
The semi-cyclic representation of $\f$ is non-degenerate if either $A$ is unital or $\f$ is lower semi-continuous.
On a von Neumann algebra, there exists a weight that is not lower semi-continuous.
\end{prb}

\begin{prb}[Completely additive weights]
Let $\f$ be a \emph{completely additive} weight on a von Neumann algebra in the sense that for every orthogonal family $\{p_\alpha\}$ of projections we have $\f(\sum_\alpha p_\alpha)=\sum_\alpha\f(p_\alpha)$.
\begin{parts}
\item A completely additive state on a von Neumann algebra is normal.
\item A completely additive and lower semi-continuous weight on a commutative von Neumann algebra is normal.
\end{parts}
\end{prb}




\chapter{Classifications}





\part{Factors}


\chapter{Type III factors}


\chapter{Amenable factors}


\chapter{Type II factors}

\begin{prb}
Let $M$ be a von Neumann algebra.
Since every $\sigma$-weakly closed ideal of $M$ admits a unit $z$ so that we have $zM,Mz\subset I\subset zIz\subset zMz$, and it implies $z$ is a central projection of $M$.
A von Neumann algebra $M$ on $H$ is called a \emph{factor} if $M\cap M'=\C\id_H$, which is equivalent to that there are only two $\sigma$-weakly closed ideals of $M$.
In a factor, every ideal of $M$ is $\sigma$-weakly dense in $M$
\end{prb}


\section{}
\begin{prb}[Crossed products]
A p.m.p.~action $\Gamma\curvearrowleft(X,\mu)$ gives
\[\alpha:\Gamma\to\Aut(L^\infty(X)),\]
which has the Koopman representation
\[\sigma:\Gamma\to B(L^2(X)).\]
Then, we have a injective $*$-homomorphism
\[C_c(\Gamma,L^\infty(X))\to B(L^2(X)\otimes\ell^2(\Gamma))=B(\ell^2(\Gamma,L^2(X))),\]
whose element $s\mapsto x_s$ is written in
\[\sum_{s\in\Gamma,\ fin}(x_s\otimes1)(\sigma_s\otimes\lambda_s).\]

\begin{parts}
\item $L(\Gamma)$ is a II$_1$ factor if and only if $\Gamma$ is a i.c.c.~group.
\item $L^\infty(X)$ is a m.a.s.a.~of $L^\infty(X)\rtimes\Gamma$ if and only if the p.m.p.~action $\Gamma\curvearrowleft X$ is free.
\item $L^\infty(X)\rtimes\Gamma$ is a II$_1$ factor if and only if the p.m.p.~action $\Gamma\curvearrowleft X$ is ergodic.
\end{parts}
\end{prb}


\section{Ergodic theory}
\section{Rigidity theory}
\section{Free probability}
\section{}
Existentially closed II$_1$ factors






\part{Subfactors}


\chapter{Standard invariant}

The way how quantum systems are decomposed.
And has Galois analogy.

\begin{prb}[Jones index theorem]
A \emph{subfactor} of a factor $M$ is a factor $N$ containing $1_M$.
\end{prb}

Tensor categories and topological invariants of 3-folds.
Ergodic flows.


Ocneanu's paragroups
Popa's $\lambda$-lattices
Jones' planar algebras
Quantum entropy





\end{document}