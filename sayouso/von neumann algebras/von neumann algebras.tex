\documentclass{../../large}
\usepackage{../../ikhanchoi}


\begin{document}
\title{Von Neumann Algebras}
\author{Ikhan Choi}
\maketitle
\tableofcontents

\iffalse
injectivity
Connes embeddability
property Gamma
Connes' bicentralizer problem
Shlyakhtenko semicircular system
group stability
bimodule

\fi


\part{Fundamentals}


\chapter{}

\section{}


\begin{prb}[Support projections of operators]
Let $x$ be an element of a von Neumenna algebra $M$.
The \emph{left support projection} of $x$ is the minimal projection $p\in M$ such that $x=px$, denoted by $s_l(x)$.
The \emph{right support projection} of $x$ is defined as the left support projection of $x^*$.
The projections $s_l(x)$ and $1-s_r(x)$ are also called the \emph{range} and \emph{kernel} projections of $x$, respectively.
\begin{parts}
\item Support projections of $x$ uniquely exist.
\item $x^*yx=0$ if and only if $s_l(x)ys_l(x)=0$ for every $y\in M$.
\item We have $s_r(x)=s_r(x^*x)=s_r(|x|)$. In particular, $s_l(x)=s_r(x)$ if $x$ is normal.
\item If $x^*x\le y^*y$, then there is a unique $v\in M$ such that $x=vy$ and $s_r(v)\le s_l(y)$.
\item There is unique $v\in M$ such that the polar decomposition $x=v|x|$ holds and that $s_r(x)=v^*v$. Moreover, $x^*=v^*|x^*|$ and $s_l(x)=vv^*$. In particular, $s_l(x)$ and $s_r(x)$ are Murray-von Neumann equivalent.
\end{parts}
\end{prb}
\begin{pf}
(a)
Let $x\in M$.
Since $\im x=\im(xx^*)^{\frac12}$, we may assume $0\le x\le1$.
Then, $x^{2^{-n}}$ is an increasing sequence in $M$ bounded by one, so it converges strongly to some $p\in M_+$.
We can check $p^2=p$ by...
We can check $p$ is the range projection of $x$ by...


(e)
Since $x^*x\le|x|^*|x|$, there is a unique $v\in M$ such that $x=v|x|$ and $v=vs_l(|x|)=vs_r(x)$.
Then, $s_r(x)-v^*v=s_r(x)(1-v^*v)s_r(x)=0$ from $|x|(1-v^*v)|x|=|x|^2-|x|^2=0$, and $s_l(x)-vv^*=s_l(x)(1-vv^*)s_l(x)=0$ from $x^*(1-vv^*)x=|x|^2-|x|^2=0$.
The partial isometry $v$ is unique since $s_r(x)=v^*v$ implies $s_r(v)=s_r(v^*v)=s_r(s_r(x))=s_r(x)$.
Similarly, $s_l(v)=s_l(x)$.
The equality $xv^*=|x^*|$ follows from $xv^*=v|x|v^*\ge0$ and $|xv^*|^2=vx^*xv^*=v|x|^2v^*=xx^*=|x^*|^2$.
\end{pf}



\begin{prb}[Support projections of states]
\end{prb}


\begin{prb}[Cyclic and separating vectors]
A vector state is separating iff it is faithful.

If $M\subset B(H)$ admits a separating vector, then every normal state is a vector state. (T:V.1.12, J:7.1.4?)
\end{prb}



\begin{prb}[Countable decomposable von Neumann algebras]
Let $M$ be a von Neumann algebra.
A projection $p\in M$ is called \emph{countably decomposable} if mutually orthogonal nonzero projections majorized by $p$ are at most countable, and we say $M$ is \emph{countably decomposable} if the identity is.
The followings are all equivalent.
\begin{parts}
\item $M$ is countably decomposable.
\item $M$ admits a faithful normal state.
\item $M$ admits a module with a cyclic and separating vector.
\item The unit ball of $M$ is metrizable in the $\sigma$-strong topology.
\end{parts}
\end{prb}
\begin{pf}
\end{pf}

\begin{prb}[Separable von Neumann algebras]
Let $M$ be a von Neumann algebra.
The followings are all equivalent.
\begin{parts}
\item $M$ has the separable predual.
\item $M$ admits a faithful separable module.
\item $M$ is countably decomposable and countably generated.
\item The unit ball of $M$ is metrizable in the $\sigma$-weak topology.
\end{parts}
\end{prb}
\begin{pf}
\end{pf}




\chapter{Modular theory}


\section{Weights}


\begin{prb}[Ideals associated to weights]
left ideal, definition ideal
\end{prb}


\begin{prb}[Semi-cyclic representations]
Let $A$ be a C$^*$-algebra.
A \emph{semi-cyclic representation} is a representation $\pi:A\to B(H)$ together with a linear map $\psi:\fn\to H$ from a left ideal $\fn$ of $A$ into $H$ with dense range, such that $\pi(x)\psi(y)=\psi(xy)$ for $x\in A$ and $y\in\fn$.

For a semi-cyclic representation, if we denote $\fm:=\fn^*\fn$, then we have a bilinear form
\[\Theta:\fm\times\pi(A)'\to\C:(y^*x,z)\mapsto\<z\psi(x),\psi(y)\>.\]
With this, we can construct a linear map $\theta:\fm\to(\pi(A)')_*$ and its transpose $\theta^*:\pi(A)'\to\fm^\#$.

Consider a weight $\f$.
\begin{parts}
\item A (it might require some condition here if $A$ is not W$^*$) weight on $A$ defines a semi-cyclic representation and vice versa?
\item If $A=M$ is a von Neumann algebra, then we can let $\theta_*:\pi(M)'\to M_*$ to have $\theta^{**}=\theta$.
\item $\theta^*$ is bijective onto the space of linear functionals on $\fm$ absolutely continuous with respect to $\f$. (bounded Radon-Nikodym)
\end{parts}
\end{prb}



\begin{prb}[Normal weights]
Let $M$ be a von Neumann algebra.
Let $\omega$ be a weight of $M$.
\begin{parts}
\item $\omega$ is normal.
\item $\omega$ is $\sigma$-weakly lower semi-continuous.
\item $\omega$ is the supremum of a set of normal positive linear functionals.
\end{parts}
\end{prb}
\begin{pf}
(c)$\Rightarrow$(b)$\Rightarrow$(a) are clear.

(a)$\Rightarrow$(b)


Suppose first $M$ is countably decomposable so that $B$ is metrizable.




\end{pf}




\section{Hilbert algebras}



\begin{prb}
A \emph{left Hilbert algebra} is a $*$-algebra $A$ together with an inner product such that the left multiplication defines a nondegenerate $*$-homomorphism $\lambda:A\to B(H)$, where $H:=\bar A$, and the involution is a closable antilinear operator whose domain contains $A$.

If an involution is an isometry, then it is also a right Hilbert algebra, which is the unimodular case.

\end{prb}






\section{Traces}

\begin{prb}[Semi-finite and tracial von Neumann algebras]
Let $M$ be a von Neumann algebra.
We say $M$ is \emph{semi-finite} if it admits a faithful normal semi-finite trace, and \emph{tracial} if it admits a faithful normal tracial state.
\begin{parts}
\item regular representation and antilinear isometric involution $J$. $L(G)=\rho(G)'$
\item $M$ is semi-finite if and only if type III does not occur in the direct sum.

\item A factor $M$ has at most one tracial state, which is normal and faithful.
\item A factor is tracial if and only if it is type II$_1$.
\end{parts}
\end{prb}


\begin{prb}[Semi-finite traces]
Let $M$ be a von Neumann algebra and $\tau$ is a trace.
For a trace $\tau$
\begin{parts}
\item $\tau$ is semi-finite if and only if $x\in M^+$ has a net $x_\alpha\in L^1(M,\tau)^+$ such that $x_\alpha\uparrow x$ strongly.
\item Let $\tau$ be normal and faithful. Then, $\tau$ is semi-finite if and only if
\[\tau(x)=\sup\{\,\tau(y):y\le x,\ y\in L^1(M,\tau)^+\,\}\quad\text{ for }\quad x\in M^+.\]
\end{parts}
\end{prb}

\begin{prb}[Uniformly hyperfinite algebras]
Let $A$ be a uniformly hyperfinite algebra.
\begin{parts}
\item Every matrix algebra admits a unique tracial state.
\item Every UHF algebra admits a unique tracial state.
\item Every hyperfinite 
\end{parts}
\end{prb}


measurable operators,
unbounded operators affilated with $M$,
noncommutative $L^p$ spaces for semi-finite con Neumann algebras,
noncommutative $L^p$ space for general von Neumann algebras: by Haagerup(crossed product), and by Kosaki-Terp(complex interpolation).

On semi-finite von Neumann algebras, measurable operators are affiliated.
On a finite von Neumann algebras, affiliated operators are measurable.


\begin{itemize}
\item density of $C(X)$ in $L^p(X,\mu)$
\item H\"older inequality
\item Radon-Nikodym
\item Riesz representation
\item Fubini
\item maximality of $L^\infty$ in $B(L^2)$
\end{itemize}




\section{Modular automorphisms}


\begin{prb}[Unitary group]
\begin{parts}
\item $U(H)$ is strongly$^*$ complete.
\item $U(H)$ is not strongly complete.
\item $U(H)$ is weakly relatively compact.
\end{parts}
\end{prb}


Let $A$ be a C$^*$-algebra.
Then, $\bar{U(A)\cap B(1,r)}^{s*}=U(A'')\cap B(1,r)$.
In particular, $U(A)$ is strongly$^*$ dense in $U(A'')$.
(Kaplansky?)





\section*{Exercises}
\begin{prb}[Lower semi-continuous weights]
Let $\f$ be a weight on a C$^*$-algebra $A$.
The semi-cyclic representation of $\f$ is non-degenerate if either $A$ is unital or $\f$ is lower semi-continuous.
On a von Neumann algebra, there exists a weight that is not lower semi-continuous.
\end{prb}

\begin{prb}[Completely additive weights]
Let $\f$ be a \emph{completely additive} weight on a von Neumann algebra in the sense that for every orthogonal family $\{p_\alpha\}$ of projections we have $\f(\sum_\alpha p_\alpha)=\sum_\alpha\f(p_\alpha)$.
\begin{parts}
\item A completely additive state on a von Neumann algebra is normal.
\item A completely additive and lower semi-continuous weight on a commutative von Neumann algebra is normal.
\end{parts}
\end{prb}



\chapter{Direct integral}


\section{Commutative von Neumann algebras}

$\sigma$-field is a unital $\sigma$-ring.
$\sigma$-ideal is an ideal of a $\sigma$-ring which is a $\sigma$-ring.
$\sigma$-ideal is sometimes called the measure class because it corresponds to an equivalence class of measures up to absolute continuity.


\begin{prb}[Enhanced measurable spaces]
An \emph{enhanced measurable space} is a measurable space $(X,M)$ together with a $\sigma$-ideal $N$ of $M$.
A morphism between enhanced measurable spaces is a partial function $f:X_1\to X_2$ on a conegligible set such that $f^*$ induces a ring homomorphism $M_2/N_2\to M_1/N_1$.
\begin{parts}
\item Maharam's theoem: every enhanced measurable space is isomorphic to the disjoint union of $\{0,1\}^I$, where $I$ is an aribitrary cardinality...?
\item A $\sigma$-finite enhanced measurable space is isomorphic to a enhenced measurable space induced from a standard probability space...?
\item For $\sigma$-finite enhanced measurable spaces, a $*$-homomorphism $L^\infty(X_2)\to L^\infty(X_1)$ induces a morphism $X_1\to X_2$...?
\end{parts}
\end{prb}

\begin{prb}[Maharam classification]
\end{prb}


\begin{prb}
Noncommutative $L^p$ spaces for a general weight?
\begin{parts}
\item For $1\le p<\infty$, $C_0(X)\to L^p(X,\mu)$ is a bounded linear maps of dense range.
\item $L^\infty(X,\mu)$ is a m.a.s.a.~of $B(L^2(X,\mu))$.
\end{parts}
\end{prb}
\begin{pf}
We will show bounded linear maps $L^\infty(X,\mu)'\to M(X)$ and $L^\infty(X,\mu)\to M(X)$ have the same image.
Let $y\in L^\infty(X,\mu)'$ and define $\mu_y\in M(X)$ by
\[\mu_y(a):=\<\pi_\mu(a)y\psi_\mu,\psi_\mu\>.\]
We claim that $\mu_y$ factors through $L^1(X,\mu)$.
\end{pf}

Monotone convergence theorem states that a measure on a countably decomposable(?) enhanced measurable space $X$ uniquely defines a `countably' normal weight on the space of all measurable functions.
Note that a `countably' normal weight is normal on a countably decomposable von Neumann algebra.



\begin{prb}[Maximal commutative subalgebras]
A commutative von Neumann algebra $M$ is m.a.s.a.~if and only if it admits a cyclic vector.
In this case, $M$ is spatially isomorphic to some $L^\infty$(if separable?).
\end{prb}
\begin{pf}
\end{pf}

separable commutative von Neumann algebra is generated by one self-adjoint element.

hyperstonean sapces





\section{Tensor products}

$L^2(X,\mu,H)=L^2(X,\mu)\otimes H$
vector or operator-valued integrals

\section{Measurable fields}

\begin{prb}[Effros Borel structure]
\end{prb}

\begin{prb}[Decomposition of states]
\end{prb}


\section{Types}

finite, infinite, purely infinite, properly infinite, abelian projections




\bigskip
Type I factors.
It possess a minimal projection.
It is isomorphic to the whole $B(H)$ for some Hilbert space.
Therefore, it is classified by the cardinality of $H$.

Type II factors.
No minimal projection, but there are non-zero finite projections so that every projection can be ``halved'' by two Murray-von Neumann equivalent projections.

In type II$_1$ factors, the identity is a finite projection
Also, Murray and von Neumann showed there is a unique finite tracial state and the set of traces of projections is $[0,1]$.
Examples of II$_1$ factors include crossed product, tensor product, free product, ultraproduct.
Free probability theory attacks the free groups factors, which are type II$_1$.

In type II$_\infty$ factors.
There is a unique semifinite tracial state up to rescaling and the set of traces of projections is $[0,\infty]$.

In type III factors no non-zero finite projections exists.
Classified the $\lambda\in[0,1]$ appeared in its Connes spectrum, they are denoted by III$_\lambda$.
Tomita-Takesaki theory.
It is represented as the crossed product of a type II$_\infty$ factor and $\R$.

Amenability, equivalently hyperfiniteness is a very nice condition in von Neumann algebra theory.
Group-measure space construction can construct them.
There are unique hyperfinite type II$_1$ and II$_\infty$ factors, and their property is well-known.
Fundamental groups of type II factors, discrete group theory, Kazhdan's property (T) are used.

Tensor product factors such as Araki-Woods factors and Powers factors.









\part{Factors}

\chapter{Type II factors}

\begin{prb}
Let $M$ be a von Neumann algebra.
Since every $\sigma$-weakly closed ideal of $M$ admits a unit $z$ so that we have $zM,Mz\subset I\subset zIz\subset zMz$, and it implies $z$ is a central projection of $M$.
A von Neumann algebra $M$ on $H$ is called a \emph{factor} if $M\cap M'=\C\id_H$, which is equivalent to that there are only two $\sigma$-weakly closed ideals of $M$.
In a factor, every ideal of $M$ is $\sigma$-weakly dense in $M$
\end{prb}


\section{}
\begin{prb}[Crossed products]
A p.m.p.~action $\Gamma\curvearrowleft(X,\mu)$ gives
\[\alpha:\Gamma\to\Aut(L^\infty(X)),\]
which has the Koopman representation
\[\sigma:\Gamma\to B(L^2(X)).\]
Then, we have a injective $*$-homomorphism
\[C_c(\Gamma,L^\infty(X))\to B(L^2(X)\otimes\ell^2(\Gamma))=B(\ell^2(\Gamma,L^2(X))),\]
whose element $s\mapsto x_s$ is written in
\[\sum_{s\in\Gamma,\ fin}(x_s\otimes1)(\sigma_s\otimes\lambda_s).\]

\begin{parts}
\item $L(\Gamma)$ is a II$_1$ factor if and only if $\Gamma$ is a i.c.c.~group.
\item $L^\infty(X)$ is a m.a.s.a.~of $L^\infty(X)\rtimes\Gamma$ if and only if the p.m.p.~action $\Gamma\curvearrowleft X$ is free.
\item $L^\infty(X)\rtimes\Gamma$ is a II$_1$ factor if and only if the p.m.p.~action $\Gamma\curvearrowleft X$ is ergodic.
\end{parts}
\end{prb}


\section{Ergodic theory}
\section{Rigidity theory}
\section{Free probability}
\section{}
Existentially closed II$_1$ factors





\chapter{Type III factors}



\part{Subfactors}


\chapter{Standard invariant}

The way how quantum systems are decomposed.
And has Galois analogy.

\begin{prb}[Jones index theorem]
A \emph{subfactor} of a factor $M$ is a factor $N$ containing $1_M$.
\end{prb}

Tensor categories and topological invariants of 3-folds.
Ergodic flows.


Ocneanu's paragroups
Popa's $\lambda$-lattices
Jones' planar algebras
Quantum entropy



\part{Noncommutative probability}



\end{document}