\documentclass{../../large}
\usepackage{../../ikhanchoi}


\begin{document}
\title{Von Neumann Algebras}
\author{Ikhan Choi}
\maketitle
\tableofcontents

\iffalse
injectivity
Connes embeddability
Existentially closed II$_1$ factors
property Gamma
Connes' bicentralizer problem
Shlyakhtenko semicircular system
free probability
group stability

Direct integral!
Types!
Traces!

\fi


\part{}


\chapter{}


\section{Commutative von Neumann algebras}


\begin{prb}[Maximal commutative subalgebras]
A commutative von Neumann algebra $M$ is m.a.s.a.~if and only if it admits a cyclic vector.
In this case, $M$ is spatially isomorphic to some $L^\infty$(if separable?).
\end{prb}

separable commutative von Neumann algebra is generated by one self-adjoint element.

hyperstonean sapces





\section{Direct integral}

\begin{prb}[Decomposition of states]
\end{prb}

tensor product





\section{Types}

partial ordering on projection lattice

Type I factors.
It possess a minimal projection.
It is isomorphic to the whole $B(H)$ for some Hilbert space.
Therefore, it is classified by the cardinality of $H$.

Type II factors.
No minimal projection, but there are non-zero finite projections so that every projection can be ``halved'' by two Murray-von Neumann equivalent projections.

In type II$_1$ factors, the identity is a finite projection
Also, Murray and von Neumann showed there is a unique finite tracial state and the set of traces of projections is $[0,1]$.
Examples of II$_1$ factors include crossed product, tensor product, free product, ultraproduct.
Free probability theory attacks the free groups factors, which are type II$_1$.

In type II$_\infty$ factors.
There is a unique semifinite tracial state up to rescaling and the set of traces of projections is $[0,\infty]$.

In type III factors no non-zero finite projections exists.
Classified the $\lambda\in[0,1]$ appeared in its Connes spectrum, they are denoted by III$_\lambda$.
Tomita-Takesaki theory.
It is represented as the crossed product of a type II$_\infty$ factor and $\R$.

Amenability, equivalently hyperfiniteness is a very nice condition in von Neumann algebra theory.
Group-measure space construction can construct them.
There are unique hyperfinite type II$_1$ and II$_\infty$ factors, and their property is well-known.
Fundamental groups of type II factors, discrete group theory, Kazhdan's property (T) are used.

Tensor product factors such as Araki-Woods factors and Powers factors.




\chapter{Weights}

\section{Hilbert algebras}






\section{Traces}

\begin{prb}[Tracial von Neumann algebras]
Let $M$ be a von Neumann algebra.
We say $M$ is \emph{tracial} if there is a faithful normal tracial state on $M$.
A tracial factor is also called a \emph{II$_1$ factor}.
\begin{parts}
\item regular representation and anti-linear isometric involution $J$. $L(G)=\rho(G)'$
\item A factor $M$ has at most one tracial state, which is normal and faithful.
\end{parts}
\end{prb}





\begin{prb}[Semi-finite traces]
Let $M$ be a von Neumann algebra and $\tau$ is a trace.
For a trace $\tau$
\begin{parts}
\item $\tau$ is semi-finite if and only if $x\in M^+$ has a net $x_\alpha\in L^1(M,\tau)^+$ such that $x_\alpha\uparrow x$ strongly.
\item Let $\tau$ be normal and faithful. Then, $\tau$ is semi-finite if and only if
\[\tau(x)=\sup\{\,\tau(y):y\le x,\ y\in L^1(M,\tau)^+\,\}\quad\text{ for }\quad x\in M^+.\]
\end{parts}
\end{prb}

\begin{prb}[Uniformly hyperfinite algebras]
Let $A$ be a uniformly hyperfinite algebra.
\begin{parts}
\item Every matrix algebra admits a unique tracial state.
\item Every UHF algebra admits a unique tracial state.
\item Every hyperfinite 
\end{parts}
\end{prb}









\chapter{Modular theory}



\section{Automorphism groups}

\begin{prb}[Unitary group]
\begin{parts}
\item $U(H)$ is strongly$^*$ complete.
\item $U(H)$ is not strongly complete.
\item $U(H)$ is weakly relatively compact.
\end{parts}
\end{prb}


Let $A$ be a C$^*$-algebra.
Then, $\bar{U(A)\cap B(1,r)}^{s*}=U(A'')\cap B(1,r)$.
In particular, $U(A)$ is strongly$^*$ dense in $U(A'')$.
(Kaplansky?)


\part{Factors}
\chapter{Type II factors}
ergodic theory, rigidity theory

\chapter{Type III factors}



\part{Subfactors}


\chapter{Standard invariant}

The way how quantum systems are decomposed.
And has Galois analogy.

\begin{prb}[Jones index theorem]
A \emph{subfactor} of a factor $M$ is a factor $N$ containing $1_M$.
\end{prb}

Tensor categories and topological invariants of 3-folds.
Ergodic flows.


Ocneanu's paragroups
Popa's $\lambda$-lattices
Jones' planar algebras
Quantum entropy




\end{document}