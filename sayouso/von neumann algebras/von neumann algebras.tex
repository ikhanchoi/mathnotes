\documentclass{../../large}
\usepackage{../../ikhanchoi}


\begin{document}
\title{Von Neumann Algebras}
\author{Ikhan Choi}
\maketitle
\tableofcontents

\iffalse
injectivity
Connes embeddability
property Gamma
Connes' bicentralizer problem
Shlyakhtenko semicircular system
group stability
bimodule
noncommutative probability
\fi


\part{}




\chapter{Modular theory}



\section{Weights and semi-cyclic representations}

\begin{prb}[Weights]
Let $M$ be a von Neumann algebra.
A \emph{weight} is an additive homogeneous function $\f:M^+\to[0,\infty]$ in the sense that
\[\f(x+y)=\f(x)+\f(y),\qquad\f(tx)=t\f(x),\qquad x,y\in M^+,\ t\ge0,\]
where we use the convention $0\cdot\infty=0$.
Define
\[\fN_\f:=\{x\in M:\f(x^*x)<\infty\},\qquad\fA_\f:=\fN_\f^*\cap\fN_\f,\qquad\fM_\f:=\fN_\f^*\fN_\f.\]
It follows easily that $\fN_\f$ is a left ideal of $M$ equipped with a sesqui-linear form $\<x,y\>:=\f(y^*x)$ for $x,y\in\fN_\f$, $\fA_\f$ is a hereditary $*$-subalgebra of $M$, and $\fM_\f$ is a hereditary $*$-subalgebra of $M$ equipped with a linear functional which extends $\f$ by the polarization identity.

Let $\f$ be a weight on $M$.
We say $\f$ is \emph{faithful} if $\f(x)=0$ implies $x=0$ for $x\in M^+$, \emph{semi-finite} if $\fM_\f$or equivalently $\fN_\f$ is $\sigma$-weakly dense in $M$, and \emph{normal} if it is written as the supremum of some set of normal positive linear functionals.
\begin{parts}
\item Every von Neumann algebra admits a faithful semi-finite normal weight.
\item $\fM_\f^+=\{x^*x:x\in\fN_\f\}$ and $\fM_\f=\{y^*x:x,y\in\fN_\f\}$.
%\item opposite weight
\end{parts}
\end{prb}
\begin{pf}
(a)
Let $\{\omega_i\}_{i\in I}$ be a maximal family of normal states on $M$ such that the support projections $p_i:=s(\omega_i)$ are mutually orthogonal, taken with the Zorn lemma.
From the maximality, we have the $\sigma$-strong$^*$ limit $\sum_ip_i=1$.
Define a weight $\f:M^+\to[0,\infty]$ by
\[\f(x):=\sum_i\omega_i(x),\qquad x\in M^+.\]
It is faithful since $\f(x^*x)=0$ means $\omega_i(x^*x)=0$ for all $i$, which implies $xp_i=0$ and
\[\<x\xi,\xi\>=\<\sum_ip_i\xi,x^*\xi\>=\sum_i\<p_i\xi,x^*\xi\>=\sum_i\<xp_i\xi,\xi\>=0.\]
It is semi-finite since an increasing net $p_J:=\sum_{i\in J}p_i\in\fM_\f$ converges to the unit, where $J$ runs through all finite subsets of $I$.
It is normal by definition.

(b)
Let $z:=\sum_{i=1}^ny_i^*x_i\in\fM_\f^+$ for some $x_i,y_i\in\fN_\f$.
The polarization writes
\[z=\frac14\sum_{i=1}^n\sum_{k=0}^3i^k|x_i+i^ky_i|^2\]
so it follows from $z^*=z$ that
\[z=\frac12\sum_{i=1}^n(|x_i+y_i|^2-|x_i-y_i|^2)\le\frac12\sum_{i=1}^n|x_i+y_i|^2.\]
Since $\f(z)<\infty$ by $x_i+y_i\in\fN_\f$, if $x:=z^{\frac12}\in\fN_\f$, then $z=x^*x$.

Let $z:=\sum_{i=1}^ny_i^*x_i\in\fM_\f$ for some $x_i,y_i\in\fM_\f$.
Let $x:=(\sum_{i=1}^nx_i^*x_i)^{\frac12}\in\fN_\f$.
Since $x_i^*x_i\le x^2$, we have $v_i\in M$ such that $x_i=v_ix$.
If we let $y:=\sum_{i=1}^nv_i^*y_i\in\fN_\f$, then
\[z=\sum_{i=1}^ny_i^*x_i=\sum_{i=1}^ny_i^*v_ix=\Bigr(\sum_{i=1}^nv_i^*y_i\Bigr)^*x=y^*x.\qedhere\]
\end{pf}



\begin{prb}[Semi-cyclic representations]
Let $A$ be a $*$-algebra.
A \emph{semi-cyclic representation} is a representation $\pi:A\to B(H)$ together with a densely defined left $A$-linear operator $\Lambda:\dom\Lambda\subset A\to H$ of dense image.
\begin{align*}
\cF_\Lambda:=\{\omega\in M_*^+:\omega(x^*x)\le\|\Lambda(x)\|^2,\ x\in\dom\Lambda\},\\
\cG_\Lambda:=\{\omega\in M_*^+:\omega(x^*x)<\|\Lambda(x)\|^2,\ x\in\dom\Lambda\}.
\end{align*}
\begin{parts}
\item 
\end{parts}
\end{prb}



\begin{prb}[Normal semi-cyclic representations]
Let $M$ be a von Neumann algebra.
A semi-cyclic representation $(\pi,\Lambda)$ of $M$ is called \emph{normal} if $\pi$ is normal and $\Lambda$ is $\sigma$-weakly closed.

For a weight $\f$ on $M$, on the Hilbert space $H_\f$ constructed by separation and completion of the sesqui-linear form on $\fN_\f$, define $\pi_\f:M\to B(H_\f)$ by the left multiplication and $\Lambda_\f:\dom\Lambda_\f\subset H_\f\to H_\f$ as the canonical map with $\dom\Lambda_\f:=\fN_\f$.
Then, $(\pi_\f,\Lambda_\f)$ is clearly a semi-cyclic representation, called the \emph{Gelfand-Naimark-Segal representation} of $\f$.
Conversely, for a semi-cyclic representation $(\pi,\Lambda)$ of $M$ on a Hilbert space $H$, define $\f_\Lambda:=\sup\{\omega:\omega\in\cF_\Lambda\}$.
Then, $\f_\Lambda$ is clearly a weight.
\begin{parts}
\item If $\f$ is normal, then $(\pi_\f,\Lambda_\f)$ is normal, and $\f=\f_{\Lambda_\f}$.
\item If $(\pi,\Lambda)$ is normal, then $\f_\Lambda$ is normal, and there exists a unitary operator $u:H\to H_{\f_\Lambda}$ satisfying $\pi_{\f_\Lambda}=(\Ad u)\pi$ and $\Lambda_{\f_\Lambda}=u\Lambda$.
\item For a normal $\f$, $\f$ is faithful if and only if $\Lambda_\f$ is injective, and $\f$ is semi-finite if and only if $\Lambda_\f$ is $\sigma$-weakly densely defined.
\end{parts}
\end{prb}
\begin{pf}
(a)
We first show $\pi_\f$ is normal.
The proof is almost same as the normality of cyclic representation associated to normal states.
Consider the adjoint $\pi_\f^*:B(H)_*\to M^*$.
For a normal state $\omega\in B(H)_*$ of the form $\omega=\omega_{\Lambda_\f(x)}$ for some $x\in\fN_\f$, since
\[\pi_\f^*(\omega)(y)=\omega_{\Lambda_\f(x)}(\pi_\f(y))=\<\pi_\f(y)\Lambda_\f(x),\Lambda_\f(x)\>=\f(x^*yx),\qquad y\in M,\]
and since $\f$ is order continuous, we can see that $\pi_\f^*(\omega)$ is also order continuous, which means that it is contained in $M_*$.
Because the image of $\Lambda_\f$ is dense, the linear span of states of the form $\omega_{\Lambda_\f(x)}$ with $x\in\fN_\f$ is norm-dense in $B(H)_*$ by the inequality
\[\|\omega_\xi-\omega_\eta\|\le\|\xi-\eta\|\|\xi+\eta\|,\qquad\xi,\eta\in H.\]
Since $M_*$ is norm-closed $M^*$, so $\pi_\f^*$ maps normal states of $B(H)$ to normal states $M_*$.

Next, we show $\Lambda_\f$ is $\sigma$-weakly closed.
Let $\Gamma(\Lambda_\f):=\{(x,\Lambda_\f(x)):x\in\fN_\f\}\subset M\times H$ be the graph of $\Lambda_\f$.
Note that for closedness of convex subsets of $M\times H$ we do not have to distinguish $\sigma$-weak topology from $\sigma$-strong$^*$ topology on $M$ and weak topology from norm topology on $H$.
Suppose a net $x_i\in\fN_\f$ satisfies $x_i\to x$ $\sigma$-strongly$^*$ in $M$ and $\Lambda_\f(x_i)\to\xi$ stronlgy in $H$.
Since $\f$ is normal, we have
\[\f=\sup_{\omega\in\cF_\f}\omega,\qquad\cF_\f:=\cF_{\Lambda_\f}=\{\omega\in M_*^+:\omega\le\f\}.\]
It follows that $x\in\fN_\f$ from
\[\f(x^*x)=\sup_{\omega\in\cF_\f}\omega(x^*x)=\sup_{\omega\in\cF_\f}\lim_i\omega(x_i^*x_i)\le\lim_i\f(x_i^*x_i)=\lim_i\|\Lambda_\f(x_i)\|^2=\|\xi\|^2<\infty.\]
For fixed $\e>0$ and any $y\in\fN_\f$, if we take $\omega\in\cF_\f$ such that $\omega(y^*y)>\f(y^*y)-\e$ and let $h_\omega\in(M')_1^+$ be the Radon-Nikodym derivative of $\omega$ with respect to $\f$ in the commutant, then
\begin{align*}
\|h_\omega\Lambda(y)-\Lambda(y)\|^2
&=\<h_\omega^2\Lambda(y),\Lambda(y)\>-2\<h_\omega\Lambda(y),\Lambda(y)\>+\<\Lambda(y),\Lambda(y)\>\\
&\le-\<h_\omega\Lambda(y),\Lambda(y)\>+\<\Lambda(y),\Lambda(y)\>\\
&=-\omega(y^*y)+\f(y^*y)<\e
\end{align*}
and the convergence $\Lambda(x_i)\to\xi$ in norm imply
\[\<\xi-\Lambda_\f(x),\Lambda_\f(y)\>\approx\<\xi-\Lambda_\f(x),h_\omega\Lambda_\f(y)\>\approx\<\Lambda(x_i)-\Lambda(x),h_\omega\Lambda(y)\>=\omega(y^*(x_i-x)),\qquad\e\to0,\]
and the $\sigma$-weak convergence $x_i\to x$ implies $\omega(y^*(x_i-x))\to0$, we can conclude that $\<\xi-\Lambda_\f(x),\Lambda_\f(y)\>$ vanishes for all $y\in\fN_\f$.
Since $\Lambda_\f$ has dense image, we finally have $\xi=\Lambda_\f(x)$.
Therefore, the graph of $\Lambda_\f$ is $\sigma$-weakly closed.

For $\omega\in M_*^+$, we have $\omega\le\f$ if and only if $\omega(x^*x)\le\f(x^*x)=\|\Lambda_\f(x)\|^2$ for $x\in\dom\Lambda_\f$, which means that $\cF_\f=\cF_{\Lambda_\f}$ and $\f=\f_{\Lambda_\f}$.

(b)
The weight $\f_\Lambda$ is obviously normal by definition.
Let $x\in\dom\Lambda$.
Then, we have
\[\f_\Lambda(x^*x)=\sup_{\omega\in\cF_\Lambda}\omega(x^*x)\le\|\Lambda(x)\|^2\]
by definition of $\f_\Lambda$ and $\cF_\Lambda$, so we get $x\in\fN_{\f_\Lambda}=\dom\Lambda_{\f_\Lambda}$ and $\|\Lambda_{\f_\Lambda}(x)\|\le\|\Lambda(x)\|$.
Conversely, let $x\in\dom\Lambda_{\f_\Lambda}$.
We have $\f_\Lambda(x^*x)=\|\Lambda_{\f_\Lambda}(x)\|^2<\infty$.
We claim that if $\f_\Lambda(x^*x)\le1$, then $x\in\dom\Lambda$ and $\|\Lambda(x)\|^2\le1$.
As a corollary, the inequality $\|\Lambda(x)\|\le\|\Lambda_{\f_\Lambda}(x)\|$ follows by scaling from this claim.

We prove the claim.
Since the graph of $\Lambda$ is $\sigma$-weakly closed by assumption, and since the projection $\dom\Lambda\times H_1\to\dom\Lambda$ is a closed map due to the tube lemma and the weak compactness of $H_1$, the image
\[\{y:\|\Lambda(y)\|\le1,\ y\in\dom\Lambda\}\subset M\]
of the graph of $\Lambda$ under this projection is $\sigma$-weakly closed.
Since the positive part of this set is also $\sigma$-weakly closed and the square root is strongly continuous, if we temporarily consider a sufficiently large representation of $M$ in which every normal state is a vector state so that a strong and $\sigma$-strong topology coincide on $M$, we can conclude that the inverse image under the square root
\[C:=\{y^*y:\|\Lambda(y)\|\le1,\ y\in\dom\Lambda\}\subset M\]
is $\sigma$-weakly closed.
Note that $C$ is also convex, and $\omega\in M_*^+$ is contained in $\cF_\Lambda$ if and only if
\[\sup_{y^*y\in C}\omega(y^*y)\le1.\]

If $x^*x\notin C$, then, by the Hahn-Banach separation, there is $\omega\in M_*^{sa}$ such that
\[\sup_{y^*y\in C}\omega(y^*y)\le1<\omega(x^*x),\]
and since we may assume $\omega$ is positive, we have $\omega\in\cF_\Lambda$.
Thus, we have $\omega(x^*x)\le\f_\Lambda(x^*x)\le1$ by definition of $\f_\Lambda$, which leads a contradiction, so we get $x^*x\in C$.
Now there is $y\in\dom\Lambda$ satisfying $x^*x=y^*y$ so that there is $v\in M$ with $x=vy$, and because $\dom\Lambda$ is a left ideal of $M$, we finally have $x\in\dom\Lambda$, and $x^*x\in C$ says that $\|\Lambda(x)\|\le1$, hence the claim follows.

In conclusion, we have $\dom\Lambda_{\f_\Lambda}=\dom\Lambda$ and $\|\Lambda_{\f_\Lambda}(x)\|=\|\Lambda(x)\|$ on it.
Since the images of $\Lambda$ and $\Lambda_{\f_\Lambda}$ are dense in $H$ and $H_{\f_\Lambda}$ respectively, we can define the unitary $u:H\to H_{\f_\Lambda}$ such that $u\Lambda(y):=\Lambda_{\f_\Lambda}(y)$, and hence that $u\pi(x)u^*=\pi_{\f_\Lambda}(x)$ because
\[u^*\pi_{\f_\Lambda}(x)u\Lambda(y)=u^*\pi_{\f_\Lambda}(x)\Lambda_{\f_\Lambda}(y)=u^*\Lambda_{\f_\Lambda}(xy)=\Lambda(xy)=\pi(x)\Lambda(y),\]
for $x,y\in\dom\Lambda_{\f_\Lambda}$.
\end{pf}


\begin{prb}[Countability of von Neumann algebras]
Let $M$ be a von Neumann algebra.
A projection $p\in M$ is called \emph{countably decomposable} if mutually orthogonal nonzero projections majorized by $p$ are at most countable, and we say $M$ is \emph{countably decomposable} if the identity is.

\begin{parts}
\item $M$ is countably decomposable if and only if it admits a faithful normal state.
\item $M$ is countably decomposable if and only if $M_1$ is metrizable in the $\sigma$-strong topology.
\item $M$ has separable predual if and only if it is countably decomposable and countably generated.
\item $M$ has separable predual if and only if it faithfully acts on a separable Hilbert space.
\item $M$ has separable predual if and only if $M_1$ is metrizable in the $\sigma$-weak topology.
\end{parts}
\end{prb}
\begin{pf}
\end{pf}


\begin{prb}[Normal weights]
Let $M$ be a von Neumann algebra, and $\f$ be a weight on $M$.
\begin{parts}
\item If $\f$ is order continuous, then it is $\sigma$-weakly lower semi-continuous.
\item If $\f$ is $\sigma$-weakly lower semi-continuous, then it is normal.
\end{parts}
\end{prb}
\begin{pf}
Since the product topology of the $\sigma$-weak topology on $\fN_\f$ and the weak topology on $H$ is the weak$^*$ topology on the dual Banach space $(M_*\oplus_1H)^*\cong M\oplus_\infty H$, it suffices to show the graph is closed in this weak$^*$ topology.
In the spirit of the Krein-S\v mulian theorem, consider the unit ball
\[\Gamma(\Lambda_\f)_1=\{(x,\Lambda_\f(x)):x\in\fN_\f,\ \|x\|\le1,\ \|\Lambda_\f(x)\|\le1\},\]
which is also convex.
\end{pf}

\begin{prb}[Lower semi-continuous weights]
Let $\f$ be a weight on a C$^*$-algebra $A$.
The semi-cyclic representation of $\f$ is non-degenerate if either $A$ is unital or $\f$ is lower semi-continuous.
On a von Neumann algebra, there exists a weight that is not lower semi-continuous.
\end{prb}






\section{Hilbert algebras}

\begin{prb}[Left and right Hilbert algebras]
A \emph{left Hilbert algebra} is an involutive inner product algebra $A$, where the involution is denoted by $\xi\mapsto\xi^\sharp$, such that if we let $H$ be the completion of $A$, then
\begin{enumerate}[(i)]
\item the left multiplication defines a $*$-homomorphism $L_0:A\subset H\to B(H)$ which is non-degenerate,
\item the involution defines an anti-linear operator $S_0:A\subset H\to H$ which is closable,
\end{enumerate}
precisely defined such that $L_0(\xi)\eta:=\xi\eta$ and $S_0\xi:=\xi^\sharp$ for $\xi,\eta\in A$.

For $\eta\in H$, we define $R(\eta):A\to H$ and $F\eta:A\to\C$ such that $R(\eta)\xi:=L_0(\xi)\eta$ and $F\eta(\xi)=\<\eta,S_0\xi\>$ for $\xi\in A$, and by introducing
\[\dom R:=\{\eta\in H:R(\eta)\text{ is bounded}\},\qquad\dom F:=\{\eta\in H:F\eta\text{ is bounded}\},\]
we recognize $R:\dom R\subset H\to B(H)$ and $F:\dom F\subset H\to H$ as a densely defined anti-homomorphism and a densely defined anti-linear operator.
Let $A':=\dom R\cap\dom F$.
We will show $A'$ is dense in $H$.

Symmetrically, for $\xi\in H$, we can define $L(\xi):A'\to H$ and $S\xi:A'\to\C$ such that $L(\xi)\eta:=R(\eta)\xi$ and $S\xi(\eta)=\<\xi,F\eta\>$ for $\eta\in A'$, and by introducing
\[\dom L:=\{\xi\in H:L(\xi)\text{ is bounded}\},\qquad\dom S:=\{\xi\in H:S\xi\text{ is bounded}\},\]
we recognize $L:\dom L\subset H\to B(H)$ and $S:\dom S\subset H\to H$ as a densely defined homomorphism and a densely defined anti-linear operator.
Let $A'':=\dom L\cap\dom S$.
If $A=A''$, then we say $A$ is \emph{full}.
\begin{parts}
\item $A'$ is an involutive algebra with the involution $\eta\mapsto\eta^\flat:=F\eta$ and the multiplication $(\eta,\zeta)\mapsto\eta\zeta:=R(\zeta)\eta$.
\item $A'$ is a full right Hilbert algebra in the sense that $R$ is non-degenerate and $F$ is closed.
\item $L$ and $S$ are injective, and are the $\sigma$-weak closure and closure of $L_0$ and $S_0$, resepctively.
\item $A_0$ is a core of $L$ and $S$.
\end{parts}
\end{prb}
\begin{pf}
(a)


\end{pf}


\begin{prb}[Semi-cyclic representations and Hilbert algebras]
Let $M$ be a von Neumann algebra on a Hilbert space $H$.

\begin{parts}
\item For a full left Hilbert algebra $A\subset H$ such that $\bar A=H$ and $L(A)''=M$, if we define $\Lambda:\dom\Lambda\subset M\to H$ as the inverse of $L:\dom L\subset H\to M$, then $\Lambda$ is a densely defined $\sigma$-weakly closed left $M$-linear operator of dense image such that $A=\Lambda((\dom\Lambda)^*\cap(\dom\Lambda))$.
\item For a densely defined $\sigma$-weakly closed left $M$-linear operator $\Lambda:\dom\Lambda\subset M\to H$ of dense image, if we define $A:=\Lambda((\dom\Lambda)^*\cap(\dom\Lambda))$, then $A$ is a natural full left Hilbert algebra such that $\bar A=H$ and $L(A)''=M$ such that $\Lambda$ is the inverse of $L$.
\end{parts}
\end{prb}
\begin{pf}
(a)
The domain and image of $\Lambda$ is dense in $M$ and $H$ because the image and domain of $L$ is dense in $M$ and $H$.
The graph of the $\Lambda$ is $\sigma$-weakly closed since its inverse $L$ is $\sigma$-weakly closed.
To check the left $M$-linearity, let $x\in M$ and $\xi\in\dom L$.
Since $\dom\Lambda$ is a $\sigma$-weakly dense $*$-subalgebra of $M$, it admits an approximate unit $e_i\in(\dom\Lambda)_1^+$ with $e_i\to1$ $\sigma$-strongly$^*$.
Because $\dom\Lambda$ is hereditary, we have a net $e_ixe_i\in\dom\Lambda$ satisfying $e_ixe_i\to x$ $\sigma$-strongly$^*$.
Then, we have $e_ixe_i\xi\to x\xi$ and a $\sigma$-weak limit
\[L(e_ixe_i\xi)=L(L(\Lambda(e_ixe_i))\xi)=L(\Lambda(e_ixe_i)\xi)=L(\Lambda(e_ixe_i))L(\xi)=e_ixe_iL(\xi)\to xL(\xi),\]
so the closedness of $L$ implies that $L(x\xi)=xL(\xi)$.

(b)



(c?d?)
The main difficulty is the closability of $\Lambda(x)\mapsto\Lambda(x^*)$.


\end{pf}


\begin{prb}[Hilbert algebras by cyclic separating vectors]
\end{prb}

\begin{prb}[Hilbert algebras by locally compact groups]
\end{prb}




\begin{prb}[Modular operator and modular conjugation]
Let $A$ be a left Hilbert algebra.
\[S=J\Delta^{\frac12}.\]

Tomita algebra
analytic elements
\end{prb}


\begin{prb}[Tomita-Takesaki commutation theorem]
Wick rotation
The \emph{imaginary time} is $\frac i\hbar t$.
Let $H$ be a self-adjoint operator of bounded from below.
Consider
\[\Delta_\beta:=e^{-\beta H},\qquad u_t:=e^{\frac i\hbar tH}.\]
Then, $\Delta$ is an invertible trace-class operator.
The unitary operator $u_t^*$ is called the \emph{propagator}.


The one-parameter automorphism $\sigma_t$ has the formal infinitesimal generator $\frac i\hbar\ad H=\frac i\hbar[H,-]$ with
\[\sigma_t:=\Ad u_t=e^{\frac i\hbar tH}\cdot e^{-\frac i\hbar tH}=e^{\frac i\hbar t\ad_H}.\]

\[\begin{tikzcd}[column sep={3.5em,between origins},row sep={2.5em,between origins}]
&A\ar{rr}{u_t=\Delta^{it}}&&A\ar{dd}{\lambda}\\
A'\ar[<->]{ur}{J}\ar[swap]{rr}{\Delta^{it}}\ar[swap]{dd}{\rho}&&A'\ar[<->,swap]{ur}{J}\ar{dd}{\rho}&\\
&&&\lambda(A)\\
\rho(A')\ar[swap]{rr}{\sigma_t=\Ad\Delta^{it}}&&\rho(A')\ar[<->,swap]{ur}{\Ad J}&.
\end{tikzcd}\]
\begin{parts}
\item Fourier inversion
\item $(e^{-s}+\Delta)^{-1}:A'\to A\cap\dom F$.
\item commutation theorem
\end{parts}
\end{prb}





\section{Standard forms}


\begin{prb}[Standard forms]
Induction
\end{prb}

\begin{prb}[Existence of standard forms]
\end{prb}

\begin{prb}[Uniqueness of standard forms]
\end{prb}

\begin{prb}[Unitary implementation]
Powers-St\o rmer inequality
\end{prb}





\section{Kubo-Martin-Schwinger weights}
Radon-Nikodym

centralizers and commuting weights





\section*{Exercises}

\begin{prb}[Completely additive weights]
Let $\f$ be a \emph{completely additive} weight on a von Neumann algebra in the sense that for every orthogonal family $\{p_\alpha\}$ of projections we have $\f(\sum_\alpha p_\alpha)=\sum_\alpha\f(p_\alpha)$.
\begin{parts}
\item A completely additive state on a von Neumann algebra is normal.
\item A completely additive and lower semi-continuous weight on a commutative von Neumann algebra is normal.
\end{parts}
\end{prb}















\chapter{}


\section{Commutative von Neumann algebras}


\begin{prb}
For a normal weight $\f$ of a commutative von Neumanna algebra $M$, $\f$ is semi-finite if and only if for every projection $p\in M$ with $\f(p)=\infty$, there is another projection $q\in M$ such that $q\le p$ and $0<\f(q)<\infty$.
\end{prb}
\begin{pf}
($\Rightarrow$)
Take $e\in M^+$ such that $0<\f(ep)<\infty$.
Approximate $ep$ from below by the simple functions $s=\sum_ia_ip_i$ such that $0<\f(ep)-\e<\f(s)\le\f(ep)<\infty$ by the normality, where $a_i\ge0$ and $p_i$ are mutually orthogonal.
Then, $q:=\sum_ip_i$ satisfies the property.

($\Leftarrow$)
Suppose $\fm$ is not $\sigma$-weakly dense in $M$.
Its $\sigma$-weak closure is given by $pMp$ for some projection $p\in M$.
Then, $1-p$ is a counterexample of the contradictory assumption.
\end{pf}



Monotone convergence theorem states that a measure on a countably decomposable(?) enhanced measurable space $X$ uniquely defines a `countably' normal weight on the space of all measurable functions.
Note that a `countably' normal weight is normal on a countably decomposable von Neumann algebra.



\begin{prb}[Maximal commutative subalgebras]
A commutative von Neumann algebra $M$ is m.a.s.a.~if and only if it admits a cyclic vector.
In this case, $M$ is spatially isomorphic to some $L^\infty$(if separable?).

$L^\infty(\mu)$ is a m.a.s.a.~of $B(L^2(\mu))$.
\end{prb}
\begin{pf}
We will show bounded linear maps $L^\infty(\mu)'\to M(X)$ and $L^\infty(\mu)\to M(X)$ have the same image.
Let $y\in L^\infty(\mu)'$ and define $\mu_y\in M(X)$ by
\[\mu_y(a):=\<\pi_\mu(a)y\psi_\mu,\psi_\mu\>.\]
We claim that $\mu_y$ factors through $L^1(\mu)$.
\end{pf}

separable commutative von Neumann algebra is generated by one self-adjoint element.

\begin{prb}[semi-finite lower semi-continuous weights]
Let $\Omega$ be a locally compact Hausdorff space.
There is a one-to-one correspondence between semi-finite lower semi-continuous weights of $C_0(\Omega)$ and positive linear functionals on $C_c(\Omega)$.

A semi-finite lower semi-continuous weight on a C$^*$-algebra $A$ is uniquely extended to a semi-finite normal weight on $A^{**}$, and to $\pi(A)''$, where $\pi$ is the semi-cyclic representation.
\end{prb}
\begin{pf}
Let $\f$ be a positive linear functional on $C_c(\Omega)$.
We can extend it to $\f:C_0(\Omega)^+\to[0,\infty]$ by letting
\[\f(f):=\sup\{\f(g):g\le f,\ g\in C_c(\Omega)\}.\]
Since $C_0(\Omega)\cap L^1(\f)$ is dense in $C_0(\Omega)$ by compact truncation, $\f$ is semi-finite.
If $f_n\in C_0(\Omega)$ such that $\int|f_n|\le1$ and $f_n\to f$ uniformly, then taking compact $K\subset\Omega$ such that $\int_{K^c}|f|<\e$, we can prove $\int|f|\le1$, so $\f$ is lower semi-continuous.
By taking the restriction on $C_c(\Omega)$, we can reconstrcut the original linear functional.

Conversely, let $\f$ be a semi-finite lower semi-continuous weight on $C_0(\Omega)$.
If there is a point $\omega\in\Omega$ such that $\f(f)=\infty$ whenever $f\in C_0(\Omega)^+$ and $f(\omega)>0$, then $\fm\subset C_0(\Omega\setminus\{\omega\})$, which contradicts to the assumption that $\f$ is semi-finite.
Then, using compactness, we can prove $C_c(\Omega)\subset\fm$.
Now we can check $\f(f)=\sup\{\f(g):g\le f,\ g\in C_c(\Omega)\}$ by constructing an increasing net in $C_c(\Omega)$ that converges to $f$ uniformly.
\end{pf}




\begin{prb}
The set of projections is a complete orthomodular lattice.
If $M$ is commutative, then the set of projections is a complete boolean algebra.
\end{prb}

\begin{itemize}
\item commutative ring - distributive lattice - coherent locale
\item clean ring+$\alpha$ - boolean algebra - stone space
\item - complete boolean algebra - stonean space
\item commutative von Neumann algebra - localizable boolean algebra - hyperstonean space
\end{itemize}


A \emph{frame} is a partially ordered set $F$ that admits a finite meets and arbitrary joins, and for any $a\in F$ the map $F\to F:x\mapsto x\wedge a$ preserves suprema.
A \emph{locale} is an object of the opposite category of frames.
An element of a locale is called \emph{open}.


A locale is called \emph{coherent} if the set of compact opens is closed under finite meets and every open is the join of compact opens, i.e.~generates opens.
It is known that a coherent locale is spatial.
\begin{enumerate}[(i)]
\item $X$ is a coherent space.
\item $X$ is a (compact) sober space such that the set of compact open subsets is closed under finite intersections and forms a base.
\item $X$ is homeomorphic to the underlying space of an affine scheme.
\end{enumerate}
A morphism of $\mathrm{CohLoc}$ is a compact open preserving local morphism.
A morphism of $\mathrm{DistLat}$ is just a lattice morphism.
We can consider the compact open functor $\mathrm{CohLoc}\to\mathrm{DistLat}^{\mathrm{op}}$ and the ideal functor $\mathrm{DistLat}^{\mathrm{op}}\to\mathrm{CohLoc}$.
They form a categorical equivalence between the category of coherent locales and the opposite category of distributive lattices with lattice morphisms (i.e.~preserving finite meets and joins).

A locale is called \emph{Stone} if it is a coherent locale in which every open is the join of all subopens of it.
\begin{enumerate}[(i)]
\item $X$ is a Stone space.
\item $X$ is totally disconnected and compact Hausdorff.
\item $X$ is a compact zero-dimesional sober space.
\item $X$ is a compact zero-dimesional Hausdorff space.
\item $X$ is coherent and Hausdorff.
\end{enumerate}
A morphism of $\mathrm{StoneLoc}$ is a compact open (clopen) preserving locale morphism.
A morphism of $\mathrm{BoolLat}$ is just a lattice morphism.





A locale is called \emph{Stonean} if it is a Stone locale in which the (unique) complement of any element is clopen.
A morphism of $\mathrm{StoneanLoc}$ is an open locale morphism.
A morphism of $\mathrm{CpltBoolLat}$ is a continuous lattice morphism.

A locale is called \emph{Hyperstonean} if...
A boolean lattice is called \emph{localizable} if it is complete, and the identity is approximated by elements admitting a faithful continuous valuation on their compression.
The category $\mathrm{LBAlg}$ admits small products, and the products are preserved by the forgetful functor $\mathrm{LBAlg}\to\mathrm{BAlg}$.



\[*\]

\begin{prb}[Boolean algebra]
A \emph{boolean ring} is a ring in which every element is idempotent, which is automatically commutative.
A \emph{boolean algebra} is a unital boolean ring.
A \emph{boolean lattice} is a complemented distributive lattice.
\begin{parts}
\item There is a one-to-one correspondence between boolean rings and boolean lattices.
\item The category of boolean algebras with unital homomorphisms and the category of Stone spaces with continuous maps are equivalent.
\item The category of complete boolean algebras with order continuous unital homomorphisms and the category of Stonean spaces with open continuous maps are equivalent.
In the Stonean space, the join and meet is realized as the closure of union and the interior of intersection, respectively.
\end{parts}
\end{prb}


\begin{prb}[Measurable algebras]
For a boolean algebra, existences of sequential suprema and sequential infima are equivalent.
A boolean algebra is called a \emph{measurable algebra} if it is order $\sigma$-complete.
\begin{parts}
\item (Loomis-Sikorski representation) Every measurable algebra $\cL$ is realized as $\cM/\cM\cap\cN$ from a enhanced measurable space $(X,\cM,\cN)$.
\item (Dedekind completion) Every boolean algbera $\cL$
\end{parts}
\end{prb}
\begin{pf}
(a)
Let $X$ be the Stone space of $\cL$, $\cM$ the set of clopen subsets, and $\cN$ the set of meager sets.
Then, $\cM$ is a $\sigma$-algebra on $X$ and $\cN$ is a $\sigma$-ideal of $X$.

(b)
complete extension of order continuous homomorphisms and universal property.
regular open algebra of $X$.
\end{pf}



\begin{prb}[Measure algebras]
A \emph{measure} on a measurable algebra $\cL$ is a completely additive monotone function $\cL\to[0,\infty]$.
A \emph{measure algebra} is a measurable algebra together with a faithful measure.

Let $(X,M,\mu)$ be a measure space, which is not necessarily faithful.
There is a canonically associated measure algebra $(\cM/\cM\cap\cN,\mu)$, which is faithful, where $\cN:=\mu^{-1}(0)$.

\end{prb}


\begin{prb}[Localizable measure algebras]

For a measure space $(X,\cM,\mu)$, the completion always does not change the measure algebra, and the complete locally determined version
\[\tilde\cM:=\{E\subset X:E\cap A\in\cM\triangle\cN,\ \mu(A)<\infty\},\qquad\tilde\mu(E):=\sup\{\mu(E\cap A):\mu(A)<\infty\}\]
does not change the measure algebra when the measure space is localizble.


\begin{parts}
\item Every localizable measure algebra is obtained from a compact decomposable measure space.
\item A $\sigma$-finite measure space is compact decomposable.
\end{parts}
\end{prb}









\begin{itemize}
\item $\mathrm{HSTop}$: hyperstonean spaces with open continuous maps,
\item $\mathrm{HSLoc}$: hyperstonean locales with open localic maps,
\item $\mathrm{LBAlg}$: localizable boolean lattices with continuous lattice homomorphisms,
\item $\mathrm{CW^*Alg}$: commutative W$^*$ algebras with normal $*$-homomorphisms.
\end{itemize}

\[\begin{tikzcd}
\mathrm{HSTop} \rar[shift left]{top}&
\mathrm{HSLoc} \lar[shift left]{sp}\rar[shift left]{clopen}&
\mathrm{LBAlg}^\op=\mathrm{MLoc} \lar[shift left]{ideal}\rar[shift left]{L^\infty}&
\mathrm{CW^*Alg}^\op \lar[shift left]{proj}
\end{tikzcd}\]


\begin{prb}[]
\begin{parts}
\item Construction of projection lattice functor.
\item Construction of $L^\infty$ functor.
\item Equivalence.
\end{parts}
\end{prb}
\begin{pf}
(b)
Let $L$ be a measurable locale.
For $\F\in\{\R,\C\}$, define $L^\infty(L,\F)$ to be the set of all bounded localic maps $x:L\to\F$, which are given by the opposite of lattice homomorphism $x^{-1}:\mathrm{top}(\F)\to L$ which preserves finite meets and arbitrary joins, and factors an open ball of $\F$.
We can define a normed $*$-algebra structure on $L^\infty(L,\F)$ such that
\begin{gather*}
(x+y)^{-1}(U):=\bigvee_{U_x+U_y\subset U}(x^{-1}(U_x)\wedge y^{-1}(U_y)),\qquad(xy)^{-1}(U):=\bigvee_{U_xU_y\subset U}(x^{-1}(U_x)\wedge y^{-1}(U_y)),\\
(x^*)^{-1}(U):=x^{-1}(\{\bar z:z\in U\}),\qquad\|x\|=\inf\{\sup_{z\in U}|z|:x^{-1}(U)=1\in L\}.
\end{gather*}
Using the axioms of locales, for example that the meet with a single element preserves arbitrary joins, we can manually check that $L^\infty(L,\F)$ is a commutative normed $*$-algebra, and in particular the C$^*$-identity when $\F=\C$.
Furthermore, since $L^\infty(L,\C)$ is the complexification of $L^\infty(L,\R)$, if we prove $L^\infty(L,\R)$ has a predual, then the completeness with respect to norm follows automatically, so $L^\infty(L,\C)$ becomes a C$^*$-algebra with a predual, i.e.~a von Neumann algebra.

Define $L^1(L,\R)$ the real linear span of continuous valuations on $L$, equipped with the variation norm.
Recall that a continuous valuation is a monotone function $v:L\to[0,\infty)$ such that $v(0)=0$ and $v(p)+v(q)=v(p\vee q)+v(p\wedge q)$, which preserves directed suprema.
Note that $L^\infty(L,\R)$....
\end{pf}




$\sigma$-field is a unital $\sigma$-ring.
$\sigma$-ideal is an ideal of a $\sigma$-ring which is a $\sigma$-ring.
$\sigma$-ideal is sometimes called the measure class because it corresponds to an equivalence class of measures up to absolute continuity.


\begin{prb}[Enhanced measurable spaces]
An \emph{enhanced measurable space} is a measurable space $(X,M)$ together with a $\sigma$-ideal $N$ of $M$.
A morphism between enhanced measurable spaces is a partial function $f:X_1\to X_2$ on a conegligible set such that $f^*$ induces a ring homomorphism $M_2/N_2\to M_1/N_1$.
\begin{parts}
\item Maharam's theoem: every enhanced measurable space is isomorphic to the disjoint union of $\{0,1\}^I$, where $I$ is an aribitrary cardinality...?
\item A $\sigma$-finite enhanced measurable space is isomorphic to a enhenced measurable space induced from a standard probability space...?
\item For $\sigma$-finite enhanced measurable spaces, a $*$-homomorphism $L^\infty(X_2)\to L^\infty(X_1)$ induces a morphism $X_1\to X_2$...?
\end{parts}
\end{prb}


Premaps: 

Strict maps: an a.e.~equivalence class of premaps. For each strict map with non-empty codomain, there is a everywhere defined representative.


Quotients on morphisms:
\[\mathrm{PreEMS}\to\mathrm{StrictEMS}\to\mathrm{EMS}.\]
Fully faithful functors:
\[\mathrm{REMS}\to\mathrm{CDEMS}\to\mathrm{DEMS}\to\mathrm{LEMS}, \mathrm{DEMS}\to\mathrm{LDEMS}.\]
The functor $\mathrm{LEMS}\to\mathrm{LBAlg}:(X,M,N)\mapsto M/N$ is a well-defined essentially surjective functor, which is fully faithful on the full subcategory $\mathrm{CDEMS}$.

We say a enhanced measurable space is \emph{decomposable} or \emph{strictly localizable} if it is isomorphic to the small coproduct of countably decomposable enhanced measurable spaces.

$\mathrm{DEMS}$ is a full subcategory of $\mathrm{PreEMS}$, but not of $\mathrm{EMS}$, and we embed it to $\mathrm{LDEMS}$




\subsubsection*{Complete Boolean algebras}





\subsubsection*{Measure algebras}

\begin{defn*}
An (complete) \emph{enhanced measurable space} is a set $X$ together with a $\sigma$-algebra $\cM$ on $X$ and a $\sigma$-ideal $\cN$ of $\cP(X)$.
\end{defn*}
\begin{defn*}
A \emph{measurable algebra} is another name of a $\sigma$-complete Boolean algebra.
A \emph{measure} on a measurable algebra $\cA$ is a countably additive monotone function $\mu:\cA\to[0,\infty]$.
A \emph{measure algbera} is a measurable algebra together with a measure.
We consider continuous homomorphisms and measure-preserving continuous homomorphisms as morphisms of the categories of measurable algebras and measure algebras.
\end{defn*}

In the below diagrams, morphisms of each category are supposed to be as follows: negligible reflecting measurable maps between enhanced measurable spaces, continuous homomorphisms between measurable algebras, and unital normal $*$-homomorphisms between von Neumann algebras.
The arrow $\twoheadrightarrow$ means an essentially surjective functor.

\[\begin{tikzcd}
\tab{measure space\\$(X,\cM,\mu)$} \rar\dar &
\tab{measure algebra\\$((\cM\triangle\cN)/\cN,\mu)$} \dar \\
\tab{enhanced measurable space\\$(X,\cM,\cN)$} \rar[->>] &
\tab{measurable algebra\\$(\cM\triangle\cN)/\cN$} &
\tab{commutative\\von Neumann algebra} \lar[hook]
\end{tikzcd}\]
above functors are fully faithful?
Essential surjectivity of the horizontal functors are by the Loomis-Sikorski representation theorem.
It states that every $\sigma$-complete Boolean algebra is isomorphic to $\cM/\cM\cap\cN$, where $(X,\cM,\cN)$ is an enhanced measurable space.

\begin{prb}[Measure space]
\end{prb}


\subsubsection*{Maharam classification}



\begin{prb}[Localizable measure algebras]
A measure algbera $(\cA,\mu)$ is called \emph{localizable} if $\cA$ is complete and $\mu$ is semi-finite.
A measurable algebra $\cA$ is called \emph{localizable} if it is complete and it admits a semi-finite measure.
\[\begin{tikzcd}
\tab{localizable\\measure space\\$(X,\cM,\mu)$} \rar[->>]\dar[->>] &
\tab{localizable\\measure algebra\\$((\cM\triangle\cN)/\cN,\mu)$} \dar[->>]\rar[<->] &
\tab{commutative\\von Neumann algebra\\with a f.s.n.~weight} \dar[->>] \\
\tab{localizable\\enhanced measurable space\\$(X,\cM,\cN)$} \rar[->>] &
\tab{localizable\\measurable algebra\\$(\cM\triangle\cN)/\cN$} \rar[<->] &
\tab{commutative\\von Neumann algebra}
\end{tikzcd}\]

\begin{parts}
\item A $\sigma$-finite measure algebra is localizable.
\end{parts}
\end{prb}

\begin{prb}[Maharam classification]
Let $\cA$ be an localizable measurable algebra.
A \emph{Maharam type} or just a \emph{type} of $\cA$ is the smallest cardinal $\tau(\cA)$ of any dense subset of $\cA$.
If $\tau(\cA)=\tau(\cA\wedge a)$ for all non-zero $a\in\cA$, then we say $\cA$ is \emph{Maharam homogeneous}.
For an infinite cardinal $\kappa$, the \emph{Maharam component} of type $\kappa$ is the supremum $e_\kappa$ of any non-zero elements $a\in\cA$ such that $A\wedge a$ is Maharam homogeneous of type $\kappa$.

A \emph{cellularity} of a Boolean algebra $\cA$ is the supremum $c(\cA)$ of the cardinalities of any disjoint subset of $\cA\setminus\{0\}$.
Note that the cellularity is either zero or infinite if $\cA$ is atomless, and $\cA\wedge e_\kappa$ is atomless if $\kappa$ is infinite.
We define a cellularity function $c:\mathrm{InfCard}\to\mathrm{InfCard}\cup\{0\}$ such that $c(\kappa):=c(\cA\wedge e_\kappa)$.

\begin{parts}
\item 
\item All Maharam homogeneous probability algebras of same type are isomorphic.
\item A measure algebra $(\cB_\kappa,\mu)$ from the measure space $(\{0,1\}^\kappa,\mu)$ is Maharam homogeneous probability algebra of type $\kappa$.
\item A
\end{parts}
\end{prb}



A disjoint union and product of localizable measurable algebras is passed to the direct product and the tensor product.
Every commutative von neumann algebra can be realized as $L^\infty$ of the disjoint union, or equivalently, the direct product of $L^\infty$, of countably decomposable enhanced measurable spaces $\{0,1\}^\kappa$.
Every countably decomposable commutative von Neumann algebra is the tensor product of $\ell^\infty$'s.

The invariants of localizable measure algebras:
\[m:\mathrm{InfCard}\to\mathrm{Card},\qquad n:(0,\infty)\to\mathrm{Card}.\]

The invariants of localizable measurable algebras:
\[c:\mathrm{InfCard}\to\mathrm{InfCard}\cup\{0\},\qquad a\in\mathrm{Card},\]
where
\[c(\kappa):=\begin{cases}\omega&,0<m(\kappa)<\infty\\m(\kappa)&,\text{ otherwise}\end{cases},\qquad a:=\sum_{r\in(0,\infty)}n(r).\]

We have $m(\kappa)\le c(\kappa)$ and $m(\kappa)<c(\kappa)$ only if $c(\kappa)=\omega$.

For an atomless commutative von Neumann algebra $M$ (no minimal projections), countable decomposability says that there is a countable collection of cardinals $S$ such that $c(\kappa)=\omega$ if $\kappa\in S$ and $c(\kappa)=0$, i.e.~the countable product of $L^\infty(\{0,1\}^\kappa)$, and separability says that $c(\omega)=\omega$ and $c(\kappa)=0$ for $\kappa>\omega$, i.e.~$L^\infty(\{0,1\}^\omega)$.



\begin{prb}[$\sigma$-finite and standard measure spaces]
Let $(X,\mu)$ be a localizable measure space.
\begin{parts}
\item $(X,\mu)$ is $\sigma$-finite if and only if $L^\infty(\mu)$ is countably decomposable.
\item $(X,\mu)$ is standard if and only if $L^\infty(\mu)$ is separable.
\end{parts}
\end{prb}


\begin{prb}[Radon measures]
\end{prb}










\section{Tensor products}

$L^2(X,\mu,H)=L^2(X,\mu)\otimes H$
vector or operator-valued integrals


\section{Direct integrals}

\begin{prb}[Effros Borel structure]
\end{prb}

\begin{prb}[Decomposition of states]
\end{prb}




\section{Projections}

abelian, finite, purely infinite, properly infinite projections

central projection = union of components
central support = a kind of minimal union of components
centrally orthogonal

\begin{prb}
We say a von Neumann algebra is of \emph{type I} if every non-zero central projection has a non-zero abelian subprojection, \emph{type II} if has no non-zero abelian projection and every non-zero central projection has a non-zero finite subprojection, and \emph{type III} if it has no non-zero finite projection.

For a type II von Neumann algebra, it is called of \emph{type II$_1$} if it is finite, and of \emph{type II$_\infty$} if it has no central finite projections.

A projection is called \emph{semi-finite} if every non-zero central subprojection has a non-zero finite subprojection.
\end{prb}

\begin{itemize}
\item type I = semi-abelian
\item type II = purely non-abelian semi-finite
\item type III= purely infinite
\end{itemize}
V.1.35. For a purely non-abelian von Neumann algebra, every projection is the sum of two equivalent orthogonal projections.

\begin{prb}[Type I]
Let $M$ be a von Neumann algebra of type I.
Then, there are families $\{M_\kappa\}_\kappa$ and $\{H_\kappa\}_\kappa$ of commutative von Neumann algebras $M_\kappa$ and Hilbert spaces $H_\kappa$ satisfying $\dim H_\kappa=\kappa$, indexed by cardinals $\kappa$, such that
\[M\cong\bigoplus_{\kappa\in\mathrm{Card}}M_\kappa\bar\otimes B(H_\kappa).\]
If $M$ is a factor, then $M\cong B(H)$ for a Hilbert space $H$.
\end{prb}




finite: f.n.~center-valued trace, \\
semi-finite: f.s.n.~extended center-valued trace, 2.34

tensor product and types





Type I factors.
It possess a minimal projection.
It is isomorphic to the whole $B(H)$ for some Hilbert space.
Therefore, it is classified by the cardinality of $H$.

Type II factors.
No minimal projection, but there are non-zero finite projections so that every projection can be ``halved'' by two Murray-von Neumann equivalent projections.

In type II$_1$ factors, the identity is a finite projection
Also, Murray and von Neumann showed there is a unique finite tracial state and the set of traces of projections is $[0,1]$.
Examples of II$_1$ factors include crossed product, tensor product, free product, ultraproduct.
Free probability theory attacks the free groups factors, which are type II$_1$.

In type II$_\infty$ factors.
There is a unique semifinite tracial state up to rescaling and the set of traces of projections is $[0,\infty]$.

In type III factors no non-zero finite projections exists.
Classified the $\lambda\in[0,1]$ appeared in its Connes spectrum, they are denoted by III$_\lambda$.
Tomita-Takesaki theory.
It is represented as the crossed product of a type II$_\infty$ factor and $\R$.

\begin{itemize}
\item Type III$_{0<\lambda<1}$ factor: unique $N\rtimes_\alpha\Z$, $N$ II$_\infty$ factor,
\item Type III$_1$ factor: unique $N\rtimes_\alpha\R$, $N$ II$_\infty$ factor,
\item Type III$_0$ factor: one-to-one correpondence with nontransitive ergodic flows.
\end{itemize}

Amenability, equivalently hyperfiniteness is a very nice condition in von Neumann algebra theory.
Group-measure space construction can construct them.
There are unique hyperfinite type II$_1$ and II$_\infty$ factors, and their property is well-known.
Fundamental groups of type II factors, discrete group theory, Kazhdan's property (T) are used.

Tensor product factors such as Araki-Woods factors and Powers factors.








cyclic group actions implies the classification of injective factors.

\begin{itemize}
\item cyclic groups: Connes (II, III$<1$), Haagerup (III$_1$),
\item finite groups: Jones (II$_1$)
\item discrete amenable groups: Ocneanu (II$_1$), 
\item property T:
\item one-parameter:
\item compact abelian: Takesaki duality?
\end{itemize}

Type I:
Every automorphism of type I factor is inner.
Cocycle conjugacy classes of actions of $\Gamma$ on the injective type I factor $B(\ell^2)$ is correponded to $H^2(\Gamma,\T)$.

approximately inner automorphisms
centrally trivial automorphisms
pointwise inner automorphisms

minimal action






\chapter{Non-commutative integral}



\begin{prb}[Semi-finite and tracial von Neumann algebras]
Let $M$ be a von Neumann algebra.
We say $M$ is \emph{semi-finite} if it admits a faithful semi-finite normal trace, and \emph{tracial} if it admits a faithful normal tracial state.
\begin{parts}
\item regular representation and antilinear isometric involution $J$. $L(G)=\rho(G)'$
\item $M$ is semi-finite if and only if type III does not occur in the direct sum.

\item A factor $M$ has at most one tracial state, which is normal and faithful.
\item A factor is tracial if and only if it is type II$_1$.
\end{parts}
\end{prb}


\begin{prb}[Semi-finite traces]
Let $M$ be a von Neumann algebra and $\tau$ is a trace.
For a trace $\tau$
\begin{parts}
\item $\tau$ is semi-finite if and only if $x\in M^+$ has a net $x_\alpha\in L^1(M,\tau)^+$ such that $x_\alpha\uparrow x$ strongly.
\item Let $\tau$ be normal and faithful. Then, $\tau$ is semi-finite if and only if
\[\tau(x)=\sup\{\,\tau(y):y\le x,\ y\in L^1(M,\tau)^+\,\}\quad\text{ for }\quad x\in M^+.\]
\end{parts}
\end{prb}

\begin{prb}[Uniformly hyperfinite algebras]
Let $A$ be a uniformly hyperfinite algebra.
\begin{parts}
\item Every matrix algebra admits a unique tracial state.
\item Every UHF algebra admits a unique tracial state.
\item Every hyperfinite 
\end{parts}
\end{prb}


measurable operators,
unbounded operators affilated with $M$,
noncommutative $L^p$ spaces for semi-finite con Neumann algebras,
noncommutative $L^p$ space for general von Neumann algebras: by Haagerup(crossed product), and by Kosaki-Terp(complex interpolation).

On semi-finite von Neumann algebras, measurable operators are affiliated.
On a finite von Neumann algebras, affiliated operators are measurable.


\begin{itemize}
\item density of $C(X)$ in $L^p(X,\mu)$
\item H\"older inequality
\item Radon-Nikodym
\item Riesz representation
\item Fubini
\item maximality of $L^\infty$ in $B(L^2)$
\end{itemize}




The sequentiality of a net is required for the relation between the almost everywhere convergence and the local convergence in measure.
In particular, an almost everywhere convergent net might not converges locally in measure.
Monotone, bounded, dominated convergence theorems are true for nets that converge locally in measure.

For a localizable measure space $(X,\mu)$, $L_\loc^0(\mu)\cong S(L^\infty(\mu))$.






\part{Constructions}




\chapter{Group actions}



\section{Crossed products}
(Pettis integral and one-parameter case is dealt with in functional analysis, on general dual pairs)

group algebras

Dual weights

Takesaki duality

\section{Spectral analysis}






\section{Classification of group actions}

cyclic, discrete, abelian, flow

kahzdahn property T, compact

Rokhlin property

(kazhdan T, some properties like pointwise inner, etc.)


\chapter{}
\chapter{}




\part{Factors}


\chapter{Type III factors}
\section{Connes invariants}

\section{Flow of weights}



\chapter{Amenable factors}

Injectivity and semi-discreteness are compatible with direct sum.

\begin{prb}
\begin{parts}
\item If $M$ is injective, then $M'$ is injective.
\item If $M$ is injective and semi-finite, then $M$ is semi-discrete.
\item If $M$ is injective, then $M$ is semi-discrete.
\end{parts}
\end{prb}
\begin{pf}
$M_{\mathrm{II}_\infty}$ is the union of type II$_1$ corners?
So we may assume $M$ is of type II$_1$?

Let $\tau$ be a faithful normal tracial state on $M$.
Since $M$ is injective, $\tau$ is amenable.
\end{pf}

\chapter{Type II factors}

\begin{prb}
Let $M$ be a von Neumann algebra.
Since every $\sigma$-weakly closed ideal of $M$ admits a unit $z$ so that we have $zM,Mz\subset I\subset zIz\subset zMz$, and it implies $z$ is a central projection of $M$.
A von Neumann algebra $M$ on $H$ is called a \emph{factor} if $M\cap M'=\C\id_H$, which is equivalent to that there are only two $\sigma$-weakly closed ideals of $M$.
In a factor, every ideal of $M$ is $\sigma$-weakly dense in $M$
\end{prb}


\section{}
\begin{prb}[Crossed products]
A p.m.p.~action $\Gamma\curvearrowleft(X,\mu)$ gives
\[\alpha:\Gamma\to\Aut(L^\infty(X)),\]
which has the Koopman representation
\[\sigma:\Gamma\to B(L^2(X)).\]
Then, we have a injective $*$-homomorphism
\[C_c(\Gamma,L^\infty(X))\to B(L^2(X)\otimes\ell^2(\Gamma))=B(\ell^2(\Gamma,L^2(X))),\]
whose element $s\mapsto x_s$ is written in
\[\sum_{s\in\Gamma,\ fin}(x_s\otimes1)(\sigma_s\otimes\lambda_s).\]

\begin{parts}
\item $L(\Gamma)$ is a II$_1$ factor if and only if $\Gamma$ is a i.c.c.~group.
\item $L^\infty(X)$ is a m.a.s.a.~of $L^\infty(X)\rtimes\Gamma$ if and only if the p.m.p.~action $\Gamma\curvearrowleft X$ is free.
\item $L^\infty(X)\rtimes\Gamma$ is a II$_1$ factor if and only if the p.m.p.~action $\Gamma\curvearrowleft X$ is ergodic.
\end{parts}
\end{prb}


\section{Ergodic theory}
\section{Rigidity theory}
\section{Free probability}
\section{}
Existentially closed II$_1$ factors






\part{Subfactors}


\chapter{Standard invariant}

The way how quantum systems are decomposed.
And has Galois analogy.

\begin{prb}[Jones index theorem]
A \emph{subfactor} of a factor $M$ is a factor $N$ containing $1_M$.
\end{prb}

Tensor categories and topological invariants of 3-folds.
Ergodic flows.


Ocneanu's paragroups
Popa's $\lambda$-lattices
Jones' planar algebras
Quantum entropy





\end{document}