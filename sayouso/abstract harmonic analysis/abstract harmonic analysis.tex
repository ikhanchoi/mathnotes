\documentclass{../../large}
\usepackage{../../ikhanchoi}


\begin{document}
\title{Abstract Harmonic Analysis}
\author{Ikhan Choi}
\maketitle
\tableofcontents



\part{}

\chapter{Hopf $*$-algebras}
\section{}
Multiplier Hopf $*$-algebras

Algebraic quantum groups

Hopf C$^*$-algebras

idempotent ring assumption


\chapter{Locally compact groups}


\section{}

\begin{prb}[Non-$\sigma$-finite measures]
Following technical issues are important
\begin{parts}
\item The Fubini theorem
\item The Radon-Nikodym theorem
\item The dual space of $L^1$ space
\end{parts}
\end{prb}

\begin{prb}[Existence of the Haar measure]
\end{prb}

\begin{prb}[Left and right uniformities]
\end{prb}

\begin{prb}[Modular functions]
\end{prb}

\begin{prb}[Uniformly continuous functions]
$G$ acts on $C_{lu}(G)$ and $L^1(G)$ continuously with respect to the point-norm topology.
A function on $G$ is left uniformly continuous if and only if it is written as $f*x$ for some $f\in L^1(G)$ and $x\in L^\infty(G)$.
$g\in C_c(G)$ is two-sided uniformly continuous.
\end{prb}



\begin{prb}[Structures on a locally compact group]
For a locally compact group $G$, consider $A:=C_c(G)$.
It is a left Hilbert algebra by the existence of the left Haar measure
\[(f*g)(s):=\int f(t)g(t^{-1}s)\,dt,\qquad\<f,g\>:=\int\bar{g(s)}f(s)\,ds,\qquad f^\sharp(s):=\delta(s^{-1})\bar{f(s^{-1})}.\]
and is a commutative counital multiplier Hopf $*$-algebra by the group structure.
\[(fg)(s):=f(s)g(s),\qquad\Delta f(s,t)=f(st),\qquad f^*(s):=\bar{f(s)},\qquad Sf(s)=f(s^{-1}).\]
Since the image of the comultiplication does not belong to $C_c(G)\otimes C_c(G)$, we need to do something unless $G$ is finite.
They satisfy a compatibility condition $\<fg,h\>=\<f,g^*h\>$.

With the integral notation $\lambda(f)=\int f(s)\lambda_s\,ds$, we can write

We start from this structures.

From now on, we are going to exclude any measure theory and the theory of non-commutative $L^p$ spaces.
First, we have the completion $H=:L^2(G)$.
Consider two representations
\[\lambda:(C_c(G),*,^\sharp)\to B(L^2(G)),\qquad m:(C_c(G),\cdot,^*)\to B(L^2(G)).\]
\begin{parts}
\item $\lambda$ is well-defined.
\item $m$ is well-defined.
\end{parts}
\end{prb}
\begin{pf}
The multiplication representation $m$ is well-defined because for $f\in C_c(G)$ we have $f^*f\in C_c(G)\subset L^2(G)$ so
\[\|m(f)g\|^2=\<fg,fg\>=\<f^*fg,g\>,\qquad g\in C_c(G).\]
\end{pf}








\section{}

We use the notation $L^p(G)$ for the non-commutative $L^p$-spaces constructed with the left Haar measure on $G$, which is a faithful semi-finite normal weight of $L^\infty(G)$.
The predual of $L^\infty(G)$ can be identified with $L^1(G)$.
The regular representation on $L^2(G)$ is the Gelfand-Naimark-Segal representation associated with the left Haar measure.

Density of $C_c(G)$?

\begin{prb}[Convolution algebra]
Let $G$ be a locally compact group.
Then, $L^1(G)$ is a hermitian Banach $*$-algebra such that
\[(f*g)(x):=(f\otimes g)\Delta(x),\qquad f,g\in L^1(G),\ x\in L^\infty(G).\]
Importance of $L^1$ instead of $C_c$: representation equivalence and predual.
\begin{parts}
\item $L^1(G)$ has a two-sided approximate unit in $C_c(G)$.
\item $\alpha:G\to\Aut(L^1(G))$ is point-norm continuous.
\item $\lambda:G\to U(L^2(G))$ and $\lambda:L^1(G)\to B(L^2(G))$ are strongly continuous.
\item Convolution inequalities.
\item Representation theory equivalence.
\end{parts}
\begin{pf}
Let $(U_\alpha)$ be a directed set of open neighborhoods of the identity $e$ of $G$.
By the Urysohn lemma, there is $e_\alpha\in C_c(U)^+$ such that $\|e_\alpha\|_1=1$ for each $\alpha$.
We claim that $e_\alpha$ is a two-sided approximate unit for $L^1(G)$.
Suppose $g\in C_c(G)$, which is two-sided uniformly continuous.
For any $\e>0$, take $\alpha_0$ such that $\|g-\lambda_sg\|<\e$ and $\|g-\rho_sg\|<\e$ for all $s\in U_\alpha$ for $\alpha\succ\alpha_0$.
Then, we have
\begin{align*}
\|e_\alpha*g-g\|_1
&=\int|e_\alpha*g(t)-g(t)|\,dt\le\iint e_\alpha(s)|g(s^{-1}t)-g(t)|\,ds\,dt\\
&=\int_{U_\alpha}e_\alpha(s)\|\lambda_sg-g\|_1\,ds<\e\int e_\alpha(s)\,ds\le\e,
\end{align*}
and
\begin{align*}
\|g*e_\alpha-g\|_1
&=\int|g*e_\alpha(s)-g(s)|\,ds\le\iint|g(t)-g(s)|e_\alpha(t^{-1}s)\,dt\,ds\\
&=\iint|g(t)-g(ts)|e_\alpha(s)\,dt\,ds=\int\|g-\rho_sg\|_1e_\alpha(s)\,ds<\e\int e_\alpha(s)\,ds\le\e,
\end{align*}
and they imply $\lim_\alpha\|e_\alpha*g-g\|_1=\lim_\alpha\|g*e_\alpha-g\|_1=0$.
We can approximate $f\in L^1(G)$ with compactly supported continuous functions by the $\e/3$ argument.
\end{pf}

\end{prb}

Note that we have
\begin{align*}
|\<\lambda(\xi)\eta,\zeta\>|^2
&=|\iint\xi(t)\eta(t^{-1}s)\bar{\zeta(s)}\,ds\,dt|^2\\
&\le\iint|\xi(t)||\eta(t^{-1}s)|^2\,ds\,dt\cdot\iint|\xi(t)||\zeta(s)|^2\,ds\,dt\\
&=\|\xi\|_1^2\|\eta\|_2^2\|\zeta\|_2^2
\end{align*}
and
\begin{align*}
|\<\rho(\xi)\eta,\zeta\>|^2
&=|\iint\eta(t)\xi(t^{-1}s)\bar{\zeta(s)}\,ds\,dt|^2\\
&\le\iint|\xi(t^{-1}s)||\eta(t)|^2\,ds\,dt\cdot\iint|\xi(t^{-1}s)||\zeta(s)|^2\,ds\,dt\\
&=\|\xi\|_1\|F\xi\|_1\|\eta\|_2^2\|\zeta\|_2^2
\end{align*}
imply
\[\|\lambda(\xi)\|_{2\to2}\le\|\xi\|_1,\qquad\|\rho(\xi)\|_{2\to2}\le\sqrt{\|\xi\|_1\|F\xi\|_1}.\]
The equalities do not hold, consider $\|\lambda(\xi)\|=\|\hat\xi\|_\infty$ if $G=\R$.









\[\begin{tikzcd}
G \ar{r} & M(G) & \, & \,\\
L_1(G) \ar[hook]{ur}\ar[hook]{r}\ar[dashed]{d}{*} & C^*(G) \ar[two heads]{r}\ar[dashed]{d}{*} & C_r^*(G) \ar[hook]{r}\ar[dashed]{d}{*} & L(G) \ar[dashed]{d}{*\text{ with }\sigma w}\\
L^\infty(G) & B(G) \ar[hook]{l} & C_r^*(G)^* \ar[hook]{l} & A(G) \ar[hook]{l}\\
& C_0(G) \ar[hook]{ul} & &
\end{tikzcd}\]



\section{}



\begin{prb}[Plancherel theorem]
With the left Haar measure on a Banach $*$-algebra $L^1(G)$ or $M(G)$, we want to construct a faithful semi-finite normal weight called the \emph{Planceherel weight}, and describe the corresponding semi-cyclic representation and left Hilbert algebra for $C^*_r(G)$ and $W^*_r(G)$.

By analyze the decomposition of the canonical representation of $C_r^*(G)$ and $W_r^*(G)$ in $B(L^2(G))$....?
Then, we can consider a unitary operator from $L^2(G)$ to the square integrable section space of a bundle on $\hat G$...

\end{prb}
\begin{pf}

\end{pf}


\begin{prb}[Fourier algebra]
The Fourier algebra is the algebra $A(G)$ of matrix coefficients of the regular representation, i.e.~the space spanned by functions $s\mapsto\<\lambda(s)\xi,\xi\>$ for $\xi\in L^2(\hat G)$.

It is a dense Banach subalgebra of $C_0(G)$ such that $A(G)\to W_r^*(G)_*:\eta^*\xi\mapsto\omega_{\xi,\eta}$ is an isometric isomorphism.

positive definite functions


\end{prb}
\begin{pf}

\end{pf}







\begin{prb}[Locally compact abelian groups]
Let $G$ be a locally compact abelian group.
Since every irreducible representation of a locally compact abelian group is one-dimensional, we introduce the notation $\<s,p\>=p_s\in\T$.
The \emph{Fourier transform} of an integrable function $f\in L^1(\hat G)$ is defined as
\[\cF f(p):=\int_G\bar{\<s,p\>}f(s)\,ds,\qquad ,\ p\in\hat G,\]
and the \emph{Fourier-Stieltjes transform} of a finite complex measure $\mu\in M(G)$ is defined as
\[\cF\mu(p):=\int_G\bar{\<s,p\>}\,d\mu(s),\qquad p\in\hat G.\]

\begin{parts}
\item The compact open topology of $C(G)$ and the weak$^*$ topology of $L^\infty(G)$ coincide on $\hat G$, which provides a locally compact abelian group.
\item The Fourier transform defines a $*$-homomorphism $\cF:L^1(G)\to C_0(\hat G)$ which is injective with norm dense image.
\item The Fourier-Stieltjes transform defines a $*$-homomorphism $\cF:M(G)\to L^\infty(\hat G)$ which is weakly$^*$ continuous and injective with weakly$^*$ dense image in $C_b(\hat G)$.
\item 
The canonical homomorphism $\Phi:G\to\hhat G$ defined such that $\Phi(s)(p)=\<s,p\>$ for $s\in G$ and $p\in\hat G$ is a topological isomorphism.
\item Fourier inversion..?
\end{parts}
\end{prb}
\begin{pf}
(b)
The Fourier transform is realized as the composition
\[\cF:L^1(G)\to C_r^*(G)\to C^*(G)\to C_0(\hat G)\to C_0(\hat G).\]

The first map is the extension of the regular representation $\lambda:C_c(G)\to B(L^2(G))$ using the inequality $\|\lambda(f)\|\le\|f\|_{L^1}$.
It has dense image by the definition of $C_r^*(G)$, the norm closure of the image of $\lambda$.
It is also injective because if $f\in L^1(G)$ satisfies $\<\lambda(f)\xi,\eta\>=0$ for all $\xi,\eta\in L^2(G)$, then it means that $\<f,a\>=0$ for every $a\in A(G)$ by definition of the Fourier algebra, which implies that $f=0$ because $L^1(G)\subset M(G)=C_0(G)^*$ and $A(G)$ is dense in $C_0(G)$.

The second map is the inverse of the canonical map $C^*(G)\to C_r^*(G)$ taken thanks to the amenability of locally compact abelian groups.
The third map is the Gelfand transform, which is a $*$-isomorphism for commutative C$^*$-algebras.
The last map is the induced map from the inverse map of the domain $\hat G$, clearly a $*$-isomorphism.

Therefore, the Fourier transform $\cF:L^1(G)\to C_0(\hat G)$ is an injective $*$-homomorphism with with dense image.

(c)
The Fourier-Stieltjes transform is realized as the composition
\[\cF:M(G)\to W_r^*(G)\to L^\infty(\hat G).\]

The first map is the weakly$^*$ continuous extension of the regular representation $\lambda:C_c(G)\to B(L^2(G))$, and the weak$^*$ continuity follows from the fact that the Fourier algebra $A(G)$ belong to $C_0(G)$.
The injectivity follows from the density of $A(G)$ in $C_0(G)$ as we did in the part (b), and the weakly$^*$ dense image is also by the definition of $W_r^*(G)$.

The second map is obtained by taking double commutant for the $*$-isomorphism $C_r^*(G)\to C_0(\hat G)$ composed from the last three maps described in the part (b), which is in fact the restriction the $*$-isomorphism $B(L^2(G))\to B(L^2(\hat G))$ defined by the Fourier transform $L^2(G)\to L^2(\hat G)$ in the Plancherel theorem and the reflection on the domain $\hat G$.

Therefore, the Fourier-Stieltjes transform $\cF:M(G)\to L^\infty(\hat G)$ is an injective $*$-homomorphism with with weakly$^*$ dense image.
The continuous and boundedness of $\cF(\mu)$ is because it is a multiplier of $C_0(\hat G)$.

(d)
Consider the inverse $\cF^{-1}:\cF(L^1(\hat G))\to L^1(\hat G)$ of the Fourier transform $\cF:L^1(\hat G)\to C_0(\hhat G)$ and the weak$^*$ transpose $\cF^t:L^1(\hat G)\to C_0(G)$
of the Fourier-Stieltjes transform $\cF:M(G)\to L^\infty(\hat G)$.
Their composition is equal to the restriction of the restriction map $C_0(\hhat G)\to C_0(G)$ along $\Phi$.

isometry?
\end{pf}


\section{}

\begin{prb}[Fell absorption principle]
Structure operator $w\in U(L^2(G,G))$ such that $w\xi(s,t)=\xi(s,st)$ or $w\in L^\infty(G)\bar\otimes L(G)$ such that $w(\lambda_s\otimes\lambda_s)w^*=\lambda_s\otimes1$.
If $w(x\otimes x)w^*=x\otimes1$, then $x=\lambda_s$ for some $s\in G$.
\end{prb}



\section{Spectral synthesis}









\part{Topological quantum groups}



\chapter{Kac algebras}




\chapter{Compact quantum groups}


\chapter{Locally compact quantum groups}
\section{Multiplicative unitaries}


\part{Representation categories}


\chapter{Representations of compact groups}
\section{Peter-Weyl theorem}
\section{Tannaka-Krein duality}
\section{Mackey machine}
Example of non-compact Lie groups,
Wigner classification



\end{document}







