\documentclass{../../large}
\usepackage{../../ikhanchoi}


\begin{document}
\title{Abstract Harmonic Analysis}
\author{Ikhan Choi}
\maketitle
\tableofcontents



\part{Fourier analysis on groups}
\chapter{Locally compact groups}
\section{Topological groups}
\section{Haar measures}

\begin{prb}[Non-$\sigma$-finite measures]
Following technical issues are important
\begin{parts}
\item Positive linear functionals on $C_c$
\item The Fubini theorem
\item The Radon-Nikodym theorem
\item The dual space of $L^1$ space
\end{parts}
\end{prb}

\begin{prb}[Radon measures]
Let $\Omega$ be a locally compact Hausdorff space.
A \emph{Radon measure} is a Borel measure $\mu$ on $\Omega$ such that
\begin{enumerate}[(i)]
\item $\mu$ is outer regular for every Borel set: $\mu(E)=\inf\{\mu(U):E\subset U,\,U\text{ open}\}$ for Borel $E$,
\item $\mu$ is inner regular for every open set: $\mu(U)=\inf\{\mu(K):K\subset U,\,K\text{ compact}\}$ for open $U$,
\item $\mu$ is locally finite.
\end{enumerate}
\begin{parts}
\item A $\sigma$-finite Radon measure is regular.
\item If every open subset of $\Omega$ is $\sigma$-compact, then a locally finite Borel measure is Radon.
\item $C_c(\Omega)$ is dense in $L^p(\mu)$ for $1\le p<\infty$.
\end{parts}
\end{prb}

\begin{prb}[Riesz-Markov-Kakutani representation theorem for $C_c$]
Let $\Omega$ be a locally compact Hausdorff space and consider the following map:
\[\begin{array}{ccc}
\{\text{Radon measures on $\Omega$}\} & \xrightarrow{\sim} & \{\text{positive linear functionals on $C_c(\Omega,\R)$}\},\\
\mu & \mapsto & (f\mapsto\int f\,d\mu).
\end{array}\]
\begin{parts}
\item a
\end{parts}
\end{prb}

\begin{prb}[Existence of the Haar measure]
\end{prb}




\section{Group algebras}
\begin{prb}[Modular functions]
\end{prb}
\begin{prb}[Convolution]
\end{prb}
\begin{prb}[Positive definite functions]
Bochner theorem
\end{prb}
\begin{prb}[Fourier-Stieltjes algebra]
\end{prb}
\begin{prb}[GNS construction for locally compact groups]
Let $G$ be a locally compact group.
By a state of $C^*(G)$, we could construct the GNS representation of $G$.
An analog of GNS construction for $L^1(G)$ without completion is doable, when given a function of positive type on $G$, instead of a state.
\end{prb}


\[\begin{tikzcd}
G \ar{r} & M(G) & \, & \,\\
L_1(G) \ar[hook]{ur}\ar[hook]{r}\ar[dashed]{d}{*} & C^*(G) \ar[two heads]{r}\ar[dashed]{d}{*} & C_r^*(G) \ar[hook]{r}\ar[dashed]{d}{*} & L(G) \ar[dashed]{d}{*\text{ with }\sigma w}\\
L^\infty(G) & B(G) \ar[hook]{l} & C_r^*(G)^* \ar[hook]{l} & A(G) \ar[hook]{l}\\
& C_0(G) \ar[hook]{ul} & &
\end{tikzcd}\]


\begin{prb}[Uniformly continuous functions]
$G$ acts on $C_{lu}(G)$ and $L^1(G)$ continuously with respect to the point-norm topology.
A function on $G$ is left uniformly continuous if and only if it is written as $f*x$ for some $f\in L^1(G)$ and $x\in L^\infty(G)$.
\end{prb}



\section{Pontryagin duality}

\begin{prb}[Dual group]
\end{prb}
\begin{prb}[Fourier inversion theorem]
\end{prb}
\begin{prb}[Plancherel's theorem]
\end{prb}


\section{Structure theorems}

\section*{Exercises}
\begin{prb}
\end{prb}

\section*{Problems}
\begin{enumerate}
\item Let $\Omega$ be a topological space. For every positive linear functional $I$ on $C_c(\Omega,\R)$, show that there exists a Borel measure $\mu$ on $\Omega$ such that $I(f)=\int f\,d\mu$ for all $f\in C_c(\Omega,\R)$. (Hint: Consider the uncountable wedge sum of circles as an example.)
\end{enumerate}
\begin{sol}
1.
The constructed Carath\'eodory measure $\mu$ on $\Omega$ is outer regular Borel measure, but we do not have local finiteness.
Everything is same to when $\Omega$ is locally compact Hausdorff except that $\mu(\supp f)$ may be infinite.
Now it is enough to show $I(\min\{f,\frac1n\})$ converges to zero as $n\to\infty$ for $f\in C_c(\Omega,[0,1])$.

Let $U:=f^{-1}((0,1])$.
For $g\in C_0(U,[0,1])$, it clearly has compact support, and and it is also continuous because $g^{-1}((a,1])$ is open in $U$ and $g^{-1}([a,1])$ is closed in $K$ for any $0<a\le1$, so that we have $C_0(U)\subset C_c(X)$.
We also have $f_1\in C_0(U)$ since $f_1^{-1}([\e,1])$ is a compact set in $U$ for every $\e>0$.
Therefore, $I$ is a positive linear functional on $C_0(U)$.
Since a positive linear map between C$^*$-algebras is bounded, there is a constant $C$ such that $I(g)\le C$ for all $g\in C_0(U,[0,1])$, and it proves $I(f_1)\le C/n\to0$ as $n\to\infty$.
Therefore, $I(f)=\int f\,d\mu$.
\end{sol}



\section{Spectral synthesis}








\chapter{Representation theory}

\begin{prb}[Schur's lemma]
\end{prb}

\begin{prb}[Operator-valued Fourier transform]
\end{prb}


\section{Group C$^*$-algerbas}

\begin{itemize}
\item How can we describe $L^1$-norm intrinsically?
\item How can we show the equivalences between representations of $G$, $C_c(G)$, $L^1(G)$, and $C^*(G)$?\\Note that $\|\cdot\|_r\le\|\cdot\|\le\|\cdot\|_1$ and $L^1(G)\hookrightarrow C^*(G)\twoheadrightarrow C^*_r(G)$.
\item How to show and interpret the inclusion $L^1(G)\hookrightarrow C^*_r(G)$.
\end{itemize}


Since it is not easy to introduce the quantum dual of $G$ for now, we cannot discuss $L^1(G)$ as the Fourier algebra, the predual of the quantum group von Neumann algebra.
($A(G)=L(G)_*=L^1(\hat G)$ and also is the closed linear span of matrix coefficients of the left regular representation.)



\chapter{Compact groups}
\section{Peter-Weyl theorem}
\section{Tannaka-Krein duality}
\section{Example of compact Lie groups}

\chapter{Mackey machine}
\section{Example of non-compact Lie groups}
Wigner classification








\chapter{Kac algebras}




\part{Topological quantum groups}
\chapter{Compact quantum groups}
\chapter{Locally compact quantum groups}
\section{Multiplicative unitaries}


\chapter*{}

\subsection{Measures on locally compact Hausdorff spaces}
compact  closed set not containing infty
open     open not containing infty
closed   closed set containing infty

for a measure that ``vanishes at infty'' = tight
two definitions of inner regularity is equivalent.

IRK -> IRF
IRK + sigma finite -> tight

Thm. The measure contructed by RMK is lf and regular(cpt version).
1. open set is approx by cpt sets (by def of rho, if X is LCH)
2. meas set is approx by opn sets (by def of outer meas)
3. sigma finite set is approx by cpt sets (by thm)

Consider
\begin{cd}
\text{regBorel}_{fin} \rar[hook]\dar[hook]& \text{Borel}_{fin} \rar\dar[hook]& \text{Baire}_{fin} \rar\dar[hook]& C_b^{*+} \dar[->>]&\\
\text{regBorel}_{locfin} \rar[hook]& \text{Borel}_{locfin} \rar& \text{Baire}_{locfin} \rar& C_c^{*+}\rar[hook]& \text{pos lin on }C_c.
\end{cd}
for locally compact Hausdorff $X$.

$\text{Borel}_{locfin}\to \text{pos lin on }C_c$ is surjective for all topological spaces.

$\text{regBorel}_{fin}\to C_b^{*+}$ is injective for normal spaces.

$\text{regBorel}_{locfin}\to C_c^{*+}$ is injective for locally compact Hausdorff spaces.(maybe)

\begin{lem}
Let $\mu$ be a Borel measure on a LCH $X$.
Then, $\mu$ is inner regular on open sets iff
\[\mu(U)=\|\mu\|_{C_c(U)^*}\]
for every open $U$ in $X$.
\end{lem}
\begin{pf}
($\Leftarrow$)
($\ge$)
For $f\in C_c(U)$, we have
\[|\int f\,d\mu|=|\int_Uf\,d\mu|\le\mu(U)\,\|f\|.\]

($\le$)
Since $\mu$ is inner regular on $U$, there is a compact set $K\subset U$ such that $\mu(U)-\mu(K)<\e$ (for the case $\mu(U)=\infty$, we can deal with separately).
We can find a nonnegative function $f\in C_c(U)$ with $f|_K \equiv 1$ and $f\le1$ by the construction of Urysohn.
Then, for all $\e>0$ we have
\[\mu(U)<\mu(K)+\e\le\int f\,d\mu+\e\le\|\mu\|_{C_c^*(U)}+\e.\]

($\Rightarrow$)
Let $f\in C_c(U)$ be a function such that $\|f\|=1$ and
\[\mu(U)-\e<\int f\,d\mu.\]
Let $K=\supp(f)$.
Then
\[\mu(K)\ge\int f>\mu(U)-\e.\]
\end{pf}
% 이런 거 쓸 때 메져가 유한인지 무한인지 케이스 나누고 증명쓰는 게 좋겠다

\begin{prop}
A Radon measure is inner regular on all $\sigma$-finite Borel sets.(Folland's)
\end{prop}
\begin{pf}
First we approximate Borel sets of finite measure, with compact sets.
Let $E$ be a Borel set with $\mu(E)<\infty$ and $U$ be an open set containing $E$.
By outer regularity, there is an open set $V\supset U-E$ such that
\[\mu(V)<\mu(U-E)+\frac\e2.\]
By inner regularity, there is a compact set $K\subset U$ such that
\[\mu(K)>\mu(U)-\frac\e2.\]
Then, we have a compact set $K-V\subset K-(U-E)\subset E$ such that
\begin{align*}
\mu(K-V)&\ge\mu(K)-\mu(V)\\
&>\left(\mu(U)-\frac\e2\right)-\left(\mu(U-E)+\frac\e2\right)\\
&\ge\mu(E)-\e.
\end{align*}
It implies that a Radon measure is inner regular on Borel sets of finite measures.

Suppose $E$ is a $\sigma$-finite Borel set so that $E=\bigcup_{n=1}^\infty E_n$ with $\mu(E_n)<\infty$.
We may assume $E_n$ are pairwise disjoint.
Let $K_n$ be a compact subset of $E_n$ such that
\[\mu(K_n)>\mu(E_n)-\frac\e{2^n},\]
and define $K=\bigcup_{n=1}^\infty K_n\subset E$.
Then,
\[\mu(K)=\sum_{n=1}^\infty\mu(K_n)>\sum_{n=1}^\infty\left(\mu(E_n)-\frac\e{2^n}\right)=\mu(E)-\e.\]
Therefore, a Radon measure is inner regular on all $\sigma$-finite Borel sets.
\end{pf}

\begin{thm}
If every open set in $X$ is $\sigma$-compact(i.e. Borel sets and Baire sets coincide), then every locally finite Borel measure is regular.
\end{thm}
\begin{prop}
In a second countable space, every open set is $\sigma$-compact(i.e. Borel sets and Baire sets coincide).
\end{prop}

Two corollaries are presented as follows:
\begin{rd}[column sep={120pt,between origins}]
\parbox{7em}{\centering locally finite \\ Borel regular} \rar &
\parbox{5em}{\centering Radon} \rar \lar[dashed, bend right, swap]{$X$ is $\sigma$-compact} &
\parbox{7em}{\centering locally finite \\ Borel} \ar[dashed, bend left]{ll}{$X$ is second countable}
\end{rd}
\begin{prb}
Let $X$ be compact.
A positive linear functional $\rho$ on $C(X)$ is bounded with norm $\rho(1)$.
\end{prb}
\begin{pf}
Since $0\le\rho(\|f\|\pm f)=\|f\|\rho(1)\pm\rho(f)$, we have $|\rho(f)|\le\rho(1)\|f\|$.
\end{pf}

\begin{prb}
Let $X$ be a locally compact Hausdorff space.
\begin{parts}
\item The Baire $\sigma$-algebra is generated by compact $G_\delta$ sets.
\item If $X$ is second countable, then every Baire set is Borel.
\end{parts}
\end{prb}
\begin{sol}
(b)
(A second countable locally compact space is $\sigma$-compact.

Since $X$ is $\sigma$-compact and Hausdorff, every closed set is a countable union of compact sets, so the Borel $\sigma$-algebra on $X$ is generated by compact sets.)

Since locally compact Hausdorff space is regular, the Urysohn metrization implies $X$ is metrizable, and every closed sets in metrizable space is $G_\delta$ set.
\end{sol}
%%%%%%%%%%%%%%%
\begin{prb}
Let $X$ be compact.
There is a map from the set of finite Baire measures to the set of positive linear functionals on $C(X)$.
\end{prb}
\begin{sol}
A function in $C(X)$ is Baire measurable and bounded.
Thus the integration is well-defined.
\end{sol}

\begin{prb}
Let $X$ be compact.
There is a map from the set of positive linear functionals on $C(X)$ to the set of finite regular Borel measures.
\end{prb}
\begin{sol}
i. and ii. and iii. of Theorem 7.2.
\end{sol}

\begin{prb}
Let $X$ be compact.
Let $\rho$ be a positive linear functional on $C(X)$.
Let $\nu$ be the regular Borel measure associated to $\rho$.
Then, $\rho(f)=\int f\,d\nu$.
\end{prb}
\begin{sol}
iv. of Theorem 7.2.
\end{sol}

\begin{prb}
Let $X$ be compact.
Let $\nu$ be a finite regular Borel measure.
Let $\nu'$ be the regular Borel measure associated to the positive linear functional $f\mapsto\int f\,d\nu$.
Then, $\nu=\nu'$ on Borel sets.
\end{prb}
\begin{sol}
Theorem 7.8.
\end{sol}

The two results above establish the correspondence between positive linear functionals and regular Borel measures.
The following is an additional topic: Borel extension of Baire measures.
\begin{prb}
Let $X$ be compact.
Let $\mu$ be a finite Baire measure.
Let $\nu$ be the regular Borel measure associated to the positive linear functional $f\mapsto\int f\,d\mu$.
Then, $\mu=\nu$ on Baire sets.
\end{prb}
\begin{sol}
Let $\mu,\nu$ be finite Baire measures.
Enough to show if $\int f\,d\mu=\int f\,d\nu$ then $\mu=\nu$ according to the preceding two results.

Enough to show the regularity of Baire measures.
\end{sol}

\end{document}







