\documentclass{../../large}
\usepackage{../../ikhanchoi}


\begin{document}
\title{Abstract Harmonic Analysis}
\author{Ikhan Choi}
\maketitle
\tableofcontents



\part{Fourier analysis on groups}

\chapter{Locally compact groups}
\section{Haar measures}

\begin{prb}[Non-$\sigma$-finite measures]
Following technical issues are important
\begin{parts}
\item The Fubini theorem
\item The Radon-Nikodym theorem
\item The dual space of $L^1$ space
\end{parts}
\end{prb}

\begin{prb}[Existence of the Haar measure]
\end{prb}




\section{Group algebras}

Which one comes from the theory of Hilbert algebras?
Which one comes from the locally compact groups(maybe for the existence of approximate unit?)?

\begin{prb}[Modular functions]
\end{prb}
\begin{prb}[Convolution]
\end{prb}
\begin{prb}[Positive definite functions]
Bochner theorem
\end{prb}
\begin{prb}[Fourier-Stieltjes algebra]
\end{prb}
\begin{prb}[GNS construction for locally compact groups]
Let $G$ be a locally compact group.
By a state of $C^*(G)$, we could construct the GNS representation of $G$.
An analog of GNS construction for $L^1(G)$ without completion is doable, when given a function of positive type on $G$, instead of a state.
\end{prb}


\[\begin{tikzcd}
G \ar{r} & M(G) & \, & \,\\
L_1(G) \ar[hook]{ur}\ar[hook]{r}\ar[dashed]{d}{*} & C^*(G) \ar[two heads]{r}\ar[dashed]{d}{*} & C_r^*(G) \ar[hook]{r}\ar[dashed]{d}{*} & L(G) \ar[dashed]{d}{*\text{ with }\sigma w}\\
L^\infty(G) & B(G) \ar[hook]{l} & C_r^*(G)^* \ar[hook]{l} & A(G) \ar[hook]{l}\\
& C_0(G) \ar[hook]{ul} & &
\end{tikzcd}\]


\begin{prb}[Uniformly continuous functions]
$G$ acts on $C_{lu}(G)$ and $L^1(G)$ continuously with respect to the point-norm topology.
A function on $G$ is left uniformly continuous if and only if it is written as $f*x$ for some $f\in L^1(G)$ and $x\in L^\infty(G)$.
\end{prb}



\section{Pontryagin duality}

\begin{prb}[Dual group]
\end{prb}
\begin{prb}[Fourier inversion theorem]
\end{prb}
\begin{prb}[Plancherel's theorem]
\end{prb}


\section{Structure theorems}



\section{Spectral synthesis}








\chapter{Representation theory}

\begin{prb}[Schur's lemma]
\end{prb}

\begin{prb}[Operator-valued Fourier transform]
\end{prb}



Since it is not easy to introduce the quantum dual of $G$ for now, we cannot discuss $L^1(G)$ as the Fourier algebra, the predual of the quantum group von Neumann algebra.
($A(G)=L(G)_*=L^1(\hat G)$ and also is the closed linear span of matrix coefficients of the left regular representation.)



\chapter{Compact groups}
\section{Peter-Weyl theorem}
\section{Tannaka-Krein duality}
\section{Example of compact Lie groups}

\chapter{Mackey machine}
\section{Example of non-compact Lie groups}
Wigner classification








\chapter{Kac algebras}




\part{Topological quantum groups}
\chapter{Compact quantum groups}
\chapter{Locally compact quantum groups}
\section{Multiplicative unitaries}


\part{Tensor categories}


\end{document}







