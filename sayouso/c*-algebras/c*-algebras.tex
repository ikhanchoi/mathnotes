\documentclass{../../large}
\usepackage{../../ikhanchoi}

\DeclareMathOperator{\Mor}{Mor}

\begin{document}
\title{C$^*$-Algebras}
\author{Ikhan Choi}
\maketitle
\tableofcontents

\part{Constructions}


\chapter{Operator systems and spaces}
\section{Completely positive maps}

$|\f(a)|^2\le\|\f\|\f(|a|^2)\le\|\f\|^2\|a\|^2$.
If $\omega$ is a state, then
$|\omega(a)|^2\le\omega(|a|^2)\le\|a\|^2$.

category of operator systems

\begin{prb}[Choi-Effros characterization]
\end{prb}

\begin{prb}[Stinespring dilation]
\end{prb}

tensor product of c.p.~maps (minimal and maximal)

\begin{prb}[Arveson extension]
Trick
\end{prb}

\section{Completely bounded maps}



\section{Subalgebras}


\begin{prb}[Hereditary C$^*$-subalgebra]
state extension, representation extension(not ideal?)
\end{prb}

\begin{prb}[Conditional expectation]
\end{prb}

\begin{prb}[Ideals]
\end{prb}


\begin{prb}[Enveloping C$^*$-algeberas]
Let $A$ be a $*$-algebra.
A \emph{C$^*$-norm} is an submultiplicative norm satisfying the C$^*$-identity.
Does $A$ have enough $*$-representations?
\begin{parts}
\item A complete C$^*$-norm is unique if it exists.
\item For every C$^*$-norm $\alpha$ on $A$, there is a $*$-isometry $\pi:A\to B(H)$.
\item For maximal C$^*$-norm, there is a universal property. The maximal C$^*$-norm can be obtained by running through cyclic representations.
\end{parts}
\end{prb}


\section{Tensor products}

\begin{prb}[Maximal tensor products]
Let $A$ and $B$ be C$^*$-algebras.
\begin{parts}
\item A commuting pair of $*$-homomorphisms $\pi:A\to B(H)$ and $\pi':B\to B(H)$ corresponds to a $*$-homomorphism $\Pi:A\odot B\to B(H)$ via the relation $\Pi(a\otimes b)=\pi(a)\pi'(b)$.
\item $A\odot B$ admits a $*$-representation and every norms induced from these $*$-representations are uniformly bounded. So, we can define a maximal tensor norm on $A\odot B$.
\item $a\otimes-:B\to A\odot B$ is bounded for each $a\in A$ with respect to any C$^*$-norm on $A\odot B$. [BO, 3.2.5]
\end{parts}
\end{prb}


\begin{prb}[Minimal tensor product]
spatiality
\end{prb}
\begin{prb}[Takesaki theorem]
\end{prb}

Tensors with $M_n(\C)$, $C_0(X)$.


\begin{prb}[Haagerup tensor product]
\end{prb}


\section*{Exercises}
\begin{prb}
Let $B$ be a hereditary C$^*$-subalgebra of a C$^*$-algebra $A$.
Let $a\in A_+$.
If for any $\e>0$ there is $b\in B_+$ such that $a-\e\le b$, then $a\in B_+$.
\end{prb}
\begin{pf}
To catch the idea, suppose $A$ is abelian.
We want to approximate $a$ by the elements of $B$ in norm.
To do this, for each $\e>0$, we want to construct $b'\in B_+$ such that $a-\e\le b'\le a+\e$ using $b$.
Taking $b'=\min\{a,b\}$ is impossible in non-abelian case, but we can put $b'=\frac a{b+\e}b$.
For a simpler proof, $b'=(\frac{\sqrt{ab}}{\sqrt b+\sqrt\e})^2$ is a better choice.

Define
\[b':=\frac{\sqrt b}{\sqrt b+\sqrt\e}a\frac{\sqrt b}{\sqrt b+\sqrt\e}.\]
Then,
\[\|\sqrt a-\sqrt a\frac{\sqrt b}{\sqrt b+\sqrt\e}\|^2=\|\frac{\sqrt\e}{\sqrt b+\sqrt\e}a\frac{\sqrt\e}{\sqrt b+\sqrt\e}\|\le\e\]
implies
\[\lim_{\e\to0}b'=\lim_{\e\to0}\frac{\sqrt b}{\sqrt b+\sqrt\e}\sqrt a\cdot\sqrt a\frac{\sqrt b}{\sqrt b+\sqrt\e}=\sqrt a\cdot\sqrt a=a.\]
\end{pf}









\chapter{Hilbert modules}

\section{Hilbert modules}

\begin{prb}
A \emph{Hilbert $A$-module} is a complex linear space $\cE$ together with
\begin{enumerate}[(i)]
\item a ring homomorphism $A^{op}\to\End_\C(\cE)$,
\item an $A$-valued inner product $\<\cdot,\cdot\>:\cE\times\cE\to A$, which is $A$-linear in second argument,
\end{enumerate}
which is complete with respect to the norm $\|\xi\|:=\|\<\xi,\xi\>\|^{\frac12}$.
\end{prb}

constructions:
direct sum, tensor product, localization

examples:
A itself


\section{Multiplier algebras}
\begin{prb}[Double centralizer characterization]
Let $A$ be a C$^*$-algebra.
A \emph{double centralizer} of $A$ is a pair $(L,R)$ of bounded linear maps on $A$ such that $aL(b)=R(a)b$ for all $a,b\in A$.
The \emph{multiplier algebra} $M(A)$ of $A$ is defined to be the set of all double centralizers of $A$.
There is another characterization $M(A):=L_A(A)$, the set of adjointable operators to itself.
\end{prb}
\begin{prb}[Cohen factorization theorem]
\end{prb}
\begin{prb}[Strict topology]
\begin{parts}
\item $\|\pi(a-e_\alpha a)\xi\|^2$
\end{parts}
\end{prb}
\begin{prb}[Essential ideals]
\begin{parts}
\item Hilbert C$^*$-module description
\end{parts}
\end{prb}


\begin{prb}[Examples of multiplier algebras]
\begin{parts}
\item $M(K(H))\cong B(H)$.
\item $M(C_0(\Omega))\cong C_b(\Omega)$.
\end{parts}
\end{prb}
\begin{pf}
(a)

(b)
First we claim $C_0(\Omega)$ is an essential ideal of $C_b(\Omega)$.
Since $C_b(\Omega)\cong C(\beta\Omega)$, and since closed ideals of $C(\beta\Omega)$ are corresponded to open subsets of $\beta\Omega$, $C_0(\Omega)\cap J$ is not trivial for every closed ideal $J$ of $C_b(\Omega)$.

Now we have an injective $*$-homomorphism $C_b(\Omega)\to M(C_0(\Omega))$, for which we want to show the surjectivity.
Let $g\in M(C_0(\Omega))_+$.
\end{pf}






Induced representations and Morita equivalence







\chapter{Examples}



\section{Group algebras}
type I, subhomogeneous


crystallographic
discrete heisenberg
free groups
projectionless of $C_r^*(F_2)$




\section{Crossed products}

\begin{prb}[C$^*$-dynamical system]
A \emph{C$^*$-dynamical system} is a triple $(A,G,\alpha)$ consisting of a C$^*$-algebra $A$, a locally compact group $G$, and group homomorphism $\alpha:G\to\Aut(A)$ that is continuous in the point-norm topology.
\begin{parts}
\item For fixed $G$, there is an equivalence between categories of locally compact transformation groups and C$^*$-dynamical system on abelian C$^*$-algebras.
\end{parts}
\end{prb}


On $U(H)$, the strict topology and the strong operator topology are equal.
Therefore, we have three topologies to consider: strong, weak, and $\sigma$-weak.

\begin{prb}
Let $G$ be a locally compact group.
A unitary representation $u:G\to B(H)$ induces a C$^*$-dynamical system $(B(H),G,\Ad)$ by $\Ad:G\to\Aut(B(H))$.

A \emph{covariant representation} of a C$^*$-dynamical system is an equivariant $*$-homomorphism $(A,G,\alpha)\to(B(H),G,\Ad)$.
\end{prb}

Note that we have a homeomorphism $\Aut(K(H))\cong PU(H)$ between the point-norm topology and the strong operator topology.

$\Z$-action, $\Homeo$-action, left multiplication of subgroup
induced representation
regular representation $(C_0(G),G,lt)\to(B(L^2(G)),G,\lambda)$.


commutative case

group algebra: completion of $C_c(G)$ with reps,
crossed product: completion of $C_c(G,A)$ with cov reps,




\section{Groupoid algebras}






\section{Graph algebras}





\section{Pimsner algebras}


\begin{prb}[C$^*$-correspondences]
Let $A$ be a C$^*$-algebra.
A \emph{C$^*$-correspondence} over $A$ is a right Hilbert $A$-module $\cE$ together with a $*$-homomorphism $\f:A\to B(\cE)$, called the \emph{left action}.
We say $\cE$ is \emph{faithful} or \emph{non-degenerate} if $\f$ is faithful or non-degenerate, respectively.
\begin{parts}
\item If $\f:A\to M(B)$ is a unital completely positive map, then we can construct a natural $A$-$B$-correspondence $\cE$ by mimicking the GNS construction on $A\odot B$.
\item If $\f:A\to M(B)$ is a non-degenerate $*$-homomorphism, $\f\in\Mor(A,B)$ in other words, then we can associate a canonical $A$-$B$-correspondence $B$ such that the left action is realized with $\f$.
More precisely, $\iota:\cE\to B:a\otimes b\mapsto\f(a)b$ provides a well-defined linear isomorphism (surjectivity follows from the density of $\f(A)B$ in $B$ and the Cohen factorization theorem) and the two actions on $\cE$ is described by $\iota(a\xi b)=\f(a)\iota(\xi)b$.
\end{parts}
\end{prb}


\begin{prb}
Let $\cE$ be a C$^*$-correspondence over a C$^*$-algebra $A$.
Let $B$ be a C$^*$-algebra and see it as a trivial C$^*$-correspondence over $B$.
A \emph{representation} of $\cE$ on $B$ is a pair $(\pi,\tau)$ of a $*$-homomorphism $\pi:A\to B$ and a linear map $\tau:\cE\to B$ such that
\[\pi(\<\xi,\eta\>)=\tau(\xi)^*\tau(\eta),\qquad\tau(\f(a)\xi)=\pi(a)\tau(\xi).\]
We define an ideal
\[J(\cE):=\f^{-1}(K(\cE))\cap\f^{-1}(0)^\perp.\]

A \emph{covariant representation} is a representation of $\cE$ such that
\[\psi(\f(a))=\pi(a),\qquad a\in J(\cE).\]
\begin{parts}
\item
Let $(A,\Z,\alpha)$ be a C$^*$-dynamical system and consider the canonical C$^*$-correspondence $A$ over $A$ with the left action $\f:=\alpha_1\in\Aut(A)\subset\Mor(A)$.
This correspondence is full, faithful, and non-degenerate.
Note that also we have $J(A)=\f^{-1}(A)\cap A=A$.
If $(\pi,\tau)$ is an any representation of this C$^*$-correspondence $A$ on $B$, then 
\end{parts}
\end{prb}

How can we decribe representations of C$^*$-correspondence $A$ with left action $\f\in\Aut(A)$ in terms of covariant representations of the C$^*$-dynamical system $(A,\Z,\alpha)$ with $\alpha_n=\f^n$?



as a morphism
sub and quotient, direct sum, tensor product,

Toeplitz-Cuntz
Toeplitz-Pimsner
Cuntz-Pimsner
Cuntz-Krieger




Let $\f\in\Aut(A)$.







Coactions and Fell bundles



KK-theory
C$^*$-algebraic quantum groups



\section{Free products}






\part{Properties}
\chapter{Approximation properties}
\section{Nuclearity and exactness}

finite dimensional[BO, 3.3.2], abelian
permanence properties
completely positive approximation property

$M_n(\C)$, $K(H)$, $C_0(X)$.

a separable C$^*$-algebra is nuclear if and only if every factor representation is hyperfinite.



quotients of nuclear
local reflexivity



Extension properties
weak expectation property
relatively weakly injective
maximal tensor product inclusion problem


\section{Quasi-diagonality}
Voiculescu theorem


\begin{prb}
An operator $x\in B(H)$ is called \emph{quasi-diagonal} if there is a net of projections $p_i\in B(H)$ such that $[p_i,x]$ and $p-\id_H$ converge strongly to zero.
A C$^*$-algebra is called \emph{quasi-diagonal} if it admits a faithful representation whose image is quasi-diagonal.
\end{prb}

faithful non-degenerate essential representations of a quasi-diagonal C$^*$-algebra are all quasi-diagonal

\section{AF-embeddability}






\chapter{Amenability}


\section{Amenable groups}


\section{Amenable actions}
crossed products
$Z_2$-grading
Connes-Feldman-Weiss
Anantharaman-Delaroche
Gromov boundaries
approximately central structure?
dynamical Kirchberg-Phillips

stably finite
dynamical Elliott program

Ornstein-Weiss-Rokhlin lemma

\section{Exact groups}
Exact groups

\section{Other properties}
Kazdahn property (T)
factorization property
Haagerrup property


Kaplansky conjecture



\chapter{Simplicity}


Furstenburg boundary



















\part{Invariants}
\chapter{Operator K-theory}

\section{Homotopy of C$^*$-algebras}

\begin{prb}[Homotopy of $*$-homomorphisms]
Let $A,B$ be C$^*$-algebras.
Two $*$-homomorphisms in $\Mor(A,B)$ are said to be \emph{homotopic} if they are connected by a path in $\Mor(A,B)$ that is continuous with the point-norm topology.
\begin{parts}
\item For pointed compact Hausdorff spaces $(X,x_0),(Y,y_0)$, two pointed maps $\f_0,\f_1:X\to Y$ are homotopic if and only if $\f_0^*,\f_1^*:C_0(Y\setminus\{y_0\})\to C_0(X\setminus\{x_0\})$ are homotopic.
\end{parts}
\end{prb}
\begin{pf}
(a)
Suppose $\f_0$ and $\f_1$ are connected by a homotopy $\f_t$.
Fixing $g\in C_0(Y)$ and $t_0\in I$, we want to show
\[\lim_{t\to t_0}\sup_{x\in X}|g(\f_t(x))-g(\f_{t_0}(x))|=0.\]
Since the function $g$ is uniformly continuous, with respect to an arbitrarily chosen uniformity on $Y$, so that there is an entourage $E\subset Y\times Y$ such that $(y,y')\in E\circ E$ implies $|g(y)-g(y')|<\e$.
Using compactness we have a finite sequence $(y_i)_{i=1}^n\subset Y$ such that for every $y$ there is $y_i$ satisfying $(y,y')\in E$.
Then, $\f^{-1}(E[y_i])$ is a finite open cover of $X\times I$, so we have $\delta$ such that $|t-t_0|<\delta$ implies for any $x\in X$ the existence of $i$ satisfying $(\f_t(x),y_i)\in E$ and $(\f_{t_0}(x),y_i)\in E$, which deduces the desired inequality.

Conversely, suppose $\f_0^*$ and $\f_1^*$ are connected by a homotopy $\f_t^*$.
By taking dual, we can induce $\f_t:X\to Y$ such that $g(\f_t(x))=(\f_t^*g)(x)$ for each $g\in C(Y)$ from $\f_t^*$ via the embedding $X\to M(X)$ by Dirac measures.
Let $V$ be an open neighborhood of $\f_{t_0}(x_0)$ and take $g\in C(Y)$ such that $g(\f_{t_0}(x_0))=1$ and $g(y)=0$ for $y\notin V$.
Now we have an open neighborhood $U$ of $x_0$ such that $x\in U$ implies $|(\f_{t_0}^*g)(x)-(\f_{t_0}^*g)(x_0)|<\frac12$.
Also we have $\delta>0$ such that $|t-t_0|<\delta$ implies $\|\f_t^*g-\f_{t_0}^*g\|<\frac12$.
Therefore, $(x,t)\in U\times(t_0-\delta,t_0+\delta)$ implies $g(\f_t(x))>0$, hence $\f_t(x)\in V$, which means $X\times I\to Y:(x,t)\mapsto\f_t(x)$ is continuous.
\end{pf}

We have $\tilde K^n(X,x_0)=K_n(C_0(X\setminus\{x_0\}))$ for a pointed compact Hausdorff space $X$.
Now then since the inclusion $\{x_0\}\to X$ induces the section so that
\[0\to K_0(C_0(X\setminus\{x_0\}))\to K_0(C(X))\to K_0(\{x_0\})\to0\]
splits, we have
\[K^0(X)=\tilde K^0(X,x_0)\oplus\Z=K_0(C_0(X\setminus\{x_0\}))\oplus K_0(\{x_0\})=K_0(C(X))\]
for a compact connected Hausdorff space $X$.
The additivity of $K_0$ and $K^0$ removes the connectedness condition.

\[K_0(\C)=\Z,\quad K_0(C_0(\R))=0,\quad K_1(C_0(\R))=K_0(C_0(\R^2))=\Z\]
\[K^0(*)=\Z,\quad K^0(S^1)=\Z,\quad K^1(S^1)=K^0(S^2)=\Z[x]/(x-1)^2\]





\section{Brown-Douglas-Fillmore theory}
\begin{prb}[Haagerup property]
\end{prb}

Baum-Connes conjecture
Non-commutative geometry
Elliott theorem




\section{Approximately finite algebras}
Elliott conjecture: amenable simple separable C$^*$-algerbas are classified by K-theory.
Brattelli diagram



\section{Fredholm theory of Mishchenko and Fomenko}


















\part{Classification}
\chapter{Simple nuclear algebras}
\section{Elliott invariant}

\section{Classifiability}
Jiang-Su stability
Universal coefficient theorem

Toms-Winter conjecture
strongly self-absorbing
nuclear dimension




successful in Kirchberg algebras


https://arxiv.org/pdf/2307.06480.pdf

Elliott classification problem
Kirchberg-Phillipes theorem

operator K-theory and its pairing with traces

$\cZ$-stability, Rosenberg-Schochet universal coefficient theorem

Connes-Haagerup classification of injective factors

Kirchberg: unital simple separable $\cZ$-stable algebra is either purely infinte or stably finite.
Haagerup, Blackadar, Handelman: unital simple stably finite algebra has a trace.

Glimm: uniformly hyperfinite algebras
Murray-von Neumann: hyperfinite II$_1$ factors




\chapter{Continuous fields}


\section{Banach bundles}

\begin{prb}[Banach bundles]
A \emph{Banach bundle}, introduced by Fell, is a continuous open surjection $\pi:E\to X$ between topological spaces together with Banach space structure on each fiber $\pi^{-1}(x)$ such that:
\begin{enumerate}[(i)]
\item the addition $\{(e,e'):\pi(e)=\pi(e')\}\subset E\times E\to E:(e,e')\mapsto e+e'$ is continuous,
\item the scalar multiplication $\C\times E\to E:(\lambda,e)\mapsto\lambda e$ is continuous,
\item the norm $E\to\R_{\ge0}:e\mapsto\|e\|$ is continuous,
\item the family of subsets
\[\{e\in B:\pi(e)\in U,\ \|e\|<r\}_{U\in N(x),r>0}\]
forms a neighborhood basis of $0\in\pi^{-1}(x)$ in $E$.
\end{enumerate}
The forth condition is equivalent to that if $\|e_i\|\to0$ and $\pi(e_i)\to x$ then $e_i\to 0_x\in\pi^{-1}(x)$.
\begin{parts}
\item For a Banach bundle $E\to X$, if $X$ is locally compact Hausdorff and every fiber $E_x$ shares a same finite dimension, then the bundle is locally trivial.
\end{parts}
\end{prb}


\begin{prb}[Continuous fields of Banach spaces]
\end{prb}



\begin{prb}[Hilbert bundles]
A \emph{Hilbert bundle} is a Banach bundle whose norm function satisfies the parallelogram law.

\begin{parts}
\item On a compact $X$, there is an equivalence between the category of Hilbert $C(X)$-modules and the category of Hilbert bundles over $X$.
\item On a compact $X$, there is an equivalence between the category of algebraically finitely generated Hilbert $C(X)$-modules and the category of classical locally trivial finite-rank complex vector bundle over $X$.
It is due to that finitely generatedness implies the projectivity and the Serre-Swan theorem.
\end{parts}
\end{prb}

\section{Dixmier-Douady theory}


Fell's condition

A C$^*$-algebra $A$ is called \emph{continuous trace} if the set of all $a\in\cA$ such that $\hat A\to\R_{\ge0}:\pi\mapsto\tr(\pi(a^*a))$ is continuous is dense in $A$.



Dadarlat-Pennig theory




\end{document}