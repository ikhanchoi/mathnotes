\documentclass{../../large}
\usepackage{../../ikhanchoi}

\newcommand{\Prim}{\operatorname{Prim}}



\begin{document}
\title{C$^*$-Algebras}
\author{Ikhan Choi}
\maketitle
\tableofcontents

\part{Constructions}


\chapter{Completely positive maps}
\section{Operator spaces}

\section{Operator systems}

\begin{prb}[Choi-Effros characterization]

\end{prb}

\begin{prb}[Von Neumann inequality]
\end{prb}


The set $M_n(A)^+$ is linearly spanned by elements of the form $[a_i^*a_j]\in M_n(A)$ for $[a_i]\in A^n$.
A linear map $\f:A\to B$ is completely positive if
\[\f(a_i^*a_j)\]


\begin{prb}[$n$-positive maps]
Let $S$ be an operator space.
Let $A$ and $B$ be C$^*$-algebras.
\begin{parts}
\item (Cauchy-Schwarz inequality)
If $\f:A\to B$ is a 2-positive map, then $\lim_\alpha\|\f(e_\alpha)\|=\|\f\|$ for any approximate unit $(e_\alpha)$ of $A$, and
\[\f(a)^*\f(a)\le\|\f\|\f(a^*a),\qquad a\in A.\]
\item (Multiplicative domain)
Let $\f:A\to B$ be a 4-positive map with $\|\f\|=1$.
If $a\in A$ satisfies $\f(a)^*\f(a)=\f(a^*a)$, then $\f(b)\f(a)=\f(ba)$ for all $b\in A$.
In particular, if $\f:B\to C$ is an extension of a $*$-homomorphism $\pi:A\to C$, then $\f(ab)=\pi(a)\f(b)$ and $\f(ba)=\f(b)\pi(a)$ for $a\in A$ and $b\in B$.
\end{parts}
\end{prb}
\begin{pf}
(a)
It suffices to show
\[\f(a)^*\f(a)\le\lim_\alpha\|\f(e_\alpha)\|\f(a^*a),\qquad a\in A,\]
since
\[\frac{\|\f(a)\|^2}{\|a\|^2}\le\lim_\alpha\|\f(e_\alpha)\|\frac{\|\f(a^*a)\|}{\|a^*a\|}\]
implies $\|\f\|^2\le\lim_\alpha\|\f(e_\alpha)\|\|\f\|$.
Suppose $B$ acts on a Hilbert space $H$ non-degenerately and faithfully.
Since $\f$ is 2-positive, we have
\[\mat{\f(e_\alpha^2)&\f(e_\alpha a)\\\f(a^*e_\alpha)&\f(a^*a)}=\f^{(2)}\left(\mat{e_\alpha^2&e_\alpha a\\a^*e_\alpha&a^*a}\right)=\f^{(2)}\left(\mat{e_\alpha&a\\0&0}^*\mat{e_\alpha&a\\0&0}\right)\ge0,\]
and it is equivalent to
\[\<\f(e_\alpha^2)\xi,\xi\>+2\Re\<\f(e_\alpha a)\eta,\xi\>+\<\f(a^*a)\eta,\eta\>\ge0,\qquad\xi,\eta\in H,\quad a\in A.\]
We put $\xi:=-(\|\f(e_\alpha)\|+\e)^{-1}\f(e_\alpha a)\eta$ for $\e>0$ to get
\[\f(e_\alpha a)^*\f(e_\alpha a)
\le\f(e_\alpha a)^*[2-(\|\f(e_\alpha)\|+\e)^{-1}\f(e_\alpha^2)]\f(e_\alpha a)
\le(\|\f(e_\alpha)\|+\e)\f(a^*a)\]
We have the desired inequality by taking limits for $\alpha$ and $\e$.

(b)
Since the second inflation $\f^{(2)}$ is 2-positive, we may write the Cauchy-Schwarz inequality
\[\f^{(2)}\left(\mat{a&b\\0&0}\right)^*\f^{(2)}\left(\mat{a&b\\0&0}\right)\le\f^{(2)}\left(\mat{a^*a&a^*b\\b^*a&b^*b}\right),\]
so
\[\mat{0&\f(a^*b)-\f(a^*)\f(b)\\\f(b^*a)-\f(b^*)\f(a)&\f(b^*b)-\f(b^*)\f(b)}\ge0,\]
which implies $\f(b^*a)-\f(b^*)\f(a)=0$ for any $b\in A$.

Note that $\|\pi\|=1$ if $\pi$ is not trivial.
Using the above argument for $a$ and $a^*$, we are done.
\end{pf}




\begin{prb}[Russo-Dye theorem]
If $C(X)\to B$ is positive, then it is c.p.
\end{prb}


\begin{prb}[Completely positive maps for matrix algebras]
Let $A$ be a C$^*$-algebra.
\begin{parts}
\item Choi matrix
\item
There is a one-to-one correspondence
\[\mathrm{CP}(M_n(\C),A)\to M_n(A)_+:\f\mapsto[\f(e_{ij})].\]
\item
Let $A$ be unital.
There is a one-to-one correspondence
\[\mathrm{CP}(A,M_n(\C))\to M_n(A)^*_+:\f\mapsto(s_\f:[a_{ij}]\mapsto\sum_{i,j}\<\f(a_{ij})e_j,e_i\>).\]
\item The above correspondences are (maybe?) isometric if we endow the complete norm on $\mathrm{CP}$.
\end{parts}
\end{prb}
\begin{pf}
(b)


\end{pf}

\section{Dilations and Extensions}

A linear map $\f:A\to B(H)$ is completely positive if and only if
\[\sum_{i,j}\<\f(a_i^*a_j)\xi_j,\xi_i\>\ge0,\qquad (a_i)\in A^n,\ (\xi_i)\in H^n.\]

\begin{prb}[Stinespring dilation]
Let $A$ be a C$^*$-algebra and $\f:A\to B(H)$ be a c.p.~map.
A \emph{Stinespring dilation} of $\f$ is a pair $(\pi,V)$ of a representation $\pi:A\to B(K)$ and a bounded linear operator $V:H\to K$ such that $\f(a)=V^*\pi(a)V$ for $a\in A$.
\[\begin{tikzcd}
B(K) \ar{dr}{V^*\cdot V} &\\
A \ar[swap]{r}{\f} \ar{u}{\pi} & B(H)
\end{tikzcd}\]
\begin{parts}
\item $\f$ has a Stinespring dilation $(\pi,V)$ such that $\bar{\pi(A)VH}=K$.
\item For a non-degenerate Stinespring dilation $(\pi,V)$ of $\f$, the operator $V$ is an isometry if and only if $\sup_\alpha\f(e_\alpha)=1$.
\end{parts}
\end{prb}
\begin{pf}
(a)
As we have done in the construction of the GNS representation, define a sesquilinear form on the algebraic tensor product $A\odot H$ such that
\[\<a\otimes\xi,b\otimes\eta\>:=\<\f(b^*a)\xi,\eta\>,\qquad a\otimes\xi,b\otimes\eta\in A\odot H.\]
It is positive semi-definite since the complete positivity of $\f$ implies
\[\<\sum_ja_j\otimes\xi_j,\sum_ia_i\otimes\xi_i\>=\sum_{i,j}\<\f(a_i^*a_j)\xi_j,\xi_i\>\ge0,\qquad a_i\otimes\xi_i\in A\odot H.\]
Then, we obtain a Hilbert space $K:=\bar{A\odot H/N}$, where $N:=\{\eta\in A\odot H:\<\eta,\eta\>=0\}$.
The above construction of a Hilbert space is sometimes called the separation and completion.

Define $\pi:A\to B(K)$ such that
\[\pi(a)(b\otimes\eta+N):=ab\otimes\eta+N,\qquad a\in A,\quad b\otimes\eta+N\in K,\]
and $V:H\to K$ such that
\[\<V\xi,b\otimes\eta+N\>:=\<\f(b^*)\xi,\eta\>,\qquad\xi\in H,\quad b\otimes\eta+N\in K.\]
The operator $V$ is well-defined by the Cauchy-Schwarz inequality
\begin{align*}
|\<\f(b^*)\xi,\eta\>|^2&=|\<\xi,\f(b)\eta\>|^2\le\|\xi\|^2\<\f(b^*)\f(b)\eta,\eta\>\\
&\le\|\xi\|^2\|\f\|\<\f(b^*b)\eta,\eta\>=\|\xi\|^2\|\f\|\|b\otimes\eta+N\|^2.
\end{align*}
Then, we can check $\pi(a)V\xi=a\otimes\xi+N$ for $a\in A$ and $\xi\in H$ from
\begin{align*}
\<\pi(a)V\xi,b\otimes\eta+N\>&=\<V\xi,a^*b\otimes\eta+N\>=\<\f(b^*a)\xi,\eta\>\\
&=\<a\otimes\xi+N,b\otimes\eta+N\>,\qquad b\otimes\eta+N\in K,
\end{align*}
so it follows that $V^*\pi(a)V=\f(a)$ for $a\in A$ from
\[\<V^*\pi(a)V\xi,\eta\>=\<V\xi,a^*\otimes\eta+N\>=\<\f(a)\xi,\eta\>,\qquad\xi,\eta\in H,\]
and the condition $\bar{\pi(A)VH}=K$.


\end{pf}


\begin{prb}[Voiculescu theorem]
Let $A$ be a unital C$^*$-algebra.
Let $\pi:A\to B(K)$ be a faithful non-degenerate representation and $\f:A\to B(H)$ be a u.c.p.~map.
Suppose further that $\f|_{\pi^{-1}(K(K))}=0$.

When do we need the faithfulness of $\pi$?
When do we need the unitality of $\f$?
When do we need the separability of $A$?
\begin{parts}
\item $\f$ is weakly$^*$ approximated by vector states, if $H$ is one-dimensional. (Glimm)
\item $\f$ is approximated by isometry conjugations in $L(A,B(H))$, if $H$ is finite-dimensional. (?)
\item $\f$ is approximated by isometry conjugations in $\f+L(A,K(H))$, if $H,K$ are separable.
\end{parts}
\end{prb}
\begin{pf}
(a)
Hahn-Banach separation and Weyl-von Neumann theorem.

(b)
correspondence for c.p.~maps to matrix algebras.

(c)
quasi-central approximate unit and block diagonal c.p.~maps.

\end{pf}


\begin{prb}[Arveson extension]
Let $A\subset B$ be C$^*$-algebras.
Let $\f:A\to B(H)$ be a c.p.~map and consider the following diagram:
\[\begin{tikzcd}
B\ar[dashed]{dr}{\tilde\f}&\\
A\ar{u}\ar[swap]{r}{\f}&B(H).
\end{tikzcd}\]
\begin{parts}
\item The norm preserving c.p.~extension $\tilde\f$ of $\f$ exists if $B$ is unital and $1_B\in A$.
\item The norm preserving c.p.~extension $\tilde\f$ of $\f$ exists if $\cA$ is unital and $B=A\oplus\C$.
\item The norm preserving c.p.~extension $\tilde\f$ of $\f$ exists if $\cA$ is non-unital and $B=\tilde\cA$.
\item The norm preserving c.p.~extension $\tilde\f$ of $\f$ always exists.
\end{parts}
\end{prb}



\begin{prb}[Representation extension]
Let $I$ be a left ideal of a C$^*$-algebra $B$.
For a representation $\pi:I\to B(H)$, there is a representation $\tilde\pi:B\to B(H)$ which extends $\pi$.
If $\pi$ is non-degenerate, the extension is unique and $\pi(e_\alpha b)\to\tilde\pi(b)$ and $\pi(be_\alpha)\to\tilde\pi(b)$ strongly for $b\in B$, where $e_i$ is an approximate unit of $I$.
The same holds for Hilbert module representations.
\end{prb}
\begin{pf}
We may assume $\pi$ is non-degenerate by replacing $H$ to $\bar{\pi(I)H}$.
Define $\tilde\pi:B\to B(H)$ such that
\[\tilde\pi(b)(\pi(a)\xi):=\pi(ba)\xi,\qquad a\in I,\ \xi\in H.\]
The well-definedness is from
\[\|\pi(ba)\xi\|^2=\<\pi(a^*b^*ba)\xi,\xi\>\le\|b\|^2\<\pi(a^*a)\xi,\xi\>=\|b\|^2\|\pi(a)\xi\|^2.\]
It is clearly a $*$-homomorphism and extends $\pi$.

For the uniqueness, if $\pi$ is non-degenerate and $\tilde\pi$ is a $*$-homomorphism which extends $\pi$, then
\[\tilde\pi(b)(\pi(a)\xi)=\tilde\pi(b)\tilde\pi(a)\xi=\tilde\pi(ba)\xi=\pi(ba)\xi,\]
which is unique by the density of $\pi(I)H$ in $H$.
\end{pf}

extension of representations for ideals

unique extension of c.p.~maps for hereditary subalgebras.






\section{Tensor products}

\begin{prb}[Maximal tensor products]
Let $A$ and $B$ be C$^*$-algebras.
\begin{parts}
\item (restrictions) A commuting pair of $*$-homomorphisms $\pi:A\to B(H)$ and $\pi':B\to B(H)$ corresponds to a $*$-homomorphism $\Pi:A\odot B\to B(H)$ via the relation $\Pi(a\otimes b)=\pi(a)\pi'(b)$.
\item $A\odot B$ admits a $*$-representation and every norms induced from these $*$-representations are uniformly bounded. So, we can define a maximal tensor norm on $A\odot B$.
\item $a\otimes-:B\to A\odot B$ is a bounded linear map for each $a\in A$ with respect to any C$^*$-norm on $A\odot B$. [BO, 3.2.5]
\end{parts}
\end{prb}


\begin{prb}[Minimal tensor product]
spatiality
\end{prb}
\begin{prb}[Takesaki theorem]
\end{prb}

Tensors with $M_n(\C)$, $C_0(X)$.


\begin{prb}[Haagerup tensor product]
\end{prb}

Trick

\section*{Exercises}
\begin{prb}
Let $A$ be a hereditary C$^*$-subalgebra of a C$^*$-algebra $B$ and let $b\in B_+$.
If for any $\e>0$ there is $a\in A_+$ such that $b-a\le\e$, then $b\in A$.
\end{prb}
\begin{pf}
For $a\in A_+$ satisfying $b\le a+\e\le(a^{\frac12}+\e^{\frac12})^2$, define
\[a_\e:=a^{\frac12}(a^{\frac12}+\e^{\frac12})^{-1}ba^{\frac12}(a^{\frac12}+\e^{\frac12})^{-1}\in A.\]
Then, 
\[\|b^{\frac12}-b^{\frac12}a^{\frac12}(a^{\frac12}+\e^{\frac12})^{-1}\|^2=\e\|(a^{\frac12}+\e^{\frac12})^{-1}b(a^{\frac12}+\e^{\frac12})^{-1}\|\le\e.\]
Thus $a_\e\to b$ in norm as $\e\to0$.
\end{pf}









\chapter{Hilbert modules}

\section{Hilbert modules}

\begin{prb}[Banach modules]
Let $A$ be a Banach algebra.
A \emph{Banach $A$-module} is a Banach space $E$ which is a $A$-module such that the action is bounded.
\begin{parts}
\item (Cohen factorization theorem) If $A$ has a left approximate unit, then $AE$ is closed in $E$.
\end{parts}
\end{prb}
\begin{pf}
Suppose $\xi\in\bar{AE}$.
We will construct a sequence $a_n$ in the unitization $\tilde A$ such that $a_n^{-1}\xi$ and $a_n$ are both Cauchy in $E$ and $\tilde A$ respectively, but the limit of $a_n$ is in $A$.
In order for this, we first need to check $a_n^{-1}\in\tilde A\setminus A$ can act on $E$, which is easy anyway.

Let $a_0=1\in\tilde A$ and suppose we have defined $a_n\in\tilde A$ such that $\|1-a_n\|\le1-2^{-n}$.
Since $\xi\in\bar{AE}$, we have $b\eta\in AE$ such that $\|\xi-b\eta\|<2^{-(3n+1)}$.
Since $A$ has an approximate unit, we have $e_n\in A$ such that $\|e_n\|\le1$, $\|1-e_n\|\le1$ (really?), and $\|(1-e_n)a_n^{-1}b\|\|\eta\|<2^{-(2n+1)}$.
Now inductively define
\[a_{n+1}:=a_n-2^{-(n+1)}(1-e_n)\in\tilde A.\]
Since $\|1-a_{n+1}\|\le1-2^{-(n+1)}$, every term in the sequence $a_n$ is invertible such that $\|a_n^{-1}\|\le2^n$.

Then, we can check $a_n$ converges to an element of $A$ because
\[a_n=a_0+\sum_{k=1}^n2^{-k}(1-e_{k-1})\to\sum_{k=1}^\infty2^{-k}e_{k-1}.\]
We can also check that $a_n^{-1}\xi$ is Cauchy because the identity
\[a_{n+1}^{-1}-a_n^{-1}=a_{n+1}^{-1}(a_n-a_{n+1})a_n^{-1}=2^{-(n+1)}a_{n+1}^{-1}(1-e)a_n^{-1}\]
is applied to get
\begin{align*}
\|(a_{n+1}^{-1}-a_n^{-1})\xi\|
&\le\|a_{n+1}^{-1}-a_n^{-1}\|\|\xi-b\eta\|+\|(a_{n+1}^{-1}-a_n^{-1})b\|\|\eta\|\\
&\le2^{-(n+1)}\|a_{n+1}^{-1}\|\|a_n^{-1}\|\|\xi-b\eta\|+2^{-(n+1)}\|a_{n+1}^{-1}\|\|(1-e)a_n^{-1}b\|\|\eta\|\\
&\le2^{-(n+1)}\cdot2^{n+1}\cdot2^n\cdot2^{-(3n+1)}+2^{-(n+1)}\cdot2^{n+1}\cdot2^{-(2n+1)}\\
&\le2^{-(2n+1)}+2^{-(2n+1)}=2^{-2n}.
\end{align*}
It implies that there is $\zeta\in E$ such that $a_n^{-1}\xi\to\zeta$ and $\|a_n^{-1}\xi-\zeta\|\le2^{-(2n-1)}$.

Then,
\[\|\xi-a\zeta\|\le\|a_n\|\|a_n^{-1}\xi-\zeta\|+\|a_n-a\|\|\zeta\|\le2^{-(n-1)}+2^{-n}\|\zeta\|\]
deduces that $\xi=a\zeta$.
\end{pf}

\begin{prb}[Finsler modules]
Let $A$ be a C$^*$-algebra.
\end{prb}




\begin{prb}[Hilbert modules]
Let $B$ be a C$^*$-algebra.
A \emph{right Hilbert $B$-module} or simply a \emph{Hilbert $B$-module} is a right module $E$ over the complex algebra $B$ which is not involutive, together with a map $\<-,-\>:E\times E\to B$ such that for $\xi,\eta\in E$ and $b\in B$ we have
\begin{enumerate}[(i)]
\item $\<\xi,\xi\>\ge0$ and $\<\xi,\xi\>=0$ if and only if $\xi=0$,
\item $\<\eta,\xi b\>=\<\eta,\xi\>b$,
\item $\<\eta,\xi\>^*=\<\xi,\eta\>$,
\end{enumerate}
and $E$ is Banach with respect to the norm $\|\xi\|:=\|\<\xi,\xi\>\|^{\frac12}$.
The map $\<-,-\>$ is called the \emph{$B$-valued inner product}.
It is a non-commutative analogue of Hermitian bundles.
Even though the complex scalars act on $E$ from right in the rigorous sense, we will frequently write the scalar multiplication at left.
\begin{parts}
\item The right action by $b$ is bounded and the norm is coincides with $B$. It does not preserve the involutions and is not adjointable in general.
\item The right action is always non-degenerate. In particular, it follows that $\xi1=\xi$ for $\xi\in E$ if $A$ is unital.
\item The right action is faithful if and only if $E$ is full, i.e.~the ideal $\<E,E\>$ of $A$ is dense in $A$.
\item Examples: $B$ itself, $B^n$, $\ell^2(\N,B)$, etc.
\item direct sum, tensor product, localization
\end{parts}
\end{prb}
\begin{pf}

(c)
Consider the approximate unit $e_i$ of $\<E,E\>$.
Then, we can show $\xi e_i\to\xi$ in $E$ for each $\xi\in E$, so $EB$ is dense in $E$.
\end{pf}

\begin{prb}[Adjointable and compact operators]
Let $E$ and $F$ be Hilbert $B$-modules over a C$^*$-algebra $B$.
An operator $T:E\to F$ is called an \emph{adjointable operator} if there is an operator $T^*:F\to E$ such that $\<T\xi,\eta\>=\<\xi,T^*\eta\>$ for all $\xi\in E$ and $\eta\in F$, and called \emph{compact} if it is a norm limit of adjointable operators of the form $\theta_{\eta,\xi}:E\to F$ with $\xi\in E$ and $\eta\in F$, where $\theta_{\eta,\xi}:=\eta\<\xi,-\>$, which has an adjoint $\theta_{\xi,\eta}$
The Banach spaces of all adjointable and compact operators $E\to F$ are denoted by $B(E,F)$ and $K(E,F)$ respectively, and these will not be used in the sense of Banach spaces.
\begin{parts}
\item An adjointable operator is a bounded $B$-module map.
\item $K(E)$ is a closed essential ideal of a C$^*$-algebra $B(E)$.
\item
\end{parts}
\end{prb}
\begin{pf}
The $B$-linearity is clear.
The boundedness follows from the uniform boundedness principle.

\end{pf}


\begin{prb}[Weak topologies for Hilbert modules]
Let $E$ and $F$ be Hilbert $B$-modules for a C$^*$-algebra $B$.
The \emph{strict topology} refers to the strong$^*$ operator topology of $B(E)$.



On the trivial Hilbert $B$-module $B$, $b_i\to0$ strictly iff $b_i,b_i^*\to0$ weakly.
If $B$ is unital, the strict topology on $B$ and the norm topology coincide.
An adjointable operator is weakly continuous.
\end{prb}



On Hilbert modules:
\begin{itemize}
\item polarization identity? OK,
\[\<\eta,\xi\>=\frac14\sum_{k=0}^3i^k\<\xi+i^k\eta,\xi+i^k\eta\>\]
\item unbounded adjointable operators and spectral theory?
\item polar decomposition? especially for unbounded adjointable operators?
\item bounded sesquilinear form?
\item Riesz representation? OK for adjointable operator $l:E\to B$, there is $\eta:=l^*1$ (The classical Riesz representation states that every bounded linear functional is automatically adjointable in the sense of Hilbert $\C$-modules)
\item alaoglu?
\item uniform boundedness principle?
\item 
\end{itemize}
















\begin{prb}[Multiplier algebra]
Four descriptions for a multiplier algebra:
double centralizers vs essential ideal vs multipliers in von Neumann algebra vs Hilbert module

1.
Let $B$ be a C$^*$-algebra.
A \emph{double centralizer} of $B$ is a pair $(L,R)$ of bounded linear maps on $A$ such that $aL(b)=R(a)b$ for all $a,b\in B$.
The \emph{multiplier algebra} $M(B)$ of $B$ is defined to be the set of all double centralizers of $B$.
There is another characterization of $M(B)$ as the set of adjointable operators to itself.
Even if the notation $B(B)$ may cause confusion, we can write $M(B)$ to avoid this.

2.
An ideal $I$ of $B$ is called an \emph{essential} if it is a full Hilbert $B$-submodule of $B$.



Every C$^*$-algebra $A$ is a correspondence over $M(A)$.

\begin{parts}
\item $\|\pi(a-e_\alpha a)\xi\|^2$
\item If $a_\alpha$ are unitary, the convergences in the strict topology and the weak topology(how to define this?) coincide.
\item If $a_\alpha$ are increasing, the convergences in the strict topology and the weak topology(how to define this?) coincide.
\item $M(K(E))\cong B(E)$.
\item $M(C_0(\Omega))\cong C_b(\Omega)$.
\end{parts}
\end{prb}
\begin{pf}
First we claim $C_0(\Omega)$ is an essential ideal of $C_b(\Omega)$.
Since $C_b(\Omega)\cong C(\beta\Omega)$, and since closed ideals of $C(\beta\Omega)$ are corresponded to open subsets of $\beta\Omega$, $C_0(\Omega)\cap J$ is not trivial for every closed ideal $J$ of $C_b(\Omega)$.

Now we have an injective $*$-homomorphism $C_b(\Omega)\to M(C_0(\Omega))$, for which we want to show the surjectivity.
Let $g\in M(C_0(\Omega))_+$.


\end{pf}

characterization in an inclusion into a von Neumann algebra.

relations between Hilbert $B(H)$-modules and representations





\begin{prb}
C$^*$-algebras together with a non-degenerate representation $C_0(X)\to Z(M(A))$.
\end{prb}



\begin{prb}[Dauns-Hoffman theorem]
\end{prb}


\begin{prb}
\,
\begin{parts}
\item $\mathrm{LCH}_{\mathrm{prop}}$ is equivalent to $\mathrm{CC^*Alg}_{\mathrm{nondeg}}$.
\item $\mathrm{CH}_*$ is equivalent to $\mathrm{CC^*Alg}$.
\item $\mathrm{LCH}$ is equivalent to $\mathrm{CC^*Alg}_{\mathrm{mor}}$.
\end{parts}
\end{prb}


부분적 결과에 대한 인용?
위상적 구조를 가지지 않은 통상의 벡터공간을 공역으로 하는 인자화대수로부터 적절한 조건에서 정점대수를 도출할 수 있다는 것을 보였다
예상의 의미가 잘 안 와닿아서 여러번 읽었다...(장의 집합이라든가)
본 문제 혹은 예상의 테크니컬한 어려움은 무엇이며 그것을 어떻게 해결하고자 하는가
연구2: 게이지변환을 이용한 컨포멀 벡터의 구성방법이 적혀있으나, 어디까지가 알려진 거고 어디부터가 본인이 하려고 하는 부분인지.. 있는 거 같긴 한데 분명하지는 않은 듯?

難航するが같은 표현은 부정적인 이미지를 줄 수 있을 것 같음


\section{C$^*$-correspondences}


\begin{prb}[C$^*$-correspondences]
Let $A$ and $B$ be C$^*$-algebras.
A \emph{C$^*$-correspondence}, \emph{C$^*$-bimodule}, or just simply a \emph{correspondence} over $A$ and $B$, or from $A$ to $B$, is a Hilbert $B$-module $E$ together with a $*$-homomorphism $\f:A\to B(E)$, called the \emph{left action}.
We say $E$ is \emph{faithful} or \emph{non-degenerate} if the left action is faithful or non-degenerate, respectively.
\begin{parts}
\item If $\f:A\to M(B)$ is a unital completely positive map, then we can construct a natural correspondence $E$ from $A$ to $B$ by mimicking the GNS construction on $A\odot B$.
\item If $\f:A\to M(B)$ is a non-degenerate $*$-homomorphism, $\f\in\Mor(A,B)$ in other words, then we can associate a canonical $A$-$B$-correspondence $B$ such that the left action is realized with $\f$.
More precisely, $\iota:E\to B:a\otimes b\mapsto\f(a)b$ provides a well-defined linear isomorphism (surjectivity follows from the density of $\f(A)B$ in $B$ and the Cohen factorization theorem) and the two actions on $E$ is described by $\iota(a\xi b)=\f(a)\iota(\xi)b$.
\end{parts}
\end{prb}


\begin{prb}[Pimsner construction]
C$^*$-correspondences over $A$ can be interpreted as a generalized automorphism on $A$, and the Pimsner construction defines a new C$^*$-algebra generated by the generalized cyclic action provided by a C$^*$-correspondence.
Let $E$ be a C$^*$-correspondence over a C$^*$-algebra $A$.
Let $B$ be a C$^*$-algebra and see it as a trivial C$^*$-correspondence over $B$.
A \emph{Toeplitz representation} of $E$ on $B$ is a pair $(\pi,\tau)$ of a $*$-homomorphism $\pi:A\to B$ and a linear map $\tau:E\to B$ such that
\[\pi(\<\xi,\eta\>)=\tau(\xi)^*\tau(\eta),\qquad\tau(\f(a)\xi)=\pi(a)\tau(\xi).\]
We define the \emph{Katsura ideal}
\[J(E):=\f^{-1}(K(E))\cap\f^{-1}(0)^\perp.\]

We say a Toeplitz representation of $E$ is \emph{covariant} if
\[\psi(\f(a))=\pi(a),\qquad a\in J(E).\]
\begin{parts}
\item
Let $(A,\Z,\alpha)$ be a C$^*$-dynamical system and consider the canonical C$^*$-correspondence $A$ over $A$ with the left action $\f:=\alpha_1\in\Aut(A)\subset\Mor(A)$.
This correspondence is full, faithful, and non-degenerate.
Note that also we have $J(A)=\f^{-1}(A)\cap A=A$.
If $(\pi,\tau)$ is an any representation of this C$^*$-correspondence $A$ on $B$, then 
\end{parts}
\end{prb}

How can we decribe representations of C$^*$-correspondence $A$ with left action $\f\in\Aut(A)$ in terms of covariant representations of the C$^*$-dynamical system $(A,\Z,\alpha)$ with $\alpha_n=\f^n$?



as a morphism
sub and quotient, direct sum, tensor product,

Toeplitz-Cuntz
Toeplitz-Pimsner
Cuntz-Pimsner
Cuntz-Krieger



Subproduct systems


\section{Morita equivalence}



Induced representations?







\chapter{Constructions}


\section{Categorical constructions}

inverse limits: direct sum, direct product, restricted direct sum, locally C$^*$-algebras.

Infinite direct sums and direct products are ill-behaved in the category of C$^*$-algebras.
An infinite direct sum must be interpreted as complete Hausdorff spaces, not a pointed compact Hausdorff space.
For example, after adding a base point, the spectrum of $\bigoplus_{i=1}^\infty C_0(\R)$ corresponds to the Hawaiian earing, and the spectrum of $\prod_{i=1}^\infty C_0(\R)$ corresponds to the Stone-\v Cech compactification of the infinite wedge of circles.
We cannot describe the infinite wedge of circles in terms of C$^*$-algebras, so we need locally C$^*$-algebras.


direct limits: filtered limits, tensor products, free products, amalgamated free products.




\begin{prb}[Locally C$^*$-algebras]
A \emph{locally C$^*$-algebra} is a complete topological $*$-algebra whose topology is generated by C$^*$-semi-norms.
We adopt the convention that a \emph{homomorphism} between locally C$^*$-algebras means a continuous $*$-homomorphism.
\begin{parts}
\item A topological $*$-algebra is a locally C$^*$-algebra if and only if it is an inverse limit of unital C$^*$-algebras.
\end{parts}
\end{prb}
\begin{pf}
(a)
Let $A$ be a locally C$^*$-algebra.
The set of continuous C$^*$-seminorms on $A$ is a directed set.
Construct an inverse system...
Since every C$^*$-algebra is a maximal ideal of a unital C$^*$-algebra of codimension one, we may assume that the objects in this inverse system is unital...
Also, elements of $A$ are represented by coherent sequences.
\end{pf}




\section{Crossed products}


\begin{prb}[Group algebras]
Let $G$ be a locally compact group.

\end{prb}


type I, subhomogeneous


crystallographic
discrete heisenberg
free groups
projectionless of $C_r^*(F_2)$



\begin{prb}[Enveloping C$^*$-algeberas]
Let $A$ be a $*$-algebra.
A \emph{C$^*$-norm} is an submultiplicative norm satisfying the C$^*$-identity.
Does $A$ have enough $*$-representations?
\begin{parts}
\item A complete C$^*$-norm is unique if it exists.
\item For every C$^*$-norm $\alpha$ on $A$, there is a $*$-isometry $\pi:A\to B(H)$.
\item For maximal C$^*$-norm, there is a universal property. The maximal C$^*$-norm can be obtained by running through cyclic representations.
\end{parts}
\end{prb}




\begin{prb}[C$^*$-dynamical system]
Let $G$ be a locally compact group.
A \emph{C$^*$-dynamical system} or a \emph{$G$-C$^*$-algebra} is a C$^*$-algebra $A$ together with a group homomorphism $\alpha:G\to\Aut(A)$ that is continuous in the point-norm topology.
We will often write a triple $(A,G,\alpha)$ instead of $A$ to refer to a C$^*$-dynamical system.
\begin{parts}
\item There is an equivalence between categories of locally compact transformation groups and C$^*$-dynamical system on abelian C$^*$-algebras.
\end{parts}
\end{prb}


On $U(H)$, the strict topology and the strong operator topology are equal.
Therefore, we have three topologies to consider: strong, weak, and $\sigma$-weak.

\begin{prb}[Covariant representation]
Let $G$ be a locally compact group.

A \emph{covariant representation} of a C$^*$-dynamical system $(A,G,\alpha)$ is a $G$-equivariant $*$-homomorphism $\pi:(A,G,\alpha)\to(B(H),G,\beta)$ for a C$^*$-dynamical system $(B(H),G,\beta)$, where a Hilbert space $H$.
\begin{parts}
\item
There exists a unitary representation $u:G\to B(H)$ such that $\pi(\alpha_sa)=u_s\pi(a)u_s^*$.
\item (Integrated form)
There is a one-to-one correspondence between covariant representations of $(A,G,\alpha)$ and $*$-representations of $L^1(G,A)$. (non-degenerate)
\end{parts}
\end{prb}

Note that we have a homeomorphism $\Aut(K(H))\cong PU(H)$ between the point-norm topology and the strong operator topology.

$\Z$-action, $\Homeo$-action, left multiplication of subgroup
induced representation
regular representation $(C_0(G),G,\lambda)\to(B(L^2(G)),G,\lambda)$.


commutative case




\section{Graph algebras}






\section{Groupoid algebras}









\part{Properties}
\chapter{Approximation properties}
\section{Nuclearity and exactness}

finite dimensional[BO, 3.3.2], abelian, AF
permanence properties


\begin{prb}[Completely positive approximation property]
Let $A$ be a C$^*$-algebra.
We say $A$ has the \emph{completely positive approximation property} if the identity is contained in the point-norm, or equivalently the point-weak closure of $\cF$ in $L(A)$.
\begin{parts}
\item If $A$ has the completely positive approximation property, then $A$ is nuclear.
\item If $A$ is nuclear, then $A$ has the completely positive approximation property.
\end{parts}
\end{prb}
\begin{pf}

(b)



Let $E\subset A$ and $F\subset A^*$ be finite subsets and fix $\e>0$.
We want to find completely positive contractions $\f:A\to M_n(\C)$ and $\psi:M_n(\C)\to A$ such that
\[|l(a)-l(\psi\circ\f(a))|<\e,\qquad a\in E,\ l\in F.\]
To implement the approximation, we would like to regard a bounded linear operator on $A$ as a state of a tensor product of C$^*$-algebras, which maps $\theta\in L(A)$ to the linear functional characterized by $a\otimes l\mapsto l(\theta(a))$.
However, since $A^*$ is not a C$^*$-algebra, we embed $A^*$ locally in $B(H)$ through the Radon-Nikodym type result.
Let $\pi:A\to B(H)$ be the cyclic representation obtained from a positive linear functional that dominates $F$ and $\Omega$ the cyclic vector such that there is a linear map $\pi':F\to\pi(A)'$ satisfying
\[l(a)=\<\pi(a)\pi'(l)\Omega,\Omega\>,\qquad a\in E,\ l\in F.\]
Now the duality of $A$ and $F$ is embodied in the tensor product representation
\[\pi\times i:A\otimes_{\max}\pi(A)'\to B(H)\]
together with a cyclic vector $\Omega\in H$.
Here the nuclearity is used to write $A\otimes_{\max}\pi(A)'=A\otimes_{\min}\pi(A)'$.

If we take any faithful representation $\rho:A\to B(K)$, then we obtain a faithful representation
\[\rho\otimes i:A\otimes_{\min}\pi(A)'\to B(K\otimes H).\]
By the Hahn-Banach separation, the state $(\pi\times i)^*\omega_\Omega$ on $A\otimes_{\min}\pi(A)'$ can be approximated weakly$^*$ by convex combinations of vector states in $B(K\otimes H)$.
In particular, by the density of $\pi(A)\Omega$ in $H$, we have algebraic tensors $(t_k)_{k=1}^m\subset K\odot\pi(A)\Omega$ such that
\[\Bigl|\omega_\Omega((\pi\times i)(a\otimes\pi'(l)))-\sum_{k=1}^m\lambda_k\omega_{t_k}((\rho\otimes i)(a\otimes\pi'(l)))\Bigr|<\e\tag{\dagger}\]
for all $a\in E$ and $l\in F$, where $\lambda_k\ge0$, $\sum_{k=1}^m\lambda=1$.

If we write each element $t\in K\odot\pi(A)\Omega$ as
\[t=\sum_{i=1}^n\eta_i\otimes\pi(b_i)\Omega,\qquad\eta_i\in K,\ b_i\in A,\]
then
\begin{align*}
\omega_t((\rho\otimes i)(a\otimes\pi'(l)))
&=\left\<(\rho(a)\otimes\pi'(l))\Bigl(\sum_{j=1}^n\eta_j\otimes\pi(b_j)\Omega\Bigr),\Bigl(\sum_{i=1}^n\eta_i\otimes\pi(b_i)\Omega\Bigr)\right\>\\
&=\sum_{i,j=1}^n\<\rho(a)\eta_j,\eta_i\>\<\pi'(l)\pi(b_i^*b_j)\Omega,\Omega\>\\
&=l\Bigl(\sum_{i,j=1}^n\<\rho(a)\eta_j,\eta_i\>b_i^*b_j\Bigr).
\end{align*}
If we define completely positive maps $\f:A\to M_n(\C)$ and $\psi:M_n(\C)\to A$ for each $\tau$ such that
\[\f(a):=[\<\rho(a)\eta_j,\eta_i\>],\quad\psi([\delta_{ik}\delta_{jl}]):=b_k^*b_l,\]
then we have $\omega_t(a\otimes\pi'(l))=l(\psi\circ\f(a))$.
We may assume $\f$ and $\psi$ are contractive by adjusting their norms.

Since $\mu(a\otimes\pi'(l))=l(a)$ and since the completely positive contractions which factor through a matrix algebra form a convex set, we have completely positive contractions $\f:A\to M_n(\C)$ and $\psi:M_n(\C)\to A$ such that the inequality (\dagger) is rewritten as
\[|l(a)-l(\psi\circ\f(a))|<\e,\]
so we are done.
\end{pf}

The set $\cF$ of factorable maps is a convex set of $L(A)$.
Note that we have an embedding
\[L(A)\hookrightarrow L(A,A^{**})=\lim_{\substack{\longleftarrow\\F}}L(A,F^*).\]
We have a continuous bijection
\[(A\hat\otimes_\pi F)^*\to L(A,F^*).\]
If we let $M:=\pi(A)''\subset B(H)$ be the GNS representation for $F$, then the Radon-Nikodym theorem on commutant gives rise to a continuous map
\[(A\mathrel{\hat\otimes}_\pi M')^*\to(A\mathrel{\hat\otimes}_\pi F)^*.\]

\[\begin{tikzcd}
B(K\otimes\pi(A)\Omega)^* \dar[->>] & B(\pi(A)\Omega)^* \dar[->>] &\\
(A\otimes_{\min}M')^*\rar[>->]
&(A\otimes_{\max}M')^*\rar[>->]
&(A\mathrel{\hat\otimes}_\pi M')^*
\end{tikzcd}\]
The first map is in fact surjective by the nuclearity.





quotients of nuclear
local reflexivity



\begin{prb}
A C$^*$-algebra $C$ is called \emph{injective} every completely positive map $\f:A\to C$ from a C$^*$-subalgebra $A$ of a C$^*$-algebra $B$ is extended to a completely positive map $\tilde\f:B\to C$.
A von Neumann algebra is called injective if it is injective as a C$^*$-algebra.
(operator subsystem? unital?)

The C$^*$-algebra $B(H)$ is injective, and its image of completely positive idempotent is injective.
A von Neumann algebra on $M$ on $H$ is injective if and only if there is a conditional expectation $B(H)\to M$.

\end{prb}



$A^{**}$ semi-discrete -> $A$ nuclear is done by four step approximation

The reverse implication follows from $A$ is nuclear -> $A'$ is injective -> $A''$ is injective -> $A''$ is semi-discrete.


Let $A$ be nuclear.
Note $A^{**}=I^{**}\oplus(A/I)^{**}$.
Since $A^{**}$ is semi-discrete, $(A/I)^{**}$ is semi-discrete.
Therefore, $A/I$ is nuclear.








a separable C$^*$-algebra is nuclear if and only if every factor representation is hyperfinite.

Extension properties
weak expectation property
relatively weakly injective
maximal tensor product inclusion problem



excision: Akemann-Anderson-Pedersen

\section{Quasi-diagonality}

\begin{prb}[Weyl-von Neumann theorem]
A self-adjoint bounded operator is quasi-diagonal.
\end{prb}

\begin{prb}[Glimm lemma]
If a state $\omega$ of $B(H)$ vanishes on $K(H)$, then it is a weak$^*$ limit of vector states.
\end{prb}

\begin{prb}[Voiculescu theorem]
\end{prb}


\begin{prb}[Quasi-diagonal algebras]
An operator $a\in B(H)$ is called \emph{quasi-diagonal} if there is a net of projections $p_i\in B(H)$ such that $[p_i,a]\to0$ in norm and $p_i\uparrow\id_H$ strongly.
A C$^*$-algebra is called \emph{quasi-diagonal} if it admits a faithful representation whose image is quasi-diagonal.
\end{prb}

faithful non-degenerate essential representations of a quasi-diagonal C$^*$-algebra are all quasi-diagonal

locally quasi-diagonal

\section{AF-embeddability}













\chapter{Amenability}


\section{Amenable groups}


\section{Amenable actions}
crossed products
$Z_2$-grading
Connes-Feldman-Weiss
Anantharaman-Delaroche
Gromov boundaries
approximately central structure?
dynamical Kirchberg-Phillips

stably finite
dynamical Elliott program

Ornstein-Weiss-Rokhlin lemma

\section{Exact groups}
Exact groups

\section{Other properties}
Kazdahn property (T)
factorization property
Haagerrup property


Kaplansky conjecture




A state $\tau$ on $A$ is called an \emph{amenable trace} if there is a state $\omega$ of $B(H)$ such that $\omega$ extends $\tau$ and $\omega(uxu^*)=\omega(x)$ for $x\in B(H)$ and $u\in U(A)$.
It is automatically tracial.
The amenability of a trace does not depend on the choice of faithful representation of $A$, using the Arveson extension and the multiplicaitve domain.

For a discrete group $\Gamma$, $C^*_r(\Gamma)$ is amenable if and only if has an amenable tracial state.
Note that a mean is a state of $\ell^\infty(\Gamma)$, which may not be normal.





\chapter{Simplicity}


Furstenburg boundary



















\part{Invariants}
\chapter{Operator K-theory}

\section{Zeroth K-groups}
Three pictures: projections of $M_n(A)$(standard), projections of $A\otimes K(H)$(recall that $K(H)$ is AF and hence nuclear), algebraically finitely generated projective Hilbert modules over $A$.


\begin{prb}[Equivalences of projections]
Let $A$ be a unital C$^*$-algebra.
Let $p$ and $q$ be projections in $A$.
Recall that they are called \emph{Murray-von Neumann equivalent} or just equivalent, denoted $p\sim q$, if $p=v^*v$ and $q=vv^*$ for some $v\in A$, \emph{unitarily equivalent}, denoted by $p\sim_uq$, if $p=u^*qu$ for some $u\in U(A)$, and \emph{homotopic}, denoted by $p\sim_hq$, if there is a continuous path in $P(A)$ connecting them.
\begin{parts}
\item If $p\sim_hq$, then $p\sim_uq$, and if $p\sim_uq$, then $p\sim q$.
\item If $p\sim q$, then $p\oplus0\sim_uq\oplus0$ in $M_2(A)$.
\item If $p\sim_uq$, then $p\oplus0\sim_hq\oplus0$ in $M_2(A)$.
\end{parts}
\end{prb}


Almost projection: if $\|a^2-a\|<\e$, then $\|p-a\|<2\e$ for some $p\in A$.

If $p\in A=\colim_i A_i$, then $\|p_i-p\|<\e$ for some $p_i\in A_i$.


\begin{prb}[Properties of $K(H)$]
Let $H$ be a separable Hilbert space.


\end{prb}

\begin{prb}[Definition of zeroth K-group]
Let $A$ be a unital C$^*$-algebra.
Define $V(A):=\bigcup_{n=1}^\infty P(M_n(A))/\sim$.
It gives a functor from the category of unital C$^*$-algebras to the category of ordered abelian monoid with cancellation property.
If $A$ is unital, we define $K_0(A):=G(V(A))$, the Grothendieck group of the monoid $V(A)$.
Its elements can be described by $[p]-[p_n]$.
\begin{parts}
\item $V(M_n(\C))\cong\Z_{\ge0}$ because two projections are equivalent iff they have same range dimensions, so $K_0(M_n(\C))\cong\Z$.
\item $V(K(H))\cong\Z_{\ge0}=\mathrm{Card}_{<\omega}$, $V(B(H))\cong\mathrm{Card}_{\le\dim H}$, $V(Q(H))\cong\{0\}\cup(\mathrm{Card}_{\ge\omega}\cap\mathrm{Card}_{\le\dim H})$, so $K_0(B(H))\cong K_0(Q(H))\cong0$. (Weyl-von Neumann theorem: self-adjoint elements of $Q(H)$ with same spectrum are unitarily equivalent)
\item $K_0(C(S^2))\cong\Z^2$.
\item For a II$_1$ factor $M$, $K_0(M)\cong\R$.
\item $K_0(\cO_n)\cong\Z/(n-1)\Z$.
\end{parts}
\end{prb}


\begin{prb}[Relative K-theory]
We want to discuss the exactness of K-theory.
For this, we have to consider pairs of C$^*$-algebras.
We define a \emph{pair} of C$^*$-algebras as a surjective $*$-homomorphism between unital C$^*$-algebras.
Let $\pi:A\to B$ is a pair of C$^*$-algebras.
Then, $K_0(A,B)$ can be concretely described or defined by the set of equivalence classes of $(p,q,v)$, where $p$ and $q$ are projections in $M_\infty(A)$ and $v\in M_\infty(A)$ satisfies $\pi(p)=\pi(v^*v)$ and $\pi(q)=\pi(vv^*)$.
In fact, we can show $K_0(A,B)$ only depends on the kernel $I:=\ker\pi$.
It is called the excision theorem.
For a general non-unital C$^*$-algebra $I$, it is well-defined that
\[K_0(I):=K_0(A,A/I),\]
where $A$ is any unitization of $I$.
We can show that if $I$ is unital, then it is naturally isomorphic to the original without-base-point definition of K-theory.(for example, $K_0(A)\cong K_0(A\oplus\C,\C)$ for unital $A$)
In particular, since $K_1(\C)=0$, the six-term exact sequence implies that $K_0(I)\cong\ker(K_0(\tilde I)\to K_0(\C))$, and since $0\to I\to\tilde I\to\C\to0$ splits, we have $K_0(I)\oplus\Z\cong K_0(\tilde I)$.
A generally non-unital C$^*$-algebra is the non-commutative analogue of the pointed quotient of compact pairs.


Even if $A$ and $B$ are non-unital, one can check the followings are exact:
\[K_0(I)\to K_0(A)\to K_0(B)\]
$[p,q,v]\mapsto[p]-[q]\mapsto$...

When we consider exact sequences, we may think every algebra $A$ in K-theory as a pair $(B,C)$ such that $B/C\cong A$!
If the algebra $A$ is unital, then it is also possible to think it as a space without base point, as in the definition of $K_0(A)$.
The basic way to think is to consider non-unital C$^*$-algebras $A$ and $K_0(A)$ as the paired or pointed version.
But we do not need the tilde.

\[\{\text{pair of spaces}\}\twoheadrightarrow\{\text{pointed spaces}\}\hookleftarrow\{\text{spaces}\}\]
\[\{\text{pair of C$^*$-algebras}\}\twoheadrightarrow\{\text{C$^*$-algebras}\}\hookleftarrow\{\text{unital C$^*$-algebras}\}\]
The first map is quotient.
The second map is adjoining a new point for space, the inclusion for algebras.
The first two categories are indistinguishable in generalized cohomology theoreis or homotopy theories.
We do not have to introduce the notation $\tilde K$, because we basically consider the unital algebra $C(X)$ not as a pointed space $(X,x_0)$ (like in topology), but as $(X\cup*,*)$, i.e.~$K(C_0(X))=\tilde K(X\cup*,*)$ for compact or non-compact $X$.
\end{prb}

As $K_0:\mathrm{C^*Alg}\to\mathrm{grAb}$, $K_0$ satisfies the axioms for cohomology theories
\begin{itemize}
\item functoriality
\item homotopy invariance
\item FINITE product-preserving$^*$
\item half-exactness
\item long exactness
\end{itemize}
with additional properties
\begin{itemize}
\item lax symmetric monoidal functor
\item filtered colimit-preserving
\item $\K$-stable
\item partial order
\item ring axioms for $K_0$ only on commutatives
\end{itemize}

$^*$Here we only consider finite product-preserving because the infinite direct product does not mean the infinite wedge sum in the category of C$^*$-algebras.
We need to consider locally C$^*$-algebras.



\begin{prb}[Homotopy of $*$-homomorphisms]
Let $A,B$ be C$^*$-algebras.
Two $*$-homomorphisms in $\Mor(A,B)$ are said to be \emph{homotopic} if they are connected by a path in $\Mor(A,B)$ that is continuous with the point-norm topology.
\begin{parts}
\item For pointed compact Hausdorff spaces $(X,x_0),(Y,y_0)$, two pointed maps $\f_0,\f_1:X\to Y$ are homotopic if and only if $\f_0^*,\f_1^*:C_0(Y\setminus\{y_0\})\to C_0(X\setminus\{x_0\})$ are homotopic.
\end{parts}
\end{prb}
\begin{pf}
(a)
Suppose $\f_0$ and $\f_1$ are connected by a homotopy $\f_t$.
Fixing $g\in C_0(Y)$ and $t_0\in I$, we want to show
\[\lim_{t\to t_0}\sup_{x\in X}|g(\f_t(x))-g(\f_{t_0}(x))|=0.\]
Since the function $g$ is uniformly continuous, with respect to an arbitrarily chosen uniformity on $Y$, so that there is an entourage $E\subset Y\times Y$ such that $(y,y')\in E\circ E$ implies $|g(y)-g(y')|<\e$.
Using compactness we have a finite sequence $(y_i)_{i=1}^n\subset Y$ such that for every $y$ there is $y_i$ satisfying $(y,y')\in E$.
Then, $\f^{-1}(E[y_i])$ is a finite open cover of $X\times I$, so we have $\delta$ such that $|t-t_0|<\delta$ implies for any $x\in X$ the existence of $i$ satisfying $(\f_t(x),y_i)\in E$ and $(\f_{t_0}(x),y_i)\in E$, which deduces the desired inequality.

Conversely, suppose $\f_0^*$ and $\f_1^*$ are connected by a homotopy $\f_t^*$.
By taking dual, we can induce $\f_t:X\to Y$ such that $g(\f_t(x))=(\f_t^*g)(x)$ for each $g\in C(Y)$ from $\f_t^*$ via the embedding $X\to M(X)$ by Dirac measures.
Let $V$ be an open neighborhood of $\f_{t_0}(x_0)$ and take $g\in C(Y)$ such that $g(\f_{t_0}(x_0))=1$ and $g(y)=0$ for $y\notin V$.
Now we have an open neighborhood $U$ of $x_0$ such that $x\in U$ implies $|(\f_{t_0}^*g)(x)-(\f_{t_0}^*g)(x_0)|<\frac12$.
Also we have $\delta>0$ such that $|t-t_0|<\delta$ implies $\|\f_t^*g-\f_{t_0}^*g\|<\frac12$.
Therefore, $(x,t)\in U\times(t_0-\delta,t_0+\delta)$ implies $g(\f_t(x))>0$, hence $\f_t(x)\in V$, which means $X\times I\to Y:(x,t)\mapsto\f_t(x)$ is continuous.
\end{pf}


\[K_0(\C)=\Z,\quad K_0(C_0(\R))=0,\quad K_1(C_0(\R))=K_0(C_0(\R^2))=\Z\]
\[K^0(*)=\Z,\quad K^0(S^1)=\Z,\quad K^1(S^1)=K^0(S^2)=\Z[x]/(x-1)^2\]


Let $X$ be a locally compact Hausdorff space, and $(X_+,*)=(X\sqcup\{*\},*)$ be the associated pointed compact Hausdorff space.
Then, the K-theory with compact supports has
\[K_0(X)=K_0(X_+,*)=\tilde K_0(X_+)=K^0(C_0(X)).\]



\section{First K-groups}


$K_1$ satisfies
long exactness(triangulated structure), bott periodicity, ring structure?

$K(\C)\cong\Z[\beta^{\pm1}]$.

\[CB:=\{f\in B\otimes C([0,1]):f(0)=0\},\qquad C_\f:=\{(a,f)\in A\oplus CB:f(1)=\f(a)\}.\]

The mapping cone can be defined by an exact sequence
\[0\to C_\f\to M_\f\to B\to0,\]
or alternatively by the pullback
\[\begin{tikzcd}
C_\f\rar\dar\ar[dr, phantom, "\lrcorner", very near start] & CB\dar{f\mapsto f(1)} \\
A \rar & B.
\end{tikzcd}\]

The suspension can defined by an exact sequence
\[0\to\Sigma B\to CB\to B\to0,\]
or alternatively by the pullback
\[\begin{tikzcd}
\Sigma B\rar\dar\ar[dr, phantom, "\lrcorner", very near start] & CB\dar{f\mapsto f(1)} \\
CB \rar & B.
\end{tikzcd}\]

We can see that $CB$ is contractible, and $\Sigma B$ is homotopic to the pullback $C_\f\oplus_ACA$.

distinguished triangle
\[\Sigma B\to C_\f\to A\xrightarrow{\f}B\]

Do not forget to describe the induced maps for K-groups!



$K_{-1}(A):=K_0(\Sigma A)$.





local Banach algebras



\begin{prb}[Pimsner-Voiculescu exact sequence]
\end{prb}
Connes-Thom isomorphism


\section{Cuntz semigroup}

nuclear dimension




\chapter{KK-theory}



\section{Kasparov picture}

\begin{itemize}
\item Kasparov stabilization theorem
\item Kasparov-Stinespring theorem
\item Kasparov-Voiculescu theorem
\item Kasparov-Weyl-von Neumann theorem
\item Kasparov technical theorem
\end{itemize}

\begin{prb}[Group equivariant correspondences]
Let $G$ be a locally compact group.
Let $(A,\alpha)$ and $(B,\beta)$ be $G$-C$^*$-algebras.
An \emph{equivariant correspondence} from $(A,\alpha)$ to $(B,\beta)$ is a correspondence $E$ from $A$ to $B$ together with a strongly continiuous map $u:G\to L(E)$ satisfying
\[u_s(a\xi b)=\alpha_s(a)u_s(\xi)\beta_s(b),\qquad\beta_s(\<\eta,\xi\>)=\<u_s\eta,u_s\xi\>,\]
for $a\in A$, $b\in B$, $s\in G$, and $\xi,\eta\in E$.
It generalizes covariant representations of $A$ and equivariant Hilbert modules over $B$.
The map $u$ is called a \emph{group action} on $E$ of $G$, and it is not in general $B$-linear unless the action $\beta$ on $B$ is trivial.
For an equivariant correspondence $(E,u)$ from $(A,\alpha)$ to $(B,\beta)$, the adjoint action $\Ad u$ acts continuously on $K(E)$ and strictly continuously on $B(E)$.
\begin{parts}
\item If $E$ is a super-correspondence from $A$ to $B$, then $(L^2(G)\otimes E,\lambda\otimes1)$ is naturally an equivariant super-correspondence from $(A,\alpha)$ to $(B,\beta)$.
If $E$ is faithful, non-degenerate, and full, then so is $L^2(G)\otimes E$, respectively.
\item interior tensor product and coalgebra structure from the group...
\end{parts}
\end{prb}
\begin{pf}
(a)
Define the super-correspondence $L^2(G)\otimes E$ from $A$ to $B$ with the natural grading, such that the left action of $A$, the right action of $B$, and the $B$-valued inner product is defined by
\[(a\xi b)(t):=\alpha_t^{-1}(a)\xi(t)\beta_t^{-1}(b),\qquad\<\eta,\xi\>:=\int_G\beta_t(\<\eta(t),\xi(t)\>)\,dt,\]
for $a\in A$, $b\in B$, $t\in G$, and $\xi,\eta\in C_c(G,E)$.
The group action on $L^2(G)\otimes E$ by $G$ is given by $\lambda\otimes1$.

We can check the above three structures preserve the grading and are all equivariant.

(Faithfulness)
Suppose $a\xi=0$ for all $\xi\in L^2(G)\otimes E$.
Then, for $f\otimes\xi_0\in C_c(G)\otimes E$,
\[0=(a(f\otimes\xi_0))(t)=f(t)\otimes(\alpha_t^{-1}(a)\xi_0)\]
implies $f(e)\otimes(a\xi_0)=0$ by putting $t=e$, so $a\xi_0=0$ and $a=0$.

(Fullness)
Because the a Hilbert module is full iff the right action is faithful, we can prove it in a similar way to faithfulness of the left action.

(Non-degeneracy)
If $e_i\in A$ is a quasi-central approximate unit such that $\alpha_t(e_i)-e_i\to0$ in $A$ compactly on $G$ (it can be shown whithout the condition that $A$ is $\sigma$-unital, Lemma 2.12 of Ozawa), then
\[(e_i\xi-\xi)(t)=(\alpha_t^{-1}(e_i)-1)\xi(t)=(\alpha_t^{-1}(e_i)-e_i)\xi(t)+(e_i-1)\xi(t)\]
\begin{align*}
|\xi-e_i\xi|^2
&=\int_G\beta_t(|((1-e_i)\xi)(t)|^2)\,dt\\
&=\int_G\beta_t(|(1-\alpha_t^{-1}(e_i))\xi(t)|^2)\,dt\\
&\le2\int_G\beta_t(|(1-e_i)\xi(t)|^2+|(e_i-\alpha_t^{-1}(e_i))\xi(t)|^2)\,dt\to0
\end{align*}
taking compact set outside which we have $\|\xi\|<\e$.

\end{pf}


\begin{prb}[Correspondences over commutative C$^*$-algebras]
Let $X$ be a locally compact Hausdorff space.
Let $A$ and $B$ be $C_0(X)$-C$^*$-algebras.

For equivariant versions, we do not require the compatibility of $G$ and $C_0(X)$ on $E$, which is satisfied automatically.
\begin{parts}
\item 
\item For a $C_0(X)$-C$^*$-algebra $A$, there exists a faithful non-degenerate correspondence $E$ from $A$ to some $C_0(X)$-W$^*$-algebra $B$.
\item tensor products of $G$-C$^*$-algebras
\end{parts}
\end{prb}
\begin{pf}


(b)
We will choose $B=C_0(X)^{**}$. ($C_0(X)^{**}$ is not a $C_0(X)$-algebra...)
Fix a state $\omega$ on $A$.
Since $C_0(X)^{**}\subset Z(A^{**})$, there is a conditional expectation $\f:A^{**}\to C_0(X)^{**}$, which factors through $\omega^{**}=\omega^{**}\f$ because $C_0(X)^{**}\subset Z(A^{**})$ is unital.
Since $\f$ is completely positive, the Stinespring construction on $A\odot C_0(X)$ gives rise to a C$^*$-correspondence $E_\omega$ from $A$ to $C_0(X)^{**}$.
Define $E:=\bigoplus_{\omega\in S(A)}E_\omega$.
If $a\in A$ acts trivially on $E$, which means $\f(a^*a)=0$ and $\omega(a^*a)=0$.
Thus $A$ acts failfully on $E$.

\end{pf}

\begin{prb}[Kasparov stabilization theorem]
Let $G$ be a locally compact group.
Let $(B,\beta)$ be a $G$-C$^*$-algebra.
Let $(E,u)$ be an equivariant Hilbert module over $(B,\beta)$.
Let $H_B:=\ell^2\otimes L^2(G)\otimes B$.
If $E$ is countably generated, then there is a equivariant $B$-linear isomteric isomorphism $E\to H_B\oplus E$.
\begin{parts}
\item non-equivaraint version.
\item equivariant version.
\end{parts}
\end{prb}
\begin{pf}
(a)
The Hilbert $B$-module $E$ is countably generated if and only if there is a dense range adjointable operator
\[\ell^2\otimes B\to E.\]


(b)
Let $H_E:=\ell^2\otimes L^2(G)\otimes E$.

We have
\begin{align*}
H_B
&=\ell^2\otimes L^2(G)\otimes B\\
&=\ell^2\otimes L^2(G)\otimes\ell^2\otimes B\\
&=\ell^2\otimes L^2(G)\otimes(E_0\oplus(\ell^2\otimes B))\\
&=(\ell^2\otimes L^2(G)\otimes E_0)\oplus(\ell^2\otimes L^2(G)\otimes\ell^2\otimes B))\\
&=(\ell^2\otimes L^2(G)\otimes E)\oplus H_B\\
&=H_E\oplus H_B,
\end{align*}
where all the identities mean equivariant isometric $B$-linear isomorphisms.

Since $G$ is compact, we have an equivariant linear isometry $\C\to L^2(G)$.
It gives rise to direct sums $L^2(G)=\C\oplus \C^\perp$, and we get $L^2(G)\otimes E=E\oplus E^\perp$ by tensoring, that is, $E$ is complemented Hilbert $B$-submodule of $L^2(G)$.
We have
\begin{align*}
E\oplus H_E
&=E\oplus(\ell^2\otimes L^2(G)\otimes E)\\
&=E\oplus(\ell^2\otimes(E\oplus E^\perp))\\
&=E\oplus(\ell^2\otimes E)\oplus(\ell^2\otimes E^\perp)\\
&=((\C\oplus\ell^2)\otimes E)\oplus(\ell^2\otimes E^\perp)\\
&=(\ell^2\otimes E)\oplus(\ell^2\otimes E^\perp)\\
&=\ell^2\otimes(E\oplus E^\perp)\\
&=\ell^2\otimes L^2(G)\otimes E\\
&=H_E.
\end{align*}
Therefore,
\[H_B=H_E\oplus H_B=E\oplus H_E\oplus H_B=E\oplus H_B.\]
\end{pf}

\begin{prb}[Quasi-central approximate units]
Let $A$ be a $\sigma$-unital C$^*$-algebra.
Let $Y$ be a locally compact $\sigma$-compact Hausdorff subset contained in a faithful representation $B(H)$ of $A$.
Then, there is an increasing sequential approximate unit $e_i$ for $A$ such that $[y,e_i]\to0$ in $A$ compactly on $Y$.
\end{prb}
\begin{pf}
Let $e_i$ be an approximate unit of $A$.
Take any compact $K\subset Y$.
Let $\Lambda$ be the algebraic convex closure of $e_i$.
Define a bounded linear operator
\[L:A\to C(K,A):a\mapsto(y\mapsto[y,a]).\]
Our goal is to show the closure $L\Lambda$ in $C(K,A)$ contains zero.
Suppose not so that there is $l\in C(K,A)^*$ such that
\[0<\inf_{v\in\Lambda}\Re l(Lv).\]
We claim that $Le_i\to0$ weakly in $C(K,A)$.
We can show that it converges in
\[\sigma(A\otimes C(K),A^*\odot\mathrm{span}\,\mathrm{PS}(C(K))).\]
To enhance the convergence, we need to introduce vector measures and require for an approximate unit to be a sequence for applying the bounded convergence theorem!!!!
I think we can show this using the measure topology (maybe).
\end{pf}

\begin{prb}[Kasparov technical theorem]
Let $G$ be a locally compact $\sigma$-compact group.
Let $J$ and $A_1$ be $\sigma$-unital $G$-C$^*$-algebras such that $A_1\subset M(J)$.
Suppose
\begin{enumerate}[(i)]
\item $\Delta$ is a norm separable subset of $M(J)$ such that $[\Delta,A_1]\subset A_1$,
\item $G$, a locally compact $\sigma$-compact group, acts on $A_1$ so that $GA_1\subset A_1$,
\item $A_2$ is a $\sigma$-unital graded C$^*$-subalgebra of $M(J)$ such that $A_1A_2\subset J$,
\item $\f$ is a bounded function $G\to M(J)$ such that $\f(G)A_1,A_1\f(G)\subset J$ and $g\mapsto\f(g)a,a\f(g)$ are norm continuous for every $a\in A_1+J$.
\end{enumerate}
Then, there is $M_2\in M(J)$ with $0\le M\le1$ such that $(1-M_2)A_1\subset J$ and
\begin{enumerate}[(i)]
\item $[\Delta,M_2]\subset J$,
\item $GM_2-M_2\subset J$,
\item $M_2A_2\subset J$,
\item $\f(G)M_2,M_2\f(G)\subset J$ and $g\mapsto\f(g)M_2,M_2\f(g)$ are norm continuous.
\end{enumerate}
\end{prb}
\begin{pf}




\end{pf}


If $G$ acts on a C$^*$-algebra $A$, then $\alpha_s^{**}:A^{**}\to A^{**}$ satisfies $\alpha_s^{**}L^\infty(\omega)=L^\infty(\alpha_s^*\omega)$.
We can construct a unitary $u_s:L^2(\omega)\to L^2(\alpha_s^*\omega)$?
Radon-Nikodym derivatives are mapped by this unitary?


\begin{prb}[Kasparov cycles]
Let $G$ be a locally compact $\sigma$-compact group and $X$ a locally compact $\sigma$-compact Hausdorff space.
Let $(A,\alpha)$ and $(B,\alpha)$ be $\sigma$-unital $G$-$C_0(X)$-C$^*$-algebras.
A \emph{Kasparov cycle} or \emph{Kasparov module} from $(A,\alpha)$ to $(B,\beta)$ is a countably generated super-correspondence $(E,u)$ from $(A,\alpha)$ to $(B,\beta)$ which is $G$-equivariant over $C_0(X)$, together with an odd adjointable operator $F\in B(E)$ such that
\begin{enumerate}[(i)]
\item $[F,A]\subset K(E)$ and $FA\subset B(E)$ is $G$-continuous,
\item $(F-F^*)A\cup(F^2-1)A\subset K(E)$ and $GF-F\subset K(E)$.
\end{enumerate}
The sets of unitary equivalence classes of Kasparov cycles and the unitary equivalence classes of degenerate Kasparov cycles are denoted by $E^G(A,B)$ and $D^G(A,B)$, where the $G$-$C_0(X)$-structures are usually omitted in notation.
\begin{parts}
\item $E^G(\C,\C)$
\item $E^G(\C,B)$,
\item $E^G(A,\C)$, Fredholm module
\end{parts}
\end{prb}


\begin{prb}[KK-groups]
Let $(E,F)$ be a Kasparov cycle from $(A,\alpha)$ to $(B\otimes C([0,1]),\beta\otimes\id)$.
Then, for each $t\in[0,1]$, we can restrict it to a Kasparov cycle $(E_t,F_t):=(E\otimes_{B\otimes C([0,1])}B,F\otimes1)$ from $(A,\alpha)$ to $(B,\beta)$ as follows.
First, we check that $E_t$ is a countably generated super-correspondence over $G$ and $C_0(X)$.
If we introduce the evaluation maps
\begin{gather*}
\mathrm{ev}_t:=\id\otimes\delta_t:B\otimes C([0,1])\to B,\\
\mathrm{ev}_t:E\to E_t:\xi(b\otimes1)\mapsto\xi b,\qquad\mathrm{ev}_t:B(E)\to B(E_t):T\mapsto T\otimes1,
\end{gather*}
which all commute the structures given by $G$ and $C_0(X)$, then the first defines the left action used in the interior tensor product $E_t=E\otimes_{B\otimes C([0,1])}B$, the second is well-defined because the right action is non-degenerate and
\[\|\xi\otimes b\|^2\le\cdots\le\|\xi(b\otimes1)\|^2,\]
and the third maps compact operators to compact operators.
Then, the two-sided actions and inner product is given by
\[a(\xi(t))b=(a\xi b)(t),\qquad\<\eta(t),\xi(t)\>=\<\eta,\xi\>(t),\qquad T(t)\xi(t)=(T\xi)(t),\]
hence $E_t$ is a super-correspondence from $A$ to $B$.
Second, we check that $F_t$ is a Fredholm operator.

\[[F_t,a],\quad (F_t-F_t^*)a,\quad (F_t^2-1)a\]
and group action continuity.

Then, $(E,F)$ is called a \emph{homotopy} between $(E_0,F_0)$ and $(E_1,F_1)$.


compact perturbation, 




The set of homotopy classes of Kasparov cycles is denoted by $KK^G(A,B)$, where the actions $\alpha$ and $\beta$ are usually omitted in notation.
The set theoretic issue does not occur because we only consider countably generated correspondences.

\begin{parts}
\item $KK^G(A,B)$ is an abelian group given by direct sum.
\item $KK^G$ is a homotopy invariant bivariant functor.
\item $KK^G$ preserves finite products. (infinite direct sum for the first argument)
\end{parts}
\end{prb}
\begin{pf}
(a)
well-definedness

associativity: clear

identity: clear

inverse: two homotopies; rotation from the sum with opposite to degenerate, trivial homotopy from degenerate to zero.

$C([0,1],B)\subset B\otimes C([0,1])$

$C([0,1],(E\oplus-E)\otimes_BB)\subset (E\oplus-E)\otimes_B(B\otimes C([0,1]))$

$C([0,1],B((E\oplus-E)\otimes_BB))\subset B((E\oplus-E)\otimes_B(B\otimes C([0,1])))$

We need to check the rotational matrix satisfies the Fredholm conditions

commutativity: clear

(b)

(c)

Suppose $\f_0,\f_1:A\rightrightarrows A'$ are homotopic.
We calim $\f_0^*,\f_1^*:KK^G(A',B)\rightrightarrows KK^G(A,B)$ are equal.

Suppose $\psi_0,\psi_1: B\rightrightarrows B'$ are homotopic.
We will show $\psi_{0*},\psi_{1*}:KK^G(A,B)\rightrightarrows KK^G(A,B')$.

\end{pf}




\begin{prb}[Kasparov product]
Let $E_1$ be a Hilbert module over $B$, and $E_2$ be a super-correspondence from $B$ to $C$.
Let $E_{12}:=E_1\otimes_B E_2$.
For $F_2\in B(E_2)$, we say $F_{12}\in B(E_{12})$ is a \emph{$F_2$-connection} for $E_1$ if
\[F_{12}T_{\xi_1}-T_{\xi_1}F_2\in K(E_2\oplus E_{12}),\quad
T_{\xi_1}^*F_{12}-F_2T_{\xi_1}^*\in K(E_{12}\oplus E_2),\qquad\xi_1\in E_1.\]
(How about $G$-$C_0(X)$-equivariant version?)

Let $(A,\alpha)$, $(B,\beta)$, and $(C,\gamma)$ be $G$-C$^*$-algebras.
Let $(E_1,F_1)$ and $(E_2,F_2)$ be Kasparov cycles from $(A,\alpha)$ to $(B,\beta)$ and from $(B,\beta)$ to $(C,\gamma)$, and let $E_{12}:=E_1\otimes_BE_2$.
We say a Kasparov cycle $(E_{12},F_{12})$ is a \emph{Kasparov product} if $F_{12}$ is a $F_2$-connection for $E_1$ and $a^*[1\otimes F_2,F_{12}]a\ge0$ in $Q(E_{12})$ for all $a\in A$.
\begin{parts}
\item 
\end{parts}
\end{prb}




(half and long exactness?)
(extension of k theory and k homology?)
(direct sum, pullback, interior tensor product, pushout, exterior tensor product?)

cap product
ring structure, $R(G)$-module structures




inverses
equivariant imprimitivity bimodules



\begin{prb}[Examples of Kasparov cycles]

For a complete Riemannian manifold $M$, $(L^2(\Lambda T^*M),m,D(1+D^*D)^{-\frac12})$, where $D:=d+d^*$ is the Hodge-Dirac operator and $D^*D$ is the Laplace-de Rham operator, is a Kasparov module from $C_0(M)$ to $\C$.
\end{prb}

\section{Extension theory}

K-homology: dual algebras, extension theory.



\begin{prb}[Weyl-von Neumann theorem]
Let $A$ be a C$^*$-algebra.
We say $a,b\in A$ are called \emph{approximately unitarily equivalent}, denoted $a\sim_ab$, if $\Ad U(A)(a)$ and $\Ad U(A)(b)$ have same closures.



$\pi(U(H))\subset U(Q(H))$ is proper.

essentially unitarily equivalent: same orbit in $Q(H)$ by $\pi(U(H))$.

If same spectrum in $Q(H)$, then they are essentially unitarily equivalent.
We can prove this by the Weyl-von Neumann theorem.

Weyl-von Neumann: every bounded self-adjoint operator on a separable Hilbert space is an arbitrarily small compact perturbation of a diagonal operator($\sigma=\sigma_p$).
\end{prb}



\section{Cuntz-Thomsen picture}


stable uniqueness theorem(Lin or Dadarlat-Eilers)


\chapter{}








\part{Classification}
\chapter{Simple nuclear algebras}


\section{AF-algebras}

Glimm's classification of UHF algebras
Bratteli diagram
Elliott's intertwining argument

Separable AF-algebras are classified by pointed ordered $K_0$.


\section{Kirchberg-Phillips theorem}

\section{Classifiability}
Jiang-Su stability
Universal coefficient theorem

Toms-Winter conjecture
strongly self-absorbing
nuclear dimension




successful in Kirchberg algebras


https://arxiv.org/pdf/2307.06480.pdf

Elliott classification problem
Kirchberg-Phillipes theorem

operator K-theory and its pairing with traces

$\cZ$-stability, Rosenberg-Schochet universal coefficient theorem

Connes-Haagerup classification of injective factors

Kirchberg: unital simple separable $\cZ$-stable algebra is either purely infinte or stably finite.
Haagerup, Blackadar, Handelman: unital simple stably finite algebra has a trace.

Glimm: uniformly hyperfinite algebras
Murray-von Neumann: hyperfinite II$_1$ factors




\section{Inclusions}










\chapter{Continuous fields}


\section{Fell bundles}

\begin{prb}[Banach bundles]
A \emph{Banach bundle}, introduced by Fell, which is possibly not locally trivial, is a continuous open surjection $\pi:E\to X$ between topological spaces together with Banach space structure on each fiber $\pi^{-1}(x)$ such that:
\begin{enumerate}[(i)]
\item the addition $\{(e,e'):\pi(e)=\pi(e')\}\subset E\times E\to E:(e,e')\mapsto e+e'$ is continuous,
\item the scalar multiplication $\C\times E\to E:(\lambda,e)\mapsto\lambda e$ is continuous,
\item the norm $E\to\R_{\ge0}:e\mapsto\|e\|$ is continuous,
\item the family of subsets
\[\{e\in B:\pi(e)\in U,\ \|e\|<r\}_{U\in N(x),r>0}\]
forms a neighborhood basis of $0\in\pi^{-1}(x)$ in $E$.
\end{enumerate}
The forth condition is equivalent to that if $\|e_i\|\to0$ and $\pi(e_i)\to x$ then $e_i\to 0_x\in\pi^{-1}(x)$.
\begin{parts}
\item For a Banach bundle $E\to X$, if $X$ is locally compact Hausdorff and every fiber $E_x$ shares a same finite dimension, then the bundle is locally trivial.
\end{parts}
\end{prb}


\begin{prb}[Continuous fields of Banach spaces]


\end{prb}




span of $a[D,b]$
completion of the span of the gradient of test functions,
dual of Borel time-dependent vector field,

For discussion of tangent vectors:
sufficiently many absolutely continuous curves?

compact metric space

\begin{prb}[Hilbert bundles]
A \emph{Hilbert bundle} is a Banach bundle whose norm function satisfies the parallelogram law.

\begin{parts}
\item On a compact $X$, there is an equivalence between the category of Hilbert $C(X)$-modules and the category of Hilbert bundles over $X$.
\item On a compact $X$, there is an equivalence between the category of algebraically finitely generated Hilbert $C(X)$-modules and the category of classical locally trivial finite-rank complex vector bundle over $X$.
It is due to that finitely generatedness implies the projectivity and the Serre-Swan theorem.
\end{parts}
\end{prb}



\section{Dixmier-Douady theory}


Fell's condition

A C$^*$-algebra $A$ is called \emph{continuous trace} if the set of all $a\in\cA$ such that $\hat A\to\R_{\ge0}:\pi\mapsto\tr(\pi(a^*a))$ is continuous is dense in $A$.



Dadarlat-Pennig theory


Coactions and Fell bundles



\section{C$^*$-dynamics}

Izumi-Matui
Rokhlin property
Evans-Kishimoto intertwining argument
dynamical Kirchberg-Phillips
Tikusis-White-Winter




\end{document}