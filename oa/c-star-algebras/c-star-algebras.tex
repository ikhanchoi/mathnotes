\documentclass{../../large}
\usepackage{../../ikhanchoi}

\newcommand{\Prim}{\operatorname{Prim}}



\begin{document}
\title{C$^*$-Algebras}
\author{Ikhan Choi}
\maketitle
\tableofcontents

\part{}



\chapter{Hilbert modules}

\section{Hilbert modules}

\begin{prb}[Banach modules]
Recall that the category of Banach spaces with contractive linear operators as its morphisms is a bicomplete symmetric monoidal category, with respect to the projective tensor product.
A Banach algebra is an internal semi-group in this category.
A \emph{Banach module} over a Banach algebra $A$ is a Banach space $E$ equipped with a contractive linear operator $A\mathrel{\hat\otimes_\pi} E\to E$.
\begin{parts}
\item (Cohen factorization theorem) If $A$ has a bounded left approximate unit, then $AE$ is closed in $E$.
\end{parts}
\end{prb}
\begin{pf}
It is clear if $A$ is unital, so assume $A$ is not unital.
Suppose $\xi\in\bar{AE}$.
We will construct a sequence $a_n\in\tilde A^\times$ of invertible elements of the unitization $\tilde A:=A\oplus_1\C$ with $\ell^1$-norm, which is not equal to the C$^*$-norm even if $A$ is a C$^*$-algebra, such that $a_n\to a\in A$ in $\tilde A$ and $a_n^{-1}\xi\to\eta$ in $E$.
If such a sequence exists, then since $a_n$ is bounded by its norm convergence, we can deduce $\xi=a\eta$ by
\[\|\xi-a\eta\|\le\|a_n\|\|a_n^{-1}\xi-\eta\|+\|a_n-a\|\|\eta\|\to0,\qquad n\to\infty.\]

Take $0<r\le4^{-1}(1+\sup_i\|e_i\|)^{-1}$, where $e_i$ is a bounded approximate unit of $A$.
Inductively define a sequence $a_n\in\tilde A$ such that
\[a_0:=1,\qquad a_{n+1}:=a_n-r(1-r)^n(1-e_n),\qquad n\ge0,\]
where $e_n\in A$ is also inductively chosen in the net $e_i$ such that
\[\|(1-e_n)4r((1-r)^n-a_n)\|\|a_n^{-1}\|\le1,\qquad4r(1-r)^n\|a_n^{-1}\|\|\eta_n\|\|(1-e_n)a_n^{-1}b_n\|<2^{-n},\]
and $b_n\in A$ and $\eta_n\in E$ are chosen such that
\[(1-r)^n\|a_n^{-1}\|^2\|\xi-b_n\eta_n\|<2^{-n}.\]

To verify that the choices of $e_n$ and $b_n\eta_n$ are possible, we need to show $a_n\in\tilde A^\times$, $(1-r)^n-a_n\in A$, and $a_n^{-1}b_n\in A$ for each step on $n$.
The third one is clear after shown that $a_n$ is invertible.
The first two can be checked inductively as follows.
First, observe that $a_0\in\tilde A^\times$ and $(1-r)^0a_0\in A$.
For $n\ge1$, the invertibility of $a_n$ follows from that $a_{n-1}$ is invertible and
\begin{align*}
\|1-a_na_{n-1}^{-1}\|
&=\|(a_{n-1}-a_n)a_{n-1}^{-1}\|\\
&=\|(1-e_{n-1})r(1-r)^{n-1}a_{n-1}^{-1}\|\\
&\le \|(1-e_{n-1})r((1-r)^{n-1}-a_{n-1})\|\|a_{n-1}^{-1}\|+\|(1-e_{n-1})r\|\\
&\le4^{-1}+4^{-1}=2^{-1}
\end{align*}
with a bound
\[\|a_n^{-1}\|\le\|a_{n-1}^{-1}\|\|a_{n-1}a_n^{-1}\|\le\|a_{n-1}^{-1}\|(1-2^{-1})^{-1}=2\|a_{n-1}^{-1}\|,\]
and we can also easily see that $(1-r)^{n-1}-a_{n-1}\in A$ and $e_{n-1}\in A$ imply that
\begin{align*}
(1-r)^n-a_n
&=(1-r)^n+r(1-r)^{n-1}(1-e_{n-1})-a_{n-1}\\
&=(1-r)^{n-1}-r(1-r)^{n-1}e_{n-1}-a_{n-1}
\end{align*}
belongs to $A$.
Therefore, the sequence $a_n$ is well-defined.

Then, $a_n$ converges to an element of $A$ by
\[a_n=a_0+\sum_{k=0}^{n-1}(a_{k+1}-a_k)=1+\sum_{k=0}^{n-1}r(1-r)^k(e_k-1)\to\sum_{k=0}^\infty r(1-r)^ke_k,\qquad n\to\infty.\]
We can also check that the sequence $a_n^{-1}\xi$ is Cauchy in $E$ because
\begin{align*}
\|(a_{n+1}^{-1}-a_n^{-1})\xi\|
&=\|a_{n+1}^{-1}(a_n-a_{n+1})a_n^{-1}\xi\|
=r(1-r)^n\|a_{n+1}^{-1}(1-e_n)a_n^{-1}\xi\|\\
&\le 2r(1-r)^n\|a_n^{-1}\|\|(1-e_n)a_n^{-1}\xi\|
\end{align*}
and
\begin{align*}
\|(1-e_n)a_n^{-1}\xi\|
&\le\|(1-e_n)a_n^{-1}(\xi-b_n\eta_n)\|+\|(1-e_n)a_n^{-1}b_n\eta_n\|\\
&\le(4r)^{-1}\|a_n^{-1}\|\|\xi-b_n\eta_n\|+\|(1-e_n)a_n^{-1}b_n\|\|\eta_n\|
\end{align*}
imply $\|(a_{n+1}^{-1}-a_n^{-1})\xi\|<2^{-n}$ for all $n\ge0$.
This completes the proof.
\end{pf}

\begin{prb}[Finsler modules]
Let $E$ be a right module over a C$^*$-algebra $B$.
A \emph{right norm} on $E$ is a map $|\cdot|:E\to B^+$ such that
\begin{enumerate}[(i)]
\item $|\xi|=0$ if and only if $\xi=0$,
\item $\||\xi+\eta|\|\|\le\||\xi|\|+\||\eta|\|$, for $\xi,\eta\in E$,
\item $|\xi b|^2=b^*|\xi|^2b$, for $\xi\in E$ and $b\in B$.
\end{enumerate}
There exists a unique Finsler structure on a Banach module if it exists.
\end{prb}


\begin{prb}[Hilbert modules]
Let $E$ be a right module over a C$^*$-algebra $B$.
A \emph{right inner product} on $E$ is a complex sesquilinear map $\<-,-\>:E\times E\to B$ such that
\begin{enumerate}[(i)]
\item we always have $\<\xi,\xi\>\ge0$, and $\<\xi,\xi\>=0$ if and only if $\xi=0$, for $\xi\in E$,
\item $\<\eta,\xi b\>=\<\eta,\xi\>b$, for $\xi,\eta\in E$ and $b\in B$,
\item $\<\eta,\xi\>^*=\<\xi,\eta\>$, for $\xi,\eta\in E$.
\end{enumerate}
The map $\<-,-\>$ is called the \emph{$B$-valued right inner product}.
It is a non-commutative analogue of Hermitian bundles.
Even though the complex scalars act on $E$ from right in the rigorous sense, we will frequently write the scalar multiplication at left.
When the ideal $\<E,E\>$ of $B$ is dense in $B$, we say $E$ is \emph{full}.
On a Banach module $E$ over $B$, the right inner product is unique if it exists.
\begin{parts}
\item The right action by $b$ is bounded and the operator norm is coincides with $B$, but does not preserve the involution and is not adjointable in general.
\item The right action is always non-degenerate, but is faithful if and only if $E$ is full.
\item Examples: $B$ itself, $B^n$, $\ell^2(\N,B)$, etc.
\item direct sum, tensor product, localization
\end{parts}
\end{prb}
\begin{pf}

(b)
The non-degeneracy of the right action is because $\xi e_i\to\xi$ in $E$ for each $\xi\in E$, where $e_i$ is an approximate unit $e_i$ of $\<E,E\>$.
\end{pf}

\begin{prb}[Adjointable and compact operators]
Let $E$ and $F$ be Hilbert $B$-modules over a C$^*$-algebra $B$.
An operator $T:E\to F$ is called an \emph{adjointable operator} if there is an operator $T^*:F\to E$ such that $\<\eta,T\xi\>=\<T^*\eta,\xi\>$ for all $\xi\in F$ and $\eta\in E$, and called \emph{compact} if it is a norm limit of the linear combination of adjointable operators of the form $\theta_{\xi,\eta}:E\to F$ with $\xi\in F$ and $\eta\in E$, where $\theta_{\xi,\eta}:=\xi\<\eta,-\>$, which has an adjoint $\theta_{\eta,\xi}$
The Banach spaces of all adjointable and compact operators $E\to F$ are denoted by $B(E,F)$ and $K(E,F)$ respectively, and these will not be used in the sense of Banach spaces.
\begin{parts}
\item An adjointable operator is a bounded $B$-module map.
\item $K(E)$ is a closed essential ideal of a C$^*$-algebra $B(E)$.
\item
\end{parts}
\end{prb}
\begin{pf}
The $B$-linearity is clear.

Let $E$ and $F$ be right Hilbert $A$-modules and let $x\in B(E,F)$ be an $A$-adjointable operator.
Consider a conjugate-linear map $F\to L(F,A):\eta\mapsto\<-,\eta\>$, and we can check it is an isometry by considering $\eta(|\eta|+\e)^{-1}$.
For each $\eta'\in F$, we have
\[|\<\eta',x\xi\>|=|\<x^*\eta',\xi\>|\le\|x^*\eta'\|\]
whenever $\xi\in E$ satisfies $\|\xi\|\le1$, so the set $\{x\xi:\xi\in E,\ \|\xi\|\le1\}\subset F$ is bounded in $L(F,A)$ by the uniform boundedness principle, and it implies the boundedness in $F$.

\end{pf}

The closed range condition becomes more differentiated for an adjointable operator.
Consider a C$^*$-algebra $B:=C([-1,1])$ and its right Hilbert modules
\[E_1:=C([-1,0]\cup[\tfrac12,1]),\qquad E_2:=C([-1,-\tfrac12]\cup[0,1]),\qquad E_3:=C([-1,1]).\]
We can consider a self-adjoint(really?) adjointable operator $T_i\in B(E_i)$ on each $E_i$ by the multiplication of the ramp function $t\mapsto\max\{0,t\}$.
Then,
\begin{gather*}
\ker T_1=C([-1,0]),\qquad\overline\ran\,T_1=C([\tfrac12,1])\\
\ker T_2=C([-1,-\tfrac12]),\qquad\overline\ran\,T_2=C_0((0,1])\\
\ker T_3=C_0([-1,0)),\qquad\overline\ran\,T_3=C_0((0,1]).
\end{gather*}
In general for an adjointable operator $T$, the closedness of $\ran T$ implies the complementedness of $\ran T$ and $\ran T^*$, and the complementedness of $\overline\ran\,T$ implies the completementedness of $\ker T^*$.

The completementedness is corresponded to continuity, and the closedness is corresponded to upper semi-continuity.

$(\ran T)^{\perp\perp}=\overline\ran\,T$ (really?) may not be complemented.


When does an adjointable operator have complemented range?


\begin{prb}[Complemented submodules]
\end{prb}

\begin{prb}[Weak topologies for Hilbert modules]
Let $E$ and $F$ be Hilbert $B$-modules for a C$^*$-algebra $B$.
The \emph{strict topology} refers to the strong$^*$ operator topology of $B(E)$.



On the trivial Hilbert $B$-module $B$, $b_i\to0$ strictly iff $b_i,b_i^*\to0$ weakly.
If $B$ is unital, the strict topology on $B$ and the norm topology coincide.
An adjointable operator is weakly continuous.





\end{prb}



On Hilbert modules:
\begin{itemize}
\item polarization identity? OK,
\[\<\eta,\xi\>=\frac14\sum_{k=0}^3i^k\<\xi+i^k\eta,\xi+i^k\eta\>,\qquad \xi,\eta\in E.\]
\item bounded sesquilinear form?
\item Riesz representation? OK for adjointable operator $l:E\to B$, there is $\eta:=l^*1$ if $B$ is unital.
\item alaoglu? the strict topology is complete. for compactness criterion, we need more. functional calculus for strictly compact functions?
\item uniform boundedness principle?
\item 
\item unbounded adjointable operators and spectral theory?
\item polar decomposition? especially for unbounded adjointable operators?
\end{itemize}
















\begin{prb}[Multiplier algebra]
Four descriptions for a multiplier algebra:
double centralizers vs essential ideal vs multipliers in von Neumann algebra vs Hilbert module

1.
Let $B$ be a C$^*$-algebra.
A \emph{double centralizer} of $B$ is a pair $(L,R)$ of bounded linear maps on $A$ such that $aL(b)=R(a)b$ for all $a,b\in B$.
The \emph{multiplier algebra} $M(B)$ of $B$ is defined to be the set of all double centralizers of $B$.
There is another characterization of $M(B)$ as the set of adjointable operators to itself.
Even if the notation $B(B)$ may cause confusion, we can write $M(B)$ to avoid this.

2.
An ideal $I$ of $B$ is called an \emph{essential} if it is a full Hilbert $B$-submodule of $B$, or equivalently $B$ faithfully acts on $I$.



Every C$^*$-algebra $A$ is a correspondence over $M(A)$.

There is a characterization in an inclusion into a von Neumann algebra.

relations between Hilbert $B(H)$-modules and representations?


\begin{parts}
\item $\|\pi(a-e_ia)\xi\|^2$
\item If $u_i$ are unitary, the convergences in the strict topology and the weak topology(how to define this?) coincide.
\item If $a_i$ are increasing, the convergences in the strict topology and the weak topology(how to define this?) coincide.
\item $M(K(E))\cong B(E)$.
\item $M(C_0(\Omega))\cong C_b(\Omega)$.
\end{parts}
\end{prb}
\begin{pf}
First we claim $C_0(\Omega)$ is an essential ideal of $C_b(\Omega)$.
Since $C_b(\Omega)\cong C(\beta\Omega)$, and since closed ideals of $C(\beta\Omega)$ are corresponded to open subsets of $\beta\Omega$, $C_0(\Omega)\cap J$ is not trivial for every closed ideal $J$ of $C_b(\Omega)$.

Now we have an injective $*$-homomorphism $C_b(\Omega)\to M(C_0(\Omega))$, for which we want to show the surjectivity.
Let $g\in M(C_0(\Omega))_+$.


\end{pf}

polar decomposition in the multiplier algebra? yes if ranges of $x$ and $x^*$ are complemented.

If $v\in M(A)$ satisfies $v^*v\in A$, then $v\in A$ since
\begin{align*}
\|v-v(v^*v)^{\frac1n}\|^2
&=\|(v^*-(v^*v)^{\frac1n}v^*)(v-v(v^*v)^{\frac1n})\|\\
&=\|v^*v-2(v^*v)^{\frac1n+1}+(v^*v)^{\frac2n+1}\|=\|v^*v(1-(v^*v)^{\frac1n})^2\|\to0,\qquad n\to\infty,
\end{align*}
assuming $\|v\|\le1$.




\begin{prb}
Let $A$ and $B$ be C$^*$-algebras, and $E$ be a right Hilbert $A$-module.
A linear map $B\to B(E)$ is called \emph{strict} if it is strictly continuous on bounded parts.
\begin{parts}
\item $K(E)$ is strictly dense in $B(E)$ by approximate unit.
\item Let $\f:B\to B(E)$ be a $*$-homomorphism, that is, $E$ be a right Hilbert bimodule over $(B,A)$. If $BE$ is complemented in $E$, then $\f$ is strictly continuously extended to $\f:M(B)\to B(E)$.
\item The strict topology is complete.
\item $M(K(E))=B(E)$, and the strict topologies with respect to $K(E)$ and $E$ coincide on bounded parts.
\item $\cK\otimes B=K(\ell^2\otimes B)$.
\end{parts}
\end{prb}

\begin{pf}
(a)
Clear.

(b)
Note that $BE$ is closed by the Cohen factorization.
Suppose $BE=pE$ for a projection $p\in B(E)$.
Then,
\[\|[p,b]\xi\|^2=\|b(1-p)\xi\|^2=\<(1-p)\xi,b^*b(1-p)\xi\>=\<(1-p)\xi,pb^*b(1-p)\xi\>=0,\qquad\xi\in E\]
implies $[b,p]=0$ for all $b\in B$.
If $b_i$ is a net in $M(B)$ such that $b_i\to0$ strictly in $M(B)$, then for each $\xi\in E$ we have $b\eta\in BE$ such that $p\xi=b\eta$, the limit for $i$
\[\|b_i\xi\|=\|pb_i\xi\|=\|b_ip\xi\|=\|b_ib\eta\|\le\|b_ib\|\|\eta\|\to0\]
implies that $b_i\to0$ strictly in $B(E)$.
(if $\overline{BE}$ is complemented, then we only have that $M(B)\to B(E)$ is strictly continuous on bounded parts)

Conversely, take a projection $p:=\f(1)\in B(E)$


(c)
If $x_i$ is a net in $B(E)$ which is strictly Cauchy.
Then, $x_i\xi$ and $x_i^*\xi$ converges in $E$.
If we define $x$ such that $x\xi:=\lim_ix_i\xi$ for $\xi\in E$, then the convergence of $x_i^*\xi$ implies that $x$ admits an adjoint $x^*$ given by $x^*\eta:=\lim_ix_i^*\eta$ for $\eta\in E$.

(d)
Since $\overline{K(E)E}=E$ is complemented in $E$, by the part (b), we have a $*$-homomorphism $M(K(E))\to B(E)$ that is strictly continuous on bounded parts.
In fact, $K(E)E$ is itself complemented so that the $*$-homomorphism is strictly continuous on the whole space, but it is not important in here.
To show the bijectivity, we will show the two strict topologies coincide on the unit ball of a common strictly dense $*$-subalgebra $K(E)$.
Suppose $k_i$ is a bounded net in $K(E)$ such that $k_i\to0$ strictly in $B(E)$.
For each $k\in K(E)$, by approximating $k$ with finite-rank operators, we can see $k_i\to0$ strictly in $M(K(E))$.
The converse is clear in the sense that the topology of $M(K(E))$ is clearly stronger than of $B(E)$.


(e)
The nuclearity of $\cK$.
\end{pf}




\begin{prb}
non-degenerate means that it is strict and extended to a unital.

every completely positive map $B\to B(H)$ is strict.
\end{prb}




\section{Morita equivalence}



Induced representations?




\section{Kasparov-Stinespring}
We define normality as the strict continuity on bounded parts?

We first consider normal completely positive maps $\f:A\to M(B)$.
By the separation and completion of $A\odot B$, we obtain a right Hilbert bimodule $E$ over $(A,B)$ together with the cocyclic data $v\in B(E,B)$ such that $\f=\Ad v$ on $B$.
Then, we have a normal left $A$-module map of strictly dense range $\Lambda_\f:A\to B(B,E)$ defined such that $\Lambda_\f(a):=av^*$.

(We did not check the normality and strict density of $\Lambda_\f$)

stabilization?






Contrary to von Neumann algebras in which polar decompositions are possible, we do not have the equivalence between the left ideals $\fN$ and hereditary $*$-subalgebras $\fM$.


\begin{prb}[C$^*$-valued weight]
Let $A$ and $B$ be C$^*$-algebras.
Recall that a subalgebra of $A$ is called \emph{hereditary} if its positive cone is hereditary in the positive cone of $A$.
A \emph{C$^*$-valued weight} is a partially defined completely positive map $\f:\dom\f\subset A\to M(B)$ on a hereditary $*$-subalgebra of $A$.
We use the notation $\fM_\f:=\dom\f$ and $\fN_\f:=\{x\in M:x^*x\in\fM_\f\}$, and $\fN_\f$ is then a left ideal of $M$.


A \emph{semi-cyclic representation} of $A$ over $B$ is a right Hilbert bimodule $E$ over $(A,B)$ together with a partially defined linear operator $\Lambda:\dom\Lambda\subset A\to B(B,E)$ such that 
\begin{enumerate}[(i)]
\item $\dom\Lambda$ is a left ideal of $A$ and $\Lambda$ is left $A$-linear,
\item $\ran\Lambda$ is strictly dense in $B(B,E)$.
\end{enumerate}
\emph{associated} if $\dom\Lambda=\fN_\f$ and $\f(a^*a)=|\Lambda(a)|^2\in M(B)$ for $a\in\dom\fN_\f$.

\begin{parts}
\item always non-degenerate so that $M(A)$ acts on $E$.
\item If $a\in\fN_\f$ satisfies $\f(a^*a)\in B$, then $\Lambda_\f(a)\in K(B,E)=E$. In particular, if $\f:\fM_\f\to B\subset M(B)$, then $\Lambda_\f:\fN_\f\to K(B,E)=E\subset B(B,E)$.
\item Commutant Radon-Nikodym: Let $\rho:A\to B(E)$ be an everywhere defined completely positivie map.
\end{parts}

\end{prb}

\begin{prb}
Lower semi-continuity of partially defined completely positive map on a hereditary $*$-subalgebra.
closedness
\end{prb}






\chapter{Operator spaces}
\section{Operator spaces}

\begin{prb}
Let $E$ be an operator space.
\begin{parts}
\item If $\f:E\to A$ is a completely bounded linear map to a C$^*$-algebra, then $\|\f\|_{cb}=\|\f\|$.
\end{parts}
\end{prb}

\begin{prb}[Ruan characterization]
Let $E$ be an operator space.
\begin{parts}
\item Let $(E,E^*)$ be a real dual pair. Let $P$ be a convex cone in $E$ and $K^*$ be a compact convex subset of $E^*$.
If for every $x\in P$ there is $x^*\in K^*$ such that $\<x,x^*\>\ge0$, then there is $x^*\in K^*$ such that for every $x\in P$ we have $\<x,x^*\>\ge0$.
\item For $\xi^*\in(\cK\otimes E)^*$ with $\|\xi^*\|=1$, there exist states $\rho_1$ and $\rho_2$ of $\cK$ such that
\[|\xi^*(a_1^*\eta a_2)|\le\rho_1(a_1^*a_1)^{\frac12}\|\eta\|\rho_2(a_2^*a_2)^{\frac12},\qquad a_1,a_2\in\cK,\ \eta\in \cK\otimes E.\]
\item For $\xi\in\cK\otimes E$, there is a complete contraction $\f:E\to B(H)$ such that $\|\f(\xi)\|=\|\xi\|$.
\end{parts}
\end{prb}
\begin{pf}
(a)
Let $K^*_x:=\{x^*\in K^*:\<x,x^*\>\ge0\}$ for $x\in P$, which is non-empty by assumption.
It suffices to show the intersection is non-empty.
To this end, we prove that they have the finite intersection property.
Let $(x_j)_{j=1}^n$ be a finite sequence in $P$ and assume $\bigcap_jK^*_{x_j}=\varnothing$.
It means that
\[\{(\<x_1,x^*\>,\cdots,\<x_n,x^*\>):x^*\in K\}\cap(\R^n)^+=\varnothing.\]
Find a positive linear functional $l$ on $\R^n$ separating the above two disjoint sets, and represent $l$ as a row vector $(l_1,\cdots,l_n)$.
Then, $x:=\sum_{j=1}^nl_jx_j$ is contained in $P$ and satisfies $K^*_x=\varnothing$ by
\[\<x,x^*\>=\sum_{j=1}^nl_j\<x_j,x^*\>=l(\<x_1,x^*\>,\cdots,\<x_n,x^*\>)<0\qquad x^*\in K^*,\]
which is a contradiction.
Therefore, the finite intersection property follows.


(b)
Rotating the scalar phase of $a_1$ and adjusting the positive scalar multiples of $a_1$ and $a_2$ to have $\rho_1(a_1^*a_1)=\rho_2(a_2^*a_2)$, it suffices to prove the existence of states $\rho_1$ and $\rho_2$ of $\cK$ satisfying
\[2\Re\xi^*(a_1^*\eta a_2)\le\rho_1(a_1^*a_1)+\rho_2(a_2^*a_2),\qquad a_1,a_2\in\cK,\ \eta\in E\otimes\cK,\ \|\eta\|=1.\]
Let $K^*:=S(\tilde\cK)\times S(\tilde\cK)\subset(\tilde\cK\oplus\tilde\cK)^*$ be a weakly$^*$ compact set by the unitality of $\tilde\cK$, and let
\[P:=\{(a_1^*a_1-\Re\xi^*(a_1^*\eta a_2),a_2^*a_2-\Re\xi^*(a_1^*\eta a_2)):a_1,a_2\in\cK,\ \eta\in\cK\otimes E,\ \|\eta\|=1\}\]
be a convex cone in $\tilde\cK\oplus\tilde\cK$, which follows from $\|\eta\oplus\eta'\|=\max\{\|\eta\|,\|\eta'\|\}$ for $\eta,\eta'\in\cK\otimes E$.
For each $a_1,a_2\in\cK$ and $\eta\in\cK\otimes E$ with $\|\eta\|=1$, if we select $(\rho_1,\rho_2)\in K^*$ such that $\rho_1(a_1^*a_1)=\|a_1^*a_1\|$ and $\rho_2(a_2^*a_2)=\|a_2^*a_2\|$, then the inequality
\[2\Re\xi^*(a_1^*\eta a_2)\le2\|a_1\|\|a_2\|\le\|a_1\|^2+\|a_2\|^2=\rho_1(a_1^*a_1)+\rho_2(a_2^*a_2)\]
implies the assumption of the part (a), hence there is $(\rho_1,\rho_2)\in K^*$ satisfying the desired inequality.

Finally we need to show the states $\rho_1$ and $\rho_2$ of $\tilde\cK$ are indeed states of $\cK$.
If we choose an approximate $e_i$ of $\cK$ and a vector $\xi\in\cK\otimes E$ such that $|\xi^*(\xi)|>1-\e$ with $\|\xi\|=1$, then our inequality
\[|\xi^*(e_i^{\frac12}\xi e_i^{\frac12})|\le\rho_1(e_i)^{\frac12}\rho_2(e_i)^{\frac12}\le1\]
proves the limits $\rho_1(e_i)\to1$ and $\rho_2(e_i)\to1$ by $\e\to0$, so the states $\rho_1$ and $\rho_2$ are reduced to states of $\cK$.

(c)
Let $\xi^*\in(\cK\otimes E)^*$ with $\|\xi^*\|\le1$.
Choose states $\rho_1$ and $\rho_2$ of $\cK$ using the part (b) such that
\[|\xi^*(a_1^*\eta a_2)|\le\rho_1(a_1^*a_1)^{\frac12}\|\eta\|\rho_2(a_2^*a_2)^{\frac12},\qquad a_1,a_2\in\cK,\ \eta\in\cK\otimes E.\]
Let $\pi_1:\cK\to B(H_1)$ and $\pi_2:\cK\to B(H_2)$ be the cyclic representations associated to $\rho_1$ and $\rho_2$ together with the canonical cyclic vectors $\Omega_1$ and $\Omega_2$ respectively.
Then, the inequality is written as
\[|\xi^*(a_1^*\eta a_2)|\le\|\pi_1(a_1)\Omega_1\|\|\eta\|\|\|\pi_2(a_2)\Omega_2\|,\qquad a_1,a_2\in\cK,\ \eta\in\cK\otimes E.\]
This implies, considering a bounded sesquilinear form, that every element $\xi^*\in(\cK\otimes E)^*$ with $\|\xi^*\|\le1$ defines a contraction $\psi:\cK\otimes E\to B(H_2,H_1)$ such that
\[\xi^*(a_1^*\eta a_2)=\<\psi(\eta)\pi_2(a_2)\Omega_2,\pi_1(a_1)\Omega_1\>,\qquad a_1,a_2\in\cK,\ \eta\in\cK\otimes E.\]



We may assume $\|\xi\|=1$.
By the Hahn-Banach extension, there is $\xi^*\in(\cK\otimes E)^*$ such that $\|\xi^*\|=1$ and $\xi^*(\xi)=1$.

If we induce $\f:E\to B(H)$ from $\psi$ via the complete isometric inclusions $E\subset\cK\otimes E$ and $B(H_2,H_1)\subset B(H)$, where $H:=H_1\oplus H_2$, then it is clearly a complete contraction.
We also have $\|\f(\xi)\|$ because




Do we have $E\subset M(\cK\otimes E)$?
\end{pf}




\section{Operator systems}

\begin{prb}[Choi-Effros characterization]

\end{prb}

\begin{prb}[Von Neumann inequality]
\end{prb}


The set $M_n(A)^+$ is linearly spanned by the rank-one projections $[a_i^*a_j]\in M_n(A)$ for $[a_i^*]\in A^n$, so a linear map $\f:A\to B$ is completely positive if and only if $[\f(a_i^*a_j)]\ge0$ in $M_n(A)$ for $[a_i]\in A^n$.


\begin{prb}[$n$-positive maps]
Let $S$ be an operator space.
Let $A$ and $B$ be C$^*$-algebras.
\begin{parts}
\item (Cauchy-Schwarz inequality)
If $\f:A\to B$ is a 2-positive map, then $\lim_i\|\f(e_i)\|=\|\f\|$ for any approximate unit $e_i$ of $A$, and
\[\f(a)^*\f(a)\le\|\f\|\f(a^*a),\qquad a\in A.\]
\item (Multiplicative domain)
Let $\f:A\to B$ be a 4-positive map with $\|\f\|=1$.
If $a\in A$ satisfies $\f(a)^*\f(a)=\f(a^*a)$, then $\f(b)\f(a)=\f(ba)$ for all $b\in A$.
In particular, if $\f:B\to C$ is an extension of a $*$-homomorphism $\pi:A\to C$, then $\f(ab)=\pi(a)\f(b)$ and $\f(ba)=\f(b)\pi(a)$ for $a\in A$ and $b\in B$.
\end{parts}
\end{prb}
\begin{pf}
(a)
It suffices to show
\[\f(a)^*\f(a)\le\lim_\alpha\|\f(e_\alpha)\|\f(a^*a),\qquad a\in A,\]
since
\[\frac{\|\f(a)\|^2}{\|a\|^2}\le\lim_\alpha\|\f(e_\alpha)\|\frac{\|\f(a^*a)\|}{\|a^*a\|}\]
implies $\|\f\|^2\le\lim_\alpha\|\f(e_\alpha)\|\|\f\|$.
Suppose $B$ acts on a Hilbert space $H$ non-degenerately and faithfully.
Since $\f$ is 2-positive, we have
\[\mat[b]{\f(e_\alpha^2)&\f(e_\alpha a)\\\f(a^*e_\alpha)&\f(a^*a)}=\f^{(2)}\left(\mat[b]{e_\alpha^2&e_\alpha a\\a^*e_\alpha&a^*a}\right)=\f^{(2)}\left(\mat[b]{e_\alpha&a\\0&0}^*\mat[b]{e_\alpha&a\\0&0}\right)\ge0,\]
and it is equivalent to
\[\<\f(e_\alpha^2)\xi,\xi\>+2\Re\<\f(e_\alpha a)\eta,\xi\>+\<\f(a^*a)\eta,\eta\>\ge0,\qquad\xi,\eta\in H,\quad a\in A.\]
We put $\xi:=-(\|\f(e_\alpha)\|+\e)^{-1}\f(e_\alpha a)\eta$ for $\e>0$ to get
\[\f(e_\alpha a)^*\f(e_\alpha a)
\le\f(e_\alpha a)^*[2-(\|\f(e_\alpha)\|+\e)^{-1}\f(e_\alpha^2)]\f(e_\alpha a)
\le(\|\f(e_\alpha)\|+\e)\f(a^*a)\]
We have the desired inequality by taking limits for $\alpha$ and $\e$.

(b)
Since the second inflation $\f^{(2)}$ is 2-positive, we may write the Cauchy-Schwarz inequality
\[\f^{(2)}\left(\mat[b]{a&b\\0&0}\right)^*\f^{(2)}\left(\mat[b]{a&b\\0&0}\right)\le\f^{(2)}\left(\mat[b]{a^*a&a^*b\\b^*a&b^*b}\right),\]
so
\[\mat[b]{0&\f(a^*b)-\f(a^*)\f(b)\\\f(b^*a)-\f(b^*)\f(a)&\f(b^*b)-\f(b^*)\f(b)}\ge0,\]
which implies $\f(b^*a)-\f(b^*)\f(a)=0$ for any $b\in A$.

Note that $\|\pi\|=1$ if $\pi$ is not trivial.
Using the above argument for $a$ and $a^*$, we are done.
\end{pf}




\begin{prb}[Russo-Dye theorem]
If $C(X)\to B$ is positive, then it is c.p.
\end{prb}


\begin{prb}[Completely positive maps for matrix algebras]
Let $A$ be a C$^*$-algebra.
\begin{parts}
\item Choi matrix
\item
There is a one-to-one correspondence
\[\mathrm{CP}(M_n(\C),A)\to M_n(A)_+:\f\mapsto[\f(e_{ij})].\]
\item
Let $A$ be unital.
There is a one-to-one correspondence
\[\mathrm{CP}(A,M_n(\C))\to M_n(A)^*_+:\f\mapsto(s_\f:[a_{ij}]\mapsto\sum_{i,j}\<\f(a_{ij})e_j,e_i\>).\]
\item The above correspondences are (maybe?) isometric if we endow the complete norm on $\mathrm{CP}$.
\end{parts}
\end{prb}
\begin{pf}
(b)


\end{pf}

\section{Dilations and Extensions}

A linear map $\f:A\to B(H)$ is completely positive if and only if
\[\sum_{i,j}\<\f(a_i^*a_j)\xi_j,\xi_i\>\ge0,\qquad [a_i]\in A^n,\ [\xi_i]\in H^n.\]

\begin{prb}[Stinespring dilation]
Let $A$ be a C$^*$-algebra and $\f:A\to B(H)$ be a completely positive linear map.
A \emph{Stinespring dilation} of $\f$ is a representation $\pi:A\to B(K)$ together with a bounded linear operator $v:K\to H$ such that $\f=(\Ad v)\pi$.
\[\begin{tikzcd}[row sep=small]
A \rar{\f}\dar[swap]{\pi} & B(H) \dar[equals]\\
B(K) \rar{\Ad v} & B(H)
\end{tikzcd}\]
\begin{parts}
\item $\f$ has a Stinespring dilation $(\pi,v)$ such that $\bar{\pi(A)v^*H}=K$.
\item For a non-degenerate Stinespring dilation $(\pi,v)$ of $\f$, the operator $v$ is a coisometry if and only if $\sup_i\f(e_i)=1$.
\end{parts}
\end{prb}
\begin{pf}
(a)
As we have done in the construction of the GNS representation, define a sesquilinear form on the algebraic tensor product $A\odot H$ such that
\[\<a_2\otimes\xi_2,a_1\otimes\xi_1\>:=\<\f(a_1^*a_2)\xi_2,\xi_1\>,\qquad a_1\otimes\xi_1,a_2\otimes\xi_2\in A\odot H.\]
It is positive semi-definite since the complete positivity of $\f$ implies
\[\<\sum_ja_j\otimes\xi_j,\sum_ia_i\otimes\xi_i\>=\sum_{i,j}\<\f(a_i^*a_j)\xi_j,\xi_i\>\ge0,\qquad a_i\otimes\xi_i\in A\odot H.\]
Then, we obtain a Hilbert space $K:=\bar{(A\odot H)/\{\xi\in A\odot H:\<\xi,\xi\>=0\}}$.
This kind of construction of a Hilbert space is sometimes called the separation and completion.

Define $\pi:A\to B(K)$ such that
\[\pi(a)(b\dot\otimes\eta):=ab\dot\otimes\eta,\qquad a\in A,\quad b\dot\otimes\eta\in K,\]
and $v:K\to H$ such that
\[v(b\dot\otimes\eta):=\f(b)\eta,\qquad b\dot\otimes\eta\in K,\]
which is well-defined since the Cauchy-Schwarz inequality $|\f(b)|^2\le\|\f\|\f(b^*b)$ implies that
\[\|\f(b)\eta\|^2=\<|\f(b)|^2\eta,\eta\>\le\|\f\|\<\f(b^*b)\eta,\eta\>=\|\f\|\|b\dot\otimes\eta\|,\qquad b\dot\otimes\eta\in K.\]

Then, we can check $\pi(a)v^*\xi=a\dot\otimes\xi$ for $a\in A$ and $\xi\in H$ from
\begin{align*}
\<\pi(a)v^*\xi,b\dot\otimes\eta\>&=\<v^*\xi,a^*b\dot\otimes\eta\>=\<\f(b^*a)\xi,\eta\>=\<a\dot\otimes\xi,b\dot\otimes\eta\>,
\end{align*}
so it follows that $v\pi(a)v^*=\f(a)$ for $a\in A$ from
\[\<v\pi(a)v^*\xi,\eta\>=\<v^*\xi,a^*\dot\otimes\eta\>=\<\f(a)\xi,\eta\>,\qquad\xi,\eta\in H,\]
and the condition $\bar{\pi(A)v^*H}=K$.


\end{pf}



\begin{prb}[Arveson extension]
Let $A\subset B$ be C$^*$-algebras.
Let $\f:A\to B(H)$ be a c.p.~map and consider the following diagram:
\[\begin{tikzcd}[sep=small]
B\ar[dashed]{dr}{\tilde\f}&\\
A\ar{u}\ar[swap]{r}{\f}&B(H).
\end{tikzcd}\]
\begin{parts}
\item The norm preserving c.p.~extension $\tilde\f$ of $\f$ exists if $B$ is unital and $1_B\in A$.
\item The norm preserving c.p.~extension $\tilde\f$ of $\f$ exists if $\cA$ is unital and $B=A\oplus\C$.
\item The norm preserving c.p.~extension $\tilde\f$ of $\f$ exists if $\cA$ is non-unital and $B=\tilde\cA$.
\item The norm preserving c.p.~extension $\tilde\f$ of $\f$ always exists.
\end{parts}
\end{prb}



\begin{prb}[Representation extension]
Let $I$ be a left ideal of a C$^*$-algebra $B$.
For a representation $\pi:I\to B(H)$, there is a representation $\tilde\pi:B\to B(H)$ which extends $\pi$.
If $\pi$ is non-degenerate, the extension is unique and $\pi(e_\alpha b)\to\tilde\pi(b)$ and $\pi(be_\alpha)\to\tilde\pi(b)$ strongly for $b\in B$, where $e_i$ is an approximate unit of $I$.
The same holds for Hilbert module representations.
\end{prb}
\begin{pf}
We may assume $\pi$ is non-degenerate by replacing $H$ to $\bar{\pi(I)H}$.
Define $\tilde\pi:B\to B(H)$ such that
\[\tilde\pi(b)(\pi(a)\xi):=\pi(ba)\xi,\qquad a\in I,\ \xi\in H.\]
The well-definedness is from
\[\|\pi(ba)\xi\|^2=\<\pi(a^*b^*ba)\xi,\xi\>\le\|b\|^2\<\pi(a^*a)\xi,\xi\>=\|b\|^2\|\pi(a)\xi\|^2.\]
It is clearly a $*$-homomorphism and extends $\pi$.

For the uniqueness, if $\pi$ is non-degenerate and $\tilde\pi$ is a $*$-homomorphism which extends $\pi$, then
\[\tilde\pi(b)(\pi(a)\xi)=\tilde\pi(b)\tilde\pi(a)\xi=\tilde\pi(ba)\xi=\pi(ba)\xi,\]
which is unique by the density of $\pi(I)H$ in $H$.
\end{pf}

extension of representations for ideals

unique extension of c.p.~maps for hereditary subalgebras.







\begin{prb}[Haagerup tensor product]
\end{prb}

Trick

\section*{Exercises}
\begin{prb}
Let $A$ be a hereditary C$^*$-subalgebra of a C$^*$-algebra $B$ and let $b\in B_+$.
If for any $\e>0$ there is $a\in A_+$ such that $b\le a+\e$, then $b\in A$.
\end{prb}
\begin{pf}
For $a\in A_+$ satisfying $b\le a+\e\le(a^{\frac12}+\e^{\frac12})^2$, define
\[a_\e:=a^{\frac12}(a^{\frac12}+\e^{\frac12})^{-1}ba^{\frac12}(a^{\frac12}+\e^{\frac12})^{-1}\in A.\]
Then, 
\[\|b^{\frac12}-b^{\frac12}a^{\frac12}(a^{\frac12}+\e^{\frac12})^{-1}\|^2=\e\|(a^{\frac12}+\e^{\frac12})^{-1}b(a^{\frac12}+\e^{\frac12})^{-1}\|\le\e.\]
Thus $a_\e\to b$ in norm as $\e\to0$.
\end{pf}












\chapter{Categorical constructions}



inverse limits: direct sum, direct product, restricted direct sum, locally C$^*$-algebras.

Infinite direct sums and direct products are ill-behaved in the category of C$^*$-algebras.
An infinite direct sum must be interpreted as complete Hausdorff spaces, not a pointed compact Hausdorff space.
For example, after adding a base point, the spectrum of $\bigoplus_{i=1}^\infty C_0(\R)$ corresponds to the Hawaiian earing, and the spectrum of $\prod_{i=1}^\infty C_0(\R)$ corresponds to the Stone-\v Cech compactification of the infinite wedge of circles.
We cannot describe the infinite wedge of circles in terms of C$^*$-algebras, so we need locally C$^*$-algebras.


\begin{prb}[Locally C$^*$-algebras]
A \emph{locally C$^*$-algebra} is a complete topological $*$-algebra whose topology is generated by C$^*$-semi-norms.
We adopt the convention that a \emph{homomorphism} between locally C$^*$-algebras means a continuous $*$-homomorphism.
\begin{parts}
\item A topological $*$-algebra is a locally C$^*$-algebra if and only if it is an inverse limit of unital C$^*$-algebras.
\end{parts}
\end{prb}
\begin{pf}
(a)
Let $A$ be a locally C$^*$-algebra.
The set of continuous C$^*$-seminorms on $A$ is a directed set.
Construct an inverse system...
Since every C$^*$-algebra is a maximal ideal of a unital C$^*$-algebra of codimension one, we may assume that the objects in this inverse system is unital...
Also, elements of $A$ are represented by coherent sequences.
\end{pf}



\section{Tensor products}




\begin{prb}[Maximal tensor products]
Let $A$ and $B$ be C$^*$-algebras.
\[\|\sum_ia_i\otimes b_i\|_{\max}:=\sup_\Pi\|\sum_i\Pi(a_i\otimes b_i)\|,\]
where $\Pi:A\odot B\to B(H)$ run over all representations.
\begin{parts}
\item (restrictions) A commuting pair of $*$-homomorphisms $\pi:A\to B(H)$ and $\pi':B\to B(H)$ corresponds to a $*$-homomorphism $\Pi:A\odot B\to B(H)$ via the relation $\Pi(a\otimes b)=\pi(a)\pi'(b)$.
\item $A\odot B$ admits a $*$-representation and every norms induced from these $*$-representations are uniformly bounded. So, we can define a maximal tensor norm on $A\odot B$.
\item $a\otimes-:B\to A\odot B$ is a bounded linear map for each $a\in A$ with respect to any C$^*$-norm on $A\odot B$. [BO, 3.2.5]
\end{parts}
\end{prb}


\begin{prb}[Minimal tensor product]
\[\|\sum_ia_i\otimes b_i\|_{\min}:=\sup_{\pi,\pi'}\|\sum_i\pi(a_i)\otimes\pi'(b_i)\|,\]
where $\pi:A\to B(H)$ and $\pi':B\to B(H')$ run over all representations.
If $\pi$ and $\pi'$ are cyclic with cyclic vectors $\Omega$ and $\Omega'$, then since $\pi(A)\Omega\odot\pi'(B)\Omega'$ is dense in $H\otimes H'$, we have
\[\|\pi(a_i)\otimes\pi'(b_i)\|=\sup_{a,b}\frac{\<(\pi(a_i)\otimes\pi'(b'))(\pi(a)\Omega\otimes\pi'(b)\Omega'),(\pi(a)\Omega\otimes\pi'(b)\Omega')\>}{\<(\pi(a)\Omega\otimes\pi'(b)\Omega'),(\pi(a)\Omega\otimes\pi'(b)\Omega')\>}.\]

\begin{parts}
\item $B(H)\odot B(K)\to B(H\otimes K)$ is injective.
\item If $\pi$ and $\pi'$ are any faithful representations, then $\|\sum_ia_i\otimes b_i\|_{\min}=\|\sum_i\pi(a_i)\otimes\pi'(b_i)\|$.
\end{parts}
\end{prb}
\begin{prb}[Takesaki theorem]
\end{prb}

Tensors with $M_n(\C)$, $C_0(X)$.



finite dimensional[BO, 3.3.2], abelian, AF
permanence properties


\begin{prb}[Completely positive approximation property]
Let $A$ be a C$^*$-algebra.
We say $A$ has the \emph{completely positive approximation property} if the identity is contained in the point-norm, or equivalently the point-weak closure of $\cF$ in $L(A)$.
\begin{parts}
\item If $A$ has the completely positive approximation property, then $A$ is nuclear.
\item If $A$ is nuclear, then $A$ has the completely positive approximation property.
\end{parts}
\end{prb}
\begin{pf}

(b)
For finite sequences $a_k$ in $A$ and $\omega_k$ in $A^*$, we want to construct completely positive contractions $\f:A\to M_n(\C)$ and $\psi:M_n(\C)\to A$ such that
\[|\omega_k(a_k)-\omega_k(\psi\circ\f(a_k))|<\e,\qquad k\ge0.\]
If we consider a representation $\pi_2:A\to B(H_2)$ with a unit cyclic vector $\Omega\in H$ such that $\omega_k$ admits the commutant Radon-Nikodym derivative $h_k$ for each $k$, then the composition can be written as
\[\omega_k(a_k)=\<h_k\pi_2(a_k)\Omega,\Omega\>,\qquad k\ge0.\]
It suggest to consider a state on $\tilde\omega:A\otimes_{\max}\pi_2(A)'$ defined such that
\[\tilde\omega(a\otimes h)=\<h\pi_2(a)\Omega,\Omega\>,\qquad a\in A,\ h\in\pi_2(A)'.\]

With a faithful representation $\pi_1:A\to B(H_1)$, consider the faithful representation $\pi_1\otimes\id:A\otimes_{\min}\pi_2(A)'\to B(H_1\otimes H_2)$.
If
\[\sum_{i,j=1}^n\<(-)(\eta_j\otimes\pi(b_j)\Omega),\eta_i\otimes\pi(b_i)\Omega\>\]
is a vector state of $B(H_1\otimes H_2)$ for finite sequences $\eta_j$ in $H_1$ and $\pi_2(b_j)\Omega$ in $H_2$ with index $1\le j\le n$, and if we define $\f:A\to M_n(\C)$ and $\psi:M_n(\C)\to A$ by
\[\f(a):=(\<\pi_1(a)\eta_j,\eta_i\>),\qquad\psi((\delta_{ii'}\delta_{jj'})):=b_{i'}^*b_{j'}^*,\qquad a\in A,\]
then
\begin{align*}
\<((\pi_1\otimes\id)(a\otimes h))(\eta_j\otimes\pi(b_j)\Omega),\eta_i\otimes\pi(b_i)\Omega\>
&=\<\pi_1(a)\eta_j\otimes h\pi_2(b_j)\Omega,\eta_i\otimes\pi(b_i)\Omega\>\\
&=\<\pi_1(a)\eta_j,\eta_i\>\<h\pi_2(b_i^*b_j)\Omega,\Omega\>\\
&=\<h\pi_2(\<\pi_1(a)\eta_j,\eta_i\>b_i^*b_j)\Omega,\Omega\>\\
&=\<h\pi_2(\psi\circ\f(a))\Omega,\Omega\>\\
&=\tilde\omega(\psi\circ\f(a)\otimes h),
\end{align*}
where the sum $\sum_{i,j}$ is omitted.
Since such states induced from vector states of $B(H_1\otimes H_2)$ via $\pi_1\otimes\id$ approximates the state space of $A\otimes_{\min}\pi_2(A)'$ by the Hahn-Banach separation, and since factorable maps form a convex set, we are done.

\end{pf}

The set $\cF$ of factorable maps is a convex set of $L(A)$.
Note that we have an embedding
\[L(A)\hookrightarrow L(A,A^{**})=\lim_{\substack{\longleftarrow\\F}}L(A,F^*).\]
We have a continuous bijection
\[(A\hat\otimes_\pi F)^*\to L(A,F^*).\]
If we let $M:=\pi(A)''\subset B(H)$ be the GNS representation for $F$, then the Radon-Nikodym theorem on commutant gives rise to a continuous map
\[(A\mathrel{\hat\otimes}_\pi M')^*\to(A\mathrel{\hat\otimes}_\pi F)^*.\]




quotients of nuclear
local reflexivity



\begin{prb}
A C$^*$-algebra $C$ is called \emph{injective} every completely positive map $\f:A\to C$ from a C$^*$-subalgebra $A$ of a C$^*$-algebra $B$ is extended to a completely positive map $\tilde\f:B\to C$.
A von Neumann algebra is called injective if it is injective as a C$^*$-algebra.
(operator subsystem? unital?)

The C$^*$-algebra $B(H)$ is injective, and its image of completely positive idempotent is injective.
A von Neumann algebra on $M$ on $H$ is injective if and only if there is a conditional expectation $B(H)\to M$.

\end{prb}



$A^{**}$ semi-discrete -> $A$ nuclear is done by four step approximation

The reverse implication follows from $A$ is nuclear -> $A'$ is injective -> $A''$ is injective -> $A''$ is semi-discrete.


Let $A$ be nuclear.
Note $A^{**}=I^{**}\oplus(A/I)^{**}$.
Since $A^{**}$ is semi-discrete, $(A/I)^{**}$ is semi-discrete.
Therefore, $A/I$ is nuclear.








a separable C$^*$-algebra is nuclear if and only if every factor representation is hyperfinite.

Extension properties
weak expectation property
relatively weakly injective
maximal tensor product inclusion problem



excision: Akemann-Anderson-Pedersen







\section{Free products}

amalgamated


\section{Colimits}

\begin{prb}[Uniformly hyper-finite algebras]
\end{prb}

\begin{prb}[Approximately finite algebras]
\end{prb}


AF embeddability


\begin{prb}[Fermion algebra]

\end{prb}









\section{Universal algebras}

\begin{prb}[Cuntz algebras]
\end{prb}
\begin{prb}[Cuntz-Krieger algebras]
\end{prb}
\begin{prb}[Graph algebras]
\end{prb}




\begin{prb}[C$^*$-correspondences]
Let $A$ and $B$ be C$^*$-algebras.
A \emph{correspondence} or a \emph{right Hilbert bimodule} over $(A,B)$ is a Hilbert $B$-module $E$ together with a $*$-homomorphism $\f:A\to B(E)$.
We say $E$ is \emph{faithful} or \emph{non-degenerate} if the left action is faithful or non-degenerate, respectively.
\begin{parts}
\item If $\f:A\to M(B)$ is a non-degenerate completely positive map, then we can construct a natural correspondence $E$ from $A$ to $B$ by mimicking the GNS construction on $A\odot B$.
\item If $\f:A\to M(B)$ is a non-degenerate $*$-homomorphism, $\f\in\Mor(A,B)$ in other words, then we can associate a canonical right Hilbert bimodule $E:=B$ over $(A,B)$ such that the left action is realized with $\f$.
\end{parts}
\end{prb}


\begin{prb}[Pimsner construction]
Let $A$ and $B$ be C$^*$-algebras.
Let $E$ be a right Hilbert bimodule over $(A,A)$.
A \emph{representation} of $E$ on $B$ is a pair $(\pi,\tau)$ of a $*$-homomorphism $\pi:A\to B$ and a linear map $\tau:E\to B$ such that
\[\pi(\<\eta,\xi\>)=\tau(\eta)^*\tau(\xi),\qquad\tau(\f(a)\xi)=\pi(a)\tau(\xi),\qquad\xi,\eta\in E,\ a\in A.\]
We automatically have $\tau(\xi a)=\tau(\xi)\pi(a)$ by computing the norm of their difference.

Given a representation $(\pi,\tau)$ of $E$ on $B$, we have a canonical $*$-homomorphism $\psi:K(E)\to C^*(\pi,\tau):\theta_{\xi,\eta}\mapsto\tau(\xi)\tau(\eta)^*$.
We define the \emph{Katsura ideal}
\[J(E):=\f^{-1}(K(E))\cap\f^{-1}(0)^\perp,\]
where $\f:A\to B(E)$ denotes the left action.
We say a Toeplitz representation of $E$ is \emph{covariant} if
\[\psi(\f(a))=\pi(a),\qquad a\in J(E).\]

The \emph{Toeplitz-Pimsner algebra} is the C$^*$-algebra generated by the universal representation, and the \emph{Cuntz-Pimsner algebra} is the C$^*$-algebra generated by the universal covariant representation.
\begin{parts}
\item
Let $(A,\Z,\alpha)$ be a C$^*$-dynamical system and consider the canonical C$^*$-correspondence $E=A$ over $(A,A)$ with the left action $\f:=\alpha_1\in\Aut(A)\subset\Mor(A)$.
This correspondence is full, faithful, and non-degenerate.
Note that also we have $J(A)=A$.

A covariant representation of the dynamical system is a representation $\pi:A\to B(H)$ together with a unitary $u\in U(H)$ such that $\pi(\f(a))=u\pi(a)u^*$.
If $(\pi,\tau)$ is an any representation of this C$^*$-correspondence $E$ on $B$, then the most natrual choice of $\tau$ and the associated $\psi$ is
\[\tau(a):=u^*\pi(a),\qquad\psi(a):=u^*\pi(a)u.\]


\end{parts}
\end{prb}

\begin{pf}

We also have extended linear maps $\tau^n:E^{\otimes n}\to C^*(\pi,\tau)$ and $*$-homomorphisms $\psi^n:K(E^{\otimes n})\to C^*(\pi,\tau)$.
\end{pf}

\begin{prb}[Gauge invariant uniqueness theorem]
\end{prb}

How can we decribe representations of C$^*$-correspondence $A$ with left action $\f\in\Aut(A)$ in terms of covariant representations of the C$^*$-dynamical system $(A,\Z,\alpha)$ with $\alpha_n=\f^n$?



as a morphism
sub and quotient, direct sum, tensor product,

Toeplitz-Cuntz
Toeplitz-Pimsner
Cuntz-Pimsner
Cuntz-Krieger



Subproduct systems




















\part{Dynamics}






\chapter{Group actions}

\section{Amenability}



crossed products
$Z_2$-grading
Connes-Feldman-Weiss
Anantharaman-Delaroche
Gromov boundaries
approximately central structure?
dynamical Kirchberg-Phillips

stably finite
dynamical Elliott program

Ornstein-Weiss-Rokhlin lemma



Exact groups


Other properties:
Kazdahn property (T)
factorization property
Haagerrup property


Kaplansky conjecture




A state $\tau$ on $A$ is called an \emph{amenable trace} if there is a state $\omega$ of $B(H)$ such that $\omega$ extends $\tau$ and $\omega(uxu^*)=\omega(x)$ for $x\in B(H)$ and $u\in U(A)$.
It is automatically tracial.
The amenability of a trace does not depend on the choice of faithful representation of $A$, using the Arveson extension and the multiplicaitve domain.

For a discrete group $\Gamma$, $C^*_r(\Gamma)$ is amenable if and only if has an amenable tracial state.
Note that a mean is a state of $\ell^\infty(\Gamma)$, which may not be normal.






\section{Crossed products}


\begin{prb}[Group algebras]
Let $G$ be a locally compact group.

\end{prb}


type I, subhomogeneous


crystallographic
discrete heisenberg
free groups
projectionless of $C_r^*(F_2)$



\begin{prb}[Enveloping C$^*$-algeberas]
Let $A$ be a $*$-algebra.
A \emph{C$^*$-norm} is an submultiplicative norm satisfying the C$^*$-identity.
Does $A$ have enough $*$-representations?
\begin{parts}
\item A complete C$^*$-norm is unique if it exists.
\item For every C$^*$-norm $\alpha$ on $A$, there is a $*$-isometry $\pi:A\to B(H)$.
\item For maximal C$^*$-norm, there is a universal property. The maximal C$^*$-norm can be obtained by running through cyclic representations.
\end{parts}
\end{prb}




\begin{prb}[C$^*$-dynamical system]
Let $G$ be a locally compact group.
A \emph{C$^*$-dynamical system} or a \emph{$G$-C$^*$-algebra} is a C$^*$-algebra $A$ together with a group homomorphism $\alpha:G\to\Aut(A)$ that is continuous in the point-norm topology.
We will often write a triple $(A,G,\alpha)$ instead of $A$ to refer to a C$^*$-dynamical system.
\begin{parts}
\item There is an equivalence between categories of locally compact transformation groups and C$^*$-dynamical system on abelian C$^*$-algebras.
\end{parts}
\end{prb}


On $U(H)$, the strict topology and the strong operator topology are equal.
Therefore, we have three topologies to consider: strong, weak, and $\sigma$-weak.

\begin{prb}[Covariant representation]
Let $G$ be a locally compact group.

A \emph{covariant representation} of a C$^*$-dynamical system $(A,G,\alpha)$ is a $G$-equivariant $*$-homomorphism $\pi:(A,G,\alpha)\to(B(H),G,\beta)$ for a C$^*$-dynamical system $(B(H),G,\beta)$, where a Hilbert space $H$.
\begin{parts}
\item
There exists a unitary representation $u:G\to B(H)$ such that $\pi(\alpha_sa)=u_s\pi(a)u_s^*$.
\item (Integrated form)
There is a one-to-one correspondence between covariant representations of $(A,G,\alpha)$ and $*$-representations of $L^1(G,A)$. (non-degenerate)
\end{parts}
\end{prb}

Note that we have a homeomorphism $\Aut(K(H))\cong PU(H)$ between the point-norm topology and the strong operator topology.

$\Z$-action, $\Homeo$-action, left multiplication of subgroup
induced representation
regular representation $(C_0(G),G,\lambda)\to(B(L^2(G)),G,\lambda)$.


commutative case




\chapter{Groupoid algebras}

\section{}

groupoid models for C$^*$-algebras (non-commutative spaces)

\begin{prb}[Action groupoids]
$G\ltimes C_0(X)$ and $C^*(G\ltimes X)$..?
\end{prb}

\begin{prb}[Deaconu-Renault groupoids]
\end{prb}

\section{Fell bundles}











\chapter{Continuous fields}


\section{Upper semi-continuous fields}

\begin{prb}[Fields of C$^*$-algebras]
Let $X$ be a locally compact Hausdorff space.


There are one-to-one correspondences among
\begin{parts}
\item non-degenerate $*$-homomorphisms $C_0(X)\to Z(M(A))$,
\item continuous maps $\mathrm{Prim}(A)\to X$,
\item upper semi-continuous fields of C$^*$-algebras over $X$, also called (H)-bundles.
\end{parts}
and among
\begin{parts}
\item continuous open maps $\mathrm{Prim}(A)\to X$,
\item continuous fields of C$^*$-algebras over $X$, also called (F)-bundles.
\end{parts}

The Dauns-Hoffman theorem states that there is a natural $*$-isomorphism
\[Z(M(A))\to C_b(\mathrm{Prim}(A)).\]

tensor products?
\end{prb}



\begin{prb}[Upper semi-continuous fields of C$^*$-algebras]
Let $X$ be a locally compact Hausdorff space.
Let $\{A(x)\}$ be a \emph{field of C$^*$-algebras} over $X$, which is just a family of C$^*$-algebras parametrized by $x\in X$.
The field $\{A(x)\}$ is called \emph{upper semi-continuous} if the total space $E:=\coprod_{x\in X}A(x)$ admits a topology such that
\begin{enumerate}[(i)]
\item $+:E\times_XE\to E$ is continuous,
\item $\cdot:\C\times E\to E$ is continuous,
\item $\|\cdot\|:E\to\R_{\ge0}$ is upper semi-continuous,
\item $A(x)\to E$ is a topological embedding for each $x\in X$,
\item $E\to X$ is an open quotient map.
\end{enumerate}
(Which conditions are redundant?)
Let $A$ be the set of continuous sections $X\to E$ vanishing at infinity with respect to the canonical proejction $E\to X$.
We want to show $A$ is a C$^*$-algebra together with a natural non-degenerate $*$-homomorphism $C_0(X)\to Z(M(A))$.
\begin{parts}
\item The topology on $E$ is uniquely determined....?
\end{parts}
\end{prb}
\begin{pf}


Define a norm on $A$ such that $\|a\|=\sup_{x\in X}\|a(x)\|$ for each $a\in A$.


$\Gamma_0(\coprod_{x\in X}A(x))$ is a $C_0(X)$-module.
\end{pf}






\begin{prb}
Let $X$ be a locally compact Hausdorff space.

\begin{parts}
\item If $\{A(x)\}$ is an upper semi-continuous field of C$^*$-algebras over $X$, then the space of sections $A$ vanishing at infinity is a C$^*$-algebra together with a natural non-degenerate $*$-homomorphism $C_0(X)\to Z(M(A))$.
\item If $A$ is a C$^*$-algbera and together with a non-degenerate $*$-homomorphism $C_0(X)\to Z(M(A))$, then there is an upper semi-continuous field of C$^*$-algebras over $X$ such that $A$ is $*$-isomorphic to the space of sections vanishing at infinity.
\item If fibers $A(x)$ are all $*$-isomorphic, then the field is continuous and $\Gamma_0\cong C_0(X,A(x))$ for any $x\in X$.
\end{parts}
\end{prb}
\begin{pf}


(b)
For each $x\in X$, since $AC_0(X\setminus\{x\})$ is a closed ideal with Cohen factorization, the fiber $A(x):=A/AC_0(X\setminus\{x\})$ of $A$ is a well-defined C$^*$-algebra.
Denote by $a(x)\in A(x)$ the image of $a\in A$ under the quotient $*$-homomorphism $A\to A(x)$.
Define the total space $E:=\coprod_{x\in X}A(x)$ with the quotient topology induced from the evaluation map $A\times X\to E$, and let $\Gamma_0$ be the set of continuous sections $a:X\to E$ of the projection map $\pi:E\to X:a(x)\mapsto x$ vanishing at infinity.
We claim that $\{A(x)\}$ is an upper semi-continuous field over $X$ and there is a natural $*$-isomorphism $A\to\Gamma_0$.

---

As a first step, we prove the evaluation map $A\times X\to E$ is open.
It suffices to prove that the cylindrical open set $\{(a,x):\|a\|<1,\ x\in U\}$ in $A\times X$ is mapped onto the open set $\{\xi:\|\xi\|<1,\ \pi(\xi)\in U\}$ in $E$ under the evaluation map, where $U$ is an open subset of $X$.
Suppose $\xi\in E$ satisfies $\|\xi\|<1$ and $\pi(\xi)\in U$.
Take $(a_0,x_0)\in A\times U$ satisfying $a_0(x_0)=\xi$.
Since $x\mapsto\|a_0(x)\|$ is upper semi-continuous, we can take a decreasing sequence $(U_n)_{n=0}^\infty$ of open neighborhoods of $x_0$ in $X$ such that $\|a_0(x)\|<(1+2^{-n})\|a_0(x_0)\|$ on $x\in U_n$.
For each $n$, construct $f_n\in C_0(X)$ such that $\supp f_n\subset U_n$ and $0\le f_n(x)\le f_n(x_0)=2^{-n}$ on $x\in X$.
If we define $f(x):=\sum_{k=1}^\infty f_k(x)$, then $fa_0$ satisfies $(fa_0)(x_0)=f(x_0)a_0(x_0)=a_0(x_0)$ and $\|fa_0\|\le\|a_0(x_0)\|<1$ since
\[\|(fa_0)(x)\|\le\sum_{k=1}^nf_k(x)\|a_0(x)\|<\sum_{k=1}^n2^{-k}(1+2^{-(n+1)})\|a_0(x_0)\|\le\|a_0(x_0)\|,\qquad x\in U_{n+1}\setminus U_n.\]
Therefore, the evaluation map is open.

Before proceeding the argument, we check that $\Gamma_0$ is closed under addition.
Let $a,b\in\Gamma_0$ and let $(x_i)_{i\in I}$ be a net in $X$ such that $x_i\to x$, which has $a(x_i)\to a(x)$ by continuity of $a$.
Although we do not have $a,b\in A$ yet, but we can take $a_0,b_0\in A$ such that $a(x)=a_0(x)$ and $b(x)=b_0(x)$.
Let $\cU$ be the filtered set of open neighborhoods of $(a_0,x)$ in $A\times X$, and let $K:=I\times\cU$ be the product directed set reversing the order of $\cU$.
Take a subnet $x_k$ of $x_i$ by defining a cofinal map $K\to I:(i,U)\mapsto i'$ such that $i'\succ i$ and $a(x_{i'})\in\ev(U)$, which can be done because the evaluation is an open map and the convergence $a(x_i)\to a_0(x)$.
Choosing $a_k\in U$ for each $k\in K$ such that $\ev(a_k,x_k)=a(x_k)$, we have $a_k\to a_0$ in $A$ and $a_k(x_k)=a(x_k)$ in $E$.
By doing the same for $b$, we may also assume that there is a net $b_k\in A$ such that $b_k\to b_0$ and $b_k(x_k)=b(x_k)$.
From $a_k+b_k\to a_0+b_0$ in $A$, we have the limit for $k$
\[a(x_k)+b(x_k)=a_k(x_k)+b_k(x_k)=(a_k+b_k)(x_k)\to(a_0+b_0)(x)=a_0(x)+b_0(x)=a(x)+b(x).\]
Since $a(x_i)$ and $b(x_i)$ are Cauchy, we have $a(x_i)+b(x_i)\to a(x)+b(x)$, so $a+b\in\Gamma_0$.


---

Now we prove the $*$-isomorphism.
If $a\in A$ satsfies $a(x)=0$ for each $x\in X$, meaning that we have a factorization $a=a_xf_x\in AC_0(X)$ such that $f_x(x)=0$, then since every pure state $\delta$ of $A$ is restricted to a point measure $\delta_x$ on $X$ so that $\delta(a)=\delta(a_xf_x)=\delta(a_x)\delta_x(f_x)=0$, we have $a=0$.
To show that $A\to\Gamma_0$ has dense image,
partition of unity.

(c)
By the quotient topology on the total space, we have a commutative diagram
\[\begin{tikzcd}[sep=small]
A\times X \rar\dar\ar{dr} & A(x)\\
X & E \uar\lar
\end{tikzcd}\]
of natural contiinuous maps.
The composition with $E\to A(x)$ gives rise to a map $\Gamma_0\to C_0(X,A(x))$, and the inverse is taken by the lift using the universality of coproduct, so we are done.
\end{pf}




\begin{prb}[Continuous fields of Banach spaces]
Let $X$ be a locally compact Hausdorff space.
A \emph{continuous field of Banach spaces} over $X$ is a family $\{V(x)\}$ of Banach spaces parametrized by $x\in X$ such that the total space $E:=\coprod_{x\in X}E(x)$ admits a topology such that
\begin{enumerate}[(i)]
\item $+:E\times_XE\to E$ is continuous,
\item $\cdot:\C\times E\to E$ is continuous,
\item $\|\cdot\|:E\times\R_{\ge0}$ is continuous,
\item $V(x)\to E$ is a topological embedding,
\item $E\to X$ is an open quotient map.
\end{enumerate}
(Which conditions are redundant?)
Let $\Gamma_0$ be the set of continuous sections $X\to E$ vanishing at infinity with respect to the canonical proejction $\pi:E\to X$.
\begin{parts}
\item If $\{V(x)\}$ is a continuous field of Banach spaces over $X$, then $\Gamma_0$ is a Banach module over $C_0(X)$.
\item If $V$ is a Banach module over $C_0(X)$, then it has a structure of continuous field of Banach spaces.
\end{parts}
\end{prb}
\begin{pf}
(a)


(b)


\end{pf}


\begin{prb}[Continuous fields of Hilbert modules]
Let $X$ be a locally compact Hausdorff space.






\begin{parts}
\item There is an equivalence between the category of Hilbert modules over $X$ and the category of coontinuous fields of Hilbert spaces over $C_0(X)$.
\end{parts}
\end{prb}


\begin{prb}[Vector bundles]
Let $X$ be a compact Hausdorff space.
Let $E$ be a finitely generated projective module over $C(X)$.
The rank is defined on each connected component, and it is uniformly bounded.
A ($C(X)$-linear continuous) group action on $E$ is corresponded to (fiber-preserving continuous) group action on $V$.
Where does the local triviality of principal $G$-bundles come from?


manifold structure on total space

some constructions like duals, tensors, symmetric, and exterior algebras, can it be interpreted as the module operations? and projectivity is preserved?
how about submodules, kernels, and images for the projectivity? how can their topologies on total spaces be described?

paracompactness? existence of metrics and isomorphism with duals? The section space of the tangent bundle of the long line is not an algebraically finitely generated projective module. (which condition is failed?)

\begin{parts}
\item There is an equivalence between the category of algebraically finitely generated projective Hilbert modules over $C(X)$ and the category of locally trivial finite-rank complex vector bundles over $X$. (Serre-Swan with Hermitian metric)
\end{parts}
\end{prb}

\begin{pf}
If $E$ if a algebraically finitely generated projective module so that $E$ is a complemented Hilbert submodule of $C(X)^n$, then we can identify $E$ as a continuous function $p:X\to M_n(\C)$ such that $p(x)$ is a projection for all $x\in X$.
Since the rank function on $M_n(\C)$ is lower semi-continuous, two integer-valued functions $\rk p$ and $\rk(1-p)=n-\rk p$ on $X$ are lower semi-continuous, which means that $\rk p:X\to\Z_{\ge0}$ is locally constant.
Therefore, $E$ is locally trivial.

\end{pf}







A $C(X)$-linear operator (Fr\'echet modules) and a $C(X)$-adjointable operator (Hilbert modules)?


span of $a[D,b]$
completion of the span of the gradient of test functions,
dual of Borel time-dependent vector field,

For discussion of tangent vectors as in optimal transport theory:
sufficiently many absolutely continuous curves?




\section{Dixmier-Douady theory}

\begin{prb}[Dixmier-Douady theory]
	
\end{prb}


Fell's condition

A C$^*$-algebra $A$ is called \emph{continuous trace} if the set of all $a\in\cA$ such that $\hat A\to\R_{\ge0}:\pi\mapsto\tr(\pi(a^*a))$ is continuous is dense in $A$.



Dadarlat-Pennig theory







\section{Fell bundles}



Coactions and Fell bundles



























\part{Classification}
\chapter{Operator K-theory}

\section{Zeroth K-groups}

\begin{itemize}
\item functoriality
\item homotopy invariance
\item filtered cocontinuity and opertor stability
\item half-exactness, split-exactness
\item long exact sequence -> section 3
\end{itemize}

with additional properties
\begin{itemize}
\item lax symmetric monoidal functor
\item partial order
\item ring axioms for $K_0$ only on commutatives
\end{itemize}




\begin{prb}[Zeroth K-group of unital algebras]
Let $A$ be a unital C$^*$-algebra.
We define $K_0(A)$ by the algebraic K-theory of a unital ring $A$, the Grothendieck group of the category of algebraically finitely generated projective right $A$-modules.
Equivalently, if we define $V(A):=P(M_\infty(A))/\sim$, then it gives a functor $V$ from the category of unital C$^*$-algebras to the category of ordered abelian monoid with cancellation property, so we can define $K_0(A):=G(V(A))$, the Grothendieck group of the monoid $V(A)$.
Its elements can be described by $[p]-[q]$ for projections $p$ and $q$ in $M_n(A)$ for some $n\ge0$.
\begin{parts}
\item $V(M_n(\C))\cong\Z_{\ge0}$ because two projections are equivalent iff they have same range dimensions, so $K_0(M_n(\C))\cong\Z$.
\item $K_0(B(H))\cong K_0(Q(H))\cong0$ if $H$ is infinite-dimensional, since
\[V(K(H))\cong\Z_{\ge0}=\mathrm{Card}_{<\omega},\qquad V(B(H))\cong\mathrm{Card}_{\le\dim H},\qquad V(Q(H))\cong\{0\}\cup(\mathrm{Card}_{\ge\omega}\cap\mathrm{Card}_{\le\dim H}).\]
The corner map $P(A)\to K_0(A)$ is sometimes called the \emph{dimension function} of a C$^*$-algebra $A$.
\item $K_0(C(S^2))\cong\Z^2$.
\item For a II$_1$ factor $M$, $K_0(M)\cong\R$.
\item $K_0(\cO_n)\cong\Z/(n-1)\Z$.
\end{parts}
\end{prb}


\begin{prb}[Zeroth K-group of non-unital algebras]
We want to discuss the exactness of K-theory.
For this, we have to consider pairs of C$^*$-algebras.
We define a \emph{pair} of C$^*$-algebras as a surjective $*$-homomorphism between unital C$^*$-algebras.
Let $\pi:A\to B$ is a pair of C$^*$-algebras.
Then, $K_0(A,B)$ can be concretely described or defined by the set of equivalence classes of $(p,q,v)$, where $p$ and $q$ are projections in $\cK\otimes A$ and $v\in M(\cK\otimes A)$ satisfies $\pi(p)=\pi(v^*v)$ and $\pi(q)=\pi(vv^*)$.
In fact, we can show $K_0(A,B)$ only depends on the kernel $I:=\ker\pi$.
It is called the excision theorem.



For a general non-unital C$^*$-algebra $I$, it is well-defined that
\[K_0(I):=K_0(A,A/I),\]
where $A$ is any unitization of $I$.
We can show that if $I$ is unital, then it is naturally isomorphic to the original without-base-point definition of K-theory.(for example, $K_0(A)\cong K_0(A\oplus\C,\C)$ for unital $A$)
In particular, since $K_1(\C)=0$, the six-term exact sequence implies that $K_0(I)\cong\ker(K_0(\tilde I)\to K_0(\C))$, and since $0\to I\to\tilde I\to\C\to0$ splits, we have $K_0(I)\oplus\Z\cong K_0(\tilde I)$.
A generally non-unital C$^*$-algebra is the non-commutative analogue of the pointed quotient of compact pairs.


Even if $A$ and $B$ are non-unital, one can check the followings are exact:
\[K_0(I)\to K_0(A)\to K_0(B)\]
$[p,q,v]\mapsto[p]-[q]\mapsto$...

We do not have to introduce the notation $\tilde K$, because we basically consider the unital algebra $C(X)$ not as a pointed space $(X,x_0)$ (like in topology), but as $(X\cup*,*)$, i.e.~$K(C_0(X))=\tilde K(X\cup*,*)$ for compact or non-compact $X$.
\end{prb}




\begin{prb}[Equivalences of projections]
Let $A$ be a C$^*$-algebra.
Denote by $P(A)$ and $U(A)$ the set of all projections and unitaries in $A$ respectively.
Let $p$ and $q$ be projections of $A$.
Recall that they are called
\begin{enumerate}[(i)]
\item \emph{Murray-von Neumann equivalent}, denoted $p\sim q$, if $p=v^*v$ and $q=vv^*$ for some $v\in A$,
\item \emph{unitarily equivalent}, denoted by $p\sim_uq$, if $upu^*=q$ for some unitary $u$ in $\tilde A$,
\item \emph{homotopic}, denoted by $p\sim_hq$, if there is a norm continuous path of projections in $A$ connecting them.
\end{enumerate}
\begin{parts}
\item $p\sim_uq$ implies $p\sim q$, and $p\sim q$ implies $p\oplus0\sim_uq\oplus0$ in $M_2(A)$.
\item $p\sim_hq$ implies $p\sim_uq$, and $p\sim_uq$ implies $p\oplus0\sim_hq\oplus0$ in $M_2(A)$.
\end{parts}
\end{prb}



\begin{pf}

We give some lemmas.
First, if $a\in A$ satisfies $0\le a\le1$ and $\|a-a^2\|<\e\le4^{-1}$, then $p:=1_{(\frac12,1]}(a)$ defines a projection in $A$ such that $\|a-p\|<2\e$, because $\sigma(a)\subset[0,\frac{1-\sqrt{1-4\e}}2)\cup(\frac{1+\sqrt{1-4\e}}2,1]$.

Second, if two projections $p$ and $q$ of $A$ satisfy $\|p-q\|<1$, then they are homotopic.
To prove this, with the complement of the non-commutative symmetric difference $v:=1-p-q+2qp$ in the standard unitization $\tilde A$, which satisfies $vp=qp=qv$, define a path $v_t:=(1-t)+tv$ connecting the unit and $v$ in $\tilde A^\times$, which is because
\[\|1-v_t\|=t\|1-v\|\le\|1-v\|=\|p+q-2qp\|=\|(1-2q)(p-q)\|\le\|p-q\|<1.\]
The polar decomposition for invertibles defines a path of unitaries $u_t:=v_t|v_t|^{-1}$ in $\tilde A$ from the unit to $u:=u_1$ such that $v^*vp=v^*qv=pv^*v$ implies $|v_t|p=p|v_t|$ and $upu^*=q$, and the norm continuity of the functional calculus on the annulus $\{\lambda\in\C:1-\|p-q\|\le|\lambda|\le2\}$ where the function $\lambda\mapsto\lambda|\lambda|^{-1}$ is uniformly approximated by polynomials implies that $u_t$ is a norm continuous path, so $p$ and $q$ are homotopic through the path $u_tpu_t^*$ of projections in $A$.


Third, if $p\in A=\colim_i A_i$ is a projection, then $\|p_i-p\|<\e$ for some $p_i\in A_i$.



(a)
If $p\sim_u q$, then there is a unitary $u$ such that $upu^*=q$, so $v:=up$ satisfies $p=v^*v$ and $q=vv^*$.


(b)
If $p\sim_h q$, then there is a norm-continuous path $p_t$ of projections such $p_0=p$ and $p_1=q$, so 

\end{pf}

\begin{prb}
We consider the following three categories.
$I(M_\infty(A))$, $P(M_\infty(A))$, $P(\cK\otimes A)$

Skolem-Noether theorem!! every automorphism of a matrix ring is inner, so we have a unique isomorphism $M_n(M_\infty(R))\to M_\infty(R)$ up to equivalence.

Why should we consider projection picture for C$^*$-algebras instead of idempotents?
Morphisms between projections are bounded.
Easily constructed by the functional calculus.

Why can we consider projection picture for C$^*$-algebras instead of idempotents?
If $e\in I(M_n(A))$, then $s_l(e)\in P(M_n(A))$, for unital $A$?
Polar decomposition?



\begin{enumerate}[(i)]
\item The category $P(M_\infty(A))$ of projections of $\cK\otimes A$ such that a morphism from $p$ to $q$ is defined by an element of $q(\cK\otimes A)p$.
\item The category $\mathrm{Hilb}_{\mathrm{proj,fg}}(A)$ of algebraically finitely generated projective Hilbert $A$-modules such that a morphism is defined by a $A$-adjointable operator.
\end{enumerate}
The forgetful functor from the category of algebraically finitely generated projective Hilbert $A$-modules to the category of algebraically finitely generated projective $A$-modules preserves isomorphic classes.

(In the category $P(\cK\otimes A)$, two projections are isomorphic if and only if they are Murray-von Neumann equivalent in $\cK\otimes A$.
Suppose two projections $p$ and $q$ in $\cK\otimes A$ have $x\in q(\cK\otimes A)p$ and $y\in p(\cK\otimes A)q$ such that $yx=p$ and $xy=q$.
Since $s_r(x)=s_r(xp)\le s_r(p)=s_r(yx)\le s_r(x)$ and $s_l(x)=s_l(qx)\le s_l(q)=s_l(xy)\le s_l(x)$ give $s_r(x)=p$ and $s_l(x)=q$, and it implies that $x$ and its adjoint in $B(\ell^2\otimes A)$ have complemented ranges because $s_l(x),s_r(x)\in M(\cK\otimes A)$, so the polar decomposition theorem for adjointable operators proves that $s_l(x)$ and $s_r(x)$ are von Neumann equivalent by a multiplier $v\in M(\cK\otimes A)$, where $v^*v=s_r(x)$ and $vv^*=s_l(x)$, but we automatically have $v\in\cK\otimes A$ since $v^*v=p\in\cK\otimes A$.
Conversely, a Murray-von Neumann equivalent projections of $\cK\otimes A$ are isomorphic in the category $P(\cK\otimes A)$ since if $p=v^*v$ and $q=vv^*$ for some $v\in\cK\otimes A$, then $(1-p)v^*v(1-p)=0$ implies $vp=v$ and similarly $v=qv$.)

We want to show a functor $p\mapsto p(\ell^2\otimes A)$ is an equivalence.

First, we prove the well-definedness.
Functoriality is clear.
Let $p\in\cK\otimes A$ be a projection.
We show that $p(\ell^2\otimes A)$ is a finitely generated projective algebraic right $A$-module.
For $\e\le12^{-1}$, fix a projection $q\in\cK\otimes\C1\subset\cK\otimes A$ such that $\|p-qpq\|<\e$ by considering the usual approximate unit consisting of finite-dimensional projections in $\cK$.
Since $0\le qpq\le1$ and $\|qpq-(qpq)^2\|<3\e\le4^{-1}$, there is a projection $r:=1_{(\frac12,1]}(qpq)\in\cK\otimes A$ such that $\|qpq-r\|<6\e$.
Then, $\|p-r\|<7\e\le1$ implies the existence of unitary $u\in M(\cK\otimes A)$ such that $p\sim uru^*$.
Thus, $p(\ell^2\otimes A)$ is isomorphic to $r(\ell^2\otimes A)$ as $A$-modules.
Since $r\le qpq\le q$, we have a direct sum decomposition $r(\ell^2\otimes A)\oplus(q-r)(\ell^2\otimes A)=q(\ell^2\otimes A)$ of a finitely generated free $A$-module, so $r(\ell^2\otimes A)$ and hence $p(\ell^2\otimes A)$ is a finitely generated projective $A$-module.
We can choose the inner product as the one naturally induced from $\ell^2\otimes A$.

Next, we prove the essential surjectivity.
Let $P$ be an algebraically finitely generated projective $A$-module.
Find a $A$-module $Q$ and $n\in\Z_{\ge0}$ such that $P\oplus Q\cong A^n$ as $A$-modules.
Take an idempotent $A$-module map $e$ on $A^n$ onto the image of $P$, and regard it as an element of $M_n(A)\subset B(A^n)$.
Then, $eA^n$ is isomorphic to $P$ as $B$-modules, and the idempotency of $e$ implies the closed range, so we can apply the polar decomposition of $e$ in the algebra of adjointable operators $B(A^n)$ to find a projection $p\in M_n(A)$ such that the ranges of $e$ and $p$ are equal.
Therefore, $P$ is isomorphic to $p(\ell^2\otimes A)$ as $A$-modules.

Finally, we prove the full faithfulness.
For projections $p,q\in\cK\otimes A$, since $x\in\cK\otimes A$ such that $xp=x=qx$ defines an $A$-adjointable operator $p(\ell^2\otimes A)\to q(\ell^2\otimes A)$, and a $A$-adjointable operator $p(\ell^2\otimes A)\to q(\ell^2\otimes A)$ gives rise to an element $x\in B(\ell^2\otimes A)=M(\cK\otimes A)$ such that $xp=x=qx$.
\end{prb}




\begin{prb}[Homotopy invariance]
Let $A$ and $B$ be C$^*$-algebras.
Two $*$-homomorphisms in $A\to B$ are said to be \emph{homotopic} if they are connected by a point-norm continuous path of $*$-homomorphisms $A\to B$.
\begin{parts}
\item For pointed compact Hausdorff spaces $(X,x_0),(Y,y_0)$, two pointed maps $f_0,f_1:X\to Y$ are homotopic if and only if $f_0^*,f_1^*:C_0(Y\setminus\{y_0\})\to C_0(X\setminus\{x_0\})$ are homotopic.
\item $A\to IA$ induces isomorphisms?
\end{parts}
\end{prb}
\begin{pf}
(a)
Suppose $f_0$ and $f_1$ are connected by a homotopy $f_t$.
Fixing $g\in C_0(Y\setminus\{y_0\})$ and $t_0\in I$, we want to show
\[\lim_{t\to t_0}\sup_{x\in X}|g(f_t(x))-g(f_{t_0}(x))|=0.\]
Since the function $g$ is uniformly continuous, with respect to an arbitrarily chosen uniformity on $Y$, so that there is an entourage $E\subset Y\times Y$ such that $(y,y')\in E\circ E$ implies $|g(y)-g(y')|<\e$.
Using compactness we have a finite sequence $(y_i)_{i=1}^n\subset Y$ such that for every $y$ there is $y_i$ satisfying $(y,y')\in E$.
Then, $f^{-1}(E[y_i])$ is a finite open cover of $X\times I$, so we have $\delta$ such that $|t-t_0|<\delta$ implies for any $x\in X$ the existence of $i$ satisfying $(f_t(x),y_i)\in E$ and $(f_{t_0}(x),y_i)\in E$, which deduces the desired inequality.

Conversely, suppose $f_0^*$ and $f_1^*$ are connected by a homotopy $f_t^*$.
By taking dual, we can induce $f_t:X\to Y$ such that $g(f_t(x))=(f_t^*g)(x)$ for each $g\in C(Y)$ from $f_t^*$ via the embedding $X\to M(X)$ by Dirac measures.
Let $V$ be an open neighborhood of $f_{t_0}(x_0)$ and take $g\in C(Y)$ such that $g(\f_{t_0}(x_0))=1$ and $g(y)=0$ for $y\notin V$.
Now we have an open neighborhood $U$ of $x_0$ such that $x\in U$ implies $|(f_{t_0}^*g)(x)-(f_{t_0}^*g)(x_0)|<\frac12$.
Also we have $\delta>0$ such that $|t-t_0|<\delta$ implies $\|f_t^*g-f_{t_0}^*g\|<\frac12$.
Therefore, $(x,t)\in U\times(t_0-\delta,t_0+\delta)$ implies $g(f_t(x))>0$, hence $f_t(x)\in V$, which means $X\times I\to Y:(x,t)\mapsto f_t(x)$ is continuous.
\end{pf}


\[K_0(\C)=\Z,\quad K_0(C_0(\R))=0,\quad K_1(C_0(\R))=K_0(C_0(\R^2))=\Z\]
\[K^0(*)=\Z,\quad K^0(S^1)=\Z,\quad K^1(S^1)=K^0(S^2)=\Z[x]/(x-1)^2\]




The K-theory with compact supports has
\[K_0(X)=K_0(X_+,*)=\tilde K_0(X_+)=K^0(C_0(X))...\]



\begin{prb}[Filtered cocontinuity and operator stability]

Since the Gronthendieck group completion is a left adjoint functor so that it is cocontinuous, we only need to show that the functor $V$ is filtered cocontinuous.

We need projection and unitary approximation in $M_n(A)$ by $M_n(A_i)$.

\end{prb}



\begin{prb}[Half-exactness and split-exactness]

\end{prb}













\section{First K-groups}

As the Whitehead group $K_1(A):=GL_\infty(A)/[GL_\infty(A),GL_\infty(A)]$.

relative



\section{Six-term exact sequences}


$\cK(\ell^2(\N))$, $\cI:=C([0,1])$, $\cS:=C_0(\R)$, $\cC:=C_0(\R_{\ge0})$

$K_{-n}(A):=K_0(\Sigma^nA)$.

How can we define $K_2$?


\begin{prb}[Index map]
In the theory of C$^*$-algebras, the connecting homomorphism $K_1(A/I)\to K_0(I)$ is usually called the index map.
Mayer-Vietoris
\end{prb}

\begin{prb}[Bott periodicty]

suspension picture $K_0\to K_1\Sigma$

Bott periodicity $K_1\to K_0\Sigma$

\end{prb}


$K(\C)\cong\Z[\beta^{\pm1}]$ as rings.

The suspension can defined by an exact sequence
\[0\to\Sigma B\to CB\to B\to0,\]
or alternatively by the pullback
\[\begin{tikzcd}[sep=small]
\Sigma B\rar\dar\pullback & CB\dar{b\mapsto b(1)} \\
CB \rar & B.
\end{tikzcd}\]

For $\f:A\to B$,
\[\cone(\f):=\{(a,b)\in A\oplus C_0((0,1],B):b(1)=\f(a)\},\qquad\cyl(\f):=\{(a,b)\in A\oplus C_0([0,1],B):b(1)=\f(a)\}.\]
The mapping cone is a homotopy cokernel geometrically and is a homotopy kernel algebraically.
The mapping cylinder is a fibrant? cofibrant? replacement.
The mapping cone can be defined by an exact sequence of C$^*$-algebras
\[0\to \cone(\f)\to \cyl(\f)\to B\to0,\]
or alternatively by the pullback
\[\begin{tikzcd}[sep=small]
\cone(\f)\rar\dar\pullback & CB\dar{b\mapsto b(1)} \\
A \rar & B.
\end{tikzcd}\]


We can see that $CB$ is contractible, and $\Sigma B$ is homotopic to the pullback $\cone(\f)\oplus_ACA$.

distinguished triangle
\[\Sigma B\to \cone(\f)\to A\xrightarrow{\f}B\]

Do not forget to describe the induced maps for K-groups!








local Banach algebras



\begin{prb}[Pimsner-Voiculescu exact sequence]
\end{prb}
Connes-Thom isomorphism


\section{Cuntz semigroup}

nuclear dimension





\begin{prb}[Ordered K-theory]
In operator K-theory, an \emph{ordered abelian group} usually means a directed partially ordered abelian group, which is equivalent to an abelian group $K$ with a subset $P\subset K$ such that
\[P-P=K,\qquad P\cap-P=\{0\},\qquad P+P\subset P.\]
For a partially ordered abelian group $K$ with an order unit $e$, a \emph{state} on $K$ is a positive homomorphism $K\to\R$ which maps $e$ to one.
The set of states is endowed with the product topology which is compact and convex.

only for stably finite unitals?
\end{prb}




\chapter{Elliott program}


\section{Elliott invariant}
Let $A$ be a unital C$^*$-algebra.
\[(K_0(A),K_1(A),T(A),[1]_0,T(A)\times K_0(A)\to\R)\]
\begin{itemize}
\item $T(A)$ is the compact convex set of tracial states endowed with the weak$^*$ topology,
\item $[1]_0$ is the K$_0$-class of the unit
\item $T(A)\to S(K_0(A))$ is the evaluation, which is known to be surjective if $A$ is unital and exact.
\end{itemize}

Glimm's classification of UHF algebras
Bratteli diagram
Elliott's intertwining argument


\begin{prb}
Let $A$ be a C$^*$-algebra.
We say a projection is
\begin{enumerate}[(i)]
\item \emph{infinite} if it is Murray-von Neumann equivalent to a proper subprojection,
\item \emph{finite} if it is not infinite,
\item \emph{properly infinite} if every non-zero quotient projection is infinite,
\item \emph{purely infinite} if every non-zero subprojection is infinite.
\end{enumerate}
Here, a quotient projection refers to the image of a projection in a quotient.
These definitions do not change in the unitization $\tilde A$ when $A$ is non-unital.

Suppose $A$ is unital.
We say $A$ is \emph{stably finite} if every projection of $A\otimes\cK$ is finite. (II$_1$)
If $A$ is stably finite, then $T(A)\ne\varnothing$ so that we have $K_0(A)$ a pointed directed partially ordered abelian group.

stable rank one, real rank zero

\begin{parts}
\item $p$ is properly infinite if and only if $p$ has two orthogonal subprojections Murray-von Neumann equivalent to $p$.
\item Every unital properly infinite C$^*$-algebra does not admit a tracial state.
\item Every unital stably finite exact C$^*$-algebra admits a tracial state. (Blackadar-Handelman-Haagerup)
\end{parts}
\end{prb}

\begin{prb}[AF-algebras]
An \emph{AF-algebra} or an \emph{approximately finite-dimensional C$^*$-algebra} is a C$^*$-algebra given by the filtered colimit of finite-dimensional C$^*$-algebras.

A \emph{Bratelli diagram} is a sequence of finite bipartite multi-graphs $(V_n,V_{n+1})$.
Bratelli proved that an equivalence class of Bratelli diagrams is a complete invariant of separable unital AF-algebras.

A \emph{UHF-algebra} or a \emph{uniformly hyperfinite algebra} is a separable unital AF-algebra described by a linear Bratelli diagram.
Glimm proved that a super-natural number is a complete invariant of UHF-algebras.

Let $A$ be a C$^*$-algebra.
We say $A$ has the \emph{cancellation property} if the monoid $V(A)=P(\cK\otimes A)/\sim$ is cancellative.

The image of the dimension function $P(A)\to K_0(A)$ is called the \emph{dimension range} of $A$.
\begin{parts}
\item For a separable AF-algebra $A$, there is an increasing sequence of finite-dimensional unital C$^*$-subalgebras whose union is dense.
\item It is known that every AF-algebras has stable rank one and real rank zero.
\item permanence properties

\item Let $F$ and $B$ be C$^*$-algebras such that $F$ is finite-dimensional and $B$ is unital with the cancellation property. If a group homomorphism $K_0(F)\to K_0(B)$ preserves dimension ranges, then 

If $A$ is unital AF, then the dimension range is the set of positive elements majorized by $[1]$, so unital AF-algebras
\end{parts}
\end{prb}
\begin{pf}
\end{pf}




locally finite-dimensional:
there is a net of (contractive) linear maps $A\to A$ whose image is contained in a finite-dimensional C$^*$-subalgebra of $A$ convergent to the identity map in the point-norm topology.

locally nulcear:
there is a net of (contractive) linear maps $A\to A$ whose image is contained in a nuclear C$^*$-subalgebra of $A$ convergent to the identity map in the point-norm topology.

Q. can we assume the maps are c.p.? difference between nuclearity and exactness?

principle of local reflexivity
there is a net of (contractive) linear maps $X^{**}\to X^{**}$ whose image is contained in $X$ convergent to the identity map in the point-weak$^*$ topology.

locally reflexive:
there is a net of contractive completely positive maps $F\subset A^{**}\to A^{**}$ whose image is contained in $A$ convergent to the inclusion in the point-weak$^*$ topology.

for every finite-dimensional subspace $F$, the inclusion is approximated by maps with some property in $L(F,A)$?

independent of the choice of $A\subset B(H)$ or $A^{**}\subset B(H)$?




\section{Universal coefficient theorem}

Quasi-diagonality

\begin{prb}[Weyl-von Neumann theorem]
A self-adjoint bounded operator is quasi-diagonal.
\end{prb}

\begin{prb}[Glimm lemma]
If a state $\omega$ of $B(H)$ vanishes on $K(H)$, then it is a weak$^*$ limit of vector states.
\end{prb}

\begin{prb}[Voiculescu theorem]
\end{prb}


\begin{prb}[Quasi-diagonal algebras]
An operator $a\in B(H)$ is called \emph{quasi-diagonal} if there is a net of projections $p_i\in B(H)$ such that $[p_i,a]\to0$ in norm and $p_i\uparrow\id_H$ strongly.
A C$^*$-algebra is called \emph{quasi-diagonal} if it admits a faithful representation whose image is quasi-diagonal.
\end{prb}

faithful non-degenerate essential representations of a quasi-diagonal C$^*$-algebra are all quasi-diagonal

locally quasi-diagonal





\section{Jiang-Su stability}
nuclear dimension


Toms-Winter conjecture
strongly self-absorbing




successful in Kirchberg algebras


https://arxiv.org/pdf/2307.06480.pdf

Elliott classification problem
Kirchberg-Phillipes theorem

operator K-theory and its pairing with traces

$\cZ$-stability, Rosenberg-Schochet universal coefficient theorem

Connes-Haagerup classification of injective factors

Kirchberg: unital simple separable $\cZ$-stable algebra is either purely infinte or stably finite.
Haagerup, Blackadar, Handelman: unital simple stably finite algebra has a trace.

Glimm: uniformly hyperfinite algebras
Murray-von Neumann: hyperfinite II$_1$ factors





\section{C$^*$-dynamics}

Izumi-Matui
Rokhlin property
Evans-Kishimoto intertwining argument
dynamical Kirchberg-Phillips
Tikusis-White-Winter







\chapter{Purely infinite algebras}

\section{Kirchberg embeddings}
A separable C$^*$-algebra is exact if and only if it is embedded in $\cO_2$.

Every Kirchberg algebra is $\cO_\infty$-stable.

\chapter{Stably finite algebras}





\end{document}