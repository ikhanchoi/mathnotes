\documentclass{../../large}
\usepackage{../../ikhanchoi}


\begin{document}
\title{Algebraic quantum field theory}
\author{Ikhan Choi}
\maketitle
\tableofcontents



\chapter{}

Axiomatic: Osterwalder-Schrader, Wightman, Haag-Kastler

CFT

Statistical physics: Gibbs state by DLR equation, Lieb-Robinson bound, quantum theory

\section{}

\begin{prb}[Geometric quantization]
On a closed symplectic manifold $(X,\omega)$, fix a \emph{pre-quantum line bundle} on $X$, which is a structured line bundle $(L,\nabla)\to X$ such that the curvature is the symplectic form $\omega$.
Define the \emph{pre-quantum operator} $Q:C^\infty(X)\to\End(\Gamma^\infty(L))$ such that
\[Q(f):=-i\nabla_{X_f}+f,\]
where $X_f$ is the Hamiltonian vector field associated to $f\in C^\infty(X)$ with respect to the symplectic structure.
polarization and metaplectic correction to construct the Hilbert space.


I think geometric quantizatioin does not work for field theory in general, and the Chern-Simons theory is the one very exceptional.
\end{prb}

\begin{prb}[Deformation quantization]
\end{prb}

\begin{prb}[Path integral quantization]
It generates a FQFT as follows.
If we describe the Hilbert space $H$ in FQFT as function spaces to describe the propagator corresponded to the world-volume $\Sigma$ as a integral operator, we want to compute the kernel in terms of action functional as
\[k(x,y)=\int_{\mathrm{Conf}_\Sigma}[D\f]e^{iS(\f)}.\]

For almost cases(perhaps), it is done by $\sigma$-models, in which we set the configuration as $\mathrm{Conf}_\Sigma:=\mathrm{Map}(\Sigma,X)$ of maps from a world-volume $\Sigma$ to the target space $X$.
Each function $f\in C^\infty(X)$ defines an operator
\end{prb}


\begin{prb}
Action functionals
\begin{parts}
\item Classical mechanics: Let $\mathrm{Conf}_\Sigma:=C^\infty(\Sigma,X)$, where $\Sigma=[t_i,t_f]$ and $X=T^*\R^3=T\R^3$. For a given background potential $V:X\to\R$, we can define the action $S(\f):=\int_\Sigma L(t,\f,\dot\f)\,dt$, where $L(t,x,v):=\frac m2\|v\|^2+V(x)$, $m>0$ is the mass parameter.
\item General relativity: the configuration space is the space of pseudo-metrics, Einstein-Hilbert action
\item Electromagnetism and Yang-Mills theory: the configuration space is the space of principal bundles with connections, and the action is $S(A):=\frac12\<F,F\>$.
\item Chern-Simons theory: the configuration space is the space of principal bundles with connections on three-dimensional manifolds, and the action is $S(\omega):=\mathrm{CS}_Q(\omega)$, where $Q$ is a symmetric polynomial in Chern-Weil theory.
\end{parts}
Except classical mechanics, $\Sigma$ is the space-time.

In principle, the configuration space is the section space of a fiber bundle, and the Lagrangian is a map from the configuration space to the space of densities, which defines an action functional.
\end{prb}


\section{}

Consider a Poincar\'e equivariant topological left(really?) module $H=L^2(H_m)$ over a commutative algebra of the Minkowski space $A=\cS(\R_x^{1,d-1})$.
We have a partially defined map $A\to H$ (fourier restriction).


An AQFT on Minkowski space can be seen as a causal conical cyclic Poincar\'e covariant representation $\phi:H\to B(FH)$ of the left $A$-module $H$.

For an ideal $I$ supported on $O\subset\R_x^{1,d-1}$ of $A$, we can produce a smaller $A$-module $IH$ and the algebra generated by $IH$ in $B(FH)$ can be considered.



\begin{prb}[Wightman axioms]
Let $\R_x^{1,d-1}$ be the Minkowski space and $\cP_+^\uparrow$ the connected component of the Poincar\'e group.
A \emph{Wightman field} is a linear map $\phi:\cS(\R_x^{1,d-1})\to\End(\cD)$, where $\cD$ is an inner product space with completion $\cH$, such that
\begin{enumerate}[(i)]
\item Covariance: there is a representation $U:\cP_+^\uparrow\to U(\cH)$ such that $\Ad U(\gamma)\phi(f)=\phi(\gamma^*f)$,
\item Causality: if the supports of $f$ and $g$ are space-like separated, then $[\phi(f),\phi(g)]=0$ on $D$,
\item Conicality: 
\item Cyclicity: there is a Poincar\'e invaraint cyclic vector $\Omega$ in the sense that the span of the set $\{\phi(f_1)\cdots\phi(f_n)\Omega\}$ is dense in $\cD$.
\end{enumerate}
\end{prb}


\begin{prb}[Free massive bosonic fields]
Let $m>0$, called the mass of a scalar particle, and let $d=1+1$ be a positive integer, called the dimension.
Note $\cP_+^\uparrow=\R^{1,1}\rtimes\R$.
On the mass shell $H_m:=\{p\in\R^{1,1}_p:(p,p)=p_0^2-p_1^2=m^2,\ p_0>0\}$, the induced metric is Riemannian with the volume form $(p_1^2+m^2)^{-\frac12}dp_1$, so we can define $L^2(H_m)$.

For $f\in\cS(\R^{1,1}_x)$, consider the restriction of the Fourier transform $\hat f\in L^2(H_m)$.
\[\hat f(p)=\int_{\R^{1,1}}e^{i(x,p)}f(x)\,d^2x,\qquad p\in H_m,\]
where $d^2x$ is the Lebesgue measure on $\R^{1,1}$, which is Lorentz invariant.
Via the Bosonic Fock space construction $\cF^+(L^2(H_m))$, we define a operator-valued distribution
\[\phi:f\mapsto a^\dagger(\hat f)+a(\hat f),\]
$\phi(f)$ is defined densely on $\cF^+(L^2(H_m))$.


\begin{parts}
\item $\phi$ is covariant.
\item $\phi$ is local.
\item $\phi$ has positive energy.
\item $\phi$ admits a vaccum.
\item $\phi$ has linear energy bound. In particular, it defines a Araki-Haag-Kastler net.
\end{parts}
\end{prb}
\begin{pf}
(a)
Consider a representation $U_m:\cP_+^\uparrow\to U(L^2(H_m))$ on $L^2(H_m)$, defined by
\[(U_m(a,\Lambda)\Psi)(p):=e^{i(a,p)}\Psi(\Lambda^{-1}p),\qquad(a,\Lambda)\in\cP_+^\uparrow,\ \Psi\in L^2(H_m).\]
The action $U_m:\cP_+^\uparrow\to U(L^2(H_m))$ is extended to $\Gamma(U_m):\cP_+^\uparrow\to U(\cF^+(L^2(H_m)))$, called the second quantization.
Then, since $\cF(a,\Lambda)\cF^{-1}$ maps
\[(p\mapsto\int e^{i(x,p)}f(x)\,d^2x)\quad\text{ to }\quad(p\mapsto\int e^{i(x,p)}f(\Lambda^{-1}(x-a))\,d^2x=e^{i(a,p)}\int e^{i(x,\Lambda^{-1}p)}f(x)\,d^2x),\]
so it is covariant.

(b)
Define the left and right wedges
\[W_L:=\{x\in\R^{1,1}:|x_0|\le-x_1\},\qquad W_R:=\{x\in\R^{1,1}:|x_0|\le x_1\}.\]
Suppose $f$ and $g$ are Schwartz functions supported on $W_L$ and $W_R$ respectively.
Write
\[[\phi(f),\phi(g)]=[a^\dagger(\hat f),a(\hat g)]+[a(\hat f),a^\dagger(\hat g)].\]
analytic continuation and residue theorem...
If $f(x)\ne0$ and $g(y)\ne0$, then $x$ and $y$ are contained in the interior of $W_L$ and $W_R$, so $(x,y)>0$.

\end{pf}


What is the interactions?


Conformal nets, vertex operator algebras, and Segal's picture.
Factorization algebra?




\section{}

A stochastic process is a family of $*$-homomorphisms $\f_t:A\to(N,\tau)$ indexed by $t$.

A stochastic process $\{\f_t\}$ is called \emph{Markov} if there is a unital positive linear semi-group action $\alpha$ on $A$ such that $\f_t(a)=\f_0(\alpha_t(a))$.

For a unital positive linear semi-group action $\alpha$ on $A$ and an initial condition in $A^*$, then we can construct a Markov process $\f_t:A\to(N,\tau)$.
The algbera $N$ does not depend on the initial condition, but the trace $\tau$ is determined by the initial condition.



\end{document}







