\documentclass{../../large}
\usepackage{../../ikhanchoi}

\renewcommand{\sf}{\operatorname{sf}}

\begin{document}
\title{Non-commutative geometry}
\author{Ikhan Choi}
\maketitle
\tableofcontents



\part{Non-commutative spaces}





\chapter{Bivariant K-theory}



\section{Higson characterization}

\begin{prb}[Bivariant K-theory]
According to Higson, a \emph{bivariant K-theory} is defined as the initial homotopy invariant, oeprator stable, split exact functor $\mathrm{kk_0}:\mathrm{C^*Alg}_{(\mathrm{sep})}\to\mathrm{KK}_{0,(\mathrm{sep})}$ to an additive category.
The additive category $\mathrm{KK}_{0,(\mathrm{sep})}$ is called the \emph{Kasparov category}.

Consider separable nuclear C$^*$-algebras
\[\cI:=C([0,1]),\qquad\cK=K(\ell^2),\qquad\cS:=C_0(\R).\]
For a functor from $\mathrm{C^*Alg}_{(\mathrm{sep})}$, the homotopy invariance means that the constant function $*$-homomorphism $A\to\cI\otimes A$ is mapped to an isomorphism, and the \emph{operator stability}, \emph{matrix stability}, or just the \emph{stability} means that the corner embedding $*$-homomorphism $A\to\cK\otimes A$ is mapped to an isomorphism, for all (separable) C$^*$-algebras $A$.
The operator stability means the stability under the stabilization functor $\cK\otimes-$, whose term comes from the stable equivalence of vector bundles.
See the standard projection picture of the operator K-theory.

Recall that the tensor products $\otimes_{\max}$ and $\otimes_{\min}$ define symmetric monoidal structures on $\mathrm{C^*Alg}_{(\mathrm{sep})}$.
Each tensor product induces a corresponding symmetric monoidal structure on the Kasparov category such that the bivariant K-theory is monoidal(?).
\end{prb}


\begin{prb}
The functor $\Sigma:\mathrm{KK}_{0,(\mathrm{sep})}^\op\to\mathrm{KK}_{0,(\mathrm{sep})}^\op$ defined as the opposite functor $\Sigma:=(\mathrm{kk}_0(\cS)\otimes-)^\op=\mathrm{kk}_0(\cS\otimes-)^{\op}$ induced by the nuclear C$^*$-algebra $\cS$ is called the \emph{suspension}, and Meyer-Nest showed it defines a triangulated category structure on the opposite Kasparov category $\mathrm{KK}_{0,(\mathrm{sep})}^\op$(?).

Recall that the category $\mathrm{CH}_*$ of pointed compact Hausdorff spaces is triagulated and there is an equivalence $\mathrm{CH}_*\to\mathrm{CC^*Alg}^\op$ with the category of commutative C$^*$-algebras.
The functor
\[\mathrm{CH}_*\to\mathrm{C^*Alg}^\op\to\mathrm{KK}_0^\op\]
preserves the triangulated category structure(?).
\end{prb}

Bott periodicity can be proved by abstract non-sense?
Cuntz showed $\mathrm{kk}_0(\cT_0)\cong0$, where $\cT_0$ is the \emph{non-unital Toeplitz algebra}, together with an exact sequence
\[0\to\cK\to\cT_0\to\cS\to0(?).\]

\section{Kasparov picture}



\begin{itemize}
\item Kasparov-Stinespring theorem
\item Kasparov-Voiculescu theorem
\item Kasparov-Weyl-von Neumann theorem
\end{itemize}

Super:
\begin{itemize}
\item positive/negative: convenient to use $\pm$ for super-modules
\item even(zero)/odd(one): convenient in the Koszul sign rules by the ring structure of degree, and conventional adjectives in super-algebras
\end{itemize}


\begin{prb}[Kasparov modules]
Let $A$ and $B$ be C$^*$-algebras.
We regard $A$ and $B$ to have trivial $\Z/2\Z$-gradings.
Note that a $\Z/2\Z$-grading is techinically same as a $\Z/2\Z$-action.
An \emph{even Kasparov module} or simply a \emph{Kasparov module} over $(A,B)$ is a pair $(E,F)$ consisting of
\begin{enumerate}[(i)]
\item a countably generated right Hilbert module $E$ over $B$ together with a $*$-homomorphism $A\to B(E)$,
\item a bounded linear operator $\textstyle F=\left(\begin{smallmatrix}0&F_-\\F_+&0\end{smallmatrix}\right)$ on $E$ such that
\begin{enumerate}[(\text{ii}.i)]
\item $F\in B(E)$,
\item $[F,a]\in K(E)$ for $a\in A$,
\item $a(F-F^*)\in K(E)$ for $a\in A$,
\item $a(F^2-1)\in K(E)$ for $a\in A$,
\end{enumerate}
\item a $\Z/2\Z$-grading on $E$ such that the bimodule actions by $(A,B)$ are even and the operator $F$ is odd.
\end{enumerate}
An \emph{odd Kasparov module} is an even Kasparov module with trivial $\Z/2\Z$-grading.

Let $(E_0,F_0)$ and $(E_1,F_1)$ be even Kasparov modules over $(A,B)$.
A \emph{unitary equivalence} between them is an even $A$-linear $B$-adjointable isometric isomorphism $u:E_0\to E_1$ with $\Ad u(F_0)=F_1$.
A \emph{homotopy} between them is an even Kasparov module $(E,F)$ over $(A,C([0,1],B))$ such that there are unitary equivalences $(E(0),F(0))=(E_0,F_0)$ and $(E(1),F(1))=(E_1,F_1)$, where $(E(t),F(t)):=(E\otimes_{\delta_t}B,F\otimes1)$ and $\delta_t:C([0,1],B)\to B:b\mapsto b(t)$ for each $t\in[0,1]$.
We can check $(E(t),F(t))$ is an even Kasparov module over $(A,B)$ with simple computations.
A homotopy $(E,F)$ between $(E_0,F_0)$ and $(E_1,F_1)$ is particularly called an \emph{operator homotopy} if $E=E_0[0,1]$ with $E_0=E_1$, $t\mapsto a(t)$ is constant, and $t\mapsto F(t)$ is norm continuous.

We can take a representative $F$ being self-adjoint because $\left(\begin{smallmatrix}0&F_-\\F_+&0\end{smallmatrix}\right)$ and $\left(\begin{smallmatrix}0&F_+^*\\F_+&0\end{smallmatrix}\right)$ defines operator homotopic Kasparov modules.
\begin{parts}
\item compact perturbation
\item operator homotopy
\item degenerate
\end{parts}
\end{prb}
\begin{pf}
(b)



(c)
Let $(E_0,F_0)$ be a degenerate Kasparov cycle from $A$ to $B$.
Define a Kasparov cycle $(E,F)$ from $A$ to $B[0,1]$ such that $E:=E_0[0,1)$

($C_b([0,1),K(E_0))\not\subset K(E_0[0,1))$ in general.)

(d)
\end{pf}


\begin{prb}[Equivariant Hilbert super-bimodules]
Let $A$ and $B$ be C$^*$-algebras with actions of a locally compact group $G$.
A \emph{Hilbert bimodule} is a Banach bimodule over $(A,B)$ such that the norm is induced from a $B$-valued inner product for which the left action of $A$ is adjointable.
A Hilbert bimodule is sometimes called a \emph{correspondence}.
We say a Hilbert super-bimodule $E$ over $(A,B)$ is \emph{equivariant} if it consists of a strongly continiuous even action $u:G\to L(E)$ on $E$ such that
\[u_s(a\xi)=\alpha_s(a)u_s(\xi),\qquad\beta_s(\<\eta,\xi\>)=\<u_s(\eta),u_s(\xi)\>,\qquad
\begin{gathered}
a\in A,\ b\in B,\\s\in G,\ \xi,\eta\in E.
\end{gathered}\]
Note that it follows $u_s(\xi b)=u_s(\xi)\beta_s(b)$ automatically.
It generalizes covariant representations of $A$ and equivariant Hilbert modules over $B$, with additional $\Z/2\Z$-grading.
An equivariant super-correspondence is in fact technically the same as the equivariant correspondence over $G\times\Z/2\Z$, in which the action of $\Z/2\Z$ corresponds to the parity operator $\gamma:\xi_\pm\mapsto\pm\xi_\pm$ given by the grading, but we will not consider it as an action.
We also avoid omitting parantheses in the notation $u_s(\xi)$ because the group action $u$ is not $B$-linear in general unless the group action on $B$ is trivial.
One can check that for an equivariant super-correspondence $E$ from $A$ to $B$ the adjoint action $\Ad u$ acts continuously on $K(E)$ and strictly continuously on $B(E)$.
\begin{parts}
\item If $E$ is an equivariant super-correspondence from $A$ to $B$, then $(L^2(G)\otimes E,\lambda\otimes u)$ is naturally an equivariant super-correspondence from $A$ to $B$.
If $E$ is faithful, non-degenerate, and full, then so is $L^2(G)\otimes E$, respectively.
Fell's absorption.
\item interior tensor product and coalgebra structure from the group...
\end{parts}
\end{prb}
\begin{pf}
(a)
\iffalse
(Faithfulness)
Suppose $a\xi=0$ for all $\xi\in L^2(G)\otimes E$.
Then, for $f\otimes\xi_0\in C_c(G)\otimes E$,
\[0=(a(f\otimes\xi_0))(t)=f(t)\otimes(\alpha_t^{-1}(a)\xi_0)\]
implies $f(e)\otimes(a\xi_0)=0$ by putting $t=e$, so $a\xi_0=0$ and $a=0$.

(Fullness)
Because a Hilbert module is full iff the right action is faithful, we can prove it in a similar way to faithfulness of the left action.

(Non-degeneracy)
If $e_i\in A$ is a quasi-central approximate unit such that $\alpha_t(e_i)-e_i\to0$ in $A$ compactly on $G$ (it can be shown whithout the condition that $A$ is $\sigma$-unital, Lemma 2.12 of Ozawa), then
\[(e_i\xi-\xi)(t)=(\alpha_t^{-1}(e_i)-1)\xi(t)=(\alpha_t^{-1}(e_i)-e_i)\xi(t)+(e_i-1)\xi(t)\]
\begin{align*}
|\xi-e_i\xi|^2
&=\int_G\beta_t(|((1-e_i)\xi)(t)|^2)\,dt\\
&=\int_G\beta_t(|(1-\alpha_t^{-1}(e_i))\xi(t)|^2)\,dt\\
&\le2\int_G\beta_t(|(1-e_i)\xi(t)|^2+|(e_i-\alpha_t^{-1}(e_i))\xi(t)|^2)\,dt\to0
\end{align*}
taking compact set outside which we have $\|\xi\|<\e$.
\fi


Define a super-correspondence $(L^2(G)\otimes E,\lambda\otimes u)$ from $A$ to $B$ such that the correspondence structure is defined as it is
\[(a\xi)(t):=a\xi(t),\quad(\xi b)(t):=\xi(t)b,\quad\<\eta,\xi\>=\int_G\<\eta(t),\xi(t)\>\,dt,\]
which is equivariant since
\begin{align*}
((\lambda_s\otimes u_s)(a\xi))(t)
&=u_s((a\xi)(s^{-1}t))
=u_s(a\xi(s^{-1}t))
=\alpha_s(a)u_s(\xi(s^{-1}t))\\
&=\alpha_s(a)((\lambda_s\otimes u_s)(\xi))(t)
=(\alpha_s(a)(\lambda_s\otimes u_s)(\xi))(t),\\
((\lambda_s\otimes u_s)(\xi b))(t)
&=u_s((\xi b)(s^{-1}t))
=u_s(\xi(s^{-1}t)b)
=u_s(\xi(s^{-1}t))\beta_s(b)\\
&=((\lambda_s\otimes u_s)(\xi))(t)\beta_s(b)
=((\lambda_s\otimes u_s)(\xi)\beta_s(b))(t),\\
\beta_s(|\xi|^2)
=\int_G\beta_s(|\xi(t)|^2)\,dt
&=\int_G|(u_s(\xi))(t)|^2\,dt
=\int_G|((\lambda_s\otimes u_s)(\xi))(t)|^2\,dt
=|(\lambda_s\otimes u_s)(\xi)|^2.
\end{align*}

Fell absorption:
Define a super-correspondence $(L^2(G)\otimes_\beta E_0,\lambda\otimes1)$ from $A$ to $B$ such that
\[(a\xi_0)(t):=\alpha_t^{-1}(a)\xi_0(t),\quad
(\xi_0b)(t):=\xi(t)\beta_t^{-1}(b),\quad
\<\eta_0,\xi_0\>:=\int_G\beta_t(\<\eta_0(t),\xi_0(t)\>)\,dt,\]
which can be also shown to be equivariant.
We have an analogue of the Fell absorption in the sense that there is an equivariant even $A$-linear $B$-adjointable isometric isomorphism $w:L^2(G)\otimes_\beta E_0\to L^2(G)\otimes E$ defined such that $(w\xi_0)(t):=u_t(\xi_0(t))$.


\end{pf}


\begin{prb}[Continuous fields of super-correspondences]
Let $A$ and $B$ be $C_0(X)$-algebras, where $X$ be a locally compact Hausdorff space.
Take notice that $X$ should not be interpreted as a pointed compact Hausdorff space when we consider $C_0(X)$-algebras.
We say a super-correspondence $E$ from $A$ to $B$ is said to be \emph{over $C_0(X)$} if $f\xi=\xi f$ for $f\in C_0(X)$ and $\xi\in E$.
For equivariant versions for which $X$ is a $G$-space, we do not require the compatibility of $G$ and $C_0(X)$ on $E$, because it is satisfied automatically.





\begin{parts}
\item 
\item Define $B[0,1]:=B\otimes C([0,1])$ and $E[0,1]:=E\otimes_BB[0,1]$. We have a natural isomorphisms
\begin{gather*}C([0,1],B)=B[0,1],\quad C([0,1],E)=E[0,1],\\
C([0,1],K(E))=K(E[0,1]),\quad C([0,1],B(E)_{\mathrm{strict}})=B(E[0,1])
\end{gather*}
as C$^*$-algebras, and in the first three the identifications are equivariant.
If $F\in C([0,1],B(E)_{\mathrm{norm}})$ and $F(t)$ is $G$-continuous for each $t\in[0,1]$, then $F$ is $G$-continuous in $B(E[0,1])$. The evaluation maps are all well-defined.
\item For a $C_0(X)$-algebra $A$, there exists a faithful non-degenerate correspondence $E$ from $A$ to some $C_0(X)$-algebra $B$.
\item tensor products of $C_0(X)$-C$^*$-algebras
\end{parts}
\end{prb}
\begin{pf}

(b)
Two C$^*$-algebras $C_0(X,B)$ and $B\otimes C_0(X)$ have a common dense $*$-subalgebra $B\odot C_0(X)$, and the induced C$^*$-norms coincide by the nuclearity of $C_0(X)$, so the identity on the dense $*$-algebra extends to a $*$-isomorphism between the two C$^*$-algebras.

Two Banach spaces $C_0(X,E)$ and $E\otimes_BC_0(X,B)$ have a common dense $*$-subalgebra $E\odot_BC_0(X,B)$, and the induced norms are given by
\[\|\sum_i\xi_i\otimes b_i\|^2_{C_0(X,E)}=\sup_{x\in X}\|\sum_{i,j}b_j^*(x)\<\xi_j,\xi_i\>b_i(x)\|_B=\|\sum_i\xi_i\otimes b_i\|^2_{E\otimes_BC_0(X,B)},\]
so the $B$-adjointable isometric isomorphism.

For $C_0(X,K(E))=K(C_0(X,E))$ and $C_0(X,B(E)_{\mathrm{strict}})=B(C_0(X,E))$, we consider a dense $*$-subalgebra $\theta_{E\odot C_0(X)}$.
For $T:=\sum_i\theta_{\xi_i\otimes f_i,\eta_i\otimes g_i}$, norms from $C_0(X,K(E))$, $K(C_0(X,E))$ and semi-norms from $C_0(X,B(E)_{\mathrm{strict}})$, $B(C_0(X,E))$ are
\[\sup_{x\in X}\sup_{\xi\in E_1}\|T(x)\xi\|_E,\qquad\sup_{\xi\in C_0(X,E)_1}\sup_{x\in X}\|(T\xi)(x)\|_E,\]
\[\sup_{x\in X}\|T(x)\xi\|_E,\quad\xi\in E,\qquad
\sup_{x\in X}\|(T\xi)(x)\|_E,\quad\xi\in C_0(X,E).\]
(For the last two, we omit the seminorms associated to adjoints.)
One can see that the first two are equal by temporarily fixing points $x\in X$.
The last two families of semi-norms are same but the generating vectors are different as $E$ and $C_0(X,E)$, so in order to extend the convergence for $\xi\in C_0(X,E)$ from $\xi\in E\odot C_0(X)$, we can use the boundedness of a strictly convergent net so that the two topologies are same on the bounded part.
Note that the norm topology on $B(E)$ cannot make use of the density of an appropriate $*$-subalgebra.


(c)
We will choose $B=C_0(X)^{**}$. ($C_0(X)^{**}$ is not a $C_0(X)$-algebra...)
Fix a state $\omega$ on $A$.
Since $C_0(X)^{**}\subset Z(A^{**})$, there is a conditional expectation $\f:A^{**}\to C_0(X)^{**}$, which factors through $\omega^{**}=\omega^{**}\f$ because $C_0(X)^{**}\subset Z(A^{**})$ is unital.
Since $\f$ is completely positive, the Stinespring construction on $A\odot C_0(X)$ gives rise to a C$^*$-correspondence $E_\omega$ from $A$ to $C_0(X)^{**}$.
Define $E:=\bigoplus_{\omega\in S(A)}E_\omega$.
If $a\in A$ acts trivially on $E$, which means $\f(a^*a)=0$ and $\omega(a^*a)=0$.
Thus $A$ acts failfully on $E$.

For equivariant version, first take $A\to B(E_0)$.
Define $A\to B(L^2(G)\otimes_\beta E_0)$ such that
\[(a\xi_0)(t):=\alpha_t^{-1}(a)\xi_0(t).\]
Using the Fell absorption $B(L^2(G)\otimes_\beta E_0)=B(L^2(G)\otimes E)$, we have $A\to B(L^2(G)\otimes E)$ such that
\[(a\xi)(t)=u_t(\alpha_t^{-1}(a)\xi(t)).\]
Note that $A\to B(E_0)$ is not equivariant, so
\[u_t(\alpha^{-1}(a)\xi(t))\ne au_t(\xi(t)).\]

\end{pf}

\begin{prb}[Equivariant Kasparov modules]

$a[u_s,F]\in K(E)$ for $a\in A$ and $s\in G$, and $aF\in B(E)$ is $G$-continuous for $a\in A$,
\end{prb}


\begin{prb}[Functorial properties of KK-theory]
The set of homotopy classes of countably generated Kasparov modules is denoted by $KK^G(A,B)$.
The set theoretic issue does not occur because we only consider countably generated correspondences.

\begin{parts}
\item $KK^G(A,B)$ is an abelian group.
\item $KK^G$ is a homotopy invariant bivariant functor.
\item $KK^G$ is split-exact.
\item $KK^G$ is stable.
\end{parts}
\end{prb}
\begin{pf}
(a)
well-definedness

associativity: clear

identity: clear

inverse:
Let $(E,F)$ be a Kasparov cycle from $A$ to $B$.
Let $U\in B(E,-E)^{\mathrm{odd}}$ be the identity operator.
Note that $[U,a]=0$ and $\Ad u_s(U)=U$ for $a\in A$ and $s\in G$.
We prove that $-(E,F):=(-E,-UFU^*)$ is the inverse.
Consider
\[\bar E:=(E\oplus-E)[0,1],\qquad \bar F(t):=\mat{c(t)F&s(t)U^*\\s(t)U&-c(t)UFU^*}\in B(E\oplus-E),\qquad t\in[0,1],\]
where $c(t):=\cos\frac\pi2t$ and $s(t):=\sin\frac\pi2t$, with an identification $\bar F\in B(\bar E)$ obtained from the norm continuity of $\bar F:[0,1]\to B(E\oplus-E)$.
If we prove $(\bar E,\bar F)$ is a Kasparov cycle from $A$ to $B[0,1]$, then it becomes an operator homotopy between $(E\oplus-E,F\oplus-UFU^*)$ and a degenerate Kasparov cycle.
Since $\bar E$ is clearly a countably generated super-correspondence, it suffices to check $\bar F$ satisfies the conditions in the definition of Kasparov cycles.

(b)

Suppose $\f_0,\f_1:A\rightrightarrows A'$ are homotopic.
We calim $\f_0^*,\f_1^*:KK^G(A',B)\rightrightarrows KK^G(A,B)$ are equal.

Suppose $\psi_0,\psi_1: B\rightrightarrows B'$ are homotopic.
We will show $\psi_{0*},\psi_{1*}:KK^G(A,B)\rightrightarrows KK^G(A,B')$.


(c)
Here we prove $KK^G$ preserves finite biproduct.
The only non-trivial part is the injectivity of
\[KK^G(A_1\oplus A_2,B)\to KK^G(A_1,B)\oplus KK^G(A_2,B).\]
Let $(E_0,F_0)\in KK^G(A_1\oplus A_2,B)$.
Define a Kasparov cycle $(E,F)$ from $A_1\oplus A_2$ to $B[0,1]$ such that
\[E:=E_0\otimes_BBV,\quad V:=([0,1]\times\{0\})\cup(\{0\}\times[0,1]),\quad F:=F_0\otimes1,\]
where the correspondnece structure on $E$ is given by
\[((a_1,a_2)\xi b)(s,t):=\begin{cases}
(a_1,(1-s)a_2)\xi(s,0)b(s)&\text{ if }s\ne0,\\
(a_1,a_2)\xi(0,0)b(0)&\text{ if }(s,t)=(0,0),\\
((1-t)a_1,a_2)\xi(0,t)b(t)&\text{ if }t\ne0,
\end{cases}
\qquad
\begin{gathered}
(a_1,a_2)\in A_1\oplus A_2,\ b\in B[0,1],\\
\xi\in E,\ (s,t)\in V.
\end{gathered}\]
and
\[\<\eta,\xi\>(t):=\begin{cases}
\<\eta(0,0),\xi(0,0)\>&\text{ if }t=0,\\
\frac{1+t}2(\<\eta(t,0),\xi(t,0)\>+\<\eta(0,t),\xi(0,t)\>)&\text{ if }t\ne0,
\end{cases}
\qquad \xi,\eta\in E,\ t\in[0,1].\]
Then, $(E,F)$ is a homotopy between $(E_0,F_0)$ and $((_{A_1}E_0)\oplus(_{A_2}E_0),F_0\oplus F_0)$, so we are done.
\end{pf}



\begin{prb}[Stabilization theorem]
Let $E$ be an equivariant Hilbert module over a C$^*$-algebra $B$ with an action of a compact group $G$.
Denote by $B_0$ and $E_0$ the C$^*$-algebra $B$ and the Hilbert module $E$ over $B_0$ with trivial gradings and trivial group actions.
Let $H_{B_0}:=\ell^2\otimes B_0$ and $H_B:=(\ell^2\otimes L^2(G)\otimes B,\id\otimes\lambda\otimes\beta)$ be the standard Hilbert modules respectively over $B_0$ and $B$, where $H_B$ is graded and equivariant with non-trivial grading on $\ell^2=\ell^2_+\oplus\ell^2_-$.
Suppose $E$ is countably generated as a Hilbert $B$-module.
\begin{parts}
\item There is a $B_0$-adjointable isometric isomorphism $H_{B_0}=E_0\oplus H_{B_0}$.
\item There is an equivariant even $B$-adjointable isomteric isomorphism $H_B=E\oplus H_B$.
\end{parts}
\end{prb}
\begin{pf}
(a)
The Hilbert module $E_0$ over $B_0$ is countably generated if and only if there is a dense range $B_0$-adjointable operator $H_{B_0}\to E_0$. (I think it is false)


(b)
Since the grading is technically nothing but an action of $\Z/2\Z$ and the product group $G\times\Z/2\Z$ is still compact, it only needs to consider group actions.
By the part (a), there is a $B_0$-adjointable isometric isomorphism $T_0:H_{B_0}=E_0\oplus H_{B_0}$ which is equivariant with respect to trivial actions, and the tensor product $1\otimes_\beta T_0:L^2(G)\otimes_\beta H_{B_0}=L^2(G)\otimes_\beta(E_0\oplus H_{B_0})$ is an equivariant $B$-adjointable isometric isomorphism because the equivariance follows from $(\lambda_s\otimes1)(1\otimes_\beta T)=(1\otimes_\beta T)(\lambda_s\otimes1)$ and the adjointability is due to $(1\otimes_\beta T)^*=1\otimes_\beta T^*$.
Since $\ell^2=\ell^2\oplus\ell^2=\ell^2\otimes\ell^2$ and the tensor product of Hilbert spaces is commutative, by applying the Fell absorption principle for Hilbert modules three times, we have
\begin{align*}
H_B
&=\ell^2\otimes L^2(G)\otimes B\\
&=\ell^2\otimes L^2(G)\otimes(\ell^2\otimes B)\\
&=\ell^2\otimes L^2(G)\otimes_\beta H_{B_0}\\
&=\ell^2\otimes L^2(G)\otimes_\beta(E_0\oplus H_{B_0})\\
&=(\ell^2\otimes L^2(G)\otimes_\beta E_0)\oplus(\ell^2\otimes L^2(G)\otimes_\beta H_{B_0})\\
&=(\ell^2\otimes L^2(G)\otimes E)\oplus(\ell^2\otimes L^2(G)\otimes B)\\
&=H_E\oplus H_B,
\end{align*}
where $H_E:=(\ell^2\otimes L^2(G)\otimes E,\id\otimes\lambda\otimes u)$, and all the identities mean equivariant $B$-adjointable isometric isomorphisms.

Since $G$ is compact, we have an equivariant linear isometry $\C\to L^2(G)$.
It gives rise to a direct sum decomposition $L^2(G)=\C\oplus \C^\perp$, so $E^\perp:=\C^\perp\otimes E$ implies
\begin{align*}
E\oplus H_E
&=E\oplus(\ell^2\otimes L^2(G)\otimes E)\\
&=E\oplus(\ell^2\otimes(\C\oplus\C^\perp)\otimes E)\\
&=E\oplus(\ell^2\otimes(E\oplus E^\perp))\\
&=E\oplus(\ell^2\otimes E)\oplus(\ell^2\otimes E^\perp)\\
&=((\C\oplus\ell^2)\otimes E)\oplus(\ell^2\otimes E^\perp)\\
&=(\ell^2\otimes E)\oplus(\ell^2\otimes E^\perp)\\
&=\ell^2\otimes(E\oplus E^\perp)\\
&=\ell^2\otimes L^2(G)\otimes E\\
&=H_E,
\end{align*}
where all the identities mean equivariant $B$-adjointable isometric isomorphisms.
Therefore,
\[H_B=H_E\oplus H_B=E\oplus H_E\oplus H_B=E\oplus H_B.\qedhere\]
\end{pf}

\begin{prb}[Technical theorem]
Let $J$ be a C$^*$-algebra with a continuous action of a locally compact group $G$, and $A_1$ and $A_2$ be C$^*$-subalgebras of the multiplier algebra $M(J)$.
Suppose $\f$ ($\f(s)=[u_s,\hat F_2]$) is a bounded function $G\to M(J)$ such that
\begin{enumerate}[(i)]
\item $\Delta$ is a norm separable subset of $M(J)$ such that $[\Delta,A_1]\subset A_1$, and $G$ acts on $A_1$,
\item $A_1(A_2\cup\f(G)\cup\f(G)^*)\subset J$,
\item $G$-action on $A_1$ is continuous, and $s\mapsto a_1\f(s),a_1\f(s)^*$ are norm continuous for every $a_1\in A_1+J$.
\end{enumerate}
Assume that $J$, $A_1$, $A_2$ are $\sigma$-unital, and that $G$ is $\sigma$-compact.
Then, there is $M,N\in M(J)^{\mathrm{ev}}$ with $0\le M,N\le1$ and $M+N=1$ such that
\begin{enumerate}[(i)]
\item $[\Delta,M],[\Delta,N]\subset J$ and $[u_s,M],[u_s,N]\in J$ for $s\in G$,
\item $MA_1\subset J$ and $N(A_2\cup\f(G)\cup\f(G)^*)\subset J$,
\item $M$ and $N$ are $G$-continuous, and $s\mapsto N\f(s)$ is norm continuous.
\end{enumerate}
\end{prb}



\begin{pf}
(a)
We first show a lemma on the existence of sequential quasi-central approximate unit.
The statement we want to prove is as follows:
for a $\sigma$-unital C$^*$-algebra $A$ and a strictly locally compact $\sigma$-compact Hausdorff subset $\Delta\subset M(A)$, there is a countable directed approximate unit $e_n\in A^+$ of $A$ such that $[e_n,-]\to0$ compactly in $C(\Delta,A)$.

Or assuming $\Delta$ is a norm separable closed subspace, we can construct a bounded linear map $\Delta\to\ell^1(\N):d\mapsto([e_n-e_{n-1},d])_n.$

Let $e_n$ be a sequential approximate unit of $A$.
Take any compact $K\subset Y$.
Let $\Lambda$ be the algebraic convex closure of $e_n$.
Define a bounded linear operator
\[L:A\to C(K,A):a\mapsto[a,-].\]
Our goal is to show the closure of the image $L\Lambda$ in $C(K,A)$ contains zero.
Suppose not so that there is $l\in C(K,A)^*$ such that
\[0<\inf_{v\in\Lambda}\Re l(Lv).\]
We claim that $Le_i\to0$ weakly in $C(K,A)$.
We can show that it converges in
\[\sigma(A\otimes C(K),A^*\odot\mathrm{span}\,\mathrm{PS}(C(K))).\]
To enhance the convergence, we need to introduce vector measures and require for an approximate unit to be a sequence for applying the bounded convergence theorem!!!!
I think we can show this using the measure topology (maybe).

(b)
We now prove the main theorem.
We do not consider $G$-actions.
All we have are $A_1,A_2,\Delta\subset M(J)$ such that $[\Delta,A_1]\subset J$ and $A_1A_2\subset J$.
We may assume $\Delta$ is countable and norm compact.

Take strictly positive $h_1\in A_1$, $h_2\in A_2$, $k\in J$.
Take an approximate unit $e_n\in A_1$ quasi-central for $\Delta$.
We may assume
\[\|e_nh_1-h_1\|<2^{-n},\qquad\|[d,e_n]\|<2^{-n},\]
where $d\in\Delta$.
Take an approximate unit $v_n\in J$ such that $v_0=0$ and
\[\|(v_n-v_{n-1})^{\frac12}w\|\le\|w(1-v_{n-1})w\|^{\frac12}<2^{-n},\qquad\|[z,(v_n-v_{n-1})^{\frac12}]\|<2^{-n},\]
where $w\in\{k,e_nh_2\}$ and $z\in\Delta\cup\{h_1,h_2\}$.
The second inequality can be done by approximate the square root with polynomials in $v_n-v_{n-1}$.

The sum of $(v_n-v_{n-1})^{\frac12}(1-e_n)(v_n-v_{n-1})^{\frac12}$ is strictly Cauchy in $M(J)$ because
\[\|(v_n-v_{n-1})^{\frac12}(1-e_n)(v_n-v_{n-1})^{\frac12}k\|\le\|(v_n-v_{n-1})^{\frac12}k\|<2^{-n},\]
and defined to be $M$.
We have $[\Delta,M]\subset J$ because
\[\|[d,(v_n-v_{n-1})^{\frac12}(1-e_n)(v_n-v_{n-1})^{\frac12}]\|\le2\|[d,(v_n-v_{n-1})^{\frac12}\|+\|[d,e_n]\|<3\cdot2^{-n}.\]
We have $MA_1\subset J$ because
\[\|(v_n-v_{n-1})^{\frac12}(1-e_n)(v_n-v_{n-1})^{\frac12}h_1\|\le\|[(v_n-v_{n-1})^{\frac12},h_1]\|+\|h_1-e_nh_1\|<2\cdot2^{-n}.\]
We have $NA_2\subset J$ because
\[\|(v_n-v_{n-1})^{\frac12}e_n(v_n-v_{n-1})^{\frac12}h_2\|\le\|[(v_n-v_{n-1})^{\frac12},h_2]\|+\|(v_n-v_{n-1})^{\frac12}e_nh_2\|<2\cdot2^{-n}.\]
\end{pf}

If $hk\in\cK$, then there is a projection $M(\cK)$ which essentially separates $h$ and $k$.


Let $H:=\ell^2$ and take $s\in B(H)^+$ such that $s$ is not invertible but $1_{\{0\}}(s)=0$ and the projections $p_n:=1_{[(n+1)^{-1},n^{-1})}(s)$ are infinite in $B(H)$ with $e_n:=\sum_{k\le n}p_k\to1$ strictly in $B(H)$.
Since $sx\in K(H)$ implies $p_nx\in K(H)$ by polynomial approximation via functional calculus, and since $p_nx\in K(H)$ implies $sx\in K(H)$ by $e_nsx\in K(H)$, so $B:=\{x\in B(H):xs\in K(H),sx\in K(H)\}=\{x\in B(H):xp_n\in K(H),\ p_nx\in K(H)\}$.

If $x\in B(H)$ with $0\le x\le1$ essentially annihilates $C^*(s)$, that is, $x\in B$, then $p_nxp_n\in K(p_nH)$ implies $\|p_n(1-x)p_n\|\ge1$.
By measuring the norm of $p_n(1-x)p_n$ as an operator on $K(p_nH)$, we can find $b=\sum_np_nbp_n\in B(H)$ such that $\|b\|=1$ and $p_nbp_n\in K(p_nH)$ and $\|(p_n(1-x)p_n)(p_nbp_n)\|\ge2^{-1}$.
Then, $b\in B$ since $p_nb=p_nbp_n=bp_n\in K(H)$.
Since $b$ commutes with $p_n$ by construction, we have $\|p_n(1-x)bp_n\|\ge2^{-1}$, and it implies $(1-x)b\notin K(H)$.
(If $k\in K(H)^+$, then $\|p_nkp_n\|=\|k^{\frac12}(e_n-e_{n-1})k^{\frac12}\|\to0$.)


Non-commutative Tietze extension.



\begin{prb}[How to use the technical theorem]
Let $A$, $B$, and $C$ be C$^*$-algebras with actions of a locally compact group $G$, and let $(E_1,F_1)$ and $(E_2,F_2)$ be equivariant Kasparov modules over $(A,B)$ and $(B,C)$ respectively.
For the interior tensor product $E_{12}:=E_1\otimes_BE_2$, let
\[J:=K(E_{12}),\qquad K_1:=K(E_1)\otimes\C,\qquad K_2:=\{x\in B(E_{12}):K_1x\cup xK_1\subset J\}\]
be C$^*$-subalgebras of $M(J)=B(E_{12})$.
The essential annihilator $K_2$ has a counterexample such that $M$ does not exist.
Consider a subset $\Delta$ and C$^*$-subalgebras $A_1$ and $A_2$ of $M(J)$.

\begin{enumerate}[(i)]
\item $[\Delta,A_1]\subset A_1$,
%\item $GA_1\subset A_1$ and $s\mapsto[u_s,a_1],a_1[u_s,\hat F_2],a_1[u_s,\hat F_2]^*$ are norm continuous for every $a_1\in A_1+J$,
\item $A_1A_2\subset J$,
\end{enumerate}
Under some countability conditions, the technical theorem states that there exist $M$ and $N$ are even adjointable positive operators on $E_{12}$ with $M+N=1$ such that
\begin{enumerate}[(i)]
\item $[\Delta,M]\subset J$ (and $[\Delta,N]\subset J$)
%\item $[u_G,M],[u_G,N]\subset J$ and $s\mapsto[u_s,N],N[u_s,\hat F_2]$ are norm continuous,
\item $MA_1,NA_2\subset J$.
\end{enumerate}


reformulation of the existence of an element of $M(J)$ to the closedness of a subset of $J^*$?

Restatement: if $B_1,B_2$ are orthogonal $\sigma$-unital C$^*$-subalgebras of a unital C$^*$-algebra $Q$, and if $[\Delta,B_1]\subset B_1$, then there is $m\in Q$ commuting with $\Delta$ such that $mb=0$ for $b\in B_1$ and $mb=b$ for $b\in B_2$.


If every $m$ cannot separate $B_1$ and $B_2$, then

When $\Delta=0$ and every algebras are separable, since $\pi(A_1)\pi(A_2)=0$ implies $C^*(\pi(A_1),\pi(A_2))\cong\pi(A_1)\oplus\pi(A_2)$, and since the surjection $C^*(A_1,A_2)\to C^*(\pi(A_1),\pi(A_2))$ is extended to a \emph{surjection} $M(C^*(A_1,A_2))\to M(C^*(\pi(A_1),\pi(A_2)))$, the technical theorem can be easily proved.




(If we prove $F_{12}-F_{12}'\in K_2$, then we have $N^{\frac12}F_{12}\equiv N^{\frac12}F_{12}'$.)


Let $\hat F_1$ and $\hat F_2$ be odd adjointable operators on $E_{12}$.
Let
\[F_{12}:=M^{\frac12}\hat F_1+N^{\frac12}\hat F_2.\]

\[D_{12}=D_1\otimes1+1\otimes D_2+(s-(\id\otimes1_B))\otimes1,\qquad s:E_1\to E_1\otimes\tilde B\]

\[M=\frac{\frac12+\hat D_1^2}{1+\hat D_1^2+\hat D_2^2},\quad N=\frac{\frac12+\hat D_2^2}{1+\hat D_1^2+\hat D_2^2}\]




\begin{parts}
\item Such $M$ and $N$ exist if $\Delta$ is separable, $A_1$ and $A_2$ are $\sigma$-unital, and $G$ is $\sigma$-compact.
\item If
\begin{align*}
\Delta&\supset\{\hat F_1,\hat F_2\}\cup A,\\
A_1+J&\supset\{[\hat F_1,a],[u_s,\hat F_1]a,(\hat F_1-\hat F_1^*)a,(\hat F_1^2-1)a:a\in A,\ s\in G\},\quad\text{$\hat F_1a$ is $G$-continuous},\\
A_2+J&\supset\{[\hat F_2,a],[u_s,\hat F_2]a,(\hat F_2-\hat F_2^*)a,(\hat F_2^2-1)a:a\in A,\ s\in G\},\\
A_1\cup A_2+J&\supset\{[\hat F_1,\hat F_2]\},
\end{align*}
then $(E_{12},F_{12})$ is an equivariant Kasparov module over $(A,C)$.
We cannot assume $\hat F_2a$ is $G$-continuous in construction, so the condition (ii) in the technical theorem becomes longer than (i) and (iii).
In particular, the second row is fulfilled if $\hat F_1=F_1\otimes1$ and $A_1\supset K_1$, the third row is fulfilled if $(E_{12},\hat F_2)$ is a Kasparov module.
When we apply the technical theorem, these two conditions are always satisfied, except the existence of the Kasparov product.
\item If $\hat F_2$ satisfies the connection condition for $F_2$ and if
\[\Delta\supset\{\hat F_1,\hat F_2\},\qquad
A_1+J\supset K_1,\]
then $F_{12}$ satisfies the connection condition for $F_2$.
\item If $\hat F_1$ satisfies the positivity condition for $F_1$ and if
\[\Delta\supset\{F_1\otimes1,\hat F_1\}\cup A,\qquad
A_2+J\supset\{[F_1\otimes1,\hat F_2]\},\]
then $F_{12}$ satisfies the positivity condition for $F_1$.

(If )
\item
If $\hat F_1=F_1\otimes1$ and $\hat F_2$ satisfies the connection condition for $F_2$, and if
\begin{align*}
\Delta&=\{\hat F_1,\hat F_2\}\cup A,\\
A_1&=K_1+C^*(\{[\hat F_1,a],[u_s,\hat F_1]a,(\hat F_1-\hat F_1^*)a,(\hat F_1^2-1)a:a\in A,\ s\in G\}),\\
A_2&=C^*([\hat F_1,\hat F_2])+C^*(\{[\hat F_2,a],[u_s,\hat F_2]a,(\hat F_2-\hat F_2^*)a,(\hat F_2^2-1)a:a\in A,\ s\in G\}),
\end{align*}
then all the assumptions of the technical theorem, and $(E_{12},F_{12})$ is a Kasparov product of $(E_1,F_1)$ and $(E_2,F_2)$.

Note that $\hat F_1=F_1\otimes1$ implies
\[[\hat F_1,a],[u_s,\hat F_1]a,(\hat F_1-\hat F_1^*)a,(\hat F_1^2-1)a\in K_1\]
and the connection condition $\hat F_2$ for $F_2$ implies
\[[\hat F_2,a],[u_s,\hat F_2]a,(\hat F_2-\hat F_2^*)a,(\hat F_2^2-1)a\in K_2\]


positivity condition implies $[\hat F_1,\hat F_2]C^*(\{[\hat F_1,a],[u_s,\hat F_1]a,(\hat F_1-\hat F_1^*)a,(\hat F_1^2-1)a:a\in A,\ s\in G\})\subset J$?


If we add some elements in $\Delta,A_1,A_2$ in this basic form, and if we check that the added elements do not violate the conditions $[\Delta,A_1]\subset A_1$ and $A_1A_2\subset J$, then $(E_{12},F_{12})$ is still a Kasparov product of $(E_1,F_1)$ and $(E_2,F_2)$.
\end{parts}
\end{prb}
\begin{pf}
(b)
We prove that from (ii) to (v).
Modulo $K(E_{12})$, we have
\begin{align*}
[F_{12},a]&\equiv M^{\frac12}[\hat F_1,a]+N^{\frac12}[\hat F_2,a]\equiv0,\\
[u_s,F_{12}]a&\equiv M^{\frac12}[u_s,\hat F_1]a+N^{\frac12}[u_s,\hat F_2]a\equiv0,\\
(F_{12}-F_{12}^*)a&\equiv(\hat F_1-\hat F_1^*)aM^{\frac12}+(\hat F_2-\hat F_2^*)aN^{\frac12}\equiv0,\\
(F_{12}^2-1)a&\equiv M(\hat F_1^2-1)a+N(\hat F_2^2-1)a+M^{\frac12}N^{\frac12}[\hat F_1,\hat F_2]a\equiv0.
\end{align*}
For the $G$-continuity of $F_{12}a$, since $s\mapsto[u_s,M],[u_s,N],N[u_s,\hat F_2]$ are norm continuous, the map
\begin{align*}
s\mapsto[u_s,F_{12}a]
&=[u_s,M^{\frac12}\hat F_1a+N^{\frac12}\hat F_2a]\\
&=[u_s,M^{\frac12}]\hat F_1a+M^{\frac12}[u_s,\hat F_1a]\\
&+[u_s,N^{\frac12}]\hat F_2a+N^{\frac12}[u_s,\hat F_2]a+N^{\frac12}\hat F_2[u_s,a]
\end{align*}
is norm continuous, so $(E_{12},F_{12})$ is a Kasparov module over $(A,C)$.

(c)
Because $\theta_{\xi_1}\otimes1\in A_1$ for $\xi_1\in E_1$ implies $M^{\frac12}T_{\xi_1}T_{\xi_1}^*M^{\frac12}=M^{\frac12}(\theta_{\xi_1}\otimes1)M^{\frac12}\equiv0$ and $M^{\frac12}T_{\xi_1}\equiv0$ by the polar decomposition, we have
\begin{align*}
F_{12}T_{\xi_1}-T_{\xi_1}F_2
&=M^{\frac12}\hat F_1T_{\xi_1}+N^{\frac12}\hat F_2T_{\xi_1}-T_{\xi_1}F_2\\
&\equiv\hat F_1M^{\frac12}T_{\xi_1}+N^{\frac12}T_{\xi_1}F_2-T_{\xi_1}F_2\\
&\equiv0-(N^{\frac12}+1)^{-1}MT_{\xi_1}F_2\equiv0,
\end{align*}
and
\begin{align*}
F_{12}^*T_{\xi_1}-T_{\xi_1}F_2^*
&=\hat F_1^*M^{\frac12}T_{\xi_1}+\hat F_2^*N^{\frac12}T_{\xi_1}-T_{\xi_1}F_2^*\\
&\equiv0+N^{\frac12}\hat F_2^*T_{\xi_1}-T_{\xi_1}F_2^*\\
&\equiv N^{\frac12}T_{\xi_1}F_2^*-T_{\xi_1}F_2^*\\
&\equiv-(N^{\frac12}+1)^{-1}MT_{\xi_1}F_2^*\equiv0.
\end{align*}

(d)
It easily follows from
\begin{align*}
a^*[F_1\otimes1,F_{12}]a
&\equiv a^*M^{\frac12}[F_1\otimes1,\hat F_1]a+a^*N^{\frac12}[F_1\otimes1,\hat F_2]a\\
&\equiv2M^{\frac14}a^*[F_1\otimes1,\hat F_1]aM^{\frac14}+0\ge0.
\end{align*}

(e)
Let $\theta_{\xi_1,\eta_1}\otimes1=T_{\xi_1}T_{\eta_1}^*\in K_1$ with $\xi_1,\eta_1\in E_1$.
We can check $\hat F_2K_1\subset K_1+J$ by applying the polar decomposition on
\begin{align*}
|F_{12}T_{\xi_1}T_{\eta_1}^*|^2
&\equiv|T_{\e(\xi_1)}F_2T_{\eta_1}^*|^2
=T_{\eta_1}F_2^*T_{\e(\xi_1)}^*T_{\e(\xi_1)}F_2T_{\eta_1}^*
=T_{\eta_1}F_2^*|\e(\xi_1)|^2F_2T_{\eta_1}^*\\
&\equiv T_{\eta_1}F_2^*F_2|\xi_1|^2T_{\eta_1}^*
\equiv T_{\eta_1}|\xi_1|^2T_{\eta_1}^*
=T_{\eta_1}T_{\xi_1}^*T_{\xi_1}T_{\eta_1}^*
=(\theta_{\xi_1,\eta_1}^*\theta_{\xi_1,\eta_1})\otimes1,
\end{align*}
and $[\hat F_2,K_1]\subset J$ by
\[[\hat F_2,T_{\xi_1}T_{\eta_1}^*]=\hat F_2T_{\xi_1}T_{\eta_1}^*-T_{\e(\xi_1)}T_{\e(\eta_1)}^*\hat F_2\equiv T_{\e(\xi_1)}F_2T_{\eta_1}^*-T_{\e(\xi_1)}F_2T_{\eta_1}^*=0.\]


\begin{align*}
K_1[\hat F_2,A]&\subset[\hat F_2,K_1A]+[\hat F_2,K_1]A\subset[\hat F_2,K_1]+JA\subset J,\\
T_{\xi_1}T_{\eta_1}^*(\hat F_2-\hat F_2^*)&\equiv T_{\xi_1}(F_2-F_2^*)T_{\e(\eta_1)}^*=\lim_iT_{\xi_1}(e_i(F_2-F_2^*))T_{\e(\eta_1)}^*\equiv0,\\
T_{\xi_1}T_{\eta_1}^*(\hat F_2^2-1)&\equiv T_{\xi_1}(\hat F_2^2-1)T_{\eta_1}^*=\lim_iT_{\xi_1}(e_i(\hat F_2^2-1))T_{\eta_1}^*\equiv0.
\end{align*}

The condition $[\Delta,A_1]\subset A_1$ follows from $[A,K_1]\cup[\hat F_1,K_1]\subset[B_1,K_1]\subset K_1$ and $[\hat F_2,K_1]\subset J$.
The condition $A_1(A_2\cup\f(G)\cup\f(G)^*)\subset J$ holds because $K_1u_s=K_1$ and $K_1\hat F_1\subset K_1$ imply
\begin{align*}
K_1[u_s,\hat F_2]&\subset[K_1u_s,\hat F_2]+[K_1,\hat F_2]u_s\subset J,\\
K_1[u_s,\hat F_2]^*&=K_1[u_s,\hat F_2^*]\subset[K_1u_s,\hat F_2^*]+[K_1,\hat F_2^*]u_s\subset J,\\
K_1[\hat F_1,\hat F_2]&\subset[K_1\hat F_1,\hat F_2]+[K_1,\hat F_2]\hat F_1\subset J.
\end{align*}
The map $s\mapsto a_1[u_s,\hat F_2]$ is continuous for $a_1\in A_1$ since $a_1\hat F_2\in A_1$ so that
\[a_1[u_s,\hat F_2]=[u_s,a_1\hat F_2]-[u_s,a_1]\hat F_2\to0,\]
and we can do similarly for $s\mapsto a_1[u_s,\hat F_2^*]$.
\end{pf}





\begin{prb}[Kasparov product]
Let $A,B,C$ be $G$-C$^*$-algebras and $G$ be a locally compact group.
For Kasparov modules $(E_1,F_1)$ and $(E_2,F_2)$ from $A$ to $B$ and from $B$ to $C$, a \emph{Kasparov product} of $(E_1,F_1)$ and $(E_2,F_2)$ is a Kasparov module $(E_{12},F_{12})$ from $A$ to $C$ such that $E_{12}:=E_1\otimes_BE_2$ and $F_{12}$ satisfies
\begin{enumerate}[(i)]
\item the connection condition for $F_2$:
\[F_{12}T_{\xi_1}-T_{\e(\xi_1)}F_2,\quad F_{12}^*T_{\xi_1}-T_{\e(\xi_1)}F_2^*\in K(E_2,E_{12}),\qquad\xi_1\in E_1,\]
\item the positivity condition for $F_1$:
\[a^*[F_1\otimes1,F_{12}]a\ge0\in Q(E_{12}),\qquad a\in A.\]
\end{enumerate}

Suppose $A$ is separable and $G$ is $\sigma$-compact.
\begin{parts}
\item The Kasparov product exists.
\item The Kasparov product is unique up to operator homotopy.
\item The Kasparov product is independent up to homotopy.
\item The Kasparov product is associative and has units, i.e.~the Kasparov category is additive.
\end{parts}
\end{prb}
\begin{pf}
(a)
As the first step, we prove the following lemma:
for a countably generated super-Hilbert module $E_1$ over $B$ and a super-correspondence $E_2$ from $B$ to $C$ with trivial group actions, if $F_2\in B(E_2)^{\mathrm{odd}}$ satisfies $[F_2,b]\in K(E_2)$ for $b\in B$, then there exists $\hat F_2\in B(E_{12})^{\mathrm{odd}}$ satisfying the connection condition for $F_2$.
Let $\tilde B$ be the unitization of $B$.
Then, $E_1$ and $E_2$ are naturally considered as a super-Hilbert module over $\tilde B$ and a super-correspondence from $\tilde B$ to $C$, respectively.
Since $E_1$ is countably generated, regarding the grading temporarily as a $\Z/2\Z$-action, we can apply the equivariant stabilization theorem for $\Z/2\Z$-action to construct an even projection $P_1\in B(H_{\tilde B})^{\mathrm{ev}}$ such that $E_1=P_1H_{\tilde B}$, where $H_{\tilde B}:=\ell^2\otimes\tilde B$ and $\ell^2=\ell^2_+\oplus\ell^2_-$ has a non-trivial $\Z/2\Z$-grading.
Then, we have even $\tilde B$-adjointable isometry
\[E_{12}\subset H_{\tilde B}\otimes_{\tilde B} E_2=\ell^2\otimes\tilde BE_2\subset\ell^2\otimes E_2,\]
with a retract
\[\begin{tikzcd}
\ell^2\otimes E_2 \rar[->>]{1\otimes1_{\tilde B}}&
\ell^2\otimes\tilde BE_2=
H_{\tilde B}\otimes_{\tilde B} E_2 \rar[->>]{P_1\otimes_{\tilde B}1}&
E_1\otimes_{\tilde B}E_2=
E_{12},
\end{tikzcd}\]
where
\[1\otimes1_{\tilde B}\in B(\ell^2\otimes E_2)^{\mathrm{ev}},\qquad P_1\otimes_{\tilde B}1\in B(H_{\tilde B}\otimes_{\tilde B}E_2)^{\mathrm{ev}}\]
are projections onto complemented super-Hilbert submodules of $\ell^2\otimes E_2$ over $C$.
Observe that the tensor product $1\otimes F_2$ is well-defined on $\ell^2\otimes E_2$.
Define $\hat F_2\in B(E_{12})^{\mathrm{odd}}$ by the compression of $1\otimes F_2\in B(\ell^2\otimes E_2)^{\mathrm{odd}}$ using the inclusion of $E_{12}\subset\ell^2\otimes E_2$.
To check $\hat F_2$ satisfies the connection condition for $F_2$, take $\xi_1\in E_1\subset H_{\tilde B}$.
Since $\ell^2\odot\tilde B$ is dense in $H_{\tilde B}$, we may assume $\xi_1=e\otimes b$ for $e\in\ell^2$ and $b\in\tilde B$.
Then,
\begin{align*}
\hat F_2T_{\xi_1}\xi_2
&=(P_1\otimes_{\tilde B}1)(1\otimes1_{\tilde B})(1\otimes F_2)(1\otimes1_{\tilde B})(P_1\otimes_{\tilde B}1)(\xi_1\otimes_{\tilde B_0}\xi_2)\\
&=(P_1\otimes_{\tilde B}1)(1\otimes1_{\tilde B})(1\otimes F_2)(1\otimes1_{\tilde B})(\xi_1\otimes_{\tilde B}\xi_2)\\
&=(P_1\otimes_{\tilde B}1)(1\otimes1_{\tilde B})(1\otimes F_2)(1\otimes1_{\tilde B})(e\otimes b\xi_2)\\
&=(P_1\otimes_{\tilde B}1)(1\otimes1_{\tilde B})(1\otimes F_2)(e\otimes b\xi_2)\\
&=(P_1\otimes_{\tilde B}1)(1\otimes1_{\tilde B})(\e(e)\otimes F_2b\xi_2)\\
&=(P_1\otimes_{\tilde B}1)(\e(e)\otimes1_{\tilde B}F_2b\xi_2),\\
T_{\e(\xi_1)}F_2\xi_2
&=\e(\xi_1)\otimes_{\tilde B_0}F_2\xi_2\\
&=(P_1\otimes_{\tilde B}1)(\e(\xi_1)\otimes_{\tilde B}F_2\xi_2)\\
&=(P_1\otimes_{\tilde B}1)(\e(e)\otimes bF_2\xi_2),
\end{align*}
so
\[\hat F_2T_{\xi_1}-T_{\e(\xi_1)}F_2=(P_1\otimes_{\tilde B_0}1)(\e(e)\otimes1_{\tilde B})[F_2,b]\in K(E_2,E_{12}).\]
By definition of $\hat F_2$, we can also check the same for $\hat F_2^*$.
Therefore, $\hat F_2$ satisfies the connection condition for $F_2$.

Now we construct $F_{12}\in B(E_{12})^{\mathrm{odd}}$ such that $(E_{12},F_{12})$ is a Kasparov module, and the connection and positivity conditions are satisfied.
Let $\hat F_1:=F_1\otimes1$, and take $\hat F_2\in B(E_{12})^{\mathrm{ev}}$ satisfying the connection condition for $F_2$.
Let
\begin{align*}
\Delta&:=\{\hat F_1,\hat F_2\}\cup A,\\
A_1&:=K_1,\\
A_2&:=C^*([\hat F_1,\hat F_2],\ [\hat F_2,a],\ (\hat F_2-\hat F_2^*)a,\ (\hat F_2^2-1)a:a\in A).
\end{align*}
By the technical theorem, $F_{12}:=M^{\frac12}\hat F_1+N^{\frac12}\hat F_2$ is a Kasparov product.




(b)
Suppose both $F_{12}$ and $F'_{12}$ on $E_{12}$ define Kasparov products of $(E_1,F_1)$ and $(E_2,F_2)$.
Let
\begin{align*}
\Delta&:=\{\hat F_1,F_{12},F_{12}'\}\cup A,\\
A_1&:=K_1,\\
A_2&:=C^*([\hat F_1,F_{12}],[\hat F_1,F_{12}'],F_{12}-F_{12}').
\end{align*}
We only need to check $A_1(F_{12}-F_{12}')\subset J$ to apply the technical theorem: for $T_{\xi_1}T_{\eta_1}^*\in K_1$, we have
\[T_{\xi_1}T_{\eta_1}^*(F_{12}-F_{12}')\equiv T_{\xi_1}(F_2-F_2)T_{\e(\eta_1)}^*=0.\]
Define $F_{12}'':=M^{\frac12}\hat F_1+N^{\frac12}F_{12}$, using the technical theorem.
Since $N^{\frac12}F_{12}\equiv N^{\frac12}F_{12}'$ implies
\begin{align*}
a^*[F_{12}'',F_{12}]a\equiv a^*M^{\frac12}[\hat F_1,F_{12}]a+a^*N^{\frac12}[F_{12},F_{12}]a\ge0,\\
a^*[F_{12}'',F_{12}']a\equiv a^*M^{\frac12}[\hat F_1,F_{12}']a+a^*N^{\frac12}[F_{12},F_{12}]a\ge0,
\end{align*}
the operators $F_{12}$ and $F_{12}'$ are connected by two operator homotopies via $F_{12}''$ by the following lemma: if $(E_0,F_0)$ and $(E_0,F_1)$ are Kasparov modules over $(A,B)$ such that $a^*[F_0,F_1]a\ge0$ in $Q(E_0)$ for every $a\in A$, then they are operator homotopic.



Here is the proof.
Let $C:=\{T\in B(E_0):[T,a]\in K(E_0),\ a\in A\}$ and $I:=\{T\in B(E_0):Ta,aT\in K(E_0),\ a\in A\}$.
Then, $I$ is a closed ideal of a C$^*$-algebra $C$, and let all congruence notation denote the identity modulo $I$ in the rest of the proof.
We can check $[F_0,F_1]\in C$ and $[F_0,F_1]\ge0$ modulo $I$ so that there is an even operator $P\in C^+$ such that $[F_0,F_1]\equiv P$, and $F_0^2\equiv F_1^2\equiv 1$ implies that $P$ commutes with $F_0$ and $F_1$ modulo $I$.
Define
\[F(t):=(1+c(t)s(t)P)^{-\frac12}(c(t)F_0+s(t)F_1),\qquad t\in[0,1],\]
where $c(t):=\cos\frac\pi2t$ and $s(t):=\sin\frac\pi2t$.
It is norm continuous, $G$-invariant modulo $I$, and $G$-continuous.
We also have for each $t\in[0,1]$ that $F(t)\in C$, $F(t)\equiv F(t)^*$, and
\[F(t)^2\equiv(1+c(t)s(t)P)^{-1}(c(t)^2F_0^2+c(t)s(t)[F_0,F_1]+s(t)^2F_1(t)^2)\equiv1.\]
Thus it defines an operator homotopy between $F_0$ and $F_1$.



(c)


(d)
Consider  the Kasparov products $F_{12}$, $F_{23}$, $F_{1(23)}$.
Note that $F_{1(23)}$ satisfies the positivity condition for $F_1$ and the connection condition for $F_{23}$ and $F_3$.
Let
\begin{align*}
\Delta&:=\{\hhat F_1,\hat F_{12},F_{1(23)}\}\cup A,\\
A_1&:=K_1+K_{12},\\
A_2&:=C^*([\hhat F_1,F_{1(23)}],[\hat F_{12},F_{1(23)}]).
\end{align*}
It suffices to check
\[
[\hhat F_1,K_{12}]\subset A_1,\quad[\hat F_{12},K_1]\subset A_1,\quad
K_1[\hat F_{12},F_{1(23)}]\subset J,\quad K_{12}[\hhat F_1,F_{1(23)}]\subset J,
\]
to apply the technical theorem.
The second is clear from $[\hat F_{12},K_1]\subset J$, and the other can be proved as
\begin{align*}
[\hhat F_1,K_{12}]&\subset[B_1,K_{12}]\subset[B_{12},K_{12}]\subset K_{12},\\
K_1[\hat F_{12},F_{1(23)}]
&\subset[K_1\hat F_{12},F_{1(23)}]+[K_1,F_{1(23)}]\hat F_{12}
\subset[K_1+K_{12},F_{1(23)}]+J\hat F_{12}\subset J,\\
K_{12}[\hhat F_1,F_{1(23)}]
&\subset[K_{12}\hhat F_1,F_{1(23)}]+[K_{12},F_{1(23)}]\hhat F_1
\subset[K_{12},F_{1(23)}]+J\hhat F_1
\subset J.
\end{align*}
If we define $F_{(12)3}:=M^{\frac12}\hhat F_1+N^{\frac12}F_{1(23)}$ using the technical theorem, then it satisfies the connection condition for $F_3$ because so is $F_{1(23)}$, and satisfies the positivity condition $F_{12}$ because $\hat F_{12}\in\Delta$ and $[\hat F_{12},F_{1(23)}]\in A_2$ give
\[a^*[\hat F_{12},F_{(12)3}]a\equiv a^*M^{\frac12}[\hat F_{12},\hhat F_1]a+a^*N^{\frac12}[\hat F_{12},F_{1(23)}]a\ge0,\]
so $F_{(12)3}$ is a Kasparov product of $F_{12}$ and $F_3$.
On the other hand,
\[a^*[F_{(12)3},F_{1(23)}]a\equiv a^*M^{\frac12}[\hhat F_1,F_{1(23)}]a+a^*N^{\frac12}[F_{1(23)},F_{1(23)}]a\ge0\]
implies $F_{(12)3}$ and $F_{1(23)}$ are operator homotopic.
\end{pf}


Assumptions for representatives of Kasparov cycles:
\begin{itemize}
\item non-degenerate
\item standard Hilbert module
\item operator homotopy
\item self-adjoint norm one by compact perturbation
\item norm continuous fredholm when $E$ and $B$ are continuous fields over $X$...?
\end{itemize}





\begin{prb}[K-theory picture]
Let $B$ be a C$^*$-algebra.
Consider
\[KK(\C,B)\to K_0(B):(E,F)\mapsto[\ker F_+]-[\ker F_+^*],\]
where $K_0(B)$ is described in the Serre-Swan picture.
NO, it cannot be defined.
\end{prb}
\begin{pf}
We first show the well-definedness.
Let $(E,F)$ be a Kasparov cycle over $A$ and $B$.
We can show $F_+\in B(E_+,E_-)$ is a Fredholm operator with a parametrix $F_-\in B(E_-,E_+)$.
Since the compact operator $1-F_-F_+$ acts on the Hilbert submodule $\ker F_-F_+\subset E_+$ as the identity, the ideal of finite-rank operators in $B(\ker F_-F_+)$ becomes norm dense, which approximates the unit so that it contains an invertible.
Hence the identity operator has finite rank, and we see that $\ker F_-F_+$ is an algebraically finitely generated module over $B$.
Since $\ker F_+\subset\ker F_-F_+$ and the kernel of an adjointable operator is always complemented(WRONG), $\ker F_+$ is projective.
We can do similarly for $\ker F_+^*$.
As a remark, we can prove $\ran F_+$ is also complemented by the approximation property of compact operators and the open mapping theorem, but it is not essential part in the proof.

Suppose $(E,F_0)$ and $(E,F_1)$ are operator homotopic Kasparov cycles.


The injectivity

The surjectivity

\end{pf}

\begin{prb}[Atiyah-J\"anich theorem in Kasparov picture]
\end{prb}
\begin{pf}
Let $B=C(X)$
Let $H_B=\ell^2\otimes B$.
For $F_+\in B(E_+,E_-)$, by stabilization theorem, we may assume $F_+\in B(H_B)$.
It defines a strictly continuous map $X\to\mathrm{Fred}(\ell^2)$.
We may assume it is norm continuous?
\end{pf}

(direct sum, pullback, interior tensor product, pushout, exterior tensor product?)

ring structure, $R(G)$-module structures




inverses
equivariant imprimitivity bimodules






\section{Cuntz picture}


We begin with historical development of extension theory.

An extension of an algebra is an analogue of an embedding of a space.

Dual algebras and K-homology.

$Q(A)$ is calle the \emph{outer multiplier algebra} or the \emph{Calkin algebra} or the...

\begin{prb}[Weyl-von Neumann theorem]

Recall that neither $U(M(\cK))\to U(Q(\cK))$ nor $N(M(\cK))\to N(Q(\cK))$ is surjective because a unilateral right shift $s\in M(\cK)$ provides a unitary in $Q(\cK)$ of index $-1$, while every invertible in $N(M(\cK))$ should have index zero.
The Fredholm index is a group homomorphism $Q(\cK)^\times\to\Z$, which induces the isomorphism $K_1(Q(\cK))\to K_0(\cK)\cong\Z$.

Recall also that for $x\in M(\cK)$ the \emph{essential spectrum} of $x$ is the set of $\lambda\in\C$ such that $\lambda-x$ is not a Fredholm operator.
By the Atkinson theorem, $\lambda\in\C$ belongs to the essential spectrum of $x$ if and only if $\lambda-x$ is not invertible in $Q(\cK)$, so it is just the spectrum of $x$ in $Q(\cK)$.
It can be also described by the complement of the set of isolated eigenvalues of finite multiplicity in the spectrum in $M(\cK)$.

Two elements of $Q(\cK)$ are called \emph{essentially unitarily equivalent} if same orbit in $Q(\cK)$ by the inner action of $U(M(\cK)))$.
Since $U(M(\cK))\to U(Q(\cK))$ is a proper inclusion, the essential unitary equivalence is stronger than the unitary equivalence in $Q(\cK)$.
If two normal elements $x$ and $y$ of $M(\cK)$ are essentially unitarily equivalent, then their essential spectra coincide.
The Weyl-von Neumann-Berg theorem states the converse also holds.
The original Weyl-von Neumann theorem states that every bounded self-adjoint operator on a separable Hilbert space is an arbitrarily small compact perturbation of a diagonal operator($\sigma=\sigma_p$).

An element of $Q(\cK)$ is called \emph{essentially normal} if it is just normal in $Q(\cK)$.
Note that $N(M(\cK))\to N(Q(\cK))$ is not surjective, while $M(\cK)^{sa}\to Q(\cK)^{sa}$ is surjective.
If two normal elements $x$ and $y$ of $Q(\cK)$ are essentially unitarily equivalent, then their essential spectra coincide and indices of $\lambda-x$ and $\lambda-y$ coincide for all $\lambda$ outside the essential spectrum.
The Brown-Douglas-Fillmore theorem states that the converse also holds.

For an essentially normal operator with essential spectrum $X$, there is a corresponding Busby invariant, a unital injective $*$-homomorhism $C(X)\to Q(\cK)$.
It defines a class of $K_1(C(X))$ in the extension picture.





Let $E$ be a separable unital C$^*$-algebra.
Two maps $E\to B$ are called \emph{approximately unitarily equivalent} if the orbits under the multiplier inner action of $B$ have the same closure in the point-norm topology.

essentially unitarily equivalent iff approximately unitarily equivalent?

basically unitary equivalence takes multiplier unitaries, not unitization unitaries.

\end{prb}


\begin{prb}[Essential extensions and Busby invariants]
Let $A$ be a unital C$^*$-algebra.
An \emph{essential extension} of $A$ by $\cK$ is a unital C$^*$-algebra $E$ together with a surjective $*$-homomorphism $E\to A$ whose kernel is an essential ideal of $E$ $*$-isomorphic to $\cK$.
A \emph{Busby invariant} of $A$ is a unital injective $*$-homomorphism $A\to Q(\cK)$.
We define $\Ext(A)=\Ext(A,\cK)$ as the set of all equivalence classes of essential extensions by $\cK$.
Brown-Douglas-Fillmore investigated this group.
\begin{parts}
\item There is a natural bijection between $\Ext(A)$ and the set of all unitary equivalence classes of Busby invariants.
\item Since there is a natural injective $*$-homomorphism $Q(\cK)\oplus Q(\cK)\to Q(\cK)$?, $\Ext(A)$ is an abelian semi-group.
\end{parts}


If the corresponding essential extension of a Busby invariant $A\to Q(\cK)$ is split, then since it means that there is a $*$-homomorphism $A\to M(\cK)$ factors through the Busby invariant, we can construct an Eilenberg swindle $A\to M(\cK)\cong M(\cK\otimes \cK)\to Q(\cK)$? so the split extension is zero in $\Ext(A)$? (by the Voiculescu theorem, $\Ext(A)$ is an abelian monoid if $A$ is separable.)

An extension by $\cK$ is called \emph{semi-split} if it is a direct summand of a split extension.
Using the Stinespring dilation, we can prove an extension by $\cK$ is semi-split if and only if it has a completely positive section.


Let $A$ be a unital separable C$^*$-algebra.
For a representation $A\to M(\cK)$ together with a projection $p\in B(H)$ satisfying $[p,a]\in\cK$ for $a\in A$, then $A\to Q(pH)$ defines a \emph{generalized Toeplitz extension}
\[0\to K(pH)\to E\to A\to0.\]

A unital injective $*$-homomorphism $A\to Q(\cK)$ is semi-split if and only if it is isomorphic to a generalized Toeplitz extension.

There is a way to define the K-homology by the Spanier-Whitehead duality?. (dual C$^*$-algebra)


We will see that if $A$ is separable nuclear unital, by Choi-Effros, then the extension group is isomorphic to the first K-homology group $\Ext(A):=\Ext(A,\cK)\cong K^1(A)=KK_1(A,\C)=KK(A,\cS)$.

index pairing $\Ext(C(X))\to\Hom(\pi_1(X),\Z)$...


It is known that $KK(A,B)\cong\Ext(A,\cS\otimes B)$
\end{prb}

The Weyl-von Neumann theorem states that self-adjoint elements of $Q(H)$ with same spectrum are all unitarily equivalent.(?)

\begin{prb}
Almost commuting matrices
\end{prb}

\begin{prb}[Voiculescu theorem]
Let $E$ be a unital separable C$^*$-algebra.
Let $\pi:E\to M(\cK)$ be a non-degenerate representation and $\sigma:E\to M(\cK)$ be a completely positive linear map such that $\sigma|_{\pi^{-1}(\cK)}=0$.
For example, we can consider $\sigma$ obtained by a section of a semi-split Busby invariant.

Then, $\sigma\lesssim\pi$, i.e.~there is a coisometry $v\in M(\cK)$ such that $\sigma=(\Ad v)\pi$ in $Q(\cK)$...?

($\pi$ and $\pi\oplus\sigma$ is approximately unitarily equivalent in $Q(\cK)$. If $\sigma$ is a $*$-homomorphism, then $\pi$ and $\pi\oplus\sigma$ is unitarily equivalent in $Q(\cK)$.)

(When do we need the faithfulness of $\pi$?
When do we need the unitality of $\sigma$?
When do we need the separability of $E$?)
\begin{parts}
\item $\sigma$ is weakly$^*$ approximated by vector states, if $H$ is one-dimensional. (Glimm)
\item $\sigma$ is approximated by isometry conjugations in $L(E,B(H))$, if $H$ is finite-dimensional. (?)
\item $\sigma$ is approximated by isometry conjugations in $\f+L(A,K(H))$, if $H,K$ are separable.
\end{parts}
\end{prb}
\begin{pf}
(a)
Hahn-Banach separation and Weyl-von Neumann theorem.

(b)
correspondence for completely positive linear maps to matrix algebras.

(c)
quasi-central approximate unit and block diagonal c.p.~maps.





Let $\pi:A\to B(H)$ be a representation of a C$^*$-algebra and $p\in B(H)$ is a projection.
If $A\to Q(pH)$ is a $*$-homomorphism, then $A\to Q((1-p)H)$ is also a $*$-homomorphism, and $pA(1-p)\cup(1-p)Ap\subset K(H)$.
\end{pf}



(Pimsner, Popa, Voiculescu, Kasparov)
For $A$ separable and $B$ $\sigma$-unital(do we really need this $\sigma$-unitality?),
The group $\Ext(A,B)$ of all equivalence classes of essential extensions
\[0\to\cK\otimes B\to E\to A\to0.\]






stable uniqueness theorem(Lin or Dadarlat-Eilers)


\begin{prb}[Cuntz two-fold extension functor]
Let $A$ be a C$^*$-algebra.
Let $\iota$ and $\bar\iota$ be the canonical inclusions $A\to A*A$ into the free product.
The algebra $qA$ is defined as the C$^*$-algebra generated by $q(a):=\iota(a)-\bar\iota(a)$ for $a\in A$ in $A*A$.
We have $q:A\to qA$.
\end{prb}

\begin{prb}
\[KK(A,B):=[qA,\cK\otimes B]\]

\end{prb}


A \emph{Cuntz pair} or a \emph{quasi-homomorphism} is a pair of $*$-homomorphisms $\varphi_\pm:A\to M(\cK\otimes B)$ such that $\varphi_+-\varphi_-$ maps into $\cK\otimes B$.
A homotopy is defined by a point-strict continuous path.

\section{Baaj-Julg picture}

\begin{prb}[Unbounded Kasparov modules]
Let $A$ and $B$ be C$^*$-algebras.
An \emph{unbounded Kasparov module} over $(A,B)$ is a pair $(\cA,E,D)$ consisting of
\begin{enumerate}[(i)]
\item a dense $*$-subalgebra $\cA$ of $A$,
\item a right Hilbert module $E$ over $(A,B)$,
\item an densely defined linear operator $D$ on $E$,
\end{enumerate}
such that 
\begin{enumerate}[(i)]
\item $D$ is regular,
\item $\cA$ acts on $\dom D$ and $\bar{[D,a]}$ is adjointable for $a\in\cA$,\hfill(Lipschitz condition)
\item $D$ is self-adjoint,
\item $a(D+i)^{-1}$ is compact for $a\in\cA$.\hfill(compact resolvent condition)
\end{enumerate}
We say an unbounded Kasparov module $(\cA,E,D)$ is \emph{even} if it is equipped with a $\Z/2\Z$-grading such that $D$ is odd, and \emph{odd} otherwise.
An unbounded Kasparov module $(\cA,E,D)$ is called \emph{countably generated} if $E$ is countably generated over $B$.
\end{prb}



\chapter{Cyclic theory}



Let $A$ be an associative algebra over a field $k$.
The cyclic bicomplex is defined by $CC_{p,q}(A):=A^{\otimes(q+1)}$ for $p,q\in\Z_{\ge0}$.
\[\begin{tikzcd}
&\vdots\dar{b}&\vdots\dar{-b'}&\vdots\dar{b}\\
\cdots\rar{1-\lambda}&A^{\otimes3} \rar{Q}\dar{b} & A^{\otimes3} \rar{1-\lambda}\dar{-b'} & A^{\otimes3} \dar{b}\\
\cdots\rar{1-\lambda}&A^{\otimes2} \rar{Q}\dar{b} & A^{\otimes2} \rar{1-\lambda}\dar{-b'} & A^{\otimes2} \dar{b}\\
\cdots\rar{1-\lambda}&A \rar{Q} & A \rar{1-\lambda} & A
\end{tikzcd}\]

$b:A^{\otimes(q+1)}\to A^{\otimes q}$ be the Hochschild operator
\[b(a_0\otimes\cdots\otimes a_q):=b'(a_0,\cdots,a_q)+(-1)^qa_qa_0\otimes\cdots\otimes a_{q-1},\]
where
\[b'(a_0\otimes\cdots\otimes a_q):=\sum_{j=0}^{q-1}(-1)^ja_0\otimes\cdots\otimes a_ja_{j+1}\otimes\cdots\otimes a_q.\]
Connes signed cyclic permutation
\[\lambda(a_0\otimes\cdots\otimes a_q):=(-1)^qa_q\otimes a_0\otimes\cdots\otimes a_{q-1},\]
and $Q:=\sum_{j=0}^q\lambda^j$ is computed as
\[Q(a_0\otimes\cdots\otimes a_q):=\sum_{j=0}^q(-1)^{jq}a_j\otimes\cdots\otimes a_q\otimes a_0\otimes\cdots\otimes a_{j-1}.\]


Rows are exact at $p>0$.
Let $D_n$ be the total complex of the cyclic bicomplex $CC_{p,q}$.
The homology group of $D_n$ is denoted by $HC_nA$.

Taking the first two columns at $p=0$ and $p=1$, the homology group of the total complex is the \emph{Hochschild homology group}.
When $A$ is unital, then the second column at $p=1$ is exact so that we obtain the usual Hochschild homology only using the first column $p=0$.



Connes SBI sequence


A \emph{normalized $(b,B)$-cochain} is a finite collection $\{\phi_m\}$ of continuous multilinear functionals on $\cA$









\chapter{}


\section{Operator spaces}

\begin{prb}[Guess on the general role of $\cK$]
We define $\cK$ as the C$^*$-algebra of compact operators on a separable Hilbert space $\ell^2$, equipped with
\begin{enumerate}[(i)]
\item canonical inclusions $M_n(\C)\to\cK$ into the upper left corners,
\item a $\cK$-bimodule $*$-isomorphism $M_2(\cK)\to\cK$,
\end{enumerate}
as an internal structure of $\cK$.

A \emph{stabilized Banach space} is a Banach bimodule $\tilde E$ over $\cK$ such that there is a $\cK$-bimodule isomorphism $M_2(\tilde E)\to\tilde E$. (compatibility with $M_2(\cK)\to\cK$? uniqueness of bimodule isomorphism?)
There is a categorical equivalence between operator spaces with complete isometries and stabilized Banach spaces...?

Let $A$ be a C$^*$-algebra together with a $*$-isomorphism $M_2(A)\to A$.


A stabilized Banach bimodule $\tilde E$ over $\cK\otimes A$ and $\cK\otimes B$ such that there is a bimodule isomorphism $M_2(\tilde E)\to\tilde E$...?
countably generatedness....

A stabilized C$^*$-algebra is a C$^*$-algebra $\tilde A$ equipped with $\cK\to M(\tilde A)$ and a $\cK$-bimodule $*$-isomorphism $M_2(\tilde A)\to\tilde A$...?
A stable C$^*$-algebra has many stabilized structures.

stabilization functor is left adjoint to the forgetful functor from stable C$^*$-algebras to C$^*$-algebras.
\end{prb}



\begin{prb}[Lemmas for stable representations]
\begin{enumerate}[(i)]
\item $B(\ell^2\otimes H)=M(\cK\otimes K(H))$?
\item $B(\ell^2\otimes H)=B_\C(\cK\otimes H)$ or $B_\cK(\cK\otimes H)$?
\item $\cK\otimes B(H)=K(\ell^2\otimes H)$?
\item 
\end{enumerate}

representation as a $*$-homomorphism $A\to M(K(H))$.
\end{prb}

\begin{prb}[Strict topology and generalized von Neumann algebras]
Is it possible to define the $\sigma$-strict topology as the induced topology on $B(E)$ from the strict topology of $B(\cK\otimes E)$?

Dual pair theory on $\sigma$-strictly continuous linear functionals?
It cannot be a weak$^*$ topology. Then, what is it?
\end{prb}



\section{Right Hilbert bimodules}

\begin{prb}[Structures on Banach bimodules]
Let $E$ be a Banach bimodule over C$^*$-algebras $A$ and $B$.
We say $E$ is \emph{right Hilbert} if there is a right inner product is a map $\<\cdot,\cdot\>:E\times E\to B$ such that the left action of $A$ is adjointable.
The \emph{imprimitivity bimodule} is a Banach bimodule $E$ over $A$ and $B$ which is both-sided Hilbert and satisfies ${}_A\<\xi,\eta\>\zeta=\xi\<\eta,\zeta\>_B$ for $\xi,\eta,\zeta\in E$.
On an imprimitivity bimodule $E$ over $A=B$, there is an involution $E\to E:\xi\mapsto $
The operator-valued inner product is unique if it exists (Phillips-Weaver)
\end{prb}

\begin{prb}[Interior tensor products of Banach bimodules]
\end{prb}

\begin{prb}[Stabilized Banach spaces]
\end{prb}







\section{Kasparov-Stinespring construction}





\chapter{}




\section{Derivations}
\begin{prb}[Universal unbounded derivations]
\end{prb}

\begin{prb}[Dirac operators]

A commutator by the Dirac operator is not defined on the whole algebra $A$.
\end{prb}


$\Omega B$ is the universal differential non-negatively graded algebra.



unbounded derivations $[D_2,-]$.


An inner product on $\Omega^1B\subset B\otimes B$ is the non-commutative analogue of Riemannian structure on $B$.
The moduli space of Hilbert structures on $\Omega^1B$?
Is there a canonical Banach structure on $\Omega^1B$?
The norm on $\Omega^1_{D_2}B$ is not fixed.



The universal derivation $B\to\Omega^1(B)$ is odd.
We want to construct a right inner product on $\Omega^1B$

Dual Hilbert module?


It may be impossible to assume $B$ is a C$^*$-algebra...
For example, it would be an operator algebra.


For a derivation $\delta:\cA\to\cK\otimes A$, is it possible to consider the crossed product by $\delta$?

\begin{prb}[Inner derivations]
\end{prb}

\section{Connections}


\begin{prb}[Connections from derivations]
Let $E$ be a Hilbert module $E$ over a C$^*$-algebra $B$.
Let $\nabla:\dom\nabla\subset E\to E\otimes\Omega^1(B)$ be a connection with respect to the universal derivation $d:B\to\Omega^1(B)$.
By the universality of $d$, to any derivation $\delta:B\to\Omega$ to a bimodule $\Omega$ over $B$ we can associate a connection $\nabla_\delta:E\to E\otimes_B\Omega$ satisfying
\[\nabla_\delta(\xi b)=\nabla_\delta(\xi)b+\xi\otimes\delta(b),\qquad \xi\in E,\ b\in B.\]
As a special case, if $\delta=X:B\to B$ is a vector field for $B=C(M)$, then we have $\nabla_X:E\to E$.

What is the exact definition of bimodule $\Omega$ in here?

A derivation cannot be defined on the whole algebra $B$.
A connection cannot be defined on the whole module $E$.
We need to consider operator algebras $\cB$ and $\cE$...?





\end{prb}

\begin{prb}[Bimodule of 1-forms]
Let $B$ be an associative algbera.
The $B$-bimodule of 1-forms is defined by $\Omega^1(B):=\tilde B\otimes B$ as vector spaces, whose elements $b_0\otimes b_1\in\Omega^1(B)$ will be denoted by $b_0\,db_1$, together with the universal derivation $d:B\to\Omega^1(B):b\mapsto db$.
The $B$-bimodule structure on $\Omega^1(B)$ is given by
\[b(b_0\,db_1):=bb_0\,db_1,\qquad(b_0\,db_1)b:=b_0\,d(b_1b)-b_0b_1\,db,\qquad b_0\in\tilde B,\ b_1,b\in B.\]

We have the short exact sequence of $B$-bimodules
\[0\to\Omega^1(B)\to\tilde B\otimes\tilde B\to\tilde B\to0,\]
where
\[\Omega^1(B)\to\tilde B\otimes\tilde B:b_0\,db_1\mapsto b_0\otimes b_1-b_0b_1\otimes1,\qquad\tilde B\otimes\tilde B\to\tilde B:b\otimes b'\mapsto bb'.\]
\end{prb}




\begin{prb}
Taking the interior tensor $E_1\otimes_B-$, we have a short exact sequence of right $B$-modules
\[0\to E_1\otimes_B\Omega^1(B)\to E_1\otimes\tilde B\to E_1\to0,\]
(why is this still exact after tensoring? projectivity of $E_1$?)
where
\[E_1\otimes_B\Omega^1(B)\to E_1\otimes\tilde B:\xi_1\otimes_Bb_0\,db_1\mapsto\xi_1b_0\otimes b_1-\xi_1b_0b_1\otimes1,\qquad E_1\otimes\tilde B\to E_1:\xi_1\otimes b\mapsto \xi_1b.\]
Then, there is a one-to-one correspondence
\[\{\text{right $B$-linear splits $s:E_1\to E_1\otimes\tilde B$}\}\leftrightarrow
\{\text{$d$-connections $\nabla:E_1\to E_1\otimes_B\Omega^1(B)$}\},\]
given by
\[\nabla(\xi_1)=s(\xi_1)-\xi_1\otimes1,\qquad\xi_1\in E_1.\]
The connection proprty can be checked as
\[\nabla(\xi_1b)-\nabla(\xi_1)b=\xi_1\otimes b-\xi_1b\otimes1=\xi_1\otimes_Bdb.\]

\end{prb}

\begin{prb}
We start from the exact sequence of vector spaces
\[E_1\otimes_B\Omega^1(B)\otimes_BE_2\to E_1\otimes E_2\to E_1\otimes_BE_2\to0.\]
(left exact? I don't know yet)
Since $D_1$ is right $B$-linear but $D_2$ is not left $B$-linear, so we choose a split
\[s\otimes_B1:E_1\otimes_BE_2\to E_1\otimes E_2:\xi_1\otimes_B\xi_2\mapsto s(\xi_1)\xi_2,\]
where $s:E_1\to E_1\otimes\tilde B$, to define a twisted operator $1\otimes_\nabla D_2$ on $E_1\otimes_BE_2$.
Let $\Omega^1_{D_2}(B)\subset\End_C(E_2)$ be the $B$-bimodule generated by $b_0[D_2,b_1]$ for $b_0\in\tilde B$ and $b_1\in B$,


\[\left\{\begin{tabular}{c}
right $B$-linear splits $s:E_1\to E_1\otimes\tilde B$\\such that (?)
\end{tabular}\right\}\leftrightarrow
\{\text{$D_2$-connections $\nabla:E_1\to E_1\otimes_B\Omega^1_{D_2}(B)$}\},\]
given by
\[\nabla(\xi_1)=(1\otimes D_2)(s\otimes_B1)(\xi_1\otimes_B-)-\xi_1\otimes D_2.\]
(How can we formally establish such correspondence? For example, from $D_2$-connection, can we retrieve $s$?)
The connection property can be checked as
\[\nabla(\xi_1b)-\nabla(\xi_1)b=\xi_1\otimes D_2b-\xi_1b\otimes D_2=\xi_1\otimes_B[D_2,b].\]
(We must have the right $B$-linear inclusion
$\Omega^1_{D_2}(B)\subset\tilde B\otimes\End_C(E_2)$ with $b_0[D_2,b_1]\mapsto b_0\otimes D_2b_1-b_0b_1\otimes D_2$. Is this well-defined?)
We can also see
\begin{align*}
(1\otimes_\nabla D_2)(\xi_1\otimes_B\xi_2)
&=\xi_1\otimes D_2\xi_2+\nabla(\xi_1)\xi_2\\
&=\xi_1\otimes D_2\xi_2+(1\otimes D_2)s(\xi_1)\xi_2-\xi_1\otimes D_2\xi_2\\
&=(1\otimes D_2)(s\otimes_B1)(\xi_1\otimes_B\xi_2)
\end{align*}
\end{prb}

\begin{prb}

$\Omega^2_{D_2}(B)$

space of connctions
\end{prb}




\[\begin{tikzcd}[sep=small]
E_1\otimes_B\Omega^1(B)\rar&
E_1\otimes\tilde B \rar\dar&
E_1\otimes_B\tilde B\rar\dar[dashed]\lar[bend right,dashed,swap]{s}&0\\
E_1\otimes_B\Omega^1(B)\rar&
E_1\otimes\tilde B\rar&
E_1\otimes_B\tilde B\rar&0
\end{tikzcd}\]
\[\begin{tikzcd}[sep=small]
E_1\otimes E_2 \rar\dar{1\otimes D_2}&
E_1\otimes_BE_2\rar\dar[dashed]{1\otimes_\nabla D_2}\lar[bend right,dashed,swap]{s\otimes_B1}&0\\
E_1\otimes E_2\rar&
E_1\otimes_BE_2\rar&0
\end{tikzcd}\]


, we may have to perturb the split map $s_1:E_1\to E_1\otimes\tilde B$ to make $D_1\otimes1+1\otimes_{\nabla^1}D_2$ self-adjoint regular.

I think we can compute the moduli space of connections when $E_1$ can be stabilized.
$\tilde B\to\tilde B\otimes\tilde B:b\mapsto1\otimes b$ induces the \emph{Grassmann connection}.
\[(D_1\otimes1)(1\otimes_{\nabla^1}D_2)(\xi_1\otimes\xi_2)=(-1)^{\deg\xi_1}(D_1\xi_1\otimes D_2\xi_2+(D_1\otimes1)\nabla_{D_2}^1(\xi_1)\xi_2)\]
\[(1\otimes_{\nabla^1}D_2)(D_1\otimes1)(\xi_1\otimes\xi_2)=(-1)^{\deg\xi_1+1}(D_1\xi_1\otimes D_2\xi_2+\nabla_{D_2}^1(D_1\xi_1)\xi_2)\]

\[[a,b]=ab-(-1)^{\deg a\deg b}ba\]
\[[D_1\otimes1,1\otimes_{\nabla^1}D_2](\xi_1\otimes\xi_2)=\]



\section{Kasparov category}

\begin{prb}[Definition of Kasparov modules]
Let $A$ and $B$ be C$^*$-algebras.
A \emph{differential Kasparov module} over $(A,B)$ is a triple $(E,D,\nabla)$ consisting of
\begin{enumerate}[(i)]
\item a right Hilbert bimodule $E$ over $(A,B)$,
\item odd self-adjoint regular $D$ on $E$ with $[D,a]\in B(E)$ and $a(D+i)^{-1}\in K(E)$ for $a\in\cA$,
\item $\nabla:E\to E\otimes_B\Omega^1B$ is a connection for the universal derivation.
\end{enumerate}

Necessary conditions:
\begin{itemize}
\item $\nabla^2:\dom\nabla^2\subset E_2\to E_2\otimes_C\Omega^1(C)$ defines a derivation $[\nabla^2,-]:B\to B(E_2,E_2\otimes_C\Omega^1(C))$ to a bimodule over $B$.
\end{itemize}


\begin{parts}
\item existence of connection? (we don't care which does not admit a connection for the universal derivation?)
\item definition of smoothness?
\end{parts}
\end{prb}


\begin{prb}[Homotopy equivalences]
How can we restrict an unbounded operator to define $D(0)$ and $D(1)$ on the boundary?

How can we construct a connection $E\to E\otimes\Omega^1(B[0,1])$?
operator-algebraic formulation of transgression?
\begin{parts}
\item transitivity of equivalence?
\item nice representatives?
\end{parts}
\end{prb}

\begin{prb}[Smooth equivalences]

\begin{parts}
\item transitivity of equivalence?
\item nice representatives?
\end{parts}
\end{prb}

\begin{prb}[Composition product]
Let $A$, $B$, and $C$ be C$^*$-algebras, and let $(E_1,D_1,\nabla^1)$ and $(E_2,D_2,\nabla^2)$ be differential Kasparov cycles from $A$ to $B$ and from $B$ to $C$, respectively.

Since the connection $\nabla^2:\dom\nabla^2\subset E_2\to E_2\otimes_C\Omega^1(C)$ defines a derivation $[\nabla^2,-]:B\to B(E_2)\otimes_C\Omega^1(C)$ to a bimodule over $B$, we have a connection
\[\nabla^1_{\nabla^2}:E_1\to E_1\otimes_BB(E_2,E_2\otimes_C\Omega^1(C))\]
satisfying
\[\nabla^1_{\nabla^2}(\xi_1b)=\nabla^1_{\nabla^2}(\xi_1)b+\xi_1\otimes[\nabla_2,b].\]
The product connection is
\[\nabla^{12}:E_{12}\to E_{12}\otimes_C\Omega^1(C)\]
defined such that
\[\nabla^{12}(\xi_1\otimes\xi_2):=\nabla^1_{\nabla_2}(\xi_1)\xi_2+\xi_1\otimes\nabla^2(\xi_2)\]

Since the Dirac operator $D_2:E_2\to E_2$ defines a derivation $[D_2,-]:B\to B(E_2)$ to a bimodule over $B$, we have a connection
\[\nabla^1_{D_2}:E_1\to E_1\otimes_BB(E_2),\]
satisfying
\[\nabla^1_{D_2}(\xi_1b)=\nabla^1_{D_2}(\xi_1)b+\xi_1\otimes[D_2,b].\]
The product Dirac operator is
\[D_{12}:=D_1\otimes1+1\otimes_{\nabla^1}D_2,\]
where $1\otimes_{\nabla^1}D_2$ is the twisted Dirac operator defined such that
\[(1\otimes_{\nabla^1}D_2)(\xi_1\otimes\xi_2):=(-1)^{\partial\xi_1}(\xi_1\otimes D_2\xi_2+\nabla^1_{D_2}(\xi_1)\xi_2).\]

\end{prb}


\begin{prb}[Exterior product]
\end{prb}


\section{Higson characterization}



\begin{prb}[Continuity]
\end{prb}

\begin{prb}[Homotopy invariance]


Homotopy invariance like:
if $A(0)$ and $A(1)$ are homotopic, then $KK(A(0),B)\cong KK(A(1),B)$.
(I think homotopy of C$^*$-algebras can be defined as a continuous field of C$^*$-algebras over the parameter space $I$)

\end{prb}


\begin{prb}[Split exactness]
additivity follows
\end{prb}

\begin{prb}[Stability]
\end{prb}


\begin{prb}[Higson universality]
universal among homotopy invariant split exact stable functors to additive categories.
\end{prb}


\section{Relation to other pictures}
\begin{prb}[Bounded picture]
\end{prb}
\begin{prb}[Unbounded picture]
\end{prb}













\part{Geometric applications}







\chapter{Index pairing}



\section{Spectral triples}


\begin{prb}[Elliptic operators]
Let $X$ be an $n$-dimensional smooth manifold.
Let $S=S_+\oplus S_-$ be a graded smooth complex vector bundles on $X$.
We will put a measure on $X$ and a Hermitian structure on $S$ to consider the Hilbert space of sections.
Let $D:\Gamma^\infty(S_+)\to\Gamma^\infty(S_-)$ be a \emph{linear partial differential operator}, a linear operator that is locally a polynomial in the operators $\partial_i$ with smooth matrix-valued coefficients $a_I\in\Gamma^\infty(\Hom(S_+,S_-))$ for each multi-index $I$.
By the Peetre theorem, a local linear operator is a linear partial differential operator.
The \emph{symbol} of $D$ can be defined equivalently as either of
\begin{enumerate}[(i)]
\item a linear bundle map $\sigma(D):T^*X\to\Hom(S_+,S_-)$ over $X$ such that
\[\sigma(D)|_{(x,\xi)}:=\sum_{i=1}^na_i(x)\frac{\partial}{\partial x^i},\qquad(x,\xi)\in T^*X,\]
\item a $C^\infty(X)$-module map $\sigma(D):\Omega^1(X)\to\Gamma^\infty(\Hom(S_+,S_-))$ such that
\[\sigma(D)(df):=[D,f]=\sum_{i=1}^na_i\partial_if,\qquad f\in C^\infty(X),\]
\item a smooth section $\sigma(D)\in\Gamma^\infty(T^*X,\pi^*\Hom(S_+,S_-))=\Gamma^\infty(\Hom(T^*X,\Hom(S_+,S_-)))=\Gamma^\infty(TX\otimes S_+^*\otimes S_-)$ such that blabla.
\end{enumerate}
When $S_+$ and $S_-$ have same rank, we say $D$ is \emph{elliptic} if its symbol $\sigma(D)_x:T_x^*X\to\Hom(S_{+,x},S_{-,x})$ at each $x\in X$ is invertible for all $\xi\in T_x^*X\setminus\{0\}$.
\end{prb}

\begin{prb}[Topological indices]
Let $X$ be a locally compact Hausdorff space.
The \emph{Thom space} $\mathrm{Th}(V)$ of a vector bundle $V\to X$ is the homotopy class of the pair $(V,V_0)$, where $S_0$ is the complement of the zero section in $V$, or the pair of the disk bundle and the sphere bundle.
It is homotopic to the one-point compactification of the total space $S$ if $X$ is compact.

Let $X$ be a smooth manifold.
For an embedding $X\hookrightarrow\R^n$, the \emph{topological index} is defined as the composition
\[K(T^*X)\xrightarrow{\sim}K(T^*N)\to K(T^*\R^n)\cong\Z,\]
where the first map is the Thom isomoprhism established because $T^*N$ can be given a complex vector bundle structure over $T^*X$, and the second map is the induced map of the quotient map $(T^*\R^n_+,*)\to(T^*N_+,*)$.

\end{prb}

\begin{prb}[]
Let $X$ be a smooth manifold.
An elliptic operator $D_+:\Gamma^\infty(S_+)\to\Gamma^\infty(S_-)$ defines a linear bundle map $\sigma(D_+):T^*X\to\Hom(S_+,S_-)$, and the ellipticity of $D$ implies that the symbol defines a K-theory class
\[[\{\ker\sigma(D)_{(x,\xi)}\}]-[\{\ker\sigma(D)^*_{(x,\xi)}\}]\in K^0(\operatorname{Th}(T^*X)).\]

An elliptic operator $D_+:\Gamma^\infty(S_+)\to\Gamma^\infty(S_-)$ defines a K-theory class of $K^0(\operatorname{Th}(T^*X))$ and a K-homology class of $K^0(C_0(X))$.



\end{prb}




Let $S\to X$ be a super-Hermitian bundle over a closed Riemannian manifold.
Then, we have a unital super-representation $C(X)\to B(L^2(S))$ given by multiplication.
For example, if $X$ is spin, then it canonically defines a super-Hermitian bundle $S\to X$ called the spinor bundle, and every Dirac operator on it gives a same K-homology class, i.e.~a spin structure (in fact, also a spin$^c$ structure) on $X$ canonically gives rise to a class of $K_0(X)$, called the fundamental class.

A \emph{Dirac operator} on $S$ is an odd differential operator $D:\Gamma^\infty(S)\to\Gamma^\infty(S)$ such that the symbol satisfies $\sigma(x,\xi)=-\|\xi\|$ on each fiber $S_x$. ($D^2=\Delta$)
Then, every Dirac operator is elliptic, and hence Fredholm.

A \emph{Dirac bundle} on $X$ is a super-Hermitian bundle $S$ together with a real bundle map $c:\Omega^1(X)\to\Gamma(\End_\C(S))$, called the \emph{Clifford multiplication}, such that the square of $\xi$ is $-\|\xi\|$ in $\End_\C(S)$.
In other words, a Dirac bundle structure on $S$ is an equivalence class of Dirac operators having same symbol.(differs to Lawson-Michelson)


A \emph{Dirac type operator} on $S$ is an odd differential operator $D$ on $\Gamma^\infty(S)$ of first order such that $[D,f]=c(df)$ for $f\in C^\infty(X)$.
Automatically elliptic, and hence Fredholm.
For example, when $S_\pm:=\Lambda_\pm T^*X$ and $V=V_+$ is a vector bundle on $X$, if we have a connection $\nabla^V:\Gamma^\infty(V)\to\Gamma^\infty(V\otimes T^*X)$ on $V$, then there is the \emph{twisted Dirac type operator}
\[1\otimes_{\nabla^V}D_+:=(1\otimes c)(\nabla^V\otimes1+1\otimes\nabla^S):\Gamma^\infty(V\otimes S_+)\to\Gamma^\infty(V\otimes S_-),\]
where $c\nabla^S=D_+$.
We can write
\[(1\otimes_{\nabla^V}D_+)(\xi\otimes\psi):=\xi\otimes(D_+\psi)+\nabla^V(\xi)\psi,\qquad\xi\in\Gamma^\infty(V),\ \psi\in\Gamma^\infty(S).\]


\begin{prb}[Spectral triples]
A \emph{spectral triple} is an unbounded Kasparov module $(H,D)$ over $(A,\C)$ together with a dense $*$-subalgebra $\cA$ in the domain of $[D,-]$.
\end{prb}

\begin{prb}[Spectral triples by elliptic operators]
Let $X$ be a closed smooth manifold.
Let $D_+:\Gamma^\infty(S_+)\to\Gamma^\infty(S_-)$ be an elliptic operator on a graded Hermitian bundle $S\to X$.
Then, we have an even spectral triple
\[(C^\infty(X),L^2(S),D).\]
\end{prb}


\begin{prb}[Spectral triples by Hodge-de Rham Dirac operators]
Let $X$ be a closed Riemannian manifold.
By specifying a Riemannian metric, the space $\Gamma(\Lambda T^*X)$ of continuous sections admits a canonical right Hilbert module structure over $C^\infty(X)$.
Fix a faithful state on $C(X)$ induced by the normalized volume form on $X$ and consider its Gelfand-Naimark-Segal representation $L^2(X)$.
Then, the right Hilbert bimodule $\Gamma(\Lambda T^*X)$ over $C(X)$ can be tensor producted with the representation $L^2(X)$ of $C(X)$ to provide a representation $L^2(\Lambda T^*X)$ of $C(X)$.

Consider the even spectral triple
\[(C^\infty(X),L^2(\Lambda T^*X),d+d^*),\]
where, $d+d^*$ is the \emph{Hodge-de Rham Dirac operator}.
The \emph{Hodge-de Rham Laplacian} $\Delta$.

Dolbeault operator with respect to the almost complex structure on $T^*X$?
\end{prb}


\begin{prb}[Spectral triples by signature operators]
\end{prb}

\begin{prb}[Spectral triples by spin$^c$-Dirac operators]
\end{prb}

\begin{prb}[Spectral triples by Dolbeault-Dirac operators]
\end{prb}







\section{}



\begin{prb}[Even index pairing]
For a projection $p\in M_k(A)$ for some $k\ge0$ and a Fredholm module $(H,F)$ over $A$,
\[\begin{array}{ccccc}
K_0(A)&\times&K^0(A)&\to&\Z\\
\left[(p(\ell^2\otimes A),0)\right]&&\left[(H,F)\right]&\mapsto&\Ind(p(1\otimes F_+)p\in B(pH^k))
\end{array}\]
\end{prb}
\begin{pf}
Well-defineness of index pairing.

Coincidence with Kasparov product.
\end{pf}


\begin{prb}[Twisted Dirac operators]
\end{prb}

\begin{prb}
Hodge de Rham: Euler characteristic

signature: L-genus? Pontryagin-Hirzebruch class?

spin$^c$: \^A-genus
\end{prb}

What is the relation between the index theoretic compression $p(1\otimes F_+)p$ and the twisted Dirac operator $1\otimes_\nabla F_+$?

\begin{prb}[Atiyah-Singer index theorem of twisted Dirac operators]
Let $X$ be a compact manifold.
Let $V$ be a complex vector bundle over $X$, and $D_+:\Gamma^\infty(S_+)\to\Gamma^\infty(S_-)$ be a Dirac type operator.
Then, for any connection $\nabla$ on $V$, we have
\[\Ind(1\otimes_\nabla D_+)=\int_X\operatorname{Ch}(V)\,\hat A(S),\]
where $\mathrm{Ch}(V)$ is the Chern character of $V$ and $\hat A(S)$ is

\end{prb}





\section{Generalized Toeplitz operators}



\begin{prb}[Fredholm modules by Toeplitz operators]
Consider the odd Fredholm module
\[(C^\infty(S^1),L^2(S^1),F),\]
where $F:=2P-1$ and $P:L^2(S^1)\to H^2(S^1)$ is the projection onto the Hardy space.

Toeplitz operator $T_a:=PaP$ for $a\in C(S^1)$.

Generalized Toeplitz operator $PuP$ for $u\in U_k(A)$.
\end{prb}




\begin{prb}[Mishchenko and Fomenko index theorem]
\end{prb}








\chapter{Index formulae}

\section{Chern characters}

\begin{prb}[Summability]
Let $A$ be a C$^*$-algebra, and $(H,F)$ be a Fredholm module over $A$.
For $p\in\Z_{\ge1}$, we say $(H,F)$ is \emph{$(p+1)$-summable} if $[F,a]\in L^{p+1}(H)$ for $a\in A$.

\end{prb}

Let $A$ be a C$^*$-algebra.
Let $(H,F)$ be a normalized $(p+1)$-summable Fredholm module over $A$.
For $n\ge p$ with the same parity of the Fredholm module $(H,F)$, we define a cyclic cocycle $\mathrm{Ch}$ by
\[\mathrm{Ch}_n(H,F)(a_0,\cdots,a_n):=\frac{\lambda_n}2\Tr(\gamma[F,a_0]\cdots[F,a_n]).\]


\section{Spectral flows}

We have $\bar{F(N)}=K(N)$ since $0\le x\le1$ with $x\in F(x)$ has $\tau(x)<\tau(s_l(x))<\infty$, and it implies $x\in K(N)$.

\begin{prb}[Breuer-Fredholm operators]
Let $N$ be a semi-finite von Neumann algebra with a faithful semi-finite normal trace $\tau$.
Let $F(N)$ be the $*$-ideal of $N$ whose support projections have finite traces, and let $K(N)$ be the norm closed ideal of $N$ generated by projections of finite traces, and $\pi:N\to Q(N):=N/K(N)$ be the canonical surjection.
An element $f\in N$ is called \emph{Breuer-Fredholm} if $f$ and $f^*$ has kernel projections of finite trace, and its \emph{index} is defined as the difference of traces of kernel projections $\Ind(f):=\tau(1-s_r(f))-\tau(1-s_r(f^*))$.
We do not require the closed range.
\begin{parts}
\item $f\in N$ is Breuer-Fredholm of index zero if $1-f\in K(N)$.
\item $f\in N$ is Breuer-Fredholm if and only if $\pi(f)$ is invertible in $Q(N)$.
\item $\Ind(f^*)=-\Ind(f)$ and $\Ind(f_1f_2)=\Ind(f_1)+\Ind(f_2)$.
\end{parts}
\end{prb}
\begin{pf}
(a)
There is $g\in N$ such that $1-g\in F(N)$ and $\|g-f\|<1$.
If we let $s:=1-(g-f)$ and $h:=1-(1-g)s^{-1}$, then $h\in N$ satisfies $1-h\in F(N)$.

We have a Breuer-Fredholm element $h$ by $1-s_r(h)\le s_l(1-h)$ and $1-s_l(h)\le s_r(1-h)$.
Since $e:=s_l(1-h)\vee s_r(1-h)$ is a finite projection such that $1-s_r(h)\le e$ and $s_r(e-(1-h))\le e$, and since $s_r(h)-(1-e)$ is a projection fixing $e-(1-h)$ from right and $1-e+s_r(e-(1-h))$ is a projection fixing $h$ from right, we have
\[1-s_r(h)=e-s_r(e-(1-h))\sim e-s_l(e-(1-h))=1-s_l(h).\]

From $hs(1-s_r(hs))=0$ we have $hs_l(s(1-s_r(hs)))=0$, and it implies $s_r(h)\le 1-s_l(s(1-s_r(hs)))$ and
\[1-s_r(h)\ge s_l(s(1-s_r(hs))\sim s_l(1-s_r(hs))=1-s_r(hs).\]
Similarly we can show $1-s_r(h)\precsim1-s_r(hs)$, we have $1-s_r(hs)\sim1-s_r(h)$.
We finally have
\begin{align*}
1-s_r(f)&=1-s_r(s-(1-f_0))=1-s_r(hs)\sim1-s_r(h)\\
&\sim1-s_l(h)=1-s_l(hs)=1-s_l(s-(1-f_0))=1-s_l(f),
\end{align*}
so that $\tau(1-s_r(f))=\tau(1-s_r(h))<\infty$ and the index of $f$ is zero.


$xy=0$ implies $|x|y=0$ since $y^*|x|^2y=|xy|^2=0$,
$|x|y=0$ implies $s_r(x)y=0$ since $s_r(x)$ can be computed as a series of $|x|$.



(b)
Suppose $f$ is Breuer-Fredholm.
We prove $\pi(f)$ is invertible.

Find a finite projection $e$ in $N$ such that $s_r(f)-s_r((1-e)f)\sim s_l(f)\wedge e$.

$s_r((1-e)f)\sim s_l((1-e)f)=1-e$
\end{pf}

\begin{prb}
Let $N$ be a semi-finite von Neumann algebra with a faithful semi-finite normal trace $\tau$.

(a)
Let $p$ and $q$ be projections in $N$ such that $\|\pi(p)-\pi(q)\|<1$.
Then, $p-s_r(qp)$ and $q-s_r(pq)$ have finite traces so that we can define
\[\Ind(p,q)=\tau(p-s_r(qp))-\tau(q-s_r(pq)),\]
also called the \emph{essential codimension}.

(b)
Let $\{f_t\}$ be a norm continuous path of Breuer-Fredholm operators in $N$, parametrized by $t\in[0,1]$.
Then, $t\mapsto\pi(p_t)$ is norm continuous, where $p_t:=1_{[0,\infty)}(f_t)$.

(c)
Now we can define for a sufficiently fine finite partition $\{t_i\}$ of $[0,1]$ the \emph{spectral flow} by
\[\sf(\{f_t\}):=\sum_i\Ind(p_{t_{i-1}},p_{t_i}),\]
which does not depend on the choice of the partition $\{t_i\}$.
\end{prb}
\begin{pf}
(a)
Since $\|\pi(p)-\pi(pqp)\|\le\|\pi(p)-\pi(q)\|<1$ implies $\pi(pqp)$ is invertible in $Q(pNp)$, $pqp$ is Breuer-Fredholm in the corner $pNp$.
Then, $p-s_r(qp)\le p-s_r(pqp)$ has finite trace.
Similar for $q-s_r(pq)$.

(b)
Let $f\in N$ be a Breuer-Fredholm element in $N$ so that $\pi(f)$ is invertible in $Q(N)$.
Let $\e>0$ such that $[-\e,e]\cap\sigma(\pi(f))=\varnothing$.
If we take $u,l\in C_b(\R)$ such that $l\le1_{[0,\infty)}\le u$ on $\R$ and $l=1_{[0,\infty)}=u$ on $\R\setminus[-\e,\e]$.
Then, we have $1_{[0,\infty)}(\pi(f))=\pi(1_{[0,\infty)}(f))$ by
\[1_{[0,\infty)}(\pi(f))=l(\pi(f))=\pi(l(f))\le\pi(1_{[0,\infty)}(f))\le\pi(u(f))=u(\pi(f))=1_{[0,\infty)}(\pi(f)).\]
Therefore, since $t\mapsto\pi(f_t)$ is norm continuous path of invertible elements, the positive part
\[t\mapsto1_{[0,\infty)}(\pi(f_t))=\pi(1_{[0,\infty)}(f_t))=\pi(p_t)\]
is also norm continuous.

(c)
It suffices to show $\Ind(p,r)=\Ind(p,q)+\Ind(q,r)$ for projections $p,q,r\in N$ satisfying $\|\pi(p)-\pi(q)\|<\frac12$ and $\|\pi(q)-\pi(r)\|<\frac12$.
Since $\|pi(r-rqpr)\|<1$, we can find $k\in K(rNr)$ such that $\|r-rqpr+k\|<1$.
Therefore, $rqpr-k$ is invertible in $rNr$ so that $\Ind_r(rqpr)=\Ind_r(rqpr-k)=0$.
\[0=\Ind_r(rqpr)=\Ind_{q,r}(rq)+\Ind_{p,q}(qp)-\Ind_{p,r}(rp).\]
\end{pf}


\begin{prb}[Continuous fields of Fredholm operators]
Let $X$ be a compact Hausdorff space.
A continuous field of Fredholm operators $\{F_x\in B(\ell^2)\}_{x\in X}$ has the same data of the Kasparov module $(\ell^2\otimes C(X),F)$ on the standard Hilbert module with some additional conditions... in the sense that
\[F_+(x)=F_x:B(\ell^2\otimes C(X)\otimes_{\delta_x}\C)\cong B(\ell^2),\qquad x\in X.\]

It defines a K-theory class by the graded algebraically finitely generated projective Hilbert module $\ker F$ over $C(X)$. (really projective?)
\end{prb}


The parametrized Hamiltonian $\{A(x)\}_{x\in X}$ defined on a fixed Hilbert space $H$ naturally defines a K-theory class of $K_1(C_0(X))$, while the Dirac operator $-i\partial_x$ defines a K-homology class of $K^1(C_0(X))$, the Bott class.
We consider the \emph{Dirac-Schr\"odinger operator}
\[D_A=-i\partial_x+A(x):\dom D_A\subset L^2(X,H)\to L^2(X,H).\]
(signs and grading?)
Thus the spectral flow theorem states that the Dirac-Schr\"odinger operator is the unbounded Kasparov product, and in particular its index defines a class in $KK(\C,\C)$ is the spectral flow.
\[\begin{array}{ccccc}
K_0(C(S^1))&\times&K^0(C(S^1))&\to&\Z\\
\left[(H\otimes C(S^1),A?)\right]&&\left[(L^2(S^1),-i\partial_x)\right]&\mapsto&\sf
\end{array}\]


odd index pairing or the spectral flow pairing

Callias-type theorems and Toeplitz-type theorems



\subsection{Equivariant KK-class associated to a circle action}

Let $A$ be a C$^*$-algebra and $\sigma:\T\to\Aut(A)$ be an action with spectral subspace assumption.
Then, the faithful conditional expectation $\e:A\to A^\sigma$ defines a $\T$-equivariant full right Hilbert bimodule $E:=\bar A$ over $(A,A^\sigma)$.


(By assuming periodicitiy of $\sigma$, we can obtain spectral projections by continuous functional calculus.)

We can obtain a class of $KK^\T_1(A,A^\sigma)$.

\subsection{}

Let $A$ be a C$^*$-algebra and $\sigma:\T\to\Aut(A)$ be an action with spectral subspace assumption.
Let $\f$ be an invaraint faithful densely defined lower semi-continuous weight on $A$.
Let $M:=\pi_\f(A)''$ and $N:=$


\subsection{Modular index pairing}
Let $N$ be a semi-finite von Neumann algebra with a faithful semi-finite normal trace $\tau$.
Let $D_1$ and $D_2$ be closed self-adjoint operators affiliated with $N$, with $P_1:=1_{[0,\infty)}(D_1)$ and $P_2:=1_{[0,\infty)}(D_2)$ the positive spectral projections.
If $D_1-D_2\in B(H)$ and $P_1P_2$ is Breuer-Fredholm, then the spectral flow is defined by $\sf(D_1,D_2):=\index_\tau(P_1P_2)$.
In the case when $P_1$ and $P_2$ are finite, then $\sf(D_1,D_2)=\tau(P_2)-\tau(P_1)$.





\section{Local index formulae}
















\chapter{Topological models}

\section{K-homology}

\begin{prb}[Fredholm modules]
It is intoduced in [Ati70].
analytic K-homology
\end{prb}


\begin{prb}[Baum-Douglas K-homology]
Let $X$ be a paracompact Hausdorff space, and $A$ be a closed subspace of $X$.
A \emph{Baum-Douglas geometric cycle} of the pair $(X,A)$ is a triple $(Z,E,\f)$ of 
\begin{enumerate}[(i)]
\item a smooth compact spin$^c$ manifold $Z$ possibly with boundary,
\item a smooth Hermitian bundle $E\to Z$,
\item a continuous map $\f:Z\to X$ such that $\f(\partial Z)\subset A$.
\end{enumerate}
\end{prb}


\begin{prb}[Connective K-homology]
It is introduced in [Seg77]
\end{prb}





For a complex vector bundle $Y\to X$, $\beta:K(X)\to K(Y)$ is an isomorphism, and the inverse $\alpha:K(Y)\to K(X)$ can be constructed the family of elliptic operators parametrized by $X$ blabla.
We have $\beta\in KK(C(X),C(Y))$ and $\alpha\in KK(C(Y),C(X))$.

\section{Correspondences}

The topological analogues of algebraic correspondences are first studied by Baum, Connes, and Skandalis in 1980s.
Emerson-Meyer enhanced the notions in [EM10]
Let $X$ and $Y$ be paracompact Hausdorff spaces.
A \emph{correspondence} from $X$ to $Y$ is a compactly supported K-theory datum of a locally compact Hausdorff space $Z$, together with continuous maps $b:Z\to X$ and $f:Z\to Y$.
Connes-Skandalis assume that $f$ is K-oriented.



Then, $b_*(j(E)\otimes_Zf_!)\in KK(X,Y)$, where $j(E)\in K(Z,Z)$.








\chapter{Baum-Connes conjecture}









\chapter{Coarse geometry}


\section{}

A \emph{coarse structure} on a set $X$ is an ideal $\cE$ on $X\times X$, in the sense that $\cE$ is a downward closed and upward directed set on $X\times X$, such that
\begin{enumerate}[(i)]
\item for every $E\in\cE$, we have $\Delta\subset E$,\hfill(reflexivity)
\item for every $E,E'\in\cE$, there is $E''\in\cE$ such that $E\circ E'\subset E''$,\hfill(triangle inequality)
\item for every $E\in\cE$, we have $E^{-1}\in\cE$.\hfill(symmetry)
\end{enumerate}

Each element $\cP(X\times Y)$ is one-to-one corresponded to union-preserving functions $\cP(X)\to\cP(Y)$.

In a metric space $X$, for each $r>0$, we have a fattening $E_r:\cP(X)\to\cP(X)$ that preserves union, and $E_r\cup E_{r'}=E_{\max\{r,r'\}}$.

In a locally compact group $G$, for each relatively compact $B\subset G$, we have $E_B^L:S\mapsto BS$, $E_B^R:S\mapsto SB^{-1}$, $E_B^{LR}:S\mapsto BSB^{-1}$, and they define three coarse structures on $G$.

A canonical bornology $\cB:=\{E(S):E\in\cE,\ |S|<\infty\}$ can be defined.

Two subsets $S_1,S_2$ of a coarse space $X$ are called \emph{asymptotically disjoint} if for every entourage $E_1,E_2$ the intersection $E_1(S_1)\cap E_2(S_2)$ is bounded.

\section{Quantum Hall effect}



A wave function is a section $\psi$ of a $\rU(1)$-line bundle $L\to M$, more generally, a section $\psi$ of a Hermitian $G$-vector bundle $V\to M$.
Galilean invariant momentum operator?
We fix a connection $\nabla$ on $L$, which can be forgotten if $M$ is contractible and no magnetic fields are considered.
A $\rU(N)$-gauge fixing is just a choice of a local field of orthonormal frames on the intersection of two charts of $M$, making $V$ locally trivially $\C^N$ with the standard basis.
After gauge fixing, the covariant derivative is represented locally as $\nabla_j\psi=(\partial_j-iA_j)\psi$, where $A=\sum_jA_jdx^j$ is a $\R=-i\fu(1)$-valued one-form.

If $(M,\nabla)$ has non-trivial holonomy (the Aharonov-Bohm flux), then it comes into play.


On $M=\R^2$ with a connection $\nabla$ such that the curvature form is $\cF^\nabla=b\cdot\vol_M$ for some $b\in\R\setminus\{0\}$, then the \emph{Landau operator}, which is a free Hamiltonian defined by the connection Laplacian or the Bochner Laplacian $H:=\nabla^*\nabla=-\tr(\nabla^2)\ge0$, its spectrum is known to be $\sigma(H)=(2\N+1)|b|$.
This discreteness of the spectrum is called the \emph{Landau quantization}.


We consider a spin manifold $M$ with the spinor bundle $S$, and the sections of $S\otimes V\to M$ represent the space of wave functions.
curvature of a spin connection
the Laplacians


\section{}

Loop spaces,
Loop groups




\chapter{Quantum metric spaces}


\section{}
Let $X$ be a metrizable compact space, or equivalently a commutative separable C$^*$-algebra.
A metric $d$ which topologize $X$ gives rise to a semi-norm $L$ on a dense $*$-subalgebra of $C(X)$ defined such that
\[L(f):=\sup_{x\ne y}\frac{|f(x)-f(y)|}{d(x,y)},\]
called the Lipschitz semi-norm.
The Lipschitz semi-norm not only recovers the original metric $d$ by
\[d(x,y)=\sup_{L(f)\le1}|f(x)-f(y)|,\]
but also defines a metric $d$ on the state space $\Prob(X)$ such that
\[d(\mu,\nu):=\sup_{L(f)\le1}|(\mu-\nu)(f)|,\]
called the Monge-Kantorovich metric.





\chapter{Topics}



\section{Non-commutative tori}



\section{Foliations}
The paper by Connes-Skandalis








\part{Physical applications}

\chapter{D-branes}

\chapter{Quantum Hall effects}

\chapter{Standard models}

We want to consider connections on an unbounded Kasparov module for $KK((\C\oplus\H)\otimes C(M),C(M))$.



YM:
\[A\subset\End_B(E_1)\otimes C(M),\qquad B:=C(M),\qquad C=\C\]
The $D_2$-connection on $E_1$ gives an unbounded Kasparov module over $A$ and $B$, and the unbounded Kasparov product $(H_{12},D_{12})$ over $A$ can be called the YM spectral triple.

Glashow-Weinberg-Salam:
\[A:=(\C\oplus M_2(\C))\otimes C(M),\qquad B:=C(M),\qquad C=\C\]
with
\[(E_1,D_1,\nabla^1):=((\C\oplus\C^2)\otimes C(M),T,d),\qquad(E_2,D_2,\nabla^2):=(L^2(S),D_2,0).\]
Then, the internal gauge group is $U(A)\approx C(M,\rU(1)\times\rU(2))$, and the adjointability of $A$ implies that the gauge group preserves the fibers of the bundle $E_1$.
The group $\rU(1)\times\rU(2)$ acts on the fiber space $\C\oplus\C^2$, and the group $C(M,\rU(1)\times\rU(2))$ acts on the total space $(\C\oplus\C^2)\times M$ of the trivial associated vector bundle.
Scalar fields are elements of one-form space $\Omega_{D_1}^1(A)$, can be computed to have two components, are Higgs bosons.
Gauge fields are elements of the $D_2$-connection space $\Hom_B(E_1,E_1\otimes\Omega_{D_2}^1(B))$ on the trivial bundle $E_1$, have the form $\nabla^1+\omega$, where $\omega\in\Omega_{D_2}^1(B)\otimes(\fu(1)\oplus\fu(2))$ represents the pair of the $B$ and $W$ bosons.

$\Omega_{D_{12}}^1(A)$ is a subspace of $\Omega_{D_1}^1(A)\oplus\Hom_B(E_1,E_1\otimes\Omega_{D_2}^1(B))$.

(By replacing $M_2(\C)$ to the quaternions, we obtain $\SU(2)$ instead of $\rU(2)$.)

A matter field, equivalently a fermion, is a vector in $E_1\otimes_BE_2=(\C\oplus\C^2)\otimes L^2(S)$.
The interaction action functional is defined by $\<D_{12}\psi,\psi\>$ on $E_1\otimes_BE_2$?



A charge is a generator of a gauge group.
For $\rU(1)\times\SU(2)$, the generator $Y$ for $\rU(1)$ is called the weak hypercharge, and the generators $T$ for $\SU(2)$ are called the weak isospins.

$T_i$ give iso

$Y=-1$ left-handed lepton doublet $L_l=(\nu_l,e_l)$

$Y=-2$ right-handed electron singlet $e_r$

$Y=1/3$ left-handed quark doublet $Q_l=(u_l,d_l)$

$Y=4/3$ right-handed up quark singlet $u_r$

$Y=-2/3$ right-handed down quark singlet $d_r$

$Y=1$ scalar doublet $\phi$

singlet -> four dimensional
(left/right)-handed singlet -> two dimensional

If we let $S$ be of dimension $4$, then $\rU(1)$ and $Y$ acts on the fiber space of dimension $(15+1)\times2$.
We need $8$(?)-dimensional representation of $\rU(1)\times\SU(2)$ instead of $3$-dimensional representation $\C\oplus\C^2$.
fundamental representation?



For the Seiberg-Witten, the gauge group is $\rU(1)$.
The Dirac operator on $L^2(S)$ is determined by a connection.
The moduli space of paris of a vector and a connection satisfying the monopole equation... and its dimension can be computed by the Atiyah-Singer????



\chapter{Anomalies}




\end{document}







