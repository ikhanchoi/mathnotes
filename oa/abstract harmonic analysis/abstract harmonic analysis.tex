\documentclass{../../large}
\usepackage{../../ikhanchoi}


\begin{document}
\title{Abstract Harmonic Analysis}
\author{Ikhan Choi}
\maketitle
\tableofcontents



\part{Locally compact groups}




\chapter{Locally compact groups}


\section{Haar measures}


\begin{prb}[Existence of the Haar measure]
\end{prb}

\begin{prb}[Left and right uniformities]
\end{prb}

\begin{prb}[Modular functions]
\end{prb}

\begin{prb}[Uniformly continuous functions]
$G$ acts on $C_u(G)$ and $L^1(G)$ continuously with respect to the point-norm topology.
A function on $G$ is left uniformly continuous if and only if it is written as $f*x$ for some $f\in L^1(G)$ and $x\in L^\infty(G)$.
$g\in C_c(G)$ is two-sided uniformly continuous.
\end{prb}




\section{Convolution algebras}

We use the notation $L^p(G)$ for the non-commutative $L^p$-spaces constructed with the left Haar measure on $G$, which is a faithful semi-finite normal weight of $L^\infty(G)$.
The predual of $L^\infty(G)$ can be identified with $L^1(G)$.
The regular representation on $L^2(G)$ is the Gelfand-Naimark-Segal representation associated with the left Haar measure.

\begin{prb}[Convolution algebras of integrable functions]
Let $G$ be a locally compact group.
Then, $L^1(G)$ is a hermitian Banach $*$-algebra such that
\[(f*g)(x):=(f\otimes g)\Delta(x),\qquad f,g\in L^1(G),\ x\in L^\infty(G).\]
\begin{parts}
\item $L^1(G)$ has a two-sided approximate unit in $C_c(G)$.
\item $\alpha:G\to\Aut(L^1(G))$ is point-norm continuous.
\item $\lambda:G\to U(L^2(G))$ and $\lambda:L^1(G)\to B(L^2(G))$ are strongly continuous.
\item Convolution inequalities.
\item Representation theory equivalence.
\end{parts}
\begin{pf}
Let $U_i$ be a net of open neighborhoods of the identity $e$ of $G$.
By the Urysohn lemma, there is $e_i\in C_c(U_i)^+$ such that $\|e_i\|_1=1$ for each $i$.
We claim that $e_i$ is a two-sided approximate unit for $L^1(G)$.
Suppose $g\in C_c(G)$, which is two-sided uniformly continuous.
For any $\e>0$, choose $i_0$ such that $\|g-\lambda_sg\|<\e$ and $\|g-\rho_sg\|<\e$ for all $s\in U_i$ for $i\succ i_0$.
Then, we have
\begin{align*}
\|e_i*g-g\|_1
&=\int|e_i*g(t)-g(t)|\,dt
\le\iint e_i(s)|g(s^{-1}t)-g(t)|\,ds\,dt\\
&=\int_{U_i}e_i(s)\|\lambda_sg-g\|_1\,ds
<\e\int e_i(s)\,ds\le\e,
\end{align*}
and
\begin{align*}
\|g*e_i-g\|_1
&=\int|g*e_i(s)-g(s)|\,ds
\le\iint|g(t)-g(s)|e_i(t^{-1}s)\,dt\,ds\\
&=\iint|g(t)-g(ts)|e_i(s)\,dt\,ds
=\int\|g-\rho_sg\|_1e_i(s)\,ds
<\e\int e_i(s)\,ds\le\e,
\end{align*}
and they imply $\lim_i\|e_i*g-g\|_1=\lim_i\|g*e_i-g\|_1=0$.
We can approximate $f\in L^1(G)$ with compactly supported continuous functions by the $\e/3$ argument.
\end{pf}

\end{prb}


\begin{prb}[Measure algebras]
\end{prb}

\begin{prb}[Group C$^*$-algebras]
\end{prb}

\begin{prb}[Group von Neumann algebras]
Let $G$ be a locally compact group.
Since $G$ is a locally compact Hausdorff space and the left Haar measure is a faithful semi-finite lower semi-continuous weight on the commutative C$^*$-algebra $C_0(G)$, we have a corresponding semi-cyclic representation $m:C_0(G)\to B(L^2(G))$ which is normally extended to a von Neumann algebra $L^\infty(G)$ with $m(L^\infty(G))=m(C_0(G))''$, and $L^1(G)$ is identified with the predual $L^\infty(G)_*$.

By the left Haar measure, $C_c(G)$ has a natural non-commutative left Hilbert algebra structure
\[(f*g)(s):=\int f(t)g(t^{-1}s)\,dt,\qquad\<f,g\>:=\int\bar{g(s)}f(s)\,ds,\qquad f^\sharp(s):=\nabla(s^{-1})\bar{f(s^{-1})},\]
where $\nabla$ is the modular function for $G$, and it induces the regular representation $\lambda:C_c(G)\to B(L^2(G))$.
By the group structure of $G$, the Hilbert algbera $C_c(G)$ is also a commutative counital multiplier Hopf $*$-algebra 
\[(fg)(s):=f(s)g(s),\qquad\Delta f(s,t)=f(st),\qquad f^*(s):=\bar{f(s)},\qquad\kappa f(s)=f(s^{-1}).\]
We start from this structures.


They satisfy a compatibility condition $\<fg,h\>=\<f,g^*h\>$.

With the integral notation $\lambda(f)=\int\lambda_sf(s)\,ds$, we can write

From now on, we are going to exclude any measure theory and the theory of non-commutative $L^p$ spaces.
First, we have the completion $H=:L^2(G)$.
Consider two representations
\[\lambda:(C_c(G),*,^\sharp)\to B(L^2(G)),\qquad m:(C_c(G),\cdot,^*)\to B(L^2(G)).\]
\begin{parts}
\item $\lambda$ is well-defined.
\item $m$ is well-defined.
\end{parts}
\end{prb}
\begin{pf}
The multiplication representation $m$ is well-defined because for $f\in C_c(G)$ we have $f^*f\in C_c(G)\subset L^2(G)$ so
\[\|m(f)g\|^2=\<fg,fg\>=\<f^*fg,g\>,\qquad g\in C_c(G).\]


blabla

Note that we have
\begin{align*}
|\<\lambda(\xi)\eta,\zeta\>|^2
&=|\iint\xi(t)\eta(t^{-1}s)\bar{\zeta(s)}\,ds\,dt|^2\\
&\le\iint|\xi(t)||\eta(t^{-1}s)|^2\,ds\,dt\cdot\iint|\xi(t)||\zeta(s)|^2\,ds\,dt\\
&=\|\xi\|_1^2\|\eta\|_2^2\|\zeta\|_2^2
\end{align*}
and
\begin{align*}
|\<\rho(\xi)\eta,\zeta\>|^2
&=|\iint\eta(t)\xi(t^{-1}s)\bar{\zeta(s)}\,ds\,dt|^2\\
&\le\iint|\xi(t^{-1}s)||\eta(t)|^2\,ds\,dt\cdot\iint|\xi(t^{-1}s)||\zeta(s)|^2\,ds\,dt\\
&=\|\xi\|_1\|F\xi\|_1\|\eta\|_2^2\|\zeta\|_2^2
\end{align*}
imply
\[\|\lambda(\xi)\|_{2\to2}\le\|\xi\|_1,\qquad\|\rho(\xi)\|_{2\to2}\le\sqrt{\|\xi\|_1\|F\xi\|_1}.\]
The equalities do not hold, consider $\|\lambda(\xi)\|=\|\hat\xi\|_\infty$ if $G=\R$.



\end{pf}








\begin{prb}[Absorption principle]
Let $G$ be a locally compact group.

\[w:\]

The \emph{structure operator} of $G$
 is an oeprator $w\in U(L^2(G\times G))$ defined such that $w\xi(s,t):=\xi(s,st)$, or $w\in L^\infty(G)\bar\otimes W_r^*(G)$ such that $\Ad w(\lambda_s\otimes\lambda_s):=\lambda_s\otimes1$.
If $w(x\otimes x)w^*=x\otimes1$, then $x=\lambda_s$ for some $s\in G$.
\begin{parts}
\item $\lambda\otimes u$ and $\lambda\otimes1$ are unitarily equivalent. It is called the \emph{Fell absorption principle}.
\end{parts}
\end{prb}
\begin{pf}

The Fell absorption principle states that the composition of equivariant operators
\[\begin{tikzcd}[row sep=tiny]
L^2(G)\otimes H \rar{\Delta\otimes1} & L^2(G)\otimes L^2(G)\otimes H \rar{1\otimes?} & L^2(G)\otimes H\\
\lambda\otimes 1 \rar[mapsto] & \lambda\otimes\lambda\otimes1 \rar[mapsto] & \lambda\otimes u
\end{tikzcd}\]
is unitary.

The structure operator is a special case of the Fell absorption operator
\[\begin{tikzcd}[row sep=tiny]
L^2(G)\otimes L^2(G) \rar{\Delta\otimes1} & L^2(G)\otimes L^2(G)\otimes L^2(G) \rar{1\otimes?} & L^2(G)\otimes L^2(G)\\
\lambda\otimes1 \rar[mapsto] & \lambda\otimes\lambda\otimes1 \rar[mapsto] & \lambda\otimes\lambda
\end{tikzcd}\]
\end{pf}

\section{Fourier and Fourier-Stieltjes algebras}


\begin{prb}[Fourier algebras]
Let $G$ be a locally compact group.
We define the \emph{Fourier algebra} by $A(G):=W_r^*(G)_*$.
\begin{parts}
\item $A(G)$ is the set of matrix coefficients of the regular representation $\lambda:G\to U(L^2(G))$, that is, the functions $s\mapsto\<\lambda(s)\xi,\eta\>$ for $\xi,\eta\in L^2(G)$.
\item $A(G)$ is a dense Banach subalgebra of $C_0(G)$. In particular, $M(G)\to W_r^*(G)$ is a dense embedding.
\end{parts}
\end{prb}
\begin{pf}

\end{pf}


\begin{prb}[Fourier-Stieltjes algebras]
Let $G$ be a locally compact group.
We define the \emph{Fourier Stieltjes algebra} by $B(G):=C^*(G)^*$.
\begin{parts}
\item $B(G)$ is the linear span of continuous positive definite functions.
\item On $B(G)_1$, the compact open topology is stronger than the weak$^*$ topology.
\item On $B(G)_1$, the strict topology with respect to $A(G)$ is equivalent to the weak$^*$ topology.
\end{parts}
\end{prb}
\begin{pf}

\end{pf}


dense embeddings among non-commutative algberas and commutative algebras:
\[\begin{tikzcd}
L^1(G) \rar\dar & C^*(G) \dar\\
M(G) \rar & W_r^*(G).
\end{tikzcd}
\qquad
\begin{tikzcd}
A(G) \rar\dar & C_0(G) \dar\\
B(G) \rar & L^\infty(G).
\end{tikzcd}\]








\section{Pontryagin duality}

\begin{prb}[Locally compact abelian groups]
Let $G$ be a locally compact abelian group.
\begin{parts}
\item Every irreducible representation of $G$ is one-dimensional, and $\hat G$ is an abelian group.
\item The compact open topology of $C(G)$ and the weak$^*$ topology of $L^\infty(G)$ coincide on $\hat G$, and $\hat G$ is locally compact Hausdorff with this topology.
\end{parts}
\end{prb}

\begin{prb}[Fourier transforms]
Let $G$ be a locally compact abelian group.
We introduce the notation $\<s,p\>:=p^{-1}(s)\in\T$ for $p\in\hat G$ and $s\in G$.
The \emph{Fourier transform} and the \emph{Fourier-Stieltjes transform} of an integrable function $f\in L^1(G)$ and a finite Radon measure $\mu\in M(G)$ are defined by
\[\cF f(p):=\int_G\<s,p\>f(s)\,ds,\qquad \cF\mu(p):=\int_G\<s,p\>\,d\mu(s)\qquad p\in\hat G.\]
\begin{parts}
\item The Fourier transform is restricted to a linear operator $B(G)\cap L^1(G)\to B(\hat G)\cap L^1(\hat G)$.
\item The Fourier transform is uniquely extended to a continuous dense $*$-homomorphism $L^1(G)\to C_0(\hat G)$.
\item The Fourier transform is uniquely extended to a continuous dense $*$-homomorphism $L^1(G)\to B(\hat G)$.
\item The Fourier transform uniquely defines a unitary operator $L^2(G)\to L^2(G)$.
\item The Fourier-Stietjes transform $M(G)\to L^\infty(G)$ is injective.
\end{parts}
\end{prb}
\begin{pf}
(a)
Let $f\in B(G)\cap L^1(G)$.


(b)

(c)

(d)
We prove the identity $\|f\|_{L^2(G)}=\|\cF f\|_{L^2(\hat G)}$ for $f\in B(G)\cap L^1(G)$ and the density of $B(G)\cap L^1(G)$ in $L^2(G)$.

(e)
Consider a commutative diagram of Banach $*$-algebras
\[\begin{tikzcd}
L^1(G) \rar{(1)}\dar & C^*(G) \rar{(3)} & C_0(\hat G) \dar\\
M(G) \rar{(2)} & W_r^*(G) \rar{(4)} & L^\infty(\hat G)
\end{tikzcd}\]
The dense injection (1) is by definition of the group C$^*$-algebra.
The dense injection (2) is by the dense inclusion $A(G)\to C_0(G)$.
The isomorphism (3) is due to the equivalence between representation theories of $G$ and $C^*(G)$ and the Gelfand duality.
The isomorphism (4) is constructed by taking double commutant of $L^1(G)$ in the Plancherel isomorphism $B(L^2(G))\to B(L^2(\hat G))$.
Since the first and third rows are respectively the Fourier transform and Fourier-Stieltjes transform, we are done.
\end{pf}

the decomposition of the regular representation and the Plancherel theorem....


\begin{prb}[Pontryagin duality]
Let $G$ be a locally compact abelian group.

\begin{parts}
\item The canonical homomorphism $\Phi:G\to\hhat G$ defined such that $\Phi(s)(p)=\<s,p\>$ for $s\in G$ and $p\in\hat G$ is a topological isomorphism.
\end{parts}
\end{prb}
\begin{pf}
It suffices to prove that the natural $*$-homomorphisms $C_0(\hhat G)\to C_0(G)$ and $M(G)\to M(\hhat G)$ have dense images.
Since the Fourier transform $L^1(G)\to B(\hat G)$ is dense, and it factors through $M(G)\to M(\hhat G)$ with an embedding $M(\hhat G)\to B(\hat G)$, so $M(G)\to M(\hhat G)$ is dense.
Since the injectivity of the Fourier-Stieltjes transform $M(G)\to L^\infty(\hat G)$ implies that its dual $L^1(\hat G)\to C_0(G)$ is dense, and it factors through $C_0(\hhat G)\to C_0(G)$ by the Fourier transform, so $C_0(\hhat G)\to C_0(G)$ is dense.
Therefore, $M(G)\to M(\hhat G)$ is a $*$-isomorphism.
\end{pf}






\chapter{Amenability}




\chapter{}




\part{Representation categories}


\chapter{Representations of compact groups}
\section{Peter-Weyl theorem}

Let $G$ be a compact group.
Every representation will assume the strong continuity and the unitarity.

Let $\pi_1$ and $\pi_2$ be representations, and suppose $\pi_1$ is irreducible.
If there is a non-zero intertwiner $v\in B(H_1,H_2)$, normalized to have norm one, then $v^*v\in\pi_1(G)'=\C1$ implies that $v$ is an isometry, so $\pi_1$ is isomorphic to a subrepresentation of $\pi_2$.
If $\pi_2$ is irreducible, then the existence of non-zero intertwiner is equivalent to that $\pi_1$ and $\pi_2$ are isomorphic.

Let $\pi_1$ and $\pi_2$ be representations.
Then, any bounded linear operator $w:H_1\to H_2$ induces an intertwiner $v:=\int_G\pi_2(s)w\pi_1(s)^*\,ds:H_1\to H_2$.
For $\xi_1,\eta_1\in H_1$ and $\xi_2,\eta_2\in H_2$, if we let $w:=\theta_{\xi_1,\xi_2}=\<\cdot,\xi_1\>\xi_2$, then
\begin{align*}
\<v\eta_1,\eta_2\>
&=\int_G\<\pi_2(s)w\pi_1(s)^*\eta_1,\eta_2\>\,ds\\
&=\int_G\<\pi_2(s)\<\pi_1(s)^*\eta_1,\xi_1\>\xi_2,\eta_2\>\,ds\\
&=\int_G\bar{\<\pi_1(s)\xi_1,\eta_1\>}\<\pi_2(s)\xi_2,\eta_2\>\,ds.
\end{align*}
This implies that matrix coefficients come from non-isomorphic irreducible representations are orthogonal.

For a representation $\pi$ of $G$, denote by $A(\pi)$ the linear span of matrix coefficients for $\pi$.
We prove $\cO(G):=\bigcup_\pi A(\pi)$ is dense in $C(G)$, where $\pi$ runs through all the finite-dimensional irreducible representations of $G$.
Here the irreducibility is redundant because every finite-dimensional representation is decomposed into the direct sum of finite-dimensional irreducible representations.

Note that for the left regular representation $\lambda:G\to U(L^2(G))$ we have $\lambda:L^1(G)\to K(L^2(G))$ and its restriction $\lambda:L^2(G)\to L^2(L^2(G))$ because $G$ is compact.
Fix $f\in C(G)$ and let $V$ be an eigenspace of the Hilbert-Schmidt operator $\lambda_f\in L^2(L^2(G))$, which is a finite-dimensional subrepresentation of $\lambda$ and satisfies $V\subset C(G)$.
Let $\{e_i\}$ be an orthonormal basis of $V$.
If $\xi\in V$, then since the contragradient representation $\lambda^*$ can be defined on $V$ and it is finite-dimensional, we have $\xi\in\cO(G)$ by
\[\xi(s)=(\lambda_s^*\xi)(e)=(\sum_i\<\lambda_s^*\xi,e_i\>e_i)(e)=\sum_ie_i(e)\<\lambda_s^*\xi,e_i\>,\]
so $V\in\cO(G)$.

For $f\in C(G)$ and $\xi\in L^2(G)$, we can see $\lambda_f\xi$ is uniformly approximated by $\cO(G)$ by the spectral truncation of $\lambda_f$.
Since $C(G)*L^2(G)$ is dense in $C(G)$, the density of $\cO(G)$ in $C(G)$ follows.

\section{Tannaka-Krein duality}
\section{Mackey machine}
Example of non-compact Lie groups,
Wigner classification

















\part{Topological quantum groups}


\chapter{Compact quantum groups}

\section{Algebraic compact quantum groups}

Multiplier Hopf $*$-algebras

Algebraic quantum groups

idempotent ring assumption



For a monoid, we can associate a bialgebra called the convolution algebra.
If the monoid is a group, then the convolution algebra becomes a Hopf algebra.

universal enveloping algebra.
$q$-deformations of the coordinate Hopf algebras $\cO(G)$ of a semi-simple complex Lie group, and the universal enveloping algebra $U(\fg)$ of a semi-simple complex Lie algebra.

If $A$ is a coalgebra and $B$ is an algebra, then $\Hom_\C(A,B)$ becomes an algebra with convolution.
If $A$ is a coalgebra, then $A^*$ is an algebra.
If $A$ is a bialgebra, then $A$ is a bimodule over $A^*$.

Duality for finite-dimensional Hopf ($*$-)algebras.
dual pairing






\begin{prb}[Algebraic compact quantum groups]
Recall that a Hopf algebra $A$ has five linear structure maps the multiplication $\mu$, unit $\eta$, comultiplication $\delta$, counit $\e$, and antipode $\kappa$.
A \emph{Hopf $*$-algebra} is a Hopf algebra $A$ together with an conjugate-linear involution $*:A\to A$ such that there are commutative diagrams
\[\begin{tikzcd}
A\otimes A \rar{\mu}\dar[swap]{(*\otimes*)\sigma_A} & A \dar{*}\\
A\otimes A \rar{\mu} & A
\end{tikzcd}
\qquad
\begin{tikzcd}
A \rar{\delta}\dar[swap]{*} & A\otimes A \dar{(*\otimes*)\sigma_A}\\
A \rar{\delta} & A\otimes A
\end{tikzcd}\]
where $\sigma_A:A\otimes A\to A\otimes A$ is the swap map.
An \emph{algebraic compact quantum group} is defined as a complex Hopf $*$-algebra $A$ together with a unital positive linear functional $h:A\to\C$ satisfying $(h\otimes\id)\delta=\eta h=(\id\otimes h)\delta$.
It is conventional to use $\G$ to denote a compact quantum group, and we will usually write the underlying Hopf $*$-algebra $A$ as $\cO(\G)$.
\begin{parts}
\item There is a categorical equivalence between commutative compact quantum groups and compact groups.
\end{parts}
\end{prb}


\section{Woronowicz compact quantum groups}

\begin{prb}[Woronowicz compact quantum groups]
From now on, the tensor product of C$^*$-algebras will always be assumed to be the minimal one, if not particularly mentioned.
In the sense of Woronowicz, a \emph{compact quantum group} is defined as a unital C$^*$-algebra $A$ together with a coassociative unital $*$-homomorphism $\delta:A\to A\otimes A$ and a state $h:A\to\C$ such that $(1\otimes h)\delta=\eta h=(h\otimes1)\delta$, where $\eta:\C\to A$ is the unit map.
The state $h$ is called the \emph{Haar state}.
When we write $\G$ to mean a compact quantum group, then the underlying C$^*$-algebra $A$ is denoted by $C(\G)$.
\begin{parts}
\item For a C$^*$-algebra $A$ with a coassociative unital $*$-homomorphism $\delta:A\to A\otimes A$, the existence of the Haar state is equivalent to the cancellation property in the sense that the linear spans of the sets $\delta(A)(A\otimes1)$ and $\delta(A)(1\otimes A)$ are respectively dense in $A\otimes A$.
\end{parts}
\end{prb}

\[C_0(G),\quad L^\infty(G),\qquad C^*(G),\quad C_r^*(F),\quad W_r^*(G)\]
\[A(G), B(G)\]

For a compact group $G$, $C(G)$ has a coalgebra structure induced from $C(G)\subset L^1(G)$.

\begin{prb}[Peter-Weyl theorem]
The $*$-subalgebra of matrix coefficients is a Hopf $*$-algebra.

\end{prb}


\begin{prb}
A \emph{compact algebraic quantum group} is a Hopf $*$-algebra with a positive integral.
For a compact quantum group $\G$, the subspace $\C(\G)$ spanned by the matrix coefficients of corepresentations is an algebraic quantum group.
\end{prb}



\begin{prb}
Let $\G$ be a compact quantum group.
A \emph{representation} of $\G$ is a corepresentation of $C(\G)$.
\end{prb}

\section{Kac algebras}

\begin{prb}[Kac algebras]
If the Haar state is a trace, then we say the compact quantum group is a \emph{Kac algebra} or is of \emph{Kac type}.
\end{prb}



\chapter{Locally compact quantum groups}

\section{Locally compact quantum groups}


Probably, a Hopf-von Neumann algebra in Enock-Schwartz is just a von Neumann bialgebra in Timmerman, a coinvolutive Hopf-von Neumann algebra in Enock-Schwartz is just a Hopf-von Neumann algebra in Timmerman.
Since a locally compact quantum group has counit and antipode as unbounded operators, I do not know if I can say there is a Hopf algebra structure.


\begin{prb}[Locally compact quantum groups]
In the sense of Kustermans-Vaes, a \emph{locally compact quantum group} is defined as a von Neumann algebra $M$ together with a coassociative unital normal $*$-homomorphism $\delta:M\to M\bar\otimes M$ and faithful semi-finite normal weights $\f$ and $\psi$ such that $(1\otimes\f)\delta=\eta\f$ on $\fM_\f$ and $(\psi\otimes1)\delta=\eta\psi$ on $\fM_\psi$, where $\eta:\C\to M$ is the unit map.
The weight $\f$ and $\psi$ are called the \emph{left} and $\emph{right Haar weights}$ respectively.
When we write $\G$ for a locally compact quantum group, the underlying von Neumann algebra is denoted by $L^\infty(\G)$.

Recall that $\fM_\f$, $\fA_\f$, $\fN_\f$, $H_\f=:L^2(\G)$, $\Lambda_\f$, $\Delta_\f$, $J_\f$.

$\fN_\f^*\fN_\psi$
\end{prb}




\begin{prb}[Fundamental multiplicative unitaries]
A \emph{multiplicative unitary} on a Hilbert space $H$ is a unitary operator $W\in B(H\otimes H)$ satisfying the pentagonal identity $W_{12}W_{13}W_{23}=W_{23}W_{12}$ in $B(H\otimes H\otimes H)$, written in the leg numbering notation.
It defines a comultiplication $\delta:H\to H\otimes H$ such that $\delta(\xi):=W(\xi\otimes1)W^*$ for $\xi\in H$.

Let $\G$ be a locally compact quantum group.
Then, there is a unique multiplicative unitary $W$ on $L^2(\G)$, called the \emph{fundamental multiplicative unitary}, such that
\[W^*(\Lambda_\f(x)\otimes\Lambda_\f(y))=(\Lambda_\f\otimes\Lambda_\f)(\delta(x)(y\otimes1)),\qquad x,y\in\fN_\f.\]

\[\begin{tikzcd}
\fN_\f\otimes\fN_\f \rar{\Lambda_\f\otimes\Lambda_\f}\dar & L^2(\G)\otimes L^2(\G) \dar{W^*}\\
\fN_\f\otimes\fN_\f \rar{\Lambda_\f\otimes\Lambda_\f} & L^2(\G)\otimes L^2(\G)
\end{tikzcd}\]
\end{prb}


\begin{prb}[Fundamental involutions]
Let $\G$ be a locally compact quantum group.
Then, there is a closed densely defined conjugate-linear involution $G:\dom G\subset L^2(\G)\to L^2(\G)$ such that
\[G\Lambda_\f((\psi\otimes\id)(\delta(x^*)(y\otimes1)))=\Lambda_\f((\psi\otimes\id)(\delta(y^*)(x\otimes1))),\qquad x,y\in\fN_\f^*\fN_\psi.\]
\end{prb}


\begin{prb}[Antipode]
$\tau_t:=\Ad|G|^{-2it}$, $(\sigma_t^\psi\otimes\tau_{-t})\delta=\delta\sigma_t^\psi$, $\delta\tau_t=(\tau_t\otimes\tau_t)\delta$, 

For the polar decomposition $G=I|G|$, the \emph{unitary antipode} is defined by $R:\dom R\subset L^\infty(\G)\to L^\infty(\G):x\mapsto Ix^*I$.
The \emph{antipode} or \emph{coinverse} is $S:=R\tau_{-\frac i2}$

\end{prb}


Kac type: trivial scaling group.


\section{Dual quantum groups}

\section{Crossed products}



\end{document}







