\documentclass{../../large}
\usepackage{../../ikhanchoi}




\begin{document}
\title{Topological Algebras}
\author{Ikhan Choi}
\maketitle
\tableofcontents



\part{Topological vector spaces}
\chapter{Locally convex spaces}

\section{Category of locally convex spaces}


complete locally convex space

bornology,
tensor products,

\begin{prb}[Bilinear forms on topological vector spaces]
We will distringuish embeddings and topological embeddings.

Topologies on the space of operators $L(E,F)$.
\end{prb}


\begin{prb}[Topological tensor products]
Let $E$ and $F$ be locally convex spaces.
The \emph{projective tensor product} of $E$ and $F$ is a locally convex space which is universal among the jointly continuous bilinear operators from $E\times F$ to a locally convex space.

We can also describe it with semi-norms.
We have
\[B_{\mathrm{jnt}}(E,F)\cong(E\hat\otimes_\pi F)^*.\]
\[(E\hat\otimes_\pi F)^*_\sigma\cong L_?(E_?,F^*_?)\]



Induced topology on $E\odot F$ from the space of separately continuous bilinear forms on $E_\sigma^*\times F_\sigma^*$ with the topology of uniform convergence on products of equicontinuous subsets of $E^*$ and $F^*$.

$\sigma$: on finite sets
$\tau$: on weakly compact sets
$\beta$: on weakly bounded sets
$\e$: on equi-continuous sets


A subset of $E^*_\sigma$ is equicontinuous iff it is contained in the polar of a neighborhood of $E$.
A subset is polar of finite sets iff 

The topology of uniform convergence on $\cG$ = The topology generated by polars of $\cG$.

$E_\e$ is the original topology


Note that we have
\[X\otimes Y\cong B_{\mathrm{jnt}}(X_\sigma^*,Y_\sigma^*)\subset B_{\mathrm{sep}}(X_\sigma^*,Y_\sigma^*).\]
The space $B_{\mathrm{sep}}(X_\sigma^*,Y_\sigma^*)$ of separately continuous bilinear forms, which has a natural topology of uniform convergence on the products of equicontinuous sets in $X_\sigma^*$ and $Y_\sigma^*$, and this topology is complete if and only if $X$ and $Y$ are complete.
The induced topology on $X\otimes Y$ is called the \emph{injective tensor product} topology.
We have $C^k(\Omega,E)\cong C^k(\Omega)\hat\otimes_\e E$ if $E$ is complete.

Note that the projective tensor product reflects the original topologies of locally convex spaces, while the injective tensor product only depends on the dual pair structure.

The dual of $X\hat\otimes_\pi Y\to X\hat\otimes_\e Y$ defines an injection $J(X,Y)\to B_{\mathrm{jnt}}(X,Y)$.
A bilinear form in $J(X,Y)$ is called to be \emph{integral}.
\end{prb}


\begin{prb}
$L(E)$ is a topological algebra
\end{prb}



\section{Vector-valued functions}

\begin{prb}[Vector-valued measurable functions]
Let $(X,\mu)$ and $(Y,\nu)$ be localizable measure spaces.
Let $(E,E^*)$ be a dual pair.

Define vector valued Lebesgue spaces as the completion?
Weakly measurable functions?
\begin{parts}
\item $L^1(X,E)$ and $L^1(X)\otimes E$: $E$ is ... and $\otimes$ is ...
\item $L^2(X,E)$ and $L^2(X)\otimes E$ if $E$ is a Hilbert space and $\otimes$ is the Hilbert space tensor product.
\item $L^\infty(X,E)$ and $L^\infty(X)\otimes E$ if $E$ is ... and $\otimes$ is ...
\item $\mu:L^1(X)\otimes E\to E\subset E^{**}$ is well-defined if $E$ is ... and $\mu$ is ...
\item What is the relation between the product measurability and the Bochner measurability.
\item $L^p(X,L^q(Y))=L^p(X)\otimes L^q(Y)$ if $\otimes$ is ...
\item $L^p(X,L^p(Y))=L^p(X\times Y)$?
\end{parts}
\end{prb}

\begin{itemize}
\item weakly integrable: $L^1(X)\otimes E\to(E^*)^\#$.
\item Dunford integrable: $L^1(X)\otimes E\to E^{**}$.
\item Pettis integrable: $L^1(X)\otimes E\to E$.
\item Bochner integrable: $L^1(X)\otimes_\pi E\to E$.
\item For a Pettis integrable function, if we check it is strongly measurable using the Pettis measurability theorem and bound it with $L^1$ norm, then it becomes Bochner integrable.
\item If $E$ is normed so that $V^*$ is Fr\'echet, then weakly integrability implies the Dunford integrability.
\end{itemize}
\begin{pf}

\end{pf}


\begin{prb}[Vector-valued continuous functions]
Let $X$ be a locally compact Hausdorff space and $(E,E^*)$ be a dual pair.
Suppose
\begin{enumerate}[(i)]
\item the closed convex hull of a compact subset is compact in $E_\sigma$,
\item $E$ is closed in the strong bidual $E^{**}_\beta$.
\end{enumerate}
An example is the case when $E$ is a Banach space.
The weak dual pair $(E,E^*)$ satisfies the assumption by the Krein-\v Smulian theorem and the completeness of $E$.
The weak$^*$ dual pair $(E^*,E)$ also satisfies the assumption by the fact that the closed convex hull of a bounded set is bounded, and the norm topology and $\beta(E^*,E_\beta)$ on $E^*$ coincide by the Goldstine theorem.
In particular, for $F\subset E^*$, the Banach space $E$ is closed in the strong bidual for the dual pair $(E,F)$ if and only if the closed unit ball $F_1=F\cap E^*_1$ is weakly$^*$ dense in the closed ball $E^*_1$.

We want to construct a canonical element of $L(C_b(X,E_\sigma),L_\sigma(M(X)_\sigma,E_\sigma))$.

\begin{parts}
\item $C(X,E_\sigma)$ and $C(X)\otimes E_\sigma$. ($X$ compact Hausdorff)
\item $C_b(X,E_\sigma)\to L(M(\beta X)_\sigma,E_\sigma)\to L(M(X)_\sigma,E_\sigma)$ if $(E,E^*)$ satisfies the two properties.
\item $C_b(X,E_\sigma)\to L(L^1(X)_\beta,E_\tau)$
\item the boundedly completeness?
\item $C_b(X,E_\tau)\to L(M(X)_\sigma,E_\tau)$?
\end{parts}
\end{prb}
\begin{pf}
(a)
Consider a common dense subset $C(X)\odot E$.
For $f\in C(X,E_\sigma)$ and for a fixed finite sequence $\xi^*_j$ in $E^*$ and $\e>0$, taking $U_{x_i}$ at each $x_i\in X$ such that $\max_j|\<f(x_i)-f(x),\xi_j^*\>|<\e$ for $x\in U_{x_i}$, then the partition of unity constructs a function $\sum_kf(x_k)\chi_k\in C(X)\odot E$ such that
\[\max_j\||\<f-\sum_kf(x_k)\chi_k,\xi^*_j\>\|=\sup_{x\in X}\sum_k\chi_k(x)\max_j|\<f(x)-f(x_k),\xi^*_j\>|<\sup_{x\in X}\sum_k\chi_k(x)\e=\e,\]
so the algebraic tensor is dense in $C(X,E_\sigma)$.

\[[f]_{\mu_j,\xi_j^*}=\int\<f(x),\xi_j^*\>\,d\mu_j(x).\]

\[ih\partial_t=H(h)\]
propagator $e^{-itH/h}$

(b)
First we have $C_b(X,E_\sigma)\to L(E^*_\beta,C_b(X)_\beta):f\mapsto(\xi^*\mapsto\<f(\cdot),\xi^*\>)$ because for a net $\xi^*_i\in E^*$ such that $\xi^*_i\to0$ in $E^*_\beta$ the weak boundedness of $f(X)\subset E_\sigma$ implies
\[\|\<f(\cdot),\xi^*_i\>\|_{C_b(X)_\beta}=\sup_{x\in X}|\<f(x),\xi^*_i\>|\to0,\qquad f\in C_b(X,E_\sigma).\]
On the other hand, for any compact subset $K\subset X$ we have $C_b(X,E_\sigma)\to L(E^*_\tau,C(K)_\beta)$ because for a net $\xi_i^*$ such that $\xi^*_i\to0$ in $E^*_\tau$ the compactness of the closed convex hull of the compact set $f(K)$ in $E_\sigma$ implies that
\[\|\<f(\cdot),\xi^*_i\>\|_{C(K)_\beta}=\sup_{x\in K}|\<f(x),\xi^*_i\>|\to0.\]

Consider
\[L(E^*_\beta,C_b(X)_\beta)\to L(E^*_\beta,C_b(X)_\sigma)\to L(M(\beta X)_\beta,E^{**}_\beta)\to L(M(X)_\beta,E^{**}_\beta)\]
and
\[L(E^*_\tau,C(K)_\beta)=L(E^*_\sigma,C(K)_\sigma)\to L(M(K)_\beta,E_\beta).\]

Note that
\[C_b(X,E_\sigma)\to L\Bigl(\colim_KM(K)_\beta,E_\beta\Bigr)\subset L(M(X)_\beta,E_\beta^{**}).\]
Since $\colim_KM(K)$ is strongly dense in $M(X)_\beta$, where $K$ runs through all compact subsets of $X$, and since $E$ is closed in $E_\beta^{**}$, ....

(c)
Fix $\xi\in C_b(X,E_\sigma)$.
\end{pf}





\begin{prb}
Let $(E,E^*)$ and $(F,F^*)$ be dual pairs.
We prove $L(E^*_\sigma,F_\sigma)=L(E^*_\tau,F_\alpha)$ as sets, where $\alpha$ is any dual topology.
\begin{parts}
\item If $T\in L(E^*_\tau,F_\sigma)$, then $T^*\in L(F_\sigma^*,E_\sigma)$, and in particular $T^*\in L(F_\tau^*,E_\sigma)$.
\item If $T\in L(E^*_\sigma,F_\sigma)$, then $T^*\in L(F_\tau^*,E_\tau)$, and in particular $T^*\in L(F_\tau^*,E_\sigma)$.
\item If $T\in L(E^*_\sigma,F_\sigma)$, then $T^*\in L(F_\beta^*,E_\beta)$.
\end{parts}
\end{prb}
\begin{pf}
(a)
If $\xi^*_i\to0$ in $E^*_\sigma$, then $T^*\xi_i^*\to0$ in $F_\sigma$ since
\[|\<\eta^*,T^*\xi_i^*\>|=|\<T\eta^*,\xi_i^*\>|\to0,\qquad\eta^*\in F^*.\]

(b)
If $\eta^*_i\to0$ in $F^*_\tau$, then $T^*\eta^*_i\to0$ in $E_\tau$ since $T$ preserves compact sets so that
\[\sup_{\xi^*\in C^*}|\<T^*\eta^*_i,\xi^*\>|=\sup_{\xi^*\in C^*}|\<\eta^*_i,T\xi^*\>|\to0.\]

(c)
If $\xi^*_i\to0$ in $E^*_\beta$, then $T^*\xi^*_i\to0$ in $F_\beta$ since $T$ preserves bounded sets so that
\[\sup_{\eta^*\in B^*}|\<\eta^*,T^*\xi^*_i\>|=\sup_{\eta^*\in B^*}|\<T\eta^*,\xi^*_i\>|\to0.\]


\end{pf}


\begin{prb}[Vector-valued differentiable functions]

\end{prb}

\begin{prb}[Vector-valued distributions]
\end{prb}



\begin{prb}[Locally compact group actions]
Let $G$ be a locally compact group and let $(E,E^*)$ be a dual pair.
Let $\alpha:G\to L_\sigma(E_\sigma)$ be a continuous bounded action.
\begin{parts}
\item $\alpha:M(\beta G)_\sigma\to L_\sigma(E_\sigma)$.
\item $\alpha:L^1(G)_\tau\to L_\sigma(E_\tau)$ if
\item $\alpha^*:G\times E_\sigma^*\to E_\sigma^*$ preserves compactness if $E_\tau$ is barrelled, and $E_\sigma^*$ has the Heine-Borel property.
\item $\alpha:G\to L_\sigma(E_\tau)$ if (a) and (b) are satisfied. (if $E=A$, then a point-weakly continuous action is point-norm continuous, and if $E=M$, then a point-$\sigma$-weakly continuous action is point-$\sigma$-strongly continuous)
\end{parts}
\end{prb}
\begin{pf}


(a)
If $(x,x^*)\in E\times E^*$, then $(s\mapsto\<\alpha_s(x),x^*\>)\in C_b(G)$ defines a continuous linear functional on $M(\beta G)$.
Thus, $\spn G\subset M(\beta G)_\sigma\to L_\sigma(E_\sigma)$ can be extended by the continuity. 


(b)

For a bounded set $B^*\in L^\infty(G)$,


$f\mapsto\<\alpha_f(x),x^*\>$ is a linear functional on $L^1(G)$ with norm....?

Let $f_n\to0$ in $L^1(G)_\tau$.
\[|\<\alpha_{f_n}(x),x^*\>|\]

(c)
Suppose $s_i$ and $x_i^*$ are nets in compact subsets of $G$ and $E^*_\sigma$.
We may assume $s_i\to e$ in $G$ and $x_i^*\to0$ in $E^*_\sigma$.
We will show that we can take a subnet such that for each $x\in E$ we have
\[|\<x,\alpha^*_{s_i}(x_i^*)\>|=|\<\alpha_{s_i}(x),x_i^*\>|\le|\<\alpha_{s_i}(x)-x,x_i^*\>|\]
converges.


For some neighborhood $U$ of zero in $E_\tau$,
$\sup_{x\in U,x^*\in C^*,s\in K}|\<\alpha_s(x),x^*\>|\le1$?

$\alpha_K(U)$ is bounded in $E_\tau$?




(d)
We claim that $E_0=E$, where
\[E_0:=\{x\in E:\lim_{s\to e}\alpha_s(x)=x\text{ in }E_\tau\}.\]
We first see that $E_0$ is closed in $E_\tau$.
Let $x_i\in E_0$ be a net such that $x_i\to x$ in $E_\tau$.
Fix $\e>0$ and weakly compact convex set $C^*\subset E^*_\sigma$.
Since the set $\alpha_K^*(C^*)$ is relatively compact in $E^*_\sigma$ by the part (c), the convergence $x_i\to x$ in the Mackey topology implies that the limit $s\to e$ gives
\begin{align*}
\sup_{x^*\in C^*}|\<\alpha_s(x)-x,x^*\>|
&\le\sup_{x^*\in C^*}|\<x-x_i,\alpha_s^*(x^*)\>|+\sup_{x^*\in C^*}|\<\alpha_s(x_i)-x_i,x^*\>|+\sup_{x^*\in C^*}|\<x_i-x,x^*\>|\\
&\to\e+0+\e,
\end{align*}
hence we have the claim $x\in E_0$ by letting $\e\to0$.

Now it suffices to show $E_0$ is dense in $E_\sigma$ by the Hahn-Banach separation and the fact that the Mackey topology is a dual topology.
Since we have a continuous linear map $\alpha:M(\beta G)_\sigma\to L(E_\sigma)_\sigma$ by the part (a), if we take a net $e_i\in C_c(G)$ such that $e_i\to\delta_0$ weakly$^*$ in $M(\beta G)_\sigma$, then for any $x\in E$, the net $\alpha_{e_i}(x)$ belongs to $E_0$ by the uniform continuity of each $e_i$ and the part (b), and it has the convergence $\alpha_{e_i}(x)\to x$ in $E_\sigma$, so we are done.



\end{pf}


\begin{prb}
Let $G$ be a compact Lie group for which the Chevalley complexification can be made.
\end{prb}





	




\section{Direct limit}
distribution theory
LF,LB spaces



\section{Differentiable spaces}



\chapter{Fr\'echet spaces}


\chapter{Banach spaces}

\section{Universal properties}
\subsection*{Notation}
\begin{tabular}{cl}
$L(X,Y)$ & the set of bounded linear operators from $X$ to $Y$\\
$B(X,Y)$ & the set of bounded bilinear forms on $X\times Y$\\
$F(X,Y)$ & the set of continuous finite-rank linear operators from $X$ to $Y$\\
$B_X$ & closed unit ball of a normed space $X$\\
$S_X$ & unit sphere of a normed space $X$\\
$X\otimes Y$ & algebraic tensor product of $X$ and $Y$\\
$X^*$ & continuous dual space\\
$X^\#$ & algebraic dual space
\end{tabular}

\begin{prb}[Algebraic tensor product of vector spaces]
Let $X$ and $Y$ be vector spaces.
The \emph{algebraic tensor product} is a vector space $X\otimes Y$ with a bilinear map $\otimes:X\times Y\to X\otimes Y$ such that the following universal property: for any vector space $Z$ and any bilinear map $\sigma:X\times Y\to Z$, there exists a unique linear map $\tilde\sigma:X\otimes Y\to Z$ such that the diagram
\[\begin{tikzcd}
X\times Y \ar{r}{\otimes}\ar[swap]{dr}{\sigma} & X\otimes Y \ar[dashed]{d}{\tilde\sigma}\\
\, & Z 
\end{tikzcd}\]
is commutative.
\begin{parts}
\item The tensor product $X\otimes Y$ always exists.
\item We have linear maps $L(X,Z)\otimes L(Y,W)\to L(X\otimes Y,Z\otimes W)$ and $B(L(X,Z),L(Y,Z))\to L(X\otimes Y,Z)$.
\item Every element $t\in X\otimes Y$ is represented as $t=\sum_{i=1}^nx_i\otimes y_i$ such that $\{x_i\}$ is linearly indpendent. In this case, if $t=0$ then $y_i=0$ for all $i$.
\end{parts}
\end{prb}
\begin{pf}
(a)
Let $T$ be the set of formal linear combinations of $X\times Y$, that is, an element of $T$ has the form $\sum_{i=1}^na_i\cdot(x_i,y_i)$ for $x_i\in X$, $y_i\in Y$, and scalars $a_i$.
Define $T_0\subset T$ to be a linear space spanned by the elements of the following four types:
\begin{gather*}
(x+x',y)-(x,y)-(x',y),\quad (x,y+y')-(x,y)-(x,y'),\\
(ax,y)-a(x,y), \quad\qquad\qquad (x,ay)-a(x,y).
\end{gather*}
Then, the quotient space $T/T_0$ satisfies the universal property with the bilinear map $X\times Y\to T/T_0:(x,y)\mapsto(x,y)+T_0$.
\end{pf}

\begin{prb}[Algebraic tensor product of involutive algebras]

\end{prb}



\section{Banach spaces}

\begin{prb}[Subcross norms]

\end{prb}

\begin{prb}[Injective tensor products]
Let $X$ and $Y$ be Banach spaces.
Define the \emph{injective norm} $\e$ on $X\otimes Y$ such that
\[\e\left(\sum_{i=1}^nx_i\otimes y_i\right):=\sup_{\substack{x^*\in B_{X^*}\\y^*\in B_{Y^*}}}\left|\sum_{i=1}^n\<x_i,x^*\>\<y_i,y^*\>\right|.\]
We denote by $X\otimes_\e Y$ the algebraic tensor product with the injective norm, and by $X\hat\otimes_\e Y$ its completion.
\begin{parts}
\item $X\otimes_\e Y$ is naturally isometrically isomorphic to $F((X^*,w^*),(Y,w))$.
\item $X^*\otimes_\e Y$ is naturally isometrically isomorphic to $F(X,Y)$.
\end{parts}
\end{prb}

\begin{prb}[Projective tensor products]
Let $X$ and $Y$ be Banach spaces.
Define the \emph{projective norm} $\pi$ on $X\otimes Y$ such that
\[\pi\left(t\right):=\inf\left\{\sum_{i=1}^n\|x_i\|\|y_i\|:t=\sum_{i=1}^nx_i\otimes y_i\right\}.\]
We denote by $X\otimes_\pi Y$ the algebraic tensor product with the projective norm, and by $X\hat\otimes_\pi Y$ its completion.
\begin{parts}
\item There are natural isometric isomorphisms $(X\otimes_\pi Y)^*\cong B(X,Y)\cong L(X,Y^*)$.
\item
\end{parts}
\end{prb}

\begin{prb}[Hilbert space tensor product]

Let $\f:H\otimes K\to L(H^*,K)$.
Then, $\lambda(\xi)=\|\f(\xi)\|$, $\gamma(\xi)=\tr(|\f(\xi)|)$, so $H\hat\otimes_\lambda K\cong K(H^*,K)$ and $H\hat\otimes_\gamma K\cong L^1(H^*,K)$.
\end{prb}


\begin{prb}[Nuclear operators]
\[X^*\otimes_\pi Y\to X^*\otimes_\e Y\xrightarrow{\sim} F(X,Y)\xrightarrow{1}K(X,Y)\]
defines
\[J:X^*\hat\otimes_\pi Y\to K(X,Y).\]
Define $N(X,Y):=\im J$.
\end{prb}

\begin{prb}[Grothendieck theorem]
Let $Y^*$ be an RNP space.
Then, there is an isometric isomorphism $(X\hat\otimes_\e Y)^*\cong N(X,Y^*)$.
\end{prb}

\section{Approximation property}

\begin{prb}[Approximation property of locally convex spaces]
\end{prb}

\begin{prb}[Approximation property of Banach spaces]
\end{prb}

\begin{prb}[Approximation property of dual Banach spaces]
\end{prb}

\begin{prb}[Mazur's goose]
\begin{parts}
\item If $X$ has a Schauder basis, then it has the approximation property.
\end{parts}
\end{prb}



\section{Nuclear spaces}










\part{Topological algebras}

\chapter{Locally convex algebras}


\chapter{Fr\'echet algebras}


For a Fr\'echet algebra $A$, 


\chapter{Banach algebras}

\end{document}