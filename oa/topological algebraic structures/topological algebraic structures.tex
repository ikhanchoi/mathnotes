\documentclass{../../large}
\usepackage{../../ikhanchoi}




\begin{document}
\title{Topological Algebraic Structures}
\author{Ikhan Choi}
\maketitle
\tableofcontents

\part{}



\chapter{Topological groups}

\chapter{Topological vector spaces}
\section{Locally convex spaces}
categorical aspects,
bornology,
tensor products,




\subsection*{Generalized Pettis integral}

\begin{prb}[Properties of dual pairs]
Let $(E,E^*)$ be a dual pair.
We say $(E,E^*)$ has the \emph{Krein property} if the closed balanced convex hull of a compact subset of $X$ is compact in the topology $\sigma(E,E^*)$, and say $(E,E^*)$ has the \emph{Goldstine property} if $E$ is $\beta(E,E^*_\beta)$-closed in the strong bidual $(E^*_\beta)^*_\beta$.

Let $E$ a Banach space.
The weak dual pair $(E,E^*)$ satisfies the Krein property by the Krein-\v Smulian theorem, and the Goldstine property by the closedness of $E$ in $E^{**}$.
If there is a predual $E_*$ of $E$, then the weak$^*$ dual pair $(E,E_*)$ satisfies the Krein property by the fact that the closed convex hull of a bounded set is bounded, and the Golstine property because the norm topology and $\beta(E,(E_*)_\beta)$ coincide by the Goldstine theorem.
In particular, a dual pair $(E,F)$ with $F\subset E^*$ has the Goldstine property if and only if the closed unit ball $F_1=F\cap E^*_1$ is weakly$^*$ dense in the closed ball $E^*_1$.
\end{prb}

\begin{prb}[Well-definedness of Pettis integral]
Let $(\Omega,\mu)$ be a localizable measure space and $(X,F)$ is a dual pair.
Let $x:\Omega\to X$ be a $\sigma(X,F)$-bounded $\sigma(X,F)$-measurable function in the sense that it determines a linear operator $F\to L^\infty(\mu)$.
By the transpose and restriction, we have a linear operator $\phi_x:L^1(\mu)\to F^\#$, which satisfies
\[\<\phi_x(f),x^*\>:=\int_\Omega \<x(s),x^*\>f(s)\,d\mu(s),\qquad f\in L^1(\mu),\ x^*\in F.\]
We usually write as
\[\phi_x(f)=\int_\Omega x(s)f(s)\,d\mu(s).\]
\begin{parts}
\item $\phi_x(L^1(\mu))\subset(F_\beta)^*$ and $\phi_x$ is always weak-$\sigma((F_\beta)^*,F)$-continuous.
\item Suppose $(X,F)$ has the Krein property.
If $x$ is $\sigma(X,F)$-compactly valued, then $\phi_x(L^1(\mu))\subset X$.
\item Suppose $(X,F)$ has the Krein and Goldstine property.
Suppose $\Omega$ is a locally compact Hausdorff space with a Radon measure $\mu$.
If $x$ is $\sigma(X,F)$-continuous, then $\phi_x(L^1(\mu))\subset X$.
(In fact, the continuity of $x$ defines $F\to C_b(\Omega)$, we can prove $\phi_x(M(\beta\Omega))\subset X$. It does not require the data of $\mu$.)
\item Suppose we have $\phi_x(L^1(\mu))\subset X$. Let $Y$ be another topological vector space and $G$ is a weakly$^*$ dense subspace of $Y^*$. If $T:X\to Y$ is a $\sigma(X,F)$-$\sigma(Y,G)$-continuous linear operator, then $T\phi_x=\phi_{T\circ x}$. In other words,
\[T\int_\Omega f(s)x(s)\,d\mu(s)=\int_\Omega f(s)Tx(s)\,d\mu(s),\qquad f\in L^1(\mu).\]
\item Suppose we have $\phi_x(L^1(\mu))\subset X$, $(X,F)$ has the Goldstine property, and $X$ is a Banach space. Then,
\[\|\int f(s)x(s)\,d\mu(s)\|\le\int\|f(s)x(s)\|\,d\mu(s),\qquad f\in L^1(\mu).\]
\end{parts}
\end{prb}
\begin{pf}
(a)
Let $B^*\subset F$ be a $\beta(F,X_\sigma)$-bounded set.
For $x^*\in F$ we have an inequality
\[|\<\phi_x(f),x^*\>|\le\int_\Omega |f(s)\<x(s),x^*\>|\,d\mu(s)\le\|f\|_{L^1}\sup_{y\in x(\Omega)}|\<y,x^*\>|,\]
and a bound
\[\sup_{x_*\in B^*}\sup_{y\in x(\Omega)}|\<y,x^*\>|<\infty\]
due to the $\sigma(X,F)$-boundedness of $x(\Omega)$, so $\phi_x(f)\in(F_\beta)^*$.
If $f_\alpha\in L^1(\mu)$ converges weakly to zero, then
\[\<\phi_x(f_\alpha),x^*\>=\int_\Omega f(s)\<x(s),x^*\>\,d\mu(s)\to0,\qquad x^*\in F\]
because $x$ is $\sigma(X,F)$-integrable so that $(s\mapsto\<x(s),x^*\>)\in L^\infty(\mu)$, so the continuity of $\phi_x$.

(b)
Fix $p\in L^\infty(\mu)$ and let $C$ be the $\sigma(X,F)$-closed balanced convex hull of $x(\Omega)\subset X$.
Then $C$ is $\sigma(X,F)$-compact by the Krein property.
Since for every $x^*\in F$ we have
\[|\<\phi_x(f),x^*\>|\le\int_\Omega|f(s)\<x(s),x^*\>|\,d\mu(s)\le\|f\|_{L^1}\sup_{y\in x(\Omega)}|\<y,x^*\>|\le\|f\|_{L^1}\sup_{y\in C}|\<y,x^*\>|,\]
the linear functional $\phi_x(f)$ on $F$ is continuous with respect to the Mackey topology $\tau(F,X)$, which is a dual topology so that $\phi_x(f)$ can be naturally identified with a vector in $(F_\tau)^*=X$.

(c)
Fix $f\in L^1(\mu)$.
By the tightness of $\mu$, there is a sequence of compact sets $K_n\subset\Omega$ such that $\int_{\Omega\setminus K_n}|f(s)|\,d\mu(s)<n^{-1}$.
Since for each $x^*\in F$ we have
\[|\<\phi_x(f)-\phi_{x|_{K_n}}(f),x^*\>|\le\int_{\Omega\setminus K_n}|f(s)|\,d\mu(s)\cdot\sup_{s\in\Omega}|\<x(s),x^*\>|<n^{-1}\sup_{y\in x(\Omega)}|\<y,x^*\>|\]
so that
\[\sup_{x^*\in B^*}|\<\phi_x(f)-\phi_{x|_{K_n}}(f),x^*\>|\le n^{-1}\sup_{x_*\in B^*}\sup_{y\in x(\Omega)}|\<y,x^*\>|\to0,\qquad n\to\infty,\]
which means that $\phi_{x|_{K_n}}(f)$ converges to $\phi_x(f)$ in $\beta((F_\beta)^*,F_\beta)$.
Since $\phi_{x|_{K_n}}(f)\in X$ by the part (b) and $X$ is closed in $\beta((F_\beta)^*,F_\beta)$ by the Goldstine property, we have $\phi_x(f)\in X$.

(d)
By the continuity of $T$, the adjoint $T^*:G\to F$ is well-defined.
The measurability of $T$ and the existence of the adjoint $T^*$ imply that the composition $T\circ x:\Omega\to Y$ is $\sigma(Y,G)$-bounded and $\sigma(Y,G)$-measurable, so the operator $\phi_{T\circ x}:L^1(\mu)\to G^\#$ is well-defined.
Then,
\begin{align*}
\<T\phi_x(f),y^*\>&=\<\phi_x(f),T^*y^*\>=\int_\Omega f(s)\<x(x),T^*y^*\>\,d\mu(s)\\
&=\int_\Omega f(s)\<Tx(s),y^*\>\,d\mu(s)=\<\phi_{T\circ x}(f),y^*\>,\qquad f\in L^1(\mu),\ y^*\in G.
\end{align*}
In particular, $\phi_{T\circ x}:L^1(\mu)\to Y$.

(e)
By the Goldstine property,
\begin{align*}
\|\int f(s)x(s)\,d\mu(s)\|
&=\sup_{x^*\in F_1}|\int f(s)x(s)\,d\mu(s)|
\le\sup_{x^*\in F_1}\int|f(s)x(s)|\,d\mu(s)\\
&\le\int\sup_{x^*\in F_1}|f(s)x(s)|\,d\mu(s)
\le\int\|f(s)x(s)\|\,d\mu(s).\qedhere\\
\end{align*}
\end{pf}

\begin{prb}[Topological tensor products]
Let $X$ and $Y$ be locally convex spaces.
The \emph{projective tensor product} is the completion $X\hat\otimes_\pi Y$ of $X\otimes Y$ with the finest locally convex topology such that the canonical bilinear map $X\times Y\to X\otimes Y$ is continuous.
We can also describe it with semi-norms.
We have
\[B_{\mathrm{jnt}}(X,Y)\cong(X\hat\otimes_\pi Y)^*.\]


Note that we have
\[X\otimes Y\cong B_{\mathrm{jnt}}(X_\sigma^*,Y_\sigma^*)\subset B_{\mathrm{sep}}(X_\sigma^*,Y_\sigma^*).\]
The space $B_{\mathrm{sep}}(X_\sigma^*,Y_\sigma^*)$ of separately continuous bilinear forms, which has a natural topology of uniform convergence on the products of equicontinuous sets in $X_\sigma^*$ and $Y_\sigma^*$, and this topology is complete if and only if $X$ and $Y$ are complete.
The induced topology on $X\otimes Y$ is called the \emph{injective tensor product} topology.
We have $C^k(\Omega,E)\cong C^k(\Omega)\hat\otimes_\e E$ if $E$ is complete.

Note that the projective tensor product reflects the original topologies of locally convex spaces, while the injective tensor product only depends on the dual pair structure.

The dual of $X\hat\otimes_\pi Y\to X\hat\otimes_\e Y$ defines an injection $J(X,Y)\to B_{\mathrm{jnt}}(X,Y)$.
A bilinear form in $J(X,Y)$ is called to be \emph{integral}.
\end{prb}


\begin{prb}[Vector-valued continuous functions]
Let $X$ be a locally compact Hausdorff space, and $(E,E^*)$ be a dual pair satisfying the two properties.

We claim there is an embedding [Tre 44.1]
\[C_0(X,E)\to C_0(X,E_\sigma)\subset L(E^*_\tau,C_0(X))=L(E^*_\sigma,C_0(X)_\sigma)=L(M(X)_\sigma, E_\sigma).\]
How about $C_c$, $C_0$, $C_b$, $C$?
See [Tre 42.2] for $L(E_\tau^*,F)=L(E_\sigma^*,F_\sigma)$.
Since $C_0(X)\odot E$ is dense in $C_0(X,E)$ for any locally convex space $E$, the above embedding gives rise to a dense embedding $C_0(X,E_\sigma)\subset C_0(X)\hat\otimes_\e E\subset B_{\mathrm{sep}}(M(X)_\sigma,E^*_\sigma)$.
\end{prb}

\begin{prb}[Vector-valued measurable functions]
We need to investigate the natural topology and its weak topology on $L_\loc^0(\mu)$.
I want to do this in measure theory.

Continuous approximations
\end{prb}

\begin{prb}[Vector-valued differentiable functions]
H\"older, Sobolev, etc.
\end{prb}


\begin{prb}[Vector-valued distributions]
\end{prb}

\begin{prb}[Relations to Bochner and Pettis integrals]
Bochner integral can be justified in terms of projective tensor products.

A weakly measurable function on $(\Omega,\mu)$ valued in $E$ gives rise to a linear map $E^*\to L_\loc^0(\mu)$.
Is it continuous?
\end{prb}


	




\section{Direct limit}
distribution theory
LF,LB spaces
\section{Differentiable spaces}


\chapter{Topological algebras}

\part{}
\chapter{Continuous fields}


\part{Fr\'echet and Banach spaces}

\chapter{}
\section{Universal properties}
\subsection*{Notation}
\begin{tabular}{cl}
$L(X,Y)$ & the set of bounded linear operators from $X$ to $Y$\\
$B(X,Y)$ & the set of bounded bilinear forms on $X\times Y$\\
$F(X,Y)$ & the set of continuous finite-rank linear operators from $X$ to $Y$\\
$B_X$ & closed unit ball of a normed space $X$\\
$S_X$ & unit sphere of a normed space $X$\\
$X\otimes Y$ & algebraic tensor product of $X$ and $Y$\\
$X^*$ & continuous dual space\\
$X^\#$ & algebraic dual space
\end{tabular}

\begin{prb}[Algebraic tensor product of vector spaces]
Let $X$ and $Y$ be vector spaces.
The \emph{algebraic tensor product} is a vector space $X\otimes Y$ with a bilinear map $\otimes:X\times Y\to X\otimes Y$ such that the following universal property: for any vector space $Z$ and any bilinear map $\sigma:X\times Y\to Z$, there exists a unique linear map $\tilde\sigma:X\otimes Y\to Z$ such that the diagram
\[\begin{tikzcd}
X\times Y \ar{r}{\otimes}\ar[swap]{dr}{\sigma} & X\otimes Y \ar[dashed]{d}{\tilde\sigma}\\
\, & Z 
\end{tikzcd}\]
is commutative.
\begin{parts}
\item The tensor product $X\otimes Y$ always exists.
\item We have linear maps $L(X,Z)\otimes L(Y,W)\to L(X\otimes Y,Z\otimes W)$ and $B(L(X,Z),L(Y,Z))\to L(X\otimes Y,Z)$.
\item Every element $t\in X\otimes Y$ is represented as $t=\sum_{i=1}^nx_i\otimes y_i$ such that $\{x_i\}$ is linearly indpendent. In this case, if $t=0$ then $y_i=0$ for all $i$.
\end{parts}
\end{prb}
\begin{pf}
(a)
Let $T$ be the set of formal linear combinations of $X\times Y$, that is, an element of $T$ has the form $\sum_{i=1}^na_i\cdot(x_i,y_i)$ for $x_i\in X$, $y_i\in Y$, and scalars $a_i$.
Define $T_0\subset T$ to be a linear space spanned by the elements of the following four types:
\begin{gather*}
(x+x',y)-(x,y)-(x',y),\quad (x,y+y')-(x,y)-(x,y'),\\
(ax,y)-a(x,y), \quad\qquad\qquad (x,ay)-a(x,y).
\end{gather*}
Then, the quotient space $T/T_0$ satisfies the universal property with the bilinear map $X\times Y\to T/T_0:(x,y)\mapsto(x,y)+T_0$.
\end{pf}

\begin{prb}[Algebraic tensor product of involutive algebras]

\end{prb}



\section{Banach spaces}

\begin{prb}[Subcross norms]

\end{prb}

\begin{prb}[Injective tensor products]
Let $X$ and $Y$ be Banach spaces.
Define the \emph{injective norm} $\e$ on $X\otimes Y$ such that
\[\e\left(\sum_{i=1}^nx_i\otimes y_i\right):=\sup_{\substack{x^*\in B_{X^*}\\y^*\in B_{Y^*}}}\left|\sum_{i=1}^n\<x_i,x^*\>\<y_i,y^*\>\right|.\]
We denote by $X\otimes_\e Y$ the algebraic tensor product with the injective norm, and by $X\hat\otimes_\e Y$ its completion.
\begin{parts}
\item $X\otimes_\e Y$ is naturally isometrically isomorphic to $F((X^*,w^*),(Y,w))$.
\item $X^*\otimes_\e Y$ is naturally isometrically isomorphic to $F(X,Y)$.
\end{parts}
\end{prb}

\begin{prb}[Projective tensor products]
Let $X$ and $Y$ be Banach spaces.
Define the \emph{projective norm} $\pi$ on $X\otimes Y$ such that
\[\pi\left(t\right):=\inf\left\{\sum_{i=1}^n\|x_i\|\|y_i\|:t=\sum_{i=1}^nx_i\otimes y_i\right\}.\]
We denote by $X\otimes_\pi Y$ the algebraic tensor product with the projective norm, and by $X\hat\otimes_\pi Y$ its completion.
\begin{parts}
\item There are natural isometric isomorphisms $(X\otimes_\pi Y)^*\cong B(X,Y)\cong L(X,Y^*)$.
\item
\end{parts}
\end{prb}

\begin{prb}[Hilbert space tensor product]

Let $\f:H\otimes K\to L(H^*,K)$.
Then, $\lambda(\xi)=\|\f(\xi)\|$, $\gamma(\xi)=\tr(|\f(\xi)|)$, so $H\hat\otimes_\lambda K\cong K(H^*,K)$ and $H\hat\otimes_\gamma K\cong L^1(H^*,K)$.
\end{prb}


\begin{prb}[Nuclear operators]
\[X^*\otimes_\pi Y\to X^*\otimes_\e Y\xrightarrow{\sim} F(X,Y)\xrightarrow{1}K(X,Y)\]
defines
\[J:X^*\hat\otimes_\pi Y\to K(X,Y).\]
Define $N(X,Y):=\im J$.
\end{prb}

\begin{prb}[Grothendieck theorem]
Let $Y^*$ be an RNP space.
Then, there is an isometric isomorphism $(X\hat\otimes_\e Y)^*\cong N(X,Y^*)$.
\end{prb}

\section{Approximation property}

\begin{prb}[Approximation property of locally convex spaces]
\end{prb}

\begin{prb}[Approximation property of Banach spaces]
\end{prb}

\begin{prb}[Approximation property of dual Banach spaces]
\end{prb}

\begin{prb}[Mazur's goose]
\begin{parts}
\item If $X$ has a Schauder basis, then it has the approximation property.
\end{parts}
\end{prb}



\section{Nuclear spaces}



\part{Fr\'echet and Banach algebras}

\chapter{Fr\'echet algebras}

\chapter{Banach algebras}

\end{document}