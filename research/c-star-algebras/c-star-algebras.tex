\documentclass{../../large}
\usepackage{../../ikhanchoi}

\newcommand{\Prim}{\operatorname{Prim}}



\begin{document}
\title{C$^*$-Algebras}
\author{Ikhan Choi}
\maketitle
\tableofcontents

\part{Constructions}


\chapter{Completely positive maps}
\section{Operator systems and spaces}


\begin{prb}[Choi-Effros characterization]

\end{prb}

\begin{prb}[Von Neumann inequality]
\end{prb}


The set $M_n(A)_+$ is linearly spanned by elements of the form $[a_i^*a_j]\in M_n(A)$ for $[a_i]\in A^n$.
A linear map $\f:A\to B$ is completely positive if
\[\f(a_i^*a_j)\]


\begin{prb}[$n$-positive maps]
Let $S$ be an operator system.
Let $A$ and $B$ be C$^*$-algebras.
\begin{parts}
\item (Cauchy-Schwarz inequality)
Let $\f:A\to B$ be a 2-positive map.
Then,
\[\f(a)^*\f(a)\le\lim_\alpha\|\f(e_\alpha)\|\f(a^*a)\] for all $a\in A$, where $e_\alpha$ be an approximate unit of $A$.
In particular, $\lim_\alpha\|\f(e_\alpha)\|=\|\f\|$.
\item (Multiplicative domain)
Let $\f:A\to B$ be a 4-positive map with $\|\f\|=1$.
If $a\in A$ satisfies $\f(a)^*\f(a)=\f(a^*a)$, then $\f(b)\f(a)=\f(ba)$ for all $b\in A$.
In particular, if $\f:B\to C$ is an extension of a $*$-homomorphism $\pi:A\to C$, then $\f(ab)=\pi(a)\f(b)$ and $\f(ba)=\f(b)\pi(a)$ for $a\in A$ and $b\in B$.
\end{parts}
\end{prb}
\begin{pf}
(a)
Consider $B$ to act on a Hilbert space $H$ non-degenerately and faithfully.
The 2-positivity of $\f$ and
\[\mat{e_\alpha^2&e_\alpha a\\a^*e_\alpha&a^*a}=\mat{e_\alpha&a\\0&0}^*\mat{e_\alpha&a\\0&0}\ge0\]
implies
\[\mat{\f(e_\alpha^2)&\f(e_\alpha a)\\\f(a^*e_\alpha)&\f(a^*a)}\ge0,\]
which is equivalent to have
\[\<\f(e_\alpha^2)\xi,\xi\>+2\Re\<\f(e_\alpha a)\eta,\xi\>+\<\f(a^*a)\eta,\eta\>\ge0\]
for any $\xi,\eta\in H$.
We put $\xi:=-(\|\f(e_\alpha)\|+\e)^{-1}\f(e_\alpha a)\eta$ for $\e>0$ to get
\begin{align*}
(\|\f(e_\alpha)\|+\e)\f(a^*a)
&\ge\f(e_\alpha a)^*(2-(\|\f(e_\alpha)\|+\e)^{-1}\f(e_\alpha^2))\f(e_\alpha a)\\
&\ge\f(e_\alpha a)^*\f(e_\alpha a).
\end{align*}
We have the desired inequality by taking limits for $\alpha$ and $\e$.

(b)
The 2-positivity of $\f_2$ gives
\[\f_2\left(\mat{a&b\\0&0}\right)^*\f_2\left(\mat{a&b\\0&0}\right)\le\f_2\left(\mat{a^*a&a^*b\\b^*a&b^*b}\right),\]
so
\[\mat{0&\f(a^*b)-\f(a^*)\f(b)\\\f(b^*a)-\f(b^*)\f(a)&\f(b^*b)-\f(b^*)\f(b)}\ge0,\]
which implies $\f(b^*a)-\f(b^*)\f(a)=0$ for any $b\in A$.

Note that $\|\pi\|=1$ if $\pi$ is not trivial.
Using the above argument for $a$ and $a^*$, we are done.
\end{pf}




\begin{prb}[Russo-Dye theorem]
If $C(X)\to B$ is positive, then it is c.p.
\end{prb}


\begin{prb}[Completely positive maps for matrix algebras]
Let $A$ be a C$^*$-algebra.
\begin{parts}
\item Choi matrix
\item
There is a one-to-one correspondence
\[\mathrm{CP}(M_n(\C),A)\to M_n(A)_+:\f\mapsto[\f(e_{ij})].\]
\item
Let $A$ be unital.
There is a one-to-one correspondence
\[\mathrm{CP}(A,M_n(\C))\to M_n(A)^*_+:\f\mapsto(s_\f:[a_{ij}]\mapsto\sum_{i,j}\<\f(a_{ij})e_j,e_i\>).\]
\item The above correspondences are (maybe?) isometric if we endow the complete norm on $\mathrm{CP}$.
\end{parts}
\end{prb}
\begin{pf}
(b)


\end{pf}

\section{Dilations and Extensions}

\begin{prb}[Stinespring dilation]
Let $A$ be a C$^*$-algebra and $\f:A\to B(H)$ is a c.p.~map.
There exist a representation $\pi:A\to B(K)$ and a bounded linear operator $V:H\to K$ such that $\f(a)=V^*\pi(a)V$ for $a\in A$.
\[\begin{tikzcd}
B(K) \ar{dr}{V^*\cdot V} &\\
A \ar[swap]{r}{\f} \ar{u}{\pi} & B(H)
\end{tikzcd}\]
\begin{parts}
\item If $\|\f\|=1$, then $V$ is an isometry.
\item 
\item We can take $\pi$ to be minimal in the sense that $\bar{\pi(A)VH}=K$.
\end{parts}
\end{prb}
\begin{pf}

\end{pf}



\begin{prb}[Arveson extension]
Let $A\subset B$ be C$^*$-algebras.
Let $\f:A\to B(H)$ be a c.p.~map and consider the following diagram:
\[\begin{tikzcd}
B\ar[dashed]{dr}{\tilde\f}&\\
A\ar{u}\ar[swap]{r}{\f}&B(H).
\end{tikzcd}\]
\begin{parts}
\item The norm preserving c.p.~extension $\tilde\f$ of $\f$ exists if $B$ is unital and $1_B\in A$.
\item The norm preserving c.p.~extension $\tilde\f$ of $\f$ exists if $\cA$ is unital and $B=A\oplus\C$.
\item The norm preserving c.p.~extension $\tilde\f$ of $\f$ exists if $\cA$ is non-unital and $B=\tilde\cA$.
\item The norm preserving c.p.~extension $\tilde\f$ of $\f$ always exists.
\end{parts}
\end{prb}

\begin{prb}[Representation extension]
Let $I$ be a closed ideal of a C$^*$-algebra $B$.
For a representation $\pi:I\to B(H)$, there is a representation $\tilde\pi:B\to B(H)$ which extends $\pi$.
If $\pi$ is non-degenerate, the extension is unique and $\pi(e_\alpha b)\to\tilde\pi(b)$ and $\pi(be_\alpha)\to\tilde\pi(b)$ strongly for $b\in B$, where $e_\alpha$ is an approximate unit of $I$.
\end{prb}
\begin{pf}
We may assume $\pi$ is non-degenerate by replacing $H$ to $\bar{\pi(I)H}$.
Define $\tilde\pi:B\to B(H)$ such that
\[\tilde\pi(b)(\pi(a)\xi):=\pi(ba)\xi,\qquad a\in I,\ \xi\in H.\]
The well-definedness is from
\[\|\pi(ba)\xi\|^2=\<\pi(a^*b^*ba)\xi,\xi\>\le\|b\|^2\<\pi(a^*a)\xi,\xi\>=\|b\|^2\|\pi(a)\xi\|^2.\]
It is clearly a $*$-homomorphism and extends $\pi$.

For the uniqueness, if $\pi$ is non-degenerate and $\tilde\pi$ is a $*$-homomorphism which extends $\pi$, then
\[\tilde\pi(b)(\pi(a)\xi)=\tilde\pi(b)\tilde\pi(a)\xi=\tilde\pi(ba)\xi=\pi(ba)\xi,\]
which is unique by the density of $\pi(I)H$ in $H$.
\end{pf}

extension of representations for ideals

unique extension of c.p.~maps for hereditary subalgebras.



\section{Completely bounded maps}




\section{Tensor products}

\begin{prb}[Maximal tensor products]
Let $A$ and $B$ be C$^*$-algebras.
\begin{parts}
\item (restrictions) A commuting pair of $*$-homomorphisms $\pi:A\to B(H)$ and $\pi':B\to B(H)$ corresponds to a $*$-homomorphism $\Pi:A\odot B\to B(H)$ via the relation $\Pi(a\otimes b)=\pi(a)\pi'(b)$.
\item $A\odot B$ admits a $*$-representation and every norms induced from these $*$-representations are uniformly bounded. So, we can define a maximal tensor norm on $A\odot B$.
\item $a\otimes-:B\to A\odot B$ is a bounded linear map for each $a\in A$ with respect to any C$^*$-norm on $A\odot B$. [BO, 3.2.5]
\end{parts}
\end{prb}


\begin{prb}[Minimal tensor product]
spatiality
\end{prb}
\begin{prb}[Takesaki theorem]
\end{prb}

Tensors with $M_n(\C)$, $C_0(X)$.


\begin{prb}[Haagerup tensor product]
\end{prb}

Trick

\section*{Exercises}
\begin{prb}
Let $A$ be a hereditary C$^*$-subalgebra of a C$^*$-algebra $B$ and let $b\in B_+$.
If for any $\e>0$ there is $a\in A_+$ such that $b-a\le\e$, then $b\in A$.
\end{prb}
\begin{pf}
For $a\in A_+$ satisfying $b\le a+\e\le(a^{\frac12}+\e^{\frac12})^2$, define
\[a_\e:=a^{\frac12}(a^{\frac12}+\e^{\frac12})^{-1}ba^{\frac12}(a^{\frac12}+\e^{\frac12})^{-1}\in A.\]
Then, 
\[\|b^{\frac12}-b^{\frac12}a^{\frac12}(a^{\frac12}+\e^{\frac12})^{-1}\|^2=\e\|(a^{\frac12}+\e^{\frac12})^{-1}b(a^{\frac12}+\e^{\frac12})^{-1}\|\le\e.\]
Thus $a_\e\to b$ in norm as $\e\to0$.
\end{pf}









\chapter{Hilbert modules}

\section{Operator algebraic modules}

\begin{prb}[Banach modules]
Let $A$ be a Banach algebra.
A \emph{Banach $A$-module} is a Banach space $\cE$ which is a $A$-module such that the action is bounded.
\begin{parts}
\item (Cohen factorization theorem) If $A$ has a left approximate unit, then $A\cE$ is closed in $\cE$.
\end{parts}
\end{prb}
\begin{pf}
Suppose $\xi$ belongs to the closure of $A\cE$ and take $\e>0$.
We will construct a decreasing sequence $a_n$ in the unitization $\tilde A$ such that $a_n^{-1}\xi$ and $a_n$ are both Cauchy.
In order to do this, we first need to check $a_n^{-1}\in\tilde A\setminus A$ can act on $\cE$, which is easy anyway.

Let $a_0=1$ and suppose we have defined $a_n\ge 2^{-n}$.
Take $b\in A$ and $\eta$ such that $\|\xi-b\eta\|<\e2^{-(2n+1)}$.
Take $e\in A$ such that $\|a_n^{-1}b-ea_n^{-1}b\|\|\eta\|<\e2^{-(n+1)}$.
Now inductively define
\[a_{n+1}:=a_n-2^{-(n+1)}(1-e)\in\tilde A\]
so that $a_{n+1}\ge2^{-(n+1)}$ is invertible.

Then, we can check $a_n$ is Cauchy whose limit point belongs to $A$, and
$a_n^{-1}\xi$ is Cauchy because by the identity
\[a_{n+1}^{-1}-a_n^{-1}=a_{n+1}^{-1}(a_n-a_{n+1})a_n^{-1}=a_{n+1}^{-1}2^{-(n+1)}(1-e)a_n^{-1}\]
we get
\begin{align*}
\|a_{n+1}^{-1}\xi-a_n^{-1}\xi\|
&\le\|a_{n+1}^{-1}-a_n^{-1}\|\|\xi-b\eta\|+\|(a_{n+1}^{-1}-a_n^{-1})b\|\|\eta\|\\
&\le2^n\cdot\e2^{-(2n+1)}+\e2^{-(n+1)}=\e2^{-n}.
\end{align*}
\end{pf}

\begin{prb}[Finsler modules]
Let $A$ be a C$^*$-algebra.
\end{prb}

\begin{prb}[Hilbert modules]
Let $A$ be a C$^*$-algebra.
A \emph{Hilbert $A$-module} is a complex linear space $\cE$ which is a right $A$-module together with a
\begin{enumerate}[(i)]
\item a ring homomorphism $A^{op}\to\End_\C(\cE)$,
\item an $A$-valued inner product $\<\cdot,\cdot\>:\cE\times\cE\to A$ which is $A$-linear in second argument,
\end{enumerate}
which is complete with respect to the norm $\|\xi\|:=\|\<\xi,\xi\>\|^{\frac12}$.
\begin{parts}
\item
\end{parts}
\end{prb}



constructions:
direct sum, tensor product, localization

examples:
A itself


\section{Multiplier algebras}
\begin{prb}[Double centralizer characterization]
Let $A$ be a C$^*$-algebra.
A \emph{double centralizer} of $A$ is a pair $(L,R)$ of bounded linear maps on $A$ such that $aL(b)=R(a)b$ for all $a,b\in A$.
The \emph{multiplier algebra} $M(A)$ of $A$ is defined to be the set of all double centralizers of $A$.
There is another characterization $M(A):=L_A(A)$, the set of adjointable operators to itself.
\end{prb}


\begin{prb}[Strict topology]
Let $a_\alpha$ be a net in $M(A)$
\begin{parts}
\item $\|\pi(a-e_\alpha a)\xi\|^2$
\item If $a_\alpha$ are unitary, the convergences in the strict topology and the weak topology(how to define this?) coincide.
\item If $a_\alpha$ are increasing, the convergences in the strict topology and the weak topology(how to define this?) coincide.
\end{parts}
\end{prb}

\begin{prb}[Essential ideals]
\begin{parts}
\item Hilbert C$^*$-module description
\end{parts}
\end{prb}


\begin{prb}[Examples of multiplier algebras]
\begin{parts}
\item $M(K(H))\cong B(H)$.
\item $M(C_0(\Omega))\cong C_b(\Omega)$.
\end{parts}
\end{prb}
\begin{pf}
(a)

(b)
First we claim $C_0(\Omega)$ is an essential ideal of $C_b(\Omega)$.
Since $C_b(\Omega)\cong C(\beta\Omega)$, and since closed ideals of $C(\beta\Omega)$ are corresponded to open subsets of $\beta\Omega$, $C_0(\Omega)\cap J$ is not trivial for every closed ideal $J$ of $C_b(\Omega)$.

Now we have an injective $*$-homomorphism $C_b(\Omega)\to M(C_0(\Omega))$, for which we want to show the surjectivity.
Let $g\in M(C_0(\Omega))_+$.
\end{pf}


\begin{prb}[Morphisms]
Let $A$ and $B$ be C$^*$-algebras.
A \emph{morphism} from $A$ to $B$ is defined to be a non-degenerate $*$-homomorphism $A\to M(B)$.
\begin{parts}
\item The inclusion of a proper closed ideal is never a morphism. A surjective $*$-homomorphism $A\to B$ is always a morphism.
\end{parts}
\end{prb}


Every morphism $A\to M(B)$ induces the following?:
\[\begin{tikzcd}
PS(B) \rar[->>]\dar & \hat B \rar[->>]\dar & \Prim(B) \dar \\
PS(A) \rar[->>] & \hat A \rar[->>] & \Prim(A).
\end{tikzcd}\]








\section{Pimsner algebras}


\begin{prb}[C$^*$-correspondences]
Let $A$ be a C$^*$-algebra.
A \emph{C$^*$-correspondence} over $A$ is a right Hilbert $A$-module $\cE$ together with a $*$-homomorphism $\f:A\to B(\cE)$, called the \emph{left action}.
We say $\cE$ is \emph{faithful} or \emph{non-degenerate} if $\f$ is faithful or non-degenerate, respectively.
\begin{parts}
\item If $\f:A\to M(B)$ is a unital completely positive map, then we can construct a natural $A$-$B$-correspondence $\cE$ by mimicking the GNS construction on $A\odot B$.
\item If $\f:A\to M(B)$ is a non-degenerate $*$-homomorphism, $\f\in\Mor(A,B)$ in other words, then we can associate a canonical $A$-$B$-correspondence $B$ such that the left action is realized with $\f$.
More precisely, $\iota:\cE\to B:a\otimes b\mapsto\f(a)b$ provides a well-defined linear isomorphism (surjectivity follows from the density of $\f(A)B$ in $B$ and the Cohen factorization theorem) and the two actions on $\cE$ is described by $\iota(a\xi b)=\f(a)\iota(\xi)b$.
\end{parts}
\end{prb}


\begin{prb}
Let $\cE$ be a C$^*$-correspondence over a C$^*$-algebra $A$.
Let $B$ be a C$^*$-algebra and see it as a trivial C$^*$-correspondence over $B$.
A \emph{representation} of $\cE$ on $B$ is a pair $(\pi,\tau)$ of a $*$-homomorphism $\pi:A\to B$ and a linear map $\tau:\cE\to B$ such that
\[\pi(\<\xi,\eta\>)=\tau(\xi)^*\tau(\eta),\qquad\tau(\f(a)\xi)=\pi(a)\tau(\xi).\]
We define the \emph{Katsura ideal}
\[J(\cE):=\f^{-1}(K(\cE))\cap\f^{-1}(0)^\perp.\]

A \emph{covariant representation} is a representation of $\cE$ such that
\[\psi(\f(a))=\pi(a),\qquad a\in J(\cE).\]
\begin{parts}
\item
Let $(A,\Z,\alpha)$ be a C$^*$-dynamical system and consider the canonical C$^*$-correspondence $A$ over $A$ with the left action $\f:=\alpha_1\in\Aut(A)\subset\Mor(A)$.
This correspondence is full, faithful, and non-degenerate.
Note that also we have $J(A)=\f^{-1}(A)\cap A=A$.
If $(\pi,\tau)$ is an any representation of this C$^*$-correspondence $A$ on $B$, then 
\end{parts}
\end{prb}

How can we decribe representations of C$^*$-correspondence $A$ with left action $\f\in\Aut(A)$ in terms of covariant representations of the C$^*$-dynamical system $(A,\Z,\alpha)$ with $\alpha_n=\f^n$?



as a morphism
sub and quotient, direct sum, tensor product,

Toeplitz-Cuntz
Toeplitz-Pimsner
Cuntz-Pimsner
Cuntz-Krieger



Subproduct systems


\section{Morita equivalence}



Induced representations?







\chapter{Examples}



\section{Crossed products}


\begin{prb}[Group algebras]
\end{prb}



type I, subhomogeneous


crystallographic
discrete heisenberg
free groups
projectionless of $C_r^*(F_2)$



\begin{prb}[Enveloping C$^*$-algeberas]
Let $A$ be a $*$-algebra.
A \emph{C$^*$-norm} is an submultiplicative norm satisfying the C$^*$-identity.
Does $A$ have enough $*$-representations?
\begin{parts}
\item A complete C$^*$-norm is unique if it exists.
\item For every C$^*$-norm $\alpha$ on $A$, there is a $*$-isometry $\pi:A\to B(H)$.
\item For maximal C$^*$-norm, there is a universal property. The maximal C$^*$-norm can be obtained by running through cyclic representations.
\end{parts}
\end{prb}




\begin{prb}[C$^*$-dynamical system]
Let $G$ be a locally compact group.
A \emph{C$^*$-dynamical system} or a \emph{$G$-C$^*$-algebra} is a C$^*$-algebra $A$ together with a group homomorphism $\alpha:G\to\Aut(A)$ that is continuous in the point-norm topology.
We will often write a triple $(A,G,\alpha)$ instead of $A$ to refer a C$^*$-dynamical system.
\begin{parts}
\item There is an equivalence between categories of locally compact transformation groups and C$^*$-dynamical system on abelian C$^*$-algebras.
\end{parts}
\end{prb}


On $U(H)$, the strict topology and the strong operator topology are equal.
Therefore, we have three topologies to consider: strong, weak, and $\sigma$-weak.

\begin{prb}[Covariant representation]
Let $G$ be a locally compact group.

A \emph{covariant representation} of a C$^*$-dynamical system $(A,G,\alpha)$ is a $G$-equivariant $*$-homomorphism $\pi:(A,G,\alpha)\to(B(H),G,\beta)$ for a C$^*$-dynamical system $(B(H),G,\beta)$, where a Hilbert space $H$.
\begin{parts}
\item
There exists a unitary representation $u:G\to B(H)$ such that $\pi(\alpha_sa)=u_s\pi(a)u_s^*$.
\item (Integrated form)
There is a one-to-one correspondence between covariant representations of $(A,G,\alpha)$ and $*$-representations of $L^1(G,A)$. (non-degenerate)
\end{parts}
\end{prb}

Note that we have a homeomorphism $\Aut(K(H))\cong PU(H)$ between the point-norm topology and the strong operator topology.

$\Z$-action, $\Homeo$-action, left multiplication of subgroup
induced representation
regular representation $(C_0(G),G,\lambda)\to(B(L^2(G)),G,\lambda)$.


commutative case




\section{Graph algebras}






\section{Groupoid algebras}





\section{Free products}






\part{Properties}
\chapter{Approximation properties}
\section{Nuclearity and exactness}

finite dimensional[BO, 3.3.2], abelian, AF
permanence properties


\begin{prb}[Completely positive approximation property]
Let $A$ be a C$^*$-algebra.
\begin{parts}
\item If $A$ has the CPAP, then $A$ is nuclear.
\item If $A$ is nuclear, then $A$ has the CPAP.
\end{parts}
\end{prb}
\begin{pf}

(b)



Let $E\subset A$ and $F\subset A^*$ be finite subsets and fix $\e>0$.
We want to find completely positive contractions $\f:A\to M_n(\C)$ and $\psi:M_n(\C)\to A$ such that
\[|l(a)-l(\psi\circ\f(a))|<\e\]
for $a\in E$ and $l\in F$.
To implement the approximation, we would like to regard a bounded linear operator on $A$ as a state of a tensor product of C$^*$-algebras, which maps $\theta\in B(A)$ to the linear functional characterized by $a\otimes l\mapsto l(\theta(a))$.
However, since $A^*$ is not a C$^*$-algebra, we embed $A^*$ locally in $B(H)$ through the Radon-Nikodym type result.
Let $\pi:A\to B(H)$ be the cyclic representation obtained from a positive linear functional that dominates $F$ and $\Omega$ the cyclic vector such that there is a linear map $\pi':F\to\pi(A)'$ satisfying
\[l(a)=\omega_\Omega(\pi(a)\pi'(l))=\<\pi(a)\pi'(l)\Omega,\Omega\>\]
for $a\in E$ and $l\in F$.
Now the duality of $A$ and $F$ is embodied in the tensor product representation
\[\pi\times i:A\otimes_{\max}\pi(A)'\to B(H)\]
together with a cyclic vector $\Omega\in H$.
Here the nuclearity is used to write $A\otimes_{\max}\pi(A)'=A\otimes_{\min}\pi(A)'$.

If we take any faithful representation $\rho:A\to B(K)$, then we obtain a fathful representation
\[\rho\otimes i:A\otimes_{\min}\pi(A)'\to B(K\otimes H).\]
By the Hahn-Banach separation, the state $(\pi\times i)^*\omega_\Omega$ on $A\otimes_{\min}\pi(A)'$ can be approximated weakly$^*$ by convex combinations of vector states in $B(K\otimes H)$.
In particular, by the density of $\pi(A)\Omega$ in $H$, we have algebraic tensors $(t_k)_{k=1}^m\subset K\odot\pi(A)\Omega$ such that
\[\Bigl|\omega_\Omega((\pi\times i)(a\otimes\pi'(l)))-\sum_{k=1}^m\lambda_k\omega_{t_k}((\rho\otimes i)(a\otimes\pi'(l)))\Bigr|<\e\tag{\dagger}\]
for all $a\in E$ and $l\in F$, where $\lambda_k\ge0$, $\sum_{k=1}^m\lambda=1$.

If we write each element $t\in K\odot\pi(A)\Omega$ as
\[t=\sum_{i=1}^n\eta_i\otimes\pi(b_i)\Omega,\qquad\eta_i\in K,\ b_i\in A,\]
then
\begin{align*}
\omega_t((\rho\otimes i)(a\otimes\pi'(l)))
&=\left\<(\rho(a)\otimes\pi'(l))\Bigl(\sum_{j=1}^n\eta_j\otimes\pi(b_j)\Omega\Bigr),\Bigl(\sum_{i=1}^n\eta_i\otimes\pi(b_i)\Omega\Bigr)\right\>\\
&=\sum_{i,j=1}^n\<\rho(a)\eta_j,\eta_i\>\<\pi'(l)\pi(b_i^*b_j)\Omega,\Omega\>\\
&=l\Bigl(\sum_{i,j=1}^n\<\rho(a)\eta_j,\eta_i\>b_i^*b_j\Bigr).
\end{align*}
If we define completely positive contractions $\f:A\to M_n(\C)$ and $\psi:M_n(\C)\to A$ for each $\tau$ such that
\[\f(a):=[\<\rho(a)\eta_j,\eta_i\>],\quad\psi([e_{ij}]):=b_i^*b_j,\]
then we have $\omega_t(a\otimes\pi'(l))=l(\psi\circ\f(a))$.

Since $\mu(a\otimes\pi'(l))=l(a)$ and since the completely positive contractions which factor through a matrix algebra form a convex set, we have completely positive contractions $\f:A\to M_n(\C)$ and $\psi:M_n(\C)\to A$ such that the inequality (\dagger) is rewritten as
\[|l(a)-l(\psi\circ\f(a))|<\e,\]
so we are done.
\end{pf}





quotients of nuclear
local reflexivity


a separable C$^*$-algebra is nuclear if and only if every factor representation is hyperfinite.

Extension properties
weak expectation property
relatively weakly injective
maximal tensor product inclusion problem



excision: Akemann-Anderson-Pedersen

\section{Quasi-diagonality}

\begin{prb}[Weyl-von Neumann theorem]
A self-adjoint bounded operator is quasi-diagonal.
\end{prb}

\begin{prb}[Glimm lemma]
If a state $\omega$ of $B(H)$ vanishes on $K(H)$, then it is a weak$^*$ limit of vector states.
\end{prb}

\begin{prb}[Voiculescu theorem]
\end{prb}


\begin{prb}[Quasi-diagonal algebras]
An operator $a\in B(H)$ is called \emph{quasi-diagonal} if there is a net of projections $p_i\in B(H)$ such that $[p_i,a]\to0$ in norm and $p_i\uparrow\id_H$ strongly.
A C$^*$-algebra is called \emph{quasi-diagonal} if it admits a faithful representation whose image is quasi-diagonal.
\end{prb}

faithful non-degenerate essential representations of a quasi-diagonal C$^*$-algebra are all quasi-diagonal

locally quasi-diagonal

\section{AF-embeddability}













\chapter{Amenability}


\section{Amenable groups}


\section{Amenable actions}
crossed products
$Z_2$-grading
Connes-Feldman-Weiss
Anantharaman-Delaroche
Gromov boundaries
approximately central structure?
dynamical Kirchberg-Phillips

stably finite
dynamical Elliott program

Ornstein-Weiss-Rokhlin lemma

\section{Exact groups}
Exact groups

\section{Other properties}
Kazdahn property (T)
factorization property
Haagerrup property


Kaplansky conjecture












\chapter{Simplicity}


Furstenburg boundary



















\part{Invariants}
\chapter{Operator K-theory}


\section{Homotopy of C$^*$-algebras}

\begin{prb}[Homotopy of $*$-homomorphisms]
Let $A,B$ be C$^*$-algebras.
Two $*$-homomorphisms in $\Mor(A,B)$ are said to be \emph{homotopic} if they are connected by a path in $\Mor(A,B)$ that is continuous with the point-norm topology.
\begin{parts}
\item For pointed compact Hausdorff spaces $(X,x_0),(Y,y_0)$, two pointed maps $\f_0,\f_1:X\to Y$ are homotopic if and only if $\f_0^*,\f_1^*:C_0(Y\setminus\{y_0\})\to C_0(X\setminus\{x_0\})$ are homotopic.
\end{parts}
\end{prb}
\begin{pf}
(a)
Suppose $\f_0$ and $\f_1$ are connected by a homotopy $\f_t$.
Fixing $g\in C_0(Y)$ and $t_0\in I$, we want to show
\[\lim_{t\to t_0}\sup_{x\in X}|g(\f_t(x))-g(\f_{t_0}(x))|=0.\]
Since the function $g$ is uniformly continuous, with respect to an arbitrarily chosen uniformity on $Y$, so that there is an entourage $E\subset Y\times Y$ such that $(y,y')\in E\circ E$ implies $|g(y)-g(y')|<\e$.
Using compactness we have a finite sequence $(y_i)_{i=1}^n\subset Y$ such that for every $y$ there is $y_i$ satisfying $(y,y')\in E$.
Then, $\f^{-1}(E[y_i])$ is a finite open cover of $X\times I$, so we have $\delta$ such that $|t-t_0|<\delta$ implies for any $x\in X$ the existence of $i$ satisfying $(\f_t(x),y_i)\in E$ and $(\f_{t_0}(x),y_i)\in E$, which deduces the desired inequality.

Conversely, suppose $\f_0^*$ and $\f_1^*$ are connected by a homotopy $\f_t^*$.
By taking dual, we can induce $\f_t:X\to Y$ such that $g(\f_t(x))=(\f_t^*g)(x)$ for each $g\in C(Y)$ from $\f_t^*$ via the embedding $X\to M(X)$ by Dirac measures.
Let $V$ be an open neighborhood of $\f_{t_0}(x_0)$ and take $g\in C(Y)$ such that $g(\f_{t_0}(x_0))=1$ and $g(y)=0$ for $y\notin V$.
Now we have an open neighborhood $U$ of $x_0$ such that $x\in U$ implies $|(\f_{t_0}^*g)(x)-(\f_{t_0}^*g)(x_0)|<\frac12$.
Also we have $\delta>0$ such that $|t-t_0|<\delta$ implies $\|\f_t^*g-\f_{t_0}^*g\|<\frac12$.
Therefore, $(x,t)\in U\times(t_0-\delta,t_0+\delta)$ implies $g(\f_t(x))>0$, hence $\f_t(x)\in V$, which means $X\times I\to Y:(x,t)\mapsto\f_t(x)$ is continuous.
\end{pf}

We have $\tilde K^n(X,x_0)=K_n(C_0(X\setminus\{x_0\}))$ for a pointed compact Hausdorff space $X$.
Now then since the inclusion $\{x_0\}\to X$ induces the section so that
\[0\to K_0(C_0(X\setminus\{x_0\}))\to K_0(C(X))\to K_0(\{x_0\})\to0\]
splits, we have
\[K^0(X)=\tilde K^0(X,x_0)\oplus\Z=K_0(C_0(X\setminus\{x_0\}))\oplus K_0(\{x_0\})=K_0(C(X))\]
for a compact connected Hausdorff space $X$.
The additivity of $K_0$ and $K^0$ removes the connectedness condition.

\[K_0(\C)=\Z,\quad K_0(C_0(\R))=0,\quad K_1(C_0(\R))=K_0(C_0(\R^2))=\Z\]
\[K^0(*)=\Z,\quad K^0(S^1)=\Z,\quad K^1(S^1)=K^0(S^2)=\Z[x]/(x-1)^2\]

\section{$K_0$ groups}

local Banach algebras

homotopy invariance
relative, reduced theory
partially ordered abelian group


\section{$K_1$ groups}

unitary
Bott periodicity
six-term exact sequence


\section{Equivariant K-theory}
\begin{prb}[Pimsner-Voiculescu exact sequence]
\end{prb}
Connes' Thom isomorphism




\chapter{KK-theory}

\section{Cuntz pairs}

\section{Kasparov modules}






\chapter{Cuntz semigroup}

nuclear dimension










\part{Classification}
\chapter{Simple nuclear algebras}


\section{AF-algebras}

Glimm's classification of UHF algebras
Bratteli diagram
Elliott's intertwining argument

Separable AF-algebras are classified by pointed ordered $K_0$.


\section{Kirchberg-Phillips theorem}

\section{Classifiability}
Jiang-Su stability
Universal coefficient theorem

Toms-Winter conjecture
strongly self-absorbing
nuclear dimension




successful in Kirchberg algebras


https://arxiv.org/pdf/2307.06480.pdf

Elliott classification problem
Kirchberg-Phillipes theorem

operator K-theory and its pairing with traces

$\cZ$-stability, Rosenberg-Schochet universal coefficient theorem

Connes-Haagerup classification of injective factors

Kirchberg: unital simple separable $\cZ$-stable algebra is either purely infinte or stably finite.
Haagerup, Blackadar, Handelman: unital simple stably finite algebra has a trace.

Glimm: uniformly hyperfinite algebras
Murray-von Neumann: hyperfinite II$_1$ factors




\section{Inclusions}










\chapter{Continuous fields}


\section{Banach bundles}

\begin{prb}[Banach bundles]
A \emph{Banach bundle}, introduced by Fell, is a continuous open surjection $\pi:E\to X$ between topological spaces together with Banach space structure on each fiber $\pi^{-1}(x)$ such that:
\begin{enumerate}[(i)]
\item the addition $\{(e,e'):\pi(e)=\pi(e')\}\subset E\times E\to E:(e,e')\mapsto e+e'$ is continuous,
\item the scalar multiplication $\C\times E\to E:(\lambda,e)\mapsto\lambda e$ is continuous,
\item the norm $E\to\R_{\ge0}:e\mapsto\|e\|$ is continuous,
\item the family of subsets
\[\{e\in B:\pi(e)\in U,\ \|e\|<r\}_{U\in N(x),r>0}\]
forms a neighborhood basis of $0\in\pi^{-1}(x)$ in $E$.
\end{enumerate}
The forth condition is equivalent to that if $\|e_i\|\to0$ and $\pi(e_i)\to x$ then $e_i\to 0_x\in\pi^{-1}(x)$.
\begin{parts}
\item For a Banach bundle $E\to X$, if $X$ is locally compact Hausdorff and every fiber $E_x$ shares a same finite dimension, then the bundle is locally trivial.
\end{parts}
\end{prb}


\begin{prb}[Continuous fields of Banach spaces]
\end{prb}



\begin{prb}[Hilbert bundles]
A \emph{Hilbert bundle} is a Banach bundle whose norm function satisfies the parallelogram law.

\begin{parts}
\item On a compact $X$, there is an equivalence between the category of Hilbert $C(X)$-modules and the category of Hilbert bundles over $X$.
\item On a compact $X$, there is an equivalence between the category of algebraically finitely generated Hilbert $C(X)$-modules and the category of classical locally trivial finite-rank complex vector bundle over $X$.
It is due to that finitely generatedness implies the projectivity and the Serre-Swan theorem.
\end{parts}
\end{prb}


\section{Primitive ideal space}

For a short exact sequence
\[\begin{tikzcd}[column sep=small]0\rar&I\rar&A\rar&B\rar&0\end{tikzcd},\]
we have
\[\begin{tikzcd}[row sep=small]
PS(I) \dar[->>]\rar[hook] & PS(A) \dar[->>] & PS(B) \dar[->>]\lar[hook'] \\
\hat I \dar[->>]\rar[hook,shorten >= 10, shorten <= 10] & \hat A \dar[->>] & \hat B \dar[->>]\lar[hook',shorten >= 10, shorten <= 10] \\
\Prim(I) \rar[hook,swap]{\text{open}} & \Prim(A) & \Prim(B) \lar[hook']{\text{closed}}
\end{tikzcd}\]

We have to understand C$^*$-algebras in the context of homotopy theory, so the pointed topological spaces must be considered.
An open set $U$ of a locally compact Hausdorff space $X$ should be recognized as the quotient space $(X,x)/(A,x)$, where $x\notin U=A^c$, hence the ideal $A(U)$ corresponds and the restriction $A(X)\to A(U)$ does not make sense.
In other words, \textbf{$A(U)$ is not an analogue of $\cO_X(U)$, but of $\cI_{(X,X\setminus U)}$}.
It is fortunate that the kernel of the restriction, an ideal, can be recognized as the function algebra of the complement, which is not the case in algebraic geometry...? (Can define the quotient $X/A$ for an analytic subset of a complex space $X$?)

Then, how can we understand the sheaf theoretic restriction on an open set in operator algebras?
How about Banach or Fr\'echet algebras?
Can we consider a ``rigid'' Zariksi topology on the spectrum? (Closed sets in C$^*$-context are too flaccid)




\begin{prb}[Modular maximal left ideals]
\end{prb}

\begin{prb}[Primitive ideals]
hull kernel topology
\[PS(A)\cong\{(\pi,\psi)\}/\sim_u,\qquad\hat A\cong\{\pi\}/\sim_u.\]

\[\begin{array}{c|ccc}
A & PS(A) & \hat A & \Prim(A) \\\hline
C(X) & X & X & X \\
K(H) & PH & * & * \\
\tilde K(H) & ? & ? & \{0,K(H)\} \\
B(H) &&&
\end{array}\]
\begin{parts}
\item $\Prim(A)$ is locally compact T$_0$ space.
\item Two maps $PS(A)\to\hat A\to\Prim(A)$ are continuous surjective open maps
\item If $A$ is type I, then $\hat A\to\Prim(A)$ is an homeomorphism.
\end{parts}

\end{prb}

\begin{prb}[Dauns-Hoffman theorem]
\end{prb}


\section{Dixmier-Douady theory}


Fell's condition

A C$^*$-algebra $A$ is called \emph{continuous trace} if the set of all $a\in\cA$ such that $\hat A\to\R_{\ge0}:\pi\mapsto\tr(\pi(a^*a))$ is continuous is dense in $A$.



Dadarlat-Pennig theory


Coactions and Fell bundles




\end{document}