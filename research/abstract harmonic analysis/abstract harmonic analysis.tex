\documentclass{../../large}
\usepackage{../../ikhanchoi}


\begin{document}
\title{Abstract Harmonic Analysis}
\author{Ikhan Choi}
\maketitle
\tableofcontents



\part{}

\chapter{Locally compact groups}
\section{Haar measures}

\begin{prb}[Non-$\sigma$-finite measures]
Following technical issues are important
\begin{parts}
\item The Fubini theorem
\item The Radon-Nikodym theorem
\item The dual space of $L^1$ space
\end{parts}
\end{prb}

\begin{prb}[Existence of the Haar measure]
\end{prb}

\begin{prb}[Left and right uniformities]
\end{prb}

\begin{prb}[Modular functions]
\end{prb}

\begin{prb}[Uniformly continuous functions]
$G$ acts on $C_{lu}(G)$ and $L^1(G)$ continuously with respect to the point-norm topology.
A function on $G$ is left uniformly continuous if and only if it is written as $f*x$ for some $f\in L^1(G)$ and $x\in L^\infty(G)$.
$g\in C_c(G)$ is two-sided uniformly continuous.
\end{prb}

\section{Group algebras}

\begin{prb}[Convolution inequalities]
\end{prb}

Justification of the following?
\[\lambda(f)=\int f(s)\lambda_s\,ds.\]

\begin{prb}[Convolution action]
Let $G$ be a locally compact group.
\begin{parts}
\item $L^1(G)$ has a two-sided approximate unit.
\item $\alpha:G\to\Aut(L^1(G))$ is point-norm continuous.
\item $\lambda:G\to U(L^2(G))$ and $\lambda:L^1(G)\to B(L^2(G))$ are strongly continuous.
\end{parts}
\begin{pf}
Let $(U_\alpha)$ be a directed set of open neighborhoods of the identity $e$ of $G$.
By Urysohn lemma, there is $e_\alpha\in C_c(U)^+$ such that $\|e_\alpha\|_1=1$ for each $\alpha$.
We claim that $e_\alpha$ is a left approximate unit for $L^1(G)$.

Suppose $g\in C_c(G)$, which is two-sided uniformly continuous.
For any $\e>0$, take $\alpha_0$ such that $\|g-\lambda_sg\|<\e$ and $\|g-\rho_sg\|<\e$ for all $s\in U_\alpha$ for $\alpha\succ\alpha_0$.
Then, we have
\begin{align*}
\|e_\alpha*g-g\|_1
&=\int|e_\alpha*g(t)-g(t)|\,dt\le\iint e_\alpha(s)|g(s^{-1}t)-g(t)|\,ds\,dt\\
&=\int e_\alpha(s)\|\lambda_sg-g\|_1\,ds<\e\int e_\alpha(s)\,ds\le\e,
\end{align*}
and
\begin{align*}
\|g*e_\alpha-g\|_1
&=\int|g*e_\alpha(s)-g(s)|\,ds\le\iint|g(t)-g(s)|e_\alpha(t^{-1}s)\,dt\,ds\\
&=\iint|g(t)-g(ts)|e_\alpha(s)\,dt\,ds=\int\|g-\rho_sg\|_1e_\alpha(s)\,ds<\e\int e_\alpha(s)\,ds\le\e,
\end{align*}
and they imply $\lim_\alpha\|e_\alpha*g-g\|_1=\lim_\alpha\|g*e_\alpha-g\|_1=0$.
For general $f\in L^1(G)$, by taking $g\in C_c(G)$ such that $\|f-g\|_1<\e$, we have
\[\|e_\alpha*f-f\|_1\le\|e_\alpha*(f-g)\|_1\]
\end{pf}

\end{prb}

Note that we have
\begin{align*}
|\<\lambda(\xi)\eta,\zeta\>|^2
&=|\iint\xi(t)\eta(t^{-1}s)\bar{\zeta(s)}\,ds\,dt|^2\\
&\le\iint|\xi(t)||\eta(t^{-1}s)|^2\,ds\,dt\cdot\iint|\xi(t)||\zeta(s)|^2\,ds\,dt\\
&=\|\xi\|_1^2\|\eta\|_2^2\|\zeta\|_2^2
\end{align*}
and
\begin{align*}
|\<\rho(\xi)\eta,\zeta\>|^2
&=|\iint\eta(t)\xi(t^{-1}s)\bar{\zeta(s)}\,ds\,dt|^2\\
&\le\iint|\xi(t^{-1}s)||\eta(t)|^2\,ds\,dt\cdot\iint|\xi(t^{-1}s)||\zeta(s)|^2\,ds\,dt\\
&=\|\xi\|_1\|F\xi\|_1\|\eta\|_2^2\|\zeta\|_2^2
\end{align*}
imply
\[\|\lambda(\xi)\|_{2\to2}\le\|\xi\|_1,\qquad\|\rho(\xi)\|_{2\to2}\le\sqrt{\|\xi\|_1\|F\xi\|_1}.\]
The equalities do not hold, consider $\|\lambda(\xi)\|=\|\hat\xi\|_\infty$ if $G=\R$.

\begin{prb}[Group algebras]
\end{prb}
\begin{prb}[Fell absorption principle]
\end{prb}

\begin{prb}[Fourier algebra]
\end{prb}
\begin{prb}[Fourier-Stieltjes algebra]
positive definite functions, Bochner theorem
\end{prb}
\begin{prb}[GNS construction for locally compact groups]
Let $G$ be a locally compact group.
By a state of $C^*(G)$, we could construct the GNS representation of $G$.
An analog of GNS construction for $L^1(G)$ without completion is doable, when given a function of positive type on $G$, instead of a state.
\end{prb}


\[\begin{tikzcd}
G \ar{r} & M(G) & \, & \,\\
L_1(G) \ar[hook]{ur}\ar[hook]{r}\ar[dashed]{d}{*} & C^*(G) \ar[two heads]{r}\ar[dashed]{d}{*} & C_r^*(G) \ar[hook]{r}\ar[dashed]{d}{*} & L(G) \ar[dashed]{d}{*\text{ with }\sigma w}\\
L^\infty(G) & B(G) \ar[hook]{l} & C_r^*(G)^* \ar[hook]{l} & A(G) \ar[hook]{l}\\
& C_0(G) \ar[hook]{ul} & &
\end{tikzcd}\]





\section{Pontryagin duality}

\begin{prb}[Dual group]
\end{prb}
\begin{prb}[Fourier inversion theorem]
\end{prb}
\begin{prb}[Plancherel's theorem]
\end{prb}


\section{Structure theorems}



\section{Spectral synthesis}








\chapter{Representation theory}

\begin{prb}[Schur's lemma]
\end{prb}

\begin{prb}[Operator-valued Fourier transform]
\end{prb}



Since it is not easy to introduce the quantum dual of $G$ for now, we cannot discuss $L^1(G)$ as the Fourier algebra, the predual of the quantum group von Neumann algebra.
($A(G)=L(G)_*=L^1(\hat G)$ and also is the closed linear span of matrix coefficients of the left regular representation.)



\chapter{Compact groups}
\section{Peter-Weyl theorem}
\section{Tannaka-Krein duality}
\section{Example of compact Lie groups}

\chapter{Mackey machine}
\section{Example of non-compact Lie groups}
Wigner classification








\chapter{Kac algebras}




\part{Topological quantum groups}
\chapter{Compact quantum groups}
\chapter{Locally compact quantum groups}
\section{Multiplicative unitaries}


\part{Tensor categories}


\end{document}







