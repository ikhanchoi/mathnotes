\documentclass{../../large}
\usepackage{../../ikhanchoi}


\begin{document}
\title{Harmonic Analysis}
\author{Ikhan Choi}
\maketitle
\tableofcontents

\part{Fourier analysis}

\chapter{Fourier series}

\section{Fourier series in $L^p$ spaces}
\begin{prb}
\[\|\hat f\|_{\ell^1(\Z)}\lesssim\|f\|_{W^{1,1+\e}(\T)}.\]
\end{prb}

Inversion theorem is an approximation problem given by $\cF^*\cF=\lim_{n\to\infty}\cF_n^*\cF$.
The condition $\hat f\in\ell^1(\Z)$ is a condition just for defining $\cF^*\hat f$ wihtout using distribution theory, and it does not affect the inversion phenomena.
The approximation, in other words, can be seen as an extension method for $\cF^*:\ell^1(\Z)\to C(\T)$ on $c_0(\Z)$.
Note that $\cF_n^*$ on $c_0(\Z)$ cannot be bounded directly without distribution theory, but $\cF_n^*\cF$ on $L^p(\T)$ can be bounded well.

\section{Summability methods}
\begin{itemize}
\item If $\cF_n^*$ is the standard partial sum, then $\cF_n^*\cF$ is the Dirichlet kernel.
\item If $\cF_n^*$ is the Ces\`aro mean, then $\cF_n^*\cF$ is the Fej\'er kernel.
\item If $\cF_r^*$ is the Abel sum, then $\cF_r^*\cF$ is the Poisson kernel.
\item In Fourier transform, we often use the Gauss-Weierstrass kernel.
\end{itemize}

The injectivity of $\cF$ is not an easy problem, which comes from the inversion theorem.



\begin{prb}[Dirichlet kernel]
The \emph{Dirichlet kernel} is a function $D_n\colon \bT\to\R$ defined by
\[D_n=\hat{\1_{|k|\le n}},\quad\text{or equivalently,}\quad\hat{D_n}=\1_{|k|\le n}.\]
This is because they are invariant under inverse, in other words, they are even.
\begin{parts}
\item
\[D_n(x)=\frac{\sin\frac{2n+1}2x}{\sin\frac12x}.\]
\item If $f\in\Lip(\bT)$, then $D_n*f\to f$ pointwisely as $n\to\infty$.
\item
\[\|D_n\|_{L^1(\bT)}\gtrsim\log n.\]
\end{parts}
\end{prb}

\begin{pf}
\begin{align*}
D_n(x)&=\sum_{k=-n}^ne^{ikx}\\
&=\frac{e^{i\frac{2n+1}2x}-e^{-i\frac{2n+1}2x}}{e^{i\frac12x}-e^{-i\frac12x}}\\
&=\frac{\sin\frac{2n+1}2x}{\sin\frac12x}.
\end{align*}

(c)
By (2) $\sin x\le x$ for $x\in[0,\pi/2]$, (3) change of variable,
\begin{align*}
\|D_n\|_{L^1(\bT)}
&=\frac1{2\pi}\int_{-\pi}^\pi|\frac{\sin\frac{2n+1}2x}{\sin\frac12x}|\,dx\\
&\ge\frac2\pi\int_0^\pi\frac{|\sin\frac{2n+1}2x|}x\,dx\\
&=\frac2\pi\int_0^{\frac{2n+1}2\pi}\frac{|\sin x|}x\,dx\\
&=\frac2\pi\sum_{k=0}^{2n}\int_{\frac k2\pi}^{\frac{k+1}2\pi}\frac{|\sin x|}x\,dx\\
&\ge\frac2\pi\sum_{k=0}^{2n}\int_0^{\frac12\pi}\frac{\sin x}{\frac{k+1}2\pi}\,dx\\
&\ge\frac4{\pi^2}\sum_{k=0}^{2n}\frac1{1+k}\\
&\ge\frac4{\pi^2}\log(2n+2).
\end{align*}
..?
\end{pf}

\begin{prb}[Fej\'er kernel]
The \emph{Fej\'er kernel} is

\begin{parts}
\item
\[K_n(x)=\frac1{n+1}\frac{\sin^2\frac{n+1}2x}{\sin^2\frac12x}.\]
\end{parts}
\end{prb}
\begin{pf}
Since
\begin{align*}
D_n(x)=
&=\frac{e^{i\frac{2n+1}2x}-e^{-i\frac{2n+1}2x}}{e^{i\frac12x}-e^{-i\frac12x}}\\
&=\frac{[e^{i\frac{2n+1}2x}-e^{-i\frac{2n+1}2x}][e^{i\frac12x}-e^{-i\frac12x}]}{[e^{i\frac12x}-e^{-i\frac12x}]^2}\\
&=\frac{[e^{i(n+1)x}+e^{-i(n+1)x}]-[e^{inx}+e^{-inx}]}{[e^{i\frac12x}-e^{-i\frac12x}]^2},
\end{align*}
by telescoping, we get
\begin{align*}
\sum_{k=0}^nD_k(x)
&=\frac{[e^{i(n+1)x}+e^{-i(n+1)x}]-[e^{i0x}+e^{-i0x}]}{[e^{i\frac12x}-e^{-i\frac12x}]^2}\\
&=\frac{[e^{i\frac{n+1}2x}-e^{-i\frac{n+1}2x}]^2}{[e^{i\frac12x}-e^{-i\frac12x}]^2}\\
&=\frac{\sin^2\frac{n+1}2x}{\sin^2\frac12x}.
\end{align*}
\end{pf}

Two important results from Fej\'er kernel:
\begin{enumerate}
\item If $f(x-)$, $f(x+)$ exist and $S_nf(x)$ converges, then $S_nf(x)\to\frac12(f(x-)+f(x+))$.
\item (If $f\in L^1(\bT)$, then $\sigma_nf\to f$ a.e.)

\item If $f\in L^1(\bT)$, then $S_nf\to f$ in $L^1$ and $L^2$.
\item If $f$ is continuous and $\hat{f}\in L^1(\Z)$, then $S_nf\to f$ uniformly.
\item Since $\sigma_nf$ is a trigonometric polynomial, the set of trigonometric polynomials are dense in $L^1(\bT)$ and $L^2(\bT)$.
\end{enumerate}



\section{Pointwise convergence of Fourier series}

BV function: Dini, Jordan's criterion
\begin{prb}[Riemann localization principle]
\end{prb}





\section*{Exercises}
\begin{prb}[Gibbs phenomenon]
\end{prb}
\begin{prb}[Du Bois-Reymond function]
\end{prb}









\chapter{Fourier transform}
\section{Fourier transform in $L^p$ space}
\begin{prb}[Riemann-Lebesgue lemma]
\end{prb}
Lp extension

Gaussian function computation: differential equation method, contour integral method
inversion theorem
\begin{prb}[Plancherel theorem]
\end{prb}

\section{Distributions}
\begin{prb}[Cauchy principal value]
indented contour, imaginary shift, Feynman's trick
\end{prb}







\section*{Exercises}
\begin{prb}[Sampling theorem]
\[\cF\1_{[-\frac12,\frac12]}(\xi)=\operatorname{sinc}(\xi/2)\]
$\operatorname{sinc}\in L^{1+\e}(\R)$.
\end{prb}
\begin{prb}[Poisson summation formula]
\end{prb}
\begin{prb}[Uncertainty principle]
\end{prb}

\begin{prb}[Multipole expansion]
Let $\rho$ be a compactly supported distribution on $\R^d$.
We want to investigate the limit behavior of $\rho(\e^{-1}x)$ as $\e\to0$.
More precisely, we want to compute an integer $k\ge d$ such that $\lim_{\e\to0+}\e^{-k}\rho(\e^{-1}x)$ defines a distribution supported at $\{0\}$, and the coefficients of derivatives of Dirac measures.

We need to introduce quantities called monopole, dipole, quadrapole, octupole, etc.
\begin{parts}
\item A distribution supported on $\{0\}$ is a linear combination of the Dirac measure and its derivatives.
\item 
\end{parts}
\end{prb}

\section*{Problems}
\begin{enumerate}
\item Find all $\alpha>0$ such that
\[\lim_{x\to\infty}x^{-\alpha}\int_0^xf(y)\,dy=0\]
for all $f\in L^3([0,\infty))$.
\end{enumerate}








\chapter{Hilbert transform}

\section{Harmonic conjugate}
\section{Kernel representation}
\section{Fourier series in $L^p$ space}









\part{Singular integral operators}
\chapter{Calder\'on-Zygmund theory}


\section{Convolution type operators}
\begin{prb}[Calder\'on-Zygmund decomposition of sets]
Let $f\in L^1(\R^d)$.
Let $E_nf$ be the conditional expectation with repect to the $\sigma$-algebra generated by dyadic cubes with side length $2^{-n}$.
Let $Mf:=\sup_nE_n|f|$ be the maximal function, and let $\Omega:=\{x:Mf(x)>\lambda\}$ for fixed $\lambda>0$.
For $x\in\Omega$ let $Q_x$ be the maximal dyadic cube such that $x\in Q_x$ and
\[\frac1{|Q_x|}\int_{Q_x}|f|>\lambda.\]
\begin{parts}
\item
$\{Q_x:x\in\Omega\}$ is a countable partition of $\Omega$.
\item
We have an weak type estimate $|\Omega|\le\frac1\lambda\|f\|_{L^1}$.
\item
$\|f\|_{L^\infty(\R^d\setminus\Omega)}\le\lambda$.
\item
For $x\in\Omega$
\[\frac1{|Q_x|}\int_{Q_x}|f|\le2^d\lambda.\]
\end{parts}
\end{prb}

\begin{prb}[Calder\'on-Zygmund decomposition of functions]
Let
\[g(x):=\begin{cases}|f(x)|&,x\notin\Omega\\\frac1{|Q_x|}\int_{Q_x}|f|&,x\in\Omega\end{cases}\]
and $b_i:=(|f|-g)\chi_{Q_i}$ so that $|f|=g+b$ where $b=\sum_ib_i$.
\begin{parts}
\item $\|g\|_{L^1}=\|f\|_{L^1}$ and $\|g\|_{L^\infty}\lesssim_d\lambda$.
\item $\|b\|_{L^1}\le2\|f\|_{L^1}$ and $\int b_i=0$.
\end{parts}
\end{prb}
\begin{pf}

\end{pf}




\begin{prb}[$L^p$ boundedness of Calder\'on-Zygmund operators]
Let $T:C_c^\infty(\R^d)\to\cD'(\R^d)$ be a \emph{singular integral operator of convolution type} in the sense that there is a function $K\in L_\loc^1(\R^d\setminus\{0\})\cap\cD'(\R^d)$ such that $Tf(x)=K*f(x)$ for all $f\in\cD(\R^d)$, whenever $x\notin\supp f$.
We say $T$ is called a \emph{Calder\'on-Zygmund} operator if
\begin{enumerate}[(i)]
\item $T$ is $L^2$-bounded: we have
\[\|Tf\|_{L^2}\lesssim\|f\|_{L^2},\]
\item $T$ satisfies the \emph{H\"ormander condition}: we have
\[\int_{|x|>2|y|}|K(x-y)-K(x)|\,dx\lesssim1\]
for every $y>0$.
\end{enumerate}

Let $f=g+b=g+\sum_ib_i$ be the Calder\'on-Zygmund decomposition, and let $\Omega^*:=\bigcup_iQ_i^*$ where $Q_i^*$ is the cube with the same center as $Q_i$ and whose sides are $2\sqrt d$ times longer. 
\begin{parts}
\item
The $L^2$-boundedness implies
\[|\{x:|Tg(x)|>\tfrac\lambda2\}|\lesssim_d\frac1\lambda\|f\|_{L^1}.\]
\item
The H\"ormander condition implies
\[|\{x:|Tb(x)|>\tfrac\lambda2\}\setminus\Omega^*|\lesssim_d\frac1\lambda\|f\|_{L^1}.\]
\item
\end{parts}
\end{prb}
\begin{pf}
(a)
Using the Chebyshev inequality and the H\"older inequality,
\[|\{x:|Tg(x)|>\frac\lambda2\}|
\le\frac4{\lambda^2}\|Tg\|_{L^2(\Omega)}^2
\le\frac{4C}{\lambda^2}\|g\|_{L^2(\Omega)}^2
\le\frac{4C}{\lambda^2}\|g\|_{L^1(\Omega)}\|g\|_{L^\infty(\Omega)}.
\]

(b)
Write
\[|\{x:|Tb(x)|>\tfrac\lambda2\}\setminus\Omega^*|
\le\frac2\lambda\int_{\R^d\setminus\Omega^*}|Tb(x)|\,dx
\le\frac2\lambda\sum_i\int_{\R^d\setminus Q_i^*}|Tb_i(x)|\,dx.\]
Since $x\in\R^d\setminus Q_i^*$ does not belong to $\supp b_i\subset Q_i$ and $\int b_i=0$, we have
\[Tb_i(x)=\int_{Q_i}K(x-y)b_i(y)\,dy=\int_{Q_i}[K(x-y)-K(x)]b_i(y)\,dy,\]
and
\[\int_{\R^d\setminus Q_i^*}|Tb_i(x)|\,dx
=\int_{Q_i}|b_i(y)|\int_{\R^d\setminus Q_i^*}|K(x-y)-K(x)|\,dx\,dy
\lesssim\|b_i\|_{L^1}.\]
(We need to show it is valid even though $b_i$ is not smooth)

(c)

\end{pf}

\begin{prb}[H\"older boundedness of Calder\'on-Zygmund operators]
\end{prb}


\section{Truncated integrals}

Homogeneous kernels

\section{$A_p$ weights}

\section{Bounded mean oscillation}


\section*{Exercises}
\begin{prb}[Size and cancellation condition]
Let $K\in L_\loc^1(\R^d\setminus\{0\})\cap\cD'(\R^d)$.
We say the condition $|K(x)|\lesssim|x|^{-d}$ for $x\ne0$ as the \emph{size condition}, and say the condition $\int_{r<|x|<R}K(x)\,dx=0$ for all $0<r<R<\infty$ as the \emph{cancellation condition}.
If $K$ satisfies the size, cancellation, and H\"ormander condition, then it is $L^2$ bounded, hence Calder\'on-Zygmund.
\end{prb}

\begin{prb}[Gradient size condition]
Let $|\nabla K(x)|\lesssim|x|^{-d-1}$ for $x\ne0$.
Then, convolution with $K$ is a Calder\'on-Zygmund operator.
\end{prb}

\begin{prb}[Riesz potential]
\end{prb}



\chapter{Littlewood-Paley theory}
\section{Littlewood-Paley decomposition}
\section{Multiplier theorems}



\chapter{Almost orthogonality}
Carleson measures, paraproducts
\section{Coltar lemma}
\section{$T(1)$ theorem}

% Fourier series
% randomness
% inner and outer functions?



\part{Oscillatory integral operators}

% oscillatory integrals - Sogge
% fourier restriction
% kakeya - Wolff
% bochner-riesz
% geometric measure
% Strichartz
% dispersive equations - Chicago and Prinston school

\chapter{Oscillatory integrals}


\begin{prb}[Justification of quadratic oscillatory integral]

\end{prb}

\begin{prb}[Stationary phase approximation]
\end{prb}
Van der Corput lemma
Dispersive equations and strichartz estimates

\chapter{Foureir restriction}
Kakeya
Bochner-Riesz
Geometric measure theory

\chapter{}









\part{Pseudo-differential operators}
% https://arxiv.org/abs/2107.12839
% https://mtaylor.web.unc.edu/notes/pseudodifferential-operators-fourier-integral-operators-and-microlocal-analysis/

% wave front sets
% wavelets
% uncertainty principle
% weyl calculus - Folland

\chapter{Pseudo-differential calculus}
\section{}

\begin{prb}[Symbol classes]
Japanese bracket $\<x\>:=(1+x^2)^{\frac12}$.
\[\<x-y\>^{-2}\<D_\xi\>^{-2}e^{i(x-y)\xi}=e^{i(x-y)\xi}\]
\end{prb}


\begin{prb}[Asymptotic expansion]
\end{prb}



\begin{prb}[Quantization]
$t$-quantization of a symbol $a$ is the $\Psi$DO defined by
\[a^t(x,D)f(x):=(2\pi)^{-d}\iint e^{i(x-y)\cdot\xi}a((1-t)x+ty),\xi)f(y)\,dy\,d\xi.\]
Kohn-Nirenberg calculus for $t=0$, Weyl calculus for $t=\frac12$.

Let $a\in S_{\rho,\delta}^m(\R^{2d})$ with $m\in\R$, $0\le\delta\le\rho\le1$ and $\delta\ne1$.
Let $t,s\in[0,1]$ with $t\ne s$.
\begin{parts}
\item There exists a unique $b\in S_{\rho,\delta}(\R^{2d})$ such that $a^t(x,D)=b^s(x,D)$.
\item $b$ is expressed as
\[b(x,\xi)=e^{i(t-s)D_xD_\xi}a(x,\xi)=(2\pi)^{-d}|t-s|^{-d}\int_{\R^{2d}}e^{-iy\eta/(t-s)}a(x+y,\xi+\eta)\,dy\,d\eta,\]
\item If $\delta<\rho$, then
\[b\sim\sum_{\alpha\in\Z_{\ge0}^d}\frac{(t-s)^{|\alpha|}}{i^{|\alpha|}\alpha!}\partial_x^\alpha\partial_\xi^\alpha a.\]
\end{parts}
\end{prb}


\begin{prb}[Formal adjoint]
extension to tempered distributions
\end{prb}



\begin{prb}[Moyal product]
Let $a\in S_{\rho,\delta}^m(\R^{2d})$ and $b\in S_{\rho,\delta}^l(\R^{2d})$.
\begin{parts}
\item there exists a unique function $a\#^tb\in S_{\rho,\delta}^{m+l}(\R^{2d})$ such that
\[a^t(x,D)b^t(x,D)=(a\#^t)^t(x,D).\]
\item It is concretely described by
\[(a\#^tb)(x,\xi)=(2\pi)^{-2}\int_{\R^{4d}}e^{-i(y\eta-z\zeta)}a(x+tz,\xi+\eta)b((1-t)y+x,\xi+\zeta)\,dy\,d\eta\,dz\,d\zeta.\]
\item If $\delta<\rho$, then
\[a\#^tb(x,\xi)\sim\sum_{k\in\Z_{\ge0}}\frac1{i^kk!}(\partial_y\partial_\eta-\partial_z\partial_\zeta)^ka((1-t)x+tz,\eta)b(tx+(1-t)y,\zeta)\Bigr|_{\substack{y=z=x\\\eta=\zeta=\xi}}.\]
\end{parts}
\end{prb}



\begin{prb}[Parametirx and elliptic operators]
\end{prb}



\section{}


\begin{prb}[Calder\'on-Vaillancourt theorem]
\end{prb}

\chapter{Semiclassical analysis}

For parameters $0\le\lambda\le1$ and $h>0$, let
\[\hat a\psi(x):=\frac1{(2\pi h)^d}\iint e^{\frac ih\<x-y,\xi\>}a((1-\lambda)x+\lambda y,\xi)\psi(y)\,dy\,d\xi.\]
For example, regardless of $h$ and $\lambda$,
\[\hat\xi\psi(x)=\frac hi\psi'(x)\]
and
\[\hat H\psi(x)=-h^2\Delta\psi(x)+V(x)\psi(x),\]
where $V:\R_x^d\times\R_\xi^d\to\R$ and $H:\R_x^d\times\R_\xi^d\to\R$ such that
\[H(x,\xi):=|\xi|^2+V(x).\]


\[\dd{t}a(t)=\{a(t),H\}=X_Ha(t)\]
\[\dd{t}\hat a(t)=\dd{t}e^{\frac iht\hat H}\hat ae^{-\frac iht\hat H}=-\frac ih[\hat a(t),\hat H]\]

\section{Heisenberg group}
\section{Phase space transforms}

\chapter{Microlocal analysis}




\end{document}