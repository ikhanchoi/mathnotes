\documentclass{../../large}
\usepackage{../../ikhanchoi}


\DeclareMathOperator{\Op}{Op}

\begin{document}
\title{Harmonic Analysis}
\author{Ikhan Choi}
\maketitle
\tableofcontents


\part{Singular integral operators}
\chapter{Calder\'on-Zygmund theory}


\section{Convolution type operators}

\begin{prb}[Calder\'on-Zygmund decomposition]
\end{prb}

\begin{prb}[Calder\'on-Zygmund decomposition of sets]
Let $f\in L^1(\R^d)$.
Let $E_nf$ be the conditional expectation with repect to the $\sigma$-algebra generated by dyadic cubes with side length $2^{-n}$.
Let $Mf:=\sup_nE_n|f|$ be the maximal function, and let $\Omega:=\{x:Mf(x)>\lambda\}$ for fixed $\lambda>0$.
For $x\in\Omega$ let $Q_x$ be the maximal dyadic cube such that $x\in Q_x$ and
\[\frac1{|Q_x|}\int_{Q_x}|f|>\lambda.\]
\begin{parts}
\item
$\{Q_x:x\in\Omega\}$ is a countable partition of $\Omega$.
\item
We have an weak type estimate $|\Omega|\le\frac1\lambda\|f\|_{L^1}$.
\item
$\|f\|_{L^\infty(\R^d\setminus\Omega)}\le\lambda$.
\item
For $x\in\Omega$
\[\frac1{|Q_x|}\int_{Q_x}|f|\le2^d\lambda.\]
\end{parts}
\end{prb}

\begin{prb}[Calder\'on-Zygmund decomposition of functions]
For $f\in L^1(\R^d)$ and $\lambda>0$, let $\Omega:=\{x:Mf(x)>\lambda\}$.

Depending on $\lambda>0$, let
\[g(x):=\begin{cases}|f(x)|&\text{ if }Mf(x)\le\lambda\\\frac1{|Q_x|}\int_{Q_x}|f|&\text{ if }Mf(x)>\lambda\end{cases}\]
and $b_i:=(|f|-g)\chi_{Q_i}$ so that $|f|=g+b$ where $b=\sum_ib_i$.
\begin{parts}
\item $\|g\|_{L^1}=\|f\|_{L^1}$ and $\|g\|_{L^\infty}\lesssim\lambda$.
\item $\|b\|_{L^1}\le2\|f\|_{L^1}$ and $\int b_i=0$.
\end{parts}
\end{prb}
\begin{pf}

\end{pf}




\begin{prb}[$L^p$ boundedness of Calder\'on-Zygmund operators]
Let $T:\cD(\R^d)\to\cD'(\R^d)$ be a \emph{singular integral operator of convolution type} in the sense that there is $K\in L_\loc^1(\R^d\setminus\{0\})\cap\cD'(\R^d)$ such that $Tf=K*f$ for all $f\in\cD(\R^d)$.
We usually say a singular integral operator is \emph{Calder\'on-Zygmund} if we can show the boundedness in $L^p$ by the Calder\'on-Zygmund decomposition or its modification.
Consider the following two conditions.
\begin{enumerate}[(i)]
\item $T$ is $L^2$-bounded: we have
\[\|Tf\|_{L^2}\lesssim\|f\|_{L^2},\]
\item $T$ satisfies the \emph{H\"ormander condition}: we have
\[\int_{|x|>2|y|}|K(x-y)-K(x)|\,dx\lesssim1,\qquad y>0.\]
\end{enumerate}

Let $f=g+b=g+\sum_ib_i$ be the Calder\'on-Zygmund decomposition at $\lambda>0$, and let $\Omega^*:=\bigcup_iQ_i^*$ where $Q_i^*$ is the cube with the same center as $Q_i$ and whose sides are $2\sqrt d$ times longer. 
\begin{parts}
\item
The H\"ormander condition implies
\[|\{x:|Tb(x)|>\frac\lambda2\}\setminus\Omega^*|\lesssim_d\frac1\lambda\|f\|_{L^1}.\]
\item
\end{parts}
\end{prb}
\begin{pf}
Using the Chebyshev inequality and the H\"older inequality, the $L^2$-boundedness of $T$ implies
\[|\{x:|Tg(x)|>\frac\lambda2\}|
\le\frac4{\lambda^2}\|Tg\|_{L^2(\R^d)}^2
\lesssim\frac4{\lambda^2}\|g\|_{L^2(\R^d)}^2
\le\frac4{\lambda^2}\|g\|_{L^1(\R^d)}\|g\|_{L^\infty(\R^d)}
\le\frac4\lambda\|f\|_{L^1(\R^d)}.
\]

Write
\[|\{x:|Tb(x)|>\tfrac\lambda2\}\setminus\Omega^*|
\le\frac2\lambda\int_{\R^d\setminus\Omega^*}|Tb(x)|\,dx
\le\frac2\lambda\sum_i\int_{\R^d\setminus Q_i^*}|Tb_i(x)|\,dx.\]
Since $x\in\R^d\setminus Q_i^*$ does not belong to $\supp b_i\subset Q_i$ and $\int b_i=0$, we have
\[Tb_i(x)=\int_{Q_i}K(x-y)b_i(y)\,dy=\int_{Q_i}[K(x-y)-K(x)]b_i(y)\,dy,\]
and
\[\int_{\R^d\setminus Q_i^*}|Tb_i(x)|\,dx
=\int_{Q_i}|b_i(y)|\int_{\R^d\setminus Q_i^*}|K(x-y)-K(x)|\,dx\,dy
\lesssim\|b_i\|_{L^1}.\]
(We need to show it is valid even though $b_i$ is not smooth)

(c)

\end{pf}

\begin{prb}[H\"older boundedness of Calder\'on-Zygmund operators]
\end{prb}


\section{Truncated integrals}


Let $E$ be a Banach space with a specified predual.
Let $T_n$ be a bounded sequence in $L(E)$.
The point-weak$^*$ topology on on $L(E)$ is complete on the closed unit ball.
The point-a.e.~convergence in $L(E_0,E)$ is stronger, where $E_0$ is a dense subspace of $E$.
Thus, if we show $T_n\to T$ in point-a.e.~in $L(E_0,E)$, then $T\in L(E)$.



Homogeneous kernels



\section{Hilbert transform}

\begin{prb}[Harmonic conjugate]
\end{prb}
\begin{prb}[Kernel representation]
\end{prb}
\begin{prb}[Fourier series in $L^p$ space]
\end{prb}






\section{$A_p$ weights}

\section{Bounded mean oscillation}


\section*{Exercises}
\begin{prb}[Size and cancellation condition]
Let $K\in L_\loc^1(\R^d\setminus\{0\})\cap\cD'(\R^d)$.
We say the condition $|K(x)|\lesssim|x|^{-d}$ for $x\ne0$ as the \emph{size condition}, and say the condition $\int_{r<|x|<R}K(x)\,dx=0$ for all $0<r<R<\infty$ as the \emph{cancellation condition}.
If $K$ satisfies the size, cancellation, and H\"ormander condition, then it is $L^2$ bounded, hence Calder\'on-Zygmund.
\end{prb}

\begin{prb}[Gradient size condition]
Let $|\nabla K(x)|\lesssim|x|^{-d-1}$ for $x\ne0$.
Then, convolution with $K$ is a Calder\'on-Zygmund operator.
\end{prb}

\begin{prb}[Riesz potential]
\end{prb}



\chapter{Littlewood-Paley theory}
\section{Littlewood-Paley decomposition}
\section{Multiplier theorems}



\chapter{Almost orthogonality}
Carleson measures, paraproducts
\section{Coltar lemma}
\section{$T(1)$ theorem}

% Fourier series
% randomness
% inner and outer functions?



\part{Oscillatory integral operators}

% oscillatory integrals - Sogge
% fourier restriction
% kakeya - Wolff
% bochner-riesz
% geometric measure
% Strichartz
% dispersive equations - Chicago and Prinston school

\chapter{Oscillatory integrals}




\begin{prb}[Justification of oscillatory integral]
For a function $\phi$ with fast growth toward infinity, we want to define a linear functional $I_\phi$ such that
\[I_\phi(a):=\int_{\R^d}e^{i\phi(x)}a(x)\,dx,\qquad a\in\cS(\R^d).\]
A linear functional of the above form is called the \emph{oscillatory integral} with \emph{phase function} $\phi$.
As a notation, we will use the above integral in the right-hand side to denote the value of $I_\phi$ even for $a\notin L^1(\R^d)$.
Then, we have pointwise justifications for integral calculus.
\begin{parts}
\item $I_\phi:A_\delta^m(\R^d)\to\C$ is well-defined and continuous, if $\phi$.
\item The change of variables is justified as follows:
\item The integral by parts is justified as follows:
\[\int_{\R^d}e^{i\phi(y)}i\partial\phi(y)a(x+y)\,dy=-\int_{\R^d}e^{i\phi(y)}\partial a(x+y)\,dy,\quad x\in\R^d,\ a\in A_\delta^m(\R^d).\]
\item The Fubini theorem is justified as follows:
\item The Fourier inversion is justified as follows:
\[a(x)=(2\pi)^{-d}\int_{\R^{2d}}e^{i(x-y)\xi}a(y)\,dy\,d\xi,\quad x\in\R^d,\ a\in A_\delta^m(\R^d).\]
\end{parts}
\end{prb}
\begin{pf}
(a)
Note that $A_\delta^m(\R^d)\cap L^1(\R^d)$ is dense in $A_\delta^m(\R^d)$.
The most difficult part is the construction and the computation of $L$ and its transpose.

(e)
Note that the function $(y,\xi)\mapsto a(y)$ belongs to $A_\delta^{m'}(\R^{2d})$ since
\end{pf}

\begin{prb}[Point evaluation of multiplier]
Let $\phi\in$ be a phase function.
We want to show the following point evaluation holds with previously justified oscillatory integral:
\[\Phi(D)a(x)=(2\pi)^{-d}\int_{\R^d}e^{i\phi(y)}a(x+y)\,dy,\qquad x\in\R^d,\ a\in A_\delta^m(\R^d),\]
where $\Phi:=\cF^*e^{i\phi}$.
Which condition for $\phi$ makes $\Phi$ be able to act on $\cS'$ by multiplication?
\end{prb}



\begin{prb}[Stationary phase approximation]

\end{prb}
\begin{pf}

\end{pf}



\begin{prb}[Van der Corput lemma]
	
\end{prb}
Dispersive equations and strichartz estimates

\section*{Exercises}
\begin{prb}[Fresnel phase]
We compute $L$ with a specific example
\end{prb}
\begin{pf}
\[(1+xQ^{-1}D)e^{\frac i2xQx}=\<x\>^2e^{\frac i2xQx}.\]
The transpose of $\<x\>^{-2}(1+xQ^{-1}D)$ is $\<x\>^{-2}(1+di-2ix^2-xD)$ for $Q=I$.

Note that $\<x\>^{-2n}\<D\>^{2n}$ is self-adjoint.

Let $Q$ be a non-degenerate symmetric bilinear form on $\R^d$.
Consider a multiplier operator $e^{\frac i2DQD}:\cS\to\cS$ such that
\[e^{\frac i2DQD}a(x):=\cF^*e^{\frac i2\xi Q\xi}\cF a(x).\]
\begin{parts}
\item
The pointwise evaluation is given by the oscillatory integral.
\[e^{\frac i2DQD}a(x)=(2\pi)^{-d}\frac{e^{\frac{i\pi}4}\sgn(Q)}{|\det Q|^{\frac12}}\int_{\R^d}e^{-\frac i2yQ^{-1}y}a(x+y)\,dy,\qquad x\in\R^d,\ a\in A_\delta^m.\]
\item
\[e^{\frac i2DQD}a(x)=\sum_{k=0}^n\frac{i^k}{2^kk!}(DQD)^ka(x)+r_n(x)\]
\end{parts}
\end{pf}


\chapter{Foureir restriction}
Kakeya
Bochner-Riesz
Geometric measure theory

\chapter{}









\part{Pseudo-differential operators}
% https://arxiv.org/abs/2107.12839
% https://mtaylor.web.unc.edu/notes/pseudodifferential-operators-fourier-integral-operators-and-microlocal-analysis/

% wave front sets
% wavelets
% uncertainty principle
% weyl calculus - Folland

\chapter{Pseudo-differential calculus}
\section{}

\begin{prb}[H\"ormander symbol classes]
Let $m,\rho,\delta\in\R$.
The \emph{H\"ormander class} $S_{\rho,\delta}^m(\R^{2d})$ of symbols is the set of smooth functions $a\in C^\infty(\R_x^d\times\R_\xi^d)$ such that
\[|\partial_x^\alpha\partial_\xi^\beta a(x,\xi)|\lesssim_{\alpha,\beta}\<\xi\>^{m+\delta|\alpha|-\rho|\beta|}\]
for each $\alpha,\beta\in\Z_{\ge0}^d$.
\begin{parts}
\item Fr\'echet space
\end{parts}
\end{prb}


\begin{prb}[Asymptotic expansion]
Let $\rho,\delta\in\R$.
Let $a_k\in S_{\rho,\delta}^{m_k}(\R^{2d})$ for a sequence $(m_k)_{k=0}^\infty\subset\R$ with $m_0$ and $m_k\downarrow-\infty$.
We want to construct $a\in S_{\rho,\delta}^{m_0}(\R^{2d})$ such that
\[a-\sum_{k=0}^{n-1}a_k\in S_{\rho,\delta}^{m_n}(\R^{2d}).\tag{\dagger}\]
The symbol $a_0$ is called the \emph{principal symbol} of $a$, or the operator $\Op^t(a)$.

Let $\chi\in C_c^\infty(\R_\xi^d,[0,1])$ be a cutoff function such that
\[\chi(\xi)=\begin{cases}1,&\text{ if }|\xi|\le1\\0,&\text{ if }|\xi|\ge2\end{cases}.\]
\begin{parts}
\item If $a\in S_{\rho,\delta}^m$, then $\chi(\e\xi)a(x,\xi)$ is uniformly bounded in $S_{\rho,\delta}^m$ for $\e\in(0,1)$ if $\rho\le1$.
\item There is $a\in S_{\rho,\delta}^{m_0}$ such that (\dagger) if $\rho\le1$.
\end{parts}
\end{prb}
\begin{pf}
(a)
On the support of $\xi\mapsto\chi(\e\xi)$ holds $\<\xi\><2|\xi|\le4\e^{-1}$ because $1<\e^{-1}$, so for each $\alpha,\beta\in\Z_{\ge0}^d$ we have
\begin{align*}
|\partial_x^\alpha\partial_\xi^\beta(\chi(\e\xi)a(x,\xi))|
&=|\sum_\tau{\beta\choose\tau}\partial_\xi^{\beta-\tau}(\chi(\e\xi))\partial_x^\alpha\partial_\xi^\tau a(x,\xi)|\\
&=|\sum_\tau{\beta\choose\tau}\e^{|\beta|-|\tau|}\partial_\xi^{\beta-\tau}\chi(\e\xi)\partial_x^\alpha\partial_\xi^\tau a(x,\xi)|\\
(\because\<\xi\>\le4\e^{-1})\quad
&\le\sum_\tau{\beta\choose\tau}(4\<\xi\>^{-1})^{|\beta|-|\tau|}|\partial_\xi^{\beta-\tau}\chi(\e\xi)||\partial_x^\alpha\partial_\xi^\tau a(x,\xi)|\\
&\lesssim\sum_{\tau}{\beta\choose\tau}\<\xi\>^{-(|\beta|-|\tau|)}\<\xi\>^{m+\delta|\alpha|-\rho|\tau|}\\
(\because\rho\le1)\quad
&\le\<\xi\>^{m+\delta|\alpha|-\rho|\beta|}.
\end{align*}

(b)
Because we have $\e^{-1}\le\<\xi\>$ on the support of $1-\chi(\e\xi)$, for each $k$ we can take a sequence $\e_k$ small enough such that
\[\max_{\substack{\alpha,\beta\in\Z_{\ge_0}^d\\|\alpha|+|\beta|\le k}}|\partial_x^\alpha\partial_\xi^\beta((1-\chi(\e_k\xi))a_k(x,\xi))|\le2^{-k}\<\xi\>^{m_k+1+\delta|\alpha|-\rho|\beta|}.\]
We may assume $\e_k\downarrow0$ so that the following sum is locally finite:
\[a(x,\xi):=\sum_{k=0}^\infty(1-\chi(\e_k\xi))a_k(x,\xi).\]
If we choose $n$ such that $m_0\ge m_n+1$, then in the expansion
\[a(x,\xi)=\sum_{k=0}^{n-1}(1-\chi(\e_k\xi))a_k(x,\xi)+\sum_{k=n}^\infty(1-\chi(\e_k\xi))a_k(x,\xi)\]
the first sum clearly belongs to $S_{\rho,\delta}^{m_0}$ and so is the second sum because
\begin{align*}
|\partial_x^\alpha\partial_\xi^\beta\sum_{k=n}^\infty(1-\chi(\e_k\xi))a_k(x,\xi)|
&\le\sum_{k=n}^\infty2^{-k}\<\xi\>^{m_{k+1}+1+\delta|\alpha|-\rho|\beta|}\\
&\le\<\xi\>^{m_n+1+\delta|\alpha|-\rho|\beta|}\\
&\le\<\xi\>^{m_0+\delta|\alpha|-\rho|\beta|}
\end{align*}
for every $\alpha,\beta\in\Z_{\ge0}^d$.
Therefore, $a\in S_{\rho,\delta}^{m_0}$.

Write
\[(a-\sum_{k=0}^{n-1}a_k)(x,\xi)=\sum_{k=0}^{n-1}\chi(\e_k\xi)a_k(x,\xi)+\sum_{k=n}^\infty(1-\chi(\e_k\xi))a_k(x,\xi).\]
The first sum belongs to $S^{-\infty}$ because it is compactly supported, and we can also show that the second sum belongs to $S_{\rho,\delta}^{m_n}$ by decomposing with $n'$ such that $m_n\ge m_n'+1$ and by considering the multiplication with a cutoff remains in the same symbol class.
\end{pf}


\begin{prb}[Quantization]
For a symbol $a$ defined on $\R^{2d}$ and $t\in[0,1]$, we want to define a pseudo-differential operator $\Op^t(a)$ such that
\[\Op^t(a)f(x):=(2\pi)^{-d}\int_{\R^{2d}}e^{i(x-y)\xi}a((1-t)x+ty,\xi)f(y)\,dy\,d\xi,\qquad f\in\cS(\R^d).\]
The operator $\Op^t(a)$ is the \emph{$t$-quantization} of the symbol $a$.
The analysis of $t$-quantizations is sometimes called the \emph{Kohn-Nirenberg calculus} for $t=0$, the \emph{Weyl calculus} for $t=\frac12$.
\begin{parts}
\item $\Op^0(a):\cS(\R^d)\to\cS'(\R^d)$ is well-defined and continuous, if $a\in\cS'(\R^2d)$.
\item $\Op^0(a):\cS(\R^d)\to\cS(\R^d)$ is well-defined and continuous, if $a\in S_{\rho,\delta}^m(\R^{2d})$ for $\delta\le1$.
\end{parts}
\end{prb}
\begin{pf}
(b)
For $\psi=e^{ih(kx-\omega t)}$, $H=i\partial_t$, $D=-i\partial_x$
 (we have $\xi=(\Ad\cF)D$),

Since $\<D_y\>^2$ is a self-adjoint partial differential operator, for any $n\in\Z_{\ge0}$ we have
\begin{align*}
\Op^0(a)f(x)
&=(2\pi)^{-d}\int_{\R^{2d}}e^{i(x-y)\xi}a(x,\xi)f(y)\,dy\,d\xi\\
(\because D_ye^{i(x-y)\xi}=-\xi e^{i(x-y)\xi})\quad
&=(2\pi)^{-d}\int_{\R^{2d}}\<\xi\>^{-2n}\<D_y\>^{2n}e^{i(x-y)\xi}a(x,\xi)f(y)\,dy\,d\xi\\
(\because\text{IBP})\quad
&=(2\pi)^{-d}\int_{\R^{2d}}e^{i(x-y)\xi}\<\xi\>^{-2n}a(x,\xi)\<D_y\>^{2n}f(y)\,dy\,d\xi.
\end{align*}
The derivatives of the integrand is integrable with respect to $\xi$ for a sufficiently large $n$ with $m+|\beta|-2n<-d$ because
\begin{align*}
&\hspace{-3em}|\partial_x^\beta(e^{i(x-y)\xi}\<\xi\>^{-2n}a(x,\xi)\<D_y\>^{2n}f(y))|\\
&=|\sum_\tau{\beta\choose\tau}(i\xi)^{\beta-\tau}e^{i(x-y)\xi}\<\xi\>^{-2n}\partial_x^\tau a(x,\xi)\<D_y\>^{2n}f(y)|\\
&\le\sum_\tau{\beta\choose\tau}\<\xi\>^{|\beta|-|\tau|}\<\xi\>^{-2n}|\partial_x^\tau a(x,\xi)||\<D_y\>^{2n}f(y)|\\
(\because a\in S_{\rho,\delta}^m)\quad
&\lesssim\sum_\tau{\beta\choose\tau}\<\xi\>^{|\beta|-|\tau|}\<\xi\>^{-2n}\<\xi\>^{m+\delta|\tau|}|\<D_y\>^{2n}f(y)|\\
(\because\delta\le1)\quad
&\lesssim\<\xi\>^{m+|\beta|-2n}|\<D_y\>^{2n}f(y)|,
\end{align*}
so the partial derivative $\partial_x$ commutes with the integral.
Since
\[x^\alpha e^{i(x-y)\xi}=(y+D_\xi)^\alpha e^{i(x-y)\xi}=\sum_{\sigma}{\alpha\choose\sigma}y^{\alpha-\sigma}D_\xi^\sigma e^{i(x-y)\xi},\]
we have an expansion
\begin{align*}
x^\alpha\partial_x^\beta\Op^0(a)f(x)
&=x^\alpha\partial_x^\beta\int_{\R^{2d}}e^{i(x-y)\xi}\<\xi\>^{-2n}a(x,\xi)\<D_y\>^{2n}f(y))\,dy\,d\xi\\
&=\int_{\R^{2d}}x^\alpha\partial_x^\beta(e^{i(x-y)\xi}\<\xi\>^{-2n}a(x,\xi)\<D_y\>^{2n}f(y))\,dy\,d\xi\\
&=\int_{\R^{2d}}\sum_{\sigma,\tau}{\alpha\choose\sigma}{\beta\choose\tau}y^{\alpha-\sigma}D_\xi^{\sigma}e^{i(x-y)\xi}(i\xi)^{\beta-\tau}\<\xi\>^{-2n}\partial_x^\tau a(x,\xi)\<D_y\>^{2n}f(y)\,dy\,d\xi\\
&=\int_{\R^{2d}}\sum_{\sigma,\tau}{\alpha\choose\sigma}{\beta\choose\tau}e^{i(x-y)\xi}(-D_\xi)^{\sigma}[(i\xi)^{\beta-\tau}\<\xi\>^{-2n}\partial_x^\tau a(x,\xi)]y^{\alpha-\sigma}\<D_y\>^{2n}f(y)\,dy\,d\xi.
\end{align*}
Here
\[\sup_{x\in\R^d}|(-D_\xi)^{\sigma}[(i\xi)^{\beta-\tau}\<\xi\>^{-2n}\partial_x^\tau a(x,\xi)]|\]
is integrable with respect to $\xi$ for sufficiently large $n$, so with this $n$ we have
\[\sup_{x\in\R^d}|x^\alpha\partial_x^\beta\Op^0(a)f(x)|\lesssim\sum_{\sigma\le\alpha}\sup_{y\in\R^d}|y^{\alpha-\sigma}\<D_y\>^{2n}f(y)|\]
for each $\alpha,\beta\in\Z_{\ge0}^d$ and all $f\in\cS(\R^d)$, which implies $\Op^0(a)f\in\cS(\R^d)$.
\end{pf}

\begin{prb}[Change of quantization]
Let $m\in\R$, .
\begin{parts}
\item $\Op^t(a)=\Op^0(e^{itD_xD_\xi}a)$.
In particular, since $M_{e^{itx\xi}}:\cS(\R^{2d})\to\cS(\R^{2d})$, $\Op^t(a):\cS(\R^{2d})\to\cS(\R^{2d})$ is well-defined and continuous.
\item $a\in S_{\rho,\delta}^m(\R^{2d})$ if and only if $e^{itD_xD_\xi}a\in S_{\rho,\delta}^m(\R^{2d})$, if $0\le\delta\le\rho\le1$ and $\delta<1$.
\item We have the formal adjoint
\[\Op^t(a)^*=\Op^{1-t}(\bar a).\]
In particular, $\Op^t(a):\cS'(\R^d)\to\cS'(\R^d)$ is well-defined and continuous for $a\in S_{\rho,\delta}^m(\R^{2d})$.
\end{parts}
\end{prb}

\begin{pf}
(a)
Note that
\begin{align*}
\Op^t(a)f(x)
&=(2\pi)^{-d}\int_{\R^{2d}}e^{i(x-y)\xi}a((1-t)x+ty,\xi)f(y)\,dy\,d\xi\\
(\because\text{Inversion on }\R^{2d})\quad
&=(2\pi)^{-3d}\int_{\R^{4d}}e^{i(x-y)\xi}e^{i((1-t)x+ty)x^*+i\xi\xi^*}\hat a(x^*,\xi^*)f(y)\,dx^*\,d\xi^*\,dy\,d\xi\\
&=(2\pi)^{-3d}\int_{\R^{4d}}e^{i(x-y+\xi^*)\xi}\hat a(x^*,\xi^*)e^{i((1-t)x+ty)x^*}f(y)\,dx^*\,d\xi^*\,dy\,d\xi\\
(\because\text{Inversion on }\R^d)\quad
&=-(2\pi)^{-2d}\int_{\R^{2d}}\hat a(x^*,y-x)e^{i((1-t)x+ty)x^*}f(y)\,dx^*\,dy\\
(\because[\xi^*/y-x])\quad
&=-(2\pi)^{-2d}\int_{\R^{2d}}e^{i(x+t\xi^*)x^*}\hat a(x^*,\xi^*)f(x+\xi^*)\,dx^*\,d\xi^*.
\end{align*}

(b)
We have the oscillatory integral
\[e^{itD_xD_\xi}a(x,\xi)=(2\pi)^{-d}|t|^{-d}\int_{\R^{2d}}e^{-it^{-1}y\eta}a(x+y,\xi+\eta)\,dy\,d\eta.\]
%\[b\sim\sum_{\alpha\in\Z_{\ge0}^d}\frac{(t-s)^{|\alpha|}}{i^{|\alpha|}\alpha!}\partial_x^\alpha\partial_\xi^\alpha a\]if $\delta<\rho$.
Enough to show
\[|\int_{\R^{2d}}e^{-it^{-1}y\eta}a(x+y,\xi+\eta)\,dy\,d\eta|\lesssim\<\xi\>^m.\]
Fix $\xi$ and
\delta\le\rho
\end{pf}




\begin{prb}[Moyal product]
Let $a\in S_{\rho,\delta}^m(\R^{2d})$ and $b\in S_{\rho,\delta}^l(\R^{2d})$.
\begin{parts}
\item there exists a unique function $a\#^tb\in S_{\rho,\delta}^{m+l}(\R^{2d})$ such that
\[a^t(x,D)b^t(x,D)=(a\#^tb)^t(x,D).\]
\item It is concretely described by
\[(a\#^tb)(x,\xi)=(2\pi)^{-2}\int_{\R^{4d}}e^{-i(y\eta-z\zeta)}a(x+tz,\xi+\eta)b((1-t)y+x,\xi+\zeta)\,dy\,d\eta\,dz\,d\zeta.\]
\item If $\delta<\rho$, then
\[a\#^tb(x,\xi)\sim\sum_{k\in\Z_{\ge0}}\frac1{i^kk!}(\partial_y\partial_\eta-\partial_z\partial_\zeta)^ka((1-t)x+tz,\eta)b(tx+(1-t)y,\zeta)\Bigr|_{\substack{y=z=x\\\eta=\zeta=\xi}}.\]
\end{parts}
\end{prb}



\begin{prb}[Parametirx and elliptic operators]
\end{prb}





\begin{prb}[Calder\'on-Vaillancourt theorem]
\end{prb}

\section*{Exercises}
Quantization of linera and quadratic exponential symbols.

\chapter{Semiclassical analysis}

We define for $h>0$ and $t\in[0,1]$
\[\Op_h^t(a)f(x):=(2\pi h)^{-d}\int_{\R^{2d}}e^{\frac ih(x-y)\xi}a((1-t)x+ty,\xi)f(y)\,dy\,d\xi,\qquad f\in\cS(\R^d).\]
\[\Op_h^w(D_xa)=[D_x,\Op_h^w(a)],\qquad\Op_h^w(hD_\xi a)=-[x,\Op_h^w(a)].\]
For example, regardless of $h>0$ and $t\in[0,1]$,
\[\Op(\xi)\psi(x)=hD\psi(x)=-ih\psi'(x)\]
and
\[\Op(H)\psi(x)=-\frac{h^2}{2m}\Delta\psi(x)+V(x)\psi(x),\]
where
\[H(x,\xi):=\frac{|\xi|^2}{2m}+V(x).\]
In physics, the operator $\Op(H)$ is frequently written as $\hat H$, which will not be used to avoid the confusion regarding the Fourier transform.


\[\dd{t}a(t)=\{a(t),H\}=X_Ha(t)\]
\[\dd{t}\hat a(t)=\dd{t}e^{\frac iht\hat H}\hat ae^{-\frac iht\hat H}=-\frac ih[\hat a(t),\hat H]\]


Let $J:\R^{2d}\to\R^{2d}:(x,\xi)\mapsto(\xi,-x)$ be a symplectomorphism, the rotation of $\frac\pi2$ in \emph{clock-wise}.
Then, we have
\[\cF_h^*\Op_h^w(a)\cF_h=\Op_h^w(J^*a).\]
Also,
\[[\Op_h^w(a),\Op_h^w(b)]=\Op_h^w(-ih\{a,b\})+O(h^2).\]

Since the Weyl quantization has a bound
\[\|\Op_h^w(a)\|_{B(L^2(\R^d))}\lesssim\|a\|_{C_b(\R^{2d})}+O(h^{\frac12}),\qquad a\in C_b(\R^{2d})\cap S_{\rho,\delta}^m(\R^{2d}),\]
for a bounded net $f_h\in L^2(\R^d)$, the positive linear functional $C_0(\R^{2d})$ defined by
\[a\mapsto\<\Op_h^w(a)f_h,f_h\>,\qquad a\in C_0(\R^{2d})\cap S_{\rho,\delta}^m(\R^{2d})\]
has a limit point in the weak$^*$ topology.
If a finite Radon measure $\mu$ on $\R^{2d}$ is a limit, then $\mu$ is called a \emph{semicalssical defect} of the net $f_h$.

Let $p$ be a symbol such that $|p(x,\xi)|\gtrsim\<\xi\>^k$ for sufficiently large $|\xi|$.
This symbol has an interpretation of the Hamiltonian.
Suppose the following two conditions are satisfied:
\[\lim_{h\to0}\|\Op_h^w(p)f_h\|_{L^2(\R^d)}=0,\qquad\|f_h\|_{L^2(\R^d)}=1.\]
Then, the support of any semicalssical defect measure $\mu$ is contained in $p^{-1}(0)$, called the \emph{characteristic variety} or the \emph{zero energy surface} of the symbol $p$.
We can understand this support restriction as saying that in the semiclassical limit $h\to0$ all the mass of solution coalesces onto a specific set in phase space.
Also we have the flow invariance $\{p,\mu\}=0$, i.e.~$\int_{\R^{2d}}\{p,a\}d\mu=0$ for all $a\in\cD(\R^{2d})$, which means that $\mu$ is invariant under the Hamiltonian flow generated by $p$.




\section{Heisenberg group}
\section{Phase space transforms}

\chapter{Microlocal analysis}




\end{document}