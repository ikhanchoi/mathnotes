\documentclass{../../large}
\usepackage{../../ikhanchoi}


\begin{document}
\title{Harmonic Analysis}
\author{Ikhan Choi}
\maketitle
\tableofcontents

\part{Fourier analysis}

\chapter{Fourier series}

\section{Fourier series in $L^p$ spaces}
\begin{prb}
\[\|\hat f\|_{\ell^1(\Z)}\lesssim\|f\|_{W^{1,1+\e}(\T)}.\]
\end{prb}

Inversion theorem is an approximation problem given by $\cF^*\cF=\lim_{n\to\infty}\cF_n^*\cF$.
The condition $\hat f\in\ell^1(\Z)$ is a condition just for defining $\cF^*\hat f$ wihtout using distribution theory, and it does not affect the inversion phenomena.
The approximation, in other words, can be seen as an extension method for $\cF^*:\ell^1(\Z)\to C(\T)$ on $c_0(\Z)$.
Note that $\cF_n^*$ on $c_0(\Z)$ cannot be bounded directly without distribution theory, but $\cF_n^*\cF$ on $L^p(\T)$ can be bounded well.

\begin{itemize}
\item If $\cF_n^*$ is the standard partial sum, then $\cF_n^*\cF$ is the Dirichlet kernel.
\item If $\cF_n^*$ is the Ces\`aro mean, then $\cF_n^*\cF$ is the Fej\'er kernel.
\item If $\cF_r^*$ is the Abel sum, then $\cF_r^*\cF$ is the Poisson kernel.
\item In Fourier transform, we often use the Gauss-Weierstrass kernel.
\end{itemize}

The injectivity of $\cF$ is not an easy problem, which comes from the inversion theorem.




\section{Pointwise convergence of Fourier series}

BV function: Dini, Jordan's criterion
\begin{prb}[Riemann localization principle]
\end{prb}


\section*{Exercises}
\begin{prb}[Gibbs phenomenon]
\end{prb}
\begin{prb}[Du Bois-Reymond function]
\end{prb}









\chapter{Fourier transform}
\section{Fourier transform in $L^p$ space}
\begin{prb}[Riemann-Lebesgue lemma]
\end{prb}
Lp extension

Gaussian function computation: differential equation method, contour integral method
inversion theorem
\begin{prb}[Plancherel theorem]
\end{prb}

\section{Tempered distributions}
\begin{prb}[Cauchy principal value]
indented contour, imaginary shift, Feynman's trick
\end{prb}



\section*{Exercises}
\begin{prb}[Sampling theorem]
\[\cF\1_{[-\frac12,\frac12]}(\xi)=\operatorname{sinc}(\xi/2)\]
$\operatorname{sinc}\in L^{1+\e}(\R)$.
\end{prb}
\begin{prb}[Poisson summation formula]
\end{prb}
\begin{prb}[Uncertainty principle]
\end{prb}


\section*{Problems}
\begin{enumerate}
\item Find all $\alpha>0$ such that
\[\lim_{x\to\infty}x^{-\alpha}\int_0^xf(y)\,dy=0\]
for all $f\in L^3([0,\infty))$.
\end{enumerate}











\chapter{}




\part{Singular integral operators}
\chapter{Cald\'eron-Zygmund theory}

\section{Hilbert transform}

\section{Calder\'on-Zygmund operators of convolution type}
\begin{prb}[Calder\'on-Zygmund decomposition of sets]
Let $E_nf$ be the conditional expectation with repect to the $\sigma$-algebra generated by dyadic cubes with side length $2^{-n}$.
Let $Mf=\sup_nE_n|f|$ be the maximal function, and let $\Omega:=\{x:Mf(x)>\lambda\}$ for fixed $\lambda>0$.
For $x\in\Omega$ let $Q_x$ be the maximal dyadic cube such that $x\in Q_x$ and
\[\frac1{|Q_x|}\int_{Q_x}|f|>\lambda.\]
\begin{parts}
\item
$\{Q_x:x\in\Omega\}$ is a countable partition of $\Omega$.
\item
We have an weak type estimate $|\Omega|\le\frac1\lambda\|f\|_{L^1}$.
\item
$\|f\|_{L^\infty(\R^d\setminus\Omega)}\le\lambda$.
\item
For $x\in\Omega$
\[\frac1{|Q_x|}\int_{Q_x}|f|\le2^d\lambda.\]
\end{parts}
\end{prb}

\begin{prb}[Calder\'on-Zygmund decomposition of functions]
Let
\[g(x):=\begin{cases}|f(x)|&,x\notin\Omega\\\frac1{|Q_x|}\int_{Q_x}|f|&,x\in\Omega\end{cases}\]
and $b_i:=(|f|-g)\chi_{Q_i}$ so that $|f|=g+b$ where $b=\sum_ib_i$.
\begin{parts}
\item $\|g\|_{L^1}=\|f\|_{L^1}$ and $\|g\|_{L^\infty}\lesssim_d\lambda$.
\item $\|b\|_{L^1}\le2\|f\|_{L^1}$ and $\int b_i=0$.
\end{parts}
\end{prb}
\begin{pf}

\end{pf}


\begin{prb}[Calder\'on-Zygmund operators of convolution type]
Let $T:C_c^\infty(\R^d)\to\cD'(\R^d)$ be a \emph{singular integral operator of convolution type} in the sense that there is $K\in L_\loc^1(\R^d\setminus\{0\})\cap\cD'(\R^d)$ such that
\[Tf(x)=\int K(x-y)f(y)\,dy\]
for all $f\in\cD(\R^d)$, whenever $x\notin\supp f$.
We say $T$ is called a \emph{Calder\'on-Zygmund} operator if
\begin{enumerate}[(i)]
\item $T$ is $L^2$-bounded: we have
\[\|Tf\|_{L^2}\lesssim\|f\|_{L^2},\]
\item $T$ satisfies the \emph{H\"ormander condition}: we have
\[\int_{|x|>2|y|}|K(x-y)-K(x)|\,dx\lesssim1\]
for every $y>0$.
\end{enumerate}

Let $f=g+b=g+\sum_ib_i$ be the Calder\'on-Zygmund decomposition, and let $\Omega^*:=\bigcup_iQ_i^*$ where $Q_i^*$ is the cube with the same center as $Q_i$ and whose sides are $2\sqrt d$ times longer. 
\begin{parts}
\item
The $L^2$-boundedness implies
\[|\{x:|Tg(x)|>\tfrac\lambda2\}|\lesssim_d\frac1\lambda\|f\|_{L^1}.\]
\item
The H\"ormander condition implies
\[|\{x:|Tb(x)|>\tfrac\lambda2\}\setminus\Omega^*|\lesssim_d\frac1\lambda\|f\|_{L^1}.\]
\item
\end{parts}
\end{prb}
\begin{pf}
(a)
Using the Chebyshev inequality and the H\"older inequality,
\[|\{x:|Tg(x)|>\frac\lambda2\}|
\le\frac4{\lambda^2}\|Tg\|_{L^2(\Omega)}^2
\le\frac{4C}{\lambda^2}\|g\|_{L^2(\Omega)}^2
\le\frac{4C}{\lambda^2}\|g\|_{L^1(\Omega)}\|g\|_{L^\infty(\Omega)}.
\]

(b)
Write
\[|\{x:|Tb(x)|>\tfrac\lambda2\}\setminus\Omega^*|
\le\frac2\lambda\int_{\R^d\setminus\Omega^*}|Tb(x)|\,dx
\le\frac2\lambda\sum_i\int_{\R^d\setminus Q_i^*}|Tb_i(x)|\,dx.\]
Since $x\in\R^d\setminus Q_i^*$ does not belong to $\supp b_i\subset Q_i$ and $\int b_i=0$, we have
\[Tb_i(x)=\int_{Q_i}K(x-y)b_i(y)\,dy=\int_{Q_i}[K(x-y)-K(x)]b_i(y)\,dy,\]
and
\[\int_{\R^d\setminus Q_i^*}|Tb_i(x)|\,dx
=\int_{Q_i}|b_i(y)|\int_{\R^d\setminus Q_i^*}|K(x-y)-K(x)|\,dx\,dy
\lesssim\|b_i\|_{L^1}.\]
(We need to show it is valid even though $b_i$ is not smooth)

(c)

\end{pf}

\section{$L^2$-boundedness of truncated integrals}

\section{Calder\'on-Zygmund operators of non-convolution type}
standard kernels



\section*{Exercises}
\begin{prb}[Gradient size condition]
Let $|\nabla K(x)|\lesssim\frac1{|x|^{d+1}}$ for $x\ne0$.
Then, convolution with $K$ is a Calder\'on-Zygmund operator.
\end{prb}




\chapter{Littlewood-Paley theory}
\section{Littlewood-Paley decomposition}
\section{Multiplier theorems}

\chapter{}


\part{Oscillatory integral operators}

\chapter{Stationary phase}

\chapter{Restriction and Kekeya problems}

\chapter{Dispersive equations}


\part{Pseudo-differential operators}
% https://arxiv.org/abs/2107.12839
\chapter{}
\section{}
$S_{\rho,\delta}^m$
\[|D_x^\alpha D_\xi^\beta a(x,\xi)|\lesssim\<\xi\>^{m-\rho|\beta|+\delta|\alpha|}.\]

Let $a$ be a symbol on $M=\R_x^d\times\R_\xi^d$.
Then, the associated $\Psi$DO is
\[T_a\psi(x):=\frac1{(2\pi)^d}\iint e^{i\<x-y,\xi\>}a(x,\xi)\psi(y)\,dy\,d\xi.\]
For parameters $0\le\lambda\le1$ and $h>0$, let
\[\hat a\psi(x):=\frac1{(2\pi h)^d}\iint e^{\frac ih\<x-y,\xi\>}a((1-\lambda)x+\lambda y,\xi)\psi(y)\,dy\,d\xi.\]
For example, regardless of $h$ and $\lambda$,
\[\hat\xi\psi(x)=\frac hi\psi'(x)\]
and
\[\hat H\psi(x)=-h^2\Delta\psi(x)+V(x)\psi(x),\]
where $V:\R_x^d\times\R_\xi^d\to\R$ and $H:\R_x^d\times\R_\xi^d\to\R$ such that
\[H(x,\xi):=|\xi|^2+V(x).\]


\[\dd{t}a(t)=\{a(t),H\}=X_Ha(t)\]
\[\dd{t}\hat a(t)=\dd{t}e^{\frac iht\hat H}\hat ae^{-\frac iht\hat H}=-\frac ih[\hat a(t),\hat H]\]


\chapter{Semiclassical analysis}
\section{Quantization}
\begin{prb}[Composition of Weyl quantization]

\end{prb}


\chapter{Microlocal analysis}




\end{document}