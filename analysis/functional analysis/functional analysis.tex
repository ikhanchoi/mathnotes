\documentclass{../../large}
\usepackage{../../ikhanchoi}

\newcommand{\Prim}{\operatorname{Prim}}

\begin{document}
\title{Functional Analysis}
\author{Ikhan Choi}
\maketitle
\tableofcontents

\part{Topological vector spaces}


\chapter{Locally convex spaces}
\section{Vector topologies}

\begin{prb}[Topological vector spaces]
We will define a \emph{topological vector space} as a vector space together with a Hausdorff vector topology.
For every topological vector space, there is a balanced neighborhood system at zero.
\end{prb}
\begin{prb}[Canonical uniformity and bornology]
\end{prb}
\begin{prb}[Continuity and boundedness of linear operators]
\end{prb}
\begin{prb}[Metrizable topological vector spaces]
Birkhoff-Kakutani
\end{prb}


\begin{prb}[Continuous linear functionals]
A linear functional $l:X\to\F$ is continuous if and only if $\ker l$ is closed, if and only if $|l|$ is continuous.
\end{prb}


\section{Seminorms and convex sets}
\begin{prb}[Locally convex spaces]
A topological vector space $X$ over $\F=\R$ or $\C$ is called a \emph{locally convex space} if there is a convex balanced neighborhood system at zero.

\end{prb}

\begin{prb}[Seminorms]
Let $X$ be a vector space over $\F=\R$ or $\C$.
A \emph{semi-norm} is a functional $p:X\to\R$ such that
\[p(x)\ge0,\qquad p(\lambda x)=|\lambda|p(x),\qquad p(x+y)\le p(x)+p(y),\qquad x,y\in X,\ \lambda\in\C.\]


For a convex balanced set $U$ such that $U-x$ is absorbing for every $x\in U$, there is a unique semi-norm $p$ on $X$ such that $U=\{x\in X:p(x)<1\}$.
The semi-norm $p$ is called the \emph{Minkowski functional} of $U$.
It is defined by
\[p(x):=\inf\{t\ge0:x\in tU\}.\]

In a given topological vector space, open convex sets correspond to continuous sublinear functionals, open convex balanced sets correspond to continuous semi-norms.

Equivalent conditions on the continuity of seminorms,
boundedness by seminorms,
normability
\end{prb}
\begin{pf}
\end{pf}

\begin{prb}
convex hulls
\end{prb}



\section{Continuous linear functionals}
\begin{prb}
Let $(x_i^*)\in X^{*n}$.
We can define $(x_i^*):X\to\F^n$.
If $x^*\in X^*$ vanishes on $\bigcap_{i=1}^n\ker x_i^*$, then $x^*$ is a linear combination of $\{x_i^*\}$.
\end{prb}



\begin{prb}[Hahn-Banach extension theorem]
Let $X_0\subset X$ be vector spaces over $\F=\R$ or $\C$.
A real functional $q:X\to\R$ is said to be \emph{sublinear} if $q(x+y)\le q(x)+q(y)$ and $q(tx)=tq(x)$ for all $x,y\in X$ and $t\in\R$.
\begin{parts}
\item For $\F=\R$ and $q:X\to\R$ sublinear, any linear functional $l_0:X_0\to\R$ satisfying $l_0\le q$ on $X_0$ admits a linear extension $l:X\to\R$ satisfying $l\le q$.
\item For $\F=\C$ and $p:X\to\R$ a semi-norm, any linear functional $l_0:X_0\to\R$ satisfying $|l_0|\le p$ on $X_0$ admits a linear extension $l:X\to\R$ satisfying $|l|\le p$.
\item If $X$ is locally convex, then a bounded linear functional $l_0:X_0\to\F$ admits a bounded linear extension $l:X\to\F$.
If $X$ is normed, we can take $l$ such that $\|l\|=\|l_0\|$.
\end{parts}
\end{prb}
\begin{pf}
(a)
Consider a partially ordered set of all linear extensions of $l_0$ dominated by $q$.
Precisely, we consider the set
\[S:=\{l:V\to\R\mid
X_0\subset V\subset X,\ l_0=l|_{X_0},\ l\le q\},\]
on which the partial order $\subset$ is given such that for elements $l:V\to\R$ and $\tilde l:\tilde V\to\R$ of $S$ we have $l\subset\tilde l$ if and only if $\tilde l$ is an extension of $l$.
The non-emptyness and the chain condition is easily satisfied, so the partially ordered set has a maximal element $l:V\to\R$ by the Zorn lemma.

Suppose $V\ne X$ and let $e\in X\setminus V$.
We want to assign an appropriate value to the vector $e$ to extend our maximal extension $l$.
The inequality
\[l(v)+l(w)=l(v+w)\le q(v+w)\le q(v-e)+q(w+e),\qquad v,w\in V\]
implies the existence of $r\in\R$ such that
\[\sup_{v\in V}(l(v)-q(v-e))\le r\le\inf_{v\in V}(-l(v)+q(v+e)).\]
If we define $\tilde V:=V+\R e$ and $\tilde l:\tilde V\to\R$ such that $\tilde l(v+te):=l(v)+tr$ for $t\in\R$, then $\tilde l$ extends $l$ and
\[\tilde l(v+te)=l(v)+tr\le
\begin{cases}l(v)+t(-l(t^{-1}v)+q(t^{-1}v+e))&,t\ge0\\l(v)+t(l(-t^{-1}v)-q(-t^{-1}v-e))&,t\le0
\end{cases}
=q(v+te),\qquad v\in V,\ t\in\R,
\]
which leads a contradiction to the maximality of $l$.
Therefore, we conclude $V=X$.

(b)
Note that the real part map $\Re:\Hom_\C(X,\C)\to\Hom_\R(X,\R)$ is bijective.
Note also that $|l|\le p$ if and only if $\Re l\le p$ for any complex linear functional $l:V\to\C$ on a complex vector space $V$.
It is because $|l|\le p$ implies $\Re l\le|l|\le p$ and conversely $\Re l\le p$ implies $|l|\le p$ by the inequality
\[|l(v)|^2=l(v)\bar{l(v)}=l(\bar{l(v)}v)=\Re l(\bar{l(v)}v)\le p(\bar{l(v)}v)=|l(v)|p(v),\qquad v\in V.\]

Since $|l_0|\le p$, we have $\Re l_0\le p$.
Using the part (a), there is a linear functional $l:X\to\C$ such that $\Re l_0=\Re l$ on $X_0$ and $\Re l\le p$.
Then, we can deduce $l_0=l$ on $X_0$ and $|l|\le p$.
\end{pf}



\begin{prb}[Hahn-Banach separation]

Let $X$ be a locally convex space over $\F=\R$ or $\C$.
Let $C$ be a closed convex subset and $K$ be a compact convex subset which are disjoint.
Then, there is a continuous linear functional $l:X\to\F$ such that
\[\sup_{c\in C}\Re l(c)<\inf_{k\in K}\Re l(k).\]
\end{prb}




\section*{Exercises}
\begin{prb}[Topology of compact convergence]
\end{prb}





\chapter{Barrelled spaces}

\section{Uniform boundedness principle}
\begin{prb}[Barreled spaces]
Let $X$ be a topological vector space.
We say a subset $B\subset X$ is \emph{absorbing} if for every $x\in X$ there is a sufficiently large $t>0$ such that $x\in tB$.
A \emph{barrel} is a convex balanced subset of $X$ which is closed and absorbing.
A \emph{barrelled space} is a topological vector space in which every barrel is a neighborhood of zero.
\end{prb}

% If a closed convex cone contains a dense subset of absorbing at a point, then it is entire?

\begin{prb}[Uniform boundedness principle]
Let $X$ and $Y$ be topological vector spaces.
We say that a family $\{T_i\}$ of continuous linear operators from $X$ to $Y$ is \emph{pointwise bounded} or \emph{strongly bounded} if $\{T_ix\}\subset Y$ is bounded for each $x\in X$, and is \emph{uniformly bounded} or \emph{equicontinuous} if 

The \emph{uniform boundedness principle} states that a pointwise bounded family is equicontinuous.
It is also frequently called the \emph{Banach-Steinhaus theorem}.
\begin{parts}
\item If $X$ is barrelled and $Y$ is locally convex, then the uniform boundedness principle holds.
\item If $X$ is complete and metrizable, then the uniform boundedness principle holds.
\end{parts}
\end{prb}
\begin{pf}
(a)
Let $V$ be a convex balanced open neighborhood of zero in $Y$.
Our goal is to construct an balanced open neighborhood $U$ of zero in $X$ such that $T_iU\subset V$ for all $i$.
Take a convex balanced open neighborhood $V'$ of zero in $Y$ such that $\bar{V'}\subset V$.
Define
\[B:=\bigcap_iT_i^{-1}\bar{V'}.\]
Then, $B$ is clearly convex, balanced, and closed.
We can see it is also absorbing because for any $x\in X$, since $\{T_ix\}$ is bounded in $Y$, there is $t>0$ such that $\{T_ix\}\subset t\bar{V'}$, which implies $x\in tB$ by definition of $B$.
Thus $B$ is a barrel in a barrelled space $X$, so it contains a convex balanced open neighborhood $U$ of zero.
From $T_iU\subset\bar{V'}\subset V$ for every $i$, we have $\{T_i\}$ is equicontinuous.
\end{pf}



\section{Baire category theorem}

\begin{prb}[Baire spaces]
A topological space is called a \emph{Baire space} if the countable intersection of open dense subsets is always dense.
\begin{parts}
\item If a topological vector space is Baire, then it is barrelled.
\item A Baire space is second category in itself.
\item A topological group that is second category in itself is Baire.
\end{parts}
\end{prb}



\begin{prb}
Let $B\subset X$ be a closed absorbing subset of a topological vector space $X$ that is Baire.
Then, $B$ has a non-empty open subset, and $B-B$ is a neighborhood of zero.
If $B$ is convex in addition, then $B$ is a neighborhood of zero.
\end{prb}


\begin{prb}[Baire category theorem]
The Baire category theorem proves many exmples of topological vector space are Baire, in particular barrelled.
\begin{parts}
\item A complete metric space is Baire.
\item A locally compact Hausdorff space is Baire.
\end{parts}
\end{prb}




\section{Open mapping theorem}

\begin{prb}[Open mapping theorem]
Let $X$ and $Y$ be topological vector spaces.
The \emph{open mapping theorem} states that continuous surjective linear operator $T:X\to Y$ is an open map.
\begin{parts}
\item If $X$ is complete metrizable and $Y$ is Baire, the open mapping theorem holds.
\item If $X$ is a Fr\'echet and $Y$ is barrelled, the open mapping theorem holds.
\end{parts}
\end{prb}

\begin{pf}
(a)
Let $U$ be balanced open neighborhoods of zero in $X$.
It is enough to prove $TU$ is a neighborhood of zero.
We first claim the closure $\bar{TU}$ is a neighborhood of zero.
Take an balanced open neighborhood $U'$ of zero in $X$ such that $U\supset U'-U'$.
Because $U'$ is absorbing and $T$ is surjective, the set $\bar{TU'}$ is closed and absorbing in a Baire space $Y$ so that $\bar{TU'}-\bar{TU'}$ is a neighborhood of zero, hence the claim follows from $\bar{TU'}-\bar{TU'}\subset\bar{TU}$.

Since $X$ is metrizable, we have a countable balanced open neighborhood system $\{U_n\}_{n=1}^\infty$ of zero in $X$ such that $\bar{U_1+U_1}\subset U$ and $U_{n+1}+U_{n+1}\subset U_n$ for all $n$.
It suffices to prove $\bar{TU_1}\subset TU$.
Take arbitrary $y_1\in\bar{TU_1}$.
We construct sequences $x_n\in U_n$ and $y_n\in\bar{TU_n}$ as follows:
Assuming $y_n\in\bar{TU_n}$, since $\bar{TU_{n+1}}$ is a neighborhood of zero as we have shown above so that $y_n+\bar{TU_{n+1}}$ is a neighborhood of a limit point $y_n$ of $TU_n$, we have $Tx_n\in y_n+\bar{TU_{n+1}}$ for some $x_n\in U_n$, and we can let $y_{n+1}:=y_n-Tx_n\in\bar{TU_{n+1}}$.
Then, the partial sum $\sum_{k=1}^nx_k$ is a Cauchy sequence because
\[\sum_{k=m+1}^nx_k\in U_{m+1}+\cdots+U_n\subset U_m,\qquad n>m\ge1,\]
and it converges to $x\in U$ by the completeness of $X$ and
\[\sum_{k=1}^nx_n\in U_1+U_2+\cdots+U_n\subset U_1+U_1,\qquad n\ge1.\]
With this $x$, we can check $Tx=y_1$ by taking limit on
\[T\sum_{k=1}^nx_k=\sum_{k=1}^nTx_k=\sum_{k=1}^n(y_k-y_{k+1})=y_1-y_{n+1}.\qedhere\]
\end{pf}

A first countable topological vector space is metrizable.
A locally complete metrizable topological vector space is complete metrizable.


\section*{Exercises}

\begin{prb}
Let $(T_n)$ be a sequence in $B(X,Y)$.
If $T_n$ coverges strongly then $\|T_n\|$ is bounded by the uniform boundedness principle.
\end{prb}

\begin{prb}
There is a closed absorbing set in $\ell^2(\Z_{\ge0})$ that is not a neighborhood of zero;
\[\bar B(0,1)\setminus\bigcup_{i=2}^\infty B(i^{-1}e_i,i^{-2})\]
is a counterexample.
\end{prb}




\begin{prb}
There is no metric $d$ on $C([0,1])$ such that $d(f_n,f)\to0$ if and only if $f_n\to f$ pointwise as $n\to\infty$ for every sequence $f_n$.
Note that this problem is slightly different to the non-metrizability of the topology of pointwise convergence.
\end{prb}

\begin{prb}
We show that there is no projection from $\ell^\infty$ onto $c_0$.
\end{prb}

\begin{prb}[Schur property]
$\ell^1$
\end{prb}

\begin{prb}
Let $\f:L^\infty([0,1])\to\ell^\infty(\N)$ be an isometric isomorphism.
Suppose $\f$ is realised as a sequence of bounded linear functionals on $L^\infty$.
\begin{parts}
\item
Show that $\f^*(\ell^1)\subset L^1$ where $\ell^1$ and $L^1$ are considered as closed linear subspaces of $(\ell^\infty)^*$ and $(L^\infty)^*$ respectively.
\item Show that $\f^*$ is indeed an isometric isomorphism, and deduce $\f$ cannot be realised as bounded linear functionals on $L^\infty$.
\end{parts}
\end{prb}


\begin{prb}[Daugavet property]
\begin{parts}
\item The real Banach space $C([0,1])$ satisfies the Daugavet property.
\end{parts}
\end{prb}
\begin{pf}
Let $T$ be a finite rank operator on $C([0,1])$, and $e_i$ be a basis of $\im T$.
Then, for some measures $\mu_i$,
\[Tf(t)=\sum_{i=1}^n\int_0^1f\,d\mu_ie_i(t).\]
Let $M:=\max\|e_i\|$.

Take $f_0$ such that $\|f_0\|=1$ and $\|Tf_0\|>\|T\|-\frac\e2$.
Reversing the sign of $f_0$ if necessary, take an open interval $\Delta$ such that $Tf_0(t)\ge\|T\|-\frac\e2$ and $|\mu_i|(\Delta)\le\frac\e{4nM}$ for all $i$.
Define $f_1$ such that $f_0=f_1$ on $\Delta^c$, $f_1(t_0)=1$ for some $t_0\in\Delta$, and $\|f_1\|=1$.
Then, $\|Tf_1-Tf_0\|\le\frac\e2$ shows $Tf_1\ge\|T\|-\e$ on $\Delta$.
Therefore,
\[\|1+T\|\ge\|f_1+Tf_1\|\ge f_1(t_0)+Tf_1(t_0)\le1+\|T\|-\e.\]
\end{pf}

\begin{prb}[Bartle-Graves theorem]
Let $E$ be a Banach space and $N$ a closed subspace.
For $\e>0$, there is a continuous homogeneous map $\rho:E/N\to E$ such that $\pi\rho(y)=y$ and $\|\rho(y)\|\le(1+\e)\|y\|$ for all $y\in E/N$.
\end{prb}
\begin{pf}
We want to construct a continuous map $\psi:S_{E/N}\to E$ with $\|\psi(y)\|\le1+\e$ for all $y\in S_{E/N}$.
If then, $\rho$ can be made from $\psi$.

For each $y_0\in S_{E/N}$, choose $x_0\in\pi^{-1}(y_0)\cap B_{1+\e}$.
There is a neighborhood $V_{y_0}\subset S_{E/N}$ of $y_0$ such that $y\in V_{y_0}$ implies $x_0$ belongs to $(\pi^{-1}(y)\cap B_{1+\e})+U_{2^{-1}}$, which is convex.
With a locally finite subcover $V_{y_\alpha}$ and a partition of unity $\eta_\alpha(y)$, define $\psi_1(y)=\sum_\alpha\eta_\alpha(y)x_\alpha$.
Then, $\psi_1(y)\in(\pi^{-1}(y)\cap B_{1+\e})+U_{2^{-1}}$.

For $i\le2$, choose for each $y_0$ the element $x_0$ in $\pi^{-1}(y_0)\cap B_{1+\e}\cap(\psi_{i-1}(y_0)+U_{2^{-{i-1}}})$.
Then, we obtain
\[\psi_i(y)\in\Bigl(\pi^{-1}(y)\cap B_{1+\e}\cap(\psi_{i-1}(y_0)+U_{2^{-{i-1}}})\Bigr)+U_{2^{-i}}.\]
Therefore, $\|\psi_i(y)-\psi_{i-1}(y)\|<2^{-{i-2}}$, so it converges uniformly to $\psi$ such that $\psi(y)\in\pi^{-1}(y)\cap B_{1+\e}$.
\end{pf}

\section*{Problems}
\begin{prb}
Let $T$ be an invertible linear operator on a normed space.
Then, $T^{-2}+\|T\|^{-2}$ is injective if it is surjective.
\end{prb}
















\chapter{Weak topologies}
\section{Dual pairs}

\begin{prb}[Dual pairs]
A \emph{dual pair} is a pair $(X,F)$ of vector spaces over a field $\F$ together with a non-degenerate bilinear form $X\times F\to\F$.
A pair $(X,F)$ of a vector space $X$ and a subspace $F$ of $X^\#$ is a natural dual pair if and only if $F$ separates points of $X$.
For a topological vector space $X$, we consider $(X,X^*)$ as a canonical dual pair associated to $X$, and if $F$ is a linear subspace of $X^*$, then $(X,F)$ is a dual pair if and only if $F$ is weakly$^*$ dense in $X^*$ by the Hahn-Banach separation.
Note that if $X$ is discrete, then $X^*=X^\#$.
If $(X,F)$ is a dual pair, then $(F,X)$ is also a dual pair.
A dual pair is never a topological notion.
When we consider a canonical dual pair $(X,X^*)$, we forget the topologies on $X$ and $X^*$ after defining $X^*$ as the continuous dual.
\end{prb}

\begin{pf}
For a linear subspace $V$ of a topological vector space $X$, $\bar V=V^{\perp\perp}$.
If $x\in\bar V$, then for $x^*\in V^\perp$, we have $\<x,x^*\>=0$ by approximation, so $x\in V^{\perp\perp}$.
Conversely, if $x\notin\bar V$, then the Hahn-Banach extension implies that there is $x^*$ such that $\<y,x^*\>=0<\<x,x^*\>$ for all $y\in V$, which means $x^*\in V^\perp$ and $x\notin V^{\perp\perp}$.

\end{pf}

\begin{prb}
Let $X$ be a locally convex space.
\begin{parts}
\item $X_\sigma$ and $X^*_\sigma$ are locally convex.
\item $(X_\sigma)^*=X^*$.
\item $(X^*_\sigma)^*=X$. Every locally convex space is a dual of a locally convex space.
\end{parts}
\end{prb}
\begin{pf}
(a)
The Hahn-Banach theorem implies the Hausdorffness.

(c)
We will only show $(X^*_\sigma)^*\subset X$.
If $x^{**}\in(X^*_\sigma)^*$, then there is a finite subset $\{x_i\}_{i\in J}$ of $X$ such that
\[|\<x^{**},x^*\>|\le\sum_{i\in J}|\<x_i,x^*\>|,\qquad x^*\in X^*.\]
Since $\bigcap_{i\in J}\ker x_i$ is a closed subspace of $\ker x^{**}$, we have $x^{**}\in\spn\{x_i\}_{i\in J}\subset X$.
\end{pf}

\begin{prb}
closure and weak closure of convex subsets
\end{prb}
\begin{pf}
Hahn-Banach
\end{pf}

\begin{prb}[Polar topologies]
For a vector space $X$ and a subspace $X^*\subset X^\#$, the Mackey topology $\tau(X,X^*)$ on $X$ is the topology of uniform convergence on weakly$^*$ compact balanced convex subsets of $X^*$.
We can show $X^*=(X_\tau)^*$, i.e.~$\tau$ is a dual topology.

Let $\alpha$ is a polar topology on $X$ generated by $\cG\subset\cP(X^*)$.
If $x^*\in(X_\alpha)^*$, then there is $\sigma(X,X^*)$-closed balanced convex $C\in\cG$ such that $|x^*|\le1$ on $C^\circ$.

$X_\sigma$, $X^*_\sigma$, $X_\tau$, $X^*_\tau$, $X_\beta$, $X^*_\beta$.
$(X^*)_\sigma=:X^*_\sigma$, $(X_\sigma)^*=X^*$.

\begin{parts}
\item If a locally convex space $X$ is barrelled or metrizable, then $X$ is Mackey, i.e.~$X_\tau=X_\beta$.
\item The Mackey topology is the finest topology such that $(X_\tau)^*=X^*$.
\end{parts}
\end{prb}

Mackey-Arens

boundedness, incompleteness


\begin{prb}[Strong bidual]

\end{prb}

\begin{prb}[Weak convergence of bounded nets]
Let $X$ be a Banach space, $D^*$ a subset of $X^*$, and $\bar{D^*}$ the norm closure of $D^*$.
For example, if $X$ has a predual $X_*\subset X^*$ and $D^*$ is dense in $X_*$, then $\sigma(X,\bar{D^*})$ is the weak$^*$ topology.
\begin{parts}
\item There is a squence $x_n\in X$ converges to zero in $\sigma(X,D^*)$ but not in $\sigma(X,\bar{D^*})$.
\item A bounded sequence $x_n\in X$ converges to zero in $\sigma(X,\bar{D^*})$ if in $\sigma(X,D^*)$.
\end{parts}
\end{prb}
\begin{pf}
(b)
Let $\xi\in\bar{D^*}$ and choose $\eta\in D^*$ such that $\|\xi-\eta\|<\e$.
Then,
\[|\<x_n,\xi\>|\le\|x_n\|\|\xi-\eta\|+|\<x_n,\eta\>|\lesssim\e+|\<x_n,\eta\>|\to\e.\]
\end{pf}



\begin{prb}[Alaoglu theorem]
Let $X$ be a topological vector space and $U$ be a balanced open neighborhood of zero.
The polar $U^\circ\subset X^*$ is weakly$^*$ compact.
\end{prb}
\begin{pf}
The algebraic dual $(X^\#,\sigma(X^\#,X))$ with weak$^*$ topology is complete because $X^\#$ is embedded into the product space $\F^X$ as a closed subspace.
Note that we have $X^*\subset X^\#$

Consider
\[B_{X^*}\to\prod_{x\in X}\|x\|B:l\mapsto(l(x))_{x\in X}.\]
Since it is an embedding into a compact space, it suffices to show the closedness of image: for $l(x):=\lim_\alpha l_\alpha(x)$, we have
\[\|l(x)\|\le\|l(x)-l_\alpha(x)\|+\|x\|\xrightarrow{\alpha\to\infty}\|x\|,\]
so $l$ is contained in the range.
\end{pf}
\begin{prb}[Eberlein-\v Smulian theorem]
\end{prb}
\begin{prb}[James' theorem]
\end{prb}






Two Krein-\v Smulian theorems




Bishop-Phelps theorem

\begin{prb}[Goldstine theorem]
Let $X$ be a normed space.
Then, $B_X$ is weakly$^*$ dense in $B_{X^{**}}$.
\end{prb}
\begin{pf}
Take $x^{**}\in B_{X^{**}}\setminus\bar{B_X}^{w*}$.
By the Hahn-Banach separation, there is $x^*\in X^*$ such that
\[\sup_{x\in B_X}\Re\<x,x^*\><\Re\<x^{**},x^*\>.\]
Since the left hand side is equal to $\|x^*\|$, we obtain a contradiction.
\end{pf}


\section{Compact convex sets}
Krein-Milman theorem
Choquet theory




\section*{Exercises}
\begin{prb}[James' space]
not reflexive but isometrically isomorphic to bidual
\end{prb}

\begin{prb}[Preduals]
Let $X$ be a Banach space.
A \emph{predual} of $X$ is a Banach space $F$ together with an isometric isomorphism $\f:X\to F^*$.
Two preduals $\f_1:X\to F_1^*$ and $\f_2:X\to F_2^*$ are said to be equivalent if there is an isometric isomorphism $\theta:F_1\to F_2$ such that $\theta^*=\f_1\f_2^{-1}$.
\begin{parts}
\item There is a one-to-one correspondence between the equivalence class of preduals of $X$ and the set of closed subspaces $X_*$ of $X^*$ such that $B_X$ is compact and Hausdorff in $(X,\sigma(X,X_*))$.
Such a subspace $X_*$ is also called a predual of $X$.
\item If $X$ admits a predual $X_*\subset X^*$, then a $\sigma(X,X_*)$-closed subspace $V$ of $X$ also admits a predual $X_*|_V$.
\end{parts}
\end{prb}
\begin{pf}
(a) Goldstine theorem for surjectivity.

(b)
It is easy if we apply the part (a).
We can show more directly.
If we let $V_*:=X_*|_V$ the image of $X_*$ under the map $X^*\to V^*$, then we have isometric injections $V\to(V_*)^*\to X$.
We can show $V$ is $\sigma(X,X_*)$ dense in $(V_*)^*$, hence the closedness proves the bijectivity of $V\to(V_*)^*$.
\end{pf}

\begin{prb}[Mazur's lemma]

\end{prb}



















\part{Banach spaces}




\chapter{Operators on Banach spaces}

\section{Bounded operators}
\begin{prb}[Bounded belowness in Banach spaces]
Let $T\in B(X,Y)$ for Banach spaces $X$ and $Y$.
The following statements are equivalent:
\begin{parts}
\item $T$ is bounded below.
\item $T$ is injective and has closed range.
\item $T$ is a topological isomorphism onto its image.
\end{parts}
\end{prb}

\begin{prb}[Bounded belowness in Hilbert spaces]
Let $T\in B(H,K)$ for Hilbert spaces $H$ and $K$.
The following statements are equivalent:
\begin{parts}
\item $T$ is bounded below.
\item $T$ is left invertible.
\item $T^*$ is right invertible.
\item $T^*T$ is invertible.
\end{parts}
\end{prb}

\begin{prb}[Injectivity and surjectivity of adjoint]
Let $T:X\to Y$ be a continuous linear operator between locally convex spaces.
\begin{parts}
\item $T^*$ is injective if and only if $T$ has dense range.
\item $T^*$ is surjective if and only if $T$ is an embedding.
\end{parts}
\end{prb}








\section{Compact operators}

$K(X,Y)$ is closed in $B(X,Y)$.
$K(X)$ is an ideal of $B(X)$.
adjoint is $K(X,Y)\to K(Y^*,X^*)$.
integral operators are compact.
riesz operator, quasi-nilpotent operator.




\section{Fredholm operators}

\begin{prb}
Let $E$ and $F$ be Fr\'echet spaces.
We say $T\in B(E,F)$ is \emph{Fredholm} if its kernel is finite-dimensional and its range is finite-codimensional.
\begin{parts}
\item A Fredholm operator $T\in B(E,F)$ has closed range.
\item If $K\in K(E)$, then $1-K$ is Fredholm.
\end{parts}
\end{prb}
\begin{pf}
(a)
For a Fredholm operator $T\in B(E,F)$, let $\ran T^\perp$ be a finite-dimensional subspace of $F$ such that there is the summation gives rise to a continuous bijection $\ran T\oplus\ran T^\perp\to F$, which is a topological isomorphism by the open mapping theorem.

(b)
Let $T:=1-K$.
Since $K$ is a compact identity on the closed subspace $\ker T$, so it is finite-dimensional.
Before we see $\ran T$ is finite-codimensional, we first show $\ran T$ is closed in $F$.
Consider a continuous bijection $T_0:\ker T^\perp\to\ran T$, where $\ker T^\perp$ is a complement of $\ker T$ in $E$.
Suppose that $T_0$ is not bounded below so that there is a sequence of unit vectors $x_n\in\ker T^\perp$ such that $Tx_n=x_n-Kx_n\to0$.
We may assume $Kx_n\to x$ for some $x\in E$ by compactness of $K$, which implies $x_n\to x$ in the closed subspace $\ker T^\perp$.
Since $Tx=x-Kx=0$, we have $x\in\ker T^\perp\cap\ker T=\{0\}$, so we obtain a contradiction $x_n\to0$.
Thus, $T_0$ is bounded below, and $\ran T=\ran T_0$ is closed.
Now we claim $(E/\ran T)^*$ is finite-dimensional.
Since the adjoint $(E/\ran T)^*\to E^*$ of the canonical projection $E\to E/\ran T$ is bounded below with image contained in $\ker T^*$, and the compactness of $K^*$ implies that $T^*=1-K^*$ has finite-dimensional kernel, $\ran T$ is finite-codimensional.
\end{pf}

\begin{prb}[Atkinson theorem]
Let $E$ and $F$ be Banach spaces.
\begin{parts}
\item An operator $T\in B(E,F)$ is Fredholm if and only if there is $S\in B(F,E)$ such that $1-ST$ and $1-TS$ is finite-rank.
\item An operator $T\in B(E)$ is Fredholm if and only if $\pi(T)$ is invertible in $Q(E)$.
\end{parts}
\end{prb}
\begin{pf}
(a)
Let $T\in B(E,F)$ be a Fredholm operator.
Since a finite-dimensional subspace $\ker T\subset E$ and a finite-codimensional subspace $\ran T\subset F$ are complemented, we have closed subspaces $\ker T^\perp$ and $\ran T^\perp$ such that $\ker T^\perp\oplus\ker T=E$ and $\ran T\oplus\ran T^\perp=F$.
By the open mapping theorem, the restriction $T_0:\ker T^\perp\to\ran T$ is a topological isomorphism.
Let $S_0:\ran T\to\ker T^\perp$ be the inverse of $T_0$ and define $S:F\to E$ such that $S:=S_0\oplus0:F\to E$.
Then, $1-ST$ and $1-TS$ are finite-rank.

(b)
By the part (a) for $E=F$, if $T\in B(E)$ is Fredholm, then $\pi(T)$ is invertible in $Q(E)$.
Conversely, if $\pi(T)$ has an inverse $\pi(S)$ in $Q(E)$ for some $S\in B(E)$, then since $1-ST$ and $1-TS$ are compact, $ST$ and $TS$ are Fredholm.
Then, $\ker(ST)\supset\ker T$ is finite-dimensional and $\ran(TS)\subset\ran T$ is finite-codimensional, so $T$ is Fredholm.
\end{pf}

\begin{prb}[Fredholm index]
locally constant, in particular, continuous.
composition makes the addition of indices.
\end{prb}

\section{Nuclear operators}
tensor products





\section*{Exercises}

\begin{prb}[Completely continuous operators]
On reflexive spaces, completely continuous operators are same with compact operators.
\end{prb}


\begin{prb}[Dunford-Pettis property]
A Banach space $X$ is said to have the \emph{Dunford-Pettis property} if all weakly compact operators $T:X\to Y$ to any Banach space $Y$ is completely continuous.
\begin{parts}
\item $X$ has the Dunford-Pettis property if and only if for every sequences $x_n\in X$ and $x^*_n\in X^*$ that converge to $x$ and $x^*$ weakly we have $x^*_n(x_n)\to x^*(x)$.
\item $C(\Omega)$ for a compact Hausdorff space $\Omega$ has the Dunford-Pettis property.
\item $L^1(\Omega)$ for a probability space $\Omega$ has the Dunford-Pettis property.
\item Infinite dimensional reflexive Banach space does not have the Dunfor-Pettis property.
\end{parts}
\end{prb}


\begin{prb}\,
\begin{parts}
\item (Mazur-Ulam, 1932) A surjective isometry $T:X\to Y$ between normed spaces is affine.
\item (Mankiewicz, 1972) Let $U,V$ be open sets in $X,Y$, normed spaces. A surjective isometry $U\to V$ is uniquely extended to a surjective isometry $X\to Y$.
\item (Mori) A surjective local isometry $T:X\to Y$ between Banach spaces is an isometry, if $X$ is separable. (Use the Baire category)
\end{parts}
\end{prb}
\begin{sol}
(a)
$T$ is continuous.
It is easy to see for continuous map $T$ that it is affine if and only if $T$ preserves the midpoint.
For $x_1\ne x_2\in X$ let $x_0$ be the midpoint.
Define inductively
\[C_1:=\{x\in X:\|x-x_1\|=\|x-x_2\|=\frac12\|x_1-x_2\|\},\qquad C_k:=\{x\in C_{k-1}:\sup_{x'\in C_{k-1}}\|x-x'\|\le\frac12\diam C_{k-1}\}.\]
Since $x_0\in C_{k-1}$ and $x'\in C_{k-1}$ imply $x_0\in C_k$ by $\|x_0-x'\|=\frac12\|(2x_0-x')-x'\|\le\frac12\diam C_{k-1}$, and since $\diam C_k\le\frac12\diam C_{k-1}$, we have $\{x_0\}=\bigcup_{k=1}^\infty C_k$.
It follows that the midpoint can be detected from the metric structure of $X$, not depending on the linear structure of $X$.
\end{sol}


\section*{Problems}
\begin{enumerate}
\item If $T\in B(L^2([0,1]))$ is a compact operator, then for any $\e>0$ there is a constant $C_\e>0$ such that
\[\|Tf\|_{L^2}\le\e\|f\|_{L^2}+C_\e\|f\|_{L^1}.\]
\end{enumerate}

\begin{pf}
1. Suppose there is $\e>0$ such that we have sequence $f_n\in L^2$ satisfying $\|f_n\|_2=1$ and
\[\|Tf_n\|_2>\e+n\|f_n\|_1.\]
By the compactness of $T$, there is a subsequence $Tf_{n_k}$ converges to $g\ne0$ in $L^2$.
Then, $\|f_{n_k}\|_1\to0$ implies $f_{n_k}\to0$ weakly in $L^2$, hence also for $Tf_{n_k}$.
It means $g=0$, which contradicts to the assumption.
\end{pf}




\chapter{Tensor products of Banach spaces}

\section{Injective and projective tensor products}

\begin{prb}[Realizations]
For Banach spaces $X$ and $Y$, $\cL(X,Y)$ and $\cB(X,Y)=\cL(X,Y^*)$ are naturally Banach spaces.
Also we have a natural algebraic inclusions of $X\otimes Y$ into $\cL(X^*,Y)\le\cB(X^*,Y^*)$, and $\cB(X,Y)^*$.
Also we have a natural algebraic inclusions of $X^*\otimes Y$ into $\cL(X,Y)\le\cB(X,Y^*)$.
\end{prb}

\begin{prb}
Let $X$ and $Y$ be a Banach spaces, and $\alpha$ be a norm on $X\otimes Y$.
We say $\alpha$ is a \emph{cross norm} if
\[\alpha(x\otimes y)=\|x\|\|y\|,\qquad x\in X,\ y\in Y,\]
and a cross norm is \emph{reasonable} if the \emph{dual norm} $\alpha^*$ on $X^*\otimes Y^*\subset(X\otimes Y,\alpha)^*$ of $\alpha$ is also a cross norm.
\[\e(u):=\|u\|_{\cB(X^*,Y^*)},\qquad\pi(u):=\|u\|_{\cB(X,Y)^*}.\]
\end{prb}

\begin{prb}[Type C and type L spaces]
\end{prb}

\begin{prb}[Duals of tensor products]
\[\cK(X,Y)\hookrightarrow X^*\hat\otimes_\e Y\leftarrow X^*\hat\otimes_\pi Y\twoheadrightarrow\cN(X,Y).\]
	
\end{prb}







\section{Vector-valued integrals}


harmonic and complex analysis


\begin{prb}[Pettis measurability theorem]
Let $(\Omega,\mu)$ be a measure space and $X$ a Banach space.
Let $f:\Omega\to X$ be a function.
We say $f$ is \emph{strongly measurable} or \emph{Bochner measurable} if it is a pointwise limit of a sequence of simple functions.

If $\mu$ is complete, then all the pointwise convergence discussed here can be relaxed to the almost everywhere convergence.
\begin{parts}
\item If $f$ is strongly measurable, then $f$ is Borel measurable.
\item If $f$ is Borel measurable, then $f$ is weakly measurable.
\item If $f$ is weakly measurable and separably valued, then $f$ is strongly measurable.
\end{parts}
\end{prb}

\begin{prb}[Pettis integrals]
\[L^1\hat\otimes_\e X\hookrightarrow\cL(X^*,L^1)\stackrel{*}{\hookrightarrow}\cL(L^\infty,X^{**}).\]
\begin{itemize}
\item Pettis integrable: $L^1\hat\otimes_\e X$,
\item weakly integrable: $\cL(X^*,L^1)$,
\item Dunford integrable: $\cL(L^\infty,X^{**})$,
\item Pettis integral: $L^1\hat\otimes_\e X\cong *^{-1}\cL(L^\infty,X)\subset\cL(X^*,L^1)$. It defines $L^1\hat\otimes_\e X\hookrightarrow\cK(L^\infty,X_\sigma)$.
\end{itemize}

\begin{parts}
\item The close graph theorem and the existence of an a.e.~convergent subsequence of an $L^1$ convergent sequence proves a weakly integrable function defines an operator in $\cL(X^*,L^1)$.
\end{parts}
\end{prb}

\begin{prb}[Bochner integrals]
Let $(\Omega,\mu)$ be a measure space and $X$ a Banach space.
Let $f:\Omega\to X$ be a strongly measurable function.
The function $f$ is said to be \emph{Bochner integrable} if there is a net of simple functions $(s_\alpha)_{\alpha\in\cA}$ such that
\[\int_\Omega\|f(\omega)-s_\alpha(\omega)\|\,d\mu(\omega)\to0\]
for $\alpha\in\cA$.

For $T\in\cL(X,Y)$ and $\mu:L^1(\mu)\to\C$, the commutative diagram for $\alpha\in\{\e,\pi\}$
\[\begin{tikzcd}
L^1(\mu)\hat\otimes_\alpha X \dar[swap]{\id\otimes T}\rar{\mu\otimes\id} & X \dar{T}\\
L^1(\mu)\hat\otimes_\alpha Y \rar{\mu\otimes\id} & Y,
\end{tikzcd}\]
which is shown with approximation by simple tensors, justifies that $T$ commutes with the integral:
\[T\int f\,d\mu=\int Tf\,d\mu.\]
The space of Bochner integrable functions $L^1\hat\otimes_\pi X$, factoring through $L^1\hat\otimes_\e X$, is naturally mapped to the space of Pettis integrals $\cK(L^\infty,X_\sigma)$.
\begin{parts}
\item $f$ is Bochner integrable if and only if $\int\|f(\omega)\|\,d\mu(\omega)<\infty$.
\item If $f$ is Bochner integrable, then it is Pettis integrable and the integrals coincides.
\end{parts}
Bochner integrable => Pettis integrable => weakly(scalarly) integrable
\end{prb}


\begin{prb}[Vector measures]
If an element of the Dunford integral $\cL(L^\infty,X^{**})$, or the Pettis integral $\cK(L^\infty,X_\sigma)$, defines a $\sigma$-weakly continuous linear operator $L^\infty\to X$, then it is called a vector measure?
\end{prb}






\section{Approximation property}
dual is Banach.
Basis problem, Mazur' duck.


\begin{prb}[Approximation property]
Every compact operator is a limit of finite-rank operators.
\begin{parts}
\item An Hilbert space has the AP.
\item
\end{parts}
\end{prb}
\begin{pf}
(a)
Let $H$ be a Hilbert space and $K\in K(H)$.
Since $\bar{KB_H}$ is a compact metric space, it is separable, which means $\bar{KH}$ is separable.
Let $(e_i)_{i=1}^\infty$ be an orthonormal basis of $\bar{KH}$, and let $P_n$ be the orthogonal projection on the space spanned by $(e_i)_{i=1}^n$.
If we let $K_n:=P_nK$, then $K_n\to K$ strongly and $K_n$ has finite rank.
Take any $\e>0$ and find, using the totally boundedness of $KB_H$, a finite subset $\{x_j\}\subset B_H$ such that for any $x\in B_H$ there is $x_j$ satisfying $\|Kx-Kx_j\|<\frac\e2$.
Then,
\begin{align*}
\|Kx-K_nx\|
&\le\|Kx-Kx_j\|+\|Kx_j-K_nx_j\|+\|P_n(Kx_j-Kx)\|\\
&\le\frac\e2+\|Kx_j-K_nx_j\|+\frac\e2.
\end{align*}
By taking the supremum on $x\in B_H$, we have
\[\|K-K_n\|\le\max_j\|Kx_j-K_nx_j\|+\e,\]
which deduces $K_n\to K$ in norm.

\end{pf}




\section*{Exercises}
Tingley problem



\chapter{Geometry of Banach spaces}







\part{Spectral theory}

\chapter{Operators on Hilbert spaces}

\section{Operator topologies}
Projections. Reducing subspaces.
Hilbert space classification by cardinal.
Riesz representation theorem.
\begin{prb}
\begin{parts}
\item A Banach space $X$ is isometrically isomorphic to a Hilbert space if there is a bounded linear projection on every closed subspace of $X$.
\end{parts}
\end{prb}

\begin{prb}[Riesz representation theorem]
Let $H$ be a Hilbert space over a field $\K$, which is either $\R$ of $\C$.


We use the bilinear form $\<-,-\>:X\times X^*\to\K$ of canonical duality.
The Riesz representation theorem states that a continuous linear functional on a Hilbert space is represented by the inner product with a vector.
\begin{parts}
\item For each $x^*\in H^*$, there is a unique $x\in H$ such that $\<y,x^*\>=\<y,x\>$ for every $y\in H$.
\item $H\to H^*:x\mapsto\<-,x\>$ is a natural linear and anti-linear isomorphism if $\K=\R$ and $\C$, respectively.
\end{parts}
\end{prb}



Let $H$ be a separable Hilbert space.
Find a positive sequence $a_n$ such that every sequence $x_n$ of unit vectors of $H$ satisfying $|\<x_i,x_j\>|\le a_j$ for all $i<j$ converges weakly to zero.



\begin{prb}[Normal operators]
For $T\in B(H)$, we have an obvious fact $(\im T)^\perp=\ker T^*$.
Suppose $T$ is normal.
\begin{parts}
\item $\ker T=\ker T^*$.
\item $T$ is bounded below if and only if $T$ is invertible.
\item If $T$ is surjective, then $T$ is invertible.
\end{parts}
\end{prb}

\begin{prb}[Invariant and Reducing subsapces]
Let $K$ be a closed subspace of $H$.
\begin{parts}
\item $K$ is reducing for $T$ if and only if $K$ is invariant for $T$ and $T^*$.
\item $K$ is reducing for $T$ if and only if $TP=PT$, where $P$ is the orthogonal projection on $K$.
\end{parts}
\end{prb}
% self adjoint operators
% invariant but not reducing for unitary operators
% eigenspaces
% matrix representation


% direct sum and tensor product of hilbert spaces

\begin{prb}[Trace class operators]
Let $K\in B(H)$.
The \emph{trace} of $K$ is
\[\Tr(K):=\sum_i\<Ke_i,e_i\>,\]
where $(e_i)\subset H$ is an orthonormal basis.
The operator $K$ is said to be in the \emph{trace-class} if $\Tr(|K|)<\infty$.
\begin{parts}
\item
The trace does not depend on the choice of $(e_i)$.
\item
$K$ is a trace class if and only if $K=\sum_i\lambda_i\theta_{e_i,e'_i}$ for some $(\lambda_i)\in\ell^1(\N)$ and orthonormal sequences $(e_i),(e'_i)\subset H$.
\item
$B(H)\to L^1(H)^*:T\mapsto\Tr(T\cdot)$ is an isometric isomorphism.
\end{parts}
\end{prb}
\begin{pf}
(b)
Applying the polar decomposition and diagonalizing the compact operator $|K|$, we are done.
Conversely, we can check the diagonalization $K^*K=\sum_i|\lambda_i|^2\theta_{y_i}$, which implies $|K|=\sum_i|\lambda_i|\theta_{y_i}$.
Thus,
\[Tr(|K|)=\sum_j\<|K|y_j,y_j\>=\sum_i|\lambda_i|<\infty.\]

\end{pf}




\begin{prb}[Six locally convex topologies]
Let $H$ be a Hilbert space.
\[T\mapsto(\|Tx\|^2+\|T^*x\|^2)^{\frac12},\qquad
T\mapsto\|Tx\|,\qquad
T\mapsto\<Tx,x\>\]
for $x\in H$.
\[T\mapsto\Bigl(\sum_{i=1}^\infty\|Tx_i\|^2+\|T^*x_i\|^2\Bigr)^{\frac12},\qquad
T\mapsto\Bigl(\sum_{i=1}^\infty\|Tx_i\|^2\Bigr)^{\frac12},\qquad
T\mapsto\Bigl|\sum_{i=1}^\infty\<Tx_i,x_i\>\Bigr|\]
for $(x_i)\in\ell^2(\N,H)$.

\begin{parts}
\item
A net $T_\alpha$ converges to $T$ strongly in $B(H)$ if and only if $\|(T_\alpha-T)^{\oplus n}\bar\xi\|\to0$ for all $\bar\xi\in H^{\oplus n}$.
\item
A net $T_\alpha$ converges to $T$ $\sigma$-strongly in $B(H)$ if and only if $\|(T_\alpha-T)^{\oplus\infty}\bar\xi\|\to0$ for all $\bar\xi\in H^{\oplus\infty}$.
\end{parts}
\end{prb}


\begin{prb}[Continuity of linear functionals]
Let $l$ be a linear functional on $B(H)$ for a Hilbert space $H$.
\begin{parts}
\item
$l$ is weakly continuous if and only if it is strongly$^*$ continuous, and in this case we have
\[l=\sum_i\lambda_i\omega_{e_i,e'_i},\qquad(\lambda_i)\in c_c,\quad(e_i),(e'_i)\subset H\text{ orthonormal}.\]
or equivalently,
\[l=\sum_i\omega_{x_i,y_i},\qquad(x_i),(y_i)\in c_c(\N,H)\]
\item
$l$ is $\sigma$-weakly continuous if and only if it is $\sigma$-strongly$^*$ continuous, and in this case we have 
\[l=\sum_i\lambda_i\omega_{e_i,e'_i},\qquad(\lambda_i)\in\ell^1,\quad(e_i),(e'_i)\subset H\text{ orthonormal}.\]
or equivalently,
\[l=\sum_i\omega_{x_i,y_i},\qquad(x_i),(y_i)\in\ell^2(\N,H)\]
\item For a convex subset of $B(H)$ is ($\sigma$-)weakly closed if and only if ($\sigma$-)strongly$^*$ closed.
\end{parts}
\end{prb}
\begin{pf}
Suppose $l$ is strongly continuous.
There exists $\bar x\in H^{\oplus n}$ such that
\[|l(T)|\le\|T^{\oplus n}\bar x\|.\]
The functional $l:A\to\C$ factors through $H^{\oplus n}$ such that
\[A\xrightarrow{\bar x}H^{\oplus n}\to\C.\]
\end{pf}

\begin{prb}[]
\,
\begin{parts}
\item On a bounded subset of $B(H)$, the weak, strong, strong$^*$ topologies coincide with the $\sigma$-weak, $\sigma$-strong, $\sigma$-strong$^*$ topologies, respectively.
\end{parts}
\end{prb}




\section{Spectral theorems}





\begin{prb}[Spectral measure]
Let $(\Omega,\cA)$ be a measurable space and $H$ a Hilbert space.
A \emph{projection-valued measure} on $\Omega$ for $H$ is a map $E:\cA\to B(H)$ with $E(\varnothing)=0$ such that $E(A)$ is a projection for every $A\in\cA$, and one of the following equivalent conditions hold:
\begin{enumerate}[(i)]
\item the set function $E_{x,y}:\cA\to\C:A\mapsto\<E(A)x,y\>$ is a complex measure on $\Omega$ for each $x,y\in H$.
\item the countable additivity holds, i.e.~$E(\bigsqcup_{i=1}^\infty A_i)=\sum_{i=1}^\infty E(A_i)$ in the strong operator topology of $B(H)$ for $(A_i)_{i=1}^\infty\subset\cM$.
\end{enumerate}
\begin{parts}
\item $E(A\cap B)=E(A)E(B)$ for $A,B\in\cM$.
\end{parts}
\end{prb}

% point spectrum, approximate point spectrum
Kato-Rellich theorem

For a densely defined closed operator $T:H\to H$, $\sigma(T^*)=\bar{\sigma(T)}$.

A multiplication operator by any Borel measurable function $\Omega\to\C$ always defines a densely defined closed normal operator.




\[-\]


\begin{prb}[Bounded Borel functions]
Let $\Omega$ be a compact Hausdorff space and denote by $B^\infty(\Omega)$ the space of bounded Borel functions on $\Omega$.
The linear combinations of projections in $B^\infty(\Omega)$ are called \emph{simple functions}.
\begin{parts}
\item There are natural inclusions $C(\Omega)\subset B^\infty(\Omega)\subset C(\Omega)^{**}$ among C$^*$-algebras. (Every bounded Borel function defines a bounded linear functional on $M(\Omega)$.)
\item $B^\infty(\Omega)$ is the norm closure of simple functions.
\item $B^\infty(\Omega)$ factors through all $L^\infty(\mu):=M(\pi_\mu)$ for GNS-representations $\pi_\mu$ of $C(\Omega)$.
\item The topology of pointwise bounded convergence on $B^\infty(\Omega)$ is stronger than the induced $\sigma$-weak topology. (It is the bounded convergence theorem.)
\end{parts}
\end{prb}

\begin{prb}[Borel functional calculus for bounded normal operators]
Let $x\in B(H)$ be a normal operator.
Consider
\[\begin{tikzcd}
C(\sigma(T))^{**} \ar[bend left]{rdd}{\tilde\pi}&&\\
B^\infty(\sigma(T))\uar[phantom,sloped,"\subset"]&&\\
C(\sigma(T)) \ar{r}{\pi}\uar[phantom,sloped,"\subset"] & W^*(T)\rar[phantom,"\subset"]& B(H).
\end{tikzcd}\]
\begin{parts}
\item If we endow the topology of pointwise bounded convergence on $B^\infty(\sigma(a))$, then the Borel functional calculus $\tilde\pi:B^\infty(\sigma(T))\to B(H)$ is strongly continuous.
\item Every von Neumann algebra is the norm closed span of projections.
\end{parts}
\end{prb}
\begin{pf}
(a)
By the bounded convergence theorem.

(b)
This is because $\sigma(a)\subset\C$ is compact so that it is separable and metrizable; every bounded measurable function is a pointwise limit of simple functions.
\end{pf}


\begin{prb}[Spectral representation]
A \emph{projection-valued measure} on a compact Hausdorff space $\Omega$ is nothing but a faithful non-degenerate representation $E:C(\Omega)\to B(H)$.
For a bounded normal operator $T\in B(H)$, there is a natural projection valued measure $\pi:C(\sigma(T))\to B(H)$, called the \emph{spectral measure}.
We now decompose $\pi=\bigoplus_\alpha\pi_\alpha$ to cyclic representations $\pi_\alpha:C(\sigma(T))\to B(H_\alpha)$ with cyclic unit vectors $\psi_\alpha$, which are not unique.
Each vector state $\psi_\alpha$ induces a probability measure $\mu_\alpha$ on $\sigma(T)$.
It is called the spectral measure associated to the cyclic vector $\psi_\alpha$.
We can check that the GNS-representation $C(\sigma(T))\to B(L^2(\mu_\alpha))$ of $\mu_\alpha$, also called a \emph{multiplication operator representation} of $C(\sigma(T))$, is unitarily equivalent to $\pi_\alpha$, so the direct sum $C(\sigma(T))\to\bigoplus_\alpha B(L^2(\mu_\alpha))$ of GNS representations can be defined.
Then, we can show the bounded normal operator $T$ is always unitarily equivalent to the multiplication operator on a finite measure space.

Multiplicity theory:
For a faithful non-degenerate representation $\pi$ of a separable abelian unital C$^*$-algebra $A$ on a separable (maybe?) Hilbert space, there is a unique canonical cyclic decomposition (up to unitary equivalence)
\[\pi\approx\bigoplus_{m=1}^\infty\pi_m^{\oplus m}:A\to\bigoplus_{m=1}^\infty B(L^2(\mu_m))^{\oplus m},\]
such that the sequence $\mu_m$ measures has disjoint supports.
Also we can show that if the measure classes of $\mu_m$, which corresponds to the equivalence classes of cyclic representations without cyclic vectors, are same, then two such representations are unitarily equivalent.
(I don't know the detailed proofs yet, for example, where to define support of a measure)
\end{prb}


To show the correspondence between the measure-theoretic spectral measure and the operator-algebraic spectral measure, note that a projection-valued measure defines a ``normal'' unital $*$-homomorphism
\[\spn P(B^\infty(X))\to B(H).\]
Then, mimick the definition of Lebesgue integral to construct a unital $*$-homomorphism $C(X)\to B(H)$.


\chapter{Unbounded operators}

\section{Densely defined and closed operators}


We almost always consider the domain of an unbounded linear operator as the union of all subspaces on which a given operator is continuously well-defined.
Between complete spaces, the subspaces may be assumed to be closed.

Densely defined operators can be seen as increasing limits of partially defined continuous linear operators.


For $X$ without condition and $Y$ normable, then the continuity of $T:X_\sigma\to Y_\sigma$ implies the boundedness of $T:X\to Y$ because if we have $x_i\to0$ and $Tx_i$ is not bounded, then the uniform boundedness principle on $Y^*$ proves $Tx_i$ does not coverges weakly to zero.


We want to realize the graph $\Gamma(T)$ as the strict inductive limit of Fr\'echet spaces $\Gamma(T_i)$ with barrelled $X_i$.
The topology on $\Gamma(T_i)$ may not come from the topology of $X_\sigma\times Y_\sigma$.
If so, by the closed graph theorem, $T_i:X_i\to Y$ are everywhere defined continuous linear operators.


Even if the weak topology on $X\times Y$ is not complete but its weakly closed subspace $\Gamma(T)$ can be seen as a Banach space.
Which topology is natural on the graph?
For closedness, weak topology is the most natural.


I think the most natural setting for densely defined closed operators is the Fr\'echet space.

\begin{prb}
Let $X$ and $Y$ be topological vector spaces.
A \emph{linear operator} from $X$ to $Y$ is a linear map $T:\dom T\to Y$, where $\dom T$ is a linear subspace of $X$.

\end{prb}

\begin{prb}
Let $X$ and $Y$ be Fr\'echet spaces.
For a closed operator $T:\dom T\subset X\to Y$, there is an increasing net $T_i:\dom T_i\subset X\to Y$ of closed operators such that $\dom T_i$ is closed and $\Gamma(T)=\bigcup_i\Gamma(T_i)$. (Consider the net of finite-dimensional subspaces)
Conversely, 

\begin{parts}
\item a
\end{parts}
\end{prb}

\begin{prb}[Adjoint operators]
Let $X$ and $Y$ be topological vector spaces.
Let $T:\dom T\subset X\to Y$ be a densely defined linear operator.
The \emph{adjoint} of $T$ is a linear operator $T^*:\dom T^*\subset Y^*_\sigma\to X^*_\sigma$ with domain
\[\dom T^*:=\{y^*\in Y^*\mid \dom T\to\C:x\mapsto\<Tx,y^*\>\text{ is continuous}\}\]
such that
\[\<x,T^*y^*\>:=\<Tx,y^*\>,\qquad x\in\dom T,\ y^*\in\dom T^*.\]
\begin{parts}
\item If $T\subset S$, then $S^*\subset T^*$.
\item $T^*:\dom T^*\subset Y^*_\sigma\to X^*_\sigma$ is always closed.
\item $T$ is closable if and only if $T^*$ is densely defined. If it is, then $T^{**}$ is the closure of $T$.
\item $T^*$ is injective if and only if $T$ has dense range, and surjective if and only if $T$ is an embedding.
\end{parts}
\end{prb}
\begin{pf}
Consider the dual pair $(X_\sigma\times Y_\sigma,Y_\sigma^*\times X_\sigma^*)$.
We claim that $\Gamma(T^*)=\Gamma(-T)^\perp$ with respect to this pairing.
One direction is clear by $\<(x,-Tx),(y^*,T^*y^*)\>=\<x,T^*y^*\>-\<Tx,y^*\>=0$ for all $x\in\dom T$ and $y^*\in\dom T^*$.
Conversely, if $(y^*,x^*)$ is contained in the right-hand side so that $0=\<(x,-Tx),(y^*,x^*)\>=\<x,x^*\>-\<Tx,y^*\>$, then the linear functional $\dom T\to\C:x\mapsto\<Tx,y^*\>$ is continuous by $\<Tx,y^*\>=\<x,x^*\>$, so we have $y^*\in\dom T^*$ and $x^*=Ty^*$ by definition of adjoint operators, hence the claim follows.

(a) Clear from the claim.

(b) It is because the complement $\Gamma(-T)^\perp$ is closed in $Y^*_\sigma\times X^*_\sigma$.

(c)
Suppose $T$ is closable.
If $y\in Y$ satisfies $\<y,y^*\>=0$ for every $y^*\in\dom T^*$, then the equation $\<(0,y),(y^*,-T^*y)\>=0$ implies $(0,y)\in\Gamma(-T^*)^\perp=\Gamma(T)^{\perp\perp}=\bar{\Gamma(T)}$, and the closability of $T$ says that $y=T0=0$, so $\dom T^*$ separates point of $Y$, which means that $\dom T^*$ is dense in $Y^*_\sigma$.
Conversely, if $T^*$ is densely defined, then we can define the double adjoint $T^{**}:\dom T^{**}\subset X\to Y$, which has the graph $\Gamma(T^{**})=\Gamma(-T^*)^\perp=\Gamma(T)^{\perp\perp}=\bar{\Gamma(T)}$, so $T$ has the closure $T^{**}$.

(d)
Suppose $T$ is bounded below.
Fix $x^*\in X^*$.
Since $T$ is bounded below, $x^*$ defines a bounded linear functional on $\dom T$ with respect to $\|x\|+\|Tx\|$, which is embedded in $Y$ as a closed subspace.
By the Hahn-Banach extension, we have an element $y^*\in Y^*$ such that $\<Tx,y^*\>=\<x,x^*\>$ for all $x\in X$, which is contained in $\dom T^*$ by the definition of $\dom T^*$.
This implies that $T$ is surjective because $T^*y^*=x^*$.

Conversely, suppose $T^*$ is surjective.
Let $F:=\{x\in\dom T:\|Tx\|\le1\}$.
Since for every $x^*\in X^*$ we have for some $y^*\in\dom T^*$ such that
\[\sup_{x\in F}|\<x,x^*\>|=\sup_{x\in F}|\<x,T^*y^*\>|=\sup_{x\in F}|\<Tx,y^*\>|\le\|y^*\|.\]
By the uniform boundedness principle, we have $\sup_{x\in F}(\|x\|+\|Tx\|)$ is bounded, we are done.
\end{pf}

\begin{prb}[Cores]
\end{prb}

\begin{prb}[Sum of unbounded operators]
\end{prb}

\begin{prb}[Composition of unbounded operators]
\end{prb}

\begin{prb}[Inverse of unbounded operators]
Let $T:\dom T\subset X\to Y$ be an injective linear operator.
\[\dom T^{-1}:=\ran T.\]
\end{prb}



\section{Symmetric and self-adjoint operators}


\begin{prb}[Symmetric operators]
Let $H$ be a Hilbert space.
A densely defined linear operator $T$ on $H$ is called \emph{symmetric} if $T\subset T^*$, equivalently,
\[\<Tx,y\>=\<x,Ty\>,\qquad x,y\in\dom T.\]
Let $T$ be a symmetric operator.
Since $\bar T=T^{**}\subset T^*$, we have no critical issue in taking closure.
If the closure of $T$ is self-adjoint, then it is called \emph{essentially self-adjoint}.
In general, instead of self-adjointness, it is easy to check a given linear operator is symmetric.
\begin{parts}
\item Every symmetric extension of $T$ is a restriction of $T^*$. In particular, $T$ has a maximal symmetric extension.
\item A maximal symmetric operator is closed.
\item A self-adjoint operator is maximal.
\item A symmetric operator is essentially self-adjoint if and only if it is indeed the unique self-adjoint extension if and only if the adjoint is symmetric.
\end{parts}
\end{prb}
\begin{pf}
(a) .
\end{pf}


\begin{prb}[Cayley transform]
There is a one-to-one correspondence between the unitary operators from $K_+$ to $K_-$, the deficiency subspaces.

If $T$ is a closed densely defined symmetric operator, then
\[Ux:=\begin{cases}0&\text{ if }x\in L^+,\\(T-i)(T+i)^{-1}x&\text{ if }x\in(L^+)^\perp,\end{cases}\]
is a partial isometry with initial and final spaces $(L^+)^\perp$ to $(L^-)^\perp$ such that $\dom T=(1-U)(L^+)^\perp$.
\begin{parts}
\item If $T$ is self-adjoint, then $1-U$ is injective and $\dom T=\ran(1-U)$.
\item The Cayley transform provides a one-to-one correspondence between self-adjoint operators $T$ and unitary operators $U$ satisfying $\ker(1-U)=0$.
\item
\end{parts}
\end{prb}


Let $T$ be a symmetric operator on a Hilbert space $H$.
We will always assume that $T$ is densely defined and closed.
We want to ask the following questions:
Is $T$ self-adjoint?
If not, does $T$ admit self-adjoint extensions?
Which self-adjoint extension generate the appropriate quantum dynamics?

\begin{ex*}
Let $T$ be a unbounded linear operator on $L^2([0,1])$ such that $Tf(x):=if'(x)$ for $f\in\dom T$, where
\[\dom T=C_c^\infty((0,1)),\]
which is symmetric.
The closure has the domain
\[\dom\bar T=H_0^1((0,1)),\]
which gives a closed symmetric operator.
The adjoint has the domain
\[\dom T^*=H^1((0,1))\subset C([0,1]),\]
which is not self-adjoint.
The family of self-adjoint extensions $\{T_\lambda\}$ can be parametrized by $\lambda\in\T$, where
\[\dom T_\lambda=\{f\in H^1((0,1)):\lambda f(0)=f(1)\}.\]



The same operator $S$ in the sense that $Sf(x):=if'(x)$
\[\dom S=\{f\in C^\infty((0,1)):\lim_{x\to0}f(x)=0\}\]
has no self-adjoint extension because we have deficiency indices $n^+=1$ and $n^-=0$ (maybe).
\end{ex*}








\begin{prb}[Non-negative symmetric operators]
Friedrichs extension.
Let $T$ be a non-negative symmetric operator on a Hilbert space $H$.
Consider an inner product $\<x,y\>_1:=\<(T+1)x,y\>$ on $\dom T$ and its completion $H_1$.
The inclusion $\dom T\subset H$ is extended to a bounded linear operator $\iota:H_1\to H$.
Since
\[\<x,y\>_1=\<(T+1)x,y\>=\<(T+1)x,\iota y\>=\<\iota^*(T+1)x,y\>_1,\qquad x,y\in\dom T\]
implies
\[x=\iota^*(T+1)x,\qquad x\in\dom T,\]
we have $\dom T=\ran(\iota^*(T+1))\subset\ran\iota^*$, so $\iota^*:H\to H_1$ has dense range and $\iota$ is injective.
Since $\iota$ is injective with dense range, $\iota\iota^*:H\to H$ is a bounded self-adjoint operator which is injective with dense range because
\[x=\iota x=\iota\iota^*(T+1)x,\qquad x\in\dom T,\] so we can define a self-adjoint operator $\tilde T:=(\iota\iota^*)^{-1}-1$ with $\dom\tilde T=\ran(\iota\iota^*)$.
We can check $\tilde T$ extends $T$ as
\[\tilde Tx=(\iota\iota^*)^{-1}x-x=(T+1)x-x=Tx,\qquad x\in\dom T.\]


Krein characterization.
\end{prb}



\section{Spectral theorems}

\begin{prb}[Strongly commuting operators]
Let $M$ be a von Neumann algebra on a Hilbert space $H$.
A closed densely defined operator $T$ on $H$ is said to be \emph{affiliated with} $M$ if 

\end{prb}



\begin{prb}[Borel functional calculus for normal operators]
Let $T$ be a normal unbounded operator on $H$.
Let $g\in\C(z,\bar z)$ be such that $g:\sigma(T)\to\C$ is an embedding to a bounded set so that $g(T)\in B(H)$.
For example, $g(z)=z/(1+|z|^2)$.
The continuous functional calculus for $g(T)$ gives
\[\Phi:C_0(\sigma(T))\to C(g(\sigma(T)))=C(\sigma(g(T)))\to B(H):f\mapsto f\circ g^{-1}\mapsto f\circ g^{-1}(g(T)).\]
We want to show $\Phi$ is independent of the choice of $g$.

Now obtain a normal extension of it to the universal von Neumann algebra of $C_0(\R)$, which contains $B^\infty(\R)$.
The restriction $B^\infty(\R)\to B(H)$ is the bounded Borel functional calculus for $T$.

For an unbounded Borel $f$ on $\sigma(T)$, there are two methods.
One method is the spectral truncation using $f1_{|f|\le n}$.
The other method is using $h\in\C(z,\bar z)$ be such that $h^{-1}:f(\sigma(T))\to\C$ is an embedding to a bounded set, $h(z)=z/(1-|z|^2)$ for example, and take the bounded Borel functional calculus with $h^{-1}\circ f$ and define $f(T):=h(h^{-1}\circ f(T))$.
We want to show $f(T)$ is independent of the choice of $h$.
\begin{parts}
\item 
\end{parts}
\end{prb}

\begin{prb}
Let $f\in B^\infty(\R)$.
\begin{parts}
\item If $f=1_{\C\setminus\{0\}}$, then $f(T)=s(T)$.
\item $f(T)$ is approximated in norm by projections: an argument on $\<E(\lambda)\xi,\xi\>$ works.
\item $f(VTV^*)=Vf(T)V^*$ for $V^*Vs(T)=s(T)V^*V=s(T)$. It follows from the commutative diagram
\[\begin{tikzcd}
C_0(\R)\ar[equals]{d}\ar{r}{\pi_T} & B(H) \ar{d}{V\cdot V^*}\\
C_0(\R)\ar{r}{\pi_{VTV^*}} & B(H).
\end{tikzcd}\]
\end{parts}
\end{prb}


\section{Infinitesimal generators}


\begin{prb}[Stone theorem]
\end{prb}

Cores and invaraint spaces?


\begin{prb}[Smooth and analytic vectors]
Cores
\begin{parts}
\item If $T$ is symmetric and $D_0$ is dense, then $T|_{D_0}$ is essentially self-adjoint.
\end{parts}
\end{prb}

\begin{prb}[Resolvent convergence]
\end{prb}





\section{Polar decomposition}

If $T:H\to H$, then for $T=V|T|$, $V$ is a partial isometry which connects from the complement of the kernel to the closure of the range as a unitary.
Same for unbounded operator.


\[\begin{tikzcd}
\coim T \rar{T} & \im T=\coim T^* \rar{T^*} & \im T^* \\
\ker T & \coker T=\ker T^* & \coker T^*
\end{tikzcd}\]

$T$ is normal then $\coim T=\im T$.

\begin{prb}[Support projections of operators]
Let $x$ be an element of a von Neumenna algebra $M$.
The \emph{left support projection} of $x$ is the minimal projection $p\in M$ such that $x=px$, denoted by $s_l(x)$.
The \emph{right support projection} of $x$ is defined as the left support projection of $x^*$.
The projections $s_l(x)$ and $1-s_r(x)$ are also called the \emph{range} and \emph{kernel} projections of $x$, respectively.

Riesz refinement?
\begin{parts}
\item Support projections of $x$ uniquely exist.
\item $x^*yx=0$ if and only if $s_l(x)ys_l(x)=0$ for every $y\in M$.
\item We have $s_r(x)=s_r(x^*x)=s_r(|x|)$. In particular, $s_l(x)=s_r(x)$ if $x$ is normal.
\item If $x^*x\le y^*y$, then there is a unique $v\in M$ such that $x=vy$ and $s_r(v)\le s_l(y)$.
\item There is unique $v\in M$ such that the polar decomposition $x=v|x|$ holds and that $s_r(x)=v^*v$. Moreover, $x^*=v^*|x^*|$ and $s_l(x)=vv^*$. In particular, $s_l(x)$ and $s_r(x)$ are Murray-von Neumann equivalent.
\end{parts}
\end{prb}
\begin{pf}
(a)
Let $x\in M$.
Since $\im x=\im(xx^*)^{\frac12}$, we may assume $0\le x\le1$.
Then, $x^{2^{-n}}$ is an increasing sequence in $M$ bounded by one, so it converges strongly to some $p\in M_+$.
We can check $p^2=p$ by...
We can check $p$ is the range projection of $x$ by...

(d)
Suppose $\id_H\in M\subset B(H)$.
The operator $v_0:\bar{yH}\to\bar{xH}:y\xi\mapsto x\xi$ is well defined because
\[\|x\xi\|^2=\<x^*x\xi,\xi\>\le\<y^*y\xi,\xi\>=\|y\xi\|^2.\]
Let $v:=v_0s_l(y)$.
Then, $x\xi=vy\xi$ for all $\xi\in H$.
If $v'\in B(H)$ satisfies $y=v'x$ and $v'=v's_l(y)$, then $y^*(v-v')^*(v-v')y=(x-x)^*(x-x)=0$ implies $0=s_l(y)(v-v')^*(v-v')s_l(y)=(v-v')^*(v-v')$, so $v$ is unique in $B(H)$.
If $u\in M'$ is unitary, then $uvu^*$ satisfies the same property $y=uvu^*x$ and $uvu^*=uvu^*s_l(y)$, so $uvu^*=v$.
Since unitaries span $M'$, we have $v\in M''=M$.

(e)
Since $x^*x\le|x|^*|x|$, there is a unique $v\in M$ such that $x=v|x|$ and $v=vs_l(|x|)=vs_r(x)$.
Then, $s_r(x)-v^*v=s_r(x)(1-v^*v)s_r(x)=0$ from $|x|(1-v^*v)|x|=|x|^2-|x|^2=0$, and $s_l(x)-vv^*=s_l(x)(1-vv^*)s_l(x)=0$ from $x^*(1-vv^*)x=|x|^2-|x|^2=0$.
The partial isometry $v$ is unique since $s_r(x)=v^*v$ implies $s_r(v)=s_r(v^*v)=s_r(s_r(x))=s_r(x)$.
Similarly, $s_l(v)=s_l(x)$.
The equality $xv^*=|x^*|$ follows from $xv^*=v|x|v^*\ge0$ and $|xv^*|^2=vx^*xv^*=v|x|^2v^*=xx^*=|x^*|^2$.
\end{pf}


\begin{prb}[Polar decomposition]
polar decomposition
polar decomposition of symmetric operator?
polar decomopsition changes spectrum or domains?

support projection
\end{prb}



\section{Decomposition of spectrum}

\begin{align*}
\sigma
&=\sigma_p\cup\sigma_c\cup\sigma_r\\
&=\sigma_{ess}\cup\sigma_d\\
&=\bar{\sigma_{pp}}\cup\sigma_{ac}\cup\sigma_{sc}.
\end{align*}


\[\sigma=\sigma_p\sqcup\sigma_c\sqcup\sigma_r=\bar{\sigma_{pp}}\cup\sigma_{ac}\sigma_{sc}=\sigma_d\sqcup\sigma_{ess,5}.\]





\section*{Exercises}


\begin{prb}[Strict topology]
Let $H$ be a Hilbert space.
Let $(T_\alpha)\subset B(H)$ and $K\in K(H)$.
\begin{parts}
\item The strong$^*$ topology and the strict topology agree on bounded sets of $B(H)$.
\end{parts}
\end{prb}

\begin{prb}[Unitary group]
Let $H$ be a Hilbert space.
\begin{parts}
\item The weak topology and the strict topology agree on $U(H)$.
\end{parts}
\end{prb}


\begin{prb}[Bounded increasing nets]
Let $T_\alpha$ be a bounded increasing net of bounded self-adjoint operators on $H$.
\begin{parts}
\item $T_\alpha$ converges strictly. In particular, $T_\alpha\to T$ strictly iff $T_\alpha\to T$ weakly.
\end{parts}
\end{prb}
\begin{pf}
Define $T$ such that
\[\<Tx,y\>:=\lim_\alpha\sum_{k=0}^3i^k\<T_\alpha(x+i^ky),x+i^ky\>.\]
The convergence is due to the monotone convergence in $\R$.
We can check it is a well-defined bounded linear operator by considering the bounded sesquilinear form.
Then, $T_\alpha\to T$ weakly by definition, and $\sigma$-strongly because the net is increasing.
\end{pf}




\begin{prb}[Distributional operators]
\begin{parts}
\item Every continuous linear operator $T:\cD(\R)\to\cD'(\R)$ naturally defines a closable densely defined operator $T:\dom T\to L^2(\R)$ with $\dom T:=\cD(\R)$.
\end{parts}
\end{prb}

\begin{prb}[Hydrogen atom]
For $V\in L^\infty(\R^d)$, the operator
\[H\psi(x):=-\frac{\hbar^2}{2m}\Delta\psi(x)-V(x)\psi(x),\qquad x\in\R^d\]
is called the \emph{Schr\"odinger operator}, and simply we write $H=-\Delta+V$.
The eigenvectors associated to the discrete spectrum is called \emph{bound eigenstates}.

Consider the Schr\"odinger operator $H:=-\Delta-|x|^{-1}$ on $L^2(\R^3)$.
We want to investigate the spectral decomposition of $H$ by diagonalization.
\begin{parts}
\item $H$ is self-adjoint.
\item $\sigma_d(H)=\{\}$
\end{parts}
\end{prb}

The orbital comes from the diagonalization of the Laplace-Beltrami operator on the unit sphere.

The periodic Schr\"odinger operator is diagonalized to the direct integral of elliptic operators defined on the Brillouin torus.



\chapter{Operator theory}
\section{Toeplitz operators}

invariant subspace problem
Beurling theorem
Hardy and Bergman and Bloch spaces
JB* triple





\part{Operator algebras}
\chapter{Banach algebras}

\section{Spectra of elements}

\begin{prb}[Banach algebras]
For a Banach algebra $A$ with multiplicative unit, there is a complete renorming such that $\|1\|=1$, i.e. we can always assume $\|1\|=1$.
It provides a definition of \emph{unital Banach algebra}.

Let $A$ be a unital Banach algebra.
\begin{parts}
\item If $\|a\|<1$, then $1-a$ is invertible. So $A^\times$ is open.
\item $A^\times\to A^\times:a\mapsto a^{-1}$ is continuous.
\item $A^\times\to A^\times:a\mapsto a^{-1}$ is differentiable.
\end{parts}
\end{prb}
\begin{pf}
(a)
We can show
\[(1-a)^{-1}=\sum_{k=0}^\infty a^k.\]
If $a$ is invertible, then $a+h=a(1+a^{-1}h)$ has the inverse $(1+a^{-1}h)^{-1}a^{-1}$ if $h$ is sufficiently small such that $\|a^{-1}h\|<1$.

(b)
Clear from
\[b^{-1}-a^{-1}=b^{-1}(a-b)a^{-1}.\]

(c)
\begin{align*}
\frac{\|b^{-1}-a^{-1}-(-a^{-1}(b-a)a^{-1})\|}{\|b-a\|}
&=\frac{\|(a^{-1}-b^{-1})(b-a)a^{-1}\|}{\|b-a\|}\\
&\le\|a^{-1}-b^{-1}\|\|a^{-1}\|\xrightarrow{b\to a}0.
\end{align*}
\end{pf}

\begin{prb}[Vector-valued complex analysis]
Let $X$ be a complex Banach space (it is known that Fr\'echet is also possible. See Rudin p.82).
If a function $f:\Omega\subset\C\to X$ on a domain is weakly holomorphic, i.e.~$f$ defines an operator $X^*\to\Hol(\Omega)$, then $f$ is clearly Bochner integrable on every contour $\gamma\subset\Omega$, and the Cauchy theorem and the Cauchy formula holds, and $f$ is strongly holomorphic, i.e.~complex differentible in norm.
\end{prb}


\begin{prb}[Spectrum and resolvent]
Let $a$ be an element of a unital Banach algebra $A$.
The \emph{spectrum} of $a$ in $A$ is defined to be the set
\[\sigma_A(a):=\{\lambda\in\C:\lambda-a\text{ is not invertible in }A\},\]
and the \emph{resolvent} of $a$ in $A$ is defined to be its complement $\rho_A(a):=\C\setminus\sigma_A(a)$.
We can now define the \emph{resolvent map} $R:\rho_A(a)\to A$ such that
\[R(\lambda)=R(\lambda;a):=(\lambda-a)^{-1}.\]
If $A$ is clear in its context, we omit it to just write $\sigma(a)$ and $\rho(a)$.
\begin{parts}
\item $\sigma(a)$ is compact.
\item $\sigma(a)$ is non-empty.
\item If $A$ is a division ring, then $A\cong\C$. This result is called the \emph{Gelfand-Mazur theorem}.
\end{parts}
\end{prb}
\begin{pf}
(a)
If $|\lambda|>\|a\|$, then $\lambda-a$ is always invertible, so the spectrum is bounded.
Closedness follows from the fact that the set of invertibles is open.

(b)
Suppose the spectrum $\sigma(a)=\varnothing$ so that the resolvent function $R:\C\to A$ is well-defined on the entire $\C$.
Note that $a\ne0$.
Since $R$ is continuous and since
\[\|(\lambda-a)^{-1}\|=\|\lambda^{-1}(1-\lambda^{-1}a)^{-1}\|
=\Bigl\|\lambda^{-1}\sum_{k=0}^\infty(\lambda^{-1}a)^k\Bigr\|
<(2\|a\|)^{-1}\sum_{k=0}^\infty2^{-k}=\|a\|^{-1}\]
on $\{\lambda\in\C:|\lambda|>2\|a\|\}$, the function $R$ is bounded.
Also, for every $l\in A^*$ we have that the function $\C\to\C:\lambda\mapsto\<R(\lambda),l\>$ is holomorphic since $a\mapsto a^{-1}$ is differentiable.

Therefore, by the Liouville theorem, the bounded entire function $\lambda\mapsto\<R(\lambda),l\>$ is constant for all $l\in A^*$.
Because $A^*$ separates points of $A$, the function $R$ is constant, which implies $a\in\C$ and contradicts to $\sigma(a)=\varnothing$.

(c)
For any $a\in A$, by the part (b), there must be $\lambda$ such that $\lambda-a$ is not invertible.
In a division ring, zero is the only non-invertible element, so $\lambda=a$.
\end{pf}

\begin{prb}[Spectral radius]
Let $a$ be an element of a unital Banach algebra $A$.
The \emph{spectral radius} of $a$ in $A$ is defined to be
\[r(a):=\sup_{\lambda\in\sigma(a)}|\lambda|.\]
\begin{parts}
\item $r(a)\le\inf_n\|a^n\|^{\frac1n}$.
\item $\limsup_n\|a^n\|^{\frac1n}\le r(a)$, i.e. $r(a)=\lim_n\|a^n\|^{\frac1n}$.
\end{parts}
\end{prb}
\begin{pf}
(a)
Since $(\lambda-a)^{-1}=\lambda^{-1}(1-\lambda^{-1}a)^{-1}$ exists if $|\lambda|>\|a\|$, we have $r(a)\le\|a\|$ for all $a\in\cA$.
For every $\lambda\in\sigma(a)$ and every integer $n\ge1$ we have
\[|\lambda|^n=|\lambda^n|\le r(a^n)\le\|a^n\|,\]
and it proves $r(a)\le\inf_n\|a^n\|^{\frac1n}$.

(b)
Consider a holomorphic function
\[f:\{\lambda\in\C:|\lambda|>r(a)\}\to\C:\lambda\mapsto\<R(\lambda),l\>\]
for each $l\in A^*$.
Since on a smaller domain $\{\lambda\in\C:|\lambda|>\|a\|\}$, the function $f$ can be given by
\[f(\lambda)=\Bigl\<\lambda^{-1}\sum_{k=0}^\infty(\lambda^{-1}a)^k,l\Bigr\>,\]
we can determine the coefficients of the Laurent series of $f$ at infinity as
\[f(\lambda)=\sum_{k=0}^\infty\<a^k,l\>\lambda^{-k-1}\]
on $\{\lambda\in\C:|\lambda|>r(a)\}$.

Take $\lambda$ such that $|\lambda|>r(a)$.
Then, the sequence $(a^k\lambda^{-k-1})_k\in\cA$ is weakly bounded, hence is bounded in norm by the uniform boundedness principle.
Let $\|a^n\|\le C_\lambda|\lambda^{n+1}|$ for all $n\ge1$.
Then,
\[\limsup_{n\to\infty}\|a^n\|^{\frac1n}\le\limsup_{n\to\infty}C_\lambda^{\frac1n}|\lambda^{n+1}|^{\frac1n}=|\lambda|.\]
If we limit $|\lambda|\downarrow r(a)$, we are done.
\end{pf}

\begin{prb}[Spectrum in closed subalgebras]
For fixed element, smaller the ambient algebra, less ``holes'' in the spectrum.
Let $A\subset B$ be a closed subalgebra containing $1_A$.
Note that $A$ may be unital even for $1_B\notin A$.
\begin{parts}
\item $B^\times$ is clopen in $A^\times\cap B$.
\end{parts}
\end{prb}





\section{Ideals}
\begin{prb}[Ideals]
\begin{parts}
\item If $I$ is a left ideal, then $A/I$ is a left $A$-module.
\end{parts}
\end{prb}

\begin{prb}[Modular left ideals]
A left ideal $I$ is called \emph{modular} if there is $e\in\cA$ such that $a-ae\in I$ for all $a\in A$.
The element $e$ is called a \emph{right modular unit} for $I$.
\begin{parts}
\item $I$ is modular if and only if $A/I$ is unital(?).
\item A proper modular left ideal is contained in a maximal left ideal.
\item $I$ is a maximal modular left ideal if and only if $I$ is a modular maximal left ideal.
\item There is a non-modular maximal ideal in the disk algebra.
\end{parts}
\end{prb}

\begin{prb}[Closed ideals]
\begin{parts}
\item closure of proper left ideal is proper left.
\item maximal modular left ideal is closed.
\end{parts}
\end{prb}


\begin{prb}[Unitization]
Let $A$ be an algebra.
Recall that we always assume algebras are associative.
Consider an embedding $A\to B(A):a\mapsto L_a$, where $L_a(b)=ab$.
Define
\[\tilde A:=\{\,L_a+\lambda\id_{B(A)}:a\in A,\lambda\in\C\,\}.\]
Note that this construction is available even for unital $A$.
\begin{parts}
\item If $A$ is normed, then $\tilde A$ is a normed algebra such that there is an isometric embedding $A\to\tilde A$.
\item If $A$ is Banach, then $\tilde A$ is a Banach algebra.
\item $A\oplus\C$ is topologically isomorphic to $\tilde A$ as normed spaces.
\end{parts}
\end{prb}
\begin{pf}
(a)
The space of bounded operators $B(A)$ is a normd algebra.
Then, $\tilde A$ is a normed $*$-algebra with induced norm
\[\|L_a+\lambda\id_{B(A)}\|=\sup_{b\in A}\frac{\|ab+\lambda b\|}{\|b\|}\]
Then, $A$ is a normed $*$-subalgebra of $\tilde A$ because the norm and involution of $A$ agree with $\tilde A$.

(b)
Suppose $(x_n,\lambda_n)$ is Cauchy in $\tilde A$.
Since $A$ is complete so that it is closed in $\tilde A$, we can induce a norm on the quotient $\tilde A/A$ so that the canonical projection is (uniformly) continuous so that $\lambda_n$ is Cauchy.
Also, the inequality $\|x\|\le\|(x,\lambda)\|+|\lambda|$ shows that $x_n$ is Cauchy in $A$.

Since a finite dimensional normed space is always Banach and $A$ is Banach, $\lambda_n$ and $x_n$ converge.
Finally, the inequality $\|(x,\lambda)\|\le\|x\|+|\lambda|$ implies that $(x_n,\lambda_n)$ converges.

(c)
Check the topology on $A\oplus\C$ in detail...
\end{pf}



unitization, homomorphisms, category(direct sum, product, etc.)

$B(\C^n)=M_n(\C)$ is simple, but $B(H)$ is not simple.

% approximate identity, norm of left multiplication


\section{Holomorphic functional calculus}


\begin{prb}[Holomorphic functional calculus]
Let $a$ be an element of a unital Banach algebra $A$.
Let $f$ be a holomorphic function on a neighborhood $U$ of $\sigma(a)$.
Let $\gamma$ be any positively oriented smooth simple closed curve in $U$ enclosing $\sigma(a)$.
Define $f(a)\in A$ by the Bochner integral
\[f(a):=\int_\gamma f(\lambda)(\lambda-a)^{-1}\,d\lambda.\]
Let $\Hol(\sigma(a))$ be the Fr\'echet algebra of all holomorphic functions on a neighborhood of $\sigma(a)$ endowed with the topology of compact convergence.
We define the \emph{holomorphic functional calculus} or the \emph{Dunford-Riesz calculus} by the map
\[\Phi:\Hol(\sigma(a))\to A:f\mapsto f(a).\]
\begin{parts}
\item $f(a)$ is independent of the choice of $\gamma$.
\item The functional calculus is an algebra homomorphism.
\item The functional calculus is bounded.
\item injective.
\item unital and $\id_\C\mapsto a$.
\item spectral mapping.
\item power series.
\end{parts}
\end{prb}
\begin{pf}
(a)


\end{pf}





\section{Gelfand theory}

Banach algebra of single generator
semisimplicity and symmetricity

\begin{prb}[Spectrum of a Banach algebra]
Let $A$ be a commutative Banach algbera.
A \emph{character} of $A$ is a non-trivial algebra homomorphism $\pi:A\to\C$.
Denote by $\sigma(A)$ the set of all characters of $A$ and endow with the weak$^*$ topology on $\sigma(A)\subset A^*$.
We call this space as the \emph{spectrum} of $A$.
\begin{parts}
\item If $A$ is unital, $\sigma(A)$ is contained in the unit sphere of $A^*$.
\item $\sigma(A)$ is locally compact and Hausdorff.
\end{parts}
\end{prb}
\begin{pf}

\end{pf}


\begin{prb}[Gelfand transform]
Let $A$ be a commutative Banach algebra.
The \emph{Gelfand transform} or the \emph{Gelfand representation} is the following algebra homomorphism
\[\Gamma:A\to C_0(\sigma(A)):a\mapsto(\pi\mapsto\pi(a)).\]
\begin{parts}
\item $\Gamma$ has the image separating points by definition.
\item $\Gamma$ has closed range if $A$ is a symmetric Banach $*$-algebra.
\item $\Gamma$ is injective if and only if $A$ is semisimple.
\item $\Gamma$ is isometric if and only if $r(a)=\|a\|$ for all $a\in A$.
\end{parts}
\end{prb}





\section*{Exercises}
\begin{prb}[Basic properties of spectrum]
Let $A$ be a unital algebra.
\begin{parts}
\item $\sigma(ab)\setminus\{0\}=\sigma(ba)\setminus\{0\}$.
\item If $\sigma(a)$ is non-empty, then $\sigma(p(a))=p(\sigma(a))$.
\end{parts}
\end{prb}
\begin{pf}
(a)
Intuitively, the inverse of $1-ab$ is $c=1+ab+abab+\cdots$.
Then, $1+bca=1+ba+baba+\cdots$ is the inverse of $1-ba$.
\end{pf}

$C_b(\Omega)$ $\ell^\infty(S)$ $L^\infty(\Omega)$ $B_b(\Omega)$ $A(\D)$
$B(X)$

\begin{prb}
In $C(\R)$, the modular ideals correspond to compact sets.
\end{prb}

\begin{prb}[Disk algebra]
\begin{parts}
\item Every continuous homomorphism is an evaluation.
\end{parts}
\end{prb}

\begin{prb}[Polynomial convexity]
(See Conway)
\end{prb}

\begin{prb}[Inclusion relation on spectra]
\begin{parts}
\item $\sigma(a+b)\subset\sigma(a)+\sigma(b)$ and $\sigma(ab)\subset\sigma(a)\sigma(b)$ for unital cases.
\item $\sigma(a^{-1})=\sigma(a)^{-1}$ for unital cases.
\item $r(a)^n=r(a^n)$.
\end{parts}
\end{prb}

\begin{prb}[Spectral radius function]
\begin{parts}
\item upper semi-continuous
\end{parts}
\end{prb}

\begin{prb}[Vector-valued complex function theory]
Let $\Omega$ be an open subset of $\C$ and $X$ a Banach space.
For a vector-valued function $f:\Omega\to X$, we say $f$ is \emph{differentiable} if the limit
\[\lim_{\lambda\to\lambda_0}(\lambda-\lambda_0)^{-1}(f(\lambda)-f(\lambda_0))\]
exists in $X$ for every $\lambda\in\Omega$, and \emph{weakly differentiable} if the limit
\[\lim_{\lambda\to\lambda_0}(\lambda-\lambda_0)^{-1}\<f(\lambda)-f(\lambda_0),x^*\>\]
exists in $\C$ for each $x^*\in X^*$ and every $\lambda\in\Omega$.
Then, the followings are all equivalent.
\begin{parts}
\item $f$ is differentiable.
\item $f$ is weakly differentiable.
\item For each $\lambda_0\in\Omega$, there is a sequence $(x_k)_{k=0}^\infty$ such that we have the power series expansion
\[f(\lambda)=\sum_{k=0}^\infty(\lambda-\lambda_0)^kx_k,\]
where the series on the right hand side converges absolutely and uniformly on any closed ball in $\Omega$ centered at $\lambda_0$.
\end{parts}
\end{prb}

\begin{prb}[Exponential of an operator]
\end{prb}







\chapter{C$^*$-algebras}

\section{Ccontinuous functional calculus}
% normal elements, real/imaginary part
\begin{prb}[$*$-algebras]
normed?
\end{prb}


\begin{prb}[C$^*$-identity]
%history
A \emph{C$^*$-algebra} is a Banach $*$-algebra $A$ satisfying the C$^*$-identity $\|a^*a\|=\|a\|^2$ for all $a\in A$.
\end{prb}


\begin{prb}[Unitization]
The \emph{unitization} or the \emph{Dorroh extension} of a C$^*$-algebra $A$ is...
\[(L_a+\lambda\id_{B(A)})^*=L_{a^*}+\bar\lambda\id_{B(A)}.\]
\end{prb}
\begin{pf}
The C$^*$-identity easily follows from the following inequality:
\begin{align*}
\|(a,\lambda)\|^2&=\sup_{\|b\|=1}\|ab+\lambda b\|^2\\
&=\sup_{\|b\|=1}\|(ab+\lambda b)^*(ab+\lambda b)\|\\
&=\sup_{\|b\|=1}\|b^*((a^*a+\lambda a^*+\bar\lambda a)b+|\lambda|^2y)\|\\
&\le\sup_{\|b\|=1}\|(a^*a+\lambda a^*+\bar\lambda a)b+|\lambda|^2b\|\\
&=\|(a,\lambda)^*(a,\lambda)\|.\qedhere
\end{align*}
\end{pf}




\begin{prb}[Gelfand-Naimark representation for C$^*$-algebras]
For a commutative C$^*$-algebra $A$, consider the Gelfand transform $\Gamma:A\to C_0(\sigma(A))$.
\begin{parts}
\item $\Gamma$ is a $*$-homomorphism.
\item $\Gamma$ is an isometry.
\item $\Gamma$ is a $*$-isomorphism.
\end{parts}
\end{prb}
\begin{pf}
(a)

(b)
Note that we have
\[\|\Gamma a\|=\sup_{\f\in\sigma(A)}|\Gamma a(\f)|=\sup_{\f\in\sigma(A)}|\f(a)|=r(a)\]
for all $a\in A$.
If we assume $a$ is self-adjoint, then since $\|a\|^2=\|a^*a\|=\|a^2\|$, the spectral radius coincides with the norm by the Beurling formula for spectral radius in Banach algebras:
\[\|\Gamma a\|=r(a)=\lim_{n\to\infty}\|a^{2^n}\|^{1/2^n}=\|a\|.\]
Hence we have for all $a\in A$ that
\[\|a\|^2=\|a^*a\|=\|\Gamma(a^*a)\|=\|(\Gamma a)^*(\Gamma a)\|=\|\Gamma a\|^2.\]

(c)
By the part (a) and (b), the image $\Gamma(A)$ is a closed unital $*$-subalgebra of $C(\sigma(A))$, and it separates points by definition.
Then, $\Gamma(A)$ is dense in $C(\sigma(A))$ by the Stone-Weierstrass theorem, which implies $\Gamma(A)=C(\sigma(A)$.
\end{pf}



\begin{prb}[Generators of a C$^*$-algebra]
joint spectrum.
\end{prb}


\begin{prb}[Continuous functional calculus]
Let $A$ be a unital C$^*$-algebra, and $a\in A$ a normal element.
Then, we have a $*$-isomorphism
\[C(\sigma(a))\to C^*(1,a):\id_{\sigma(a)}\mapsto a\]
defined by the inverse of the Gelfand transform, which we call the \emph{continuous functional calculus}.

\begin{parts}
\item spectral mapping: $\lambda\in\sigma_p(a)$ implies $f(\lambda)\in\sigma_p(f(a))$, $\lambda\in\sigma(a)$ iff $f(\lambda)\in\sigma(f(a))$, composition, ...
\end{parts}
\end{prb}




\begin{prb}[Normal elements]
Let $a$ be an element of a unital C$^*$-algebra $A$.
We say $a$ is \emph{normal}, \emph{unitary}, and \emph{self-adjoint} if $a^*a=aa^*$, $a^*a=aa^*=e$, and $a^*=a$ respectively.
For normality and self-adjointness, the definitions can be extended to non-unital C$^*$-algebras.
\begin{parts}
\item If $a$ is normal, then $a$ is unitary if and only if $\sigma(a)\subset\T$.
\item If $a$ is normal, then $a$ is self-adjoint if and only if $\sigma(a)\subset\R$.
\end{parts}
\end{prb}
\begin{pf}
(a)

(b)
We may assume $A$ is unital.
By the holomorphic functional calculus, we have
\[e^{ia}=\sum_{n=1}^\infty\frac{(ia)^n}{n!}\in A,\]
and the inverse of $e^{ia}$ is $e^{-ia}$.
Since the involution on $A$ is continuous, we can check $e^{ia}$ is unitary by
\[(e^{ia})^*=\sum_{n=1}^\infty\frac{(-ia)^n}{n!}=e^{-ia}.\]
For every $\f\in\sigma(A)$, then by the part (a) the equality
\[e^{-\Im\f(a)}=|e^{i\f(a)}|=|\f(e^{ia})|=1\]
proves $\f(a)\in\R$, hence $\sigma(a)\subset\R$.
\end{pf}

\begin{prb}[$*$-homomorphism]
Let $\f:A\to B$ be a $*$-homomorphism between C$^*$-algerbas.
\begin{parts}
\item $\f$ is determined by self-adjoint elements.
\item $\|\f\|=1$ if $\f$ is non-trivial.
\item If $\f$ has dense range, then it is surjective.
\item If $\f$ is injective, then it is an isometry.
\end{parts}
\end{prb}


\begin{prb}[Category of commutative C$^*$-algebras]
\[\begin{tikzcd}
& \mathrm{CH} \rar\dar[shift right, swap]{\text{disjoint base}} &
\mathrm{LCH} \rar &
\mathrm{cpltH} \\
\mathrm{LCH}_{\mathrm{prop}} \rar &
\mathrm{CH}_* \uar[shift right,swap]{\text{forgetful}} \uar[phantom]{\scriptscriptstyle\boldsymbol{\dashv}}
\end{tikzcd}\]

\[\begin{tikzcd}
& \mathrm{uCC^*Alg}_{\mathrm{unital}} \rar\dar[shift right, swap]{\text{inclusion}} &
\mathrm{CC^*Alg}_{\mathrm{mor}} \rar &
\mathrm{locCC^*Alg} \\
\mathrm{CC^*Alg}_{\mathrm{nondeg}} \rar &
\mathrm{CC^*Alg} \uar[shift right,swap]{\text{unitization}} \uar[phantom]{\scriptscriptstyle\boldsymbol{\vdash}}
\end{tikzcd}\]
\end{prb}


\section{States}


\begin{prb}[Positive elements]
Let $a,b$ be elements of a C$^*$-algebra $A$.
We say $a$ is \emph{positive} and write $a\ge0$ if it is normal and $\sigma(a)\subset\R_{\ge0}$.
If we define a relation $a\le b$ as $b-a\ge0$, then we can see that it is a partial order on $A$.
\begin{parts}
\item $a\ge0$ if and only if $\|\lambda-a\|\le\lambda$ for some $\lambda\ge\|a\|$.
\item If $a\ge0$ and $\sigma(b)\subset\R_{\ge0}$, then $\sigma(a+b)\subset\R_{\ge0}$.
\item $a\ge0$ if and only if $a=b^*b$ for some $b\in A$.
\end{parts}
\end{prb}
\begin{pf}
(c)
If $a\ge0$, then let $b:=a^{\frac12}$.
Conversely, we prove $b^*b\ge0$.
Let $c:=b(b^*b)_-$.
Observe that $c^*c=-(b^*b)_-^3\le0$.
Since $\sigma(c^*c)=\sigma(cc^*)$, we have $cc^*\le0$ and $c^*c+cc^*\le0$.
However, $c^*c+cc^*=2(\Re c)^2+2(\Im c)^2\ge0$.
Thus we have $c=\Re c+i\Im c=0$, which implies $(b^*b)_-=-(c^*c)^{\frac13}=0$.
\end{pf}

% Absolute value of an operator



\begin{prb}[Operator monotone operations]
\begin{parts}
\item If $0\le a\le b$, then $a^{-1}\ge b^{-1}$.
\item If $a\le b$, then $cac^*\le cbc^*$.
\end{parts}
\end{prb}





\begin{prb}[Approximate units]
Let $M(A)$


Let $I:=\{a\in A^+:\|a\|<1\}$.
Then,
for $a\ge0$,
\[\|ae_i-a\|^2=\|a(1-e_i)^2a\|^2\le\|a(1-e_i)a\|\]
If we let $a_n:=na(1+na)^{-1}$, then there is $i$ such that $e_i\ge a_n$, so $\|a(1-a_n)a\|\le n^{-1}\|a\|\to0$ implies $\|ae_i-a\|^2\to0$.

\begin{parts}
\item Exists.
\item 
\item 
\item separable.
\end{parts}
\end{prb}




\begin{prb}[Positive linear functionals]
Let $A$ be a C$^*$-algebra.
A \emph{state} of $A$ is a positive linear functional $\omega$ such that $\|\omega\|=1$.
\begin{parts}
\item For $\omega\in A^*$, $\omega$ is positive if and only if $\omega(e_i)\to\|\omega\|$.
\item If $\omega_0:V\to\C$ on a closed subspace $1\in V\subset A$ such that $\omega_0(1)=1$ and $\|\omega_0\|=1$, then $\omega_0$ is extended to a state of $A$.
\item For a normal element $a\in A$ there is a state $\omega$ such that $|\omega(a)|=\|a\|$.
\item A self-adjoint linear functional is the difference of two positive linear functional. It is called the \emph{Jordan decomposition}.
\end{parts}
\end{prb}
\begin{pf}
(a)


(b)
We may assume $A$ is unital.
If we have $\omega_0\in\spn\{1,a\}^*_1$ such that $\omega_0(1)=1$and $|\omega_0(a)|=\|a\|$, then the norm-preserving Hahn-Banach extension gives a desired state $\omega$.

Since $\sigma(a)\cup\{0\}$ is compact, there is $\lambda\in\sigma(a)$ such that $|\lambda|=\|a\|$.
The Dirac measure $\delta_\lambda$ induces a state $\omega_0$ of $C^*(1,a)$ such that $\omega_0(a)=\lambda$.
By the Hahn-Banach extension, there is extension $\omega\in A^*$ of $\omega_0$ with $\|\omega\|=1$.
Since $\omega(1)=\omega_0(1)=1$, $\omega$ is positive by the part (a), and $|\omega(a)|=|\omega_0(a)|=|\lambda|=\|a\|$.

(c)
We first show the real dual $(A^{sa})^*$ can be identified with the self adjoint part $(A^*)^{sa}$ of the complex dual.
By this identification, we can describe the weak$^*$ topology on $(A^*)^{sa}$ as $\sigma((A^*)^{sa},A^{sa})$.

We may assume $A$ is unital.
The closed unit ball of the real Banach space $(A^*)^{sa}$ is weakly$^*$ compact.
We are enough to show
\[(A^*)^{sa}_1=\bar\conv(S(A)\cup-S(A)),\]
where the closure is taken in the weak$^*$ topology, because $S(A)$ and $-S(A)$ are weakly$^*$ compact and convex due to the unit of $A$, the closure on the right-hand side is not necessary.
Suppose not and take $l\in(A^*)^{sa}_1$ which is not approximated weakly$^*$ by $\conv(S(A)\cup-S(A))$.
By the Hahn-Banach separation, there is $a\in A^{sa}$ such that
\[\sup_{\omega\in S(A)\cup-S(A)}\omega(a)<l(a).\]
If we take $\omega\in S(A)$ such that $|\omega(a)|=\|a\|$ using the part (b), then we get a contradiction to the bound $\|l\|\le1$.


\end{pf}






\section{Representations}


\begin{prb}[Non-degenerate representations]
Let $A$ be a C$^*$-algebra.
A \emph{representation} of $A$ on a Hilbert space $H$ is a $*$-homomorphism $\pi:A\to B(H)$.
We say a representation $\pi:A\to B(H)$ is \emph{non-degenerate} if $\pi(A)H$ is dense in $H$.
\begin{parts}
\item Every representation has a unique non-degenerate subrepresentation.
\item The following statements are equivalent:
\begin{enumerate}[(i)]
\item $\pi$ is non-degenerate.
\item For each $\xi\in H$ there is $a\in A$ such that $\pi(a)\xi\ne0$.
\item $\pi(e_i)\to1$ strongly for an approximate identity $e_i$ of $A$.
\end{enumerate}
\end{parts}
\end{prb}

\begin{prb}[Cyclic representations]
\emph{cyclic} if there is a vector $\psi\in H$ such that $A\psi$ is dense in $H$.
Cyclic decomposition
\end{prb}


\begin{prb}[Irreducible representations]
\emph{irreducible} if there is no proper closed subspace $K\subset H$ such that $\pi(A)K\subset K$.
The following statements are equivalent:
\begin{enumerate}[(i)]
\item $\pi$ is irreducible.
\item $\pi(A)'=\C$.
\item $\pi(A)$ is strongly dense in $B(H)$.
\item Every non-zero vector in $H$ is cyclic.
\end{enumerate}
\end{prb}


\begin{prb}[Gelfand-Naimark-Segal representation]
Let $A$ be a C$^*$-algebra, and $\omega$ be a state on $A$.
The \emph{left kernel} of $\omega$ is defined to be
\[\fn_\omega:=\{a\in A:\omega(a^*a)=0\}.\]
\begin{parts}
\item $\fn_\omega$ is a left ideal of $A$.
\item $\<a+\fn_\omega,b+\fn_\omega\>:=\omega(b^*a)$ is an inner product on $A/\fn_\omega$.
\item There is a unique representation $\pi_\omega:A\to B(H_\omega)$ such that $\pi_\omega(a)(b+\fn_\omega):=ab+\fn_\omega$ for $a,b\in A$.
\item $\pi_\omega:A\to B(H_\omega)$ is a cyclic representation.
\end{parts}
\end{prb}




\section{Ideals}

pure states, irreducible representations, primitive ideals.
\[(\mathrm{PS}(A),\hat A,\mathrm{Prim}(A))\]

For a short exact sequence
\[\begin{tikzcd}[column sep=small]0\rar&I\rar&A\rar&B\rar&0\end{tikzcd},\]
we have
\[\begin{tikzcd}[row sep=small]
PS(I) \dar[->>]\rar[hook] & PS(A) \dar[->>] & PS(B) \dar[->>]\lar[hook'] \\
\hat I \dar[->>]\rar[hook,shorten >= 10, shorten <= 10] & \hat A \dar[->>] & \hat B \dar[->>]\lar[hook',shorten >= 10, shorten <= 10] \\
\Prim(I) \rar[hook,swap]{\text{open}} & \Prim(A) & \Prim(B) \lar[hook']{\text{closed}}
\end{tikzcd}\]

We have to understand C$^*$-algebras in the context of homotopy theory, so the pointed topological spaces must be considered.
An open set $U$ of a locally compact Hausdorff space $X$ should be recognized as the quotient space $(X,x)/(A,x)$, where $x\notin U=A^c$, hence the ideal $A(U)$ corresponds and the restriction $A(X)\to A(U)$ does not make sense.
In other words, \textbf{$A(U)$ is not an analogue of $\cO_X(U)$, but of $\cI_{(X,X\setminus U)}$}.
It is fortunate that the kernel of the restriction, an ideal, can be recognized as the function algebra of the complement, which is not the case in algebraic geometry...? (Can define the quotient $X/A$ for an analytic subset of a complex space $X$?)

Then, how can we understand the sheaf theoretic restriction on an open set in operator algebras?
How about Banach or Fr\'echet algebras?
Can we consider a ``rigid'' Zariksi topology on the spectrum? (Closed sets in C$^*$-context are too flaccid)




\begin{prb}[Modular maximal left ideals]
\end{prb}

\begin{prb}[Primitive ideals]
hull kernel topology
\[PS(A)\cong\{(\pi,\psi)\}/\sim_u,\qquad\hat A\cong\{\pi\}/\sim_u.\]

\[\begin{array}{c|ccc}
A & PS(A) & \hat A & \Prim(A) \\\hline
C(X) & X & X & X \\
K(H) & PH & * & * \\
\tilde K(H) & ? & ? & \{0,K(H)\} \\
B(H) &&&
\end{array}\]
\begin{parts}
\item $\Prim(A)$ is locally compact T$_0$ space.
\item Two maps $PS(A)\to\hat A\to\Prim(A)$ are continuous surjective open maps
\item If $A$ is type I, then $\hat A\to\Prim(A)$ is an homeomorphism.
\end{parts}

\end{prb}




Every morphism $A\to M(B)$ induces the following?:
\[\begin{tikzcd}
PS(B) \rar[->>]\dar & \hat B \rar[->>]\dar & \Prim(B) \dar \\
PS(A) \rar[->>] & \hat A \rar[->>] & \Prim(A).
\end{tikzcd}\]





\section*{Exercises}

% Basic examples
%  C(X), C_0(X)
%  M_n(\C)
%  B(H), K(H), Q(H)

% Schroder-Burnstein thm of representations


\begin{prb}[Projections in $M_2(\C)$]
The space of self-adjoint elements in $M_2(\C)$ is a real vector space spanned by
\[1=\begin{pmatrix}1&0\\0&1\end{pmatrix},\qquad p:=\begin{pmatrix}1&0\\0&0\end{pmatrix},\qquad q:=\frac12\begin{pmatrix}1&1\\1&1\end{pmatrix}.\]
\begin{parts}
\item $(p-q)^2=\frac12$.
\item If we let $\lambda_\pm$ be the eigenvalues of $ap+bq$, then $\lambda_++\lambda_-=a+b$ and $\lambda_+-\lambda_-=\sqrt{a^2+b^2}$.
\item Every functional calculus $f(x)$ of self-adjoint $x$ is a linear combination of $x$ and 1.
\item $ap+bq+c\ge0$ if and only if $a+b+2c\ge\sqrt{a^2+b^2}$.
\item Every projection of rank one is given by $ap+bq+(1-a-b)/2$ for $a^2+b^2=1$.
\end{parts}
\end{prb}

\begin{prb}[Operator monotone square]
Let $A$ be a C$^*$-algebra in which the square function is operator monotone, that is, $0\le a\le b$ implies $a^2\le b^2$ for any positive elements $a$ and $b$ in $A$.
We are going to show that $A$ is necessarily commutative.
Let $a$ and $b$ denote arbitrary positive elements of $A$.
\begin{parts}
\item
Show that $ab+ba\ge0$.
\item
Let $ab=c+id$ where $c$ and $d$ are self adjoints.
Show that $d^2\le c^2$.
\item
Suppose $\lambda>0$ satisfies $\lambda d^2\le c^2$.
Show that $c^2d^2+d^2c^2-2\lambda d^4\ge0$.
\item
Show that $\lambda(cd+dc)^2\le(c^2-d^2)^2$.
\item
Show that $\sqrt{\lambda^2+2\lambda-1}\cdot d^2\le c^2$ and deduce $d=0$.
\item
Extend the result for general exponent: $A$ is commitative if $f(x)=x^\beta$ is operator monotone for $\beta>1$.
\end{parts}
\end{prb}


\begin{prb}[States on unitization]
Let $A$ be a non-unital C$^*$-algebra and $\tilde A$ be its unitization.
Let $\tilde\omega=\omega\oplus\lambda$ be a bounded linear functional on $\tilde A$, where $\omega\in A^*$ and $\lambda\in\C^*=\C$.

Since $A$ is hereditary in $\tilde A$, the extension defines a well-defined injective map $S(A)\to S(\tilde A)$.
We can identify $PS(A)$ as a subset of $PS(\tilde A)$ whose complement is a singleton.
\begin{parts}
\item $\tilde\rho$ is positive if and only if $\lambda\ge0$ and $0\le\rho\le\lambda$.
\item $\tilde\omega$ is a state if and only if $\lambda=1$ and $0\le\omega\le1$.
\item $\tilde\omega$ is a pure state if and only if $\lambda=1$ and $\omega$ is either a pure state or zero.
\end{parts}
\end{prb}


\begin{prb}[Representations of $C_0(X)$]
Let $A=C_0(X)$ and $\mu$ be a state on $A$, a regular Borel probability measure on a locally compact Hausdorff space $X$.
\begin{parts}
\item The left kernel of $\mu$ is $N_\mu=\{\,f\in A:f|_{\supp\mu}=0\,\}$.
\item $H_\mu=L^2(X,\mu)$.
\item The canonical cyclic vector is the unity function on $X$.
\end{parts}
\end{prb}

\begin{prb}[Representations of $K(H)$]
\end{prb}

\begin{prb}[Automorphism group of $K(H)$ and $B(H)$]
\end{prb}


\begin{prb}[Approximate eigenvectors]
\end{prb}


\begin{prb}[Kadison transitivity theorem]
\end{prb}

\begin{prb}[Hereditary C$^*$-algebras]
\end{prb}

\begin{prb}[Extreme points of the ball]
Let $A$ be a C$^*$-algebra and let $B_A$ be the closed unit ball of $A$.
\begin{parts}
\item Extreme points of $A_+\cap B_A$ is the projections in $A$.
\item Extreme points of $A_{sa}\cap B_A$ is the self-adjoint unitaries in $A$.
\item Every extreme point of $B_A$ is a partial isometry.
\end{parts}
\end{prb}

\section*{Problems}
\begin{enumerate}
\item* A C$^*$-algebra is commutative if and only if a function $f(x)=x(1+x)^{-1}$ is operator subadditive.
%L\"owner-Heinz inequality
\end{enumerate}







\chapter{Von Neumann algebras}

\section{Normal states}

\begin{prb}[Von Neumann algebras]
A \emph{von Neumann algebra} on a Hilbert space $H$ is a weakly closed $*$-subalgebra $M$ of $B(H)$ containing the identity operator.
We will see later that a $*$-subalgebra of $B(H)$ is weakly closed if and only if it is $\sigma$-strongly$^*$ closed.
A linear map between von Neumann algebras is called \emph{normal} if it is continuous between $\sigma$-weak topologies.
\begin{parts}
\item Every weakly closed $*$-subalgebra of $B(H)$ has a unit $p\in P(H)$.
\item A positive map $\f$ between von Neumann algebras is order continuous if and only if it is normal
\item The image of normal $*$-homomorphism is weakly closed.
\end{parts}
\end{prb}

\begin{prb}[Normal states]
Let $N\subset M$ be von Neumann algebras on a Hilbert space $H$.
The space of $\sigma$-weakly continuous linear functionals on $M$ is denoted by $M_*$.
\begin{parts}
\item $M_*$ is a predual of $M$.
\item The restriction of a normal state of $M$ on $N$ is normal.
\item A normal state of $N$ is extended to a normal state of $M$.
\item A state $\omega$ of $M$ is normal if and only if $\omega(x)=\sum_{i=1}^\infty\<x\xi_i,\xi_i\>$ for some $(\xi_i)\in\ell^2(\N,H)$.
\item $M_*$ is a closed subspace of $M^*$.
\end{parts}
\end{prb}


\begin{prb}[Support projections]
Let $M$ be a von Neumann algebra on a Hilbert space $H$.
$p$, $Mp$, $pM_*$.


\begin{parts}
\item projections of $M$, $\sigma$-closed left ideals of $M$, closed right invariant subspaces of $M_*$.
\item For a normal state $\omega$ of $M$, its left hull kernel $\fn_\omega:=\{x\in M:\f(x^*x)=0\}$ is a $\sigma$-weakly closed left ideal of $M$, so there is a projection $s(\omega)$ such that $\fn_\omega=Ms(\omega)$. This projection is called the \emph{support} or the \emph{carrier} projection of $\omega$.
\end{parts}
\end{prb}
\begin{pf}
Let $\fn$ be a $\sigma$-strongly$^*$ closed left ideal of $M$.
Then, $\fa:=\fn^*\cap\fn$ is a $\sigma$-strongly$^*$ closed $*$-subalgebra of $M$, whose unit is a projection $p\in M$.
If $xp\in Mp$, since $p\in\fn$ and $\fn$ is a left ideal, we have $xp\in\fn$.
Conversely, if $x\in\fn$, then $x^*x\in\fa$ implies $|x|\in\fa$ so that $|x|=|x|p$ since $p$ is the unit of $\fa$, and by the polar decomposition $x=v|x|$, we have $x=v|x|=v|x|p\in Mp$.
Therefore, $\fn=Mp$.
If two projections $p$ and $q$ in $M$ satisfy $Mp=Mq$, then since there is a unique unit in a $\sigma$-strongly$^*$ closed $*$-algebra $pMp=qMq$, hence $p=q$ and the uniqueness follows.

\end{pf}



\begin{prb}[Normal cyclic representations]
\begin{parts}
\item The GNS representation of a normal state is normal.
\item faithful normal states
\item Every normal state is a vector state.
\item sufficiently large representation, dependence of weak and strong topologies
\end{parts}
\end{prb}


An action admits a separating vector if and only if it admits a cyclic separating vector, which is equivalent to that the action can be realized as a cyclic representation associated to a faithful normal state, so every normal state is a vector state by the Connes cocycle.

\section{Density theorems}



\begin{prb}[Double commutant theorem]
Let $H$ be a Hilbert space.
The \emph{commutant} of a subset $A\subset B(H)$ is the von Neumann algebra $A'$ on $H$ consisting of all elements of $B(H)$ that commute every $a\in A$.
Let $A$ be a non-degenerate $*$-subalgebra of $B(H)$.
One can describe the von Neumann algebra generated by $A$ in $B(H)$ purely algebraically in terms of commutants as follows.
\begin{parts}
\item If $x\in A''$, for any $\e>0$ and $\xi\in H$ there is $a\in A$ such that $\|(x-a)\xi\|<\e$.
\item $A''$ is the $\sigma$-strong$^*$ closure of $A$.
\end{parts}
\end{prb}
\begin{pf}
(a)
Let $\xi\in H$ and let $p$ be the projection onto $\bar{A\xi}$.
We claim $px\xi=x\xi$.
For fixed $a\in A$, since $ap\eta$ and $a^*p\eta$ are in $\bar A\xi$ for any $\eta\in H$, we have $pap=ap$ and $pa^*p=a^*p$, which implies $ap=pap=(pa^*p)^*=(a^*p)^*=pa$, hence $p\in A'$.
Thus $xp=px$ for $x\in A''$.
On the other hand, observe that $a^*(1-p)\xi=(1-p)a^*\xi=a^*\xi-pa^*\xi=0$ for all $a\in A$.
Then, $\<(1-p)\xi,a\eta\>=0$ for any $\eta\in H$, and the non-degeneracy of $A$ implies $p\xi=\xi$.
Combining $xp=px$ and $p\xi=\xi$, we obtain $px\xi=xp\xi=x\xi$.

(b)
Since $A''$ is weakly closed and $A$ is self-adjoint, it suffices to show $A$ is $\sigma$-strongly dense in $A''$.
Consider the diagonal map $\Delta:B(H)\to B(\ell^2(\N,H))$, which is a injective unital normal $*$-homomorphism.
Then, $\Delta(A)$ is non-degenerate $*$-subalgebra of $B(\ell^2(\N,H))$.
We can check that $\Delta(A'')=\Delta(A)''$.
By applying the part (a) for arbitrary $\xi\in\ell^2(\N,H)$, we deduce the desired result.
\end{pf}


\begin{prb}[Kaplansky density theorem]

We say a continuous function $f:F\to\C$ on a closed set $F\subset\C$ is \emph{strongly continuous} if the functional calculus $x\mapsto f(x)$ is 
If $f$ is a 


A $*$-isomorphism between von Neumann algebras is normal.

\end{prb}



\begin{prb}[Approximate units for von Neumann algebras]
Let $M$ be a von Neumann algebra on a Hilbert space $H$.
Let $A$ be a $\sigma$-weakly dense $*$-subalgebra.
\begin{parts}
\item There is a net $e_i\in A_1^+$ such that $e_i\to1$ $\sigma$-strongly$^*$.
\item If either $A$ is hereditary in the sense that $AMA\subset A$ or $M$ is countably decomposable, then we may assume $e_i\uparrow1$.
\end{parts}
\end{prb}



\section{Borel functional calculus}

Let $A$ be a C$^*$-algebra.
For every weakly$^*$ dense subspace $F$ of $A^*$, we have a weakly$^*$-dense injection $A\to F^*$.
When can the $*$-algebra structure be extended to $F^*$ using the weak$^*$ topology?
This is related to that a von Neumann algebra has the $\sigma$-strong$^*$ topology.




\begin{prb}[Sherman-Takeda theorem]
Let $A$ be a C$^*$-algebra.
The bidual $A^{**}$ is called the \emph{enveloping von Neumann algebra} of a C$^*$-algebra $A$.
\begin{parts}
\item $A^{**}$ is a von Neumann algebra on a Hilbert space such that the dual of the canonical embedding $A\to A^{**}$ is an isometric isomorphism $(A^{**})_*\to A^*$.
Such a $*$-algebra is unique.
\item $A^{**}$ enjoys a universal property in the sense that every $*$-homomorphism $\f:A\to N$ to a von Neumann algebra $N$ has a unique normal extension $\tilde\f:A^{**}\to N$ of $\f$.
\end{parts}
\end{prb}
\begin{pf}
(a)
Let $\pi_u:A\to B(H_u)$ be the universal representation of $A$ constructed as the direct sum of all the GNS-representations of states of $A$, and let $M:=\pi_u(A)''$.
Consider the following adjoint maps
\[\pi_u:A\to M_{\sigma w},\qquad\pi_u^*:M_*\to A^*,\qquad\pi_u^{**}:A^{**}\to M.\]

Since
\[\|\pi_u^*(\omega)\|=\sup_{\substack{\|a\|\le1\\a\in A}}|\omega(\pi_u(a))|=\sup_{\substack{\|x\|\le1\\x\in M}}|\omega(x)|=\|\omega\|,\qquad \omega\in M_*\]
by the Kaplansky density theorem and the normality of $\omega$, $\pi_u^*$ is an isometry.
In fact, the same holds for any non-degenerate representations.

Let $\omega$ be a state of $A$.
Since the universal representation $\pi_u$ has the GNS representation of $\omega$ as a subrepresentation, $\omega$ is given by a vector state in $\pi_u$, which means that it gives rise to a normal state of $M$ which extends $\omega$ via $\pi_u$.
Then, the Jordan decomposition can be applied to verify that every bounded linear functional of $A$ has a normal extension on $M$, so $\pi_u^*$ is surjective.

Now the existence of a von Neumann algebra structure via the isometric isomorphism $\pi_u^{**}:A^{**}\to M$.
The uniqueness $*$-algebra structure follows from the double commutant theorem and Kaplansky density.

(b)
We can define $\tilde\f$ as the bitranspose of $\f:A\to N_{\sigma w}$, and it is a unique extension because $A$ is $\sigma$-weakly dense in $A^{**}$.
\end{pf}




\section{Predual}



\begin{prb}[Conditional expectations]
Let $A$ be a closed subalgebra of a C$^*$-algebra $B$.
Let $\f:B\to A$ be a contractive idempotent surjective linear map.
Such a map is called a \emph{conditional expectation}.
\begin{parts}
\item $\f$ is an $A$-bimodule map.
\item $\f$ is completely positive.
\end{parts}
\end{prb}

\begin{pf}
Since each conclusion of (a) and (b) still holds for restriction, we may assume $A$ and $B$ are von Neumann algebras by thinking of the bitranspose $\f^{**}:B^{**}\to A^{**}$.

(a)
Since the linear span of projections is $\sigma$-weakly dense in a von Neumann algebra, we are enough to show $p\f(b)=\f(pb)$ and $\f(bp)=\f(b)p$ for any projection $p\in A$.

Let $p\in A$ be a projection and let $b\in B$.
Note that the surjectivity of $\f$ implies that $p\f$ is also idempotent.
Then, where $1=1_B$,
\begin{align*}
(1+t)^2\|p\f((1-p)b)\|^2
&=\|p\f((1-p)b)+tp\f(p\f((1-p)b))\|^2\\
&\le\|(1-p)b+tp\f((1-p)b)\|^2\\
&=\|(1-p)b\|^2+t^2\|p\f((1-p)b)\|^2
\end{align*}
implies $p\f((1-p)b)=0$ by letting $t\to\infty$.
Putting $1_A-p$ and $1_A$ instead of $p$, we obtain
\[(1-p)\f((1-1_A+p)b)=0,\qquad\f((1-1_A)b)=0\]
respectively, which imply $(1-p)\f(pb)=0$.
Hence for any $b\in B$ we have
\[p\f(b)=p\f(pb)=\f(pb).\]
Similarly we can show $\f(b(1-p))p=0$ and $\f(bp)(1-p)=0$ for $b\in B$, we are done.

(b)
Let $[b_{ij}]\in M_n(B)_+$.
Let $\pi:A\to B(H)$ be a cyclic representation with a cyclic vector $\psi$.
Then, $[\xi_i]\in H^n$ can be replaced to $[\pi(a_i)\psi]$, so we can check the positivity of inflations $\f_n$ as
\[\sum_{i,j}\<\pi(\f(b_{ij}))\pi(a_j)\psi,\pi(a_i)\psi\>=\<\pi(\f(\sum_{i,j}a_i^*b_{ij}a_j))\psi,\psi\>\ge0,\]
because it follows $\sum_{i,j}a_i^*b_{ij}a_j\ge0$ by the positivity of $b_{ij}$ from
\[\<\pi_B(\sum_{i,j}a_i^*b_{ij}a_j)\xi,\xi\>=\sum_{i,j}\<\pi_B(b_{ij})\pi_B(a_j)\xi,\pi_B(a_i)\xi\>\ge0,\]
where $\pi_B$ is any representation of $B$.
\end{pf}



\begin{prb}[Sakai theorem]
Let $A$ be a \emph{W$^*$-algebra} or just a \emph{von Neumann algebra} in an intrinsic sense, that is, a C$^*$-algebra together with a predual $F\subset A^*$.
Consider the canonical embedding $A\subset A^{**}$ and a faithful unital normal representation $\pi:A^{**}\to B(H)$, for example, constructed by the Sherman-Takeda theorem.
We will show that every W$^*$-algebra $A$ is embedded in $A^{**}$ as a weakly$^*$ closed $*$-subalgebra, so that $A$ admits a faithful unital normal representation $\pi:A\to \pi(A)''\subset B(H)$.
In this context, a von Neumann algebra on a Hilbert space can be interpreted as just a W$^*$-algebra with a choice of a faithful unital normal representation.
\begin{parts}
\item There is an injective $*$-homomorphism $\pi:A\to A^{**}$ with weakly$^*$ closed image.
\item $\pi$ is a topological embedding with respect to $\sigma(A,F)$ and $\sigma(A^{**},A^*)$.
\item The predual $F$ is unique in $A^*$.
\end{parts}
\end{prb}
\begin{pf}
(a)
By the definition of predual, we have a linear map $\e:A^{**}\to A$ defined by the restriction on $F$, and it is a contractive idempotent surjective map, and hence is a $A$-bimodule map.
Since $A$ is dense in $\sigma(A^{**},A^*)$ by the Goldstine theorem, and since $\e$ is continuous between $\sigma(A^{**},A^*)$ and $\sigma(A,F)$, we can see that $\e$ is in fact a $A^{**}$-bimodule map, which means the kernel is a $\sigma$-weakly closed ideal of $A^{**}$.
Thus, we have a central projection $z\in A^{**}$ such that $\ker\e=(1-z)A^{**}$.
Define $\pi:A\to A^{**}$ such that $\pi(a):=za$.
It is a $*$-homomorphism because $z$ is central.
The injectivity follows from $a=\e(a)=\e(za)=\e(\pi(a))$ for $a\in A$, and $x-\e(x)\in\ker\e$ implies $zx=z\e(x)\in zA$ for $x\in A^{**}$ so that the image $\pi(A)=zA=zA^{**}$ is $\sigma$-weakly closed in $B(H)$.

(b)
Note that $\pi:A\to A^{**}$ is continuous with respect to the norm topology and $\sigma(A^{**},A^*)$ so that its adjoint can have the form $\pi^*:A^*\to A^*$.
For $\pi$ to be an embedding, it suffices to prove the equality $\pi^*(A^*)=F$.
First, suppose $l\in A^*$ satisfies $\pi^*(l)\in A^*\setminus F$.
Because $F$ is norm closed in $A^*$, by the Hahn-Banach extension, there is $x\in A^{**}$ such that $\<x,\pi^*(l)\>\ne0$ and $\<x,f\>=0$ for all $f\in F$.
Since $\<\e(x),f\>=\<x,f\>=0$ for every $f\in F$ from the definition of $\e$ and $F$ separates points $A$, we have $\e(x)=0$ and $x\in\ker\e$ implies $zx=0$.
Take a net $a_i\in A$ such that $a_i\to x$ in $\sigma(A^{**},A^*)$.
Then, if we take $\xi,\eta\in H$ such that $\<a,l\>=\<a\xi,\eta\>$, then we have
\[\<x,\pi^*(l)\>=\lim_i\<a_i,\pi^*(l)\>=\lim_i\<za_i,l\>=\lim_i\<za_i\xi,\eta\>=\lim_i\<a_i\xi,z\eta\>=\<x\xi,z\eta\>=\<zx\xi,\eta\>=0,\]
which is a contradiction, so we have $\pi^*(A^*)\subset F$.
Conversely, if $f\in F$, then we have $\<a,\pi^*(f)\>=\<za,f\>=\<a,f\>$ because $(1-z)a\in\ker\e$ acts on $F$ trivially by definition of $\e$, so $f=\pi^*(f)\in\pi^*(A^*)$.

(c)
Suppose $F_1$ and $F_2$ are preduals of $A$.
The identity $(A,\sigma(A,F_1))\to(A,\sigma(A,F_2))$ is a $*$-isomorphism between von Neumann algebras, which automatically has $\sigma$-weak continuity, so it induces the equality $\sigma(A,F_1)=\sigma(A,F_2)$ of topologies.
By taking duals for the two weak$^*$ topologies, we get $F_1=F_2$.
\end{pf}











\section*{Exercises}
\begin{prb}[Extremally disconnected space]
$\sigma(B^\infty(\Omega))$ is extremally disconnected.
\end{prb}

resolution of identity
normal operator theories: multiplicity, invariant subspaces
$L^\infty$ representation


$\sigma$-weakly closed left ideal has the form $Mp$. II.3.12

Let $\fm$ be an algebraic ideal of a von Neumann algebra $M$, and $\bar\fm$ be its $\sigma$-weak closure.
If $x\in(\bar\fm)_+$, then there is an increasing net $(x_i)\subset\fm$ converges to $x$ strongly. II.3.13



binary expansion and hereditary subalgebras

\end{document}