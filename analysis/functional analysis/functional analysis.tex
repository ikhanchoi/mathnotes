\documentclass{../../large}
\usepackage{../../ikhanchoi}

\newcommand{\Prim}{\operatorname{Prim}}

\begin{document}
\title{Functional Analysis}
\author{Ikhan Choi}
\maketitle
\tableofcontents

\part{Topological vector spaces}


\chapter{Locally convex spaces}
\section{Vector topologies}

\begin{prb}[Canonical uniformity and bornology]
\end{prb}
\begin{prb}[Metrizability]
Birkhoff-Kakutani
\end{prb}
\begin{prb}[Boundedness of linear operators]
\end{prb}




\section{Seminorms and convex sets}
\begin{prb}[Seminorms]
\[\bigcap_{i=1}^m\{:p()<1\}\]
Equivalent conditions on the continuity of seminorms
\end{prb}
\begin{pf}
\end{pf}
boundedness by seminorms, normability

\section{Continuous linear functionals}
\begin{prb}
Let $\bar{x^*}=(x_1^*,\cdots,x_n^*)\in X^{*n}$.
$\bar{x^*}:X\to\F^n$.
If $x^*\in X^*$ vanishes on $\bigcap_{i=1}^n\ker x_i^*$, then $x^*$ is a linear combination of $\{x_i^*\}$.
\end{prb}



\begin{prb}[Hahn-Banach extension]
Let $X$ be a real vector space.
Suppose $V$ is a linear subspace of $X$ and $l:V\to\R$ is a linear functional dominated by a sublinear functional $q:X\to\R$, that is, $l(v)\le q(v)$ for all $v\in V$.
\begin{parts}
\item There is a linear functional $\tilde l:X\to\R$ that extends $l$.
\item There is a linear functional $\tilde l:X\to\R$ that extends $l$ and is dominated by $q$ if $\dim X/V=1$.
\item There is a linear functional $\tilde l:X\to\R$ that extends $l$ and is dominated by $q$.
\end{parts}
\end{prb}
\begin{pf}
(a)
It can be done by the Hamel basis.

(b)
Let $e\in X\setminus V$ so that every vector $x\in X$ can be uniquely written as $x=v+te$ with $v\in V$ and $t\in\R$.
For $v_1,v_2\in V$,
\[l(v_1)+l(v_2)=l(v_1+v_2)\le q(v_1+v_2)\le q(v_1-e)+q(v_2+e)\]
implies
\[l(v_1)-q(v_1-e)\le-l(v_2)+q(v_2+e).\]
Define a linear functional $\tilde l:X\to\R$ such that $\tilde l(v)=v$ and
\[l(v)-q(v-e)\le\tilde l(e)\le-l(v)+q(v+e)\]
for all $v\in V$.
Since
\[\tilde l(v+te)=l(v)+t\tilde l(e)\le l(v)+t(-l(t^{-1}v)+q(t^{-1}v+e))=q(v+te)\]
if $t\ge0$ and
\[\tilde l(v+te)=l(v)+t\tilde l(e)\le l(v)+t(l(-t^{-1}v)-q(-t^{-1}v-e))=q(v+te)\]
if $t\le0$, we have $\tilde l(x)\in q(x)$ for all $x\in X$.

(c)
With $X$ and $q$, Consider a partially ordered set
\[\{(\tilde V,\tilde l)\mid
V\le\tilde V\le X,\ \tilde l:\tilde V\to\R\text{ is a linear extension of $l$ dominated by $q$}\}\]
such that $(V_1,l_1)\prec(V_2,l_2)$ if and only if $V_1\le V_2$ and $l_2|_{V_1}=l_1$.
The nonemptyness and the chain condition is easily satisfied, so it has a maximal element $(\tilde V,\tilde l)$ by the Zorn lemma.
By the part (b), we have $\tilde V=X$.
\end{pf}

\begin{prb}[Complex linear functionals]
Let $X$ be a complex vector space.
Consider a map
\[\begin{array}{ccc}
\{\text{$\C$-linear functionals on $X$}\}
&\to&
\{\text{$\R$-linear functionals on $X$}\}\\
l&\mapsto&\Re l.
\end{array}\]
Let $p$ be a seminorm on $X$ and $l$ a complex linear functional on $X$.
\begin{parts}
\item The above map is bijective.
\item $|l(x)|\le p(x)$ if and only if $|\Re l(x)|\le p(x)$.
\end{parts}
\end{prb}
\begin{pf}
(b)
There is $\lambda$ such that $|\lambda|=1$ and $l(\lambda x)\ge0$.
Then,
\[|l(x)|=|\lambda^{-1}l(\lambda x)|=l(\lambda x)=\Re l(\lambda x)\le p(\lambda x)=p(x).\]
\end{pf}


\begin{prb}[Hahn-Banach separation]
\end{prb}




\section*{Exercises}
\begin{prb}[Topology of compact convergence]
\end{prb}





\chapter{Barreled spaces}

\section{Uniform boundedness principle}
\begin{prb}[Barreled spaces]
Let $X$ be a topological vector space.
A \emph{barrel} is an absorbing, balanced, convex, and closed subset of $X$.
A \emph{barreled space} is a topological space in which every barrel is a neighborhood of zero.
\end{prb}

% If a closed convex cone contains a dense subset of absorbing at a point, then it is entire?

\begin{prb}[Uniform boundedness principle]
Let $X$ and $Y$ be topological vector spaces.
Let $\cF$ be a family of continuous linear operator from $X$ to $Y$.
Suppose $\bigcup_{T\in\cF}Tx$ is bounded for each $x\in D$, where $D\subset X$.
\begin{parts}
\item If $D$ is dense in $X$, then $\bigcap_{T\in\cF}T^{-1}\bar U$ is absorbing.
\item If $X$ is barreled, then $\cF$ is equicontinuous.
\end{parts}
\end{prb}



\section{Baire category theorem}

\begin{prb}[Baire spaces]
A topological space is called a \emph{Baire space} if the countable intersection of open dense subsets is always dense.
\begin{parts}
\item If a topological vector space is Baire, then it is barreled.
\item A Baire space is second category in itself.
\item A topological group that is second category in itself is Baire.
\end{parts}
\end{prb}



\begin{prb}[Absorbing sets]
Let $X$ be a topological vector space that is Baire.
A subset $U\subset X$ is said to be \emph{absorbing} if for every $x\in X$ there is a sufficiently large $t>0$ such that $x\in tU$.
Let $U\subset X$.
\begin{parts}
\item If $U$ is closed and absorbing, then $U$ has a non-empty open subset.
\item If $U$ is closed and absorbing, then $U-U$ is a neighborhood of zero.
\item If $U$ is closed, convex, and absorbing, then $U$ is a neighborhood of zero.
\end{parts}
\end{prb}


\begin{prb}[Baire category theorem]
The Baire category theorem proves many exmples of topological vector space are Baire, in particular barreled.
\begin{parts}
\item A complete metric space is Baire.
\item A locally compact Hausdorff space is Baire.
\end{parts}
\end{prb}




\section{Open mapping theorem}

\begin{prb}[Open mapping theorem]
Let $X$ be a $F$-space and $Y$ a barreled space.
Suppose $T:X\to Y$ is a continuous and surjective linear operator.
\begin{parts}
\item $\bar{TU}$ is a neighborhood of zero.
\item $TU$ is a neighborhood of zero.
\end{parts}
\end{prb}

\begin{pf}
(a)
Let $U'$ be a neighborhood of zero such that $U\supset U'-U'$.
Because $T$ is surjective, the set $\bar{TU'}$ is a closed absorbing set, so it contains a non-empty open subset, since $Y$ is barreled.
Thus, $\bar{TU}\supset\bar{TU'}-\bar{TU'}$ is a neighborhood of zero.

(b)
We claim $\bar{TU_{2^{-1}}}\subset TU_1$.
Take $y_1\in\bar{TU_{2^{-1}}}$.

Assume $y_n\in\bar{TU_{2^{-n}}}$.
Since $\bar{TU_{2^{-(n+1)}}}$ is a neighborhood of zero, we have
\[(y_n+\bar{TU_{2^{-(n+1)}}})\cap TU_{2^{-n}}\ne\varnothing.\]
Then, there is $x_n\in U_{2^{-n}}$ such that $Tx_n\in y_n+\bar{TU_{2^{-(n+1)}}}$.
Define
\[y_{n+1}:=y_n-Tx_n.\]

Then, $\sum_{n=1}^\infty x_n$ clearly converges to $x\in U_1$.
Therefore,
\[Tx=\sum_{n=1}^\infty Tx_n=\sum_{n=1}^\infty(y_n-y_{n+1})=y_1.\qedhere\]
\end{pf}


\section*{Exercises}

\begin{prb}
Let $(T_n)$ be a sequence in $B(X,Y)$.
If $T_n$ coverges strongly then $\|T_n\|$ is bounded by the uniform boundedness principle.
\end{prb}

\begin{prb}
There is a closed absorbing set in $\ell^2(\Z_{\ge0})$ that is not a neighborhood of zero;
\[\bar B(0,1)\setminus\bigcup_{i=2}^\infty B(i^{-1}e_i,i^{-2})\]
is a counterexample.
\end{prb}




\begin{prb}
There is no metric $d$ on $C([0,1])$ such that $d(f_n,f)\to0$ if and only if $f_n\to f$ pointwise as $n\to\infty$ for every sequence $f_n$.
Note that this problem is slightly different to the non-metrizability of the topology of pointwise convergence.
\end{prb}

\begin{prb}
We show that there is no projection from $\ell^\infty$ onto $c_0$.
\end{prb}

\begin{prb}[Schur property]
$\ell^1$
\end{prb}

\begin{prb}
Let $\f:L^\infty([0,1])\to\ell^\infty(\N)$ be an isometric isomorphism.
Suppose $\f$ is realised as a sequence of bounded linear functionals on $L^\infty$.
\begin{parts}
\item
Show that $\f^*(\ell^1)\subset L^1$ where $\ell^1$ and $L^1$ are considered as closed linear subspaces of $(\ell^\infty)^*$ and $(L^\infty)^*$ respectively.
\item Show that $\f^*$ is indeed an isometric isomorphism, and deduce $\f$ cannot be realised as bounded linear functionals on $L^\infty$.
\end{parts}
\end{prb}


\begin{prb}[Daugavet property]
\begin{parts}
\item The real Banach space $C([0,1])$ satisfies the Daugavet property.
\end{parts}
\end{prb}
\begin{pf}
Let $T$ be a finite rank operator on $C([0,1])$, and $e_i$ be a basis of $\im T$.
Then, for some measures $\mu_i$,
\[Tf(t)=\sum_{i=1}^n\int_0^1f\,d\mu_ie_i(t).\]
Let $M:=\max\|e_i\|$.

Take $f_0$ such that $\|f_0\|=1$ and $\|Tf_0\|>\|T\|-\frac\e2$.
Reversing the sign of $f_0$ if necessary, take an open interval $\Delta$ such that $Tf_0(t)\ge\|T\|-\frac\e2$ and $|\mu_i|(\Delta)\le\frac\e{4nM}$ for all $i$.
Define $f_1$ such that $f_0=f_1$ on $\Delta^c$, $f_1(t_0)=1$ for some $t_0\in\Delta$, and $\|f_1\|=1$.
Then, $\|Tf_1-Tf_0\|\le\frac\e2$ shows $Tf_1\ge\|T\|-\e$ on $\Delta$.
Therefore,
\[\|1+T\|\ge\|f_1+Tf_1\|\ge f_1(t_0)+Tf_1(t_0)\le1+\|T\|-\e.\]
\end{pf}

\begin{prb}[Bartle-Graves theorem]
Let $E$ be a Banach space and $N$ a closed subspace.
For $\e>0$, there is a continuous homogeneous map $\rho:E/N\to E$ such that $\pi\rho(y)=y$ and $\|\rho(y)\|\le(1+\e)\|y\|$ for all $y\in E/N$.
\end{prb}
\begin{pf}
We want to construct a continuous map $\psi:S_{E/N}\to E$ with $\|\psi(y)\|\le1+\e$ for all $y\in S_{E/N}$.
If then, $\rho$ can be made from $\psi$.

For each $y_0\in S_{E/N}$, choose $x_0\in\pi^{-1}(y_0)\cap B_{1+\e}$.
There is a neighborhood $V_{y_0}\subset S_{E/N}$ of $y_0$ such that $y\in V_{y_0}$ implies $x_0$ belongs to $(\pi^{-1}(y)\cap B_{1+\e})+U_{2^{-1}}$, which is convex.
With a locally finite subcover $V_{y_\alpha}$ and a partition of unity $\eta_\alpha(y)$, define $\psi_1(y)=\sum_\alpha\eta_\alpha(y)x_\alpha$.
Then, $\psi_1(y)\in(\pi^{-1}(y)\cap B_{1+\e})+U_{2^{-1}}$.

For $i\le2$, choose for each $y_0$ the element $x_0$ in $\pi^{-1}(y_0)\cap B_{1+\e}\cap(\psi_{i-1}(y_0)+U_{2^{-{i-1}}})$.
Then, we obtain
\[\psi_i(y)\in\Bigl(\pi^{-1}(y)\cap B_{1+\e}\cap(\psi_{i-1}(y_0)+U_{2^{-{i-1}}})\Bigr)+U_{2^{-i}}.\]
Therefore, $\|\psi_i(y)-\psi_{i-1}(y)\|<2^{-{i-2}}$, so it converges uniformly to $\psi$ such that $\psi(y)\in\pi^{-1}(y)\cap B_{1+\e}$.
\end{pf}

\section*{Problems}
\begin{prb}
Let $T$ be an invertible linear operator on a normed space.
Then, $T^{-2}+\|T\|^{-2}$ is injective if it is surjective.
\end{prb}
















\chapter{Weak topologies}
\section{Dual spaces}

\begin{prb}[Bidual]
\end{prb}

\begin{prb}
Let $X$ be a locally convex space.
The \emph{weak topology} is the topology $w$ on $X$ defined by the family of seminorms $\{x\mapsto|\<x,\xi\>|\}_{\xi\in X^*}$.
The \emph{weak$^*$ topology} is the topology $w^*$ on $X^*$ defined by the family of seminorms $\{\xi\mapsto|\<x,\xi\>|\}_{x\in X}$.
Let $J:X\to X^{**}$ be the canonical embedding.
\begin{parts}
\item $(X,w)$ and $(X^*,w^*)$ are locally convex.
\item $(X,w)^*=X^*$.
\item $(X^*,w^*)^*=X$. Every locally convex space is a dual of a locally convex space.
\end{parts}
\end{prb}
\begin{pf}
(a)
The Hahn-Banach theorem implies the Hausdorffness.

(c)
We will only show $(X^*,w^*)^*\subset X$.
If $u\in(X^*,w^*)^*$, then there are $x_1,\cdots,x_m\in X$ such that
\[|\<u,\xi\>|\le\sum_{i=1}^m|\<x_i,\xi\>|\]
for all $\xi\in X^*$.
If we let $\ker\vec x:=\bigcap_{i=1}^m\ker x_i$, then it is a closed subspace of $X^*$ such that $\ker\vec x\subset\ker u$, so we have $u\in\spn\vec x\subset X$.
\end{pf}

\begin{prb}
closure and weak closure of convex subsets
\end{prb}
\begin{pf}
Hahn-Banach
\end{pf}

\begin{prb}[Polar]
\end{prb}


boundedness, incompleteness

\begin{prb}[Weak convergence by dense set]
Let $X$ be a Banach space, $D^*$ a subset of $X^*$, and $\bar{D^*}$ the norm closure of $D^*$.
For example, if $X$ has a predual $X_*\subset X^*$ and $D^*$ is dense in $X_*$, then $\sigma(X,\bar{D^*})$ is the weak$^*$ topology.
\begin{parts}
\item There is a squence $x_n\in X$ converges to zero in $\sigma(X,D^*)$ but not in $\sigma(X,\bar{D^*})$.
\item A bounded sequence $x_n\in X$ converges to zero in $\sigma(X,\bar{D^*})$ if in $\sigma(X,D^*)$.
\end{parts}
\end{prb}
\begin{pf}
(b)
Let $\xi\in\bar{D^*}$ and choose $\eta\in D^*$ such that $\|\xi-\eta\|<\e$.
Then,
\[|\<x_n,\xi\>|\le\|x_n\|\|\xi-\eta\|+|\<x_n,\eta\>|\lesssim\e+|\<x_n,\eta\>|\to\e.\]
\end{pf}



\section{Weak compactness}
\begin{prb}[Banach-Alaoglu theorem]
\end{prb}
\begin{pf}
Consider
\[B_{X^*}\to\prod_{x\in X}\|x\|B:l\mapsto(l(x))_{x\in X}.\]
Since it is an embedding into a compact space, it suffices to show the closedness of image: for $l(x):=\lim_\alpha l_\alpha(x)$, we have
\[\|l(x)\|\le\|l(x)-l_\alpha(x)\|+\|x\|\xrightarrow{\alpha\to\infty}\|x\|,\]
so $l$ is contained in the range.
\end{pf}
\begin{prb}[Eberlein-\v Smulian theorem]
\end{prb}
\begin{prb}[James' theorem]
\end{prb}

\section{Weak density}
Bishop-Phelps theorem

\begin{prb}[Goldstine theorem]
Let $X$ be a Banach space.
Then, $B_X$ is weakly$^*$ dense in $B_{X^{**}}$.
\end{prb}
\begin{pf}
Take $x^{**}\in B_{X^{**}}\setminus\bar{B_X}^{w*}$.
By the Hahn-Banach separation, there are $x^*\in X^*$ and $r\in\R$ such that
\[\Re\<x,x^*\>\le r<\Re\<x^{**},x^*\>\]
for every $x\in B_X$.
Since the left hand side can approximate $\|x^*\|$, we have $\|x^*\|\le r$ and obtain a contradiction
\[r<\Re\<x^{**},x^*\>\le\|x^*\|\le r.\qedhere\]
\end{pf}




\section{Krein-Milman theorem}
Choquet theory


\section{Polar topologies}
Mackey-Arens


\section*{Exercises}
\begin{prb}[James' space]
not reflexive but isometrically isomorphic to bidual
\end{prb}

\begin{prb}[Preduals]
Let $X$ be a Banach space.
A \emph{predual} of $X$ is a Banach space $F$ together with an isometric isomorphism $\f:X\to F^*$.
Two preduals $\f_1:X\to F_1^*$ and $\f_2:X\to F_2^*$ are said to be equivalent if there is an isometric isomorphism $\theta:F_1\to F_2$ such that $\theta^*=\f_1\f_2^{-1}$.
\begin{parts}
\item There is a one-to-one correspondence between the equivalence class of preduals of $X$ and the set of closed subspaces $X_*$ of $X^*$ such that $B_X$ is compact and Hausdorff in $(X,\sigma(X,X_*))$.
Such a subspace $X_*$ is also called a predual of $X$.
\item If $X$ admits a predual $X_*\subset X^*$, then a $\sigma(X,X_*)$-closed subspace $V$ of $X$ also admits a predual $X_*|_V$.
\end{parts}
\end{prb}
\begin{pf}
(a) Goldstine theorem for surjectivity.

(b)
It is easy if we apply the part (a).
We can show more directly.
If we let $V_*:=X_*|_V$ the image of $X_*$ under the map $X^*\to V^*$, then we have isometric injections $V\to(V_*)^*\to X$.
We can show $V$ is $\sigma(X,X_*)$ dense in $(V_*)^*$, hence the closedness proves the bijectivity of $V\to(V_*)^*$.
\end{pf}

\begin{prb}[Mazur's lemma]

\end{prb}



















\part{Banach spaces}




\chapter{Operators on Banach spaces}

\section{Bounded operators}
\begin{prb}[Bounded belowness in Banach spaces]
Let $T\in B(X,Y)$ for Banach spaces $X$ and $Y$.
The following statements are equivalent:
\begin{parts}
\item $T$ is bounded below.
\item $T$ is injective and has closed range.
\item $T$ is a topological isomorphism onto its image.
\end{parts}
\end{prb}

\begin{prb}[Bounded belowness in Hilbert spaces]
Let $T\in B(H,K)$ for Hilbert spaces $H$ and $K$.
The following statements are equivalent:
\begin{parts}
\item $T$ is bounded below.
\item $T$ is left invertible.
\item $T^*$ is right invertible.
\item $T^*T$ is invertible.
\end{parts}
\end{prb}

\begin{prb}[Injectivity and surjectivity of adjoint]
Let $T\in B(X,Y)$ for Banach spaces $X$ and $Y$.
\begin{parts}
\item $T^*$ is injective if and only if $T$ has dense range.
\item $T^*$ is surjective if and only if $T$ is bounded below.
\end{parts}
\end{prb}








\section{Compact operators}

$K(X,Y)$ is closed in $B(X,Y)$.
$K(X)$ is an ideal of $B(X)$.
adjoint is $K(X,Y)\to K(Y^*,X^*)$.
integral operators are compact.
riesz operator, quasi-nilpotent operator.




\section{Fredholm operators}

\begin{prb}
A bounded linear operator $T:X\to Y$ between Banach spaces is called a \emph{Fredholm} operator if its kernel is finite dimensional and its range is finite codimensional.
\begin{parts}
\item A Fredholm operator $T$ has closed range.
\end{parts}
\end{prb}
\begin{pf}
(a)
Let $C$ be a finite dimensional subsapce of $Y$ such that $\im T\oplus C=Y$.
Let $\tilde T:X/\ker T\to Y$ be the induced operator of $T$.
Define $S:(X/\ker T)\oplus C\to Y$ such that $S(x+\ker T,c):=\tilde T(x+\ker T)+c$.
Then, $S$ is an topological isomorhpism between Banach spaces by the open mapping theorem, so $S(X/\ker T\oplus\{0\})=\im\tilde T=\im T$ is closed.
\end{pf}

\begin{prb}[Atkinson's theorem]
An operator $T\in B(X,Y)$ is Fredholm if and only if there is $S\in B(Y,X)$ such that $TS-I$ and $ST-I$ is finite rank.
\end{prb}

\begin{prb}[Fredholm index]
locally constant, in particular, continuous.
composition makes the addition of indices.
\end{prb}

\section{Nuclear operators}
tensor products





\section*{Exercises}

\begin{prb}[Completely continuous operators]
On reflexive spaces, completely continuous operators are same with compact operators.
\end{prb}


\begin{prb}[Dunford-Pettis property]
A Banach space $X$ is said to have the \emph{Dunford-Pettis property} if all weakly compact operators $T:X\to Y$ to any Banach space $Y$ is completely continuous.
\begin{parts}
\item $X$ has the Dunford-Pettis property if and only if for every sequences $x_n\in X$ and $x^*_n\in X^*$ that converge to $x$ and $x^*$ weakly we have $x^*_n(x_n)\to x^*(x)$.
\item $C(\Omega)$ for a compact Hausdorff space $\Omega$ has the Dunford-Pettis property.
\item $L^1(\Omega)$ for a probability space $\Omega$ has the Dunford-Pettis property.
\item Infinite dimensional reflexive Banach space does not have the Dunfor-Pettis property.
\end{parts}
\end{prb}


\begin{prb}\,
\begin{parts}
\item (Mazur-Ulam, 1932) A surjective isometry $T:X\to Y$ between normed spaces is affine.
\item (Mankiewicz, 1972) Let $U,V$ be open sets in $X,Y$, normed spaces. A surjective isometry $U\to V$ is uniquely extended to a surjective isometry $X\to Y$.
\item (Mori) A surjective local isometry $T:X\to Y$ between Banach spaces is an isometry, if $X$ is separable. (Use the Baire category)
\end{parts}
\end{prb}
\begin{sol}
(a)
$T$ is continuous.
It is easy to see for continuous map $T$ that it is affine if and only if $T$ preserves the midpoint.
For $x_1\ne x_2\in X$ let $x_0$ be the midpoint.
Define inductively
\[C_1:=\{x\in X:\|x-x_1\|=\|x-x_2\|=\frac12\|x_1-x_2\|\},\qquad C_k:=\{x\in C_{k-1}:\sup_{x'\in C_{k-1}}\|x-x'\|\le\frac12\diam C_{k-1}\}.\]
Since $x_0\in C_{k-1}$ and $x'\in C_{k-1}$ imply $x_0\in C_k$ by $\|x_0-x'\|=\frac12\|(2x_0-x')-x'\|\le\frac12\diam C_{k-1}$, and since $\diam C_k\le\frac12\diam C_{k-1}$, we have $\{x_0\}=\bigcup_{k=1}^\infty C_k$.
It follows that the midpoint can be detected from the metric structure of $X$, not depending on the linear structure of $X$.
\end{sol}


\section*{Problems}
\begin{enumerate}
\item If $T\in B(L^2([0,1]))$ is a compact operator, then for any $\e>0$ there is a constant $C_\e>0$ such that
\[\|Tf\|_{L^2}\le\e\|f\|_{L^2}+C_\e\|f\|_{L^1}.\]
\end{enumerate}

\begin{pf}
1. Suppose there is $\e>0$ such that we have sequence $f_n\in L^2$ satisfying $\|f_n\|_2=1$ and
\[\|Tf_n\|_2>\e+n\|f_n\|_1.\]
By the compactness of $T$, there is a subsequence $Tf_{n_k}$ converges to $g\ne0$ in $L^2$.
Then, $\|f_{n_k}\|_1\to0$ implies $f_{n_k}\to0$ weakly in $L^2$, hence also for $Tf_{n_k}$.
It means $g=0$, which contradicts to the assumption.
\end{pf}




\chapter{Geometry of Banach spaces}

\section{Tensor products}

\section{Approximation property}
dual is Banach.
Basis problem, Mazur' duck.


\begin{prb}[Approximation property]
Every compact operator is a limit of finite-rank operators.
\begin{parts}
\item An Hilbert space has the AP.
\item
\end{parts}
\end{prb}
\begin{pf}
(a)
Let $H$ be a Hilbert space and $K\in K(H)$.
Since $\bar{KB_H}$ is a compact metric space, it is separable, which means $\bar{KH}$ is separable.
Let $(e_i)_{i=1}^\infty$ be an orthonormal basis of $\bar{KH}$, and let $P_n$ be the orthogonal projection on the space spanned by $(e_i)_{i=1}^n$.
If we let $K_n:=P_nK$, then $K_n\to K$ strongly and $K_n$ has finite rank.
Take any $\e>0$ and find, using the totally boundedness of $KB_H$, a finite subset $\{x_j\}\subset B_H$ such that for any $x\in B_H$ there is $x_j$ satisfying $\|Kx-Kx_j\|<\frac\e2$.
Then,
\begin{align*}
\|Kx-K_nx\|
&\le\|Kx-Kx_j\|+\|Kx_j-K_nx_j\|+\|P_n(Kx_j-Kx)\|\\
&\le\frac\e2+\|Kx_j-K_nx_j\|+\frac\e2.
\end{align*}
By taking the supremum on $x\in B_H$, we have
\[\|K-K_n\|\le\max_j\|Kx_j-K_nx_j\|+\e,\]
which deduces $K_n\to K$ in norm.

\end{pf}

\section*{Exercises}
Tingley problem



\chapter{}







\part{Spectral theory}

\chapter{Operators on Hilbert spaces}

\section{Operator topologies}
Projections. Reducing subspaces.
Hilbert space classification by cardinal.
Riesz representation theorem.
\begin{prb}
\begin{parts}
\item A Banach space $X$ is isometrically isomorphic to a Hilbert space if there is a bounded linear projection on every closed subspace of $X$.
\end{parts}
\end{prb}

\begin{prb}[Riesz representation theorem]
Let $H$ be a Hilbert space over a field $\K$, which is either $\R$ of $\C$.


We use the bilinear form $\<-,-\>:X\times X^*\to\K$ of canonical duality.
The Riesz representation theorem states that a continuous linear functional on a Hilbert space is represented by the inner product with a vector.
\begin{parts}
\item For each $x^*\in H^*$, there is a unique $x\in H$ such that $\<y,x^*\>=\<y,x\>$ for every $y\in H$.
\item $H\to H^*:x\mapsto\<-,x\>$ is a natural linear and anti-linear isomorphism if $\K=\R$ and $\C$, respectively.
\end{parts}
\end{prb}



Let $H$ be a separable Hilbert space.
Find a positive sequence $a_n$ such that every sequence $x_n$ of unit vectors of $H$ satisfying $|\<x_i,x_j\>|\le a_j$ for all $i<j$ converges weakly to zero.



\begin{prb}[Normal operators]
For $T\in B(H)$, we have an obvious fact $(\im T)^\perp=\ker T^*$.
Suppose $T$ is normal.
\begin{parts}
\item $\ker T=\ker T^*$.
\item $T$ is bounded below if and only if $T$ is invertible.
\item If $T$ is surjective, then $T$ is invertible.
\end{parts}
\end{prb}

\begin{prb}[Invariant and Reducing subsapces]
Let $K$ be a closed subspace of $H$.
\begin{parts}
\item $K$ is reducing for $T$ if and only if $K$ is invariant for $T$ and $T^*$.
\item $K$ is reducing for $T$ if and only if $TP=PT$, where $P$ is the orthogonal projection on $K$.
\end{parts}
\end{prb}
% self adjoint operators
% invariant but not reducing for unitary operators
% eigenspaces
% matrix representation


% direct sum and tensor product of hilbert spaces

\begin{prb}[Trace class operators]
Let $K\in B(H)$
The \emph{trace} of $K$ is
\[\Tr(K):=\sum_i\<Ke_i,e_i\>,\]
where $(e_i)\subset H$ is an orthonormal basis.
The operator $K$ is said to be in the \emph{trace-class} if $\Tr(|K|)<\infty$.
\begin{parts}
\item
The trace does not depend on the choice of $(e_i)$.
\item
$K$ is a trace class if and only if $K=\sum_{i=1}^\infty\lambda_i\theta_{x_i,y_i}$ for some $(\lambda_i)_{i=1}^\infty\subset\ell^1(\N)$ and orthogonal sequences $(x_i)_{i=1}^\infty,(y_i)_{i=1}^\infty\subset H$.
\item
$B(H)\to L^1(H)^*:T\mapsto\Tr(T\cdot)$ is an isometric isomorphism.
\end{parts}
\end{prb}
\begin{pf}
(b)
Conversely, we can check the diagonalization $K^*K=\sum_{i=1}^\infty|\lambda_i|^2\theta_{y_i}$, which implies $|K|=\sum_{i=1}^\infty|\lambda_i|\theta_{y_i}$.
Thus,
\[Tr(|K|)=\sum_{j=1}^\infty\<|K|y_j,y_j\>=\sum_{i=1}^\infty|\lambda_i|<\infty.\]

\end{pf}




\begin{prb}
\begin{parts}
\item
A net $T_\alpha$ converges to $T$ strongly in $B(H)$ if and only if $\|(T_\alpha-T)^{\oplus n}\bar\xi\|\to0$ for all $\bar\xi\in H^{\oplus n}$.
\item
A net $T_\alpha$ converges to $T$ $\sigma$-strongly in $B(H)$ if and only if $\|(T_\alpha-T)^{\oplus\infty}\bar\xi\|\to0$ for all $\bar\xi\in H^{\oplus\infty}$.
\end{parts}
\end{prb}

\begin{prb}[Strong$^*$ operator topology]
Let $H$ be a Hilbert space.
We provides some conditions for a strongly convergent sequence to converge strongly$^*$.
Let $(T_\alpha)\subset B(H)$ and suppose $T_\alpha\to T$ strongly.
\end{prb}

\begin{prb}[Continuity of linear functionals]
Let $f$ be a linear functional on $B(H)$ for a Hilbert space $H$.
\begin{parts}
\item
$f$ is weakly continuous if and only if it is strongly$^*$ continuous, and in this case we have $f=\sum_i\omega_{x_i,y_i}$ for some $(x_i),(y_i)\in c_c(\N,H)$.
\item
$f$ is $\sigma$-weakly continuous if and only if it is $\sigma$-strongly$^*$ continuous, and in this case we have $f=\sum_i\omega_{x_i,y_i}$ for some $(x_i),(y_i)\in\ell^2(\N,H)$.
\end{parts}
\end{prb}
\begin{pf}
Suppose $f$ is strongly continuous.
There exists $\bar x\in H^{\oplus n}$ such that
\[|f(T)|\le\|T^{\oplus n}\bar x\|.\]
The functional $f:A\to\C$ factors through $H^{\oplus n}$ such that
\[A\xrightarrow{\bar x}H^{\oplus n}\to\C.\]
\end{pf}


For $\bar x=(x_i)\in\ell^2(\N,H)$,
\[p_{\bar x}^{\sigma s*}(T)=\left(\sum_i\|Tx_i\|^2+\|T^*x_i\|^2\right)^{\frac12}\qquad
p_{\bar x}^{\sigma s}(T)=\left(\sum_i\|Tx_i\|^2\right)^{\frac12}\qquad
p_{\bar x}^{\sigma w}(T)=\left|\sum_i\<Tx_i,x_i\>\right|\]







\section{Closed operators}

\begin{prb}[Closed operators]
\begin{parts}
\item a
\end{parts}
\end{prb}

\begin{prb}[Adjoint operators]
Let $T:\dom T\subset X\to Y$ be a densely defined linear operator between Banach spaces.
Define an unbounded operator $T^*:\dom T^*\subset Y^*\to X^*$ such that $\<x,T^*y^*\>:=\<Tx,y^*\>$ for all $x\in\dom T$ and $y^*\in\dom T^*$, where
\[\dom T^*:=\{y^*\in Y^*\mid \dom T\to\C:x\mapsto\<Tx,y^*\>\text{ is bounded}\}.\]
\begin{parts}
\item If $T\subset S$, then $S^*\subset T^*$.
\item $T^*$ is always closed.
\item $T$ is closable if and only if $T^*$ is densely defined. If it is, then $T^{**}$ is the closure of $T$. (Only on reflexive spaces?)
\item $T^*$ is injective if and only if $T$ has dense range, and surjective if and only if $T$ is bounded below.
\end{parts}
\end{prb}
\begin{pf}

(d)
Suppose $T$ is bounded below.
Fix $x^*\in X^*$.
Since $T$ is bounded below, $x^*$ defines a bounded linear functional on $\dom T$ with respect to $\|x\|+\|Tx\|$, which is embedded in $Y$ as a closed subspace.
By the Hahn-Banach extension, we have an element $y^*\in Y^*$ such that $\<Tx,y^*\>=\<x,x^*\>$ for all $x\in X$, which is contained in $\dom T^*$ by the definition of $\dom T^*$.
This implies that $T$ is surjective because $T^*y^*=x^*$.

Conversely, suppose $T^*$ is surjective.
Let $F:=\{x\in\dom T:\|Tx\|\le1\}$.
Since for every $x^*\in X^*$ we have for some $y^*\in\dom T^*$ such that
\[\sup_{x\in F}|\<x,x^*\>|=\sup_{x\in F}|\<x,T^*y^*\>|=\sup_{x\in F}|\<Tx,y^*\>|\le\|y^*\|.\]
By the uniform boundedness principle, we have $\sup_{x\in F}(\|x\|+\|Tx\|)$ is bounded, we are done.
\end{pf}

\begin{prb}[Operations of unbounded operators]
inverse, composition, addition
\end{prb}





\begin{prb}[Symmetric operators]
A densely defined operator $T:\dom T\to H$ is called \emph{symmetric} if
\[\<Tx,y\>=\<x,Ty\>,\qquad x,y\in\dom T.\]
Let $T$ be a densely defined symmetric operator.
If the closure of $T$ is self-adjoint, then it is called \emph{essentially self-adjoint}.
\begin{parts}
\item $T$ has the closed and densely defined closure.
\item Every symmetric extension of $T$ is a restriction of $T^*$, which is not symmetric in general. In particular, $T$ has a maximal symmetric extension.
\item A maximal symmetric operator is closed since the closure of a .
\item A self-adjoint operator is maximal.
\item A densely defined closed symmetric operator is essentially self-adjoint if and only if it is indeed the unique self-adjoint extension if and only if the adjoint is symmetric.
\end{parts}
\end{prb}


\begin{prb}[Cayley transform]
There is a one-to-one correspondence between the unitary operators from $K_+$ to $K_-$, the deficiency subspaces.
\end{prb}


Let $T$ be a symmetric operator on a Hilbert space $H$.
We will always assume that $T$ is densely defined and closed.
We want to ask the following questions:
Is $T$ self-adjoint?
If not, does $T$ admit self-adjoint extensions?
Which self-adjoint extension generate the appropriate quantum dynamics?

\begin{ex*}
Let $T:=i\,d/dx$ on $L^2([0,1])$ with
\[\dom T=H_0^1((0,1)).\]
It is densely defined and closed.
Then,
\[\dom T^*=H^1((0,1))\subset C([0,1])\]
and $T^*$ is not self-adjoint since...
The set of self-adjoint extensions is $\{T_\alpha:\alpha\in\T\}$, where
\[\dom T_\alpha=\{f\in H^1((0,1)):\alpha f(0)=f(1)\}.\]
\end{ex*}



\section{Spectral theorems}





\begin{prb}[Spectral measure]
Let $(\Omega,\cA)$ be a measurable space and $H$ a Hilbert space.
A \emph{projection-valued measure} on $\Omega$ for $H$ is a map $E:\cA\to B(H)$ with $E(\varnothing)=0$ such that $E(A)$ is a projection for every $A\in\cA$, and one of the following equivalent conditions hold:
\begin{enumerate}[(i)]
\item the set function $E_{x,y}:\cA\to\C:A\mapsto\<E(A)x,y\>$ is a complex measure on $\Omega$ for each $x,y\in H$.
\item the countable additivity holds, i.e.~$E(\bigsqcup_{i=1}^\infty A_i)=\sum_{i=1}^\infty E(A_i)$ in the strong operator topology of $B(H)$ for $(A_i)_{i=1}^\infty\subset\cM$.
\end{enumerate}
\begin{parts}
\item $E(A\cap B)=E(A)E(B)$ for $A,B\in\cM$.
\end{parts}
\end{prb}

\begin{prb}
Let $T\in B(H)$ be a normal operator.
Then, there exists a projection-valued measure $E$ on $\sigma(T)$ for $H$ such that
\[T=\int_{\sigma(T)}\lambda\,dE(\lambda).\]
This spectral measure $E$ is also called the \emph{resolution of the identity}.
\end{prb}




A multiplication operator by any Borel measurable function $\Omega\to\C$ always defines a densely defined closed normal operator.



Let $E$ be the spectral measure of a normal operator $T\in B(H)$.
If we choose $\xi\in E(B(\lambda,n^{-1}))H$, then since $E(B(\lambda,n^{-1})^c)\xi=0$, or since $E(B(\lambda,n^{-1}))\xi=\xi$, we have
\begin{align*}
\|(\lambda-T)\xi\|^2
&=\int|\lambda-z|^2\,d\<E(z)\xi,\xi\>\\
&=\int_{B(\lambda,n^{-1})}|\lambda-z|^2\,d\<E(z)\xi,\xi\>\\
&\le n^{-2}\int_{B(\lambda,n^{-1})}\,d\<E(z)\xi,\xi\>\\
&\le n^{-2}\int\,d\<E(z)\xi,\xi\>\\
&=n^{-2}\|\xi\|^2.
\end{align*}





\begin{prb}[Spectral representation]
Let $T$ be a bounded normal operator on a Hilbert space $H$.
Then, the unital C$^*$-algebra $C^*(T)$ generated by $T$ is $*$-isomorphic to $C(\sigma(T))$, and it has a canonical faithful representation $\pi:C(\sigma(T))\to B(H)$.
This representation exactly corresponds to the object called spectral measure.
We now decompose $\pi=\bigoplus_\alpha\pi_\alpha$ to cyclic representations $\pi_\alpha:C(\sigma(T))\to B(H_\alpha)$ with cyclic unit vectors $\psi_\alpha$.
Each vector state $\psi_\alpha$ induces a probability measure $\mu_\alpha$ on $\sigma(T)$.
It is called the spectral measure associated to the cyclic vector $\psi_\alpha$.
We can check that the GNS-representation of $\mu_\alpha$ is unitarily equivalent to $\pi_\alpha$.
The direct sum $C(\sigma(T))\to\bigoplus_\alpha B(L^2(\mu_\alpha))$ can be defined.
Then, we can show the bounded normal operator $T$ is always unitarily equivalent to the multiplication operator on a finite measure space.
However, in order for $T$ to be unitarily equivalent to the multiplication operator by the identity function of $\C$, $T$ must be multiplicity free, equivalently, $T$ must have a cyclic vector of $H$.
\end{prb}







Two bounded normal operators are unitarily equivalent if and only if the sequence of multiplicity measure classes are identical.

Two spectral theorems: Multiplication operator form(unitary equivalence), Projection-valued measure form(functional calculus).


% point spectrum, approximate point spectrum
Kato-Rellich theorem

For a densely defined closed operator $T:H\to H$, $\sigma(T^*)=\bar{\sigma(T)}$.


\begin{prb}[Polar decomposition]
polar decomposition
polar decomposition of symmetric operator?
polar decomopsition changes spectrum or domains?

support projection
\end{prb}

\begin{prb}[Stone theorem]
\end{prb}

\begin{prb}[Analytic vectors]
\begin{parts}
\item If $T$ is symmetric and $D_0$ is dense, then $T|_{D_0}$ is essentially self-adjoint.
\end{parts}
\end{prb}

\begin{prb}[Resolvent convergence]
\end{prb}




\section{Decomposition of spectrum}

\begin{align*}
\sigma
&=\sigma_p\cup\sigma_c\cup\sigma_r\\
&=\sigma_{ess}\cup\sigma_d\\
&=\bar{\sigma_{pp}}\cup\sigma_{ac}\cup\sigma_{sc}.
\end{align*}


\[\sigma=\sigma_p\sqcup\sigma_c\sqcup\sigma_r=\bar{\sigma_{pp}}\cup\sigma_{ac}\sigma_{sc}=\sigma_d\sqcup\sigma_{ess,5}.\]





\section*{Exercises}


\begin{prb}[Strict topology]
Let $H$ be a Hilbert space.
Let $(T_\alpha)\subset B(H)$ and $K\in K(H)$.
\begin{parts}
\item The strong$^*$ topology and the strict topology agree on bounded sets of $B(H)$.
\end{parts}
\end{prb}

\begin{prb}[Unitary group]
Let $H$ be a Hilbert space.
\begin{parts}
\item The weak topology and the strict topology agree on $U(H)$.
\end{parts}
\end{prb}


\begin{prb}[Bounded increasing nets]
Let $T_\alpha$ be a bounded increasing net of bounded self-adjoint operators on $H$.
\begin{parts}
\item $T_\alpha$ converges strictly. In particular, $T_\alpha\to T$ strictly iff $T_\alpha\to T$ weakly.
\end{parts}
\end{prb}
\begin{pf}
Define $T$ such that
\[\<Tx,y\>:=\lim_\alpha\sum_{k=0}^3i^k\<T_\alpha(x+i^ky),x+i^ky\>.\]
The convergence is due to the monotone convergence in $\R$.
We can check it is a well-defined bounded linear operator by considering the bounded sesquilinear form.
Then, $T_\alpha\to T$ weakly by definition, and $\sigma$-strongly because the net is increasing.
\end{pf}




\begin{prb}[Distributional operators]
\begin{parts}
\item Every continuous linear operator $T:\cD(\R)\to\cD'(\R)$ naturally defines a closable densely defined operator $T:\dom T\to L^2(\R)$ with $\dom T:=\cD(\R)$.
\end{parts}
\end{prb}

\begin{prb}[Hydrogen atom]
For $V\in L^\infty(\R^d)$, the operator
\[H\psi(x):=-\frac{\hbar^2}{2m}\Delta\psi(x)-V(x)\psi(x),\qquad x\in\R^d\]
is called the \emph{Schr\"odinger operator}, and simply we write $H=-\Delta+V$.
The eigenvectors associated to the discrete spectrum is called \emph{bound eigenstates}.

Consider the Schr\"odinger operator $H:=-\Delta-|x|^{-1}$ on $L^2(\R^3)$.
We want to investigate the spectral decomposition of $H$ by diagonalization.
\begin{parts}
\item $H$ is self-adjoint.
\item $\sigma_d(H)=\{\}$
\end{parts}
\end{prb}

The orbital comes from the diagonalization of the Laplace-Beltrami operator on the unit sphere.

The periodic Schr\"odinger operator is diagonalized to the direct integral of elliptic operators defined on the Brillouin torus.



\chapter{Operator theory}
\section{Toeplitz operators}

invariant subspace problem
Beurling theorem
Hardy and Bergman and Bloch spaces
JB* triple


\chapter{}




\part{Operator algebras}
\chapter{Banach algebras}

\section{Spectra of elements}

\begin{prb}[Banach algebras]
For a Banach algebra $A$ with multiplicative unit, there is a complete renorming such that $\|1\|=1$, i.e. we can always assume $\|1\|=1$.
It provides a definition of \emph{unital Banach algebra}.

Let $A$ be a unital Banach algebra.
\begin{parts}
\item If $\|a\|<1$, then $1-a$ is invertible. So $A^\times$ is open.
\item $A^\times\to A^\times:a\mapsto a^{-1}$ is continuous.
\item $A^\times\to A^\times:a\mapsto a^{-1}$ is differentiable.
\end{parts}
\end{prb}
\begin{pf}
(a)
We can show
\[(1-a)^{-1}=\sum_{k=0}^\infty a^k.\]
If $a$ is invertible, then $a+h=a(1+a^{-1}h)$ has the inverse $(1+a^{-1}h)^{-1}a^{-1}$ if $h$ is sufficiently small such that $\|a^{-1}h\|<1$.

(b)
Clear from
\[b^{-1}-a^{-1}=b^{-1}(a-b)a^{-1}.\]

(c)
\begin{align*}
\frac{\|b^{-1}-a^{-1}-(-a^{-1}(b-a)a^{-1})\|}{\|b-a\|}
&=\frac{\|(a^{-1}-b^{-1})(b-a)a^{-1}\|}{\|b-a\|}\\
&\le\|a^{-1}-b^{-1}\|\|a^{-1}\|\xrightarrow{b\to a}0.
\end{align*}
\end{pf}


\begin{prb}[Spectrum and resolvent]
Let $a$ be an element of a unital Banach algebra $A$.
The \emph{spectrum} of $a$ in $A$ is defined to be the set
\[\sigma_A(a):=\{\lambda\in\C:\lambda-a\text{ is not invertible in }A\},\]
and the \emph{resolvent} of $a$ in $A$ is defined to be its complement $\rho_A(a):=\C\setminus\sigma_A(a)$.
We can now define the \emph{resolvent map} $R:\rho_A(a)\to A$ such that
\[R(\lambda)=R(\lambda;a):=(\lambda-a)^{-1}.\]
If $A$ is clear in its context, we omit it to just write $\sigma(a)$ and $\rho(a)$.
\begin{parts}
\item $\sigma(a)$ is compact.
\item $\sigma(a)$ is non-empty.
\item If $A$ is a division ring, then $A\cong\C$. This result is called the \emph{Gelfand-Mazur theorem}.
\end{parts}
\end{prb}
\begin{pf}
(a)
If $|\lambda|>\|a\|$, then $\lambda-a$ is always invertible, so the spectrum is bounded.
Closedness follows from the fact that the set of invertibles is open.

(b)
Suppose the spectrum $\sigma(a)=\varnothing$ so that the resolvent function $R:\C\to A$ is well-defined on the entire $\C$.
Note that $a\ne0$.
Since $R$ is continuous and since
\[\|(\lambda-a)^{-1}\|=\|\lambda^{-1}(1-\lambda^{-1}a)^{-1}\|
=\Bigl\|\lambda^{-1}\sum_{k=0}^\infty(\lambda^{-1}a)^k\Bigr\|
<(2\|a\|)^{-1}\sum_{k=0}^\infty2^{-k}=\|a\|^{-1}\]
on $\{\lambda\in\C:|\lambda|>2\|a\|\}$, the function $R$ is bounded.
Also, for every $l\in A^*$ we have that the function $\C\to\C:\lambda\mapsto\<R(\lambda),l\>$ is holomorphic since $a\mapsto a^{-1}$ is differentiable.

Therefore, by the Liouville theorem, the bounded entire function $\lambda\mapsto\<R(\lambda),l\>$ is constant for all $l\in A^*$.
Because $A^*$ separates points of $A$, the function $R$ is constant, which implies $a\in\C$ and contradicts to $\sigma(a)=\varnothing$.

(c)
For any $a\in A$, by the part (b), there must be $\lambda$ such that $\lambda-a$ is not invertible.
In a division ring, zero is the only non-invertible element, so $\lambda=a$.
\end{pf}

\begin{prb}[Spectral radius]
Let $a$ be an element of a unital Banach algebra $A$.
The \emph{spectral radius} of $a$ in $A$ is defined to be
\[r(a):=\sup_{\lambda\in\sigma(a)}|\lambda|.\]
\begin{parts}
\item $r(a)\le\inf_n\|a^n\|^{\frac1n}$.
\item $\limsup_n\|a^n\|^{\frac1n}\le r(a)$, i.e. $r(a)=\lim_n\|a^n\|^{\frac1n}$.
\end{parts}
\end{prb}
\begin{pf}
(a)
Since $(\lambda-a)^{-1}=\lambda^{-1}(1-\lambda^{-1}a)^{-1}$ exists if $|\lambda|>\|a\|$, we have $r(a)\le\|a\|$ for all $a\in\cA$.
For every $\lambda\in\sigma(a)$ and every integer $n\ge1$ we have
\[|\lambda|^n=|\lambda^n|\le r(a^n)\le\|a^n\|,\]
and it proves $r(a)\le\inf_n\|a^n\|^{\frac1n}$.

(b)
Consider a holomorphic function
\[f:\{\lambda\in\C:|\lambda|>r(a)\}\to\C:\lambda\mapsto\<R(\lambda),l\>\]
for each $l\in A^*$.
Since on a smaller domain $\{\lambda\in\C:|\lambda|>\|a\|\}$, the function $f$ can be given by
\[f(\lambda)=\Bigl\<\lambda^{-1}\sum_{k=0}^\infty(\lambda^{-1}a)^k,l\Bigr\>,\]
we can determine the coefficients of the Laurent series of $f$ at infinity as
\[f(\lambda)=\sum_{k=0}^\infty\<a^k,l\>\lambda^{-k-1}\]
on $\{\lambda\in\C:|\lambda|>r(a)\}$.

Take $\lambda$ such that $|\lambda|>r(a)$.
Then, the sequence $(a^k\lambda^{-k-1})_k\in\cA$ is weakly bounded, hence is normly bounded by the uniform boundedness principle.
Let $\|a^n\|\le C_\lambda|\lambda^{n+1}|$ for all $n\ge1$.
Then,
\[\limsup_{n\to\infty}\|a^n\|^{\frac1n}\le\limsup_{n\to\infty}C_\lambda^{\frac1n}|\lambda^{n+1}|^{\frac1n}=|\lambda|.\]
If we limit $|\lambda|\downarrow r(a)$, we are done.
\end{pf}

\begin{prb}[Spectrum in closed subalgebras]
For fixed element, smaller the ambient algebra, less ``holes'' in the spectrum.
Let $A\subset B$ be a closed subalgebra containing $1_A$.
Note that $A$ may be unital even for $1_B\notin A$.
\begin{parts}
\item $B^\times$ is clopen in $A^\times\cap B$.
\end{parts}
\end{prb}





\section{Ideals}
\begin{prb}[Ideals]
\begin{parts}
\item If $I$ is a left ideal, then $A/I$ is a left $A$-module.
\end{parts}
\end{prb}

\begin{prb}[Modular left ideals]
A left ideal $I$ is called \emph{modular} if there is $e\in\cA$ such that $a-ae\in I$ for all $a\in A$.
The element $e$ is called a \emph{right modular unit} for $I$.
\begin{parts}
\item $I$ is modular if and only if $A/I$ is unital(?).
\item A proper modular left ideal is contained in a maximal left ideal.
\item $I$ is a maximal modular left ideal if and only if $I$ is a modular maximal left ideal.
\item There is a non-modular maximal ideal in the disk algebra.
\end{parts}
\end{prb}

\begin{prb}[Closed ideals]
\begin{parts}
\item closure of proper left ideal is proper left.
\item maximal modular left ideal is closed.
\end{parts}
\end{prb}


\begin{prb}[Unitization]
Let $A$ be an algebra.
Recall that we always assume algebras are associative.
Consider an embedding $A\to B(A):a\mapsto L_a$, where $L_a(b)=ab$.
Define
\[\tilde A:=\{\,L_a+\lambda\id_{B(A)}:a\in A,\lambda\in\C\,\}.\]
Note that this construction is available even for unital $A$.
\begin{parts}
\item If $A$ is normed, then $\tilde A$ is a normed algebra such that there is an isometric embedding $A\to\tilde A$.
\item If $A$ is Banach, then $\tilde A$ is a Banach algebra.
\item $A\oplus\C$ is topologically isomorphic to $\tilde A$ as normed spaces.
\end{parts}
\end{prb}
\begin{pf}
(a)
The space of bounded operators $B(A)$ is a normd algebra.
Then, $\tilde A$ is a normed $*$-algebra with induced norm
\[\|L_a+\lambda\id_{B(A)}\|=\sup_{b\in A}\frac{\|ab+\lambda b\|}{\|b\|}\]
Then, $A$ is a normed $*$-subalgebra of $\tilde A$ because the norm and involution of $A$ agree with $\tilde A$.

(b)
Suppose $(x_n,\lambda_n)$ is Cauchy in $\tilde A$.
Since $A$ is complete so that it is closed in $\tilde A$, we can induce a norm on the quotient $\tilde A/A$ so that the canonical projection is (uniformly) continuous so that $\lambda_n$ is Cauchy.
Also, the inequality $\|x\|\le\|(x,\lambda)\|+|\lambda|$ shows that $x_n$ is Cauchy in $A$.

Since a finite dimensional normed space is always Banach and $A$ is Banach, $\lambda_n$ and $x_n$ converge.
Finally, the inequality $\|(x,\lambda)\|\le\|x\|+|\lambda|$ implies that $(x_n,\lambda_n)$ converges.

(c)
Check the topology on $A\oplus\C$ in detail...
\end{pf}



unitization, homomorphisms, category(direct sum, product, etc.)

$B(\C^n)=M_n(\C)$ is simple, but $B(H)$ is not simple.

% approximate identity, norm of left multiplication


\section{Holomorphic functional calculus}

Fr\'echet space valued 

\begin{prb}[Holomorphic functional calculus]
Let $a$ be an element of a unital Banach algebra $A$.
Let $f$ be a holomorphic function on a neighborhood $U$ of $\sigma(a)$.
Let $C$ be a positively oriented smooth simple closed curve in $U$ enclosing $\sigma(a)$.
Define $f(a)\in A^{**}$ as the Dunford integral
\[\<f(a),l\>:=\int_Cf(\lambda)\<(\lambda-a)^{-1},l\>\,d\lambda,\qquad l\in A^*.\]

Let $\cO(\sigma(a))$ be the space of all holomorphic functions on a neighborhood of $\sigma(a)$ endowed with the topology of compact convergence.
Note that $\cO(\sigma(a))$ is a Fr\'echet algebra, but not Banach.
We define the \emph{holomorphic functional calculus} by the map
\[\cO(\sigma(a))\to A:f\mapsto f(a).\]
It is also called the Riesz or the \emph{Riesz-Dunford functional calculus}.
\begin{parts}
\item $f(a)\in A$, i.e. $f(a)$ is in fact given by the Pettis integral.
\item $f(a)$ is independent of the choice of $C$.
\item The functional calculus is an algebra homomorphism.
\item The functional calculus is bounded.
\item injective.
\item unital and $\id_\C\mapsto a$.
\item spectral mapping.
\item power series.
\end{parts}
\end{prb}
\begin{pf}
(a)


\end{pf}





\section{Gelfand theory}

Banach algebra of single generator
semisimplicity and symmetricity

\begin{prb}[Spectrum of a Banach algebra]
Let $A$ be a commutative Banach algbera.
A \emph{character} of $A$ is a non-trivial algebra homomorphism $\pi:A\to\C$.
Denote by $\sigma(A)$ the set of all characters of $A$ and endow with the weak$^*$ topology on $\sigma(A)\subset A^*$.
We call this space as the \emph{spectrum} of $A$.
\begin{parts}
\item If $A$ is unital, $\sigma(A)$ is contained in the unit sphere of $A^*$.
\item $\sigma(A)$ is locally compact and Hausdorff.
\end{parts}
\end{prb}
\begin{pf}

\end{pf}


\begin{prb}[Gelfand transform]
Let $A$ be a commutative Banach algebra.
The \emph{Gelfand transform} or the \emph{Gelfand representation} is the following algebra homomorphism
\[\Gamma:A\to C_0(\sigma(A)):a\mapsto(\pi\mapsto\pi(a)).\]
\begin{parts}
\item $\Gamma$ has the image separating points by definition.
\item $\Gamma$ has closed range if $A$ is a symmetric Banach $*$-algebra.
\item $\Gamma$ is injective if and only if $A$ is semisimple.
\item $\Gamma$ is isometric if and only if $r(a)=\|a\|$ for all $a\in A$.
\end{parts}
\end{prb}





\section*{Exercises}
\begin{prb}[Basic properties of spectrum]
Let $A$ be a unital algebra.
\begin{parts}
\item $\sigma(ab)\setminus\{0\}=\sigma(ba)\setminus\{0\}$.
\item If $\sigma(a)$ is non-empty, then $\sigma(p(a))=p(\sigma(a))$.
\end{parts}
\end{prb}
\begin{pf}
(a)
Intuitively, the inverse of $1-ab$ is $c=1+ab+abab+\cdots$.
Then, $1+bca=1+ba+baba+\cdots$ is the inverse of $1-ba$.
\end{pf}

$C_b(\Omega)$ $\ell^\infty(S)$ $L^\infty(\Omega)$ $B_b(\Omega)$ $A(\D)$
$B(X)$

\begin{prb}
In $C(\R)$, the modular ideals correspond to compact sets.
\end{prb}

\begin{prb}[Disk algebra]
\begin{parts}
\item Every continuous homomorphism is an evaluation.
\end{parts}
\end{prb}

\begin{prb}[Polynomial convexity]
(See Conway)
\end{prb}

\begin{prb}[Inclusion relation on spectra]
\begin{parts}
\item $\sigma(a+b)\subset\sigma(a)+\sigma(b)$ and $\sigma(ab)\subset\sigma(a)\sigma(b)$ for unital cases.
\item $\sigma(a^{-1})=\sigma(a)^{-1}$ for unital cases.
\item $r(a)^n=r(a^n)$.
\end{parts}
\end{prb}

\begin{prb}[Spectral radius function]
\begin{parts}
\item upper semi-continuous
\end{parts}
\end{prb}

\begin{prb}[Vector-valued complex function theory]
Let $\Omega$ be an open subset of $\C$ and $X$ a Banach space.
For a vector-valued function $f:\Omega\to X$, we say $f$ is \emph{differentiable} if the limit
\[\lim_{\lambda\to\lambda_0}(\lambda-\lambda_0)^{-1}(f(\lambda)-f(\lambda_0))\]
exists in $X$ for every $\lambda\in\Omega$, and \emph{weakly differentiable} if the limit
\[\lim_{\lambda\to\lambda_0}(\lambda-\lambda_0)^{-1}\<f(\lambda)-f(\lambda_0),x^*\>\]
exists in $\C$ for each $x^*\in X^*$ and every $\lambda\in\Omega$.
Then, the followings are all equivalent.
\begin{parts}
\item $f$ is differentiable.
\item $f$ is weakly differentiable.
\item For each $\lambda_0\in\Omega$, there is a sequence $(x_k)_{k=0}^\infty$ such that we have the power series expansion
\[f(\lambda)=\sum_{k=0}^\infty(\lambda-\lambda_0)^kx_k,\]
where the series on the right hand side converges absolutely and uniformly on any closed ball in $\Omega$ centered at $\lambda_0$.
\end{parts}
\end{prb}

\begin{prb}[Exponential of an operator]
\end{prb}







\chapter{C$^*$-algebras}

\section{C$^*$ identity}
% normal elements, real/imaginary part
\begin{prb}[$*$-algebras]
normed?
\end{prb}


\begin{prb}[C$^*$-identity]
%history
A \emph{C$^*$-algebra} is a Banach $*$-algebra $A$ satisfying the C$^*$-identity $\|a^*a\|=\|a\|^2$ for all $a\in A$.
\end{prb}


\begin{prb}[Unitization]
\[(L_a+\lambda\id_{B(A)})^*=L_{a^*}+\bar\lambda\id_{B(A)}.\]
\end{prb}
\begin{pf}
The C$^*$-identity easily follows from the following inequality:
\begin{align*}
\|(a,\lambda)\|^2&=\sup_{\|b\|=1}\|ab+\lambda b\|^2\\
&=\sup_{\|b\|=1}\|(ab+\lambda b)^*(ab+\lambda b)\|\\
&=\sup_{\|b\|=1}\|b^*((a^*a+\lambda a^*+\bar\lambda a)b+|\lambda|^2y)\|\\
&\le\sup_{\|b\|=1}\|(a^*a+\lambda a^*+\bar\lambda a)b+|\lambda|^2b\|\\
&=\|(a,\lambda)^*(a,\lambda)\|.\qedhere
\end{align*}
\end{pf}





\section{Continuous functional calculus}

\begin{prb}[Gelfand-Naimark representation for C$^*$-algebras]
For a commutative C$^*$-algebra $A$, consider the Gelfand transform $\Gamma:A\to C_0(\sigma(A))$.
\begin{parts}
\item $\Gamma$ is a $*$-homomorphism.
\item $\Gamma$ is an isometry.
\item $\Gamma$ is a $*$-isomorphism.
\end{parts}
\end{prb}
\begin{pf}
(a)

(b)
Note that we have
\[\|\Gamma a\|=\sup_{\f\in\sigma(A)}|\Gamma a(\f)|=\sup_{\f\in\sigma(A)}|\f(a)|=r(a)\]
for all $a\in A$.
If we assume $a$ is self-adjoint, then since $\|a\|^2=\|a^*a\|=\|a^2\|$, the spectral radius coincides with the norm by the Beurling formula for spectral radius in Banach algebras:
\[\|\Gamma a\|=r(a)=\lim_{n\to\infty}\|a^{2^n}\|^{1/2^n}=\|a\|.\]
Hence we have for all $a\in A$ that
\[\|a\|^2=\|a^*a\|=\|\Gamma(a^*a)\|=\|(\Gamma a)^*(\Gamma a)\|=\|\Gamma a\|^2.\]

(c)
By the part (a) and (b), the image $\Gamma(A)$ is a closed unital $*$-subalgebra of $C(\sigma(A))$, and it separates points by definition.
Then, $\Gamma(A)$ is dense in $C(\sigma(A))$ by the Stone-Weierstrass theorem, which implies $\Gamma(A)=C(\sigma(A)$.
\end{pf}



\begin{prb}[Generators of a C$^*$-algebra]
joint spectrum.
\end{prb}


\begin{prb}[Continuous functional calculus]
Let $A$ be a unital C$^*$-algebra, and $a\in A$ a normal element.
Then, we have a $*$-isomorphism
\[C(\sigma(a))\to\tilde C^*(1,a):\id_{\sigma(a)}\mapsto a\]
defined by the inverse of the Gelfand transform, which we call the \emph{continuous functional calculus}.

\begin{parts}
\item spectral mapping: $\lambda\in\sigma_p(a)$ implies $f(\lambda)\in\sigma_p(f(a))$, $\lambda\in\sigma(a)$ iff $f(\lambda)\in\sigma(f(a))$, composition, ...
\end{parts}
\end{prb}




\begin{prb}[Normal elements]
Let $a$ be an element of a unital C$^*$-algebra $A$.
We say $a$ is \emph{normal}, \emph{unitary}, and \emph{self-adjoint} if $a^*a=aa^*$, $a^*a=aa^*=e$, and $a^*=a$ respectively.
For normality and self-adjointness, the definitions can be extended to non-unital C$^*$-algebras.
\begin{parts}
\item If $a$ is normal, then $a$ is unitary if and only if $\sigma(a)\subset\T$.
\item If $a$ is normal, then $a$ is self-adjoint if and only if $\sigma(a)\subset\R$.
\end{parts}
\end{prb}
\begin{pf}
(a)

(b)
We may assume $A$ is unital.
By the holomorphic functional calculus, we have
\[e^{ia}=\sum_{n=1}^\infty\frac{(ia)^n}{n!}\in A,\]
and the inverse of $e^{ia}$ is $e^{-ia}$.
Since the involution on $A$ is continuous, we can check $e^{ia}$ is unitary by
\[(e^{ia})^*=\sum_{n=1}^\infty\frac{(-ia)^n}{n!}=e^{-ia}.\]
For every $\f\in\sigma(A)$, then by the part (a) the equality
\[e^{-\Im\f(a)}=|e^{i\f(a)}|=|\f(e^{ia})|=1\]
proves $\f(a)\in\R$, hence $\sigma(a)\subset\R$.
\end{pf}

\begin{prb}[$*$-homomorphism]
Let $\f:A\to B$ be a $*$-homomorphism between C$^*$-algerbas.
\begin{parts}
\item $\f$ is determined by self-adjoint elements.
\item $\|\f\|=1$ if $\f$ is non-trivial.
\item The iamge of $\f$ is closed.
\item The induced map $A/\ker\f\to B$ is an isometry.
\end{parts}
\end{prb}



\section{Positive elements}


\begin{prb}[Positive elements]
Let $a,b$ be elements of a C$^*$-algebra $A$.
We say $a$ is \emph{positive} and write $a\ge0$ if it is normal and $\sigma(a)\subset\R_{\ge0}$.
If we define a relation $a\le b$ as $b-a\ge0$, then we can see that it is a partial order on $A$.
\begin{parts}
\item $a\ge0$ if and only if $\|\lambda-a\|\le\lambda$ for some $\lambda\ge\|a\|$.
\item If $a\ge0$ and $\sigma(b)\subset\R_{\ge0}$, then $\sigma(a+b)\subset\R_{\ge0}$.
\item $a\ge0$ if and only if $a=b^*b$ for some $b\in A$.
\end{parts}
\end{prb}
\begin{pf}


Let $a:=b^*b$.
Let $a=a_+-a_-$.
Then we have $(ba_-)^*(ba_-)=a_-aa_-=-a_-^3\le0$, which also implies $(ba_-)(ba_-)^*\le0$ and
\[0\le(ba_-)^*(ba_-)+(ba_-)(ba_-)^*\le0.\]
Thus we have $ba_-=0$ and $a_-^3=0$.

\end{pf}

% Absolute value of an operator



\begin{prb}[Operator monotone operations]
\begin{parts}
\item If $0\le a\le b$, then $a^{-1}\ge b^{-1}$.
\item If $a\le b$, then $cac^*\le cbc^*$.
\end{parts}
\end{prb}



\begin{prb}[Positive linear functionals]
Let $A$ be a C$^*$-algebra.
A \emph{state} of $A$ is a positive linear functional $\omega$ such that $\|\omega\|=1$.
\begin{parts}
\item For a normal element $a\in A$ there is a state $\omega$ such that $|\omega(a)|=\|a\|$.
\item A self-adjoint linear functional is the difference of two positive linear functional. It is called the \emph{Jordan decomposition}.
\end{parts}
\end{prb}
\begin{pf}
(b)
We first show the real dual $(A^{sa})^*$ can be identified with the self adjoint part $(A^*)^{sa}$ of the complex dual.
By this identification, we can describe the weak$^*$ topology on $(A^*)^{sa}$ as $\sigma((A^*)^{sa},A^{sa})$.

We may assume $A$ is unital.
The closed unit ball of the real Banach space $(A^*)^{sa}$ is weakly$^*$ compact.
We are enough to show
\[(A^*)^{sa}_1=\bar\conv(S(A)\cup-S(A)),\]
where the closure is taken in the weak$^*$ topology, because $S(A)$ and $-S(A)$ are weakly$^*$ compact and convex due to the unit of $A$, the closure on the right-hand side is not necessary.
Suppose not and take $l\in(A^*)^{sa}_1$ which is not approximated weakly$^*$ by $\conv(S(A)\cup-S(A))$.
By the Hahn-Banach separation, there is $a\in A^{sa}$ such that
\[\sup_{\omega\in S(A)\cup-S(A)}\omega(a)<l(a).\]
If we take $\omega\in S(A)\cup-S(A)$ such that $\omega(a)=\|a\|$ using the part (a), then we get a contradiction to the bound $\|l\|\le1$.


\end{pf}



\begin{prb}[Approximate identity]
Let $e_\alpha$ be an approximate identity of $A$.
\begin{parts}
\item Exists.
\item For a positive linear functional $\omega$, we have $\lim_\alpha\omega(e_\alpha)=\|\omega\|$.
\item 
\item separable.
\end{parts}
\end{prb}




\section{Representations of C$^*$-algebras}


\begin{prb}[Non-degenerate representations]
Let $A$ be a C$^*$-algebra.
A \emph{representation} of $A$ on a Hilbert space $H$ is a $*$-homomorphism $\pi:A\to B(H)$.
We say a representation $\pi:A\to B(H)$ is \emph{non-degenerate} if $\pi(A)H$ is dense in $H$.
\begin{parts}
\item Every representation has a unique non-degenerate subrepresentation.
\item The following statements are equivalent:
\begin{enumerate}[(i)]
\item $\pi$ is non-degenerate.
\item For each $\xi\in H$ there is $a\in A$ such that $\pi(a)\xi\ne0$.
\item $\pi(e_\alpha)\to\id_H$ strongly for an approximate identity $e_\alpha$ of $A$.
\end{enumerate}
\end{parts}
\end{prb}

\begin{prb}[Cyclic representations]
\emph{cyclic} if there is a vector $\psi\in H$ such that $A\psi$ is dense in $H$.
Cyclic decomposition
\end{prb}


\begin{prb}[Irreducible representations]
\emph{irreducible} if there is no proper closed subspace $K\subset H$ such that $\pi(A)K\subset K$.
The following statements are equivalent:
\begin{enumerate}[(i)]
\item $\pi$ is irreducible.
\item $\pi(A)'=\C\id_H$.
\item $\pi(A)$ is strongly dense in $B(H)$.
\item Every non-zero vector in $H$ is cyclic.
\end{enumerate}
\end{prb}


\begin{prb}[Gelfand-Naimark-Segal representation]
Let $A$ be a C$^*$-algebra, and $\omega$ be a state on $A$.
The \emph{left kernel} of $\omega$ is defined to be
\[N_\omega:=\{a\in A:\omega(a^*a)=0\}.\]
\begin{parts}
\item $N_\omega$ is a left ideal of $A$.
\item $\<a+N,b+N\>:=\omega(b^*a)$ is an inner product on $A/N_\omega$.
\item There is a unique representation $\pi_\omega:A\to B(H_\omega)$ such that $\pi_\omega(a)(b+N_\omega):=ab+N_\omega$ for $a,b\in A$.
\item $\pi_\omega:A\to B(H_\omega)$ is a cyclic representation.
\end{parts}
\end{prb}








\section*{Exercises}

% Basic examples
%  C(X), C_0(X)
%  M_n(\C)
%  B(H), K(H), Q(H)

% Schroder-Burnstein thm of representations


\begin{prb}[Projections in $M_2(\C)$]
The space of self-adjoint elements in $M_2(\C)$ is a real vector space spanned by
\[1=\begin{pmatrix}1&0\\0&1\end{pmatrix},\qquad p:=\begin{pmatrix}1&0\\0&0\end{pmatrix},\qquad q:=\frac12\begin{pmatrix}1&1\\1&1\end{pmatrix}.\]
\begin{parts}
\item $(p-q)^2=\frac12$.
\item If we let $\lambda_\pm$ be the eigenvalues of $ap+bq$, then $\lambda_++\lambda_-=a+b$ and $\lambda_+-\lambda_-=\sqrt{a^2+b^2}$.
\item Every functional calculus $f(x)$ of self-adjoint $x$ is a linear combination of $x$ and 1.
\item $ap+bq+c\ge0$ if and only if $a+b+2c\ge\sqrt{a^2+b^2}$.
\item Every projection of rank one is given by $ap+bq+(1-a-b)/2$ for $a^2+b^2=1$.
\end{parts}
\end{prb}

\begin{prb}[Operator monotone square]
Let $A$ be a C$^*$-algebra in which the square function is operator monotone, that is, $0\le a\le b$ implies $a^2\le b^2$ for any positive elements $a$ and $b$ in $A$.
We are going to show that $A$ is necessarily commutative.
Let $a$ and $b$ denote arbitrary positive elements of $A$.
\begin{parts}
\item
Show that $ab+ba\ge0$.
\item
Let $ab=c+id$ where $c$ and $d$ are self adjoints.
Show that $d^2\le c^2$.
\item
Suppose $\lambda>0$ satisfies $\lambda d^2\le c^2$.
Show that $c^2d^2+d^2c^2-2\lambda d^4\ge0$.
\item
Show that $\lambda(cd+dc)^2\le(c^2-d^2)^2$.
\item
Show that $\sqrt{\lambda^2+2\lambda-1}\cdot d^2\le c^2$ and deduce $d=0$.
\item
Extend the result for general exponent: $A$ is commitative if $f(x)=x^\beta$ is operator monotone for $\beta>1$.
\end{parts}
\end{prb}


\begin{prb}[States on unitization]
Let $A$ be a non-unital C$^*$-algebra and $\tilde A$ be its unitization.
Let $\tilde\omega=\omega\oplus\lambda$ be a bounded linear functional on $\tilde A$, where $\omega\in A^*$ and $\lambda\in\C^*=\C$.

Since $A$ is hereditary in $\tilde A$, the extension defines a well-defined injective map $S(A)\to S(\tilde A)$.
We can identify $PS(A)$ as a subset of $PS(\tilde A)$ whose complement is a singleton.
\begin{parts}
\item $\tilde\rho$ is positive if and only if $\lambda\ge0$ and $0\le\rho\le\lambda$.
\item $\tilde\omega$ is a state if and only if $\lambda=1$ and $0\le\omega\le1$.
\item $\tilde\omega$ is a pure state if and only if $\lambda=1$ and $\omega$ is either a pure state or zero.
\end{parts}
\end{prb}


\begin{prb}[Representations of $C_0(X)$]
Let $A=C_0(X)$ and $\mu$ be a state on $A$, a regular Borel probability measure on a locally compact Hausdorff space $X$.
\begin{parts}
\item The left kernel of $\mu$ is $N_\mu=\{\,f\in A:f|_{\supp\mu}=0\,\}$.
\item $H_\mu=L^2(X,\mu)$.
\item The canonical cyclic vector is the unity function on $X$.
\end{parts}
\end{prb}

\begin{prb}[Representations of $K(H)$]
\end{prb}

\begin{prb}[Automorphism group of $K(H)$ and $B(H)$]
\end{prb}


\begin{prb}[Approximate eigenvectors]
\end{prb}


\begin{prb}[Kadison transitivity theorem]
\end{prb}

\begin{prb}[Hereditary C$^*$-algebras]
\end{prb}

\begin{prb}[Extreme points of the ball]
Let $A$ be a C$^*$-algebra and let $B_A$ be the closed unit ball of $A$.
\begin{parts}
\item Extreme points of $A_+\cap B_A$ is the projections in $A$.
\item Extreme points of $A_{sa}\cap B_A$ is the self-adjoint unitaries in $A$.
\item Every extreme point of $B_A$ is a partial isometry.
\end{parts}
\end{prb}

\section*{Problems}
\begin{enumerate}
\item* A C$^*$-algebra is commutative if and only if a function $f(x)=x(1+x)^{-1}$ is operator subadditive.
%L\"owner-Heinz inequality
\end{enumerate}







\chapter{Von Neumann algebras}


\section{Density theorems}


\begin{prb}[Von Neumann algebras]
A \emph{von Neumann algebra} on a Hilbert space $H$ is a $\sigma$-weakly closed $*$-subalgebra of $B(H)$ including $\id_H$.
A positive linear map $\f$ between von Neumann algebras is said to be \emph{normal} if $\f(\sup_\alpha x_\alpha)=\sup_\alpha\f(x_\alpha)$ for any bounded increasing net $x_\alpha$ of positive elements.
\begin{parts}
\item A positive map $\f$ is normal if and only if it is continuous between $\sigma$-weak topologies.
\end{parts}
\end{prb}

\begin{prb}[Normal states]
Let $N\subset M\subset B(H)$ be von Neumann algebras.
The space of $\sigma$-weakly continuous linear functionals on $M$ is denoted by $M_*$.
\begin{parts}
\item $M_*$ is a predual of $M$.
\item The restriction of a normal state of $M$ on $N$ is normal.
\item A normal state of $N$ is extended to a normal state of $M$.
\item A state $\omega$ of $M$ is normal if and only if $\omega(x)=\sum_{i=1}^\infty\<x\xi_i,\xi_i\>$ for some $(\xi_i)\in\ell^2(\N,H)$.
\item The GNS representation of a normal state is normal.
\end{parts}
\end{prb}

\begin{prb}[Double commutant theorem]
The \emph{commutant} of a subset $A\subset B(H)$, denoted by $A'$, is the set of all elements of $B(H)$ that commute every $a\in A$.
Suppose $A$ is a non-degenerate $*$-subalgebra of $B(H)$.
One can describe the von Neumann algebra generated by $A$ in $B(H)$ purely algebraically in terms of commutants.
\begin{parts}
\item $A''$ is weakly closed $*$-algebra.
\item If $x\in A''$, for any $\e>0$ and $\xi\in H$ there is $a\in A$ such that $\|(x-a)\xi\|<\e$.
\item $A$ is $\sigma$-strongly$^*$ dense in $A''$.
\end{parts}
\end{prb}
\begin{pf}
(a)
Suppose a net $x_\alpha\in A''$ weakly converges to $x\in B(H)$.
For any $y\in A'$,
\[\<xy\xi,\eta\>=\lim_\alpha\<x_\alpha y\xi,\eta\>=\lim_\alpha\<yx_\alpha\xi,\eta\>=\<yx\xi,\eta\>,\qquad \xi,\eta\in H.\]
Hence $x\in A''$.

(b)
We claim $x\xi\in\bar{A\xi}$ for each $\xi\in H$.
Let $p$ be the projection onto $\bar{A\xi}$.
For any $a\in A$, the operator $ap$ ranges into $\bar{A\xi}$ so that $pap=ap$, and we also have $pa^*p=a^*p$ by the self-adjointness of $A$.
It implies $ap=pa$, which deduces $p\in A'$.
Thus $xp=px$ for $x\in A''$.
On the other hand, observe that $a(1-p)\xi=(1-p)a\xi=0$ for all $a\in A$.
Then, $\<(1-p)\xi,\eta\>=0$ for any $\eta\in H=\bar{AH}$ by the non-degeneracy, so $p\xi=\xi$.
Combining $xp=px$ and $p\xi=\xi$, we obtain $x\xi=xp\xi=px\xi$ so that $x\xi\in\bar{A\xi}$.

(c)
It suffices to show $A$ is $\sigma$-strongly dense in $A''$ because $A$ is self-adjoint.
Consider $A$ as the non-degenerate $*$-subalgebra of $B(\ell^2(\N,H))$ via the diagonal map $B(H)\to B(\ell^2(\N,H))$, which is a injective normal unital $*$-homomorphism.
We can check that $A''$ does not change if we replace $B(H)$ to $B(\ell^2(\N,H))$.
By applying the part (b) for arbitrary $\xi\in\ell^2(\N,H)$, we deduce the desired result.
\end{pf}


\begin{prb}[Kaplansky density theorem]
\end{prb}







\section{Borel functional calculus}

\begin{prb}[Sherman-Takeda theorem]
Let $A$ be a C$^*$-algebra.
Define $M(\pi):=\pi(A)''$ for $\pi:A\to B(H)$ a representation.
Let $\pi_u:A\to B(H_u)$ be the universal representation of $A$, the direct sum of all the GNS-representations of states of $A$.
Consider the following three maps
\[\pi_u:A\to(M(\pi_u),\sigma w),\qquad\pi_u^*:M(\pi_u)_*\to A^*,\qquad\pi_u^{**}:A^{**}\to M(\pi_u),\]
constructed by adjoints.
\begin{parts}
\item $\pi_u^*$ is isometric.
\item $\pi_u^*$ is surjective.
In particular, $\pi_u^{**}$ is a normal $*$-isomorphsim.
\item $A^{**}$ enjoys a universal property in the sense that every $*$-homomorphism $\f:A\to M$ to a von Neumann algebra $M$ has a unique normal extension $\tilde\f:A^{**}\to M$ of $\f$.
\end{parts}
\end{prb}
\begin{pf}
(a)
It holds for any representation of $\pi:A\to B(H)$.
For each $l\in M(\pi)_*$ we have
\[\|\pi^*(l)\|=\sup_{\substack{\|a\|\le1\\a\in A}}|l(\pi(a))|=\sup_{\substack{\|x\|\le1\\x\in M(\pi)}}|l(x)|=\|l\|\] by the Kaplansky density theorem and the $\sigma$-weak continuity of $l$.

(b)
Let $\omega$ be a state of $A$.
Since the universal representation $\pi_u$ has the GNS representation of $\omega$ as a subrepresentation, $\omega$ is given by a vector state in $\pi_u$.
By restriction of this vector state, we have a normal state of $M(\pi_u)$, which extends $\omega$.
Now the Jordan decomposition can be applied to verify that every bounded linear functional of $A$ has a $\sigma$-weakly continuous extension on $M(\pi_u)$.

(c)
We can define $\tilde\f$ as the bitranspose of $\f:A\to(M,\sigma w)$, and it is a unique extension because $A$ is $\sigma$-weakly dense in $A^{**}$.
\end{pf}
\begin{rmk}
The bidual $A^{**}$ is frequently viewed as a von Neumann algebra, and we call it the \emph{enveloping von Neumann algebra} of a C$^*$-algebra $A$.
By the universal property, we have a normal $*$-homomorphism $M(\pi_u)\to M(\pi)$ that is in fact surjective for every representation $\pi$ of $A$, and it fails to be injective even if $\pi$ is faithful.
\end{rmk}


\begin{prb}[Bounded Borel functions]
Let $X$ be a compact Hausdorff space and denote by $B^\infty(X)$ the space of bounded Borel functions on $X$.
The linear combinations of projections in $B^\infty(X)$ are called \emph{simple functions}.
\begin{parts}
\item There are natural inclusions $C(X)\subset B^\infty(X)\subset C(X)^{**}$ among C$^*$-algebras.
\item $B^\infty(X)$ is the norm closure of simple functions.
\item $B^\infty(X)$ factors through all $L^\infty(X,\mu):=M(\pi_\mu)$ for GNS-representations $\pi_\mu$ of $C(X)$.
\end{parts}
\end{prb}

\begin{prb}[Borel functional calculus]
Let $x\in B(H)$ be a normal operator.
Consider
\[B^\infty(\sigma(x))\subset C(\sigma(x))^{**}\to W^*(x)\subset B(H).\]
\begin{parts}
\item If we endow the topology of pointwise convergence on $B^\infty(\sigma(a))$ and the strong operator topology on $M$, then the Borel functional calculus is continuous.
\item Every von Neumann algebra is the norm closed span of projections.
\end{parts}
\end{prb}
\begin{pf}
(a)
By the bounded convergence theorem.

(b)
This is because $\sigma(a)\subset\C$ is compact so that it is separable and metrizable; every bounded measurable function is a pointwise limit of simple functions.

\end{pf}



For normal $a\in B(H)$, the continuous functional calculus for $a$ is just a non-degenerate representation
\[C(\sigma(a))\to B(H)\]
which maps $\id_{\sigma(a)}$ to $a$.
Also, a projection valued-measure on a compact Hausdorff space $X$ is just a non-degenerate representation
\[C(X)\to B(H).\]
To show this, note that a projection-valued measure defines a ``normal'' unital $*$-homomorphism
\[\spn P(B^\infty(X))\to B(H).\]
Then, mimick the definition of Lebesgue integral to construct a unital $*$-homomorphism $C(X)\to B(H)$.




\section{Predual}



\begin{prb}[Conditional expectations]
Let $A$ be a closed subalgebra of a C$^*$-algebra $B$.
Let $\f:B\to A$ be a contractive idempotent surjective linear map.
Such a map is called a \emph{conditional expectation}.
\begin{parts}
\item $\f$ is an $A$-bimodule map.
\item $\f$ is completely positive.
\end{parts}
\end{prb}

\begin{pf}
Since each conclusion of (a) and (b) still holds for restriction, we may assume $A$ and $B$ are von Neumann algebras by thinking of the bitranspose $\f^{**}:B^{**}\to A^{**}$.

(a)
Since the linear span of projections is $\sigma$-weakly dense in a von Neumann algebra, we are enough to show $p\f(b)=\f(pb)$ and $\f(bp)=\f(b)p$ for any projection $p\in A$.

Let $p\in A$ be a projection and let $b\in B$.
Note that the surjectivity of $\f$ implies that $p\f$ is also idempotent.
Then, where $1=1_B$,
\begin{align*}
(1+t)^2\|p\f((1-p)b)\|^2
&=\|p\f((1-p)b)+tp\f(p\f((1-p)b))\|^2\\
&\le\|(1-p)b+tp\f((1-p)b)\|^2\\
&=\|(1-p)b\|^2+t^2\|p\f((1-p)b)\|^2
\end{align*}
implies $p\f((1-p)b)=0$ by letting $t\to\infty$.
Putting $1_A-p$ and $1_A$ instead of $p$, we obtain
\[(1-p)\f((1-1_A+p)b)=0,\qquad\f((1-1_A)b)=0\]
respectively, which imply $(1-p)\f(pb)=0$.
Hence for any $b\in B$ we have
\[p\f(b)=p\f(pb)=\f(pb).\]
Similarly we can show $\f(b(1-p))p=0$ and $\f(bp)(1-p)=0$ for $b\in B$, we are done.

(b)
Let $[b_{ij}]\in M_n(B)_+$.
Let $\pi:A\to B(H)$ be a cyclic representation with a cyclic vector $\psi$.
Then, $[\xi_i]\in H^n$ can be replaced to $[\pi(a_i)\psi]$, so we can check the positivity of inflations $\f_n$ as
\[\sum_{i,j}\<\pi(\f(b_{ij}))\pi(a_j)\psi,\pi(a_i)\psi\>=\<\pi(\f(\sum_{i,j}a_i^*b_{ij}a_j))\psi,\psi\>\ge0,\]
because it follows $\sum_{i,j}a_i^*b_{ij}a_j\ge0$ by the positivity of $b_{ij}$ from
\[\<\pi_B(\sum_{i,j}a_i^*b_{ij}a_j)\xi,\xi\>=\sum_{i,j}\<\pi_B(b_{ij})\pi_B(a_j)\xi,\pi_B(a_i)\xi\>\ge0,\]
where $\pi_B$ is any representation of $B$.
\end{pf}



\begin{prb}[Sakai theorem]
Suppose $A$ is a C$^*$-algebra which admits a predual $F$.
\begin{parts}
\item There is an injective $*$-homomorphism $\pi:A\to A^{**}$ with weakly$^*$ closed image.
\item $\pi$ is a topological embedding with respect to $\sigma(A,F)$ and $\sigma(A^{**},A^*)$.
\item The predual $F$ is unique in $A^*$.
\end{parts}
In particular, since $A^{**}$ admits a faithful normal representation, so does $A$.
\end{prb}
\begin{pf}
(a)
By taking the adjoint for the inclusion $i:F\hookrightarrow A^*$, we have a conditional expectation $\e:A^{**}\twoheadrightarrow A$.
Its kernel is a $A$-bimodule, and by the $\sigma$-weak density of $A$ in $A^{**}$ and the continuity of $\e$ between weak$^*$ topologies, so it is in fact a $A^{**}$-bimodule, which means it is a $\sigma$-weakly closed ideal of $A^{**}$.
Thus we have a central projection $z\in A^{**}$ such that $\ker\e=(1-z)A^{**}$.

Define $\pi:A\to A^{**}$ such that $\pi(a):=za$.
It is clearly a $*$-homomorphism.
The injectivity follows from $a=\e(a)=\e(za)$ for $a\in A$.
The image is weakly$^*$ closed because $\e(x-\e(x))=0$ implies $z(x-\e(x))=0$ for $x\in A^{**}$ so that $zA^{**}=zA$.

(b)
Since $\<a,f\>=\<\e(za),f\>=\<za,f\>$ for $a\in A$ and $f\in F$, in which the second equality holds by the definition of $\e$, it is enough to show $\sigma(zA,A^*)=\sigma(zA,F)$.

For $l\in A^*$, we claim there exists $f$ such that $\<za,l\>=\<za,f\>$.
Define $\tilde l\in A^*$ such that $\<x,\tilde l\>:=\<zx,l\>$ for $x\in A^{**}$.
Then, $\<zx,l\>=\<z^2x,l\>=\<zx,\tilde l\>$ for $x\in A^{**}$.
Suppose $\tilde l\notin F$.
Because $F$ is closed in $A^*$, there is $x\in A^{**}$ such that $\<x,\tilde l\>\ne0$ and $\<x,f\>=0$ for all $f\in F$ by the Hahn-Banach separation.
Then, $0=\<x,f\>=\<x,i(f)\>=\<\e(x),f\>$ implies $\e(x)=0$ so that $zx=0$, which leads a contradiction $\<x,\tilde l\>=\<zx,l\>=0$, so we have $\tilde l\in F$.

(c)
If closed subspaces $F_1$ and $F_2$ of $A^*$ are preduals of $A$, then $\sigma(A,F_1)=\sigma(A,F_2)$ by the part (b).
If $l\in F_1$, which is obviously continuous on $\sigma(A,F_1)$, and the continuity in $\sigma(A,F_2)$ implies that $l$ is contained in a linear span of some finitely many elements of $F_2$, hence $F_1\subset F_2$.
\end{pf}











\section*{Exercises}
\begin{prb}[Extremally disconnected space]
$\sigma(B^\infty(\Omega))$ is extremally disconnected.
\end{prb}

resolution of identity
normal operator theories: multiplicity, invariant subspaces
$L^\infty$ representation


$\sigma$-weakly closed left ideal has the form $Mp$. II.3.12

Let $\fm$ be an algebraic ideal of a von Neumann algebra $M$, and $\bar\fm$ be its $\sigma$-weak closure.
If $x\in(\bar\fm)_+$, then there is an increasing net $(x_i)\subset\fm$ converges to $x$ strongly. II.3.13



binary expansion and hereditary subalgebras

\end{document}