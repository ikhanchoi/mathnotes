\documentclass{../../large}
\usepackage{../../ikhanchoi}

\newcommand{\Prim}{\operatorname{Prim}}

\begin{document}
\title{Functional Analysis}
\author{Ikhan Choi}
\maketitle
\tableofcontents

\part{Topological vector spaces}


\chapter{Locally convex spaces}
\section{Vector topologies}

\begin{prb}[Topological vector spaces]
A vector space will always mean a real or complex vector space.
We will define a \emph{topological vector space} as a vector space together with a Hausdorff vector topology.
For every topological vector space, there is a balanced neighborhood system at zero.
\end{prb}

\begin{prb}[Canonical uniformity and bornology]
\end{prb}

\begin{prb}[Continuity and boundedness of linear operators]
\end{prb}
\begin{prb}[Metrizable topological vector spaces]
Birkhoff-Kakutani
\end{prb}


\begin{prb}[Continuous linear functionals]
A linear functional $l:X\to\F$ is continuous if and only if $\ker l$ is closed, if and only if $|l|$ is continuous.
\end{prb}



\section{Seminorms and convex sets}

\begin{prb}[Locally convex spaces]
A \emph{disk} is a convex balanced subset of a vector space.
A topological vector space $X$ is called a \emph{locally convex space} if there is a neighborhood system of disks at zero.

\end{prb}

\begin{prb}[Seminorms]
Let $X$ be a vector space.
A \emph{semi-norm} on $X$ is a functional $p:X\to\R_{\ge0}$ such that
\[p(x+y)\le p(x)+p(y),\qquad p(\lambda x)=|\lambda|p(x),\qquad x,y\in X,\ \lambda\in\C.\]


\end{prb}

\begin{prb}[Absorbing convex balanced sets]
Let $X$ be a vector space.
We say a subset $D\subset X$ is \emph{absorbing} if for every $x\in X$ there is a sufficiently large $r>0$ such that $x\in rD$.
For an absorbing disk $D$, a semi-norm on $X$ defined by $p(x):=\inf\{r\ge0:x\in rD\}$ for $x\in X$ is the unique semi-norm satisfying $p^{-1}([0,1))\subset D\subset p^{-1}([0,1])$.
The semi-norm $p$ is called the \emph{gauge} or \emph{Minkowski functional} of $D$.

In a given topological vector space, open convex sets correspond to continuous sublinear functionals, open convex balanced sets correspond to continuous semi-norms.

Equivalent conditions on the continuity of seminorms,
boundedness by seminorms,
normability
\begin{parts}
\item 
\end{parts}
\end{prb}
\begin{pf}


If $X$ is a locally convex space generated by a family of semi-norms $\{p_i\}$, then a subset $B\subset X$ is bounded if and only if $\sup_{x\in B}p_i(x)<\infty$ for each semi-norm $p_i$.
\end{pf}

Let $X$ be a locally convex space.
Let $U$ be an open absorbing convex balanced subset of $X$.
Let $p$ be the Minkowski functional of $U$.
If $x_i$ is a net in $X$ such that $x_i\to0$, then for every $\e>0$ there is $i_0$ such that $x_i\in\e U$ for $i\succ i_0$, and it implies $p(x_i)\le\e$ so we have $p(x_i)\to0$ by taking limit for $i$ and $\e\to0$.
That is, $p$ is continuous.




Let $D$ be a disk in a vector space $X$.
By separation and completion of a semi-normed space $(\spn D,p_D)$, where $p_D$ is a Minkowski functional of $D$, we obtain a Banach space $\hat X_D$, called the \emph{auxiliary Banach space} for $D$.
Let $X$ be a locally convex space, that is, a dual pair $(X,X^*)$ together with a choice of polar topology between $X_\sigma$ and $X_\tau$.
Then, $D$ is a neighborhood if and only if there is a natural continuous linear map $X\to\hat X_D$ of densen range, and $D$ is \emph{Banach} if and only if there is a natural continuous linear map $\hat X_D\to X$ that is injective.
A linear map $T:X\to Y$ between locally convex topologies is continuous if and only if for every continuous semi-norm $q$ on $Y$ there is a continuous semi-norm $p$ on $X$ such that $|p|\le|q|$.




\section{Hahn-Banach theorems}


\begin{prb}
Let $x_k^*\in X^*$ be a finite sequence.
If $x^*\in X^*$ vanishes on $\bigcap_k\ker x_k^*$, then $x^*$ is a linear combination of $x_k^*$.
\end{prb}



\begin{prb}[Hahn-Banach extension]
Let $X_0\subset X$ be vector spaces over the real or complex field $\F$.
A real functional $q:X\to\R$ is said to be \emph{sublinear} if $q(x+y)\le q(x)+q(y)$ and $q(tx)=tq(x)$ for all $x,y\in X$ and $t\in\R$.
\begin{parts}
\item For $q:X\to\R$ sublinear, any linear functional $l_0:X_0\to\R$ satisfying $l_0\le q$ on $X_0$ admits a linear extension $l:X\to\R$ satisfying $l\le q$.
\item For $p:X\to\R_{\ge0}$ a semi-norm, any linear functional $l_0:X_0\to\F$ satisfying $|l_0|\le p$ on $X_0$ admits a linear extension $l:X\to\F$ satisfying $|l|\le p$.
\item If $X$ is locally convex, then a continuous linear functional $l_0:X_0\to\F$ admits a continuous linear extension $l:X\to\F$.
If $X$ is normed, then a bounded linear functional $l_0:X_0\to\F$ admits a norm-preserving linear extension $l:X\to\F$.
\end{parts}
\end{prb}
\begin{pf}
(a)
Consider a partially ordered set of all linear extensions of $l_0$ dominated by $q$.
Precisely, we consider the set
\[\{l:V\to\R\mid\text{$V$ is a linear space between $X_0\subset X$},\ l_0=l|_{X_0},\ l\le q\},\]
on which the partial order is given by the restriction.
The non-emptyness and the chain condition is easily satisfied, so the partially ordered set has a maximal element $l:V\to\R$ by the Zorn lemma.

Suppose $V\ne X$ and choose $e\in X\setminus V$.
We want to assign an appropriate value to the vector $e$ to extend our maximal extension $l$.
The inequality
\[l(v)+l(v')=l(v+v')\le q(v+v')\le q(v-e)+q(v'+e),\qquad v,v'\in V\]
implies the existence of $r\in\R$ such that
\[\sup_{v\in V}(l(v)-q(v-e))\le r\le\inf_{v\in V}(-l(v)+q(v+e)).\]
If we define $\tilde V:=V+\R e$ and $\tilde l:\tilde V\to\R$ such that
\[\tilde l(v+te):=l(v)+tr,\qquad v\in V,\ t\in\R,\]
then $\tilde l$ extends $l$ and is dominated by $q$ as
\[l(v)+tr\le l(v)+t\cdot
\begin{cases}
-l\left(\dfrac vt\right)+q\left(\dfrac vt+e\right)&\text{ if }t>0,\\
0&\text{ if }t=0,\\
l\left(-\dfrac vt\right)-q\left(-\dfrac vt-e\right)&\text{ if }t<0,
\end{cases}
=q(v+te),\qquad v\in V,\ t\in\R,
\]
which leads a contradiction to the maximality of $l$.
Therefore, we conclude $V=X$.

(b)

Let $\F=\C$.
Note that the real part map $\Re:\Hom_\C(X,\C)\to\Hom_\R(X,\R)$ is bijective.
Note also that $|l|\le p$ if and only if $\Re l\le p$ for any complex linear functional $l:V\to\C$ on a complex vector space $V$.
It is because $|l|\le p$ implies $\Re l\le|l|\le p$ and conversely $\Re l\le p$ implies $|l|\le p$ by the inequality
\[|l(v)|^2=l(v)\bar{l(v)}=l(\bar{l(v)}v)=\Re l(\bar{l(v)}v)\le p(\bar{l(v)}v)=|l(v)|p(v),\qquad v\in V.\]
Since $|l_0|\le p$, we have $\Re l_0\le p$.
Using the part (a), there is a linear functional $l:X\to\C$ such that $\Re l_0=\Re l$ on $X_0$ and $\Re l\le p$.
Then, we can deduce $l_0=l$ on $X_0$ and $|l|\le p$.

(c)
Since $l_0:X_0\to\F$ is continuous, there is a finite family $\{p_j\}$ of continuous semi-norms such that
\[|l_0(x)|\le\sum_jp_j(x),\qquad x\in X_0.\]
The sum of semi-norms is a semi-norm, so the part (b) implies that there is a linear extension $l:X\to\F$ such that
\[|l(x)|\le\sum_jp_j(x),\qquad x\in X.\]
This inequality shows the continuity of $l$.

\end{pf}



\begin{prb}[Hahn-Banach separation]

Let $X$ be a locally convex space.
Let $C$ be a closed convex subset and $K$ be a compact convex subset of $X$ that are disjoint.
Then, there is a continuous linear functional $x^*\in X^*$ such that
\[\sup_{x\in C}\Re\<x,x^*\><\inf_{x\in K}\Re\<x,x^*\>.\]
\end{prb}





\section*{Exercises}
\begin{prb}[Topology of compact convergence]
\end{prb}





\chapter{Barrelled spaces}

\section{Uniform boundedness principle}
\begin{prb}[Barreled spaces]
Let $X$ be a topological vector space.
A \emph{barrel} is a closed absorbing convex balanced subset of $X$.
A \emph{barrelled space} is a topological vector space in which every barrel is a neighborhood of zero.
\end{prb}


A locally convex space $X$ is barrelled iff $X=X_\tau=X_\beta$.

% If a closed convex cone contains a dense subset of absorbing at a point, then it is entire?

\begin{prb}[Uniform boundedness principle]
Let $X$ and $Y$ be topological vector spaces.
We say that a family $\{T_i\}$ of continuous linear operators from $X$ to $Y$ is \emph{pointwise bounded} if $\{T_ix\}\subset Y$ is bounded for each $x\in X$, and is \emph{equicontinuous} if 

The \emph{uniform boundedness principle} states that a pointwise bounded family is equicontinuous.
It is also frequently called the \emph{Banach-Steinhaus theorem}.
\begin{parts}
\item If $X$ is barrelled and $Y$ is locally convex, then the uniform boundedness principle holds.
\item If $X$ is complete and metrizable, then the uniform boundedness principle holds.
\end{parts}
\end{prb}
\begin{pf}
(a)
Let $V$ be a neighborhood of zero in $Y$.
We may assume $V$ is closed and balanced, and we may further assume that $V$ is convex since $Y$ is locally convex.
If we define $U:=\bigcap_iT_i^{-1}V$, then $U$ is clearly a closed convex balanced set, and it is also absorbing because for any $x\in X$ the boundedness of $T_ix$ implies that there is $r\ge0$ such that $T_ix\subset rV$ for all $i$, which is equivalent to $x\in rU$ by definition of $U$.
Thuerefore, $U$ is a barrel in a barrelled space $X$, so it is a neighborhood of zero.
From $T_iU\subset V$ for every $i$, we have $\{T_i\}$ is equicontinuous.
\end{pf}



\section{Baire category theorem}

\begin{prb}[Baire spaces]
A topological space is called a \emph{Baire space} if the countable intersection of open dense subsets is always dense.
Equivalently, if the union of a sequence of closed subsets is the whole set, then at least one of such closed subsets contains an open set.
\begin{parts}
\item If a topological vector space is Baire, then it is barrelled.
\item A Baire space is second category in itself.
\item A topological group that is second category in itself is Baire.
\end{parts}
\end{prb}



\begin{prb}
Let $B\subset X$ be a closed absorbing subset of a topological vector space $X$ that is Baire.
Then, $B$ has a non-empty open subset, and $B-B$ is a neighborhood of zero.
If $B$ is convex in addition, then $B$ is a neighborhood of zero.
\end{prb}


\begin{prb}[Baire category theorem]
The Baire category theorem proves many exmples of topological vector space are Baire, in particular barrelled.
\begin{parts}
\item A completely metrizable space is Baire.
\item A locally compact Hausdorff space is Baire.
\end{parts}
\end{prb}




\section{Open mapping theorem}

\begin{prb}[Open mapping theorem]
Let $X$ and $Y$ be topological vector spaces.
The \emph{open mapping theorem} states that continuous surjective linear operator $T:X\to Y$ is an open map.
\begin{parts}
\item Suppose $X$ is completely metrizable. If $Y$ is Baire, or if $X$ is locally convex and $Y$ is barreled, the open mapping theorem holds.
\end{parts}
\end{prb}

\begin{pf}
(a)
Let $U$ be a balanced open neighborhood of zero in $X$.
It is enough to prove $TU$ is a neighborhood of zero.
We first claim the closure $\bar{TU}$ is a neighborhood of zero.
If we take a smaller balanced open neighborhood $U'$ of zero in $X$ such that $U'+U'\subset U$, then because $U'$ is absorbing and $T$ is surjective, the set $\bar{TU'}$ is closed and absorbing in $Y$.
Since $Y$ is Baire, $\bar{TU'}$ contains an open subset, so $\bar{TU'}+\bar{TU'}$ is a neighborhood of zero in $Y$.
Thus, the claim follows from $\bar{TU'}+\bar{TU'}\subset\bar{TU}$.

(If $X$ is locally convex (i.e.~Fr\'echet) and $Y$ is barreled...)

Since $X$ is metrizable, we have a countable balanced open neighborhood system $\{U_n\}_{n=0}^\infty$ of zero in $X$ such that $\bar{U_0+U_0}\subset U$ and $U_{n+1}+U_{n+1}\subset U_n$ for all $n\ge0$.
Our goal is to prove $\bar{TU_0}\subset TU$.
To prove this, we fix arbitrary $y_0\in\bar{TU_0}$ and construct a sequence $y_n\in Y$ inductively such that
\[y_{n+1}:=y_n-Tx_n,\qquad n\ge0,\]
where $x_n\in X$ is taken such that
\[x_n\in U_n,\qquad Tx_n\in y_n+\bar{TU_{n+1}},\qquad n\ge0.\]

To verify that such $x_n$ exists, it suffices to show $y_n\in\bar{TU_n}$ for all $n\ge0$ because this gets $TU_n$ to intersect the neighborhood $y_n+\bar{TU_{n+1}}$ of $y_n$.
For $n=0$ it is clear.
By the induction hypothesis that there is $x_{n-1}\in X$ satisfying all the above conditions, we get $Tx_{n-1}\in y_{n-1}+\bar{TU_n}$ and $y_n=y_{n-1}-Tx_{n-1}\in\bar{TU_n}$, so the desired sequence $x_n$ is well-defined.

Since $X$ is complete, the partial sum $\sum_{k=0}^{n-1}x_k$ is convergent to $x\in X$ because it is Cauchy by
\[\sum_{k=n'}^{n-1}x_k\in U_{n'}+\cdots+U_{n-1}\subset U_{n'-1},\qquad n>n'\ge0,\]
and in fact we have $x\in\bar{U_0+U_0}\subset U$ by
\[\sum_{k=0}^{n-1}x_k\in U_0+\cdots+U_{n-1}\subset U_0+U_0,\qquad n\ge1.\]
Because $y_n\to0$ as $n\to\infty$, we can check $y_0=Tx\in TU$ by
\[y_0=y_n+\sum_{k=0}^{n-1}(y_k-y_{k+1})=y_n+\sum_{k=0}^{n-1}Tx_k=y_n+T\sum_{k=0}^{n-1}x_k\to Tx,\qquad n\to\infty.\]
This completes the proof.
\end{pf}

A first countable topological vector space is metrizable.
A locally complete metrizable topological vector space is complete metrizable.



\section{Vector bornologies}

\begin{prb}[Bornological spaces]
A \emph{bornology} on a set $X$ is an ideal $\cB$ of $\cP(X)$ which covers $X$.
A bornology on a vector space is called a \emph{vector bornology} if it is stable under translation, scaling, and the balanced hulls.
For a topological vector space, there is a canonical bornology, sometimes called the \emph{von Neumann bornology}, defined such that $B\subset X$ is bounded if and only if every open neighborhood $U$ of zero in $X$ has $r\ge0$ such that $B\subset rU$.

Let $X$ be a locally convex space.
It is called \emph{bornological} if one of the following holds.
\begin{enumerate}[(i)]
\item every bornivorous convex balanced set is a neighborhood of zero,
\item every bounded linear operator from $X$ to any locally convex space is continuous,
\item it is an inductive limit of normable spaces
\end{enumerate}

If $X$ is metrizable, then $X$ is bornological.
If $X$ is bornological, then $X$ is Mackey.
If $X$ is bornological, then $X_\beta$ is complete.
If $X$ is bornological and boundedly complete, then $X$ is barrelled.

\end{prb}


bornological space is Mackey
barrelled space is Mackey: $X=X$



\section*{Exercises}

\begin{prb}
Let $(T_n)$ be a sequence in $B(X,Y)$.
If $T_n$ coverges strongly then $\|T_n\|$ is bounded by the uniform boundedness principle.
\end{prb}

\begin{prb}
There is a closed absorbing set in $\ell^2(\Z_{\ge0})$ that is not a neighborhood of zero;
\[\bar B(0,1)\setminus\bigcup_{i=2}^\infty B(i^{-1}e_i,i^{-2})\]
is a counterexample.
\end{prb}




\begin{prb}
There is no metric $d$ on $C([0,1])$ such that $d(f_n,f)\to0$ if and only if $f_n\to f$ pointwise as $n\to\infty$ for every sequence $f_n$.
Note that this problem is slightly different to the non-metrizability of the topology of pointwise convergence.
\end{prb}

\begin{prb}
We show that there is no projection from $\ell^\infty$ onto $c_0$.
\end{prb}

\begin{prb}[Schur property]
$\ell^1$
\end{prb}

\begin{prb}
Let $\f:L^\infty([0,1])\to\ell^\infty(\N)$ be an isometric isomorphism.
Suppose $\f$ is realised as a sequence of bounded linear functionals on $L^\infty$.
\begin{parts}
\item
Show that $\f^*(\ell^1)\subset L^1$ where $\ell^1$ and $L^1$ are considered as closed linear subspaces of $(\ell^\infty)^*$ and $(L^\infty)^*$ respectively.
\item Show that $\f^*$ is indeed an isometric isomorphism, and deduce $\f$ cannot be realised as bounded linear functionals on $L^\infty$.
\end{parts}
\end{prb}


\begin{prb}[Daugavet property]
\begin{parts}
\item The real Banach space $C([0,1])$ satisfies the Daugavet property.
\end{parts}
\end{prb}
\begin{pf}
Let $T$ be a finite rank operator on $C([0,1])$, and $e_i$ be a basis of $\im T$.
Then, for some measures $\mu_i$,
\[Tf(t)=\sum_{i=1}^n\int_0^1f\,d\mu_ie_i(t).\]
Let $M:=\max\|e_i\|$.

Take $f_0$ such that $\|f_0\|=1$ and $\|Tf_0\|>\|T\|-\frac\e2$.
Reversing the sign of $f_0$ if necessary, take an open interval $\Delta$ such that $Tf_0(t)\ge\|T\|-\frac\e2$ and $|\mu_i|(\Delta)\le\frac\e{4nM}$ for all $i$.
Define $f_1$ such that $f_0=f_1$ on $\Delta^c$, $f_1(t_0)=1$ for some $t_0\in\Delta$, and $\|f_1\|=1$.
Then, $\|Tf_1-Tf_0\|\le\frac\e2$ shows $Tf_1\ge\|T\|-\e$ on $\Delta$.
Therefore,
\[\|1+T\|\ge\|f_1+Tf_1\|\ge f_1(t_0)+Tf_1(t_0)\le1+\|T\|-\e.\]
\end{pf}

\begin{prb}[Bartle-Graves theorem]
Let $E$ be a Banach space and $N$ a closed subspace.
For $\e>0$, there is a continuous homogeneous map $\rho:E/N\to E$ such that $\pi\rho(y)=y$ and $\|\rho(y)\|\le(1+\e)\|y\|$ for all $y\in E/N$.
\end{prb}
\begin{pf}
We want to construct a continuous map $\psi:S_{E/N}\to E$ with $\|\psi(y)\|\le1+\e$ for all $y\in S_{E/N}$.
If then, $\rho$ can be made from $\psi$.

For each $y_0\in S_{E/N}$, choose $x_0\in\pi^{-1}(y_0)\cap B_{1+\e}$.
There is a neighborhood $V_{y_0}\subset S_{E/N}$ of $y_0$ such that $y\in V_{y_0}$ implies $x_0$ belongs to $(\pi^{-1}(y)\cap B_{1+\e})+U_{2^{-1}}$, which is convex.
With a locally finite subcover $V_{y_\alpha}$ and a partition of unity $\eta_\alpha(y)$, define $\psi_1(y)=\sum_\alpha\eta_\alpha(y)x_\alpha$.
Then, $\psi_1(y)\in(\pi^{-1}(y)\cap B_{1+\e})+U_{2^{-1}}$.

For $i\le2$, choose for each $y_0$ the element $x_0$ in $\pi^{-1}(y_0)\cap B_{1+\e}\cap(\psi_{i-1}(y_0)+U_{2^{-{i-1}}})$.
Then, we obtain
\[\psi_i(y)\in\Bigl(\pi^{-1}(y)\cap B_{1+\e}\cap(\psi_{i-1}(y_0)+U_{2^{-{i-1}}})\Bigr)+U_{2^{-i}}.\]
Therefore, $\|\psi_i(y)-\psi_{i-1}(y)\|<2^{-{i-2}}$, so it converges uniformly to $\psi$ such that $\psi(y)\in\pi^{-1}(y)\cap B_{1+\e}$.
\end{pf}

\section*{Problems}


Let $Y$ be a separable infinite-dimensional Banach space.
Let $\{e_i\}_{i\in I}$ be a Hamel basis of $Y$.
Since $I$ is uncountable, embed $\N\subset I$.
Let $X:=c_c(I)$ with $\ell^1$ norm.
Define $T:X\to Y$ such that $Te_i:=i^{-1}e_i$ for $i\in\N$ and $Te_i:=e_i$ for $i\notin\N$.
Then, $T$ is surjective, but we can see $T(e_i)=i^{-1}e_i$ for $i\in\N$ has a limit point zero.

\begin{prb}
Let $T$ be an invertible linear operator on a normed space.
Then, $T^{-2}+\|T\|^{-2}$ is injective if it is surjective.
\end{prb}
















\chapter{Weak topologies}
\section{Dual pairs}

\begin{prb}[Dual pairs]
Let $\F$ be the real or complex field.
A \emph{dual pair} is a pair $(X,X^*)$ of vector spaces over $\F$ together with a non-degenerate bilinear form $X\times X^*\to\F$.
A pair $(X,X^*)$ of a vector space $X$ and a subspace $X^*$ of $X^\#$ is a natural dual pair if and only if $X^*$ separates points of $X$.

Let $X$ be a topological vector space.
We can canonically define $X^*$ as the topological dual of $X$, and we consider $(X,X^*)$ as a canonical dual pair associated to $X$.
If $F$ is a linear subspace of $X^*$, then $(X,F)$ is another dual pair exactly when $F$ is weakly$^*$ dense in $X^*$ by the Hahn-Banach separation.

Note that if $X$ is discrete, then $X^*=X^\#$.
If $(X,X^*)$ is a dual pair, then $(X^*,X)$ is also a dual pair.
\end{prb}



\begin{pf}
For a linear subspace $V$ of a topological vector space $X$, $\bar V=V^{\perp\perp}$.
If $x\in\bar V$, then for $x^*\in V^\perp$, we have $\<x,x^*\>=0$ by approximation, so $x\in V^{\perp\perp}$.
Conversely, if $x\notin\bar V$, then the Hahn-Banach extension implies that there is $x^*$ such that $\<y,x^*\>=0<\<x,x^*\>$ for all $y\in V$, which means $x^*\in V^\perp$ and $x\notin V^{\perp\perp}$.

\end{pf}

\begin{prb}[Weak topology of dual pairs]
Let $(X,X^*)$ be a dual pair.

\begin{parts}
\item $X_\sigma$ and $X^*_\sigma$ are locally convex.
\item $(X_\sigma)^*=X^*$.
\item $(X^*_\sigma)^*=X$. Every locally convex space is a dual of a locally convex space.
\item A subset $B\subset X_\sigma$ is weakly bounded if and only if it is weakly totally bounded.
\item If $X^*=X^\#$, then $X_\sigma^*$ is complete.
\end{parts}
\end{prb}
\begin{pf}
(a)
The Hahn-Banach theorem implies the Hausdorffness.

(c)
We will only show $(X^*_\sigma)^*\subset X$.
If $x^{**}\in(X^*_\sigma)^*$, then there is a finite subset $\{x_i\}_{i\in J}$ of $X$ such that
\[|\<x^{**},x^*\>|\le\sum_{i\in J}|\<x_i,x^*\>|,\qquad x^*\in X^*.\]
Since $\bigcap_{i\in J}\ker x_i$ is a closed subspace of $\ker x^{**}$, we have $x^{**}\in\spn\{x_i\}_{i\in J}\subset X$.
\end{pf}

\begin{prb}
closure and weak closure of convex subsets
\end{prb}
\begin{pf}
Hahn-Banach
\end{pf}

\begin{prb}[Polar topologies]
Let $(X,X^*)$ be a dual pair.
The \emph{Mackey topology} on $X$ is the topology $\tau$ of uniform convergence on compact convex balanced subsets of $X^*_\sigma$.
The \emph{strong topology} on $X$ is the topology $\beta$ of uniform convergence on bounded convex balanced subsets of $X^*_\sigma$.
The space $X$ together with the Mackey and strong topology is denoted by $X_\tau$ and $X_\beta$ respectively.
If $X$ is a topological vector space and $(X,X^*)$ is the canonical dual pair, then we always have continuous identity operators
\[X_\beta\to X_\tau\to X\to X_\sigma.\]

Let $\alpha$ is a polar topology on $X$ generated by $\cG^*\subset\cP(X^*)$.
If $x^*\in(X_\alpha)^*$, then there is $\sigma(X,X^*)$-closed convex balanced $C^*\in\cG$ such that $\sup_{x\in(C^*)^\circ}|\<x,x^*\>|\le1$.


\begin{parts}
\item If $X$ is locally convex, then $X$ is barrelled if and only if $X=X_\beta$.
\item If $X$ is locally covex and metrizable, then $X=X_\tau$.
\item Mackey-Arens theorem
\end{parts}
\end{prb}


boundedness, incompleteness



\begin{prb}[Weak convergence of bounded nets]
Let $X$ be a Banach space, $X_0^*$ a subset of $X^*$, and $\bar{D^*}$ the norm closure of $X_0^*$.
For example, if $X$ has a predual $X_*\subset X^*$ and $X_0^*$ is dense in $X_*$, then $\sigma(X,\bar{X_0^*})$ is the weak$^*$ topology.
\begin{parts}
\item There is a squence $x_n\in X$ converges to zero in $\sigma(X,X_0^*)$ but not in $\sigma(X,\bar{X_0^*})$.
\item A bounded sequence $x_n\in X$ converges to zero in $\sigma(X,\bar{X_0^*})$ if in $\sigma(X,X_0^*)$.
\end{parts}
\end{prb}
\begin{pf}
(b)
Let $x^*\in\bar{X_0^*}$ and choose $x_0^*\in X_0^*$ such that $\|x^*-x_0^*\|<\e$.
Then,
\[|\<x_n,x^*\>|\le\|x_n\|\|x^*-x_0^*\|+|\<x_n,x_0^*\>|\lesssim\e+|\<x_n,x_0^*\>|\to\e.\]
\end{pf}



\begin{prb}[Alaoglu theorem]
Let $(X,X^*)$ be a dual pair.
If $U$ is a neighborhood of zero in $X_\tau$, then the polar $U^\circ$ is compact in $X^*_\sigma$.
\end{prb}
\begin{pf}
Note that $X^*_\sigma$ is topologically embedded into $X^\#_\sigma$, which has the Heine-Borel property by the Tychonoff theorem.
Since $U$ is absorbing, $U^\circ$ is bounded in $X^\#_\sigma$.
We can also see that $U^\circ$ is closed in $X^\#_\sigma$ because if $x_i^*\in U^\circ$ is a net such that $x_i^*\to x^*$ in $X^\#_\sigma$, then taking the limit for $i$ and the supremum on $x\in U$ in the inequality
\[|\<x,x^*\>|\le|\<x,x^*-x^*_i\>|+|\<x,x_i^*\>|\le|\<x,x^*-x^*_i\>|+1,\qquad x\in U,\]
we can deduce $x^*\in X^*$ from that $(x^*)^{-1}(\D)$ contains a neighborhood $U$ of zero in $X_\tau$, and automatically $x^*\in U^\circ$ by the inequality.
Therefore, the compactness of $U^\circ$ in $X_\sigma^\#$ and hence in $X_\sigma^*$ follows.
\end{pf}




\section{Compact convex sets}
Krein-Milman theorem
Choquet theory




\section*{Exercises}
\begin{prb}[James' space]
not reflexive but isometrically isomorphic to bidual
\end{prb}

\begin{prb}[Preduals]
Let $X$ be a Banach space.
A \emph{predual} of $X$ is a Banach space $F$ together with an isometric isomorphism $\f:X\to F^*$.
Two preduals $\f_1:X\to F_1^*$ and $\f_2:X\to F_2^*$ are said to be equivalent if there is an isometric isomorphism $\theta:F_1\to F_2$ such that $\theta^*=\f_1\f_2^{-1}$.
\begin{parts}
\item There is a one-to-one correspondence between the equivalence class of preduals of $X$ and the set of closed subspaces $X_*$ of $X^*$ such that $B_X$ is compact and Hausdorff in $(X,\sigma(X,X_*))$.
Such a subspace $X_*$ is also called a predual of $X$.
\item If $X$ admits a predual $X_*\subset X^*$, then a $\sigma(X,X_*)$-closed subspace $V$ of $X$ also admits a predual $X_*|_V$.
\end{parts}
\end{prb}
\begin{pf}
(a) Goldstine theorem for surjectivity.

(b)
It is easy if we apply the part (a).
We can show more directly.
If we let $V_*:=X_*|_V$ the image of $X_*$ under the map $X^*\to V^*$, then we have isometric injections $V\to(V_*)^*\to X$.
We can show $V$ is $\sigma(X,X_*)$ dense in $(V_*)^*$, hence the closedness proves the bijectivity of $V\to(V_*)^*$.
\end{pf}

\begin{prb}[Mazur's lemma]

\end{prb}



















\part{Banach spaces}




\chapter{Operators on Banach spaces}

\section{Bounded operators}





\begin{prb}[Bounded belowness in Banach spaces]
Let $T\in L(X,Y)$ for Banach spaces $X$ and $Y$.
The following statements are equivalent:
\begin{parts}
\item $T$ is bounded below.
\item $T$ is injective and has closed range.
\item $T$ is a topological isomorphism onto its image.
\end{parts}
\end{prb}

\begin{prb}[Bounded belowness in Hilbert spaces]
Let $T\in B(H,K)$ for Hilbert spaces $H$ and $K$.
The following statements are equivalent:
\begin{parts}
\item $T$ is bounded below.
\item $T$ is left invertible.
\item $T^*$ is right invertible.
\item $T^*T$ is invertible.
\end{parts}
\end{prb}

\begin{prb}[Injectivity and surjectivity of adjoint]
Let $T:X\to Y$ be a continuous linear operator between locally convex spaces.
\begin{parts}
\item $T^*$ is injective if and only if $T$ has dense range.
\item $T^*$ is surjective if and only if $T$ is an embedding.
\end{parts}
\end{prb}








\section{Compact operators}

$K(X,Y)$ is closed in $B(X,Y)$.
$K(X)$ is an ideal of $B(X)$.
adjoint is $K(X,Y)\to K(Y^*,X^*)$.
integral operators are compact.
riesz operator, quasi-nilpotent operator.




\section{Fredholm operators}

\begin{prb}
Let $E_1$ and $E_2$ be Fr\'echet spaces.
A \emph{Fredholm operator} is a bounded linear operator $F\in B(E_1,E_2)$ which has closed range and the kernels of $F$ and $F^*$ are finite-dimensional.
Let $F\in B(E)$.
\begin{parts}
\item $F$ is Fredholm if and only if $F^*$ is Fredholm.
\item $F$ is Fredholm of index zero if $1-F$ is compact.
\item $F$ is Fredholm if and only if $\pi(F)$ is invertible in $Q(E)$.
\end{parts}
\end{prb}
\begin{pf}
(b)
Since $1-F$ and $1-F^*$ are the compact identities on the closed subspaces $\ker F$ and $\ker F^*$ respectively, what remains is to show $F$ has closed range.
Consider the continuous dense embedding $V:E/\ker F\to\bar\ran\,F$ induced from $F$.
Suppose that $V$ is not bounded below so that there is a sequence of vectors $x_n\in E$ such that $x_n$ does not converge to zero in $E/\ker F$ but $Fx_n\to0$ in $E$.
We may assume $(1-F)x_n\to x$ for some $x\in E$ by compactness of $K$, which implies $x_n\to x$ in the closed subspace $\ker T^\perp$.
Since $Tx=x-Kx=0$, we have $x\in\ker T^\perp\cap\ker T=\{0\}$, so we obtain a contradiction $x_n\to0$.
Thus, $V$ is bounded below, so the range of $F$ is closed.
\end{pf}

\begin{prb}[Atkinson theorem]
Let $E$ and $F$ be Banach spaces.
\begin{parts}
\item An operator $T\in B(E,F)$ is Fredholm if and only if there is $S\in B(F,E)$ such that $1-ST$ and $1-TS$ is finite-rank.
\item An operator $T\in B(E)$ is Fredholm if and only if $\pi(T)$ is invertible in $Q(E)$.
\end{parts}
\end{prb}
\begin{pf}
(c)
Let $F$ be a Fredholm operator.
Note that the induced operator $V:E/\ker F\to\bar\ran\,F$ is a topological isomorphism.
Since $\ker F$ and $\ran F$ are complemented, we can define $F':=V^{-1}\oplus0$.
Then, $1-F'F$ and $1-FF'$ are of finite-rank.

Conversely if $\pi(F)$ has an inverse $\pi(F')$ in $Q(E)$ for some $F'\in B(E)$, then compactness of $1-F'F$ and $1-FF'$ implies that $F'F$ and $FF'$ are Fredholm.
Then, $\ker F\subset\ker(F'F)$ and $\ker F^*\subset\ker((FF')^*)$ are finite-dimensional, $F$ is Fredholm.
\end{pf}

\begin{prb}[Fredholm index]
locally constant, in particular, continuous.
composition makes the addition of indices.
\end{prb}


\section{}




\section*{Exercises}

\begin{prb}[Completely continuous operators]
On reflexive spaces, completely continuous operators are same with compact operators.
\end{prb}


\begin{prb}[Dunford-Pettis property]
A Banach space $X$ is said to have the \emph{Dunford-Pettis property} if all weakly compact operators $T:X\to Y$ to any Banach space $Y$ is completely continuous.
\begin{parts}
\item $X$ has the Dunford-Pettis property if and only if for every sequences $x_n\in X$ and $x^*_n\in X^*$ that converge to $x$ and $x^*$ weakly we have $x^*_n(x_n)\to x^*(x)$.
\item $C(\Omega)$ for a compact Hausdorff space $\Omega$ has the Dunford-Pettis property.
\item $L^1(\Omega)$ for a probability space $\Omega$ has the Dunford-Pettis property.
\item Infinite dimensional reflexive Banach space does not have the Dunfor-Pettis property.
\end{parts}
\end{prb}


\begin{prb}\,
\begin{parts}
\item (Mazur-Ulam, 1932) A surjective isometry $T:X\to Y$ between normed spaces is affine.
\item (Mankiewicz, 1972) Let $U,V$ be open sets in $X,Y$, normed spaces. A surjective isometry $U\to V$ is uniquely extended to a surjective isometry $X\to Y$.
\item (Mori) A surjective local isometry $T:X\to Y$ between Banach spaces is an isometry, if $X$ is separable. (Use the Baire category)
\end{parts}
\end{prb}
\begin{sol}
(a)
$T$ is continuous.
It is easy to see for continuous map $T$ that it is affine if and only if $T$ preserves the midpoint.
For $x_1\ne x_2\in X$ let $x_0$ be the midpoint.
Define inductively
\[C_1:=\{x\in X:\|x-x_1\|=\|x-x_2\|=\frac12\|x_1-x_2\|\},\qquad C_k:=\{x\in C_{k-1}:\sup_{x'\in C_{k-1}}\|x-x'\|\le\frac12\diam C_{k-1}\}.\]
Since $x_0\in C_{k-1}$ and $x'\in C_{k-1}$ imply $x_0\in C_k$ by $\|x_0-x'\|=\frac12\|(2x_0-x')-x'\|\le\frac12\diam C_{k-1}$, and since $\diam C_k\le\frac12\diam C_{k-1}$, we have $\{x_0\}=\bigcup_{k=1}^\infty C_k$.
It follows that the midpoint can be detected from the metric structure of $X$, not depending on the linear structure of $X$.
\end{sol}


\section*{Problems}
\begin{enumerate}
\item If $K\in B(L^2([0,1]))$ is a compact operator, then for any $\e>0$ there is a constant $C>0$ such that
\[\|Kf\|_{L^2}\le\e\|f\|_{L^2}+C\|f\|_{L^1}.\]
\end{enumerate}

\begin{pf}
Suppose there is $\e>0$ such that we have sequence $f_n\in L^2$ satisfying $\|f_n\|_{L^2}=1$ and $\|Kf_n\|_2>\e+n\|f_n\|_1$.
Since $K$ is compact, there is a subsequence $Kf_{n_j}$ convergent to $g\ne0$ in $L^2$.
Then, $\|f_{n_j}\|_{L^1}\to0$ and $\|f_{n_j}\|_{L^2}\le1$ imply $f_{n_j}\to0$ and hence $Kf_{n_j}\to0$ weakly in $L^2$, which implies a contradiction $g=0$.
\end{pf}




\chapter{Tensor products of Banach spaces}

\section{Injective and projective tensor products}

\begin{prb}[Realizations]
For Banach spaces $X$ and $Y$, $L(X,Y)$ and $Bi(X,Y)=L(X,Y^*)=L(Y,X^*)$ are naturally Banach spaces.
Also we have a natural algebraic inclusions of $X\otimes Y$ into $L(X^*,Y)\le Bi(X^*,Y^*)$, and $Bi(X,Y)^*$.
Also we have a natural algebraic inclusions of $X^*\otimes Y$ into $\cL(X,Y)\le\cB(X,Y^*)$.
\end{prb}

\begin{prb}
Let $X$ and $Y$ be a Banach spaces, and $\alpha$ be a norm on $X\otimes Y$.
We say $\alpha$ is a \emph{cross norm} if
\[\alpha(x\otimes y)=\|x\|\|y\|,\qquad x\in X,\ y\in Y,\]
and a cross norm is \emph{reasonable} if the \emph{dual norm} $\alpha^*$ on $X^*\otimes Y^*\subset(X\otimes Y,\alpha)^*$ of $\alpha$ is also a cross norm.
\[\e(u):=\|u\|_{\cB(X^*,Y^*)},\qquad\pi(u):=\|u\|_{\cB(X,Y)^*}.\]
\end{prb}

\begin{prb}[Type C and type L spaces]
\end{prb}

\begin{prb}[Duals of tensor products]
\[\cK(X,Y)\hookrightarrow X^*\hat\otimes_\e Y\leftarrow X^*\hat\otimes_\pi Y\twoheadrightarrow\cN(X,Y).\]
	
\end{prb}







\section{Vector-valued integrals}


harmonic and complex analysis


\begin{prb}[Pettis measurability theorem]
Let $(\Omega,\mu)$ be a measure space and $X$ a Banach space.
Let $f:\Omega\to X$ be a function.
We say $f$ is \emph{strongly measurable} or \emph{Bochner measurable} if it is a pointwise limit of a sequence of simple functions.

If $\mu$ is complete, then all the pointwise convergence discussed here can be relaxed to the almost everywhere convergence.
\begin{parts}
\item If $f$ is strongly measurable, then $f$ is Borel measurable.
\item If $f$ is Borel measurable, then $f$ is weakly measurable.
\item If $f$ is weakly measurable and separably valued, then $f$ is strongly measurable.
\end{parts}
\end{prb}

\begin{prb}[Pettis integrals]
\[L^1\hat\otimes_\e X\hookrightarrow\cL(X^*,L^1)\stackrel{*}{\hookrightarrow}\cL(L^\infty,X^{**}).\]
\begin{itemize}
\item Pettis integrable: $L^1\hat\otimes_\e X$,
\item weakly integrable: $\cL(X^*,L^1)$,
\item Dunford integrable: $\cL(L^\infty,X^{**})$,
\item Pettis integral: $L^1\hat\otimes_\e X\cong *^{-1}\cL(L^\infty,X)\subset\cL(X^*,L^1)$. It defines $L^1\hat\otimes_\e X\hookrightarrow\cK(L^\infty,X_\sigma)$.
\end{itemize}

\begin{parts}
\item The close graph theorem and the existence of an a.e.~convergent subsequence of an $L^1$ convergent sequence proves a weakly integrable function defines an operator in $\cL(X^*,L^1)$.
\end{parts}
\end{prb}

\begin{prb}[Bochner integrals]
Let $(\Omega,\mu)$ be a measure space and $X$ a Banach space.
Let $f:\Omega\to X$ be a strongly measurable function.
The function $f$ is said to be \emph{Bochner integrable} if there is a net of simple functions $(s_\alpha)_{\alpha\in\cA}$ such that
\[\int_\Omega\|f(\omega)-s_\alpha(\omega)\|\,d\mu(\omega)\to0\]
for $\alpha\in\cA$.

For $T\in\cL(X,Y)$ and $\mu:L^1(\mu)\to\C$, the commutative diagram for $\alpha\in\{\e,\pi\}$
\[\begin{tikzcd}
L^1(\mu)\hat\otimes_\alpha X \dar[swap]{\id\otimes T}\rar{\mu\otimes\id} & X \dar{T}\\
L^1(\mu)\hat\otimes_\alpha Y \rar{\mu\otimes\id} & Y,
\end{tikzcd}\]
which is shown with approximation by simple tensors, justifies that $T$ commutes with the integral:
\[T\int f\,d\mu=\int Tf\,d\mu.\]
The space of Bochner integrable functions $L^1\hat\otimes_\pi X$, factoring through $L^1\hat\otimes_\e X$, is naturally mapped to the space of Pettis integrals $\cK(L^\infty,X_\sigma)$.
\begin{parts}
\item $f$ is Bochner integrable if and only if $\int\|f(\omega)\|\,d\mu(\omega)<\infty$.
\item If $f$ is Bochner integrable, then it is Pettis integrable and the integrals coincides.
\end{parts}
Bochner integrable => Pettis integrable => weakly(scalarly) integrable
\end{prb}


\begin{prb}[Vector measures]
If an element of the Dunford integral $\cL(L^\infty,X^{**})$, or the Pettis integral $\cK(L^\infty,X_\sigma)$, defines a $\sigma$-weakly continuous linear operator $L^\infty\to X$, then it is called a vector measure?
\end{prb}






\section{Approximation property}
dual is Banach.
Basis problem, Mazur' duck.



\begin{prb}[Approximation property]
A locally convex space $X$ has the \emph{approximation property} if $F(X)$ is dense in $L_\tau(X)$.
(Schafer-Wolff do not assume the convexity of compact sets)
Recall that a locally convex space $X$ is called nuclear if $N(X)$ is dense in $L(X)$.


Every compact operator is a limit of finite-rank operators.
\begin{parts}
\item An Hilbert space has the AP.
\item If a locally convex space $X$ has the approximation property, then $X$ has a Schauder basis, and the converse holds if $X$ is separable and Fr\'echet.
\end{parts}
\end{prb}
\begin{pf}
(a)
Let $H$ be a Hilbert space and $K\in K(H)$.
Since $\bar{KB_H}$ is a compact metric space, it is separable, which means $\bar{KH}$ is separable.
Let $(e_i)_{i=1}^\infty$ be an orthonormal basis of $\bar{KH}$, and let $P_n$ be the orthogonal projection on the space spanned by $(e_i)_{i=1}^n$.
If we let $K_n:=P_nK$, then $K_n\to K$ strongly and $K_n$ has finite rank.
Take any $\e>0$ and find, using the totally boundedness of $KB_H$, a finite subset $\{x_j\}\subset B_H$ such that for any $x\in B_H$ there is $x_j$ satisfying $\|Kx-Kx_j\|<\frac\e2$.
Then,
\begin{align*}
\|Kx-K_nx\|
&\le\|Kx-Kx_j\|+\|Kx_j-K_nx_j\|+\|P_n(Kx_j-Kx)\|\\
&\le\frac\e2+\|Kx_j-K_nx_j\|+\frac\e2.
\end{align*}
By taking the supremum on $x\in B_H$, we have
\[\|K-K_n\|\le\max_j\|Kx_j-K_nx_j\|+\e,\]
which deduces $K_n\to K$ in norm.

\end{pf}




\section*{Exercises}
Tingley problem



\chapter{Geometry of Banach spaces}


\section{}


\begin{prb}[Eberlein-\v Smulian theorem]
\end{prb}
\begin{prb}[James theorem]
\end{prb}



\begin{prb}[Krein-\v Smulian theorem for weakly$^*$ closed sets]
Let $X$ be a Banach space, and let $F^*$ be a convex subset of $X^*$ whose bounded parts are weakly$^*$ closed.
\begin{parts}
\item If $F^*$ is disjoint to $B_{X^*}$, then there is $x\in X$ separating $F^*$ and $B_{X^*}$.
\item $F^*$ is weakly$^*$ closed.
\end{parts}
\end{prb}
\begin{pf}
(a)
Note that for any Banach space $X$, if $F$ is a subset of $B_X$, then we have a natural contractive linear operator $\ell^1(F)\to X$ and its dual $X^*\to\ell^\infty(F)$.
We will construct a subset $F$ of $B_X$ such that $F^*$ induces a subset of $c_0(F)$ and $F^*\cap F^\circ=\varnothing$, where $F^\circ$ denote the complex polar of $F$.
If it exists, the image of $F^*$ in $c_0(F)$ is a convex set disjoint to $B_{c_0(F)}$, so there exists a separating linear functional in $B_{\ell^1(F)}$ by the Hahn-Banach separation, and it induces a linear functional separating $F^*$ and $B_{X^*}$.

Let $F_0:=\{0\}\subset X$.
As an induction hypothesis on $n$, suppose we have $F_k$ for $0\le k\le n-1$ are finite subsets of $k^{-1}B_X$ such that
\[F^*\cap nB_{X^*}\cap\left(\bigcup_{k=0}^{n-1}F_k\right)^\circ=\varnothing.\]
If every finite subset $F_n$ of $n^{-1}B_X$ satisfies
\[F^*\cap(n+1)B_{X^*}\cap\left(\bigcup_{k=0}^{n-1}F_k\right)^\circ\cap F_n^\circ\ne\varnothing,\]
then since $F^*$ is weakly$^*$ compact on a bounded part by assumption, the finite intersection property leads a contradiction because the intersection of all complex polars $F_n^\circ$ of finite subsets $F_n$ of $n^{-1}B_X$ is $nB_{X^*}$, the polar of all union of finite subsets $F_n$ of $n^{-1}B_X$.
Thus, we have a finite subset $F_n$ of $n^{-1}B_X$ such that
\[F^*\cap(n+1)B_{X^*}\cap\left(\bigcup_{k=0}^nF_k\right)^\circ=\varnothing.\]
Then, $F:=\bigcup_{k=0}^\infty F_k$ has the property we want.

\end{pf}

\begin{prb}[Krein-\v Smulian theorem for weakly compact sets]
\end{prb}


\begin{prb}[Bishop-Phelps theorem]
\end{prb}



Let $T:X\to Y$ be a quotient.
For each $y^*\in Y^*$, if we take $y\in Y$ such that $\|y\|=1$ and $\|y^*\|<|\<y,y^*\>|+\e$ with $x\in X$ such that $Tx=y$, then since $1=\|Tx\|=\inf_{Tx'=0}\|x-x'\|$, we can find $x'$ such that $Tx'=0$ and $\|x-x'\|<1+\e$, so $\|y^*\|<|\<y,y^*\>|+\e=|\<x-x',T^*y^*\>|+\e<(1+\e)\|T^*y^*\|+\e$ implies that $T^*$ is an isometry.

Let $T:X\to Y$ be an isometry.
For each $y^*\in Y^*$, since the Hahn-Banach extension gives ${y^*}'\in Y^*$ such that $T^*{y^*}'=T^*y^*$ and $\|T^*y^*\|=\|{y^*}'\|$, we have $\inf_{T^*{y^*}'=0}\|y^*-{y^*}'\|=\inf_{T^*{y^*}'=T^*y^*}\|{y^*}'\|\le\|T^*y^*\|$, so $T^*$ is a quotient.



\part{Spectral theory}

\chapter{Operators on Hilbert spaces}

\section{}

\begin{prb}
quadratic form = symmetric bilinear form
hermitian form = conjugate-symmetric sesqui-linear form

polarization works for quadratic forms and sesquilinear forms.
Cauchy-Schwarz works for positive semi-definite quadratic forms and positive semi-definite hermitian forms

\end{prb}
\begin{pf}
Let $h$ be a positive semi-definite hermitian form on a complex vector space $H$.
For $\xi,\eta\in H$ and $\e>0$, we have
\begin{align*}
0&\le h\left(\xi-\frac{h(\xi,\eta)}{h(\eta,\eta)+\e}\eta,\ \xi-\frac{h(\xi,\eta)}{h(\eta,\eta)+\e}\eta\right)(h(\eta,\eta)+\e)\\
&=h(\xi,\xi)(h(\eta,\eta)+\e)-|h(\xi,\eta)|^2\frac{h(\eta,\eta)+2\e}{h(\eta,\eta)+\e}\\
&\le h(\xi,\xi)(h(\eta,\eta)+\e)-|h(\xi,\eta)|^2,\qquad\xi,\eta\in H.
\end{align*}
Limiting $\e\to0$, we obtain the Cauchy-Schwarz inequality.
\end{pf}

Projections. Reducing subspaces.
Hilbert space classification by cardinal.
Riesz representation theorem.
\begin{prb}
\begin{parts}
\item A Banach space $X$ is isometrically isomorphic to a Hilbert space if there is a bounded linear projection on every closed subspace of $X$.
\end{parts}
\end{prb}

\begin{prb}[Riesz representation theorem]
Let $H$ be a Hilbert space over a field $\K$, which is either $\R$ of $\C$.


We use the bilinear form $\<-,-\>:X\times X^*\to\K$ of canonical duality.
The Riesz representation theorem states that a continuous linear functional on a Hilbert space is represented by the inner product with a vector.
\begin{parts}
\item For each $x^*\in H^*$, there is a unique $x\in H$ such that $\<y,x^*\>=\<y,x\>$ for every $y\in H$.
\item $H\to H^*:x\mapsto\<-,x\>$ is a natural linear and anti-linear isomorphism if $\K=\R$ and $\C$, respectively.
\end{parts}
\end{prb}



Let $H$ be a separable Hilbert space.
Find a positive sequence $a_n$ such that every sequence $x_n$ of unit vectors of $H$ satisfying $|\<x_i,x_j\>|\le a_j$ for all $i<j$ converges weakly to zero.



\begin{prb}[Normal operators]
For $T\in B(H)$, we have an obvious fact $(\im T)^\perp=\ker T^*$.
Suppose $T$ is normal.
\begin{parts}
\item $\ker T=\ker T^*$.
\item $T$ is bounded below if and only if $T$ is invertible.
\item If $T$ is surjective, then $T$ is invertible.
\end{parts}
\end{prb}

\begin{prb}[Invariant and Reducing subsapces]
Let $K$ be a closed subspace of $H$.
\begin{parts}
\item $K$ is reducing for $T$ if and only if $K$ is invariant for $T$ and $T^*$.
\item $K$ is reducing for $T$ if and only if $TP=PT$, where $P$ is the orthogonal projection on $K$.
\end{parts}
\end{prb}
% self adjoint operators
% invariant but not reducing for unitary operators
% eigenspaces
% matrix representation


% direct sum and tensor product of hilbert spaces


\section{Compact operators}

spectral theorem of compact normal operators.


linearly spanned by rank-one projections...

$K(H)$ is the unique non-zero proper closed ideal of $B(H)$ if $H$ is separable.


analytic Skolem-Noether theorem: automorphism of $K(H)$ is inner in the identity representation $B(H)$.

states of $K(H)$ are density operators.


\section{}
\begin{prb}[Traces]
Bounded linear operators $t$ and $h$ on a Hilbert space $H$ are said to be a \emph{trace-class} and a \emph{Hilbert-Schmidt} operator respectively if
\[\sum_i\<|t|\delta_i,\delta_i\><\infty,\qquad\sum_i\<|h|^2\delta_i,\delta_i\><\infty,\]
where $(\delta_i)$ is an orthonormal basis of $H$.
\begin{parts}
\item
The trace does not depend on the choice of the orthonormal basis.
The trace is tracial.
Finite-rank operators are dense.
\item $L^2(H)$ is a Hilbert space.
\item $L^1(H)\to K(H)^*:t\mapsto\Tr(\cdot t)$ is an isometric isomorphism.
\item $B(H)\to L^1(H)^*:x\mapsto\Tr(x\cdot)$ is an isometric isomorphism.
\item $t\in B(H)$ is a trace class if and only if $t=\sum_i\lambda_i\theta_{\delta'_i,\delta_i}$ for some $(\lambda_i)\in\ell^1(\N)$ and orthonormal sequences $(\delta_i),(\delta'_i)\subset H$.
\end{parts}
\end{prb}
\begin{pf}

If $(\delta_i)$, $(\delta'_{i'})$, and $(\delta''_{i''})$ are any orthonormal bases of $H$, then by the Parseval theorem,
\[\sum_i\|h\delta_i\|^2
=\sum_{i,i'}|\<h\delta_i,\delta'_{i'}\>|^2
=\sum_{i,i'}|\<h^*\delta'_{i'},\delta_i\>|^2
=\sum_{i'}\|h^*\delta'_{i'}\|^2,\]
and similarly
\[\sum_{i'}\|h^*\delta'_{i'}\|^2=\sum_{i''}\|h\delta'_{i''}\|^2.\]
In particular, $\Tr(|h|^2)=\Tr(|h^*|^2)$.
For Hilbert-Schmidt operators $h_1$ and $h_2$ on $H$, the polarization deduces
\[\Tr(h_2^*h_1)
=\sum_{k=0}^3i^k\Tr(|h_1+i^kh_2|^2)
=\sum_{k=0}^3i^k\Tr(|h_1^*+\bar{i^k}h_2^*|^2)
=\sum_{k=0}^3i^k\Tr(|h_2^*+i^kh_1^*|^2)
=\Tr(h_1h_2^*).\]

(c)


(d)
For $t\in L^1(H)$, let $t=v|t|$ be the polar decopmosition.
Then, the boundedness follows from
\begin{align*}
|\Tr(xt)|^2
&=|\Tr(xv|t|^{\frac12}\cdot|t|^{\frac12})|^2\\
&\le|\Tr(xv|t|^{\frac12}\cdot|t|^{\frac12}v^*x^*)||\Tr(|t|^{\frac12}\cdot|t|^{\frac12})|\\
&=|\Tr(|t|^{\frac12}v^*x^*\cdot xv|t|^{\frac12})||\Tr(|t|)|\\
&\le\|x^*x\||\Tr(|t|^{\frac12}v^*v|t|^{\frac12})||\Tr(|t|)|\\
&=\|x\|^2|\Tr(|t|)|^2,\qquad x\in B(H).
\end{align*}
We can check the isometry by putting $x:=v^*$.
For the surjectivity, let $l\in L^1(H)^*$.
A sesqui-linear functional $\sigma$ on $H$ defined by
\[\sigma(\xi,\eta):=l(\theta_{\xi,\eta}),\qquad\xi,\eta\in H\]
is bounded by $\|l\|$, so there is $x\in B(H)$ such that
\[\sigma(\xi,\eta)=\<x\xi,\eta\>,\qquad\xi,\eta\in H.\]
To verify $l=\Tr(x\cdot)$ on $L^1(H)$, we may assume that $t\in L^1(H)$ has the form $t=\theta_{\xi,\eta}$ because the finite-rank operators are dense in $L^1(H)$, and we finally have for an orthonormal basis $(\delta_i)$ such that $\eta=\delta_i$ for some $i$ that
\[\Tr(xt)=\Tr(x\theta_{\xi,\eta})=\sum_i\<x\theta_{\xi,\eta}\delta_i,\delta_i\>=\sum_i\<\<\delta_i,\eta\>x\xi,\delta_i\>=\<x\xi,\eta\>=l(\theta_{\xi,\eta})=l(t).\]


(e)
Applying the polar decomposition and diagonalizing the compact operator $|t|$, we are done.
Conversely, if $t=\sum_i\lambda_i\theta_{\delta'_i,\delta_i}$, then we can check the diagonalization $t^*t=\sum_i|\lambda_i|^2\theta_{\delta_i}$, so we have $|t|=\sum_i|\lambda_i|\theta_{\delta_i}$, and
\[\Tr(|t|)=\sum_j\<|t|\delta_j,\delta_j\>=\sum_{i,j}\<|\lambda_i|\theta_{\delta_i}\delta_j,\delta_j\>=\sum_{i,j}\<|\lambda_i|\delta_{ij}\delta_i,\delta_j\>=\sum_{i,j}|\lambda_i|\delta_{ij}^2=\sum_i|\lambda_i|<\infty.\]


\end{pf}



\begin{prb}[Six locally convex operator topologies]
Let $H$ be a Hilbert space.
\[x\mapsto(\|x\xi\|^2+\|x^*\xi\|^2)^{\frac12},\qquad
x\mapsto\|x\xi\|,\qquad
x\mapsto\<x\xi,\xi\>\]
for $\xi\in H$.
\[x\mapsto\Bigl(\sum_{i=1}^\infty\|x\xi_i\|^2+\|x^*\xi_i\|^2\Bigr)^{\frac12},\qquad
x\mapsto\Bigl(\sum_{i=1}^\infty\|x\xi_i\|^2\Bigr)^{\frac12},\qquad
x\mapsto\Bigl|\sum_{i=1}^\infty\<x\xi_i,\xi_i\>\Bigr|\]
for $(\xi_i)\in\ell^2(\N,H)$.

\begin{parts}
\item
A net $T_i$ converges to $T$ strongly in $B(H)$ if and only if $\|(T_i-T)^{\oplus n}\bar\xi\|\to0$ for all $\bar\xi\in H^{\oplus n}$.
\item
A net $T_i$ converges to $T$ $\sigma$-strongly in $B(H)$ if and only if $\|(T_i-T)^{\oplus\infty}\bar\xi\|\to0$ for all $\bar\xi\in H^{\oplus\infty}$.
\end{parts}
\end{prb}


\begin{prb}[Continuity of linear functionals]
Let $l$ be a linear functional on $B(H)$ for a Hilbert space $H$.
\begin{parts}
\item
$l$ is weakly continuous if and only if it is strongly$^*$ continuous, and in this case we have
\[l=\sum_i\lambda_i\omega_{e_i,e'_i},\qquad(\lambda_i)\in c_c,\quad(e_i),(e'_i)\subset H\text{ orthonormal}.\]
or equivalently,
\[l=\sum_i\omega_{x_i,y_i},\qquad(x_i),(y_i)\in c_c(\N,H)\]
\item
$l$ is $\sigma$-weakly continuous if and only if it is $\sigma$-strongly$^*$ continuous, and in this case we have 
\[l=\sum_i\lambda_i\omega_{e_i,e'_i},\qquad(\lambda_i)\in\ell^1,\quad(e_i),(e'_i)\subset H\text{ orthonormal}.\]
or equivalently,
\[l=\sum_i\omega_{x_i,y_i},\qquad(x_i),(y_i)\in\ell^2(\N,H)\]
\item For a convex subset of $B(H)$ is ($\sigma$-)weakly closed if and only if ($\sigma$-)strongly$^*$ closed.
\end{parts}
\end{prb}
\begin{pf}
Suppose $l$ is strongly continuous.
There exists $\bar x\in H^{\oplus n}$ such that
\[|l(T)|\le\|T^{\oplus n}\bar x\|.\]
The functional $l:A\to\C$ factors through $H^{\oplus n}$ such that
\[A\xrightarrow{\bar x}H^{\oplus n}\to\C.\]
\end{pf}

\begin{prb}[]
\,

($\sigma$-) means that we have chosen the standard form.
The followings also hold in Hilbert modules.
\begin{parts}
\item On a bounded subset of $B(H)$, the weak, strong, strong$^*$ topologies coincide with the $\sigma$-weak, $\sigma$-strong, $\sigma$-strong$^*$ topologies, respectively.
\item $M$ is ($\sigma$-)strongly$^*$ complete, $M_1$ is ($\sigma$-)weakly and ($\sigma$-)strongly complete.
\item On $\rU(M)$, the ($\sigma$-)weak topology and ($\sigma$-)strong$^*$ topology coincide. However, it is not ($\sigma$-)weakly nor ($\sigma$-)stronlgy closed, but is ($\sigma$-)strongly$^*$ closed.
\end{parts}
\end{prb}




Suppose $T:X\to Y$ satisfies $\|y\|=\inf_{x\in T^{-1}(y)}\|x\|$.
If $R:Y\to Z$ is a function such that $RT:X\to Z$ is a contraction, then $R$ is a contraction, because for $y\in Y$, if we take $x\in X$ such that $Tx=y$ and $\|y\|+\e>\|x\|$, then
\[\|Ry\|=\|RTx\|\le\|x\|<\|y\|+\e\to\|y\|,\qquad\e\to0.\]
The condition is equivalent to the isometry of $T^*$?


\section{}


spectral radius formula and spectral mapping theorem are reuiqred.






\begin{prb}[Spectral radius formula]
\end{prb}

\begin{prb}[Spectral mapping theorem]

If $f(A)$ is not invertible, then in the factorization $f(z)=c\prod_k(\lambda_k-z)$, we have $\lambda=\lambda_k\in\sigma(A)$ for some $k$, so $f(\lambda)=0$.

If $f(N)$ is not invertible, then there is $x\in H$ such that $\<f(N)x,x\>=0$....?
Then, we can check $f(\<Nx,x\>)=0$ and $\<Nx,x\>\in\sigma(N)$..?






\end{prb}





\begin{prb}[Continuous functional calculus]
Let $N$ be a bounded normal linear operator on a Hilbert space $H$.
A \emph{continuous functional calculus} of $N$ is a unital $*$-homomorphism
\[\Phi:C(\sigma(N))\to B(H):f\mapsto f(N)\]
such that $\Phi(z)=N$, where $z\in C(\sigma(N))$ denotes the inclusion $\sigma(N)\to\C$.
\end{prb}
\begin{pf}

We first prove the existence.
On the polynomial ring $\C[z,\bar z]\subset C(\sigma(N))$ with a conjugate-linear involution $z\mapsto\bar z$, we have no issue for defining a unital $*$-homomorphism
\[\Phi_0:\C[z,\bar z]\to B(H):f\mapsto f(N)\]
such that $\Phi_0(z)=N$ because $N$ is normal.
First, $\C[z,\bar z]$ is a unital $*$-algebra separating points of $\sigma(N)$ so that it is uniformly dense in $C(\sigma(N))$ by the Stone-Weierstrass theorem.
Second, $B(H)$ is complete with respect to the norm topology.
Third, $\Phi_0$ is bounded by the spectral radius formula and the spectral mapping theorem for 
\[\|f(N)\|=r(f(N))=\sup_{\lambda\in\sigma(f(N))}|\lambda|=\sup_{\lambda\in\sigma(N)}|f(\lambda)|=\|f\|\]

\[\|f(N)\|^2=\||f|^2(N)\|=r(|f|^2(N))=\sup_{\lambda\in\sigma(|f|^2(N))}|\lambda|=\sup_{\lambda\in\sigma(N)}|f(\lambda)|^2=\||f|^2\|=\|f\|^2\]


Therefore, $\Phi_0$ is extended to a bounded linear map
\[\Phi:C(\sigma(N))\to B(H):f\mapsto f(N)\]
such that $\Phi(z)=N$.
Now it is enough to check $\Phi$ also preserves the multiplication and involution.



\end{pf}


\chapter{Unbounded operators}

\section{Densely defined closed operators}


We almost always consider the domain of an unbounded linear operator as the union of all subspaces on which a given operator is continuously well-defined.
Between complete spaces, the subspaces may be assumed to be closed.

Densely defined operators can be seen as increasing limits of partially defined continuous linear operators.


For $X$ without condition and $Y$ normable, then the continuity of $T:X_\sigma\to Y_\sigma$ implies the boundedness of $T:X\to Y$ because if we have $x_i\to0$ and $Tx_i$ is not bounded, then the uniform boundedness principle on $Y^*$ proves $Tx_i$ does not coverges weakly to zero.


We want to realize the graph $\Gamma(T)$ as the strict inductive limit of Fr\'echet spaces $\Gamma(T_i)$ with barrelled $X_i$.
The topology on $\Gamma(T_i)$ may not come from the topology of $X_\sigma\times Y_\sigma$.
If so, by the closed graph theorem, $T_i:X_i\to Y$ are everywhere defined continuous linear operators.


Even if the weak topology on $X\times Y$ is not complete but its weakly closed subspace $\Gamma(T)$ can be seen as a Banach space.
Which topology is natural on the graph?
For closedness, weak topology is the most natural.


I think the most natural setting for densely defined closed operators is the Fr\'echet space.

\begin{prb}
Let $X$ and $Y$ be topological vector spaces.
A \emph{linear operator} from $X$ to $Y$ is a linear map $T:\dom T\to Y$, where $\dom T$ is a linear subspace of $X$.

\end{prb}

\begin{prb}
Let $X$ and $Y$ be Fr\'echet spaces.
For a closed operator $T:\dom T\subset X\to Y$, there is an increasing net $T_i:\dom T_i\subset X\to Y$ of closed operators such that $\dom T_i$ is closed and $\Gamma(T)=\bigcup_i\Gamma(T_i)$. (Consider the net of finite-dimensional subspaces)
Conversely, 

\begin{parts}
\item a
\end{parts}
\end{prb}

\begin{prb}[Adjoint operators]
Let $(X,X^*)$ and $(Y,Y^*)$ be dual pairs.
Let $T:\dom T\subset X_\sigma\to Y$ be a densely defined linear operator.
The \emph{adjoint} of $T$ is defined by a linear operator $T^*:\dom T^*\subset Y^*_\sigma\to X^*$ with domain
\[\dom T^*:=\{y^*\in Y^*\mid \dom T\subset X_\sigma\to\C:x\mapsto\<Tx,y^*\>\text{ is continuous}\}\]
such that
\[\<x,T^*y^*\>:=\<Tx,y^*\>,\qquad x\in\dom T,\ y^*\in\dom T^*.\]
Consider the dual pair $(X\times Y,Y^*\times X^*)$ with the pairing given by $\<(x,y),(y^*,x^*)\>:=\<x,x^*\>+\<y^*,y\>$ for $(x,y)\in X\times Y$ and $(y^*,x^*)\in Y^*\times X^*$.




\begin{parts}
\item If $T\subset S$, then $S^*\subset T^*$.
\item $T^*:\dom T^*\subset Y^*_\sigma\to X^*_\sigma$ is always closed.
\item $T$ is closable if and only if $T^*$ is densely defined. If it is, then $T^{**}$ is the closure of $T$.
\end{parts}
\end{prb}
\begin{pf}
Before proofs, we first claim that the defining condition of the adjoint $T^*$ is equivalent to the equality $\gra(-T^*)=(\gra T)^\perp$ in $Y^*\times X^*$ with respect to the pairing.
One direction is clear by
\[\<(x,Tx),(y^*,-T^*y^*)\>=\<x,-T^*y^*\>+\<Tx,y^*\>=0,\qquad x\in\dom T,\ y^*\in\dom T^*.\]
Conversely if $(y^*,x^*)\in(\gra T)^\perp$, then since
\[0=\<(x,Tx),(y^*,x^*)\>=\<x,x^*\>+\<Tx,y^*\>=\<x,x^*+T^*y^*\>,\qquad x\in\dom T,\]
we have $y^*\in\dom T^*$ from the continuity of $\dom T\subset X_\sigma\to\C:x\mapsto\<Tx,y^*\>=-\<x,x^*\>$, and $x^*=-T^*y^*$ by the definition of adjoint operator $T^*$ and the desity of $\dom T$ in $X_\sigma$.
Hence the claim $(y^*,x^*)=(y^*,-T^*y^*)\in\gra(-T^*)$ follows.

(a) Clear from the claim.

(b) It is because the complement $(\gra T)^\perp$ is closed in $(Y^*\times X^*)_\sigma=Y^*_\sigma\times X^*_\sigma$.

(c)
Suppose $T$ is closable.
If $y\in Y$ satisfies $\<y,y^*\>=0$ for every $y^*\in\dom T^*$, then the equation $\<(0,y),(y^*,-T^*y)\>=0$ implies $(0,y)\in(\gra(-T^*))^\perp=(\gra T)^{\perp\perp}=\bar{\gra T}$, and the closability of $T$ implies $y=T0=0$.
It means that $\dom T^*$ separates point of $Y$, that is, $\dom T^*$ is dense in $Y^*_\sigma$.

Conversely, if $T^*$ is densely defined, then we can define the double adjoint $T^{**}:\dom T^{**}\subset X\to Y$, which has the graph $\gra T^{**}=(\gra(-T^*))^\perp=(\gra T)^{\perp\perp}=\bar{\gra T}$, so $T$ has the closure $T^{**}$.

\end{pf}







\begin{prb}[Everywhere defined continuous operators]
Let $(X,X^*)$ and $(Y,Y^*)$ be dual pairs.
We will always consider the weak topologies if not mentioned.
Let $T:X\to Y$ be an everywhere defined continuous linear operator.
It is clear that $T$ is a homeomorphism if and only if it is either an injective quotient map, a surjective embedding, or a closed embedding of dense range.
For a dense injection $T$, $T$ is a homeomorphism if and only if it is either a quotient map, surjective embedding, or a closed embedding.
\begin{parts}
\item $T$ is injective iff $T^*$ has dense range.
\item $T$ is an embedding iff $T^*$ is surjective.
\item $T$ is a closed embedding iff $T^*$ is a quotient map.
\end{parts}
\end{prb}
\begin{pf}
(a)
If the range of $T^*$ is not dense in $X^*$, then there is a non-zero $x\in X$ vanishing on the range of $T^*$ by the Hahn-Banach extension, and $Tx$ vanishes by every $y^*$ so that the kernel contains $x$ and is non-zero.
Conversely, if $T^*$ has dense range, then $T$ is clearly injective by the continuity of $T$.

(b)
If $T$ is an embedding, then every element of $X^*$ induces a partially defined continuous linear functional on the range of $T$, and the Hahn-Banach extension gives an everywhere defined continuous linear functional on $Y$, which is indeed an element of $Y^*$, so $T^*$ is surjective.
Conversely if $T^*$ is surjective, then for a net $x_i$ in $X$ such that $Tx_i\to0$ in $Y$, taking $y^*\in Y^*$ for each $x^*\in X^*$ such that $T^*y^*=x^*$, we can check easily that $x_i\to0$ in $X$, so $T$ is an embedding.

(c)
Suppose $T$ is a closed embedding.
Let $V^*$ be a neighborhood of zero in $Y^*$, and find a finite sequences $x_i\in X$ and $y_j\in Y\setminus TX$ such that $\{Tx_i,y_j\}^\circ\subset V^*$.
For $x^*\in X^*$ satisfying $\max_i|\<x_i,x^*\>|\le1$, since $TX$ is closed in $Y$, we can find $y^*\in Y^*$ by the Hahn-Banach extension theorem such that $x^*=T^*y^*$ and $\max_j|\<y_j,y^*\>|\le1$.
It implies that $\{x_i\}^\circ\subset T^*\{Tx_i,y_j\}^\circ\subset T^*V^*$, so $T^*V^*$ is a neighborhood of zero in $X^*$, which means $T^*$ is open, and hence $T^*$ is a quotient map.

Conversely, assume $T^*$ is a quotient map, which is automatically open by the fact that a quotient map onto the coset space of a topological group is open.
Let $x_i$ be a net in $X$ such that $Tx_i\to y$ in $Y$.
Since $T^*$ is open and the absolute polar $\{y\}^\circ$ is a neighborhood of zero in $Y^*$, so is the image $T^*\{y\}^\circ$ in $X^*$.
Then, $x^*\in X^*$ is contained in this image if and only if there is $y^*\in Y^*$ such that $x^*=T^*y^*$ and $|\<y,y^*\>|\le1$, which is equivalent to $\lim_i|\<x_i,x^*\>|\le1$ since $T$ is an embedding and $T^*$ is surjective.
If we consider a linear functional $x^{**}:x^*\mapsto\lim_i\<x_i,x^*\>$ on $X^*$, then it is continuous because $\{x^{**}\}^\circ=T^*\{y\}^\circ$ is a neighborhood of zero in $X^*$, so $x^{**}$ defines $x\in X$ such that $Tx=y$.


(The open mapping theorem on Banach spaces follows from the fact that $Y^*_\tau$ if always complete if $Y$ is a Banach space, called the Grothendieck completeness criterion.)

\end{pf}



\begin{prb}[Densely defined closed operators]
Let $(X,X^*)$ and $(Y,Y^*)$ be dual pairs.
We will always consider the weak topologies if not mentioned.
Let $T:\dom T\subset X\to Y$ be a densely defined closed linear operator.
Note that $\ker T$ is closed because if $x_i\in\ker T$ is a net with $x_i\to x$ in $X$, then $(x_i,Tx_i)=(x_i,0)\to(x,0)$ in $X\times Y$, so the closedness of $T$ implies that $Tx=0$.
Consider the induced operator $V:\dom V\subset X/\ker T\to\bar\ran\,T$ into the closure of the range, which is always a densely defined dense injection with $\dom V=\dom T/\ker T$ and $\ran V=\ran T$ as vector spaces.
Furthermore, we can decompose $V$ via the graph to construct a diagram of everywhere defined continuous linear operators
\[\begin{tikzcd}[sep=small]
X\rar{(1)}&X/\ker T&\gra V\lar[swap]{(2)}\rar{(3)}&\bar\ran\,T\rar{(4)}&Y
\end{tikzcd}\]
where (1) is a quotient map, (2) and (3) are dense injections, and (4) is a closed embedding.
For the operator $T$, we can consider the following six conditions.
\begin{parts}
\item (1) is a homeomorphism iff $T$ is injective.
\item (2) is an embedding iff $T$ is continuous.
\item (2) is a surjection iff $T$ is everywhere defined.
\item (3) is an embedding iff $T$ is ...
\item (3) is a surjection iff $T$ has closed range.
\item (4) is a homeomorphism iff $T$ has dense range.
\end{parts}
The proofs are clear.
\end{prb}








\begin{prb}[Weak dual operators]
Let $(X,X^*)$ and $(Y,Y^*)$ be dual pairs.
We will always consider the weak topologies if not mentioned.
Let $T:\dom T\subset X\to Y$ be a densely defined closed linear operator.
Consider the following daigram.
\[\begin{tikzcd}[sep=small]
X\rar{(1)}&X/\ker T&\gra V\lar[swap]{(2)}\rar{(3)}&\bar\ran\,T\rar{(4)}&Y\\
Y^*\rar{(1^*)}&Y^*/\ker T^*&\gra V^*\lar[swap]{(2^*)}\rar{(3^*)}&\bar\ran\,T^*\rar{(4^*)}&X^*
\end{tikzcd}\]
We have a topological isomorphism $(X/\ker T)^*\cong\bar\ran\,T^*$ since $(X/\ker T)^*\to X^*$ and $\bar\ran\,T^*\to X^*$ are closed embeddings with same range, and another topological isomorphism $(\bar\ran\,T)^*\cong Y^*/\ker T^*$ since $Y^*\to Y^*/\ker T^*$ and $Y^*\to(\bar\ran\,T)^*$ are quotient maps with same kernel.
In the above diagram, (1) and (4$^*$) are mutually duals, (4) and (1$^*$) are mutually duals.
We can also check the followings easily.
\begin{parts}
\item (2) is an embedding iff (2$^*$) is a surjection.
\item (2) is a surjection iff (2$^*$) is an embedding.
\item (2) is a homeomorphism iff (2$^*$) is a homeomorphism.
\item (3) is an embedding iff (3$^*$) is a surjection.
\item (3) is a surjection iff (3$^*$) is an embedding.
\item (3) is a homeomorphism iff (3$^*$) is a homeomorphism.
\end{parts}
We may assume $T:\dom T\subset X\to Y$ is a dense injection so that $V=T$ in the proof.

\end{prb}
\begin{pf}

(a)
If $(x,Tx)\mapsto x$ is an embedding, meaning $T$ is continuous, 
for $y^*\in Y^*$, $T^*y^*$ is a continuous linear functional on $X$

(e)
Suppose $(x,Tx)\mapsto Tx$ is a surjection.
If $T^*y_i^*\to0$, then $\<Tx,y_i^*\>\to0$ implies $y_i\to0$, so $(T^*y^*,y^*)\mapsto T^*y^*$ is an embedding.



\end{pf}


\begin{prb}[Operators on Fre\'chet spaces]
Let $(X,X^*)$ and $(Y,Y^*)$ be dual pairs.
Recall that for an everywhere defined linear operator between $X\to Y$ the weak continuity and the Mackey continuity are equivalent.

Let $T:\dom T\subset X\to Y$ be a densely defined closed between Fr\'echet spaces.
Note that (b) and (c) are equivalent, and (d) and (e) are equivalent by the open mapping theorem.
We can ask only four conditions: injectivity, boundedness, closed range, and dense range.
Note that $T$ is boundedly invertible if and only if it is bijective on its domain.
\end{prb}
\begin{pf}

$T$ is a Mackey embedding iff $T$ is surjective:
If $T$ is a Mackey embedding, then the completeness of the Mackey topology of $X_\tau$ implies that the range of $T$ is closed in the Mackey topology, so $T$ is surjective.
Conversely, if $T$ is surjective, then the direct application of the open mapping theorem on $T$ implies that $T$ is a Mackey embedding.

\end{pf}

\[-\]
We want to investigate sufficient conditions in order that continuity implies everywhere definedness.
If we may assume the closed convex hull of $\{T^*y_i^*\}$ is compact in $X^*$ for any(?) net $y_i^*\to0$, then
\[|\<x,T^*y_i^*\>|\le|\<Tx_0,y_i^*\>|+|\<x_0-x,T^*y_i^*\>|\]
can be estimated by taking $x_0$ independent of $i$ so that $x\in\dom T^{**}=\dom T$.
However, in a different way, we can see that it is enough to have the complete Mackey topology. Why is it okay?




\[-\]
<Strong bidual>

We have in general $X_\tau^*\ne X_\beta^*$.




bounded below iff 1,2,3
surjective iff 3,4
boundedly invertible iff 1,3,4

For symmetric operators, 4 implies 1
For self-adjoint or normal operators, 1,4 are equivalent

point spectrum if 1 fails
residue spectrum if 1 hold but 4 fails
continuous spectrum if 1,4 hold but 3 fails






\begin{prb}[Cores]
\end{prb}

\begin{prb}[Sum of unbounded operators]
\end{prb}

\begin{prb}[Composition of unbounded operators]
\end{prb}

\begin{prb}[Inverse of unbounded operators]
Let $T:\dom T\subset X\to Y$ be an injective linear operator.
\[\dom T^{-1}:=\ran T.\]
\end{prb}



\section{Symmetric and self-adjoint operators}


\begin{prb}[Symmetric operators]
Let $H$ be a Hilbert space.
A densely defined linear operator $A$ on $H$ is called \emph{symmetric} if $A\subset A^*$, that is, $\<Ax,y\>=\<x,Ay\>$ for $x,y\in\dom A$.
Let $A$ be a symmetric operator.
Then, $A$ is always closable with closure $A^{**}$ since $A^*$ is densely defined because $A$ is densely defined, and since $A^*$ is closed because every adjoint operator is closed.
If the closure of $A$ is self-adjoint, then it is called \emph{essentially self-adjoint}.
In general, instead of self-adjointness, it is easy to check a given linear operator is symmetric.
\begin{parts}
\item Every symmetric extension of $A$ is a restriction of $A^*$. In particular, $A$ has maximal symmetric extensions.
\item A self-adjoint operator is maximal, and a maximal symmetric operator is closed.
\item A symmetric operator is essentially self-adjoint if and only if it has a unique self-adjoint extension.
\end{parts}
\end{prb}
\begin{pf}
(a) .
\end{pf}


Let $A$ be a closed symmetric operator on a Hilbert space $H$.
We want to ask the following questions:
Is $A$ self-adjoint?
If not, does $A$ admit self-adjoint extensions?
Which self-adjoint extension generate the appropriate quantum dynamics?


\begin{prb}[Spectra of closed symmetric operators]
Let $A$ be a closed symmetric operator on a Hilbert space $H$.
We have $A$ is injective, bounded, of closed range, of dense range if and only if $A^*$ is of dense range, bounded, of closed range, injective, respectively.
\begin{parts}
\item If $\lambda\in\C\setminus\R$, then $\lambda-A$ is injective and has closed range.
\item $\C\setminus\R\to\Z_{\ge0}:\lambda\mapsto\dim\ker(\bar\lambda-A^*)$ is locally constant.
\item $A$ is self-adjoint if and only if $\sigma(A)\subset\R$.
\end{parts}
\end{prb}
\begin{pf}
(a)
By the symmetry of $A$, we have
\[\|(\lambda-A)x\|^2=\|(\Re\lambda-A)x+i\Im\lambda x\|^2=\|(\Re\lambda-A)x\|^2+\|i\Im\lambda x\|^2\gtrsim\|x\|^2,\qquad x\in\dom A.\]

(b)

(c)
If $A$ is self-adjoint, then $A\pm i=A^*\pm i$ is surjective so that $\sigma(A)\subset\R$.
Suppose conversely $\sigma(A)\subset\R$ so that $A-\lambda$ is surjective and $A^*-\lambda$ is injective for all $\lambda\in\C\setminus\R$.
Let $y\in\dom A^*$.
By the surjectivity of $A+i$, there is $x\in\dom A$ such that $(A^*+i)y=(A+i)x=(A^*+i)x$, and it implies $y=x$ by the injectivity of $A^*+i$.
Therefore, $y\in\dom A$ and $A^*=A$.
\end{pf}


\begin{prb}[Unbounded normal operators]
Let $N$ be a normal operator on a Hilbert space $H$, a densely defined closed operator such that $N^*N=NN^*$.
We define the \emph{continuous functional calculus} of $N$ as the $*$-homomorphism $\Phi:C_0(\sigma(N))\to B(H):f\mapsto f(N)$ such that
\[Nf(N)=(zf)(N)=\bar{f(N)N},\qquad N^*f(N)=(\bar zf)(N)=\bar{f(N)N^*}\]
for all $f\in C_0(\sigma(N))$ satisfying $zf\in C_0(\sigma(N))$, where $z\in C_0(\sigma(N))$ denotes the inclusion $\sigma(N)\to\C$.
Define $R:=(1+N^*N)^{-1}$ and $B:=N(1+N^*N)^{-\frac12}$, sometimes called the \emph{resolvent} and the \emph{bounded transform} of $N$.
\begin{parts}
\item $R$ is an everywhere defined bounded self-adjoint operator.
\item $B$ is an everywhere defined bounded normal operator.
\item The continuous functional calculus uniquely exists.
\end{parts}
\end{prb}
\begin{pf}
(a)
This statement is true for any densely defined closed linear operator $T$ on a Hilbert space, instead of $N$.
We prove $1+T^*T$ is a boundedly invertible self-adjoint operator.
Consider the inequality
\[\|(1+T^*T)x\|^2=\|x\|^2+2\|Tx\|^2+\|T^*Tx\|^2\ge\|x\|^2,\qquad x\in\dom T^*T.\]
The operator $1+T^*T$ is then clearly injective, and is also surjective because for any $z\in H$, since the graph of $T$ and the swapped graph of $-T^*$ in $H\oplus H$ are mutually orthogonal complements, there is $x\in\dom T$ and $y\in\dom T^*$ such that $(z,0)=(x,Tx)+(-T^*y,y)$, which gives $z=x-T^*y=x-T^*(-Tx)=(1+T^*T)x$.

It is closed because the surjectivity of $1+T^*T$ implies that the inverse $(1+T^*T)^{-1}$ is everywhere defined and bounded by the above inequality, which implies $(1+T^*T)^{-1}$ is closed so that the original operator $1+T^*T$ is also closed.
It is densely defined because if $(x,Tx)\in\gra T$ is orthogonal to $\{(y,Ty):y\in\dom T^*T\}$, then it follows from the surjectivity of $1+T^*T$ that $x=0$ by
\[0=\<x,y\>+\<Tx,Ty\>=\<x,y\>+\<x,T^*Ty\>=\<x,(1+T^*T)y\>,\qquad y\in\dom T^*T,\]
so $\dom T^*T$ is dense in $\dom T$, and also in $H$.
The self-adjointness is now clear from $0\notin\sigma(1+T^*T)$.

(b)
The operator $R$ is the everywhere defined bounded linear operator on $H$ such that $\ran R=\dom N^*N$ such that $N^*NR=1-R$.
We also have $0\le R\le1$.
We first prove $\ran R^{\frac12}\subset\dom N$ for everywhere definedness of $B=NR^{\frac12}$.
Fix an element $R^{\frac12}x\in\ran R^{\frac12}$ and arbitrary $\e>0$.
Since an injective normal operator $R^{\frac12}$ has dense range, we can take $R^{\frac12}x_0\in\ran R^{\frac12}$ satisfying $\|x-R^{\frac12}x_0\|<\e$.
Then, since
\[\|NRx_0\|^2
=\<RN^*NRx_0,x_0\>
=\<R^{\frac12}(1-R)x_0,R^{\frac12}x_0\>
=\<(1-R)R^{\frac12}x_0,R^{\frac12}x_0\>
\le\|R^{\frac12}x_0\|^2,\]
we have $R^{\frac12}x\in\dom N$ by limiting $\e\to0$ on
\begin{align*}
|\<R^{\frac12}x,N^*y\>|
&\le|\<R^{\frac12}(x-R^{\frac12}x_0),N^*y\>|+|\<Rx_0,N^*y\>|\\
&\le\|R^{\frac12}\|\|x-R^{\frac12}x_0\|\|N^*y\|+|\<NRx_0,y\>|\\
&\le\e\|N^*y\|+\|NRx_0\|\|y\|\\
&\le\e\|N^*y\|+\|R^{\frac12}x_0\|\|y\|\\
&\le\e\|N^*y\|+(\|x\|+\e)\|y\|\to\|x\|\|y\|,\qquad y\in\dom N^*.
\end{align*}
Therefore, $B$ is everywhere defined, and the boundedness, by one, automatically follows in the proof.

Now we check the normality.
Be cautious that $B^*$ is the closure of a densely defined bounded operator $R^{\frac12}N^*$ but not itself.
Let $x\in\dom N$.
Since $Rx\in\ran R=\dom N^*N\subset\dom N$, we have $N^*NRx=x-Rx\in\dom N$, so $NN^*NRx$ is well-defined in $H$ and
\[RNx=RN(1+N^*N)Rx=R(N+NN^*N)Rx=R(1+NN^*)NRx=R(1+N^*N)NRx=NRx.\]
Thus, $N^*Rx\in\dom N$ implies $RNN^*Rx=NRN^*Rx$, so
\[B^*BRx=R^{\frac12}N^*NR^{\frac32}x=R^{\frac12}(1-R)R^{\frac12}x=(R-R^2)x=RNN^*Rx=NRN^*Rx=BB^*Rx.\]
Since the injectivity of $R^2$ deduces that $R\dom N\supset R\ran R=\ran R^2$ is dense, we have $B^*B=BB^*$.



(c)
Observe that
\[\sigma(N)\to\sigma(B)\cap B(0,1):\lambda\mapsto\lambda(1+|\lambda|^2)^{-\frac12}\]
is a homeomorphism, so we can define an faithful non-degenrate representation
\[C_0(\sigma(N))\to C(\sigma(B))\to B(H),\]
where $C(\sigma(B))\to B(H)$ is the continuous functional calculus of the bounded normal operator $B$.
We want to prove it preserves the action of $N$ and $N^*$, also after extension to Borel functional calculus.

We also want to prove $\pi(C_0(\sigma(N)))''$ is the smallest von Neumann algebra where $N$ is affiliated with.

We also want to prove we can always regard $f(N)$ as normal for finite Borel function $f$, in particular densely defined and closed by taking closure.


When we do (continuous or Borel) functional calculus of polynomially unbounded functions, it is safe to fix $\xi\in\dom(1+N^*N)^n$ for sufficiently large $n$, i.e. the common core of operators occuring in computation.

For the continuous functional calculus for finitely generated commutative unital C$^*$-subalgebra of $B(H)$, we can show the existence of common core by collecting the compactly supported functions.


uniqueness..


\end{pf}



\begin{prb}
For an unbounded Borel function $f$ on $\sigma(T)$, there are two methods.
One method is the spectral truncation using $f1_{|f|\le n}$.
The other method is using $h\in\C(z,\bar z)$ be such that $h^{-1}:f(\sigma(T))\to\C$ is an embedding to a bounded set, the inverse bounded transform $z\mapsto z(1-|z|^2)^{-1}$ for example, and take the bounded Borel functional calculus with $h^{-1}\circ f$ and define $f(T):=h(h^{-1}\circ f(T))$.
We want to show $f(T)$ is independent of the choice of $h$.
\begin{parts}
\item 
\end{parts}
\end{prb}

\begin{prb}
Consider a net $\{F\}$ of all finite subsets of $[0,1]$.
Then, $1_F\uparrow1$ pointwisely in $B_b([0,1])$, but $\omega(1_F)=0$ for Radon measure $\omega\in C([0,1])^*$, which means $1_F$ does not converge to the unit $\sigma$-weakly in $C([0,1])^{**}$.
\begin{parts}
\item pointwise bounded sequential convergence
\item $1_{\{\lambda\}}(N)$ is the projection onto the eigenspace corresponding to $\lambda$.
\item $f(N)$ is approximated in norm by projections.
\item $f(VNV^*)=Vf(N)V^*$ for $V$ such that $V^*V$ is the identity on $(\ker N)^\perp=\bar\ran\,N$.
\end{parts}
\end{prb}
\begin{pf}
(b)
We may assume $\lambda=0$.
If $1_{\{\lambda\}}(N)\xi=\xi$, then
\[N\xi=N1_{\{\lambda\}}(N)\xi=(z1_{\{\lambda\}})(N)\xi=(\lambda1_{\{\lambda\}})(N)\xi=\lambda1_{\{\lambda\}}\xi=\lambda\xi.\]
Conversely let $N\xi=\lambda\xi$.
Define $f_n\in B_b(\C)$ such that
\[f_n(z):=\begin{cases}
1-n|z-\lambda|&\text{ if }|z-\lambda|\le n^{-1},\\
0&\text{ if }|z-\lambda|\ge n^{-1}.
\end{cases}\]
It satisfies $0\le1-f_n\le n|z-\lambda|$ and $f_n\downarrow 1_{\{\lambda\}}$ poinwisely.
Then,
\[\|(1-f_n(N))\xi\|^2=\<|1-f_n|^2(N)\xi,\xi\>\le n^2\<|N_\lambda|^2\xi,\xi\>=n^{-2}\|(N-\lambda)\xi\|^2=0\]
implies the strong convergence $1_{\{\lambda\}}(N)\xi=\lim_nf_n(N)\xi=\xi$.

(d)
If $N$ is bounded, then we can check the diagram
\[\begin{tikzcd}[row sep=small]
C(\sigma(N))\ar[equals]{d}\ar{r}{\Phi_N} & B(H) \ar{d}{V\cdot V^*}\\
C(\sigma(N))\ar{r}{\Phi_{VNV^*}} & B(VH)
\end{tikzcd}\]
commutes on the dense $*$-subalgebra $\C[z,\bar z]$.

$C_0(\sigma(N))\to B(VH):f\mapsto Vf(N)V^*$ satisfies the axiom of functional calculus for $VNV^*$?
We can check it is a unital $*$-homomorphism such that
\[(VNV^*)Vf(N)V^*=VNf(N)V^*=V(zf)(N)V^*\]
\end{pf}






\begin{prb}[Cayley transform]
There is a one-to-one correspondence between the unitary operators from $K_+$ to $K_-$, the deficiency subspaces.

If $A$ is a densely defined closed symmetric operator, then
\[Ux:=\begin{cases}0&\text{ if }x\in L^+,\\(T-i)(T+i)^{-1}x&\text{ if }x\in(L^+)^\perp,\end{cases}\]
is a partial isometry with initial and final spaces $(L^+)^\perp$ to $(L^-)^\perp$ such that $\dom A=(1-U)(L^+)^\perp$.
\begin{parts}
\item If $A$ is self-adjoint, then $1-U$ is injective and $\dom A=\ran(1-U)$.
\item The Cayley transform provides a one-to-one correspondence between self-adjoint operators $A$ and unitary operators $U$ satisfying $\ker(1-U)=0$.
\item
\end{parts}
\end{prb}


\begin{prb}[Kato-Rellich theorem]
\end{prb}



\begin{prb}[Non-negative symmetric operators]
Let $A$ be a non-negative symmetric operator on a Hilbert space $H$.
Define a Hilbert space $H_1$ by the completion of the inner product space $\dom A$ given such that $\<x,y\>_1:=\<(1+A)x,y\>$.
We have a dense inclusion $T:H_1\to H$ satisfying $Tx=x$ and
\[\<(1+A)x,y\>=\<T^{-1}x,T^{-1}y\>_1=\<(T^{-1})^*T^{-1}x,y\>=\<(TT^*)^{-1}x,y\>,\qquad x,y\in\dom A,\]
so that $TT^*:H\to H$ is also a dense inclusion, which is a bounded self-adjoint operator.
Define a self-adjoint operator $\tilde A:=(TT^*)^{-1}-1:\dom\tilde A\subset H\to H$ with domain $\dom\tilde A:=\ran(TT^*)$.
Then, we can check $\tilde A$ extends $A$ as
\[\tilde Ax=(TT^*)^{-1}x-x=(1+A)x-x=Ax,\qquad x\in\dom A.\]
The self-adjoint operator $\tilde A$ is called the \emph{Friedrichs extension}.

Krein characterization.

For $x\in H$, let $\<TT^*x,y\>+\<\tilde ATT^*x,Ay\>=0$ for all $y\in\dom A$.
\begin{align*}
0&=\<TT^*x,y\>+\<\tilde ATT^*x,Ay\>\\
&=\<TT^*x,y\>+\<x,Ay\>-\<TT^*x,Ay\>\\
&=\<TT^*x,y\>+\<x,Ay\>-\<\tilde ATT^*x,y\>\\
&=\<TT^*x,y\>+\<x,Ay\>-\<x,y\>+\<TT^*x,y\>\\
&=2\<TT^*x-x,y\>+\<x,(1+A)y\>\\
\end{align*}
\end{prb}


\begin{prb}[Multiplication operators]
Let $(X,\mu)$ be a localizable measure space.
For $f\in L_\loc^0(X,\mu)$ an almost everywhere finite measurable function, we define a linear operator $m(f)$ on $L^2(X,\mu)$ by multiplication
\[m(f)\xi:=f\xi,\qquad \xi\in\dom m(f):=\{\xi\in L^2(X,\mu):f\xi\in L^2(X,\mu)\}.\]

almost everywhere zero function
\begin{parts}
\item $L^0_\loc(X,\mu)$ is identified with the set of all normal operators on $L^2(X,\mu)$ affiliated with $m(L^\infty(X,\mu))$.
\end{parts}
\end{prb}
\begin{pf}
(a)
We first prove that for a measurable function $f:X\to\C$ the multiplication operator $m(f)$ is a normal operator on $L^2(X,\mu)$ affiliated with $m(L^\infty(X,\mu))$.

The operator $m(f)$ is densely defined since if we let $E_n:=f^{-1}(B(0,n))\subset X$ be a non-decreasing sequence of measurable subsets so that $E_n\uparrow X$ as $n\to\infty$ almost everywhere, then for any $\xi\in L^2(X,\mu)$ we have $E_n\xi\to\xi$ in $L^2(X,\mu)$ as $n\to\infty$ and $E_n\xi\in\dom m(f)$.

The adjoint $m(f)^*$ is densely defined since

The operator $m(f)$ is closed because if $\xi_n$ is a sequence in $\dom m(f)$ such that $\xi_n\to\xi$ and $m(f)\xi_n\to\eta$ in $L^2(X,\mu)$, then we have $\xi\in\dom m(f)$ by limit $n\to\infty$ on
\begin{align*}
|\<\xi,m(f)^*\zeta\>|
&\le|\<\xi-\xi_n,m(f)^*\zeta\>|+|\<\xi_n,m(f)^*\zeta\>|\\
&=|\<\xi-\xi_n,m(f)^*\zeta\>|+|\<m(f)\xi_n,\zeta\>|\\
&\le|\<\xi-\xi_n,m(f)^*\zeta\>|+|\<m(f)\xi_n-\eta,\zeta\>|+|\<\eta,\zeta\>|\to|\<\eta,\zeta\>|,\qquad\zeta\in\dom m(f)^*,
\end{align*}
and $m(f)\xi=\eta$ by limit $n\to\infty$ on
\begin{align*}
|\<m(f)\xi-\eta,\zeta\>|
&\le|\<m(f)\xi-m(f)\xi_n,\zeta\>|+|\<m(f)\xi_n-\eta,\zeta\>|\\
&=|\<\xi-\xi_n,m(f)^*\zeta\>|+|\<m(f)\xi_n-\eta,\zeta\>|\to0,\qquad\zeta\in\dom m(f)^*.
\end{align*}

affiliated? easy if we know $m(L^\infty(X,\mu))$ is maximal abelian subalgebra.


Conversely, let $T$ be a normal operator affiliated with $m(L^\infty(X,\mu))$.
By the spectral theorem, $1_{[-n,n]}(T)\in m(L^\infty(X,\mu))$.
We can construct a non-decreasing sequence of measurable subsets $E_n$ such that $m(1_{E_n})=1_{[-n,n]}(T)$.



Let $f_{n,i}:=T1_{E_n\cap F_i}\in L^2(F_i,\mu)$.
$f_{n,i}$ converges to $f$ locally in measure... in $L^0_\loc(X,\mu)$.
We need to check $m(f)=T$.

\end{pf}

\begin{prb}[Polar decomposition]
If $T:H\to H$, then for $T=V|T|$, $V$ is a partial isometry which connects from the complement of the kernel to the closure of the range as a unitary.
Same for unbounded operator.

\[T=\mat{T&0\\0&0}:(\ker T)^\perp\oplus\ker T\xrightarrow{|T|}(\ker T)^\perp\oplus\ker T\xrightarrow{V}\bar{\ran T}\oplus(\ran T)^\perp\]


$T$ is normal then $(\ker T)^\perp=\bar{\ran T}$.


polar decomposition
polar decomposition of symmetric operator?
polar decomopsition changes spectrum or domains?

support projection
\end{prb}


\section{Infinitesimal generators}

\begin{prb}[Stone theorem]
Let $u:\R\to U(H)$ be a weakly continuous unitary flow on a Hilbert space $H$.
Then, there is a unique self-adjoint operator $h$ on $H$ such that $e^{ish}=u_s$ for $s\in\R$.
\end{prb}
\begin{pf}
Define a one-parameter family of bounded operators
\[h_s:=\frac{u_s-1}{is},\quad e_s:=\frac1s\int_0^su_t\,dt,\qquad s\in\R\setminus\{0\},\]
where the integral is justified by the continuous $*$-homomorphism $u:M(\beta\R)_\sigma\to B_\sigma(H_\sigma):s^{-1}1_{[0,s]}\mapsto e_s$ between the weak$^*$ topology and the weak operator topology, which extends $u:\R\to U(H)$.
Since $s^{-1}1_{[0,s]}\to\delta_0$ in $M(\beta\R)$, we have $e_s\to1$ weakly in $B(H)$.
Define a linear operator $h$ on $H$ such that
\[h\xi:=\lim_{s\to0}h_s\xi,\qquad\xi\in\dom h:=\{\xi\in H:\lim_{s\to0}h_s\xi\text{ exists in }H_\sigma\},\]
where the limits are in the weak topology $H_\sigma$.



First, $\dom h$ is weakly dense in $H$ because if we choose any $\xi\in H$, then we have $u_f\xi\in\dom h$ for any function $f\in BV(\R)$ and we can take an weak$^*$ approximate unit of $M(\beta\R)$ in $BV(\R)$.
In more detail, we have $hu_f=iu_\mu$ where $f'=\mu\in M(\R)$ by
\begin{align*}
\<h_su_f\xi-iu_{f'}\xi,\eta\>
&=\int_\R\<u_t\xi,\eta\>\left(\frac{f(t)-f(t-s)}s\,dt-d\mu(t)\right)\to0,\qquad\eta\in H,\ s\to0.
\end{align*}
(We did not check yet)

Second, we prove $h$ is weakly closed.
As a lemma, we introduce an operator version of the fundamental theorem of calculus formulated as
\[h_s\xi=e_sh\xi,\qquad\xi\in\dom h,\ s\in\R\setminus\{0\},\]
which can be shown by introducing $f(s):=\<u_s\xi,\eta\>$ and $F(s):=\int_0^sf(t)\,dt$ as
\begin{align*}
\<e_sh\xi,\eta\>
&=\<h\xi,e_s^*\eta\>=\lim_{t\to0}\<h_t\xi,e_s^*\eta\>=\lim_{t\to0}\<e_sh_t\xi,\eta\>\\
&=\lim_{t\to0}\left\<\frac1s\int_0^su_r\frac{u_t-1}{it}\,dr\,\xi,\eta\right\>\\
&=\frac1{is}\lim_{t\to0}\int_0^s\frac{f(r+t)-f(r)}t\,dr\\
&=\frac1{is}\lim_{t\to0}\left(\frac{F(s+t)-F(s)}t-\frac{F(t)-F(0)}t\right)\\
&=\frac1{is}(f(s)-f(0))=\<h_s\xi,\eta\>,\qquad\eta\in H.
\end{align*}
To prove the claim, if we take a net $\xi_i\in\dom h$ such that $\xi_i\to\xi$ and $h\xi_i\to\xi'$ weakly in $H$, then for any $\e>0$ and for sufficiently small neighborhood $U$ of zero in $\R$ we have
\begin{align*}
|\<h_s\xi-\xi',\eta\>|
&\le|\<h_s(\xi-\xi_i),\eta\>|+|\<h_s\xi_i-e_s\xi',\eta\>|+|\<e_s\xi'-\xi',\eta\>|\\
&\le|\<\xi-\xi_i,h_s^*\eta\>|+|\<h\xi_i-\xi',e_s^*\eta\>|+\e,\qquad s\in U,\ \eta\in H,
\end{align*}
which implies the weak convergence $h_s\xi\to\xi'$ as $s\to 0$ by taking limit on $i$.

Next, we can prove $h$ is self-adjoint.
Let $\xi\in\dom h^*$ and $s\in\R\setminus\{0\}$.
Since $\{\eta\in H:e_s^*\eta\in\dom h\}$ is dense because if $\eta\in$ then
\[\cdots,\]
with $he_s^*\eta=h_{-s}\eta$, we have
\[\<e_sh^*\xi,\eta\>=\<\xi,he_s^*\eta\>=\<\xi,h_{-s}\eta\>=\<h_s\xi,\eta\>\]
for dense choices of $\eta$ in $H$.
It implies
\[|\<h_s\xi-h^*\xi,\eta\>|\le|\<h_s\xi-e_sh^*\xi,\eta\>|+|\<e_sh^*\xi-h^*\xi,\eta\>|=|\<(e_s-1)h^*\xi,\eta\>|\to0,\qquad\eta\in H,\ s\to0,\]
so $\xi\in\dom h$ and $h\xi=h^*\xi$.
The claim follows from the same argument but applying conversely.

Finally, we claim that the functional calculus gives $e^{ish}=u_s$ for $s\in\R$.
Since the functional calculus is a $*$-homomorphism, $e^{ish}$ is unitary on $H$ for each $s\in\R$.
\end{pf}

Cores and invaraint spaces?


\begin{prb}[Smooth and analytic vectors]
Cores
\begin{parts}
\item If $T$ is symmetric and $D_0$ is dense, then $T|_{D_0}$ is essentially self-adjoint.
\end{parts}
\end{prb}

\begin{prb}[Resolvent convergence]
\end{prb}








\section{Decomposition of spectrum}

\begin{prb}
Let $T:\dom T\subset E\to F$ be a linear operator between Banach spaces.
We define the \emph{point spectrum} and the \emph{continuous spectrum} of $T$ as
\[\sigma_p(T):=\{\lambda\in\C:\lambda-T\text{ is not injective}\},\qquad
\sigma_c(T):=\{\lambda\in\C:\lambda-T\text{ is a dense inclusion}\},\]
and the \emph{residual spectrum} as $\sigma_r(T):=\sigma(T)\setminus(\sigma_p(T)\cup\sigma_c(T))$.
\end{prb}

\begin{align*}
\sigma
&=\sigma_p\cup\sigma_c\cup\sigma_r\\
&=\sigma_{ess}\cup\sigma_d\\
&=\bar{\sigma_{pp}}\cup\sigma_{ac}\cup\sigma_{sc}.
\end{align*}


\[\sigma=\sigma_p\sqcup\sigma_c\sqcup\sigma_r=\bar{\sigma_{pp}}\cup\sigma_{ac}\sigma_{sc}=\sigma_d\sqcup\sigma_{ess,5}.\]





\section*{Exercises}


\begin{prb}[Strict topology]
Let $H$ be a Hilbert space.
Let $(T_\alpha)\subset B(H)$ and $K\in K(H)$.
\begin{parts}
\item The strong$^*$ topology and the strict topology agree on bounded sets of $B(H)$.
\end{parts}
\end{prb}

\begin{prb}[Unitary group]
Let $H$ be a Hilbert space.
\begin{parts}
\item The weak topology and the strict topology agree on $U(H)$.
\end{parts}
\end{prb}


\begin{prb}[Bounded increasing nets]
Let $T_\alpha$ be a bounded increasing net of bounded self-adjoint operators on $H$.
\begin{parts}
\item $T_\alpha$ converges strictly. In particular, $T_\alpha\to T$ strictly iff $T_\alpha\to T$ weakly.
\end{parts}
\end{prb}
\begin{pf}
Define $T$ such that
\[\<Tx,y\>:=\lim_\alpha\sum_{k=0}^3i^k\<T_\alpha(x+i^ky),x+i^ky\>.\]
The convergence is due to the monotone convergence in $\R$.
We can check it is a well-defined bounded linear operator by considering the bounded sesquilinear form.
Then, $T_\alpha\to T$ weakly by definition, and $\sigma$-strongly because the net is increasing.
\end{pf}




\begin{prb}[Distributional operators]
\begin{parts}
\item Every continuous linear operator $T:\cD(\R)\to\cD'(\R)$ naturally defines a closable densely defined operator $T:\dom T\to L^2(\R)$ with $\dom T:=\cD(\R)$.
\end{parts}
\end{prb}


\begin{prb}[Differential operators on intervals]
Let $D:\dom D\subset H\to H$ be a linear operator $H:=L^2([0,1])$ such that
\[Df(x):=if'(x),\qquad f\in\dom D:=C_c((0,1)).\]
It is symmetric.

\begin{parts}
\item $\dom\bar D=H_0^1((0,1))$.
\item $\dom D^*=H^1((0,1))\subset C([0,1])$.
\item The family of self-adjoint extensions $\{D_\alpha\}$ can be parametrized by $\alpha\in\T$, where
\[\dom D_\alpha=\{f\in H^1((0,1)):\lambda f(0)=f(1)\}.\]
\item $\tilde D$ has no self-adjoint extension if
\[\dom\tilde D=C^\infty((0,1))\cap C_0((0,1]).\]
\end{parts}
\end{prb}
\begin{pf}
(d)
has no self-adjoint extension because we have deficiency indices $n^+=1$ and $n^-=0$ (maybe).
\end{pf}

\begin{prb}[Sch\"rodinger operators]
For the \emph{potential} $V\in L^0_\loc(\R^d)\cap\cD'(\R^d)$ with the same symbol as the multiplication operator on $L^2(\R^d)$, let $H$ be a linear operator on $L^2(\R^d)$ defined by
\[H\psi:=-\frac{\hbar^2}{2m}\Delta\psi+V\psi,\qquad\psi\in\dom H:=C_c(\R^d),\]
where $\hbar$ and $m$ are positive real constants.
It is called the \emph{Schr\"odinger operator}, and simply we write $H=-\Delta+V$ by putting $\hbar=1$ and $m=\frac12$.

The eigenvectors associated to the discrete spectrum is called \emph{bound eigenstates}.


\end{prb}


\begin{prb}[Hydrogen atom]
Consider the Schr\"odinger operator $H:=-\Delta-|x|^{-1}$ defined on $L^2(\R^3)$.
We want to investigate the spectral decomposition of $H$ by diagonalization.
\begin{parts}
\item $H$ is self-adjoint.
\item $\sigma_d(H)=\{\}$
\end{parts}
\end{prb}
\begin{pf}
The orbital comes from the diagonalization of the Laplace-Beltrami operator on the unit sphere.
\end{pf}


\begin{prb}[Periodic Schr\"odinger operators]
It is diagonalized to the direct integral of elliptic operators defined on the Brillouin torus.
\end{prb}
\begin{pf}
\end{pf}





\chapter{Operator theory}
\section{Toeplitz operators}

invariant subspace problem
Beurling theorem
Hardy and Bergman and Bloch spaces
JB* triple





\part{Operator algebras}
\chapter{Banach algebras}

\section{Spectra of elements}

A \emph{Banach algebra} is a complete normed algebra, and a \emph{unital Banach algebra} is a Banach algebra that is unital and satisfies $\|1\|=1$.
(For a Banach algebra $A$ that is unital, there is a complete algebra renorming such that $\|1\|=1$.)
If an element $a$ of a unital Banach algebra $A$ satisfies $\|1-a\|<1$, then we can easily see that $a$ is invertible in $A$ with $\|a^{-1}\|\le(1-\|1-a\|)^{-1}$, because the Neumann series $\sum_{k=0}^\infty(1-a)^k$ exists in $A$ and defines an inverse of $a$.




\section{Ideals}
\begin{prb}[Ideals]
\begin{parts}
\item If $I$ is a left ideal, then $A/I$ is a left $A$-module.
\end{parts}
\end{prb}

\begin{prb}[Modular left ideals]
A left ideal $I$ is called \emph{modular} if there is $e\in A$ such that $a-ae\in I$ for all $a\in A$.
The element $e$ is called a \emph{right modular unit} for $I$.
\begin{parts}
\item $I$ is modular if and only if $A/I$ is unital(?).
\item A proper modular left ideal is contained in a maximal left ideal.
\item $I$ is a maximal modular left ideal if and only if $I$ is a modular maximal left ideal.
\item There is a non-modular maximal ideal in the disk algebra.
\end{parts}
\end{prb}

\begin{prb}[Closed ideals]
\begin{parts}
\item closure of proper left ideal is proper left.
\item maximal modular left ideal is closed.
\end{parts}
\end{prb}


\begin{prb}[Unitization]
Let $A$ be an associative complex algebra.
Since $A$ is a module over $A$ itself, there is a algebra homomorphism $A\to L(A)$, so we can define
\[\tilde A:=\{\,a+\lambda\in L(A):a\in A,\ \lambda\in\C\,\}.\]
It is called the \emph{Doroh extension} of $A$.
If $A$ is not unital, then it is usually called the \emph{unitization}.
\begin{parts}
\item If $A$ is normed, then $\tilde A$ is a normed algebra such that there is an isometric embedding $A\to\tilde A$.
\item If $A$ is Banach, then $\tilde A$ is a Banach algebra.
\item $A\oplus\C$ is topologically isomorphic to $\tilde A$ as normed spaces.
\end{parts}
\end{prb}
\begin{pf}
(a)
Since $A$ is normed, the space of bounded operators $L(A)$ has a natural normd algebra structure together with an isometry $A\to L(A)$.
Then, $\tilde A$ is a normed $*$-algebra with induced norm
\[\|a+\lambda\|_{L(A)}=\sup_{b\in A}\frac{\|ab+\lambda b\|}{\|b\|}\]
Then, $A$ is a normed $*$-subalgebra of $\tilde A$ because the norm and involution of $A$ agree with $\tilde A$.

(b)
Suppose $a_n+\lambda_n$ is Cauchy in $\tilde A$.
Since $A$ is complete so that it is closed in $\tilde A$, we can induce a norm on the quotient $\tilde A/A$ so that the canonical projection is (uniformly) continuous so that $\lambda_n$ is Cauchy.
Also, the inequality $\|a\|\le\|a+\lambda\|+|\lambda|$ shows that $a_n$ is Cauchy in $A$.

Since a finite dimensional normed space is always Banach and $A$ is Banach, $\lambda_n$ and $a_n$ converge.
Finally, the inequality $\|a+\lambda\|\le\|a\|+|\lambda|$ implies that $a_n+\lambda_n$ converges.

(c)
Check the topology on $A\oplus\C$ in detail...
\end{pf}



unitization, homomorphisms, category(direct sum, product, etc.)

$B(\C^n)=M_n(\C)$ is simple, but $B(H)$ is not simple.

% approximate identity, norm of left multiplication







\section{Gelfand theory}

\begin{prb}[Spectra of elements in unital Banach algebras]
Let $a$ be an element of a unital Banach algebra $A$.
The \emph{spectrum} of $a$ in $A$ is defined to be the set
\[\sigma_A(a):=\{\lambda\in\C:\lambda-a\text{ is not invertible in }A\},\]
and the \emph{resolvent set} of $a$ in $A$ is defined to be its complement $\rho_A(a):=\C\setminus\sigma_A(a)$.
If the ambient algebra $A$ is clear in its context, we often omit it to just write $\sigma(a)$ and $\rho(a)$.
\begin{parts}
\item $\sigma(a)$ is compact.
\item $\sigma(a)$ is non-empty.
\item If $A$ is further a division ring, then $A\cong\C$. This result is called the \emph{Gelfand-Mazur theorem}.
\end{parts}
\end{prb}
\begin{pf}
(a)
If $\lambda\in\C$ satisfies $|\lambda|>\|a\|$, then $\lambda-a=\lambda(1-\lambda^{-1}a)$ is always invertible because $\|\lambda^{-1}a\|<1$, so $\lambda\notin\sigma(a)$ and the spectrum is bounded.
To show the closedness, it suffices to prove the set of invertibles $A^\times$ is open in $A$.
For $a\in A^\times$, $a+h$ has the inverse $(1+a^{-1}h)^{-1}a^{-1}$ if $h\in A$ is sufficiently small such that $\|a^{-1}h\|<1$, so we are done.

(b)
Suppose the spectrum $\sigma(a)$ is empty so that the resolvent function $\C\to A:\lambda\mapsto(\lambda-a)^{-1}$ is well-defined on the entire domain $\C$.
Note that $a\ne0$.
Because $A^\times\to A^\times:a\mapsto a^{-1}$ is norm differentiable in the sense that
\[\frac{\|b^{-1}-a^{-1}-(-a^{-1}(b-a)a^{-1})\|}{\|b-a\|}
=\frac{\|(a^{-1}-b^{-1})(b-a)a^{-1}\|}{\|b-a\|}
\le\|a^{-1}-b^{-1}\|\|a^{-1}\|\to0,\qquad b\to a,\]
the resolvent function is weakly continuous and weakly holomorphic on $\C$.
Since
\[\|(\lambda-a)^{-1}\|=\|\lambda^{-1}(1-\lambda^{-1}a)^{-1}\|
=\Bigl\|\lambda^{-1}\sum_{k=0}^\infty(\lambda^{-1}a)^k\Bigr\|
\le(2\|a\|)^{-1}\sum_{k=0}^\infty2^{-k}=\|a\|^{-1},\qquad\lambda\in\C\setminus B(0,2\|a\|),\]
the resolvent function is bounded.
The Liouville theorem implies that it is weakly constant, which is indeed a constant function because $A^*$ separates points of $A$.
It means that $a\in\C$ and contradicts to $\sigma(a)=\varnothing$.

(c)
For any $a\in A$, by the part (b), there must be $\lambda$ such that $\lambda-a$ is not invertible.
In a division ring, zero is the only non-invertible element, so $\lambda=a$.
\end{pf}

\begin{prb}[Spectral radius]
Let $a$ be an element of a unital Banach algebra $A$.
The \emph{spectral radius} of $a$ in $A$ is defined to be
\[r(a):=\sup_{\lambda\in\sigma(a)}|\lambda|.\]
\begin{parts}
\item $r(a)\le\inf_n\|a^n\|^{\frac1n}$.
\item $\limsup_n\|a^n\|^{\frac1n}\le r(a)$, i.e. $r(a)=\lim_n\|a^n\|^{\frac1n}$.
\end{parts}
\end{prb}
\begin{pf}
(a)
Since $(\lambda-a)^{-1}=\lambda^{-1}(1-\lambda^{-1}a)^{-1}$ exists if $|\lambda|>\|a\|$, we have $r(a)\le\|a\|$ for all $a\in\cA$.
For every $\lambda\in\sigma(a)$ and every integer $n\ge1$ we have
\[|\lambda|^n=|\lambda^n|\le r(a^n)\le\|a^n\|,\]
and it proves $r(a)\le\inf_n\|a^n\|^{\frac1n}$.

(b)
Fix $\omega\in A^*$ and $a$.
Since the resolvent function $\rho(a)\to A:\lambda\mapsto(\lambda-a)^{-1}$ is weakly holomorphic, on the domain $\C\setminus\bar{B(0,r(a))}\subset\rho(a)$ we can consider the Laurent expansion
\[\omega((\lambda-a)^{-1})=\sum_{k=-\infty}^\infty a_k\lambda^k,\qquad|\lambda|>r(a),\]
with a coefficient sequence $a_k\in\C$.
Since on a smaller domain $\C\setminus\bar{B(0,\|a\|)}$ we have
\[\omega((\lambda-a)^{-1})=\omega(\lambda^{-1}(1-\lambda^{-1}a)^{-1})=\omega(\lambda^{-1}\sum_{k=0}^\infty\lambda^{-k}a^k)=\sum_{k=0}^\infty\omega(a^k)\lambda^{-k-1},\qquad|\lambda|>\|a\|,\]
we can determine the coefficients $a_k$ by the identity theorem
\[\omega((\lambda-a)^{-1})=\sum_{k=0}^\infty\omega(a^k)\lambda^{-k-1},\qquad|\lambda|>r(a).\]

It implies that if we take any positive $\lambda>r(a)$, then the sequence $a^k\lambda^{-k-1}$ in $A$ indexed by $k$ is weakly bounded, hence is bounded in norm by the uniform boundedness principle.
Let $\|a^n\|\le C_\lambda\lambda^{n+1}$ for all $n\ge1$.
Then,
\[\limsup_{n\to\infty}\|a^n\|^{\frac1n}\le\limsup_{n\to\infty}C_\lambda^{\frac1n}(\lambda^{n+1})^{\frac1n}=\lambda.\]
If we limit $\lambda\downarrow r(a)$, we are done.
\end{pf}

\begin{prb}[Spectral invariance]
For fixed element, smaller the ambient algebra, less ``holes'' in the spectrum.
Let $A\subset B$ be a closed subalgebra containing $1_A$.
Note that $A$ may be unital even for $1_B\notin A$.
\begin{parts}
\item $B^\times$ is clopen in $A^\times\cap B$.
\end{parts}
\end{prb}


semisimplicity and symmetricity

\begin{prb}[Spectra of commutative unital Banach algebras]
Let $A$ be a commutative unital Banach algbera.
A \emph{character} or a \emph{one-dimensional representation} of $A$ is a non-zero algebra homomorphism $\omega:A\to\C$.
Denote by $\hat A$ or $\sigma(A)$ the set of all characters of $A$ and endow with the weak$^*$ topology on $\hat A\subset A^*$.
We call this space the \emph{character space} or the \emph{spectrum} of $A$.
\begin{parts}
\item $\hat A$ is a closed subset of the unit sphere of $A^*$.
\item If $A$ is generated by $a\in A$ in the sense that $\C[a]$ is dense in $A$, then $\hat A$ is homeomorphic to $\sigma(a)$.
\item For $a\in A$ and $\lambda\in\C$, we have $\lambda\in\sigma(a)$ if and only if there is $\omega\in\hat A$ such that $\omega(a)=\lambda$.
\end{parts}
\end{prb}
\begin{pf}

(c)
If $\lambda\notin\sigma(a)$, then $\omega(1)=1$ implies $\lambda-\omega(a)=\omega(\lambda-a)=\omega((\lambda-a)^{-1})^{-1}$ cannot be zero.
Conversely, let $\lambda\in\sigma(a)$.
Since the closed ideal generated by $\lambda-a$ is proper, there is a maximal ideal $I$ containing $\lambda-a$.
By the Gelfand-Mazur theorem, the quotient homomorphism $A\to A/I\cong\C$ defines a one-dimensional representation $\omega\in\hat A$ such that $\omega(\lambda-a)=0$, so we are done.
\end{pf}


\begin{prb}[Gelfand transform]
Let $A$ be a commutative unital Banach algebra.
The \emph{Gelfand transform} or the \emph{Gelfand representation} is the algebra homomorphism
\[\Gamma:A\to C(\hat A):a\mapsto(\omega\mapsto\omega(a)).\]
\begin{parts}
\item $\Gamma$ has the image separating points by definition.
\item $\Gamma$ is isometric if and only if $r(a)=\|a\|$ for all $a\in A$.
\end{parts}
\end{prb}
\begin{pf}
(a)

\end{pf}


\begin{prb}[Non-unital Banach algebras]
\end{prb}


\section{Holomorphic functional calculus}


\begin{prb}[Holomorphic functional calculus]
Let $a$ be an element of a unital Banach algebra $A$.
Let $f$ be a holomorphic function on a neighborhood $U$ of $\sigma(a)$.
Let $\gamma$ be any positively oriented smooth simple closed curve in $U$ enclosing $\sigma(a)$.
Define $f(a)\in A$ by the Bochner integral
\[f(a):=\int_\gamma f(\lambda)(\lambda-a)^{-1}\,d\lambda.\]
Let $\Hol(\sigma(a))$ be the Fr\'echet algebra of all holomorphic functions on a neighborhood of $\sigma(a)$ endowed with the topology of compact convergence.
We define the \emph{holomorphic functional calculus} or the \emph{Dunford-Riesz calculus} by a faithful unital algebra homomorphism $\Phi:\Hol(\sigma(a))\to A$ such that $\Phi(zf)=a\Phi(f)=\Phi(f)a$ for all $f\in\Hol(\sigma(a))$.

Contour integrals of weakly holomorphic functions.
It is a bounded weakly continuous function on the contour, we can define the integral.


\begin{parts}
\item $f(a)$ is independent of the choice of $\gamma$.
\item spectral mapping.
\item power series.
\end{parts}
\end{prb}
\begin{pf}
(a)


\end{pf}

\section*{Exercises}
\begin{prb}[Basic properties of spectrum]
Let $A$ be a unital Banach algebra.
\begin{parts}
\item $\sigma(ab)\setminus\{0\}=\sigma(ba)\setminus\{0\}$. In particular, we cannot have $ab-ba=1$. The left and right shift operators give an counterexample.
\item If $\sigma(a)$ is non-empty, then $\sigma(p(a))=p(\sigma(a))$.
\end{parts}
\end{prb}
\begin{pf}
(a)
Intuitively, the inverse of $1-ab$ is $c=1+ab+abab+\cdots$.
Then, $1+bca=1+ba+baba+\cdots$ is the inverse of $1-ba$.
\end{pf}

$C_b(\Omega)$ $\ell^\infty(S)$ $L^\infty(\Omega)$ $B_b(\Omega)$ $A(\D)$
$B(X)$

\begin{prb}
In $C(\R)$, the modular ideals correspond to compact sets.
\end{prb}

\begin{prb}[Disk algebra]
\begin{parts}
\item Every continuous homomorphism is an evaluation.
\end{parts}
\end{prb}

\begin{prb}[Polynomial convexity]
(See Conway)
\end{prb}

\begin{prb}[Inclusion relation on spectra]
\begin{parts}
\item $\sigma(a+b)\subset\sigma(a)+\sigma(b)$ and $\sigma(ab)\subset\sigma(a)\sigma(b)$ for unital cases.
\item $\sigma(a^{-1})=\sigma(a)^{-1}$ for unital cases.
\item $r(a)^n=r(a^n)$.
\end{parts}
\end{prb}

\begin{prb}[Spectral radius function]
\begin{parts}
\item upper semi-continuous
\end{parts}
\end{prb}

\begin{prb}[Vector-valued complex function theory]
Let $\Omega$ be an open subset of $\C$ and $X$ a Banach space.
For a vector-valued function $f:\Omega\to X$, we say $f$ is \emph{differentiable} if the limit
\[\lim_{\lambda\to\lambda_0}(\lambda-\lambda_0)^{-1}(f(\lambda)-f(\lambda_0))\]
exists in $X$ for every $\lambda\in\Omega$, and \emph{weakly differentiable} if the limit
\[\lim_{\lambda\to\lambda_0}(\lambda-\lambda_0)^{-1}\<f(\lambda)-f(\lambda_0),x^*\>\]
exists in $\C$ for each $x^*\in X^*$ and every $\lambda\in\Omega$.
Then, the followings are all equivalent.
\begin{parts}
\item $f$ is differentiable.
\item $f$ is weakly differentiable.
\item For each $\lambda_0\in\Omega$, there is a sequence $(x_k)_{k=0}^\infty$ such that we have the power series expansion
\[f(\lambda)=\sum_{k=0}^\infty(\lambda-\lambda_0)^kx_k,\]
where the series on the right hand side converges absolutely and uniformly on any closed ball in $\Omega$ centered at $\lambda_0$.
\end{parts}
\end{prb}

\begin{prb}[Exponential of an operator]
\end{prb}







\chapter{C$^*$-algebras}

\section{Continuous functional calculus}
% normal elements, real/imaginary part

\begin{prb}[C$^*$-algerbas]
%history
A \emph{C$^*$-algebra} is a Banach $*$-algebra $A$ such that the norm satisfies the \emph{C$^*$-identity} $\|a^*a\|=\|a\|^2$ for all $a\in A$.
We automatically have $\|1\|=1$ in a unital C$^*$-algebra because $\|1\|=\|1^*1\|^2=\|1\|^2$.
The \emph{standard unitization} or the \emph{Dorroh extension} of a C$^*$-algebra $A$ is a C$^*$-algebra $\tilde A$ defined by a vector space $A\oplus\C$ with the multiplication, involution, and norm are given such that
\begin{parts}
\item
\end{parts}
\end{prb}
\begin{pf}
The C$^*$-identity easily follows from the following inequality:
\begin{align*}
\|(a,\lambda)\|^2&=\sup_{\|b\|=1}\|ab+\lambda b\|^2\\
&=\sup_{\|b\|=1}\|(ab+\lambda b)^*(ab+\lambda b)\|\\
&=\sup_{\|b\|=1}\|b^*((a^*a+\lambda a^*+\bar\lambda a)b+|\lambda|^2y)\|\\
&\le\sup_{\|b\|=1}\|(a^*a+\lambda a^*+\bar\lambda a)b+|\lambda|^2b\|\\
&=\|(a,\lambda)^*(a,\lambda)\|.\qedhere
\end{align*}
\end{pf}





\begin{prb}[Normal elements]
Let $a$ be an element of a C$^*$-algebra $A$.
We say $a$ is \emph{normal} if $a^*a=aa^*$, \emph{self-adjoint} if $a^*=a$, and \emph{unitary} if $A$ is unital and $a^*a=aa^*=1$, respectively.
\begin{parts}
\item A normal element $a$ is unitary if and only if $\sigma(a)\subset\T$.
\item A normal element $a$ is self-adjoint if and only if $\sigma(a)\subset\R$.
\item A normal element $a$ is self-adjoint if and only if $\omega(a)\in\R$ for all $\omega$?
\item A normal element $a$ is positive if and only if $\omega(a)\ge0$ for all $\omega$?
\end{parts}
\end{prb}
\begin{pf}
(a)
We may assume $A$ is unital.
Let $u\in A$ be unitary.
We have $\|u\|^2=\|u^*u\|=\|1\|=1$ and similarly $\|u^*\|=1$.
If $\lambda\in\sigma(u)$, then $|\lambda|\le\|u\|=1$ and $\lambda^{-1}\in\sigma(u^{-1})=\sigma(u^*)$ implying $|\lambda^{-1}|\le\|u^*\|\le1$, we have $|\lambda|=1$ so that $\sigma(u)\subset\T$.
If $\sigma(u)\subset\T$ conversely,

(b)
We may assume $A$ is unital.
By the holomorphic functional calculus, we have
\[e^{ia}=\sum_{k=0}^\infty\frac{(ia)^k}{k!}\in A,\]
and the inverse of $e^{ia}$ is $e^{-ia}$.
Since the involution on $A$ is bounded, we can check $e^{ia}$ is unitary by
\[(e^{ia})^*=\sum_{k=0}^\infty\frac{(-ia)^k}{k!}=e^{-ia}.\]
For every $\omega\in\sigma(A)$, then by the part (a) the equality
\[e^{-\Im\omega(a)}=|e^{i\omega(a)}|=|\omega(e^{ia})|=1\]
proves $\omega(a)\in\R$, hence $\sigma(a)\subset\R$.
\end{pf}



\begin{prb}[Gelfand-Naimark representation theorem]
Let $A$ be a commutative C$^*$-algebra.
Consider the Gelfand transform $\Gamma:A\to C_0(\hat A)$.
\begin{parts}
\item $\Gamma$ is a $*$-isomorphism.
\end{parts}
\end{prb}
\begin{pf}

(a)
The Gelfand transform $\Gamma$ is a $*$-homomorphism since each $\omega\in\sigma(A)$ is a homomorphism, and it preserves involution as
\[\omega(a^*)=\omega(\Re a-i\Im a)=\omega(\Re a)-i\omega(\Im a)=\bar{\omega(\Re a)+i\omega(\Im a)}=\bar{\omega(a)},\qquad a\in A,\]
because $\omega$ sends a self-adjoint element to a number contained in its spectrum, which is real.
It is also an isometry since 
\[\|\Gamma(a)\|=\sup_{\omega\in\sigma(A)}|(\Gamma(a))(\omega)|=\sup_{\omega\in\sigma(A)}|\omega(a)|=r(a),\qquad a\in A,\]
implies that we have
\[\|\Gamma(a)\|^2=\|\Gamma(a^*a)\|=r(a^*a)=\lim_{n\to\infty}\|(a^*a)^{2^n}\|^{\frac1{2^n}}=\|a^*a\|=\|a\|^2,\qquad a\in A\]
by the C$^*$-identity and the spectral radius formula.
Thus, the image $\Gamma(A)$ of an isometric $*$-homomorphism $\Gamma$ is a closed unital $*$-subalgebra of $C(\sigma(A))$, and it separates points by definition.
Then, $\Gamma(A)$ is dense in $C(\sigma(A))$ by the Stone-Weierstrass theorem, which implies $\Gamma(A)=C(\sigma(A))$.
\end{pf}

\begin{prb}[Continuous functional calculus]
Let $A$ be a unital C$^*$-algebra, and $a\in A$ a normal element.
Then, we have a $*$-isomorphism
\[C(\sigma(a))\to C^*(1,a):\id_{\sigma(a)}\mapsto a\]
defined by the inverse of the Gelfand transform, which we call the \emph{continuous functional calculus}.

joint spectrum

\begin{parts}
\item spectral mapping: $\lambda\in\sigma_p(a)$ implies $f(\lambda)\in\sigma_p(f(a))$, $\lambda\in\sigma(a)$ iff $f(\lambda)\in\sigma(f(a))$, composition, ...
\end{parts}
\end{prb}


\begin{prb}[$*$-homomorphisms]
Let $\f:A\to B$ be a $*$-homomorphism between C$^*$-algerbas.
\begin{parts}
\item $\f$ is determined by self-adjoint elements.
\item $\|\f\|=1$ if $\f$ is non-trivial.
\item The quotient of $A$ by a closed ideal $I$ is a C$^*$-algebra.
\item If $\f$ is injective, then it is an isometry.
\item If $\f$ has dense range, then it is surjective.
\end{parts}
\end{prb}
\begin{pf}

Let $\f:A\to B$ be an injective $*$-homomorphism.
We may assume $A$ and $B$ are commutative so that $A=C_0(X)$ and $B=C_0(Y)$.
Then, $\f$ induces a continuous surjective pointed map $Y_+\to X_+$.
The pullback map is an isometry.

For the surjectivity, quotient out by kernel.
\end{pf}





\section{States}


\begin{prb}[Positive elements]
Let $a,b$ be elements of a C$^*$-algebra $A$.
We say $a$ is \emph{positive} and write $a\ge0$ if it is normal and $\sigma(a)\subset\R_{\ge0}$, and the set of all positive elements of $A$ is denoted by $A^+$.
If we define a relation $a\le b$ as $b-a\ge0$, then we can see that it is a partial order on $A$.
\begin{parts}
\item $a\ge0$ if and only if $\|\lambda-a\|\le\lambda$ for some $\lambda\ge\|a\|$.
\item If $a\ge0$ and $\sigma(b)\subset\R_{\ge0}$, then $\sigma(a+b)\subset\R_{\ge0}$.
\item $a\ge0$ if and only if $a=b^*b$ for some $b\in A$.
\end{parts}
\end{prb}
\begin{pf}
(c)
If $a\ge0$, then $b:=a^{\frac12}$ gives $a=b^*b$.
Conversely, if we let $c:=b(b^*b)_-$, then $c^*c=-(b^*b)_-^3\le0$, so we have $cc^*\le0$ and $c^*c+cc^*\le0$ because $\sigma(c^*c)=\sigma(cc^*)$.
However, $c^*c+cc^*=2(\Re c)^2+2(\Im c)^2\ge0$, thus we have $c=\Re c+i\Im c=0$, which implies $(b^*b)_-=-(c^*c)^{\frac13}=0$.
\end{pf}

% Absolute value of an operator



\begin{prb}[Operator monotone operations]
\begin{parts}
\item If $0\le a\le b$, then $a^{-1}\ge b^{-1}$.
\item If $a\le b$, then $cac^*\le cbc^*$.
\item If $0\le a\le B$, then $a^p\le b^p$ for $0\le p\le1$.
\end{parts}
\end{prb}





\begin{prb}[Standard approximate units]
For a von Neumann algebra or a multiplier algebra, we can ask an approximate unit is bounded, directed, or countable.
If we consider only positive elements for an approximate unit, then countable => directed => bounded.



Let $M$ be a $\sigma$-finite von Neumann algebra, and $\cA$ be a $\sigma$-weakly dense $*$-subalgebra of $M$.
Using the Kaplansky density theorem and the $\sigma$-strong metrizability of the bounded part of $M$, take a bounded approximate unit $b_n\in\cA^+$ in $M$ such that $\|b_n\|\le1$ for all $n$ and $b_n\to1$ $\sigma$-strongly.
If we let $p_n$ be the support projection of $\sum_{k\le n}b_k$, then $p_n\uparrow1$.
Inductively define a sequence $e_n\in\cA^+$ by
\[e_0:=1-(1-b_0)^{k_0},\qquad e_n:=1-((1-e_{n-1})(1-b_n)(1-e_{n-1}))^{k_n},\]
where $k_n\in\Z_{>0}$ is taken such that $\omega(p_n-e_n)<n^{-1}$, which can be done since $((1-e_{n-1})(1-b_n)(1-e_{n-1}))^k\to1-p_n$ $\sigma$-strongly as $k\to\infty$.
Then, $e_n\uparrow1$ $\sigma$-weakly, hence $\sigma$-strongly by the monotone convergence theorem.




Let $I:=\{a\in A^+:\|a\|<1\}$.
It is directed since if $a,b\in I$, then
\[c:=(a(1-a)^{-1}+b(1-b)^{-1})(1+a(1-a)^{-1}+b(1-b)^{-1})^{-1}\]
belongs to $I$ with $a\le c$ and $b\le c$.
Define the \emph{standard approximate unit} of the C$^*$-algebra $A$ as a net $e_i\in A$ indexed on $I$ by $e_i:=i$.
If we fix any $a\in A^+$ and $\e>0$, then for any sufficiently advancing $i$ such that $e_i\ge a(a+\e)^{-1}$, by letting $\e\to0$ on
\[\|a-e_ia\|^2=\|a(1-e_i)^2a\|\le\|a(1-e_i)a\|\le\|a(\e(a+\e)^{-1})a\|\le\e\|a\|,\]
we can check $e_ia\to a$ in norm of $A$.

\begin{parts}
\item
\end{parts}
\end{prb}




\begin{prb}[States]
Let $A$ be a C$^*$-algebra.
We say a linear functional $\omega\in A^*$ is \emph{self-adjoint} if $\omega(a^*)=\bar{\omega(a)}$, and \emph{positive} if $\omega(a^*a)\ge0$, for all $a\in A$.
A \emph{state} of $A$ is defined as a normalized positive linear functional on $A$, that is, $\omega\in(A^*)^+$ with $\|\omega\|=1$.
\begin{parts}
\item For $\omega\in A^*$, $\omega$ is positive if and only if $\omega(e_i)\to\|\omega\|$.
\item Let $V$ be a closed linear subspace of $A$ containing the unit of $A$. If $\omega_0:V\to\C$ satisfies $\omega_0(1)=1$ and $\|\omega_0\|=1$, then $\omega_0$ is extended to a state of $A$.
\end{parts}
\end{prb}
\begin{pf}

\end{pf}


\begin{prb}[Pure states]
Let $A$ be a C$^*$-algebra.
A state $\omega$ of $A$ is called \emph{pure} if every positive linear functional on $A$ dominated by $\omega$ is a scalar multiple of $\omega$.

Let $\omega$ be a state of $A$.
\begin{parts}
\item If $\omega$ is multiplicative, then it is pure.
\item If $\omega$ is pure, then its restriction on the center is multiplicative.
\end{parts}
\end{prb}
\begin{pf}
(a)

(b)
Fix $z\in Z(A)$ with $0\le z\le1$ and define $\omega_z\in A^*$ such that $\omega_z(a):=\omega(za)$ for all $a\in A$.
Then, $\omega_z$ is positive by $\omega_z(a)=\omega(z^{\frac12}az^{\frac12})$ for $a\in A$, and the inequality
\[\omega_z(a)=\omega(a^{\frac12}za^{\frac12})\le\omega(a^{\frac12}a^{\frac12})=\omega(a),\qquad a\in A^+\]
implies that there is $\lambda\in\R_{\ge0}$ such that $\omega_z=\lambda\omega$ by the assumption that $\omega$ is pure.
For the standard approximate unit $e_i$ of $A$, we have $\omega(z)=\lim_i\omega_z(e_i)=\lambda\lim_i\omega(e_i)=\lambda$, so
\[\omega(za)=\omega_z(a)=\lambda\omega(a)=\omega(z)\omega(a),\qquad a\in A.\]
Since a C$^*$-algebra is linearly generated by positive elements in the closed unit ball, it implies that $\omega$ is multiplicative on the center $Z(A)$.
\end{pf}


\begin{prb}[Probability regular Borel measures]
We investigate states of the commutative C$^*$-algebra $C_0(X)$, where $X$ is a locally compact Hausdorff space.
\end{prb}

\begin{prb}[Vector states]
We investigate states of the C$^*$-algebra $K(H)$ of compact operators on a Hilbert space $H$.
\end{prb}

\section{Representations}


\begin{prb}[Non-degenerate representations]
Let $A$ be a C$^*$-algebra.
A \emph{representation} of $A$ on a Hilbert space $H$ is a $*$-homomorphism $\pi:A\to B(H)$.
We say a representation $\pi:A\to B(H)$ is \emph{non-degenerate} if $\pi(A)H$ is dense in $H$.
\begin{parts}
\item Every representation has a unique non-degenerate subrepresentation.
\item The following statements are equivalent:
\begin{enumerate}[(i)]
\item $\pi$ is non-degenerate.
\item For each $\xi\in H$ there is $a\in A$ such that $\pi(a)\xi\ne0$.
\item $\pi(e_i)\to1$ strongly for an approximate unit $e_i$ of $A$.
\end{enumerate}
\end{parts}
\end{prb}

\begin{prb}[Cyclic representations]
\emph{cyclic} if there is a vector $\Omega\in H$ such that $A\Omega$ is dense in $H$.
Cyclic decomposition
\end{prb}


\begin{prb}[Irreducible representations]
\emph{irreducible} if there is no proper closed subspace $K\subset H$ such that $\pi(A)K\subset K$.
The following statements are equivalent:
\begin{enumerate}[(i)]
\item $\pi$ is irreducible if and only if $\pi(A)'=\C$.
\item $\pi$ is irreducible if and only if every non-zero vector in $H$ is cyclic.
\end{enumerate}
\end{prb}


\begin{prb}[Gelfand-Naimark-Segal representation]
Let $A$ be a C$^*$-algebra, and $\omega$ be a state on $A$.
The \emph{left kernel} of $\omega$ is defined to be
\[\fn_\omega:=\{a\in A:\omega(a^*a)=0\}.\]
\begin{parts}
\item $\fn_\omega$ is a left ideal of $A$.
\item $\<a+\fn_\omega,b+\fn_\omega\>:=\omega(b^*a)$ is an inner product on $A/\fn_\omega$.
\item There is a unique representation $\pi_\omega:A\to B(H_\omega)$ such that $\pi_\omega(a)(b+\fn_\omega):=ab+\fn_\omega$ for $a,b\in A$.
\item $\pi_\omega:A\to B(H_\omega)$ is a cyclic representation.
\end{parts}
\end{prb}




\section{Ideals}

For a short exact sequence
\[\begin{tikzcd}[column sep=small]0\rar&I\rar&A\rar&B\rar&0\end{tikzcd},\]
we have
\[\begin{tikzcd}[row sep=small]
PS(I) \dar[->>]\rar[hook] & PS(A) \dar[->>] & PS(B) \dar[->>]\lar[hook'] \\
\hat I \dar[->>]\rar[hook,shorten >= 10, shorten <= 10] & \hat A \dar[->>] & \hat B \dar[->>]\lar[hook',shorten >= 10, shorten <= 10] \\
\Prim(I) \rar[hook,swap]{\text{open}} & \Prim(A) & \Prim(B) \lar[hook']{\text{closed}}
\end{tikzcd}\]




\begin{prb}[Modular maximal left ideals]
\end{prb}

\begin{prb}[Primitive ideals]
hull kernel topology
\[PS(A)\cong\{(\pi,\psi)\}/\sim_u,\qquad\hat A\cong\{\pi\}/\sim_u.\]

\[\begin{array}{c|ccc}
A & PS(A) & \hat A & \Prim(A) \\\hline
C(X) & X & X & X \\
K(H) & PH & * & * \\
\tilde K(H) & ? & ? & \{0,K(H)\} \\
B(H) &&&
\end{array}\]
\begin{parts}
\item $\Prim(A)$ is locally compact T$_0$ space.
\item Two maps $PS(A)\to\hat A\to\Prim(A)$ are continuous surjective open maps
\item If $A$ is type I, then $\hat A\to\Prim(A)$ is an homeomorphism.
\end{parts}

\end{prb}




Every morphism $A\to M(B)$ induces the following?:
\[\begin{tikzcd}
PS(B) \rar[->>]\dar & \hat B \rar[->>]\dar & \Prim(B) \dar \\
PS(A) \rar[->>] & \hat A \rar[->>] & \Prim(A).
\end{tikzcd}\]





\section*{Exercises}

% Basic examples
%  C(X), C_0(X)
%  M_n(\C)
%  B(H), K(H), Q(H)

% Schroder-Burnstein thm of representations


\begin{prb}[Projections in $M_2(\C)$]
The space of self-adjoint elements in $M_2(\C)$ is a real vector space spanned by
\[1=\begin{pmatrix}1&0\\0&1\end{pmatrix},\qquad p:=\begin{pmatrix}1&0\\0&0\end{pmatrix},\qquad q:=\frac12\begin{pmatrix}1&1\\1&1\end{pmatrix}.\]
\begin{parts}
\item $(p-q)^2=\frac12$.
\item If we let $\lambda_\pm$ be the eigenvalues of $ap+bq$, then $\lambda_++\lambda_-=a+b$ and $\lambda_+-\lambda_-=\sqrt{a^2+b^2}$.
\item Every functional calculus $f(x)$ of self-adjoint $x$ is a linear combination of $x$ and 1.
\item $ap+bq+c\ge0$ if and only if $a+b+2c\ge\sqrt{a^2+b^2}$.
\item Every projection of rank one is given by $ap+bq+(1-a-b)/2$ for $a^2+b^2=1$.
\end{parts}
\end{prb}

\begin{prb}[Operator monotone square]
Let $A$ be a C$^*$-algebra in which the square function is operator monotone, that is, $0\le a\le b$ implies $a^2\le b^2$ for any positive elements $a$ and $b$ in $A$.
We are going to show that $A$ is necessarily commutative.
Let $a$ and $b$ denote arbitrary positive elements of $A$.
\begin{parts}
\item
Show that $ab+ba\ge0$.
\item
Let $ab=c+id$ where $c$ and $d$ are self adjoints.
Show that $d^2\le c^2$.
\item
Suppose $\lambda>0$ satisfies $\lambda d^2\le c^2$.
Show that $c^2d^2+d^2c^2-2\lambda d^4\ge0$.
\item
Show that $\lambda(cd+dc)^2\le(c^2-d^2)^2$.
\item
Show that $\sqrt{\lambda^2+2\lambda-1}\cdot d^2\le c^2$ and deduce $d=0$.
\item
Extend the result for general exponent: $A$ is commitative if $f(x)=x^\beta$ is operator monotone for $\beta>1$.
\end{parts}
\end{prb}


\begin{prb}[States on unitization]
Let $A$ be a non-unital C$^*$-algebra and $\tilde A$ be its unitization.
Let $\tilde\omega=\omega\oplus\lambda$ be a bounded linear functional on $\tilde A$, where $\omega\in A^*$ and $\lambda\in\C^*=\C$.

Since $A$ is hereditary in $\tilde A$, the extension defines a well-defined injective map $S(A)\to S(\tilde A)$.
We can identify $PS(A)$ as a subset of $PS(\tilde A)$ whose complement is a singleton.
\begin{parts}
\item $\tilde\rho$ is positive if and only if $\lambda\ge0$ and $0\le\rho\le\lambda$.
\item $\tilde\omega$ is a state if and only if $\lambda=1$ and $0\le\omega\le1$.
\item $\tilde\omega$ is a pure state if and only if $\lambda=1$ and $\omega$ is either a pure state or zero.
\end{parts}
\end{prb}


\begin{prb}[Representations of $C_0(X)$]
Let $A=C_0(X)$ and $\mu$ be a state on $A$, a regular Borel probability measure on a locally compact Hausdorff space $X$.
\begin{parts}
\item The left kernel of $\mu$ is $\fn_\mu=\{\,f\in A:f|_{\supp\mu}=0\,\}$.
\item $H_\mu=L^2(X,\mu)$.
\item The canonical cyclic vector is the unity function on $X$.
\end{parts}
\end{prb}

\begin{prb}[Representations of $K(H)$]
\end{prb}

\begin{prb}[Automorphism group of $K(H)$ and $B(H)$]
\end{prb}


\begin{prb}[Approximate eigenvectors]
\end{prb}


\begin{prb}[Kadison transitivity theorem]
\end{prb}

\begin{prb}[Hereditary C$^*$-algebras]
\end{prb}

\begin{prb}[Extreme points of the ball]
Let $A$ be a C$^*$-algebra and let $B_A$ be the closed unit ball of $A$.
\begin{parts}
\item Extreme points of $A_+\cap B_A$ is the projections in $A$.
\item Extreme points of $A_{sa}\cap B_A$ is the self-adjoint unitaries in $A$.
\item Every extreme point of $B_A$ is a partial isometry.
\end{parts}
\end{prb}


\begin{prb}[Category of commutative C$^*$-algebras]
\[\begin{tikzcd}
& \mathrm{CH} \rar\dar[shift right, swap]{\text{disjoint base}} &
\mathrm{LCH} \rar &
\mathrm{cpltH} \\
\mathrm{LCH}_{\mathrm{prop}} \rar &
\mathrm{CH}_* \uar[shift right,swap]{\text{forgetful}} \uar[phantom]{\scriptscriptstyle\boldsymbol{\dashv}}
\end{tikzcd}\]

\[\begin{tikzcd}
& \mathrm{uCC^*Alg}_{\mathrm{unital}} \rar\dar[shift right, swap]{\text{inclusion}} &
\mathrm{CC^*Alg}_{\mathrm{mor}} \rar &
\mathrm{locCC^*Alg} \\
\mathrm{CC^*Alg}_{\mathrm{nondeg}} \rar &
\mathrm{CC^*Alg} \uar[shift right,swap]{\text{unitization}} \uar[phantom]{\scriptscriptstyle\boldsymbol{\vdash}}
\end{tikzcd}\]

The unitization is left adjoint to the inclusion functor.
\end{prb}
\begin{pf}
Let $X$ and $Y$ be locally compact Hausdorff spaces.
We show the continuous maps $\f^*:X\to Y$ corresponds to non-degenerate $*$-homomorphisms $\f:C_0(Y)\to M(C_0(X))\cong C_b(X)\cong C(\beta X)$.
If $f^*:X\to Y$ is continuous, then $\f:C_0(Y)\to C_b(X)$ is well-defined, which is non-degenerate since $e_i\in C_0(Y)$ with $e_i\uparrow 1$ so that for $f\in C_0(X)$ and arbitrary $\e>0$ we have a compact $K\subset X$ such that
\begin{align*}
\|f-\f(e_i)f\|
&=\sup_{x\in X}|(1-e_i(\f^*(x)))f(x)|\\
&\le\sup_{x\in K}|1-e_i(\f^*(x))|\|f\|+\e,
\end{align*}
hence the Dini theorem proves $\|f-\f(e_i)f\|\to0$.
Conversely, if $\f:C_0(Y)\to M(C_0(X))$ is non-degenerate, then the dual gives $\f^*:X\subset\Prob(\beta X)\to\Prob(Y)$, and the image of $x$ is pure on $M(Y)$ since it defines a character on $Y$ by
\[\f^*(x)(fg)=\f(fg)(x)=\f(f)(x)\f(g)(x)=\f^*(x)(f)\f^*(x)(g),\qquad f,g\in C_0(Y).\]
\end{pf}


\section*{Problems}
\begin{enumerate}
\item* A C$^*$-algebra is commutative if and only if a function $f(x)=x(1+x)^{-1}$ is operator subadditive.
%L\"owner-Heinz inequality
\item On a normed algebra, there is a unique C$^*$-algebra structure.
\end{enumerate}






\chapter{Von Neumann algebras}

\section{Normal states}

\begin{prb}[Von Neumann algebras]
A \emph{von Neumann algebra} on a Hilbert space $H$ is a $\sigma$-weakly closed unital $*$-subalgebra of $B(H)$.
We will see later that a $*$-subalgebra of $B(H)$ is weakly closed if and only if it is $\sigma$-strongly$^*$ closed.
A linear map between von Neumann algebras on $H$ is called \emph{normal} if it is continuous between $\sigma$-weak topologies.
We denote by $M_*$ the space of normal linear functionals on $M$.
\begin{parts}
\item $M_*$ is a predual of $M$.
\item A state on $M$ is normal if and only if it is completely additive.
As a corollary, a positive linear map between von Neumann algebras is normal if it is $\sigma$-weakly continuous on bounded parts or commutative von Neumann subalgebras.
\end{parts}
\end{prb}
\begin{pf}
(a)
To prove that $M_*$ is a predual of $M$, we need to show that $M_*$ is Banach and the map $M\to(M_*)^*$ is an isometric isomorphism.
The cokernel of the kernel of $B(H)_*\to M_*$ gives a Banach quotient map $B(H)_*\to F$, whose dual $F^*\to B(H)$ is an isometry.
The Hahn-Banach separation proves that the image of the isometry $F^*\to B(H)$ is the $\sigma$-weak closure of $M$.
Here the $\sigma$-weak closedness of $M$ is not necessary.
Thus, by taking dual for weakly$^*$ dense isometry $M\to F^*$ we get an isometry $F\to M^*$.
We know that $B(H)_*\to F$ is surjective by construction of $F$.
This implies $F$ can be identified with $M_*$ by definition of $M_*$, hence $M_*$ is Banach.

Since the norm topology is stronger than the topology generated by $M$ on $M_*$, the norm-$\sigma(M_*,M)$ continuous bijection $M_*\to M_*$ implies a weakly$^*$ dense isometry $M\to(M_*)^*$.
Because the weak$^*$ topology on $(M_*)^*$ is induced from the $\sigma$-weak topology of $B(H)$ via the dual map $(M_*)^*\to(B(H)_*)^*=B(H)$, the $\sigma$-weak closedness of $M$ implies the bijectivity of $M\to(M_*)^*$.
The norms are induced from $B(H)$, they are isometrically isomorphic.






(b)
A normal state is clearly completely additive.
Let $\omega$ be a completely additive state on $M$.
Let $\{p_i\}$ be a maximal family of orthogonal projections of $M$ such that there exist vectors $\{\xi_i\}$ in $H$ satisfying $\omega(p_ixp_i)\le\<x\xi_i,\xi_i\>$ for all $x\in M^+$, and suppose $p:=\sum_ip_i<1$.
For any fixed $\xi\in(1-p)H$ with $\|\xi\|=1$, let $\{q_j\}$ be a maximal family of orthogonal subprojections of $1-p$ such that $\omega(q_j)>\<q_j\xi,\xi\>$, and suppose further $q:=\sum_jq_j=1-p$.
Then, we have
\[\omega(1-p)=\omega(q)\ge\sum_j\omega(q_j)>\sum_j\<q_j\xi,\xi\>=\<q\xi,\xi\>=\<(1-p)\xi,\xi\>=\|\xi\|^2=1,\]
which contradicts to $\omega(1-p)\le1$, so we have $q<1-p$.
Here we did not use the complete additivity of $\omega$ yet.
Since every subprojection $r'$ of $r:=1-p-q$ satisfies $\omega(r')\le\<r'\xi,\xi\>$, and since every positive element is approximated by finite linear combinations of projections in norm, we have $\omega(rxr)\le\<rxr\xi,\xi\>$ for all $x\in M^+$.
If we consider $\{p_i\}\cup\{r\}$ with the corresponding vector $r\xi$, then it contradicts to the maximality of $\{p_i\}$, so we have $p=1$.
If we let $p_J:=\sum_{i\in J}p_i$ for any finite subset $J$ of $I$, then the complete additivity of $\omega$ implies the norm convergence $p_J\omega\to\omega$ in $M^*$ by
\[|(\omega-p_J\omega)(x)|^2=|\omega(x(1-p_J))|^2\le\omega(1)\omega((1-p_J)x^*x(1-p_J))\le\|x\|^2\omega(1-p_J)\to0\]
as $J\to I$.
For each $i$, the linear functional $p_i\omega$ is normal and so is $p_J\omega$ because it is $\sigma$-strongly continuous by
\[|\omega(xp_i)|^2\le\omega(1)\omega(p_ix^*xp_i)\le\<x^*x\xi_i,\xi_i\>=\|x\xi_i\|,\qquad x\in M,\]
so we have $\omega\in M_*$ because $M_*$ is Banach.
\end{pf}






\begin{prb}[Normal cyclic representations]
Let $M$ be a von Neumann algebra on a Hilbert space $H$.
A vector $\Omega\in H$ is called \emph{separating} if $x\Omega=0$ and $x\ge0$ imply $x=0$.

Properties for a faithful unital normal representation:
\begin{itemize}
\item admits a cyclic vector
\item admits a separating vector
\item admits a cyclic separating vector
\item every normal state is a vector state
\end{itemize}

classification of cyclic normal representation

separability and $\sigma$-finiteness and the existence of separating vectors

\begin{parts}
\item The associated cyclic representation of a normal state is normal.
\item $\xi$ is separating if and only if it is cyclic for $M'$, and it is equivalent that the vector functional $\omega_\xi$ is a faithful normla state of $M$.
\item sufficiently large representation, dependence of weak and strong topologies.
\item Radon-Nikodym
\end{parts}
\end{prb}


A $*$-isomorphism between von Neumann algebras is normal.


\begin{prb}
Jordan decomposition preserves normality?
\end{prb}


\section{Density theorems}



\begin{prb}[Double commutant theorem]
Let $H$ be a Hilbert space.
The \emph{commutant} of a subset $A\subset B(H)$ is the von Neumann algebra $A'$ on $H$ consisting of all elements of $B(H)$ that commute every $a\in A$.
Let $A$ be a non-degenerate $*$-subalgebra of $B(H)$.
By the double commutant theorem, one can describe the von Neumann algebra generated by $A$ in $B(H)$ purely algebraically in terms of commutants.
\begin{parts}
\item $A''$ is the strong closure of $A$.
\item $A''$ is the $\sigma$-strong$^*$ closure of $A$.
\item A $\sigma$-strongly$^*$ closed $*$-subalgebra of $B(H)$ is weakly closed.
\end{parts}
\end{prb}
\begin{pf}
(a)
The strong closedness of $A''$ is clear, so take $x\in A''$.
We claim $x\xi\in\bar{A\xi}$ for $\xi\in H$.
Let $p\in B(H)$ be the projection onto $\bar{A\xi}$.
First, we obtain $p\in A'$ because for any $a\in A$ the left action of $p$ fixes $ap$ and $a^*p$ since their ranges are in $A\xi\subset pH$, and $pap=ap$ and $pa^*p=a^*p$ imply $ap=pa$.
Next, since $(1-p)\xi$ is orthogonal to the dense subspace $AH$ of $H$, we have $p\xi=\xi$.
Therefore, $px\xi=xp\xi=x\xi$ implies $x\xi\in pH=\bar{A\xi}$.

(b)
Since $A''$ is weakly closed and $A$ is self-adjoint, it suffices to show $A$ is $\sigma$-strongly dense in $A''$.
Consider the diagonal inclusion $B(H)\to B(\ell^2\otimes H)$, which is an injective unital normal $*$-homomorphism.
Then, $A$ is non-degenerately represented also in $B(\ell^2\otimes H)$, and we can check that the double commutant of $A$ does not change in the new representation $A\to B(\ell^2\otimes H)$.
One way to check this is using (a).
Now by the part (a) for arbitrary vector in $\ell^2\otimes H$, we deduce the desired result.

(c)
Let $M$ be a $\sigma$-strongly$^*$ closed $*$-subalgebra of $B(H)$.
We may assume that $M$ is non-degenerate in $B(H)$.
Then, by the part (b) we have $M''=M$, which is strongly closed by the part (a).
Since a strongly closed convex set is weakly closed, $M$ is weakly closed.
\end{pf}


\begin{prb}[Kaplansky density theorem]
Let $f:F\to\C$ be a continuous function on a closed subset $F$ of $\C$.
We say $f$ is \emph{strongly continuous} if for every net $x_i\in B(H)$ of normal operators with the spectra $\sigma(x_i)\subset F$ for all $i$, the strong convergence $x_i\to x$ implies the strong convergence $f(x_i)\to f(x)$.

\begin{parts}
\item Since we only consider normal operators, strong and strong$^*$ have no difference.
\item The image of a von Neumann algebra under a normal $*$-homomorphism is a von Neumann algebra.
\end{parts}

\end{prb}
\begin{pf}
Let $A\subset C(F)$ be the set of all strongly continuous functions.
We can check $A$ is a $*$-algebra containing the polynomial algebra $\C[z,\bar z]$.
We will prove that $C_0(F)\subset A$ using the Stone-Weierstrass theorem.

Now we have
\[C_0(F)\cup\C[z,\bar z]\subset A\subset C(F).\]

If $g$ is bounded and continuous on $F$, then
\[g(z)=\frac{g(z)}{1+|z|^2}+\bar z\frac{zg(z)}{1+|z|^2},\qquad z\in F\]
implies $g\in C_0(F)+\bar zC_0(F)\subset A$.

\end{pf}



\begin{prb}[Approximate units for von Neumann algebras]
Let $M$ be a von Neumann algebra on a Hilbert space $H$.
Let $A$ be a $\sigma$-weakly dense $*$-subalgebra.
\begin{parts}
\item There is a net $e_i\in A_1^+$ such that $e_i\to1$ $\sigma$-strongly$^*$.
\item If either $A$ is hereditary in the sense that $AMA\subset A$ or $M$ is countably decomposable, then we may assume $e_i\uparrow1$.
\end{parts}
\end{prb}



\begin{parts}
\item If $\f$ is a normal $*$-homomorphism, then its image is a von Neumann algebra on $H$. (Kaplansky density is needed)
\end{parts}


\section{Projections}


\begin{prb}[Projection lattices]
Let $M$ be a von Neumann algebra. 
Let $P(M)$ be the partially ordered set of all projections in $M$, called the \emph{projection lattice} of $N$
\begin{parts}
\item The linear span of $P(M)$ is $\sigma$-weakly dense in $M$.
\item $P(M)$ is a complete orthomodular lattice.
\item 
\end{parts}
\end{prb}


$1\le s_l(x)+s_r(1-x)$

Since $\ker x\cap\ker y\subset\ker(x+y)$, we have $s_r(x+y)\le s_r(x)\vee s_r(y)$.

$p\wedge s_l(x)=s_l(px)$



\begin{prb}[Support projections for operators]
Let $M$ be a von Neumenna algebra on a Hilbert space $H$.
The \emph{left support projection} or the \emph{range projection} of $x\in M$ is the minimal projection $s_l(x)\in M$ such that $s_l(x)x=x$.

We have $s_r(x)=s_l(x^*)$.
The projections $s_l(x)$ and $1-s_r(x)$ are also called the \emph{range} and \emph{kernel} projections of $x$, respectively.

Riesz refinement?
\begin{parts}
\item The left support projection $s_l(x)\in M$ of $x\in M$ uniquely exists.
\item We have $s_l(x)H=\bar{xH}$ and $Ms_l(x)=\bar{Mx}$.
\item $x^*yx=0$ if and only if $s_l(x)ys_l(x)=0$ for every $y\in M$, and we have $s_r(x)=s(x^*x)=s(|x|)$. In particular, $s_l(x)=s_r(x)$ if $x$ is normal.
\item If $x,y\in M$ satisfies $x^*x\le y^*y$, then there is a unique $v\in M$ such that $x=vy$ and $s_r(v)\le s_l(y)$.
\item For $x\in M$, there is unique partial isometry $v\in M$ such that the polar decomposition $x=v|x|=|x^*|v$ holds with $v^*v=s_r(x)$ and $vv^*=s_l(x)$. Moreover, $x^*=v^*|x^*|=|x|v^*$.
\end{parts}
\end{prb}
\begin{pf}
(a)
Let $x\in M$.
We may assume $0\le x\le1$.
Then, $(xx^*)^{2^{-n}}$ is a bounded increasing sequence in $M$, so it converges strongly to some $s\in M_+$.
We can check that $s$ is a projection by
\[s^2=\cdots=s.\]
Now we show that the projection $s$ is the left support projection of $x$.

(d)
The operator $v_0:\bar{yH}\to\bar{xH}:y\xi\mapsto x\xi$ is well defined because
\[\|x\xi\|^2=\<x^*x\xi,\xi\>\le\<y^*y\xi,\xi\>=\|y\xi\|^2.\]
If we let $v:=v_0s_l(y)$, then we can easily check $x=vy$, and since $v(1-s_l(y))v^*=0$ implies $s_r(v)(1-s_l(y))s_r(v)=0$, we have $s_r(v)\le s_l(y)$.

For the uniqueness, if $v'\in B(H)$ satisfies $y=v'x$ and $v'=v's_l(y)$, then $y^*(v-v')^*(v-v')y=(x-x)^*(x-x)=0$ implies $0=s_l(y)(v-v')^*(v-v')s_l(y)=(v-v')^*(v-v')$, so the uniqueness of $v$ in $B(H)$ follows.
If $u\in M'$ is any unitary, then $uvu^*\in B(H)$ is a partial isometry satisfying the same properties $(uvu^*)x=uvxu^*=uyu^*=y$ and $(uvu^*)s_l(y)=uvs_l(y)u^*=uvu^*$, so the uniqueness implies $uvu^*=v$.
Since unitaries of $M'$ span $M'$, we have $v\in M''=M$.

(e)
Since $x^*x\le|x|^*|x|$, there is $v\in M$ such that $x=v|x|$ and $v=vs(|x|)=vs_r(x)$.
Then, $s_r(x)-v^*v=s_r(x)(1-v^*v)s_r(x)=0$ because $|x|(1-v^*v)|x|=|x|^2-|x|^2=0$, and $s_l(x)-vv^*=s_l(x)(1-vv^*)s_l(x)=0$ because $x^*(1-vv^*)x=|x|^2-|x|^2=0$ and $s_l(v)=s_l(x)$.
The uniqueness of $v$ follows from the part (d) since $s_r(x)=v^*v$ implies $s_r(v)=s_r(v^*v)=s_r(s_r(x))=s_r(x)=s(|x|)$.

The equality $xv^*=|x^*|$ follows from $xv^*=v|x|v^*\ge0$ and $|xv^*|^2=vx^*xv^*=v|x|^2v^*=xx^*=|x^*|^2$.
\end{pf}


\begin{prb}[Support projections for linear functionals]
Let $M$ be a von Neumann algebra on a Hilbert space $H$.
Note that $x\in M$ canonically acts on $\omega\in M_*$ from left as $(x\omega)(\cdot):=\omega(\cdot x)$.
The \emph{left support projection} of $\omega\in M_*$ is the minimal projection $s_l(\omega)\in M$ such that $s_l(\omega)\omega=\omega$.

\begin{parts}
\item The left support projection exists $s_l(\omega)\in M$ of $\omega\in M_*$ uniquely exists.
\item If $\omega\ge0$, then $s(\omega)$ can be characterized by $\fn_\omega=M(1-s(\omega))$. $\fn_{|\omega|}=M(1-s(\omega))$ for any $\omega\in M_*$?
\item $s(\omega_\xi)H=\bar{M'\xi}$.
\end{parts}
\end{prb}
\begin{pf}
(a)

(b)
Let $p\in M$ be the projection such that $\fn_\omega=Mp$.
Since $\omega(p)=0$ implies $\omega(xp)^2\le\omega(xx^*)\omega(p)=0$ for any $x\in M$, the left action of $1-p$ fixes $\omega$, so $1-p\ge s(\omega)$ by the minimality of $s(\omega)$.
Conversely, we have $\omega(1-s(\omega))=\omega(1)-(s(\omega)\omega)(1)=0$ so that $1-s(\omega)\in Mp$, hence $(1-s(\omega))p=p$ and $1-s(\omega)\le p$.
Therefore, $1-s(\omega)=p$.

\end{pf}


\begin{prb}[$\sigma$-weakly closed subalgebras]
Let $M$ be a von Neumann algebra on a Hilbert space $H$.



\begin{parts}
\item If $\fn$ is a $\sigma$-strongly$^*$ closed left ideal of $M$, then there is a unique projection $p\in M$ such that $\fn=Mp$.
\item If $V$ is a left invariant closed subspace of $M_*$, then there is a unique projection $p\in M$ such that $V=M_*p$.
\end{parts}
\end{prb}
\begin{pf}
(a)
If we define $\fa:=\fn^*\cap\fn$, then it is a $\sigma$-strongly$^*$ closed $*$-subalgebra, that is, a von Neumann subalgebra of $M$, and it admits a unit $p\in M$.
Since $p\in\fn$ and $\fn$ is a left ideal, we have $Mp\subset\fn$.
Conversely, if $x\in\fn$, then $x^*x\in\fa$ implies $|x|\in\fa$, and by the polar decomposition $x=v|x|$ we have $x=v|x|=v|x|p\in Mp$.
Therefore, $\fn=Mp$.
If two projections $p$ and $q$ in $M$ satisfy $Mp=Mq$, then since there is a unique unit in a $\sigma$-strongly$^*$ closed $*$-algebra $pMp=qMq$, hence $p=q$ and the uniqueness follows.

\end{pf}



\begin{prb}[]
The \emph{central support projection} of $x\in M$ is the smallest central projection $z(x)$ that fixes $x$ from left or right.

existence
We have $z(x)H=MxH$.


cyclic projection

\end{prb}




\section{Predual}


\begin{prb}[Jordan decomposition for C$^*$-algebras]
Let $A$ be a C$^*$-algebra
\begin{parts}
\item For a normal element $a\in A$ there is a state $\omega$ of $A$ such that $|\omega(a)|=\|a\|$.
\item A self-adjoint bounded linear functional is uniquely represented as the difference of two positive linear functional.




\end{parts}
\end{prb}
\begin{pf}
Note that we will not prove the existence of such a state by the Hahn-Banach extension.


It is trivial if $a=0$, so we let $a\ne0$, and assume $\|a\|=1$ without loss of generality.
We may assume $A$ is unital because a state $\tilde\omega$ of the unitization $\tilde A$ satisfying $\tilde\omega(a)=1$ restricts to a state $\omega:=\tilde\omega|_A$ of $A$, as the norm is exactly one by
\[1=|\omega(a)|\le\|\omega\|\le\|\tilde\omega\|=1.\]
Since $\sigma(a)\cup\{0\}$ is compact, there is $\lambda\in\sigma(a)$ such that $|\lambda|=1$.
The Dirac measure $\delta_\lambda$ induces a state $\omega_0$ of $C^*(1,a)$ via the continuous functional calculus satisfying $\omega_0(a)=\lambda$.
By the Hahn-Banach extension, there is a linear functional $\omega\in A^*$ of $\omega_0\in C^*(1,a)^*$ that is normalized as $\|\omega\|=\|\omega_0\|=1$ and positive with $\omega(1)=\omega_0(1)=1$, which means $\omega$ is a state of $A$.
Fianlly, since $|\omega(a)|=|\omega_0(a)|=|\lambda|=1$, we are done.

(b)
We may assume $A$ is unital since the positivity of linear functionals does not change under the restriction from the standard unitization $\tilde A$ onto $A$, so that $S(A)$ and $-S(A)$ are weakly$^*$ compact in $A^*$.
Then, we are enough to show
\[(A^*)^{sa}_1=\conv(S(A)\cup-S(A)).\]
Note that the right-hand is weakly$^*$-compact because if $(1-t_i)\omega_i^+-t_i\omega_i^-$ is a net in the convex hull of $S(A)\cup-S(A)$, then we can find a subnet such that $\omega_i^+$ and $\omega_i^-$ converge weakly$^*$ in $S(A)$ and $t_i$ converges in $[0,1]$, which implies the extracted subnet converges weakly$^*$ in the convex hull.

Since one inclusion is clear, suppose there exists $\omega\in(A^*)^{sa}_1$ which is not contained in the weakly$^*$ closed compact set $\conv(S(A)\cup-S(A))$.
By the Hahn-Banach separation, and by the fact that the real dual $(A^{sa})^*$ can be identified with the self-adjoint part $(A^*)^{sa}$ of the complex dual of $A$ as real locally convex spaces, there is $a\in A^{sa}$ such that
\[\sup_{\omega'\in S(A)}|\omega'(a)|=\sup_{\omega'\in S(A)\cup-S(A)}\omega'(a)<\omega(a).\]
If we take $\omega'\in S(A)$ such that $|\omega'(a)|=\|a\|$ using the part (b), then it leads to a contradiction, hence the claim follows.
\end{pf}


\begin{prb}[Sherman-Takeda theorem]
Let $A$ be a C$^*$-algebra, and let $\pi_u:A\to B(H_u)$ be the universal representation of $A$ constructed as the direct sum of the cyclic representations associated to all the states of $A$.
Let $M:=\pi_u(A)''$ be the von Neumann algebra on $H_u$ generated by $\pi_u(A)$.
The bidual $A^{**}$ is called the \emph{enveloping von Neumann algebra} of a C$^*$-algebra $A$.
\begin{parts}
\item $A^{**}$ is canonically a von Neumann algebra on a $H_u$ such that the canonical embedding $A\to A^{**}$ induces an isometric isomorphism $(A^{**})_*\to A^*$.
\item $A^{**}$ enjoys a universal property in the sense that every $*$-homomorphism $\f:A\to N$ to a von Neumann algebra $N$ has a unique normal extension $\tilde\f:A^{**}\to N$ of $\f$.
\end{parts}
\end{prb}
\begin{pf}
(a)
Consider the following adjoint maps
\[\pi_u:A\to M,\qquad\pi_u^*:M_*\to A^*,\qquad\pi_u^{**}:A^{**}\to M.\]
The adjoint $\pi_u^*:M_*\to A^*$ is an isometry since
\[\|\pi_u^*(\omega)\|=\sup_{\substack{\|a\|\le1\\a\in A}}|\omega(\pi_u(a))|=\sup_{\substack{\|x\|\le1\\x\in M}}|\omega(x)|=\|\omega\|,\qquad \omega\in M_*\]
by the non-degeneracy of the representation $\pi_u$ and the Kaplansky density.
It is also surjective because if we take $\omega\in A^*$, which can be assumed to be a state by the Jordan decomposition we proved, then the universal representation $\pi_u$ has the cyclic representation associated to $\omega$ as a subrepresentation, so $\omega$ is given by a vector state in $\pi_u$, which means that it gives rise to a normal state of $M$ which extends $\omega$ via $\pi_u$.
Now we have the isometric isomorphism $\pi_u^{**}:A^{**}\to M$, and the $*$-algebra structure on $A^{**}$ can be determined from the von Neumann algebra $M$ on $H_u$, which is unique by the Kaplansky density.

(b)
We can define $\tilde\f$ as the bitranspose of $\f:A\to N_{\sigma w}$, and it is a unique extension because $A$ is $\sigma$-weakly dense in $A^{**}$.
\end{pf}



\begin{prb}[Conditional expectations]
Let $B$ be a C$^*$-subalgebra of a C$^*$-algebra $A$.
A \emph{conditional expectation} is defined as a normalized positive $B$-bimodule map $\e:A\to B$.
\begin{parts}
\item A contractive retraction $\e:A\to B$ is a conditional expectation.
\item A conditional expectation is completely positive.
\end{parts}
\end{prb}

\begin{pf}
(a)
Taking bidual, we may assume $A$ and $B$ are von Neumann algebras $M$ and $N$ on some Hilbert spaces.
Let $x\in M$ and $p\in N$ be a projection.
Since the linear span of projections is norm dense in a von Neumann algebra, it is enough to show $p\e(x)=\e(px)$ and $\e(xp)=\e(x)p$.
Since $p\e$ is idempotent, taking the limit $t\to\infty$ on
\begin{align*}
(1+t)^2\|p\f((1-p)x)\|^2
&=\|p\e((1-p)x)+tp\e((1-p)x))\|^2\\
&=\|p\e((1-p)x)+tp\e(p\e((1-p)x))\|^2\\
&\le\|(1-p)x+tp\e((1-p)x)\|^2\\
&=\|(1-p)x\|^2+t^2\|p\e((1-p)x)\|^2,
\end{align*}
we have $p\e((1-p)x)=0$.
Let $q$ be the unit of $N$.
Substituting $q-p$ and $q$ in the place of $p$ respectively, we obtain
\[(1-p)\e((1-q+p)x)=0,\qquad\e((1-q)x)=0,\]
which imply $(1-p)\e(px)=0$, hence for any $x\in M$ we have
\[p\e(x)=p\e(px)=\e(px).\]
Similarly we can show $\e(x(1-p))p=0$ and $\e(xp)(1-p)=0$, we are done.

(b)
It follows easily from
\[\sum_{i,j}b_i^*\e(a_i^*a_j)b_j=\sum_{i,j}\e(b_i^*a_i^*a_jb_j)=\e(\sum_{i,j}b_i^*a_i^*a_jb_j)=\e(|\sum_{j}a_jb_j|^2)\ge0,\]
where $[a_i]\in A^n$ and $[b_i]\in B^n$ for $n\ge1$.
\end{pf}



\begin{prb}[Sakai theorem]
Let $M$ be a \emph{W$^*$-algebra} or just a \emph{von Neumann algebra} in the intrinsic sense that it does not depend on the choice of Hilbert spaces where it acts, that is, a C$^*$-algebra together with a predual $M_*\subset M^*$.
Consider the canonical weakly$^*$ dense embedding $M\subset M^{**}$ and a faithful unital normal representation of $M^{**}$, constructed by the Sherman-Takeda theorem for example.
We will show that every W$^*$-algebra $M$ is canonically embedded in $M^{**}$ as a weakly$^*$ closed $*$-subalgebra, but in a different way as the canonical embedding, so that $M$ admits a faithful unital normal representation.
In this context, a von Neumann algebra on a Hilbert space is equivalent to just a W$^*$-algebra together with a faithful unital normal representation.
\begin{parts}
\item There is an injective $*$-homomorphism $\pi:M\to M^{**}$ with weakly$^*$ closed image.
\item $\pi$ is a topological embedding with respect to $\sigma(M,M_*)$ and $\sigma(M^{**},M^*)$.
\item A predual of a C$^*$-algebra is unique in the dual space if it exists.
\end{parts}
\end{prb}
\begin{pf}
(a)
Note that the canonical embedding $M\to M^{**}$ between C$^*$-algberas is not continuous with respect to the weak$^*$ topologies.
For the inclusion $M_*\to M^*$, we have the dual map $\e:M^{**}\to M$ that is contractive and idempotent onto $M$, and hence is a $M$-bimodule map.
Since $\e:M^{**}\to M$ is continuous between the weak$^*$-topologies, and since $M$ is weakly$^*$ dense in $M^{**}$, we can check that the kernel of $\e$ is a weakly$^*$ closed ideal of $M^{**}$, so we have a central projection $z\in M^{**}$ such that $\ker\e=(1-z)M^{**}$.
Define $\pi:M\to M^{**}:x\mapsto zx$.
It is a $*$-homomorphism because $z$ is central, and is injective because $\pi$ is a right inverse of $\e$.
Furethermore, the idempotence of $\e$ implies that $zM^{**}=zM$, so the image $\pi(M)=zM=zM^{**}$ is weakly$^*$ closed in $M^{**}$.

(b)
Note that $\pi:M\to M^{**}$ is continuous with respect to the norm topology of $M$ and the weak$^*$ topology of $M^{**}$ so that its adjoint can have the form $\pi^*:M^*\to M^*$.
For $\pi$ to be an embedding between weak$^*$ topologies, it suffices to prove $\pi^*(M^*)=M_*$.
Suppose not and take $\f\in M^*$ satisfying $\pi^*(\f)\notin M_*$.
Because $M_*$ is norm closed in $M^*$, there is $x\in M^{**}$ by the Hahn-Banach extension theorem such that $\pi^*(\f)(x)\ne0$ and $\omega(x)=0$ for all $\omega\in M_*$.
Since $\omega(\e(x))=\omega(x)=0$ for every $\omega\in M_*$ from the definition of $\e$, and since $M_*$ separates points of $M$, we have $\e(x)=0$ so that $zx=0$.
If we take a net $e_i\in M$ such that $e_i\to z$ weakly$^*$ in $M^{**}$, then the $\sigma$-weak continuity of the multiplication by $z$ implies a contradiction
\[\pi^*(\f)(x)=\lim_i\pi^*(\f)(e_i)=\lim_i\f(ze_i)=\f(zx)=0,\]
so we can conclude $\pi^*(M^*)\subset M_*$.
Conversely, if $\omega\in M_*$, then we have $\pi^*(\omega)(x)=\omega(zx)=\omega(x)$ for every $x\in M$ because $(1-z)x\in\ker\e$ acts on $M_*$ trivially by definition of $\e$, so $\omega=\pi^*(\omega)\in\pi^*(M^*)$.

(c)
Since $*$-isomorphism between von Neumann algebras is automatically normal, we can recover the predual by taking adjoint for the identity map on $M$.
\end{pf}


\section{Types}



For von Neumann algebras, we want to compare order topology, measure topology, and operator topologies.
- commutative and non-commutative.
- $M$, $M^{sa}$, $M_1^{sa}$.
- closedness for convex sets or bounded sets.
- continuity of linear operators and functionals.




abelian, finite, purely infinite(every non-zero subprojection is infinite), properly infinite(every non-zero central subprojection is infinite)

central projection = union of components
central support = a kind of minimal union of components
centrally orthogonal


\begin{prb}[Comparison of projections]
l
\begin{parts}
\item For projections $p$ and $q$ in $M$, there is a central projection $z$ in $M$ such that $pz\precsim qz$ and $p(1-z)\succsim q(1-z)$.
\end{parts}
\end{prb}

\begin{prb}[Finite projections]
A projection in a von Neumann algebra is called \emph{finite} if there is no proper Murray-von Neumann equivalent subprojection.
Let $p$ and $q$ be projections in a von Neumann algebra $M$.
\begin{parts}
\item If $M$ is finite, then $p\sim q$ implies $1-p\sim 1-q$.
\item If $p$ and $q$ are finite, then $p\vee q$ is finite.
\item An abelian projection is finite.
\end{parts}
\end{prb}


\begin{prb}[Types of von Neumann algebras]
A von Neumann algebra is called
\begin{enumerate}[(i)]
\item \emph{type I} if every non-zero central projection has a non-zero abelian subprojection,
\item \emph{type II} if every non-zero central projection has a non-zero finite subprojection and there is no non-zero abelian projection,
\item \emph{type III} if there is no non-zero finite projection.
\end{enumerate}
\begin{parts}
\item Every von Neumann algebra $M$ is uniquely decomposed into the direct sum of three von Neumann algebras of each type.
\end{parts}
\end{prb}



V.1.35. For a purely non-abelian von Neumann algebra, every projection is the sum of two equivalent orthogonal projections.



\begin{prb}[Type I]
Let $M$ be a von Neumann algebra of type I.
Then, there are families $\{M_\kappa\}_\kappa$ and $\{H_\kappa\}_\kappa$ of commutative von Neumann algebras $M_\kappa$ and Hilbert spaces $H_\kappa$ satisfying $\dim H_\kappa=\kappa$, indexed by cardinals $\kappa$, such that
\[M\cong\bigoplus_{\kappa\in\mathrm{Card}}M_\kappa\bar\otimes B(H_\kappa).\]
If $M$ is a factor, then $M\cong B(H)$ for a Hilbert space $H$.
\begin{parts}
\item A minimal projection is abelian and a non-zero abelian projection in a factor is minimal.
\end{parts}
\end{prb}














\begin{prb}[Semi-finite and tracial von Neumann algebras]
Let $M$ be a von Neumann algebra.
We say $M$ is \emph{semi-finite} if it admits a faithful semi-finite normal trace, and \emph{tracial} if it admits a faithful normal tracial state.
\begin{parts}
\item regular representation and antilinear isometric involution $J$. $L(G)=\rho(G)'$
\item $M$ is semi-finite if and only if type III does not occur in the direct sum.

\item A factor $M$ has at most one tracial state, which is normal and faithful.
\item A factor is tracial if and only if it is type II$_1$.
\end{parts}
\end{prb}


\begin{prb}
Let $M$ be a von Neumann algebra.
A \emph{center-valued trace} is $M^+\to\hat{Z(M)}^+$.
\begin{parts}
\item There is a faithful semi-finite normal center-valued trace on $M$ if and only if $M$ is semi-finite.
\item If $M$ is a semi-finite factor, then a projection $p\in M$ is finite if and only if $\tau(p)<\infty$.
\end{parts}
\end{prb}

\begin{prb}[Semi-finite traces]
Let $M$ be a von Neumann algebra and $\tau$ is a trace.
For a trace $\tau$
\begin{parts}
\item $\tau$ is semi-finite if and only if $x\in M^+$ has a net $x_\alpha\in L^1(M,\tau)^+$ such that $x_\alpha\uparrow x$ strongly.
\item Let $\tau$ be normal and faithful. Then, $\tau$ is semi-finite if and only if
\[\tau(x)=\sup\{\,\tau(y):y\le x,\ y\in L^1(M,\tau)^+\,\}\quad\text{ for }\quad x\in M^+.\]
\end{parts}
\end{prb}

\begin{prb}[Uniformly hyperfinite algebras]
Let $A$ be a uniformly hyperfinite algebra.
\begin{parts}
\item Every matrix algebra admits a unique tracial state.
\item Every UHF algebra admits a unique tracial state.
\item Every hyperfinite 
\end{parts}
\end{prb}





\section*{Exercises}
\begin{prb}[Extremally disconnected space]
$\sigma(B^\infty(\Omega))$ is extremally disconnected.
\end{prb}

resolution of identity
normal operator theories: multiplicity, invariant subspaces
$L^\infty$ representation


$\sigma$-weakly closed left ideal has the form $Mp$. II.3.12

Let $\fm$ be an algebraic ideal of a von Neumann algebra $M$, and $\bar\fm$ be its $\sigma$-weak closure.
If $x\in(\bar\fm)_+$, then there is an increasing net $(x_i)\subset\fm$ converges to $x$ strongly. II.3.13



binary expansion and hereditary subalgebras

\end{document}