\documentclass{../../large}
\usepackage{../../ikhanchoi}

\newcommand{\wk}{\text{\sc wk}}
\newcommand{\wot}{{\text{\sc wot}}}
\newcommand{\sot}{{\text{\sc sot}}}

\begin{document}
\title{Functional Analysis}
\author{Ikhan Choi}
\maketitle
\tableofcontents

\part{Topological vector spaces}


\chapter{Locally convex spaces}
\section{Vector topologies}

\begin{prb}[Canonical uniformity and bornology]
\end{prb}
\begin{prb}[Metrizability]
Birkhoff-Kakutani
\end{prb}
\begin{prb}[Boundedness of linear operators]
\end{prb}

\section{Seminorms and convex sets}
\begin{prb}[Seminorms]
\[\bigcap_{i=1}^m\{:p()<1\}\]
Equivalent conditions on the continuity of seminorms
\end{prb}
\begin{pf}
\end{pf}
boundedness by seminorms, normability

\section{Continuous linear functionals}
\begin{prb}
Let $\bar{x^*}=(x_1^*,\cdots,x_n^*)\in X^{*n}$.
$\bar{x^*}:X\to\F^n$.
If $x^*\in X^*$ vanishes on $\bigcap_{i=1}^n\ker x_i^*$, then $x^*$ is a linear combination of $\{x_i^*\}$.
\end{prb}



\section{Hahn-Banach theorem}

\begin{prb}[Hahn-Banach theorem]
Let $X$ be a real vector space.
Suppose $V$ is a linear subspace of $X$ and $l:V\to\R$ is a linear functional dominated by a sublinear functional $q:X\to\R$, that is, $l(v)\le q(v)$ for all $v\in V$.
\begin{parts}
\item There is a linear functional $\tilde l:X\to\R$ that extends $l$.
\item There is a linear functional $\tilde l:X\to\R$ that extends $l$ and is dominated by $q$ if $\dim X/V=1$.
\item There is a linear functional $\tilde l:X\to\R$ that extends $l$ and is dominated by $q$.
\end{parts}
\end{prb}
\begin{pf}
(a)
It can be done by the Hamel basis.

(b)
Let $e\in X\setminus V$ so that every vector $x\in X$ can be uniquely written as $x=v+te$ with $v\in V$ and $t\in\R$.
For $v_1,v_2\in V$,
\[l(v_1)+l(v_2)=l(v_1+v_2)\le q(v_1+v_2)\le q(v_1-e)+q(v_2+e)\]
implies
\[l(v_1)-q(v_1-e)\le-l(v_2)+q(v_2+e).\]
Define a linear functional $\tilde l:X\to\R$ such that $\tilde l(v)=v$ and
\[l(v)-q(v-e)\le\tilde l(e)\le-l(v)+q(v+e)\]
for all $v\in V$.
Since
\[\tilde l(v+te)=l(v)+t\tilde l(e)\le l(v)+t(-l(t^{-1}v)+q(t^{-1}v+e))=q(v+te)\]
if $t\ge0$ and
\[\tilde l(v+te)=l(v)+t\tilde l(e)\le l(v)+t(l(-t^{-1}v)-q(-t^{-1}v-e))=q(v+te)\]
if $t\le0$, we have $\tilde l(x)\in q(x)$ for all $x\in X$.

(c)
With $X$ and $q$, Consider a partially ordered set
\[\{(\tilde V,\tilde l)\mid
V\le\tilde V\le X,\ \tilde l:\tilde V\to\R\text{ is a linear extension of $l$ dominated by $q$}\}\]
such that $(V_1,l_1)\prec(V_2,l_2)$ if and only if $V_1\le V_2$ and $l_2|_{V_1}=l_1$.
The nonemptyness and the chain condition is easily satisfied, so it has a maximal element $(\tilde V,\tilde l)$ by the Zorn lemma.
By the part (b), we have $\tilde V=X$.
\end{pf}

\begin{prb}[Complex linear functionals]
Let $X$ be a complex vector space.
Consider a map
\[\begin{array}{ccc}
\{\text{$\C$-linear functionals on $X$}\}
&\to&
\{\text{$\R$-linear functionals on $X$}\}\\
l&\mapsto&\Re l.
\end{array}\]
Let $p$ be a seminorm on $X$ and $l$ a complex linear functional on $X$.
\begin{parts}
\item The above map is bijective.
\item $|l(x)|\le p(x)$ if and only if $|\Re l(x)|\le p(x)$.
\end{parts}
\end{prb}
\begin{pf}
(b)
There is $\lambda$ such that $|\lambda|=1$ and $l(\lambda x)\ge0$.
Then,
\[|l(x)|=|\lambda^{-1}l(\lambda x)|=l(\lambda x)=\Re l(\lambda x)\le p(\lambda x)=p(x).\]
\end{pf}


\begin{prb}[Applications of Hahn-Banach theorem]
\end{prb}




\section*{Exercises}
\begin{prb}[Topology of compact convergence]
\end{prb}





\chapter{Barreled spaces}

\section{Uniform boundedness principle}
\begin{prb}[Barreled spaces]
Let $X$ be a topological vector space.
A \emph{barrel} is an absorbing, balanced, convex, and closed subset of $X$.
A \emph{barreled space} is a topological space in which every barrel is a neighborhood of zero.
\end{prb}

% If a closed convex cone contains a dense subset of absorbing at a point, then it is entire?

\begin{prb}[Uniform boundedness principle]
Let $X$ and $Y$ be topological vector spaces.
Let $\cF$ be a family of continuous linear operator from $X$ to $Y$.
Suppose $\bigcup_{T\in\cF}Tx$ is bounded for each $x\in D$, where $D\subset X$.
\begin{parts}
\item If $D$ is dense in $X$, then $\bigcap_{T\in\cF}T^{-1}\bar U$ is absorbing.
\item If $X$ is barreled, then $\cF$ is equicontinuous.
\end{parts}
\end{prb}



\section{Baire category theorem}

\begin{prb}[Baire spaces]
A topological space is called a \emph{Baire space} if the countable intersection of open dense subsets is always dense.
\begin{parts}
\item If a topological vector space is Baire, then it is barreled.
\item A Baire space is second category in itself.
\item A topological group that is second category in itself is Baire.
\end{parts}
\end{prb}



\begin{prb}[Absorbing sets]
Let $X$ be a topological vector space that is Baire.
A subset $U\subset X$ is said to be \emph{absorbing} if for every $x\in X$ there is a sufficiently large $t>0$ such that $x\in tU$.
Let $U\subset X$.
\begin{parts}
\item If $U$ is closed and absorbing, then $U$ has a non-empty open subset.
\item If $U$ is closed and absorbing, then $U-U$ is a neighborhood of zero.
\item If $U$ is closed, convex, and absorbing, then $U$ is a neighborhood of zero.
\end{parts}
\end{prb}


\begin{prb}[Baire category theorem]
The Baire category theorem proves many exmples of topological vector space are Baire, in particular barreled.
\begin{parts}
\item A complete metric space is Baire.
\item A locally compact Hausdorff space is Baire.
\end{parts}
\end{prb}




\section{Open mapping theorem}

\begin{prb}[Open mapping theorem]
Let $X$ be a $F$-space and $Y$ a barreled space.
Suppose $T:X\to Y$ is a continuous and surjective linear operator.
\begin{parts}
\item $\bar{TU}$ is a neighborhood of zero.
\item $TU$ is a neighborhood of zero.
\end{parts}
\end{prb}

\begin{pf}
(a)
Let $U'$ be a neighborhood of zero such that $U\supset U'-U'$.
Because $T$ is surjective, the set $\bar{TU'}$ is a closed absorbing set, so it contains a non-empty open subset, since $Y$ is barreled.
Thus, $\bar{TU}\supset\bar{TU'}-\bar{TU'}$ is a neighborhood of zero.

(b)
We claim $\bar{TU_{2^{-1}}}\subset TU_1$.
Take $y_1\in\bar{TU_{2^{-1}}}$.

Assume $y_n\in\bar{TU_{2^{-n}}}$.
Since $\bar{TU_{2^{-(n+1)}}}$ is a neighborhood of zero, we have
\[(y_n+\bar{TU_{2^{-(n+1)}}})\cap TU_{2^{-n}}\ne\varnothing.\]
Then, there is $x_n\in U_{2^{-n}}$ such that $Tx_n\in y_n+\bar{TU_{2^{-(n+1)}}}$.
Define
\[y_{n+1}:=y_n-Tx_n.\]

Then, $\sum_{n=1}^\infty x_n$ clearly converges to $x\in U_1$.
Therefore,
\[Tx=\sum_{n=1}^\infty Tx_n=\sum_{n=1}^\infty(y_n-y_{n+1})=y_1.\qedhere\]
\end{pf}


\section*{Exercises}

\begin{prb}
Let $(T_n)$ be a sequence in $B(X,Y)$.
If $T_n$ coverges strongly then $\|T_n\|$ is bounded by the uniform boundedness principle.
\end{prb}

\begin{prb}
There is a closed absorbing set in $\ell^2(\Z_{\ge0})$ that is not a neighborhood of zero;
\[\bar B(0,1)\setminus\bigcup_{i=2}^\infty B(i^{-1}e_i,i^{-2})\]
is a counterexample.
\end{prb}




\begin{prb}
There is no metric $d$ on $C([0,1])$ such that $d(f_n,f)\to0$ if and only if $f_n\to f$ pointwise as $n\to\infty$ for every sequence $f_n$.
Note that this problem is slightly different to the non-metrizability of the topology of pointwise convergence.
\end{prb}

\begin{prb}
We show that there is no projection from $\ell^\infty$ onto $c_0$.
\end{prb}

\begin{prb}[Schur property]
$\ell^1$
\end{prb}

\begin{prb}
Let $\f:L^\infty([0,1])\to\ell^\infty(\N)$ be an isometric isomorphism.
Suppose $\f$ is realised as a sequence of bounded linear functionals on $L^\infty$.
\begin{parts}
\item
Show that $\f^*(\ell^1)\subset L^1$ where $\ell^1$ and $L^1$ are considered as closed linear subspaces of $(\ell^\infty)^*$ and $(L^\infty)^*$ respectively.
\item Show that $\f^*$ is indeed an isometric isomorphism, and deduce $\f$ cannot be realised as bounded linear functionals on $L^\infty$.
\end{parts}
\end{prb}


\begin{prb}[Daugavet property]
\begin{parts}
\item The real Banach space $C([0,1])$ satisfies the Daugavet property.
\end{parts}
\end{prb}
\begin{pf}
Let $T$ be a finite rank operator on $C([0,1])$, and $e_i$ be a basis of $\im T$.
Then, for some measures $\mu_i$,
\[Tf(t)=\sum_{i=1}^n\int_0^1f\,d\mu_ie_i(t).\]
Let $M:=\max\|e_i\|$.

Take $f_0$ such that $\|f_0\|=1$ and $\|Tf_0\|>\|T\|-\frac\e2$.
Reversing the sign of $f_0$ if necessary, take an open interval $\Delta$ such that $Tf_0(t)\ge\|T\|-\frac\e2$ and $|\mu_i|(\Delta)\le\frac\e{4nM}$ for all $i$.
Define $f_1$ such that $f_0=f_1$ on $\Delta^c$, $f_1(t_0)=1$ for some $t_0\in\Delta$, and $\|f_1\|=1$.
Then, $\|Tf_1-Tf_0\|\le\frac\e2$ shows $Tf_1\ge\|T\|-\e$ on $\Delta$.
Therefore,
\[\|1+T\|\ge\|f_1+Tf_1\|\ge f_1(t_0)+Tf_1(t_0)\le1+\|T\|-\e.\]
\end{pf}


\section*{Problems}
\begin{prb}
Let $T$ be an invertible linear operator on a normed space.
Then, $T^{-2}+\|T\|^{-2}$ is injective if it is surjective.
\end{prb}
















\chapter{Weak topologies}
\section{Dual spaces}

\begin{prb}[Bidual]
\end{prb}

\begin{prb}
Let $X$ be a locally convex space.
The \emph{weak topology} is the topology $w$ on $X$ defined by the family of seminorms $\{x\mapsto|\<x,\xi\>|\}_{\xi\in X^*}$.
The \emph{weak$^*$ topology} is the topology $w^*$ on $X^*$ defined by the family of seminorms $\{\xi\mapsto|\<x,\xi\>|\}_{x\in X}$.
Let $J:X\to X^{**}$ be the canonical embedding.
\begin{parts}
\item $(X,w)$ and $(X^*,w^*)$ are locally convex.
\item $(X,w)^*=X^*$.
\item $(X^*,w^*)^*=X$. Every locally convex space is a dual of a locally convex space.
\end{parts}
\end{prb}
\begin{pf}
(a)
The Hahn-Banach theorem implies the Hausdorffness.

(c)
We will only show $(X^*,w^*)^*\subset X$.
If $u\in(X^*,w^*)^*$, then there are $x_1,\cdots,x_m\in X$ such that
\[|\<u,\xi\>|\le\sum_{i=1}^m|\<x_i,\xi\>|\]
for all $\xi\in X^*$.
If we let $\ker\vec x:=\bigcap_{i=1}^m\ker x_i$, then it is a closed subspace of $X^*$ such that $\ker\vec x\subset\ker u$, so we have $u\in\spn\vec x\subset X$.
\end{pf}

\begin{prb}
closure and weak closure of convex subsets
\end{prb}
\begin{pf}
Hahn-Banach
\end{pf}

\begin{prb}[Polar]
\end{prb}


boundedness, incompleteness

\begin{prb}[Weak convergence by dense set]
Let $X$ be a Banach space, $D^*$ a subset of $X^*$, and $\bar{D^*}$ the norm closure of $D^*$.
For example, if $X$ has a predual $X_*\subset X^*$ and $D^*$ is dense in $X_*$, then $\sigma(X,\bar{D^*})$ is the weak$^*$ topology.
\begin{parts}
\item There is a squence $x_n\in X$ converges to zero in $\sigma(X,D^*)$ but not in $\sigma(X,\bar{D^*})$.
\item A bounded sequence $x_n\in X$ converges to zero in $\sigma(X,\bar{D^*})$ if in $\sigma(X,D^*)$.
\end{parts}
\end{prb}
\begin{pf}
(b)
Let $\xi\in\bar{D^*}$ and choose $\eta\in D^*$ such that $\|\xi-\eta\|<\e$.
Then,
\[|\<x_n,\xi\>|\le\|x_n\|\|\xi-\eta\|+|\<x_n,\eta\>|\lesssim\e+|\<x_n,\eta\>|\to\e.\]
\end{pf}



\section{Weak compactness}
\begin{prb}[Banach-Alaoglu theorem]
\end{prb}
\begin{prb}[Eberlein-\v Smulian theorem]
\end{prb}
\begin{prb}[James' theorem]
\end{prb}

\section{Weak density}
Bishop-Phelps theorem

\begin{prb}[Goldstine's theorem]
Let $X$ be a Banach space and $J:X\to X^{**}$ the canonical embedding.
Our claim is that $\bar B$ is weak$^*$-dense in $\bar B_{X^{**}}$.
Let $x_0^{**}\in X^{**}$ with $\|x_0^{**}\|\le1$, and let
\[\bigcap_{i=1}^m\,\{\,x^{**}\in X^{**}:|\<x^{**}-x_0^{**},x_i^*\>|<\e\,\}\]
be an open weak$^*$-neighborhood of zero in $X^{**}$ with $\|x_i^*\|\le1$ and $\e>0$.
Let
\[S:=\bigcap_{i=1}^m\,\{\,x\in X:\<x,x_i^*\>=\<x_0^{**},x_i^*\>\,\}.\]
\begin{parts}
\item $S$ is not empty.
\item $S\cap(1+\e)\bar B_X$ is not empty for all $\e>0$.
\item $\bar B_X$ is weak$^*$-dense in $\bar B_{X^{**}}$
\end{parts}
\end{prb}
\begin{pf}
(a)

(b)
From the part (a), we have $x\in S$.
Suppose $S$ does not intersect $(1+\e)\bar B_X$.
By the Hahn-Banach theorem, there is $y^*\in X^*$ such that
\[y^*|_{S-x}=0,\quad\<x,y^*\>>1+\e,\quad\text{ and }\quad\|y^*\|=1.\]
Since $S-x=\bigcap_{i=1}^m\ker x_i^*$, the linear functional $y^*$ is a linear combination of $x_1^*,\cdots,x_m^*$, so we have
\[1+\e<\<x,y^*\>=\<x_0^{**},y^*\>\le\|x_0^{**}\|\|y^*\|\le1.\]

(c)
Take $\e>0$ such that $\e\max_{1\le i\le m}\|x_i^*\|<1$.
By the part (b), there is $y\in X$ such that $\|y\|\le1+\e$ and $\<y,x_i^*\>=\<x^{**},x_i^*\>$.
If we let $x:=(1+\e)^{-1}y$, then $x\in\bar B_X$ so that
\[|\<x-x_0^{**},x_i^*\>|=|\<x-y,x_i^*\>|=|\<\e x,x_i^*\>|\le\e\|x\|\|x_i^*\|<\e\]
for all $i$.
\end{pf}




\section{Krein-Milman theorem}
Choquet theory


\section{Polar topologies}
Mackey-Arens


\section*{Exercises}
\begin{prb}[James' space]
not reflexive but isometrically isomorphic to bidual
\end{prb}


\begin{prb}[Predual correspondence]
Let $X$ be a Banach space.
Let
\[\{\,(Y,\f)\mid\f:X\to Y^*\text{ is an isometric isorphism}\,\}\]
and
\[\{\,Z\le X^*\mid\bar{B_X}\text{ is compact Hausdorff in }(X,\sigma(X,Z))\,\}.\]

\[(Y,\f)\mapsto\im\f^*|_{J(Y)}\]

\begin{parts}
\item The map is well-defined.
\item The map is surjective. (by Goldstein)
\item The map is injective up to isomorphism for $Y$.
\end{parts}
\end{prb}

\begin{prb}
Let $X$ be a closed subspace of a Banach space $Y$ and \[i:X\to Y\] the inclusion.
Suppose $X$ and $Y$ have preduals $X_*$ and $Y_*$ respectively.
Let \[j:=i^*|_{Y_*}:Y_*\to Z\subset X^*,\]
where $Z:=i^*(Y_*)^-$.
Then we can show
\[j^*:Z^*\subset X^{**}\to Y\]
coincides with $i$ on $X\cap Z^*$.
From the existence of $X_*$ we have $X^{**}\to X$, which is restricted to define a map $k:Z^*\to X$.
\begin{cd}
&X\ar{r}{i}&Y\\
X^{**}\ar{ur}\ar{r}&Z^*\ar{u}{k}\ar{ur}{j}&
\end{cd}
We can show $k$ is an isomorphism so that we have
\[X_*\cong Y_*/Y_*\cap\ker(i^*).\]
\end{prb}

\begin{prb}[Mazur's lemma]

\end{prb}



















\part{Banach spaces}

\chapter{Fr\'echet, Banach, Hilbert spaces}

\section{Banach spaces}
dual is Banach.
Basis problem, Mazur' duck.

\section{Hilbert spaces}
Projections. Reducing subspaces.
Hilbert space classification by cardinal.
Riesz representation theorem.
\begin{prb}
\begin{parts}
\item A Banach space $X$ is isometrically isomorphic to a Hilbert space if there is a bounded linear projection on every closed subspace of $X$.
\end{parts}
\end{prb}

\begin{prb}[Riesz representation theorem]
Let $H$ be a Hilbert space over a field $\F$, which is either $\R$ of $\C$.


We use the bilinear form $\<-,-\>:X\times X^*\to\F$ of canonical duality.
\emph{Dirac} notation $\<-|-\>$ for the inner product of a complex Hilbert spaces such that $\<x,y\>=\<y|x\>$.
The Riesz representation theorem states that a continuous linear functional on a Hilbert space is represented by the inner product with a vector.
\begin{parts}
\item For each $x^*\in H^*$, there is a unique $x\in H$ such that $\<y,x^*\>=\<y,x\>$ for every $y\in H$.
\item $H\to H^*:x\mapsto\<-,x\>$ is a natural linear and anti-linear isomorphism if $\F=\R$ and $\C$, respectively.
\end{parts}
\end{prb}



\section*{Exercises}







\chapter{Bounded linear operators}
\begin{prb}[Bounded belowness in Banach spaces]
Let $T\in B(X,Y)$ for Banach spaces $X$ and $Y$.
The following statements are equivalent:
\begin{parts}
\item $T$ is bounded below.
\item $T$ is injective and has closed range.
\item $T$ is a topological isomorphism onto its image.
\end{parts}
\end{prb}

\begin{prb}[Bounded belowness in Hilbert spaces]
Let $T\in B(H,K)$ for Hilbert spaces $H$ and $K$.
The following statements are equivalent:
\begin{parts}
\item $T$ is bounded below.
\item $T$ is left invertible.
\item $T^*$ is right invertible.
\item $T^*T$ is invertible.
\end{parts}
\end{prb}

\begin{prb}[Injectivity and surjectivity of adjoint]
Let $T\in B(X,Y)$ for Banach spaces $X$ and $Y$.
\begin{parts}
\item $T^*$ is injective if and only if $T$ has dense range.
\item $T^*$ is surjective if and only if $T$ is bounded below.
\end{parts}
\end{prb}

\begin{prb}[Normal operators]
For $T\in B(H)$, we have an obvious fact $(\im T)^\perp=\ker T^*$.
Suppose $T$ is normal.
\begin{parts}
\item $\ker T=\ker T^*$.
\item $T$ is bounded below if and only if $T$ is invertible.
\item If $T$ is surjective, then $T$ is invertible.
\end{parts}
\end{prb}

\begin{prb}[Invariant and Reducing subsapces]
Let $K$ be a closed subspace of $H$.
\begin{parts}
\item $K$ is reducing for $T$ if and only if $K$ is invariant for $T$ and $T^*$.
\item $K$ is reducing for $T$ if and only if $TP=PT$, where $P$ is the orthogonal projection on $K$.
\end{parts}
\end{prb}
% self adjoint operators
% invariant but not reducing for unitary operators
% eigenspaces
% matrix representation









\chapter{Compact operators}

$K(X,Y)$ is closed in $B(X,Y)$.
$K(X)$ is an ideal of $B(X)$.
adjoint is $K(X,Y)\to K(Y^*,X^*)$.
integral operators are compact.
riesz operator, quasi-nilpotent operator.

\section{Finite-rank operators}
\section{Fredholm operators}

\begin{prb}
A bounded linear operator $T:X\to Y$ between Banach spaces is called a \emph{Fredholm} operator if its kernel is finite dimensional and its range is finite codimensional.
\begin{parts}
\item A Fredholm operator $T$ has closed range.
\end{parts}
\end{prb}
\begin{pf}
(a)
Let $C$ be a finite dimensional subsapce of $Y$ such that $\im T\oplus C=Y$.
Let $\tilde T:X/\ker T\to Y$ be the induced operator of $T$.
Define $S:(X/\ker T)\oplus C\to Y$ such that $S(x+\ker T,c):=\tilde T(x+\ker T)+c$.
Then, $S$ is an topological isomorhpism between Banach spaces by the open mapping theorem, so $S(X/\ker T\oplus\{0\})=\im\tilde T=\im T$ is closed.
\end{pf}

\begin{prb}[Atkinson's theorem]
An operator $T\in B(X,Y)$ is Fredholm if and only if there is $S\in B(Y,X)$ such that $TS-I$ and $ST-I$ is finite rank.
\end{prb}

\begin{prb}[Fredholm index]
locally constant, in particular, continuous.
composition makes the addition of indices.
\end{prb}

\section{Nuclear operators}
tensor products





\section*{Exercises}

\begin{prb}[Completely continuous operators]
On reflexive spaces, completely continuous operators are same with compact operators.
\end{prb}


\begin{prb}[Dunford-Pettis property]
A Banach space $X$ is said to have the \emph{Dunford-Pettis property} if all weakly compact operators $T:X\to Y$ to any Banach space $Y$ is completely continuous.
\begin{parts}
\item $X$ has the Dunford-Pettis property if and only if for every sequences $x_n\in X$ and $x^*_n\in X^*$ that converge to $x$ and $x^*$ weakly we have $x^*_n(x_n)\to x^*(x)$.
\item $C(\Omega)$ for a compact Hausdorff space $\Omega$ has the Dunford-Pettis property.
\item $L^1(\Omega)$ for a probability space $\Omega$ has the Dunford-Pettis property.
\item Infinite dimensional reflexive Banach space does not have the Dunfor-Pettis property.
\end{parts}
\end{prb}




\section*{Problems}
\begin{enumerate}
\item If $T\in B(L^2([0,1]))$ is a compact operator, then for any $\e>0$ there is a constant $C_\e>0$ such that
\[\|Tf\|_{L^2}\le\e\|f\|_{L^2}+C_\e\|f\|_{L^1}.\]
\end{enumerate}

\begin{pf}
1. Suppose there is $\e>0$ such that we have sequence $f_n\in L^2$ satisfying $\|f_n\|_2=1$ and
\[\|Tf_n\|_2>\e+n\|f_n\|_1.\]
By the compactness of $T$, there is a subsequence $Tf_{n_k}$ converges to $g\ne0$ in $L^2$.
Then, $\|f_{n_k}\|_1\to0$ implies $f_{n_k}\to0$ weakly in $L^2$, hence also for $Tf_{n_k}$.
It means $g=0$, which contradicts to the assumption.
\end{pf}







\part{Spectral theory}


\chapter{Normal operators}
\section{Spectral theorem for compact normal operators}
There is an orthonormal basis $E\subset H$ such that
\[T=\sum_{e\in E}\lambda_e|e\>\<e|.\]

\section{Spectral theorem for bounded normal operators}

\begin{prb}[Spectral measure]
Let $(\Omega,\cM)$ be a measurable space and $H$ a Hilbert space.
A \emph{projection valued measure} on $\Omega$ for $H$ is a map $E:\cM\to B(H)$ such that
\begin{enumerate}[(i)]
\item $E(A)$ is an orthogonal projection with $E(\varnothing)=0$,\item the set function $E_{\xi,\eta}:\cM\to\C:A\mapsto\<E(A)\xi,\eta\>$ is a complex measure on $\Omega$ for each $\xi,\eta\in H$.
\end{enumerate}
Let $\Omega$ be a locally compact Hausdorff space.
A \emph{spectral measure} is a projection valued measure $E$ on the Borel measurable space $\Omega$ such that $E_{\xi,\eta}$ is regular.
\begin{parts}
\item The condition (ii) is equivalent to the countable additivity: $E(\bigsqcup_{i=1}^\infty A_i)=\sum_{i=1}^\infty E(A_i)$ in the strong operator topology of $B(H)$ for $\{A_i\}_{i=1}^\infty\subset\cM$.
\item $E(A\cap B)=E(A)E(B)$ for $A,B\in\cM$.
\end{parts}
\end{prb}

\begin{prb}
Let $T\in B(H)$ be a normal operator.
Then, there exists a spectral measure $E$ on $\sigma(T)$ for $H$ such that
\[T=\int_{\sigma(T)}\lambda\,dE(\lambda).\]
This spectral measure $E$ is also called the \emph{resolution of the identity}.
\end{prb}



\section{Operator topologies}


\begin{prb}[Compact left multiplications and SOT]
Let $T_n$ be a sequence of bounded linear operators on a Hilbert space that converges in SOT.
For compact $K$, $T_n K$ converges in norm, but $KT_n$ generally does not unless $T$ is self-adjoint.
\end{prb}

\begin{prb}
Let $f$ be a linear functional on $B(H)$ for a Hilbert space $H$.
Then, TFAE:
\begin{parts}
\item $f$ is $\wot$-continuous,
\item $f$ is $\sot$-continuous,
\item $f(T)=\sum_{i=1}^n\<Tx_i,y_i\>$ for some $x_i,y_i$.
\end{parts}
\end{prb}
\begin{pf}
(2)$\impl$(3) is the only nontrivial implication.
By the definition of $\sot$, there exists $v\in \cH^n$ such that
\[|f(T)|\le\|T^{\oplus n}v\|.\]
The functional $f:\cA\to\C$ factors through $\cH^n$ such that
\[\cA\to{v}\cH^n\to\C.\]
\end{pf}






\chapter{Unbounded operators}



% point spectrum, approximate point spectrum
Kato-Rellich theorem




\chapter{Toeplitz operators}








\part{Operator algebras}
\chapter{Banach algebras}

\section{Spectra}

\begin{prb}[Banach algebras]

\end{prb}


\begin{prb}[Inverses in Banach algebras]
Let $\cA$ be a unital Banach algebra.
\begin{parts}
\item If $\|a\|<1$, then $1-a$ is invertible. So $\cA^\times$ is open.
\item $\cA^\times\to\cA^\times:a\mapsto a^{-1}$ is differentiable.
\end{parts}
\end{prb}



\begin{prb}[Spectrum and resolvent]
Let $a$ be an element of a unital Banach algebra $\cA$.
The \emph{spectrum} of $a$ in $\cA$ is defined to be the set
\[\sigma_\cA(a):=\{\lambda\in\C:\lambda-a\text{ is not invertible in }\cA\},\]
and the \emph{resolvent} of $a$ in $\cA$ is defined to be its complement $\rho_\cA(a):=\C\setminus\sigma_\cA(a)$.
We can now define the \emph{resolvent map} $R:\rho_\cA(a)\to\cA$ such that
\[R(\lambda)=R(\lambda;a):=(\lambda-a)^{-1}.\]
If $\cA$ is clear in its context, we omit it to just write $\sigma(a)$ and $\rho(a)$.
\begin{parts}
\item $\sigma(a)$ is compact.
\item $\sigma(a)$ is non-empty.
\item If $\cA$ is a division ring, then $\cA\cong\C$. This result is called the \emph{Gelfand-Mazur theorem}.
\end{parts}
\end{prb}
\begin{pf}
(b)
Suppose the spectrum $\sigma(a)=\varnothing$ so that $(\lambda-a)^{-1}$ exists for every $\lambda\in\C$.
Note that $a\ne0$.
Since the resolvent map $R:\C\to\cA$ is continuous and we have for $|\lambda|>2\|a\|$ that
\[\|(\lambda-a)^{-1}\|=\|\lambda^{-1}(1-\lambda^{-1}a)^{-1}\|
=\Bigl\|\lambda^{-1}\sum_{k=0}^\infty(\lambda^{-1}a)^k\Bigr\|
<(2\|a\|)^{-1}\sum_{k=0}^\infty2^{-k}=\|a\|^{-1},\]
the function $R$ is bounded.
Also, for every $l\in\cA^*$ we have that the function $\C\to\C:\lambda\mapsto\<R(\lambda),l\>$ is holomorphic since $a\mapsto a^{-1}$ is differentiable.
Therefore, by the Liouville theorem, the bounded entire function $\lambda\mapsto\<R(\lambda),l\>$ is constant for all $l\in\cA^*$.
Because $\cA^*$ separates points of $\cA$, the function $R$ is constant, which implies $a\in\C$ and contradicts to $\sigma(a)=\varnothing$.

(c)
\end{pf}

\begin{prb}[Spectral radius]
Let $a$ be an element of a unital Banach algebra $\cA$.
The \emph{spectral radius} of $a$ in $\cA$ is defined to be
\[r(a):=\sup_{\lambda\in\sigma(a)}|\lambda|.\]
\begin{parts}
\item $r(a)\le\inf_{n\ge1}\|a^n\|^{\frac1n}$ for all $a\in\cA$.
\item $\limsup_{n\to\infty}\|a^n\|^{\frac1n}\le r(a)$ for all $a\in\cA$.
\end{parts}
\end{prb}
\begin{pf}
(a)
Since $(\lambda-a)^{-1}=\lambda^{-1}(1-\lambda^{-1}a)^{-1}$ exists if $|\lambda|>\|a\|$, we have $r(a)\le\|a\|$ for all $a\in\cA$.
For every $\lambda\in\sigma(a)$ and every integer $n\ge1$ we have
\[|\lambda|^n=|\lambda^n|\le r(a^n)\le\|a^n\|,\]
and it proves $r(a)\le\inf_{n\ge1}\|a^n\|^{\frac1n}$.

(b)
On the domain $\{\lambda\in\C:|\lambda|>r(a)\}$, on which $R(\lambda)$ is well-defined, we have a holomorphic function $\lambda\mapsto\<R(\lambda),l\>$ for each $l\in\cA^*$.
By comparing to the same function but on a smaller domain $\{\lambda\in\C:|\lambda|>\|a\|\}$, we can determine the coefficients of the Laurent series of $\<R(\lambda),l\>$ at infinity as
\[\<R(\lambda),l\>=\Bigl\<\lambda^{-1}\sum_{k=0}^\infty(\lambda^{-1}a)^k,l\Bigr\>=\sum_{k=0}^\infty\<a^k,l\>\lambda^{-k-1}\]
for each $l\in\cA^*$.

It implies for each $\lambda\in\C$ with $|\lambda|>r(a)$ that the sequence $(a^k\lambda^{-k-1})_k$ in $\cA$ is weakly bounded, hence is normly bounded by the uniform boundedness principle.
Let $\|a^n\|\le C_\lambda|\lambda^{n+1}|$ for all $n\ge1$.
Then,
\[\limsup_{n\to\infty}\|a^n\|^{\frac1n}\le\limsup_{n\to\infty}C_\lambda^{\frac1n}|\lambda^{n+1}|^{\frac1n}=|\lambda|\]
for all $\lambda$ with $|\lambda|>r(a)$, so we are done.
\end{pf}

\begin{prb}[Spectrum in closed subalgebras]
For fixed element, smaller the ambient algebra, less ``holes'' in the spectrum.
Let $\cB\subset\cA$ be a closed subalgebra containing $1_\cA$.
Note that $\cB$ may be unital even for $1_\cA\notin\cB$.
\begin{parts}
\item $\cB^\times$ is clopen in $\cA^\times\cap\cB$.
\end{parts}
\end{prb}





\section{Ideals}
\begin{prb}[Ideals]
\begin{parts}
\item If $I$ is a left ideal, then $\cA/I$ is a left $\cA$-module.
\end{parts}
\end{prb}

\begin{prb}[Modular left ideals]
A left ideal $I$ is called \emph{modular} if there is $e\in\cA$ such that $a-ae\in I$ for all $a\in\cA$.
The element $e$ is called a \emph{right modular unit} for $I$.
\begin{parts}
\item $I$ is modular if and only if $\cA/I$ is unital(?).
\item A proper modular left ideal is contained in a maximal left ideal.
\item $I$ is a maximal modular left ideal if and only if $I$ is a modular maximal left ideal.
\item There is a non-modular maximal ideal in the disk algebra.
\end{parts}
\end{prb}

\begin{prb}[Closed ideals]
\begin{parts}
\item closure of proper left ideal is proper left.
\item maximal modular left ideal is closed.
\end{parts}
\end{prb}


\begin{prb}[Unitization]
Let $\cA$ be an algebra.
Recall that we always assume algebras are associative.
Consider an embedding $\cA\to B(\cA):a\mapsto L_a$, where $L_a(b)=ab$.
Define
\[\tilde\cA:=\{\,L_a+\lambda\id_{B(\cA)}:a\in\cA,\lambda\in\C\,\}.\]
Note that this construction is available even for unital $\cA$.
\begin{parts}
\item If $\cA$ is normed, then $\tilde\cA$ is a normed algebra such that there is an isometric embedding $\cA\to\tilde\cA$.
\item If $\cA$ is Banach, then $\tilde\cA$ is a Banach algebra.
\item $\cA\oplus\C$ is topologically isomorphic to $\tilde\cA$ as normed spaces.
\end{parts}
\end{prb}
\begin{pf}
(a)
The space of bounded operators $B(\cA)$ is a normd algebra.
Then, $\tilde\cA$ is a normed $*$-algebra with induced norm
\[\|L_a+\lambda\id_{B(\cA)}\|=\sup_{b\in\cA}\frac{\|ab+\lambda b\|}{\|b\|}\]
Then, $\cA$ is a normed $*$-subalgebra of $\tilde\cA$ because the norm and involution of $\cA$ agree with $\tilde\cA$.

(b)
Suppose $(x_n,\lambda_n)$ is Cauchy in $\tilde\cA$.
Since $\cA$ is complete so that it is closed in $\tilde\cA$, we can induce a norm on the quotient $\tilde\cA/\cA$ so that the canonical projection is (uniformly) continuous so that $\lambda_n$ is Cauchy.
Also, the inequality $\|x\|\le\|(x,\lambda)\|+|\lambda|$ shows that $x_n$ is Cauchy in $\cA$.

Since a finite dimensional normed space is always Banach and $\cA$ is Banach, $\lambda_n$ and $x_n$ converge.
Finally, the inequality $\|(x,\lambda)\|\le\|x\|+|\lambda|$ implies that $(x_n,\lambda_n)$ converges.

(c)
Check the topology on $\cA\oplus\C$ in detail...
\end{pf}



unitization, homomorphisms, category(direct sum, product, etc.)

$B(\C^n)$ is simple, but $B(X)$ is not simple.

% approximate identity, norm of left multiplication


\section{Holomorphic functional calculus}

\begin{prb}
Let $a$ be an element of a unital Banach algebra $\cA$.
Let $f$ be a holomorphic function on a neighborhood $U$ of $\sigma(a)$.
Let $C$ be a positively oriented smooth simple closed curve in $U$ enclosing $\sigma(a)$.
Define $f(a)\in\cA^{**}$ as the Dunford integral
\[\<f(a),l\>:=\int_Cf(\lambda)\<R(\lambda),l\>\,d\lambda\]
for all $l\in\cA^*$.

Let $\Hol(\sigma(a))$ be the space of all holomorphic functions on a neighborhood of $\sigma(a)$ endowed with the topology of compact convergence.
Note that $\Hol(\sigma(a))$ is not Banach.
We define the \emph{holomorphic functional calculus} by
\[\Hol(\sigma(a))\to\cA:f\mapsto f(a).\]
It is also called the Riesz or the Riesz-Dunford functional calculus.
\begin{parts}
\item $f(a)\in\cA$, i.e. $f(a)$ is given by the Pettis integral.
\item $f(a)$ is independent of the choice of $C$.
\item The functional calculus is an injective algebra homomorphism.
\item The functional calculus is continuous.
\item power series, 1 to 1, $\lambda$ to $a$.
\end{parts}
\end{prb}

spectral mapping




\section{Gelfand theory}

Banach algebra of single generator
semisimplicity and symmetricity

\begin{prb}[Spectrum of a Banach algebra]
Let $\cA$ be a commutative Banach algbera.
A \emph{character} of $\cA$ is a non-zero algebra homomorphism $\f:\cA\to\C$.
Denote by $\sigma(\cA)$ the set of all characters of $\cA$.
We will show that all characters are bounded.
Then, endow with the weak$^*$ topology on $\sigma(\cA)$ from the inclusion $\sigma(\cA)\subset\cA^*$.
We call this space as the \emph{spectrum} of $\cA$.
Let $\f\in\sigma(\cA)$.
\begin{parts}
\item $\|\f\|=1$.
\item If $\cA$ is unital, then $\sigma(\cA)$ is compact and Hausdorff.
\item Even if $\cA$ is non-unital, $\sigma(\cA)$ is locally compact and Hausdorff.
\end{parts}
\end{prb}


\begin{prb}[Gelfand transform]
Let $\cA$ be a commutative Banach algebra.
\[\Gamma:\cA\to C_0(\sigma(\cA)).\]
\begin{parts}
\item $\Gamma(\cA)$ separates points.
\item $\Gamma$ has closed range if
\item $\Gamma$ is injective if
\item $\Gamma$ is isometric if $r(a)=\|a\|$ for all $a\in\cA$.
\end{parts}
\end{prb}





\section*{Exercises}
\begin{prb}[Basic properties of spectrum]
Let $\cA$ be a unital algebra.
\begin{parts}
\item $\sigma(ab)\setminus\{0\}=\sigma(ba)\setminus\{0\}$.
\item If $\sigma(a)$ is non-empty, then $\sigma(p(a))=p(\sigma(a))$.
\end{parts}
\end{prb}
\begin{pf}
(a)
Intuitively, the inverse of $1-ab$ is $c=1+ab+abab+\cdots$.
Then, $1+bca=1+ba+baba+\cdots$ is the inverse of $1-ba$.
\end{pf}

$C_b(\Omega)$ $\ell^\infty(S)$ $L^\infty(\Omega)$ $B_b(\Omega)$ $A(\D)$
$B(X)$

\begin{prb}
In $C(\R)$, the modular ideals correspond to compact sets.
\end{prb}

\begin{prb}[Disk algebra]
\begin{parts}
\item Every continuous homomorphism is an evaluation.
\end{parts}
\end{prb}

\begin{prb}[Polynomial convexity]
(See Conway)
\end{prb}

\begin{prb}[Inclusion relation on spectra]
\begin{parts}
\item $\sigma(a+b)\subset\sigma(a)+\sigma(b)$ and $\sigma(ab)\subset\sigma(a)\sigma(b)$ for unital cases.
\item $\sigma(a^{-1})=\sigma(a)^{-1}$ for unital cases.
\item $r(a)^n=r(a^n)$.
\end{parts}
\end{prb}

\begin{prb}[Spectral radius function]
\begin{parts}
\item upper semi-continuous
\end{parts}
\end{prb}

\begin{prb}[Vector-valued complex function theory]
Let $\Omega$ be an open subset of $\C$ and $X$ a Banach space.
For a vector-valued function $f:\Omega\to X$, we say $f$ is \emph{differentiable} if the limit
\[\lim_{\lambda\to\lambda_0}(\lambda-\lambda_0)^{-1}(f(\lambda)-f(\lambda_0))\]
exists in $X$ for every $\lambda\in\Omega$, and \emph{weakly differentiable} if the limit
\[\lim_{\lambda\to\lambda_0}(\lambda-\lambda_0)^{-1}\<f(\lambda)-f(\lambda_0),x^*\>\]
exists in $\C$ for each $x^*\in X^*$ and every $\lambda\in\Omega$.
Then, the followings are all equivalent.
\begin{parts}
\item $f$ is differentiable.
\item $f$ is weakly differentiable.
\item For each $\lambda_0\in\Omega$, there is a sequence $(x_k)_{k=0}^\infty$ such that we have the power series expansion
\[f(\lambda)=\sum_{k=0}^\infty(\lambda-\lambda_0)^kx_k,\]
where the series on the right hand side converges absolutely and uniformly on any closed ball in $\Omega$ centered at $\lambda_0$.
\end{parts}
\end{prb}

\begin{prb}[Exponential of an operator]
\end{prb}







\chapter{C$^*$-algebras}

\section{C$^*$ identity}
% normal elements, real/imaginary part
\begin{prb}[Involutive Banach algebras]
Banach $*$-algebra: $\|a^*\|=\|a\|$.
\end{prb}


\begin{prb}[C$^*$ identity]
% history
A normed $*$-algebra $\cA$ is called a \emph{C$^*$-algebra} if
\begin{parts}
\item $\cA$ is Banach,
\item $\cA$ satisfies the C$^*$-identity: $\|x^*x\|=\|x\|^2$.
\end{parts}
\end{prb}


\begin{prb}[Unitization of C$^*$-algebras]
\[(L_a+\lambda\id_{B(\cA)})^*=L_{a^*}+\bar\lambda\id_{B(\cA)}.\]
\end{prb}
\begin{pf}
The C$^*$-identity easily follows from the following inequality:
\begin{align*}
\|(x,\lambda)\|^2&=\sup_{\|y\|=1}\|xy+\lambda y\|^2\\
&=\sup_{\|y\|=1}\|(xy+\lambda y)^*(xy+\lambda y)\|\\
&=\sup_{\|y\|=1}\|y^*((x^*x+\lambda x^*+\bar\lambda x)y+|\lambda|^2y)\|\\
&\le\sup_{\|y\|=1}\|(x^*x+\lambda x^*+\bar\lambda x)y+|\lambda|^2y\|\\
&=\|(x,\lambda)^*(x,\lambda)\|.\qedhere
\end{align*}
\end{pf}




\begin{prb}[$*$-homomorphisms]
\begin{parts}
\item determined by self-adjoint elements
\item norm-decreasing
\item
\end{parts}
\end{prb}


\section{Continuous functional calculus}

\begin{prb}[Gelfand-Naimark representation for C$^*$-algebras]
For a commutative unital C$^*$-algebra $\cA$, consider the Gelfand transform $\Gamma:\cA\to C(\sigma(\cA))$.
\begin{parts}
\item $\Gamma$ is a $*$-homomorphism.
\item $\Gamma$ is an isometry.
\item $\Gamma$ is a $*$-isomorphism.
\end{parts}
\end{prb}
\begin{pf}
(a)

(b)
Note that we have
\[\|\Gamma a\|=\sup_{\f\in\sigma(\cA)}|\Gamma a(\f)|=\sup_{\f\in\sigma(\cA)}|\f(a)|=r(a)\]
for all $a\in\cA$.
If we assume $a$ is self-adjoint, then since $\|a\|^2=\|a^*a\|=\|a^2\|$, the spectral radius coincides with the norm by the Beurling formula for spectral radius in Banach algebras:
\[\|\Gamma a\|=r(a)=\lim_{n\to\infty}\|a^{2^n}\|^{1/2^n}=\|a\|.\]
Hence we have for all $a\in\cA$ that
\[\|a\|^2=\|a^*a\|=\|\Gamma(a^*a)\|=\|(\Gamma a)^*\Gamma a\|=\|\Gamma a\|^2.\]

(c)
By the part (a) and (b), the image $\Gamma(\cA)$ is a closed unital $*$-subalgebra of $C(\sigma(\cA))$, and it separates points by definition.
Then, $\Gamma(\cA)$ is dense in $C(\sigma(\cA))$ by the Stone-Weierstrass theorem, which implies $\Gamma(\cA)=C(\sigma(\cA)$.
\end{pf}



\begin{prb}[Finitely generated C$^*$-algebras]
joint spectrum.
\end{prb}


\begin{prb}[Continuous functional calculus]
Let $\cA$ be a C$^*$-algebra, and $a\in\cA$ a normal element.
Then, we have an isometric $*$-homomorphism
\[C(\sigma(a))\to\cA\]
defined by the inverse of the Gelfand transform, which we call the \emph{continuous functional calculus}.
\begin{parts}
\item $\id\mapsto a$.
\item $(f+g)(a)=f(a)+g(a)$ and $(fg)(a)$.
\item $(f\circ g)(a)=f(g(a))$.
\end{parts}
\end{prb}



\begin{prb}[Normal elements]
Let $a$ be an element of a unital C$^*$-algebra $\cA$.
We say $a$ is \emph{normal}, \emph{unitary}, and \emph{self-adjoint} if $a^*a=aa^*$, $a^*a=aa^*=e$, and $a^*=a$ respectively.
For normality and self-adjointness, the definitions can be extended to non-unital C$^*$-algebras.
\begin{parts}
\item If $a$ is normal, then $a$ is unitary if and only if $\sigma(a)\subset\T$.
\item If $a$ is normal, then $a$ is self-adjoint if and only if $\sigma(a)\subset\R$.
\end{parts}
\end{prb}
\begin{pf}
(a)

(b)
We may assume $\cA$ is unital.
By the holomorphic functional calculus, we have
\[e^{ia}=\sum_{n=1}^\infty\frac{(ia)^n}{n!}\in\cA,\]
and the inverse of $e^{ia}$ is $e^{-ia}$.
Since the involution $^*:\cA\to\cA$ is continuous, we can check $e^{ia}$ is unitary by
\[(e^{ia})^*=\sum_{n=1}^\infty\frac{(-ia)^n}{n!}=e^{-ia}.\]
For every $\f\in\sigma(\cA)$, then by the part (a) the equality
\[e^{-\Im\f(a)}=|e^{i\f(a)}|=|\f(e^{ia})|=1\]
proves $\f(a)\in\R$, hence $\sigma(a)\subset\R$.
\end{pf}



\section{Positivitiy in C$^*$-algebras}


\begin{prb}[Positive elements]
Let $a,b$ be elements of a C$^*$-algebra $\cA$.
We say $a$ is \emph{positive} and write $a\ge0$ if it is normal and $\sigma(a)\subset\R_{\ge0}$.
If we define a relation $a\le b$ as $b-a\ge0$, then we can see that it is a partial order on $\cA$.
\begin{parts}
\item $a\ge0$ if and only if $\|\lambda-a\|\le\lambda$ for some $\lambda\ge\|a\|$.
\item If $a\ge0$ and $\sigma(b)\subset\R_{\ge0}$, then $\sigma(a+b)\subset\R_{\ge0}$.
\item If $a^*a\le0$, then $a=0$.
\item $a\ge0$ if and only if $a=b^*b$ for some $b\in\cA$.
\end{parts}
\end{prb}
\begin{pf}

\end{pf}

\begin{prb}[Absolute value of an operator]
\end{prb}


\begin{prb}[Operator monotonicity]
\begin{parts}
\item If $0\le a\le b$, then $a^{-1}\ge b^{-1}$.
\item If $a\le b$, then $cac^*\le cbc^*$.
\end{parts}
\end{prb}


\begin{prb}[Positive linear functionals]
\end{prb}


\begin{prb}[Injective $*$-homomorphism]
% https://math.stackexchange.com/questions/434706/sufficient-condition-for-a-homomorphism-between-c-algebras-being-isometric/435105#435105
\end{prb}


\begin{prb}[Approximate identity]
separable?

\end{prb}

\begin{prb}[Hereditary C$^*$-algebras]
\end{prb}



\section{Representations of C$^*$-algebras}


\begin{prb}[Representation of C$^*$-algebras]
A \emph{representation} of a C$^*$-algebra is a $*$-homomorphism $\pi:\cA\to B(H)$ for a Hilbert space $H$.
\end{prb}

\begin{prb}[Non-degenerate representations]
Let $\pi:\cA\to B(H)$ be a representation of a C$^*$-algebra $\cA$.
We say $\pi$ is \emph{non-degenerate} if $\pi(\cA)H$ is dense in $H$.
\begin{parts}
\item $\pi$ is non-degenerate.
\item For each $\xi\in H$ there is $a\in\cA$ such that $\pi(a)\xi\ne0$.
\item $\pi(e_\alpha)\to\id_H$ strongly for every approximate identity $e_\alpha$ of $\cA$.
\end{parts}
\end{prb}

\begin{prb}[Cyclic representations]
Let $\pi:\cA\to B(H)$ be a representation of a C$^*$-algebra $\cA$.
\begin{parts}
\item
\end{parts}
\end{prb}

\begin{prb}[Irreducible representations]
Let $\pi:\cA\to B(H)$ be a representation of a C$^*$-algebra $\cA$.
We say $\pi$ is irreducible if there is no proper closed subspace $K\subset H$ such that $\pi(a)K\subset K$.
\begin{parts}
\item $\pi$ is irreducible.
\item $\pi(\cA)'=\C\id_H$.
\item $\pi(\cA)$ is strongly dense in $B(H)$.
\item Every non-zero vector is cyclic.
\end{parts}
\end{prb}

\begin{prb}[Gelfand-Naimark-Segal representation]
Let $\cA$ be a C$^*$-algebra, and $\rho$ be a state on $\cA$.
The \emph{left kernel} of $\rho$ is defined to be
\[L_\rho:=\{a\in\cA:\rho(a^*a)=0\}.\]
\begin{parts}
\item $L_\rho$ is a left ideal of $\cA$.
\item $\<a+L,b+L\>:=\rho(b^*a)$ is an inner product on $\cA/L_\rho$.
\item There is a unique representation $\pi_\rho:\cA\to B(H_\rho)$ such that $\pi_\rho(a)(b+L):=ab+L$ for $a,b\in\cA$.
\item $\pi_\rho:\cA\to B(H_\rho)$ is a cyclic representation.
\end{parts}
\end{prb}




\begin{prb}[Kadison transitivity theorem]
\end{prb}

\begin{prb}[Left ideals]
\end{prb}

\begin{prb}[Primitive ideals]
\end{prb}

\begin{prb}[Hull-kernel topology]
\end{prb}



\section*{Exercises}


\begin{prb}
Let $\cB$ be a hereditary C$^*$-subalgebra of a C$^*$-algebra $\cA$.
Let $a\in\cA^+$.
If for any $\e>0$ there is $b\in\cB^+$ such that $a-\e\le b$, then $a\in B^+$.
\end{prb}
\begin{pf}
To catch the idea, suppose $\cA$ is abelian.
We want to approximate $a$ by the elements of $\cB$ in norm.
To do this, for each $\e>0$, we want to construct $b'\in\cB^+$ such that $a-\e\le b'\le a+\e$ using $b$.
Taking $b'=\min\{a,b\}$ is impossible in non-abelian case, but we can put $b'=\frac a{b+\e}b$.
For a simpler proof, $b'=(\frac{\sqrt{ab}}{\sqrt b+\sqrt\e})^2$ is a better choice.

Define
\[b':=\frac{\sqrt b}{\sqrt b+\sqrt\e}a\frac{\sqrt b}{\sqrt b+\sqrt\e}.\]
Then,
\[\|\sqrt a-\sqrt a\frac{\sqrt b}{\sqrt b+\sqrt\e}\|^2=\|\frac{\sqrt\e}{\sqrt b+\sqrt\e}a\frac{\sqrt\e}{\sqrt b+\sqrt\e}\|\le\e\]
implies
\[\lim_{\e\to0}b'=\lim_{\e\to0}\frac{\sqrt b}{\sqrt b+\sqrt\e}\sqrt a\cdot\sqrt a\frac{\sqrt b}{\sqrt b+\sqrt\e}=\sqrt a\cdot\sqrt a=a.\]
\end{pf}

\begin{prb}[Operator monotone square]
Let $\cA$ be a C$^*$-algebra in which the square function is operator monotone, that is, $0\le a\le b$ implies $a^2\le b^2$ for any positive elements $a$ and $b$ in $\cA$.
We are going to show that $\cA$ is necessarily commutative.
Let $a$ and $b$ denote arbitrary positive elements of $\cA$.
\begin{parts}
\item
Show that $ab+ba\ge0$.
\item
Let $ab=c+id$ where $c$ and $d$ are self adjoints.
Show that $d^2\le c^2$.
\item
Suppose $\lambda>0$ satisfies $\lambda d^2\le c^2$.
Show that $c^2d^2+d^2c^2-2\lambda d^4\ge0$.
\item
Show that $\lambda(cd+dc)^2\le(c^2-d^2)^2$.
\item
Show that $\sqrt{\lambda^2+2\lambda-1}\cdot d^2\le c^2$ and deduce $d=0$.
\item
Extend the result for general exponent: $\cA$ is commitative if $f(x)=x^\beta$ is operator monotone for $\beta>1$.
\end{parts}
\end{prb}


\begin{prb}[States on unitization]
Let $\cA$ and $\tilde\cA=\cA\oplus\C$ be a C$^*$-algebra and its unitization respectively.
Let $\tilde\rho=\rho\oplus\lambda$ be a bounded linear functional on $\tilde\cA$, where $\rho\in\cA^*$ and $\lambda\in\C^*=\C$.
\begin{parts}
\item $\tilde\rho$ is positive if and only if $\lambda\ge0$ and $0\le\rho\le\lambda$.
\item $\tilde\rho$ is a state if and only if $\lambda=1$ and $\rho$ is positive with $\|\rho\|\le1$.
\item $\tilde\rho$ is a pure state if and only if $\lambda=1$ and $\rho$ is either a pure state or zero.
\end{parts}
\end{prb}


\begin{prb}[Representations of $C_0(\Omega)$]
Let $\cA=C_0(\Omega)$ and $\mu$ be a state on $\cA$, a regular Borel probability measure on $\Omega$.
\begin{parts}
\item The left kernel of $\mu$ is $L_\mu=\{\,f\in\cA:f|_{\supp\mu}=0\,\}$.
\item The quotient is $\cA/L_\mu\cong C(\supp\mu)$ so that $H_\mu=L^2(\supp\mu,\mu)$.
\item The canonical cyclic vector is the unity function.
\end{parts}
\end{prb}

\begin{prb}[Representations of $K(H)$]
\end{prb}



\begin{prb}[Approximate eigenvectors]
\end{prb}


\section*{Problems}
\begin{enumerate}
\item* A C$^*$-algebra is commutative if and only if a function $f(x)=x(1+x)^{-1}$ is operator subadditive.
%L\"owner-Heinz inequality
\end{enumerate}


\chapter{Von Neumann algebras}

\section{Borel functional calculus}

\begin{prb}[Von Neumann algebras]
A C$^*$-algebra $\cA$ is called a \emph{von Neumann algebra} if there is a isometric $*$-homomorphism $\cA\to B(H)$ for a Hilbert space $H$ whose image is closed in the weak operator topology.
\end{prb}

\begin{prb}[Vigier theorem]
Increasing bounded net is convergent in strong operator topology.
The boundedness is important because we have to construct a bounded sesquilinear form using the monotone convergence in $\R$.
\end{prb}




\begin{prb}[Borel functional calculus]
Let $\cA$ be a von Neumann algebra.
\[B^\infty(\sigma(a))\to\cA.\]
\begin{parts}
\item The Borel functional calculus is in general not injective.
\item If we endow the topology of pointwise convergence on $B^\infty(\sigma(a))$ and the strong operator topology on $\cA$, then the Borel functional calculus is continuous.
\item not isometric, even if it is injective.
\item Every von Neumann algebra is the closed span of projections.
\end{parts}
\end{prb}
\begin{prb}

(b)
By the bounded convergence theorem.

(d)
This is because $\sigma(a)\subset\C$ is compact so that it is separable and metrizable; every bounded measurable function is a pointwise limit of simple functions.

\end{prb}



\section{Density theorems}

\begin{prb}[Bicommutant theorem]
Let $\cA$ be a non-degenerate C$^*$-subalgebra of $B(H)$.
\begin{parts}
\item $\cA'$ and $\cA''$ are weakly closed.
\item For $a\in\cA''$ and $\xi\in H$, there is a sequence $a_n\in\cA$ such that $a_n(\xi)\to a(\xi)$.
\item For $a\in\cA''$ and $\xi_1,\cdots,\xi_m\in H$, there is a sequence $a_n\in\cA$ such that $a_n(\xi_i)\to a(\xi_i)$ for all $i$.
\item $\cA$ is von Neumann algebra if and only if $\cA=\cA''$.
\end{parts}
\end{prb}
\begin{pf}
(b)
Let $K:=\bar{\cA\xi}$ be the cyclic subspace of $\xi$ in $H$ and $p$ its orthogonal projection.
We claim $a\xi\in K$.
For every $b\in\cA$, we have $bK\subset K$ because the multiplication by $b$ is continuous on $H$, and $b^*K\subset K$ because $\cA$ is self-adjoint.
It means that $K$ reduces all $b\in\cA$, and then $bp=pb$ implies $ap=pa$, so $K$ also reduces $a$.
Therefore, $aK\subset K$ proves $a\xi=\lim_\alpha e_\alpha a\xi\in K$, where $e_\alpha$ is an approximate identity of $\cA$.

(e)
Since $\bar\cA^\wot$ is closed convex, $\bar\cA^\sot=\bar\cA^\wot$.
Also, $\cA''$ is weakly closed, $\bar\cA^\wot\subset\cA''$.
\end{pf}


\begin{prb}[Kaplansky density theorem]
\end{prb}



\section{Predual}



\section{Factors and traces}


Every trace of factor is faithful

\begin{prb}
Normal states is a state in which the monotone convergence theorem holds.
Precisely, a state $\rho$ is \emph{normal} if a monotone net $a_\alpha$ strongly converges to $a$ then $\rho(a_\alpha)\to\rho(a)$.
\end{prb}


\section*{Exercises}
\begin{prb}[Extremally disconnected space]
$\sigma(B^\infty(\Omega))$ is extremally disconnected.
\end{prb}

resolution of identity
normal operator theories: multiplicity, invariant subspaces
$L^\infty$ representation

\end{document}