\documentclass{../../large}
\usepackage{../../ikhanchoi}


\begin{document}
\title{Differential Equations}
\author{Ikhan Choi}
\maketitle
\tableofcontents

% notes by Igor Yanovsky
% Oxford physics lecture

\part{Linear ordinary differential equations}


\chapter{Initial value problems}
\section{Constant coefficient equations}
existence uniqueness system of equations
characteristic equations
complex roots
repeated roots


\section{Variable coefficient equations}
existence uniqueness
series solution
Frobenius method
Fuch's theorem




\chapter{Boundary value problems}
\section{Second order linear equations}
Helmholtz
Bessel
Legendre
Hermite
Laguerre

\section{Orthogonal polynomials}
$L^2$ space

\section{Sturm-Liouville theory}
Eigenvalue problems
boundary conditions

\section*{Exercises}
\begin{prb}[Rayleigh-Ritz principle]
\end{prb}


\chapter{Inhomogeneous problems}
\section{Method of undetermined coefficients}
\section{Variation of parameters}
\section{Laplace transform}

discontinuous data gluing


\section*{Exercises}

\begin{prb}[Damped oscillation]
\end{prb}






\part{Nonlinear ordinary differential equations}
\chapter{First order nonlinear equations}
\section{Local existence theorems}

\begin{prb}[Picard-Lindel\"of theorem]
Consider the following initial value problem:
\[x'(t)=f(t,x(t)),\qquad x(0)=x_0.\]
Construct an approximate solution $(x_n)_{n=0}^\infty$ defined inductively such that $x_0(t)\equiv x_0$ and
\[x_{n+1}'(t)=f(t,x_n(t)),\quad x_{n+1}(0)=x_0.\]
Suppose $f$ satisfies
\[|f(t,x)|\le \frac RT,\qquad|f(t,x)-f(t,y)|\lesssim|x-y|\]
on the cylinder $[0,T]\times\bar{B(x_0,R)}$.
\begin{parts}
\item $x_n$ is in $C^1([0,T],\bar{B(x_0,R)})$.
\item $x_n$ is Cauchy in $C^1([0,T],\bar{B(x_0,R)})$.
\item The equation has a unique solution in $C^1([0,T],\bar{B(x_0,R)})$.
\end{parts}
\end{prb}
\begin{pf}
(a)
It clearly follows from the explicit formula
\[x_{n+1}(t)=x_0+\int_0^tf(s,x_n(s))\,ds.\]

(b)
Since
\[|x_1(t)-x_0(t)|\le\int_0^t|f(s,x_0)|\,ds\le Mt\]
and
\begin{align*}
|x_{n+1}(t)-x_n(t)|
&\le\int_0^t|f(s,x_n(s))-f(s,x_{n-1}(s))|\,ds\\
&\le K\int_0^t|x_n(s)-x_{n-1}(s)|\,dx\\
&\le MK^n\int_0^t\frac{s^n}{n!}\,ds\\
&=MK^n\frac{t^{n+1}}{(n+1)!},
\end{align*}
we have the convergent series
\[\sum_{n=0}^\infty\|x_{n+1}-x_n\|_\infty\le TM\frac{e^{KT}-1}{KT}.\]
Also,
\[|x'_{n+1}(t)-x'_n(t)|\le|f(t,x_n(t))-f(t,x_{n-1}(t))|\le K|x_n(t)-x_{n-1}(t)|\le MK^{n+1}\frac{t^{n+1}}{(n+1)!}.\]

(c)
Limiting check.
\end{pf}

\begin{prb}[Cauchy-Peano theorem]
\end{prb}

\begin{prb}[Carath\'eodory existence theorem]
\end{prb}


\section{Implicit equations}
integrating factor, separable equations, exact equations


\section{Global existence}
\begin{prb}[Gronwall's inequality]

\end{prb}
\begin{prb}[A priori estimate]

\end{prb}



\chapter{Dynamical systems}
\section{Equillibrium and stability}
Bifurcations\\
Stability theory
Lyapunov, invariant set

\section{Autonomous systems}


\section{Hamiltonian systems}

\section{Planar systems}
periodic orbit


\begin{prb}[Poincar\'e-Bendixon]
\end{prb}

\section*{Exercises}
\begin{prb}[Undamped pendulum]
\[x''(t)+\sin x(t)=0\]
\end{prb}
\begin{prb}[Approximated pendulum]
\[x''(t)+x(t)-\frac16x(t)^3=\alpha\]
\end{prb}

\begin{prb}[Van der Pol oscillator]
\[x''(t)-\mu(1-x(t)^2)x'(t)+x(t)=0\]
\end{prb}

\begin{prb}[Lotka-Volterra model]
Also known as predator-prey equations.
\end{prb}


\chapter{Chaos}
Attractors







\part{Linear partial differential equations}


\chapter{Laplace's equation}
\section{Harmonic functions}
\begin{prb}[Mean value property]
\end{prb}
\begin{prb}[Maximum principle]
\end{prb}


\begin{prb}[Newtonian potential]
\end{prb}
\begin{prb}[Dirichlet problem for half space]
\end{prb}
\begin{prb}[Dirichlet problem for open ball]
\end{prb}

\section{Poisson equation}
% How can we introduce the Dirac delta function in this note?
\begin{prb}[Weak derivative]

\end{prb}
\begin{prb}[Dirac delta function]
Let $\Omega$ be an open subset of $\R^d$.
The \emph{Dirac delta function} is a linear functional $\delta:C_c^\infty(\Omega)\to\R$ defined by $\delta(\f):=\f(0)$.
We conventionally use the function-like notation $\delta(x)$ to denote $\f(0)$ by
\[\int\delta(x)\f(x)\,dx.\]


\end{prb}

\begin{prb}[Fundamental solution of the Laplace equation]
Let $d\ge2$.
The \emph{Fundamental solution of the Laplace equation} is a function $\Phi:\R^d\setminus\{0\}\to\R$ that solves the boundary value problem
\[\left\{\begin{alignedat}{2}
-\Delta\Phi(x)&=\delta(x) &\quad&\text{ in }\R^d,\\
\Phi(x)&\to0 &&\text{ as }|x|\to\infty.
\end{alignedat}\right.\]
\begin{parts}
\item The funcdamental solution is given by
\[\Phi(x):=\begin{cases}-\frac1{2\pi}\log|x|&\text{ if }d=2\\\frac1{(d-2)\omega_d}\frac1{|x|^{d-2}}&\text{ if }d\ge3\end{cases}.\]
In particular, $\Phi$ and $\nabla\Phi$ are locally integrable on $\R^d$ but $\nabla^2\Phi$ is not.
\item For $u\in C_0^2(\R^d)$,
\[u(x)=-\int\Phi(x-y)\Delta u(y)\,dy.\]
\end{parts}
\end{prb}
\begin{pf}
Note that $\nabla\Phi(y)\cdot\nabla u(x-y)$ is integrable in $y$.
Then,
\begin{align*}
-\int\Phi(y)\Delta u(x-y)\,dy
&=-\int\nabla\Phi(y)\cdot\nabla u(x-y)\,dy\\
&=-\lim_{\e\to\infty}\int_{|y|\ge\e}\nabla\Phi(y)\cdot\nabla u(x-y)\,dy\\
&=-\lim_{\e\to\infty}\int_{|y|=\e}\nabla\Phi(y)u(x-y)\cdot\nu\,dS.
\end{align*}
Since
\[\nabla\Phi(x)=-\frac1{\omega_d}\frac x{|x|^d},\quad\nu=\frac x{|x|},\]
we get
\[-\int\Phi(y)\Delta u(x-y)\,dy=\lim_{\e\to\infty}\frac1{\omega_d\e^{d-1}}\int_{|y|=\e}u(x-y)\,dS_y=u(x).\]

\end{pf}

\begin{prb}[Green's function of the Poisson equation]
Let $\Omega$ be a bounded open subset of $\R^d$ for $d\ge2$.
\emph{Green's function of the Poisson equation} is a function $G:\Omega^2\setminus\{(x,x)\in\Omega\}\to\R$ that solves the boundary value problem
\[\left\{\begin{alignedat}{2}
-\Delta_yG(x,y)&=\delta(x-y)\quad & \text{ in }&y\in\Omega\setminus\{x\},\\
G(x,y)&=0 & \text{ on }&y\in\partial\Omega.
\end{alignedat}\right.\]
for each $x\in\Omega$.

Define $\phi:\Omega^2\to\R$ to be a function that solves the boundary value problem
\[\left\{\begin{alignedat}{2}
-\Delta_y\phi(x,y)&=0 & \text{ in }&y\in\Omega,\\
\phi(x,y)&=\Phi(x-y)\quad & \text{ on }&y\in\partial\Omega.
\end{alignedat}\right.\]
for each $x\in\Omega$.
Assume for the domain $\Omega$ that there exists a unique $\phi$.
\begin{parts}
\item Green's function is given by
\[G(x,y)=\Phi(x-y)-\phi(x,y),\]
where $\Phi$ is the fundamental solution of the Laplace equation.
Physically, $y\mapsto-\phi(x,y)$ has a meaning of the electric potential generated by the induced surface charge of a grounded conductor provided a point charge is at $x$.
\item The \emph{Green representation formula} holds: for $u\in C^2(\Omega)\cap C(\bar\Omega)$,
\[u(x)=-\int_\Omega G(x,y)\Delta u(y)\,dy-\int_{\partial\Omega}u(y)\nabla_yG(x,y)\cdot\nu\,dS_y.\]
\end{parts}
\end{prb}

\begin{prb}[Existence and uniqueness of Poisson equation]
representation formulas describe the solution assuming 

\end{prb}

\section{Eigenvalue problems}





\chapter{Heat equation}
\section{Heat kernel}
Duhamel's principle
\section{Separation of variables}






\chapter{Wave equation}
\section{First order partial differential equations}
\section{Initial value problems}
d'Alambert\\
Kirchhoff\\
odd reflection

\section{Boundary value problems}
Dirichlet, Neumann, Mixed

\section{Dispersive equations}





\part{Nonlinear partial differential equations}



\chapter{Geometric PDEs}
gradient flow
curvature flow

\chapter{Fluid dynamics}
\section{Conservation laws}
\section{Euler and Burger equation}
\section{Non-linear waves}
Nonlinear diffusion?
\section{Navier-Stokes equation}


\end{document}