\documentclass{../../large}
\usepackage{../../ikhanchoi}


\begin{document}
\title{Measure Theory}
\author{Ikhan Choi}
\maketitle
\tableofcontents

\part{Measures}





\chapter{Measure spaces}


\section{Measurable spaces}

\begin{prb}[Measurable spaces]

\end{prb}



\section{Measure spaces}

\begin{prb}[Definition of measures]
Let $(\Omega,\cM)$ be a measurable space.
A \emph{measure} on $\cM$ is a set function $\mu:\cM\to[0,\infty]:\varnothing\mapsto0$ that is \emph{countably additive}: we have
\[\mu\Bigl(\bigsqcup_{i=1}^\infty E_i\Bigr)=\sum_{i=1}^\infty\mu(E_i)\]
for $(E_i)_{i=1}^\infty\subset\cM$.
Here the squared cup notation reads the disjoint union.
\end{prb}

\begin{prb}[Continuity of measures]
\end{prb}

\begin{prb}[Pushforward measures]
\end{prb}

\begin{prb}[Complete measures]
\end{prb}



\section{Carath\'eodory extension}

\begin{prb}[Outer measures]
Let $\Omega$ be a set.
An \emph{outer measure} on $\Omega$ is a set function $\mu^*:\cP(\Omega)\to[0,\infty]:\varnothing\mapsto0$ such that
\begin{enumerate}[(i)]
\item $\mu^*$ is \emph{monotone}: we have
\[S_1\subset S_2\Rightarrow\mu^*(S_1)\le\mu^*(S_2)\]
for $S_1,S_2\in\cP(\Omega)$,
\item $\mu^*$ is \emph{countably subadditive}: we have
\[\mu^*\Bigl(\bigcup_{i=1}^\infty S_i\Bigr)\le\sum_{i=1}^\infty\mu^*(S_i)\]
for $(S_i)_{i=1}^\infty\subset\cP(\Omega)$.
\end{enumerate}
Compairing the definition of measures, we can see the outer measures extend the domain to the power set, but loosen the countable additivity to monotone countable subadditivity.
\begin{parts}
\item A set function $\mu^*:\cP(\Omega)\to[0,\infty]:\varnothing\mapsto0$ is an outer measure if and only if $\mu^*$ is \emph{monotonically countably subadditive}:
\[S\subset\bigcup_{i=1}^\infty S_i\Rightarrow\mu^*(S)\le\sum_{i=1}^\infty\mu^*(S_i)\]
for $S\in\cP(\Omega)$ and $(S_i)_{i=1}^\infty\subset\cP(\Omega)$.
\item
For $\varnothing\in\cA\subset\cP(\Omega)$, let $\rho:\cA\to[0,\infty]:\varnothing\mapsto0$ be a set function.
We can associate an outer measure $\mu^*:\cP(\Omega)\to[0,\infty]$ by defining as
\[\mu^*(S):=\inf\left\{\,\sum_{i=1}^\infty\rho(B_i):S\subset\bigcup_{i=1}^\infty B_i,\ B_i\in\cA\,\right\},\]
where we use the convention $\inf\varnothing=\infty$.
\end{parts}
\end{prb}
\begin{pf}
\end{pf}


\begin{prb}[Carath\'eodory measurability]
Let $\mu^*$ be an outer measure on a set $\Omega$.
We want to construct a measure by restriction of $\mu^*$ on a properly defined $\sigma$-algebra.
A subset $E\subset\Omega$ is called \emph{Carath\'eodory measurable} relative to $\mu^*$ if
\[\mu^*(S)=\mu^*(S\cap E)+\mu^*(S\setminus E)\]
for every $S\in\cP(\Omega)$.
Let $\cM$ be the collection of all Carath\'eodory measurable subsets relative to $\mu^*$.
\begin{parts}
\item $\cM$ is an algebra and $\mu^*$ is finitely additive on $\cM$.
\item $\cM$ is a $\sigma$-algebra and $\mu^*$ is countably additive on $\cM$.
\item The measure $\mu:=\mu^*|_\cM:\cM\to[0,\infty]$ is complete.
\end{parts}
\end{prb}
\begin{pf}
\end{pf}


\begin{prb}[Carath\'eodory extension theorem]
The Carath\'eodory extension is a construction method for a measure extending a given set function $\rho$.
The idea is to restrict the outer measure $\mu^*$ associated to $\rho$ in order to obtain a measure $\mu$.
We want to find a sufficient condition for $\mu$ to be a measure on a $\sigma$-algebra containing $\cA$.

For $\varnothing\in\cA\subset\cP(\Omega)$, let $\rho:\cA\to[0,\infty]:\varnothing\mapsto0$ be a set function.
Let $\mu^*:\cP(\Omega)\to[0,\infty]$ be the associated outer measure of $\rho$, and $\mu:\cM\to[0,\infty]$ the measure defined by the restriction of $\mu^*$ on Carath\'eodory measurable subsets.
\begin{parts}
\item We have $\mu^*|_\cA=\rho$ if $\rho$ satisfies the monotone countable subadditivity:
\[A\subset\bigcup_{i=1}^\infty B_i\Rightarrow\rho(A)\le\sum_{i=1}^\infty\rho(B_i)\]
for $A\in\cA$ and $(B_i)_{i=1}^\infty\subset\cA$.
\item We have $\cA\subset\cM$ if $\rho$ satisfies the following property: for every $B,A\in\cA$, and for any $\e>0$, there are $\{C_j\}_{j=1}^\infty$ and $\{D_j\}_{j=1}^\infty\subset\cA$ such that
\[B\cap A\subset\bigcup_{j=1}^\infty C_j\quad\text{ and }\quad B\setminus A\subset\bigcup_{j=1}^\infty D_j,\]
and
\[\rho(B)+\e>\sum_{j=1}^\infty\rho(C_j)+\sum_{j=1}^\infty\rho(D_j).\]
\end{parts}
\end{prb}
\begin{pf}
(a)
Clearly $\mu^*(A)\le\rho(A)$ for $A\in\cA$.
We may assume $\mu^*(A)<\infty$.
For arbitrary $\e>0$ there is $\{B_i\}_{i=1}^\infty$ such that $A\subset\bigcup_{i=1}^\infty B_i$ and
\[\mu^*(A)+\e>\sum_{i=1}^\infty\rho(B_i)\ge\rho(A).\]
Limiting $\e\to0$, we get $\mu^*(A)\ge\rho(A)$.

(b)
Let $S\in\cP(\Omega)$ and $A\in\cA$.
It is enough to check the inequality $\mu^*(S)\ge\mu^*(S\cap A)+\mu^*(S\setminus A)$ for $S$ with $\mu^*(S)<\infty$, so we may assume there is a countable family $\{B_i\}_{i=1}^\infty\subset\cA$ such that $S\subset\bigcup_{i=1}^\infty B_i$.
Then, we have $B_i\cap A\subset\bigcup_{j=1}^\infty C_{i,j}$ and $B_i\setminus A\subset\bigcup_{j=1}^\infty D_{i,j}$ satisfying
\[\mu^*(S)+\e>\sum_{i=1}^\infty(\rho(B_i)+\frac\e{2^{i+1}})
>\sum_{i,j=1}^\infty\rho(C_{i,j})+\sum_{i,j=1}^\infty\rho(D_{i,j})\ge\mu^*(S\cap A)+\mu^*(S\setminus A).\]
Therefore, $A$ is Carath\'eodory measurable relative to $\mu^*$.
\end{pf}


\begin{prb}[Uniqueness of extension of measures]
The existence of the Carath\'eodory extension provides a uniqueness theorem for the extension of measures.
The important property here is \emph{$\sigma$-finiteness}: for $\varnothing\in\cA\subset\cP(\Omega)$, let $\rho:\cA\to[0,\infty]:\varnothing\mapsto0$ be a set function.
Then, we say $\rho$ is $\sigma$-finite if there is a countable cover $(B_i)_{i=1}^\infty\subset\cA$ of $\Omega$ such that $\rho(B_i)<\infty$ for each $i$.

Let $\mu^*$ be the outer measure associated to $\rho$.
Let $\cM$ be a $\sigma$-algebra such that the restriction $\mu^*|_\cM:\cM\to[0,\infty]$ is a measure, and $\mu:\cM\to[0,\infty]$ be any measure.
Suppose further that $\mu^*(A)=\rho(A)=\mu(A)$ for all $A\in\cA$.
Let $E\in\cM$.
\begin{parts}
\item $\mu(E)\le\mu^*(E)$.
\item If $E_1,E_2\in\cM$ satisfy $\mu(E_1)=\mu^*(E_1)$ and $\mu(E_2)=\mu^*(E_2)$, then $\mu(E_1\cup E_2)=\mu^*(E_1\cup E_2)$.
\item $\mu(E)=\mu^*(E)$ if $\mu^*(E)<\infty$.
\item If $\rho$ is $\sigma$-finite, then $\mu(E)=\mu^*(E)$ for $\mu^*(E)=\infty$.
\end{parts}
\end{prb}
\begin{pf}
(a)
If $\mu^*(E)=\infty$, then $\mu(E)\le\mu^*(E)$ trivially.
Suppose $\mu^*(E)<\infty$.
By the definition of the outer measure, there is $\{B_i\}_{i=1}^\infty\subset\cA$ such that $E\subset\bigcup_{i=1}^\infty B_i$.
Also, we have
\[\mu(E)\le\mu\Bigl(\bigcup_{i=1}^\infty B_i\Bigr)\le\sum_{i=1}^\infty\mu(B_i)=\sum_{i=1}^\infty\rho(B_i)\]
whenever $E\subset\bigcup_{i=1}^\infty B_i$, so $\mu(E)\le\mu^*(E)$.

(b)
In the light of the inclusion-exclusion principle,
\[\mu^*(E_1\cup E_2)=\mu^*(E_1)+\mu^*(E_2)-\mu^*(E_1\cap E_2)\le\mu(E_1)+\mu(E_2)-\mu(E_1\cap E_2)=\mu(E_1\cup E_2)\]
proves the identity we want.

(c)
Because $\mu^*(E)<\infty$, for any $\e>0$ we have a sequence $(B_i)_{i=1}^\infty\subset\cA$ such that $E\subset\bigcup_{i=1}^\infty B_i$ and
\[\mu^*(E)+\e>\sum_{i=1}^\infty\rho(B_i).\]
Applying the part (b) inductively, we have for every $n$ that
\[\mu\Bigl(\bigcup_{i=1}^nB_i\Bigr)=\mu^*\Bigl(\bigcup_{i=1}^nB_i\Bigr),\]
and by limiting $n\to\infty$ the continuity from below gives
\[\mu\Bigl(\bigcup_{i=1}^\infty B_i\Bigr)=\mu^*\Bigl(\bigcup_{i=1}^\infty B_i\Bigr).\]
Then, we have
\[\mu^*(E)\le\mu^*\Bigl(\bigcup_{i=1}^\infty B_i\Bigr)=\mu\Bigl(\bigcup_{i=1}^\infty B_i\Bigr)=\mu\Bigl(\bigcup_{i=1}^\infty B_i\setminus E\Bigr)+\mu(E)\]
and
\[\mu\Bigl(\bigcup_{i=1}^\infty B_i\setminus E\Bigr)\le\mu^*\Bigl(\bigcup_{i=1}^\infty B_i\setminus E\Bigr)=\mu^*\Bigl(\bigcup_{i=1}^\infty B_i\Bigr)-\mu^*(E)\le\sum_{i=1}^\infty\mu^*(B_i)-\mu^*(E)=\sum_{i=1}^\infty\rho(B_i)-\mu^*(E)<\e,\]
we get $\mu^*(E)<\mu(E)+\e$ and $\mu^*(E)\le\mu(E)$ by limiting $\e\to0$.

(d)
Let $(B_i)_{i=1}^\infty\subset\cA$ be such that $\rho(B_i)<\infty$ and $\Omega=\bigcup_{i=1}^\infty B_i$.
Define $E_1:=B_1$ and $E_n:=B_n\setminus\bigcup_{i=1}^{n-1}B_i$ for $n\ge2$.
Then, $(E_i)_{i=1}^\infty$ is a pairwise disjoint cover of $\Omega$ with
\[\mu^*(E\cap E_i)\le\mu^*(E_i)\le\mu^*(B_i)=\rho(B_i)<\infty\]
for each $i$, so we have by the part (c) that
\[\mu(E)=\sum_{i=1}^\infty\mu(E\cap E_i)=\sum_{i=1}^\infty\mu^*(E\cap E_i)=\mu^*(E).\qedhere\]
\end{pf}


\section*{Exercises}

\begin{prb}[Semi-rings and semi-algebras]
We will prove a simplified Carath\'eodory extension with respect to \emph{semi-rings} and \emph{semi-algebras}.
Let $\cA$ be a collection of subsets of a set $\Omega$ such that $\varnothing\in\cA$.
We say $\cA$ is a semi-ring if it is closed under finite intersection, and the complement is a finite union of elements of $\cA$.
We say $\cA$ is a semi-algebra

Let $\cA$ be a semi-ring of sets over $\Omega$.
Suppose a set function $\rho:\cA\to[0,\infty]:\varnothing\mapsto0$ satisfies
\begin{enumerate}[(i)]
\item $\rho$ is \emph{disjointly countably subadditive}: we have
\[\rho\Bigl(\bigsqcup_{i=1}^\infty A_i\Bigr)\le\sum_{i=1}^\infty\rho(A_i)\]
for $(A_i)_{i=1}^\infty\subset\cA$,
\item $\rho$ is \emph{finitely additive}: we have
\[\rho(A_1\sqcup A_2)=\rho(A_1)+\rho(A_2)\]
for $A_1,A_2\in\cA$.
\end{enumerate}
A set function satisfying the above conditions are occasionally called a \emph{pre-measure}.
\begin{parts}
\item
\item 
\end{parts}
\end{prb}

\begin{prb}[Monotone class lemma]
A collection $\cC\subset\cP(\Omega)$ is called a \emph{monotone class} if it is closed under countable increasing unions and countable decreasing intersections.

Let $H$ be a vector space closed under bounded monotone convergence.
If $\spn\{\1_A:A\in\cA\}\subset H$ then $B^\infty(\sigma(\cA))\subset H$.
\end{prb}





\chapter{Measures on the real line}

\begin{prb}[Distribution functions]
\end{prb}

\begin{prb}[Helly selection theorem]
\end{prb}

\begin{prb}[Non-Lebesgue measurable set]
\end{prb}


\section*{Exercises}

\begin{prb}[Steinhaus theorem]
Let $\lambda$ denote the Lebesgue measure on $\R$ and let $\E\subset\R$ be a Lebesgue measurable set with $\lambda(E)>0$.
\begin{parts}
\item For any $0<\alpha<1$, there is an interval $I=(a,b)$ such that $\lambda(E\cap I)>\alpha\lambda(I)$.
\item $E-E=\{x-y:x,y\in E\}$ contains an open interval containing zero.
\end{parts}
\begin{pf}
(a)
We may assum $\lambda(E)<\infty$.
Since $\lambda$ is outer measure and $\lambda(E)\ne0$, we have an open subset $U$ of $\R$ such that $\lambda(U)<\alpha^{-1}\lambda(E)$.
Because $U$ is a countable disjoint union of open intervals $U=\bigsqcup_{i=1}^\infty(a_i,b_i)$, we have
\[\sum_{i=1}^\infty\lambda((a_i,b_i))=\lambda(U)<\alpha^{-1}\lambda(E)=\alpha^{-1}\sum_{i=1}^n\lambda(E\cap(a_i,b_i)).\]
Therefore, there is $i$ such that $\alpha\lambda((a_i,b_i))<\lambda(E\cap(a_i,b_i))$.
\end{pf}
% convolution으로 푸는 방법: continuous approximation 이 레벨에선 무리인듯
\end{prb}




\section*{Problems}
\begin{enumerate}
\item* Every Lebesgue measurable set in $\R$ of positive measure contains an arbitrarily long arithmetic progression.
\end{enumerate}

















\chapter{Measurable functions}



\section{Simple functions}
\begin{prb}[Measurability of pointwise limits]

Conversely, every measurable extended real-valued function is a pointwise limit of simple functions.

\end{prb}
\begin{pf}
Let $f(x)=\lim_{n\to\infty}s_n(x)$.

\end{pf}


\section{Almost everywhere convergence}

\begin{prb}[Almost everywhere convergence]
Let $(\Omega,\mu)$ be a measure space and let $f_n:\Omega\to\bar\R$ and $f:\Omega\to\bar\R$ be measurable functions.
The set of convergence of the sequence $f_n$ is defined as the set
\[\{x\in\Omega:\lim_{n\to\infty}f_n(x)=f(x)\},\]
and the set of divergence is defined as its complement.
We say $f_n$ converges to $f$ \emph{alomst everywhere} with respect to $\mu$ if the set of divergence is a null set in $\mu$.
We simply write
\[f_n\to f\text{ a.e.}\]
if $f_n$ converges to $f$ almost everywhere, and we frequently omit the measure $\mu$ if it has no confusion.
\begin{parts}
\item If $\mu$ is complete and, if $f_n\to f$ a.e., then $f$ is measurable.
\end{parts}
\end{prb}

\begin{prb}[Borel-Cantelli lemma]
Let $(\Omega,\mu)$ be a measure space and let $f_n:\Omega\to\bar\R$ and $f:\Omega\to\bar\R$ be a sequence of measurable functions.
Note that the set of divergence is given by
\[\bigcup_{\e>0}\bigcap_{n=1}^\infty\bigcup_{i=n}^\infty\{x:|f_n(x)-f(x)|\ge\e\}.\]
Each measurable set of the form
\[\{x:|f_n(x)-f(x)|\ge\e\}\]
is sometimes called the \emph{tail event}, coined in probability theory.
\begin{parts}
\item $f_n\to f$ a.e. if and only if for each $\e>0$ we have
\[\mu(\{x:\limsup_{n\to\infty}|f_n(x)-f(x)|\ge\e\})=0.\]
\item $f_n\to f$ a.e. if and only if for each $\e>0$ we have
\[\mu(\limsup_{n\to\infty}\{x:|f_n(x)-f(x)|\ge\e\})=0.\]
\item $f_n\to f$ a.e. if for each $\e>0$ we have
\[\sum_{n=1}^\infty\mu(\{x:|f_n(x)-f(x)|\ge\e\})<\infty.\]
\end{parts}
\end{prb}
\begin{pf}
(b)
The set of divergence of the sequence $f_n$ is given by
\[\bigcup_{m=1}^\infty\bigcap_{n=1}^\infty\bigcup_{i=n}^\infty\{x:|f_i(x)-f(x)|\ge\tfrac1m\}=\bigcup_{m=1}^\infty\,\bigcap_{n=1}^\infty(\Omega\setminus E_n^m).\]

(c)
Since
\[\mu\Bigl(\bigcup_{i=1}^\infty\{x:|f_i(x)-f(x)|\ge\e\}\Bigr)\le\sum_{i=1}^\infty\mu(\{x:|f_i(x)-f(x)|\ge\e\})<\infty,\]
we have by the continuity from above that
\begin{align*}
\mu(\limsup_{n\to\infty}\{x:|f_n(x)-f(x)|\ge\e\})
&=\mu\Bigl(\bigcap_{n=1}^\infty\bigcup_{i=n}^\infty\{x:|f_i(x)-f(x)|\ge\e\}\Bigr)\\
&=\lim_{n\to\infty}\mu\Bigl(\bigcup_{i=n}^\infty\{x:|f_i(x)-f(x)|\ge\e\}\Bigr)\\
&\le\lim_{n\to\infty}\sum_{i=n}^\infty\mu(\{x:|f_i(x)-f(x)|\ge\e\})
=0.\qedhere
\end{align*}
\end{pf}

\begin{prb}[Convergence in measure]
Let $(\Omega,\mu)$ be a measure space and let $f_n:\Omega\to\bar\R$ be a sequence of measurable functions.
We say $f_n$ converges to a measurable function $f:\Omega\to\bar\R$ \emph{in measure} if for each $\e>0$ we have
\[\lim_{n\to\infty}\mu(\{x:|f_n(x)-f(x)|\ge\e\})=0.\]
\begin{parts}
\item If $f_n\to f$ in measure, then there is a subsequence $f_{n_k}$ such that $f_{n_k}\to f$ a.e.
\item If every subsequence $f_{n_k}$ of $f_n$ has a further subsequence $f_{n_{k_j}}$ such that $f_{n_{k_j}}\to f$ a.e., then $f_n\to f$ in measure.
\end{parts}
\end{prb}
\begin{pf}
(a)
Since for each positive integer $k$ we have $\mu(\{x:|f_n(x)-f(x)|\ge\frac1k\})\to0$ as $n\to\infty$, there exists $n_k$ such that
\[\mu(\{x:|f_{n_k}(x)-f(x)|\ge\tfrac1k\})<\frac1{2^k}.\]
By the Borel-Cantelli lemma, we get
\[\mu(\limsup_{k\to\infty}\{x:|f_{n_k}(x)-f(x)|\ge\tfrac1k\})=0.\]
Then, for each $\e>0$,
\begin{align*}
\limsup_{k\to\infty}\{x:|f_{n_k}(x)-f(x)|\ge\e\}
&=\bigcap_{k=\lceil\e^{-1}\rceil}^\infty\bigcup_{j=k}^\infty\{x:|f_{n_j}(x)-f(x)|\ge\e\}\\
&\subset\bigcap_{k=\lceil\e^{-1}\rceil}^\infty\bigcup_{j=k}^\infty\{x:|f_{n_j}(x)-f(x)|\ge\tfrac1k\}\\
&=\limsup_{k\to\infty}\{x:|f_{n_k}(x)-f(x)|\ge\tfrac1k\}
\end{align*}
implies the limit superior of the tail events is a null set, hence $f_{n_k}\to f$ a.e.

(b)
\end{pf}

\begin{prb}[Egorov theorem]
Egorov's theorem informally states that an almost everywhere convergent functional sequence is ``almost'' uniformly convergent.
Through this famous theorem, we introduce a convenient ``$\e/2^m$ argument'', occasionally used throughout measure theory to construct a measurable set having a special property.

Let $(\Omega,\mu)$ be a finite measure space and let $f_n:\Omega\to\bar\R$ be a sequence of measurable functions such that $f_n\to f$ a.e.
For each positive integer $m$, which indexes the tolerance $1/m$, consider an increasing sequence of measurable subsets
\[E_n^m:=\bigcap_{i=n}^\infty\{x:|f_i(x)-f(x)|<\tfrac1m\}.\]
\begin{parts}
\item $E_n^m$ converges to a full set for each $m$.
\item For every $\e>0$ there is a measurable $K\subset\Omega$ such that $\mu(\Omega\setminus K)<\e$ and for each $m$ there is finite $n$ satisfying $K\subset E_n^m$.
\item For every $\e>0$ there is a measurable $K\subset\Omega$ such that $\mu(\Omega\setminus K)<\e$ and $f_n\to f$ uniformly on $K$.
\end{parts}
\end{prb}
\begin{pf}
(a)
Recall that the a.e. convergence $f_n\to f$ means that for every fixed $m$ the intersection
\[\bigcap_{n=1}^\infty(\Omega\setminus E_n^m)=\limsup_n\{x:|f_n(x)-f(x)|\ge\tfrac1m\}\]
is a null set.
Since $\mu(\Omega)<\infty$, it is equivalent to $E_n^m$ converges to a full set for each $m$ by the continuity from above.

(b)
For each $m$, we can find $n_m$ such that
\[\mu(\Omega\setminus E_{n_m}^m)<\frac\e{2^m}.\]
If we define
\[K:=\bigcap_{m=1}^\infty E_{n_m}^m,\]
then it satisfies the second conclusion, and also have
\[\mu(\Omega\setminus K)=\mu\Bigl(\bigcup_{m=1}^\infty(\Omega\setminus E_{n_m}^m)\Bigr)\le\sum_{m=1}^\infty\mu(\Omega\setminus E_{n_m}^m)<\sum_{m=1}^\infty\frac\e{2^m}=\e.\]


(c)
Fix $m>0$.
Since $n\ge n_m$ implies $K\subset E_{n_m}^m\subset E_n^m$, we have
\[n\ge n_m\quad\Rightarrow\quad\sup_{x\in K}|f_n(x)-f(x)|<\frac1m.\qedhere\]
\end{pf}



\section*{Exercises}
\begin{prb}[Cauchy's functional equation]
Let $f:\R\to\R$ be a function.
Cauchy's functional equation refers to the equation $f(x+y)=f(x)+f(y)$, satisfied for all $x,y\in\R$.
Suppose $f$ satisfies the Cauchy functional equation.
We ask if $f$ is linear, that is $f(x)=ax$ for all $x\in\R$, where $a:=f(1)$.
\begin{parts}
\item $f(x)=ax$ for all $x\in\Q$, but there is a nonlinear solution of Cauchy's functional equation.
\item If $f$ is conitnuous at a point, then $f$ is linear.
\item If $f$ is Lebesgue measurable, then $f$ is linear.
\end{parts}
\end{prb}











\part{Lebesgue integral}


\chapter{Convergence theorems}
\section{Definition of Lebesgue integral}
\section{Convergence theorems}

% Stein: Egorov $\to$ BCT $\to$ Fatou $\to$ MCT $\to$ L1<M\\
% Stein: BCT + L1<M $\to$ DCT\\
% Folland: MCT $\to$ Fatou $\to$ DCT $\to$ BCT

\begin{prb}[Monotone convergence theorem]
\end{prb}




\section{Radon-Nikodym theorem}

An integrable function as a measure
$\sigma$-finite measures




\chapter{Product measures}
\section{Fubini-Tonelli theorem}
\section{Lebesgue measure on Euclidean spaces}


\chapter{Measures on metric spaces}
\section{Borel measures}

\section{Riesz-Markov-Kakutani representation theorem}

locally compact

\section{Hausdorff measures}









\part{Linear operators}



\chapter{Lebesgue spaces}
\section{$L^p$ spaces}

\begin{prb}[H\"older inequality]
\end{prb}
\begin{pf}
\[\int fg\le C^p\int\frac{|f|^p}p+\frac1{C^q}\int\frac{|g|^q}q\]
Take $C$ such that
\[C^p\int\frac{|f|^p}p=\frac1{C^q}\int\frac{|g|^q}q.\]
Then,
\[C^p\int\frac{|f|^p}p+\frac1{C^q}\int\frac{|g|^q}q=2p^{-\frac1p}q^{-\frac1q}\Bigl(\int|f|^p\Bigr)^{\frac1p}\Bigl(\int|g|^p\Bigr)^{\frac1q}.\]
Note that we can show that $1\le2p^{-\frac1p}q^{-\frac1q}\le2$ and the minimum is attained only if $p=q=2$, so this method does not provide the sharpest constant.
\end{pf}


\section{$L^1$ spaces}

\begin{prb}[Convolution?]
\end{prb}
\begin{prb}[Approximate identity?]
\end{prb}
\begin{prb}[Continuity of translation?]
\end{prb}

\section{$L^2$ spaces}

\section{$L^\infty$ spaces}








\chapter{Bounded linear operators}
\section{Continuity}
Schur test

\section{Density arguments}
extension of operators

\section{Interpolation}
weak Lp, marcinkiewicz



\begin{defn}
Let $f$ be a measurable function on a measure space $(X,\mu)$.
The \emph{distribution function} $\lambda_f:[0,\infty)\to [0,\infty)$ is defined as:
\[\lambda_f(\alpha):=\mu(\{x:|f(x)|>\alpha\}).\]
\end{defn}


Do not use $\mu(\{x:|f(x)|\ge\alpha\})$.
The strict inequality implies the \emph{lower semi-continuity} of $\lambda_f$.

\begin{parts}
\item For $p>0$, we have
\[\|f\|_{L^p}^p=p\int_0^\infty\left[\alpha\cdot\mu(|f|>\alpha)^\frac1p\right]^p\,\frac{d\alpha}\alpha.\]
\end{parts}
\begin{defn}
\[\|f\|_{L^{p,q}}^q:=p\int_0^\infty\left[\alpha\cdot\mu(|f|>\alpha)^\frac1p\right]^q\,\frac{d\alpha}\alpha.\]
Also,
\[\|f\|_{L^{p,\infty}}:=\sup_{0<\alpha<\infty}\left[\alpha\cdot\mu(|f|>\alpha)^\frac1p\right].\]
\end{defn}
\begin{thm}
For $p\ge1$ we have $\|f\|_{p,\infty}\le\|f\|_p$.
\end{thm}
\begin{pf}
By the Chebyshev inequality,
\[\sup_{0<\alpha<\infty}\left[\alpha^p\cdot\mu(|f|>\alpha)\right]\le\int_0^\infty p\alpha^{p-1}\cdot\mu(|f|>\alpha)\,d\alpha=\|f\|_{L^p}^p.\]

\end{pf}





\begin{prb}[Marcinkiewicz interpolation]
Let $X$ be a $\sigma$-finite measure space and $Y$ be a measure space.
Let
\[1<p_0<p<p_1<\infty.\]
If a sublinear operator $T\colon L^{p_0}(X)+L^{p_1}(X)\to M(Y)$ has two weak-type estimates
\[\|T\|_{L^{p_0}(X)\to L^{p_0,\infty}(Y)}<\infty\quad\text{and}\quad\|T\|_{L^{p_1}(X)\to L^{p_1,\infty}(Y)}<\infty,\]
then it has a strong-type estimate
\[\|T\|_{L^p(X)\to L^p(X)}<\infty.\]
\end{prb}
\begin{pf}
Let $f\in L^p(X)$ and denote $f_h=\chi_{|f|>\alpha}f$ and $f_l=\chi_{|f|\le\alpha}f$.
It is easy to show $f_h\in L^{p_0}$ and $f_l\in L^{p_1}$.
Then,
\begin{align*} % 고치던 중
\|Tf\|_{L^p(Y)}^p&\sim\int\alpha^p\cdot\mu(|Tf|>\alpha)\,\frac{d\alpha}\alpha\\
&\lesssim\int\alpha^p\cdot\mu(|T(f\cdot\1_{|f|>\alpha})|>\alpha)\,\frac{d\alpha}\alpha+\int\alpha^p\cdot\mu(|Tf_l|>\alpha)\,\frac{d\alpha}\alpha\\
&\le\int\alpha^p\cdot\frac1{\alpha^{p_0}}\|Tf_h\|_{L^{p_0,\infty}}^{p_0}\,\frac{d\alpha}\alpha+\int\alpha^p\cdot\frac1{\alpha^{q_1}}\|Tf_l\|_{L^{p_1,\infty}}^{p_1}\,\frac{d\alpha}\alpha\\
&\lesssim\int\alpha^{p-p_0}\|f_h\|_{p_0}^{p_0}\,\frac{d\alpha}\alpha+\int\alpha^{p-p_1}\|f_l\|_{p_1}^{p_1}\,\frac{d\alpha}\alpha\\
&\sim\|f\|_p^p.
\end{align*}
by (1) Fubini, (2) Sublinearlity, (3) Chebyshev, (4) Boundedness, (5) Fubini.
\end{pf}

\begin{prb}[Hadamard's three line lemma]
Let $f$ be a bounded holomorphic function on the vertical unit stripe $\{z:0<\Re z<1\}$.
Then, for $0<\theta<1$,
\[\|f\|_{L^\infty(\Re=\theta)}\le\|f\|_{L^\infty(\Re=0)}^{1-\theta}\|f\|_{L^\infty(\Re=1)}^\theta.\]
\end{prb}
\begin{pf}
Define
\[g(z):=\frac{f(z)}{\|f\|_{L^\infty(\Re=0)}^{1-z}\|f\|_{L^\infty(\Re=1)}^z},\qquad g_n(z)=g(z)e^\frac{z^2-1}n.\]
Then we have
\begin{enumerate}
\item $g_n\to g$ pointwisely as $n\to\infty$,
\item $g_n(z)\to0$ uniformly as $\Im z\to\infty$.
\end{enumerate}
The second one is because $g$ is bounded and for $z=x+yi$ we have
\[|g_n(z)|\lesssim|e^\frac{z^2-1}n|=e^{\Re\frac{z^2-1}n}=e^\frac{x^2-y^2-1}n\le e^\frac{-y^2}n.\]

By (1), it is enough to bound $g_n$ for each $n$.
Truncating the stripe, the outer region is controlled by (2) and the interior region is controlled by the maximum modulus principle.
\end{pf}



\begin{prb}[Riesz-Thorin interpolation]
Let $X,Y$ be $\sigma$-finite measure spaces.
Let
\[\frac1{p_\theta}=(1-\theta)\frac1{p_0}+\theta\frac1{p_1},\qquad\frac1{q_\theta}=(1-\theta)\frac1{q_0}+\theta\frac1{q_1}.\]
Then,
\[\|T\|_{p_\theta\to q_\theta}\le\|T\|_{p_0\to q_0}^{1-\theta}\|T\|_{p_1\to q_1}^\theta.\]
\end{prb}
\begin{pf}
Note that
\[\|T\|_{p_\theta\to q_\theta}=\sup_f\frac{\|Tf\|_{q_\theta}}{\|f\|_{p_\theta}}=\sup_{f,g}\frac{|\<Tf,g\>|}{\|f\|_{p_\theta}\|g\|_{q'_\theta}}.\]
Consider a holomorphic function
\[z\mapsto\<Tf_z,g_z\>=\int\bar{g_z(y)}Tf_z(y)\,dy,\]
where $f_z$ and $g_z$ are defined as
\[f_z=|f|^{\frac{p_\theta}{p_0}(1-z)+\frac{p_\theta}{p_1}z}\frac f{|f|}\]
so that we have $f_\theta=f$ and
\[\|f\|_{p_\theta}^{p_\theta}=\|f_z\|_{p_x}^{p_x}\]
for $\Re z=x$.

Then,
\[|\<Tf_z,g_z\>|\le\|T\|_{p_0\to q_0}\|f_z\|_{p_0}\|g_z\|_{q'_0}=\|T\|_{p_0\to q_0}\|f\|_{p_\theta}^{p_\theta/p_0}\|g\|_{q'_\theta}^{q'_\theta/q'_0}\]
for $\Re z=0$, and
\[|\<Tf_z,g_z\>|\le\|T\|_{p_1\to q_1}\|f_z\|_{p_1}\|g_z\|_{q'_1}=\|T\|_{p_1\to q_1}\|f\|_{p_\theta}^{p_\theta/p_1}\|g\|_{q'_\theta}^{q'_\theta/q'_1}\]
for $\Re z=1$.
By Hadamard's three line lemma, we have
\[|\<Tf_z,g_z\>|\le\|T\|_{p_0\to q_0}^{1-\theta}\|T\|_{p_1\to q_1}^\theta\|f\|_{p_\theta}\|g\|_{q'_\theta}\]
for $\Re z=\theta$.
Putting $z=\theta$ in the last inequality, we get the desired result.
\end{pf}




\chapter{Convergence of linear operators}
\section{Translation and multiplication operators}

\section{Convolution type operators}
approximation of identity
Fej\'er, Poisson, box?

\section{Computation of integral transforms}











\part{Fundamental theorem of calculus}

\chapter{Weak derivatives}

The space of weakly differentiable functions with respect to all variables $=W_\loc^{1,1}$.

\begin{prb}[Product rule for weakly differentiable functions]
We want to show that if $u$, $v$, and $uv$ are weakly differentiable with respect to $x_i$, then $\pd_{x_i}(uv)=\pd_{x_i}uv+u\pd_{x_i}v$.
\begin{parts}
\item If $u$ is weakly differentiable with respect to $x_i$ and $v\in C^1$, then $\pd_{x_i}(uv)=\pd_{x_i}uv+u\pd_{x_i}v$.
\end{parts}
\end{prb}


\begin{prb}[Interchange of differentiation and integration]
Let $f:\Omega_x\times\Omega_y\to\R$ be such that $\pd_{x_i}f$ is well-defined. Suppose $f$ and $\pd_{x_i}f$ are locally integrable in $x$ and integrable $y$.

Then,
\[\pd_{x_i}\int f(x,y)\,dy=\int\pd_{x_i}f(x,y)\,dy.\]
\end{prb}


Do not think the Schwarz theorem as the condition for partial differentiation to commute.
We should understand like this: if $F$ is $C^2$ then the \emph{classical} partial differentiation commute, and if $F$ is not $C^2$ then the \emph{classical} partial derivatives of order two or more are \emph{meaningless} because it is not compatible with the generalized concept of differentiation.



\chapter{Absolutely continuity}

\begin{parts}
\item $f$ is $\Lip_\loc$ iff $f'$ is $L_\loc^\infty$
\item $f$ is $\textrm{AC}_\loc$ iff $f'$ is $L_\loc^1$
\end{parts}
\begin{parts}
\item $f$ is $\Lip$ iff $f'$ is $L^\infty$
\item $f$ is $\textrm{AC}$ iff $f'$ is $L^1$
\item $f$ is $\textrm{BV}$ iff $f'$ is a finite regular Borel measure
\end{parts}

\section{Absolute continuous measures}

\section{Absolute continuous functions}

\section{Functions of bounded variation}




\chapter{Lebesgue differentiation theorem}

\section{Hardy-Littlewood maximal function}

Let $T_m$ be a net of linear operators.
It seems to have two possible definitions of maximal functions:
\[T^*f:=\sup_m|T_mf|\]
and
\[T^*f:=\sup_{m,\ \e:|\e(x)|=1}|T_m(\e f)|.\]

\begin{prb}[Hardy-Littlewood maximal function]
The Hardy-Littlewood maximal function is just the maximal function defined with the approximate identity by the box kernel.
\end{prb}

\begin{prb}[Weak type estimate]
\[\|Mf\|_{1,\infty}\le 3^d\|f\|_{L^1(\Omega)}.\]
\begin{parts}
\item Proof by covering lemma.
\end{parts}
\end{prb}
\begin{pf}
(a)
By the inner regularity of $\mu$, there is a compact subset $K$ of $\{|Mf|>\lambda\}$ such that
\[\mu(K)>\mu(\{|Mf|>\lambda\})-\e.\]
For every $x\in K$, since $|Mf(x)|>\lambda$, we can choose an open ball $B_x$ such that
\[\frac1{\mu(B_x)}\int_{B_x}|f|>\lambda\]
if and only if
\[\mu(B_x)<\frac1\lambda\int_{B_x}|f|.\]
With these balls, extract a finite open cover $\{B_i\}_i$ of $K$.
Since the diameter of elements in this cover is clearly bounded, so the Vitali covering lemma can be applied to obtain a disjoint subcollection $\{B_k\}_k$ such that
\[K\subset\bigcup_iBi\subset\bigcup_k3B_k.\]
Therefore,
\[\mu(K)
\le\sum_k3^d\mu(B_k)
\le\frac{3^d}\lambda\sum_k\int_{B_k}|f|
\le\frac{3^d}\lambda\|f\|_1.\]
The disjointness is important in the last inequality which shows the constant does not depend on the number of $B_k$'s.
\end{pf}



\begin{prb}[Radially bounded approximate identity]
If an approximate identity $K_n$ is radially bounded, then its maximal function is dominated by the Hardy-Littlewood maximal function:
\[\sup_n|K_n*f(x)|\lesssim Mf(x)\]
for every $n$ and $x$, hence has a weak type estimate.
\end{prb}


\begin{prb}[Almost everywhere convergence of operators]
Suppose is $T_m$ is a sequence of linear operators such that the maximal function $T^*f$ is dominated by $Mf$.
If $f\in L^1(\Omega)$ and $T_mg\to g$ pointwise for $g\in C(\Omega)$, then $T_mf\to f$ a.e.
\end{prb}
\begin{pf}
Take $\e>0$ and $g\in C(\Omega)$ such that $\|f-g\|_{L^1(\Omega)}<\e$.
Since $T_mg(x)\to g(x)$ pointwise, we have
\begin{align*}
&\mu(\{x:\limsup_m|T_mf(x)-f(x)|>\lambda\})\\
&\qquad\le\mu(\{x:\limsup_m|T_mf(x)-T_mg(x)|>\tfrac\lambda2\})
+\mu(\{x:|g(x)-f(x)|>\tfrac\lambda2\})\\
&\qquad\le\mu(\{x:M(f-g)(x)>\tfrac\lambda2\})+\frac2\lambda\|f-g\|_{L^1(\Omega)}\\
&\qquad\lesssim\frac1\lambda\e
\end{align*}
for every $\lambda>0$.
Limiting $\e\to0$, we get
\[\mu(\{x:\limsup_m|T_mf(x)-f(x)|>\lambda\})=0\]
for every $\lambda>0$, hence the continuity from below implies
\[\mu(\{x:\limsup_m|T_mf(x)-f(x)|>0\})=0.\]
\end{pf}


\begin{defn}
\[f^*(x):=\lim_{r\to0+}\frac1{\mu(B)}\int_B|f(y)-f(x)|\,dy.\]
\end{defn}
\begin{thm}[Lebesgue differentiation]
$f^*=0$ a.e.
\end{thm}
\begin{pf}
Note that $f^*\le Mf+|f|$ implies
\[\|f^*\|_{1,\infty}\le\|Mf\|_{1,\infty}+\|f\|_{1,\infty}\lesssim\|f\|_1.\]
Note that $g^*=0$ for $g\in C_c$.
Approximate using $f^*=(f-g)^*$.
\end{pf}

\section*{Exercises}
\begin{prb}[Doubling measure]

\end{prb}


\end{document}


