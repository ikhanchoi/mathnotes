\documentclass{../../large}
\usepackage{../../ikhanchoi}


\begin{document}
\title{Foundations of Calculus}
\author{Ikhan Choi}
\maketitle



\chapter*{Preface}
the main objectives
the audience
the structure of the book
how to use this book
acknowledgements
references

% hypergeometric, bessel, gamma, zeta

\tableofcontents


\part{Sequences}


\chapter{Metric spaces}
% 문제의식: 노름과 메트릭 추상화, 동치는 다루지 않음
\section{Metric spaces}
\begin{prb}[Definition of metric spaces]
Let $X$ be a set.
A \emph{metric} is a function $d:X\times X\to\R_{\ge0}$ such that
\begin{parts}[(i)]
\item $d(x,y)=0$ if and only if $x=y$, \hfill(nondegeneracy)
\item $d(x,y)=d(y,x)$ for all $x,y\in X$, \hfill(symmetry)
\item $d(x,z)\le d(x,y)+d(y,z)$ for all $x,y,z\in X$. \hfill(triangle inequality)
\end{parts}
A pair $(X,d)$ of a set $X$ and a metric on $X$ is called a \emph{metric space}.
We often write it simply $X$.
\begin{parts}
\item
A normed space $X$ is a metric space with a metric defined by $d(x,y):=\|x-y\|$.
\item
A subset of a metric space is a metric space with a metric given by restriction.
\end{parts}
\end{prb}

\begin{prb}[System of open balls]
A metric is often misunderstood as something that measures a distance between two points and belongs to the study of geoemtry.
The main function of a metric is to make a system of small balls, sets of points whose distance from specified center points is less than fixed numbers.
The balls centered at each point provide a concrete images of ``system of neighborhoods at a point'' in a more intuitive sense.
In this viewpoint, a metric can be considered as a structure that lets someone accept the notion of neighborhoods more friendly.

Note that taking either $\e$ or $\delta$ in analysis really means taking a ball of the very radius.
Investigation of the distribution of open balls centered at a point is now an important problem.

Let $X$ be a metric space.
A set of the form 
\[\{y\in X:d(x,y)<\e\}\]
for $x\in X$ and $\e>0$ is called an \emph{open ball centered at $x$ with radius $\e$} and denoted by $B(x,\e)$ or $B_\e(x)$.
\end{prb}

\begin{prb}[Convergence and continuity in metric spaces]
Let $\{x_n\}_n$ be a sequence of points on a metric space $(X,d)$.
We say that a point $x$ is a \emph{limit} of the sequence or the sequence \emph{converges to $x$} if for arbitrarily small ball $B(x,\e)$, we can find $n_0$ such that $x_n\in B(x,\e)$ for all $n>n_0$.
If it is satisfied, then we write
\[\lim_{n\to\infty}x_n=x,\]
or simply $x_n\to x$ as $n\to\infty$.
We say a sequence is \emph{convergent} if it converges to a point.
If it does not converge to any points, then we say the sequence \emph{diverges}.

A function $f:X\to Y$ between metric spaces is called \emph{continuous at $x\in X$} if for any ball $B(f(x),\e)\subset Y$, there is a ball $B(x,\delta)\subset X$ such that $f(B(x,\delta))\subset B(f(x),\e)$.
The function $f$ is called \emph{continuous} if it is continuous at every point on $X$.
\begin{parts}
\item A sequence $x_n$ in a metric space $X$ converges to $x\in X$ if and only if $d(x_n,x)$ converges to zero.
\item 
Let $f:X\to Y$ be a function between two metric spaces.
If there is a constant $C$ such that $d(x,y)\le Cd(f(x),f(y))$ for all $x$ and $y$ in $X$, then $f$ is continuous.
In this case, $f$ is particularly called \emph{Lipschitz continuous} with the \emph{Lipschitz constant} $C$.
\end{parts}
\end{prb}


\begin{prb}[Separable metric spaces]
separable iff second countable iff lindelof
\end{prb}

\section{Normed spaces}
banach space

\section{Open sets and closed sets}
convergence, limit point
\section{Compact sets}
Bolzano-Weierstrass
\section{Connected sets}



\section*{Exercises}
\section*{Problems}




\chapter{Real sequences}

\section{Monotone sequences}
preserving inequalities
limsup and liminf
monotone convergence


\section{Extended real numbers}
\begin{prb}[Operations in the extended real numbers]
We can extend addition (except $\infty+(-\infty)$), subtraction, multiplication (except $\infty\times0$), division (except dividing by zero).
\end{prb}

\begin{prb}[Limits in the extended real numbers]
\end{prb}


\section{Asymptotic analysis}
sufficiently large
asymptotic expressions
growth and decay

Approximate sequences($\e/3$)

\begin{prb}[Change of limits]
\[|a_n-a|\le|a_n-b_{mn}|+|b_{mn}-b_m|+|b_m-a|\]
\[\lim_m\sup_n|a_n-b_{mn}|=0\]
\[\lim_n|b_{mn}-b_m|=0\]
\end{prb}

\[a_n=b_{mn}+c_{mn}\le b_{mn}+\e\]





\section*{Exercises}
\begin{prb}
\end{prb}
\begin{prb}[Newton method]
\end{prb}
\section*{Problems}
\begin{enumerate}
\item Every real sequence $(a_n)_{n=1}^\infty$ has a subsequence $(a_{n_k})_{k=1}^\infty$ such that $\lim_{k\to\infty}a_{n_k}=\limsup_{n\to\infty}a_n$.
\end{enumerate}





\chapter{Series}

\section{Absolute convergence}
\begin{prb}[Unconditional convergence]
\end{prb}


\section{Convergence tests}

comparison
limit comparison
cauchy condensation
integral....

ratio
root


\begin{prb}[Abel transform]
\[A_n(B_n-B_{n-1})+(A_n-A_{n-1})B_{n-1}=A_nB_n-A_{n-1}B_{n-1}\]
\[\sum_{m<k\le n}A_kb_k=A_nB_n-A_mB_m-\sum_{m<k\le n}a_kB_{k-1}.\]
\end{prb}

abel test
\begin{prb}[Dirichlet test]
\end{prb}


\begin{prb}[Mertens' theorem]
If $\sum_{k=0}^\infty a_k$ converges to $A$ absolutely and $\sum_{k=0}^\infty b_k$ converges to $B$, then their Cauchy product $\sum_{k=0}^\infty c_k$ with $c_k:=\sum_{l=0}^ka_lb_{k-l}$ converges to $AB$.
Let
\[A_n:=\sum_{k=0}^na_k,\ B_n:=\sum_{k=0}^nb_k,\quad\text{ and }\quad C_n:=\sum_{k=0}^nc_k.\]
\end{prb}
\begin{pf}
Write
\[|C_n-AB|\le|C_n-A_nB_n|+|A_nB_n-AB|.\]
Since the limit of the second term $|A_nB_n-AB|\to0$ is clear, we claim $|C_n-A_nB_n|\to0$.

Fix any $\e>0$.
Note that $|B_n|$ is bounded by some $M>0$.
Write for some $m$,
\begin{align*}
|C_n-A_nB_n|
&=|\sum_{k=0}^na_k(B_n-B_{n-k})|\\
&\le|\sum_{k=0}^ma_k(B_n-B_{n-k})|+|\sum_{k=m+1}^na_k(B_n-B_{n-k})|\\
&\le\sum_{k=0}^m|a_k||B_n-B_{n-k}|+\sum_{k=m+1}^n|a_k|\cdot2M.
\end{align*}
Since $\sum_ka_k$ converges absolutely, we can take $m$ such that
\[\sum_{k=m+1}^\infty|a_k|<\frac\e{2M}.\]
By taking limit $n\to\infty$, we have
\[\limsup_{n\to\infty}|C_n-A_nB_n|\le0+\e.\]
Since $\e>0$ is arbitrary, we have $\lim_n|C_n-A_nB_n|=0$.

\end{pf}




\section*{Exercises}
\begin{prb}[Ces\`aro mean]

\end{prb}
\begin{prb}[Recursive sine sequence]
Let $a_{n+1}=\sin a_n$ and $a_n=1$.
We can use $\sin x=x-\frac{x^3}6+O(x^5)$.
\[a_n=\sqrt3n^{-\frac12}-\frac{3\sqrt3}{20}n^{-\frac32}+o(n^{-\frac32}).\]
\end{prb}


\begin{prb}[Convergence rates of recursive sequences]
If $a_{n+1}=a_n-f(a_n)$, $f(0)=0$, $f(x)>0$ for $0<x<\e$, $f\in C^2$? then
\[f'(a_n)\sim\lim_{x\to0+}\frac{f'(x)^2}{f''(x)f(x)}\frac1n.\]
\end{prb}
\section*{Problems}
\begin{enumerate}
\item If $a_n\to0$, then $\frac1n\sum_{k=1}^na_k\to0$. (Ces\`aro mean)
\item If $a_n\ge0$ and $\sum a_n$ diverges, then $\sum\frac{a_n}{1+a_n}$ also diverges.
\item If $a_n\ge0$ and $\sum a_n<\infty$, then there are sequences $b_n\downarrow0$ and $\sum c_n<\infty$ such that $a_n=b_nc_n$. (Very special case of the Cohen factorization)
\end{enumerate}






\part{Functions}

\chapter{Continuity}
% 문제의식: 연속함수의 성질을 엄밀하게 기술할 수 있는 보조정리 증명, 연속함수 공간 개요
\section{Intermediate and extreme value theorems}

left and right limits
semicontinuous


\section{Various continuities}

Lipschitz
uniform
cauchy


\section*{Exercises}

\section*{Problems}
\begin{enumerate}
\item The set of local minima of a convex real function is connected.
\item Let $f:\R\to\R$ be continuous.
The equation $f(x)=c$ cannot have exactly two solutions for every constant $c\in\R$.
\item A continuous function that takes on no value more than twice takes on some value exactly once.
\item Let $f$ be a function that has the intermediate value property.
If the preimage of every singleton is closed, then $f$ is continuous.
\item If a continuous function $f:[0,\infty)\to\R$ has a limit at infinity, then it is uniformly continuous.
\item If $f:[0,1]^2\to\R$ is continuous, then $g:[0,1]\to\R$ defined by $g(x):=\max_{y\in[0,1]}f(x,y)$ is continuous.
\end{enumerate}







\chapter{Differentiation}
\section{Differentiability}
\begin{prb}[L'hopital's theorem]
\end{prb}

\section{Monotonicty and convexity}

\section{Taylor expansion}
\begin{prb}[Rolle's theorem]
Let $f:[a,b]\to\R$ be a function that is continuous on $[a,b]$ and differentiable on $(a,b)$.
\begin{parts}
\item If $f(a)=f(b)=0$, then there is $c\in(a,b)$ such that $f'(c)=0$.
\item Suppose $f$ is $(n+1)$-times differentiable. If $f(a)=f'(a)=\cdots=f^{(n)}(a)=0$ and $f(b)=0$, then there is $c\in(a,b)$ such that $f^{(n+1)}(c)=0$.
\end{parts}
\end{prb}
\begin{pf}
(a)
If $f\equiv0$, then it is clear.
If not, we may assume there is $x\in(a,b)$ such that $f(x)>0$ by multiplying $-1$.
Since $f$ is continuous, by the extreme value theorem, there is $c\in(a,b)$ such that $c$ attains the maximum of $f$.
Then, $f'(c)=0$.

(b)
By the induction, we have $c_n\in(a,b)$ such that $f^{(n)}(c)=0$.
By applying Rolle's theorem (the part (a)) for $f^{(n)}$, we have $c_{n+1}\in(a,c_n)$ such that $f^{(n+1)}(c_{n+1})=0$.
\end{pf}

\begin{prb}[Taylor theorem]
\end{prb}


\section{Smooth functions}

\section*{Exercises}
\begin{prb}[Variations on the mean value theorem]
Let $f$ be a differentiable function on the unit closed interval.
\begin{parts}
\item If $f(0)=0$ there is $c$ such that $cf'(c)=f(c)$. (Flett)
\item If $f(0)=0$ there is $c$ such that $cf(c)=(1-c)f'(c)$.
\end{parts}
\end{prb}
\begin{prb}[Dini derivatives]
\end{prb}
\begin{prb}[Darboux theorem]
\end{prb}

\section*{Problems}
\begin{enumerate}
\item If $\lim_{x\to\infty}f(x)=a$ and $\lim_{x\to\infty}f'(x)=b$, then $a=0$.
\item Let $f$ be a real $C^2$ function with $f(0)=0$ and $f''(0)\ne0$.
Defined a function $\xi$ such that $f(x)=xf'(\xi(x))$ with $|\xi|\le|x|$, we have $\xi'(0)=1/2$.
\item Let $f$ be a $C^2$ function such that $f(0)=f(1)=0$.
We have $\|f\|\le\frac18\|f''\|$.
\item A smooth function such that for each $x$ there is $n$ having the $n$th derivative vanish is a polynomial.
\item If a real $C^1$ function $f$ satisfies $f(x)\ne0$ for $x$ such that $f'(x)=0$, then in a bounded set there are only finite points at which $f$ vanishes.
\item Let a real function $f$ be differentiable.
For $a<a'<b<b'$ there exist $a<c<b$ and $a'<c'<b'$ such that $f(b)-f(a)=f'(c)(b-a)$ and $f(b')-f(a')=f'(c')(b'-a')$.
\item Let $f:[1,\infty)\to\R$ satisfy that $f(1)=1$ and $f'(x)=(x^2+f(x)^2)^{-1}$. Show that $\lim_{x\to\infty}f(x)$ exists in the open interval $(1,1+\frac\pi4)$.
\item If $f:(0,\infty)\to\R$ is $C^2$ and satisfies $f'(x)\le0<f(x)$ for all $x>0$, then the boundedness of $f''$ implies $f'(x)\to0$ as $x\to\infty$.
\item If a function $f:[0,1]\to\R$ that is continuous on $[0,1]$ and differentiable on $(0,1)$ satisfies $f(0)=0$ and $0\le f'(x)\le2f(x)$, then $f$ is identically zero.
\item For $C^2$ function $f:\R\to\R$ we have $\|f'\|^2\le4\|f\|\|f''\|$.
\item For a smooth function $f:\R\to\R$ such that $f'''(x)<0$, we have $\frac{f'(x)+f'(y)}2<\frac{f(x)-f(y)}{x-y}$ for all $x\ne y\in\R$.
\end{enumerate}






\chapter{Integration}

\section{Riemann integral}
tagged partition
\section{Henstock-Kurzweil intergral}
bounded compact support <-> lebesgue
\section{Improper integral}
\section{Fundamental theorem of calculus for continuous functions}

\section*{Exercises}

\section*{Problems}
\begin{enumerate}
\item Find the value of $\lim_{n\to\infty}\frac1n\left(\sum_{k=1}^n\frac1nf\left(\frac kn\right)-\int_0^1f(x)\,dx\right)$.
\item Find all $a>0$ and $b>0$ such that $\int_0^\infty x^{-b}|\tan x|^a\,dx$ converges.
\item* If $xf'(x)$ is bounded and $x^{-1}\int_0^xf\to L$ then $f(x)\to L$ as $x\to\infty$.
\item Show that for a continuous function $f:[0,1]\to\R$ we have $\int_0^1x^2f(x)\,dx=\frac13f(c)$ for some $c\in[0,1]$.
\end{enumerate}









\part{Functional sequences}

\chapter{Continuous functions}
\section{Uniform convergence}

\begin{prb}
Let $X$ be a compact metric space.
\begin{parts}
\item $C(X)$ is complete.
\end{parts}
\end{prb}
\begin{pf}
(a)
Suppose $f_m$ is a Cauchy sequence in $C(X)$.
Since $f_m$ is Cauchy pointwise, we can define the pointwise limit $f$.
We first claim that $f_m$ converges to $f$ uniformly.
Fix $\e>0$.
Write
\[|f_m(x)-f(x)|\le\|f_m-f_{m'}\|+|f_{m'}(x)-f(x)|.\]
Since $f_m$ is uniformly Cauchy, there is $m_0$ such that $m,m'>m_0$ implies
\[|f_m(x)-f(x)|<\e+|f_{m'}(x)-f(x)|.\]
Taking limit $m'\to\infty$, we have
\[|f_m(x)-f(x)|\le\e+0.\]
Taking the supremum over $x\in X$ and limit $m\to\infty$, we obtain
\[\lim_{m\to\infty}\|f_m-f\|\le\e.\]
Since $\e$ is arbitrary, we have the uniform limit $f_m\to f$.

Now we claim $f$ is continuous.
Let $x\in X$ and suppose $x_n$ converges to $x$.
Divide the error as
\[|f(x_n)-f(x)|\le|f(x_n)-f_m(x_n)|+|f_m(x_n)-f_m(x)|+|f_m(x)-f(x)|.\]
Using the uniform convergence, we can take sufficiently large $m$ such that $\|f_m-f\|<\e$, so we have
\[|f(x_n)-f(x)|<\e+|f_m(x_n)-f_m(x)|+\e.\]
Then, taking $\limsup_{n\to\infty}$ on the both-hand sides, we get
\[\limsup_{n\to\infty}|f(x_n)-f(x)|\le\e+0+\e=2\e.\]
Since $\e>0$ has been arbitrarily taken,
\[\lim_{n\to\infty}|f(x_n)-f(x)|=0.\]

(b)

\end{pf}


\section{}

\begin{prb}[Partition of unity]
\end{prb}
\begin{prb}[Urysohn lemma]
\end{prb}
\begin{prb}[Tietze extension]
\end{prb}



\section{Arzela-Ascoli theorem}

\section{Stone-Weierstrass theorem}

\begin{prb}[Bernstein polynomial]
We want to show $\R[x]$ is dense in $C([0,1],\R)$.
Let $f\in C([0,1],\R)$ and define \emph{Berstein polynomials} $B_n(f)\in\R[x]$ for each $n$ such that
\[B_n(f)(x):=\sum_{k=0}^nf\left(\frac kn\right)\binom nkx^k(1-x)^{n-k}.\]
\begin{parts}
\item $B_n(f)$ uniformly converges to $f$ on $[0,1]$.
\item There is a sequence $p_n\in\R[x]$ with $p_n(0)=0$ uniformly convergent to $x\mapsto|x|$ on $[-1,1]$.
\end{parts}
\end{prb}
\begin{pf}
(b)
Let
\[B_n(x):=\sum_{k=0}^n\left|1-\frac{2k}n\right|\binom nk(1-2x)^k(2x-1)^{n-k}.\]
Since $B_n(x)\to|x|$ uniformly on $[-1,1]$ and $B_n(0)\to0$, we have $B_n(x)-B_n(0)\to|x|$ uniformly on $[-1,1]$.
\end{pf}

\begin{prb}[Taylor series of square root]
We want to show the absolute value is approximated by polynomials in $C([-1,1],\R)$ in another way.
Let
\[f_n(x):=\sum_{k=0}^n a_k(x-1)^k\]
be the partial sum of the Taylor series of the square root function $\sqrt x$ at $x=1$.
\begin{parts}
\item By Abel's theorem, $f_n$ uniformly converges to $\sqrt x$ on $[0,1]$
\item There is a sequence $p_n\in\R[x]$ with $p_n(0)=0$ uniformly convergent to $x\mapsto|x|$ on $[-1,1]$.
\end{parts}
\end{prb}


\begin{prb}[Proof of Stone-Weierstrass theorem]
Let $X$ be a compact Hausdorff space and $S\subset C(X,\R)$.
We say that $S$ \emph{separates points} if for every distinct $x$ and $y$ in $X$ there is $f\in S$ such that $f(x)\ne f(y)$, and that $S$ \emph{vanishes nowhere} if for every $x$ in $X$ there is $f\in S$ such that $f(x)\ne0$.

Let $\cA=\bar{S\R[S]}$ be the real Banach subalgebra of $C(X,\R)$ generated by $S$.
\begin{parts}
\item $\cA$ is a lattice.
\item $\cA$ is dense in $C(X,\R)$.
\end{parts}
\end{prb}




Locally compact version,
complex version








\begin{prb}
Some examples
\begin{parts}
\item $z\R[z]$ is dense in $C([1,2],\R)$.
\item $\C[z]$ is dense in $C([0,1],\C)$.
\item $z\C[z,\bar z]$ is dense in $C(\T,\C)$.
\end{parts}
\end{prb}







\section*{Exercises}
\begin{prb}[Weierstrass' nowhere differentiable function]
\end{prb}
\section*{Problems}
\begin{enumerate}
\item* Show that a sequence of functions $f_n:[0,1]\to[0,1]$ that satisfies $|f(x)-f(y)|\le|x-y|$ whenever $|x-y|\ge\frac1n$ has a uniformly convergent subsequence.
\item Show that for a sequence of differentiable functions $f_n:\R\to\R$ that satisfies $|f_n'(x)|\le1$ for all $n\ge1$ and $x\in\R$ its pointwise limit is continuous if it exists.
\item Show that a sequence of $C^1$ functions $f_n:[0,1]\to\R$ such that $|f_n'(x)|\le x^{-\frac12}$ for $x\in(0,1]$ and $\int_0^1f_n(x)\,dx=0$ for all $n\ge1$ has a uniformly convergent subsequence.
\end{enumerate}

\chapter{Differentiable functions}
\section{Differentiable class}
completeness
\section{H\"older spaces}

\section{Analytic functions}

Power series
uniform convergence and absolute convergence, abel theorem?
differentiation
convergence of radius, complex domain
sum, product, composition, reciprocal?
closed under uniform convergence
identity theorem


\section*{Problems}

	
\begin{enumerate}
\item Show that if $f:(-1,1)\to\R$ is a smooth function such that $|f^{(n)}(x)|\le1$ for all $n\ge1$ uniformly then $f$ is analytic.
\end{enumerate}


\chapter{Integrable functions}
\section{}

\begin{prb}[Lebesgue criterion of Riemann integrability]
\end{prb}













\part{Multivariable Calculus}
\chapter{Fre\'chet derivatives}
% 편미분이 그냥 계산도구라는 것을 배운다
% 편미분의 교환도 사실은 항상 다 되는 걸... 배울 수 있나
% 접공간 개념을 배운다
\section{Tangent spaces}
\begin{prb}[Vector fields]

\end{prb}

\section{Inverse function theorem}






\chapter{Differential forms}
% 다음 두 단원 동안: 좌표 변환, 미분 형식, 적분 정의
\section{Multilinear algebra}

\begin{prb}[Tensor product]
\end{prb}

\begin{prb}[Wedge product]
\end{prb}



\begin{prb}[One-forms]
\end{prb}


\begin{prb}[Multiple integral]
volume forms,
stone weierstrass and fubini
\end{prb}



\section{Vector calculus}

\begin{prb}[Exterior derivative]
\end{prb}

\begin{prb}[Musical isomorphisms]
\end{prb}

\begin{prb}[Inner product of differential forms]
ONB
\end{prb}

\begin{prb}[Hodge star operator]
Identification of 2-forms and vector fields
\end{prb}

\begin{prb}[Gradient, curl, and divergence]
\end{prb}

\begin{prb}[Potentials]
\end{prb}

\begin{prb}[Vector calculus identities]
\end{prb}


\section*{Exercises}

\begin{prb}[Multivariable Taylor's theorem]
Symmetric product
\end{prb}

\begin{prb}[Vector analysis in two dimension]
\end{prb}

\begin{prb}[Geometric algebra]
\end{prb}








\chapter{Stokes theorems}

\section{Local coordinates}

\begin{prb}[Spherical coordinates]
Let $U=\R^3\setminus\{\,(x,y,z):x=0,\ y\ge0\,\}$.
\[(x,y,z)=(r\sin\theta\cos\f,r\sin\theta\sin\f,r\cos\theta)\]
for $(r,\theta,\f)\in(0,\infty)\times(0,\pi)\times(0,2\pi)$.
Orthonormal bases are
\[\left(\pd_r,\ \frac1r\pd_\theta,\ \frac1{r\sin\theta}\pd_\f\right),\]
\[(dr,\ r\,d\theta,\ r\sin\theta\,d\f),\]
\[(r^2\sin\theta\,d\theta\wedge d\f,\ r\sin\theta\,d\f\wedge dr,\ r\,dr\wedge d\theta).\]
\begin{parts}
\item
\item The Laplacian is given by
\[\Delta f=\frac1{r^2}\pd{r}\left(r^2\pd{f}{r}\right)+\frac1{r^2\sin\theta}\pd{\theta}\left(\sin\theta\pd{f}{\theta}\right)+\frac1{r^2\sin^2\theta}\pd[2]{f}{\f}.\]
\end{parts}
\end{prb}
\begin{pf}
Write $df$ in the orthonormal basis
\begin{align*}
df&=\pd{f}{r}\,dr+\pd{f}{\theta}\,d\theta+\pd{f}{\f}\,d\f\\
&=\left(\pd{f}{r}\right)\,dr+\left(\frac1r\pd{f}{\theta}\right)\,r\,d\theta+\left(\frac1{r\sin\theta}\pd{f}{\f}\right)\,r\sin\theta\,d\f.
\end{align*}
After taking the Hodge star operator
\begin{align*}
{}*df&=\left(\pd{f}{r}\right)\,r^2\sin\theta\,d\theta\wedge d\f+\left(\frac1r\pd{f}{\theta}\right)\,r\sin\theta\,d\f\wedge dr+\left(\frac1{r\sin\theta}\pd{f}{\f}\right)\,r\,dr\wedge d\theta\\
&=r^2\sin\theta\pd{f}{r}\,d\theta\wedge d\f+\sin\theta\pd{f}{\theta}\,d\f\wedge dr+\frac1{\sin\theta}\pd{f}{\f}\,dr\wedge\theta,
\end{align*}
the differential is computed as
\begin{align*}
d*df&=d\left(r^2\sin\theta\pd{f}{r}\right)\,d\theta\wedge d\f+d\left(\sin\theta\pd{f}{\theta}\right)\,d\f\wedge dr+d\left(\frac1{\sin\theta}\pd{f}{\f}\right)\,dr\wedge\theta\\
&=\left[\sin\theta\pd{r}\left(r^2\pd{f}{r}\right)+\pd{\theta}\left(\sin\theta\pd{f}{\theta}\right)+\frac1{\sin\theta}\pd[2]{f}{\f}\right]\,dr\wedge d\theta\wedge d\f,
\end{align*}
so that we have
\begin{align*}
\Delta f={}*d*df&=\frac1{r^2\sin\theta}\left[\sin\theta\pd{r}\left(r^2\pd{f}{r}\right)+\pd{\theta}\left(\sin\theta\pd{f}{\theta}\right)+\frac1{\sin\theta}\pd[2]{f}{\f}\right]\\
&=\frac1{r^2}\pd{r}\left(r^2\pd{f}{r}\right)+\frac1{r^2\sin\theta}\pd{\theta}\left(\sin\theta\pd{f}{\theta}\right)+\frac1{r^2\sin^2\theta}\pd[2]{f}{\f}
\end{align*}

\end{pf}





\section{Integration on curves and surfaces}

\begin{prb}[Line integral]
\end{prb}

\begin{prb}[Surface integral]
\end{prb}


\section{Stokes theorems}
% 미분기하보다 피디이스럽게
\begin{prb}[Bump functions]
\end{prb}

\begin{prb}[Partition of unity]
\end{prb}

\begin{prb}
\end{prb}


\end{document}