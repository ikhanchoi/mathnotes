\documentclass{../../large}
\usepackage{../../ikhanchoi}

\newcommand{\Bun}{\operatorname{Bun}}
\newcommand{\PSh}{\operatorname{PSh}}
\newcommand{\Sh}{\operatorname{Sh}}

\begin{document}
\title{Complex Analysis}
\author{Ikhan Choi}
\maketitle
\tableofcontents


\part{One complex variable}


\chapter{Holomorphic functions}
\section{Cauchy theory}

\begin{prb}[Holomorphic functions]
A \emph{domain} in $\C$ means a non-empty connected open subset of the complex plane $\C$.
A complex valued function $f$ defined on a domain $\Omega\subset\C$ is called \emph{holomorphic} if it is $C^1$ and complex differentiable, that is, the following limit exists for every $a\in\Omega$:
\[f'(a):=\lim_{z\to a}\frac{f(z)-f(a)}{z-a}.\]
The set of all holomorphic functions on $\Omega$ is denoted by $\Hol(\Omega)$ or $\cO(\Omega)$.
Cauchy-Riemann equation can be interpreted as several ways: the matrix representation of $df$ corresponds to a complex number via $x+iy\mapsto\left(\mat[small]{x&-y\\y&x}\right)$, the closedness of the 1-form $f(z)\,dz$.

Let $f\in C^1(\Omega,\C)$ on a domain $\Omega\subset\C$.
Write $z=x+iy$ and $f(z)=u(x,y)+iv(x,y)$.
\begin{parts}
\item $f$ is holomorphic if and only if it satisfies the Cauchy-Riemann equation in $\Omega$.
\end{parts}
\end{prb}
\begin{pf}
We may assume $a=0\in\Omega$.
Since $f$ is $C^1$, we have the Taylor expansion
\[f(z)-f(0)=u_x(0)x+u_y(0)y+i(v_x(0)x+v_y(0)y)+o(|z|),\qquad z\to0.\]

($\Rightarrow$)
Let $y=0$ so that $z=x$.
Then,
\[f(z)-f(0)=u_x(0)x+iv_x(0)x+o(|x|),\qquad x\to0\]
implies $f'(0)=u_x(0)+iv_x(0)$.
If we let $x=0$ so that $z=iy$, then
\[f(z)-f(0)=u_y(0)(y)+iv_y(0)y+o(|y|),\qquad y\to0\]
implies $f'(0)=-iu_y(0)+v_y(0)$.


($\Leftarrow$)
If the Cauchy-Riemann equation implies
\[f(z)-f(0)=u_x(0)z+iv_x(0)z+o(|z|),\qquad z\to0.\]
\end{pf}


\begin{prb}[Contour integral]
We mean by a \emph{contour} on a domain $\Omega\subset\C$ is a formal sum $\gamma=\sum_{i=1}^n\gamma_i$ with $n\ge1$ of $C^1$ paths $\gamma_i:[a_i,b_i]\to\Omega$ such that $\gamma_i(b_i)=\gamma_{i+1}(a_{i+1})$ for all $1\le i<n$ and $\gamma_n(b_n)=\gamma_1(a_1)$, which we call the components of $\gamma$.
In other words, a contour can just be regarded as a piecewise $C^1$ closed curve.
A formal sum of contours on $\Omega$ whose components are all defined on the unit interval is called a \emph{$C^1$ singular 1-cycle} on $\Omega$.

The \emph{contour integral} of $f\in\Hol(\Omega)$ along a contour $\gamma=\sum_{i=1}^n\gamma_i$ is defined by
\[\int_\gamma f(z)\,dz:=\sum_{i=1}^n\int_{a_i}^{b_i}\gamma_i^*(f(z)\,dz)=\sum_{i=1}^n\int_{a_i}^{b_i}f(\gamma_i(t))\gamma_i'(t)\,dt.\]
\begin{parts}
\item The contour integral does not depend on the choice of $\Omega$ containing $\gamma$, and on the reparametrization of $\gamma$.
\item
If we denote by $|z|=1$ the contour $\gamma(\theta):=e^{i\theta}$ with $\theta\in[0,2\pi]$, then for $n\in\Z$ we have
\[\int_{|z|=1}z^n\,dz=\begin{cases}2\pi i&\text{ if }n=-1,\\0&\text{ otherwise }.\end{cases}\]
\end{parts}
\end{prb}

\begin{prb}[Cauchy theorem]
We mean by a \emph{triangle} in a domain $\Omega\subset\C$ a map $\sigma:\Delta\to\Omega$ that has a $C^1$ extension on a neighborhood of $\Delta$, where
\[\Delta:=\{(x,y)\in\R^2:x\ge0,\ y\ge0,\ x+y\le1\}.\]
The \emph{boundary} of a triangle $\sigma$ is a contour defined as $\partial\sigma=\gamma_1+\gamma_2+\gamma_3$, where
\[\gamma_1(t):=\sigma(t,0),\quad\gamma_2(t):=\sigma(1-t,t),\quad\gamma_3(t):=\sigma(0,1-t),\qquad t\in[0,1],\]
and a formal sum of the boundary of triangles is called a \emph{$C^1$ singular 1-boundary} on $\Omega$.
\begin{parts}
\item A contour on $\Omega$ whose components are defined on the unit interval is null-homotopic if and only if it is the sum of the boundary of some triangles in $\Omega$.
\item $\Omega$ is contractible if and only if $\Omega$ is simply connected.
\item If $\Omega$ is simply connected, then for a contour $\gamma$ and a holomorphic function $f$ on $\Omega$,
\[\int_\gamma f(z)\,dz=0.\]
\end{parts}
\end{prb}
\begin{pf}
(a)
$C^1$ approximation...

(c)
Since $f$ is holomorphic, the 1-form $f(z)\,dz$ is closed.
The Stokes theorem writes
\[\int_{\partial\sigma}f(z)\,dz=\int_\sigma d(f(z)\,dz)=0\]
for arbitrary triangle $\sigma:\Delta\to\Omega$.
\end{pf}


\begin{prb}[Cauchy integral formula]
Let $f$ be a holomorphic function on a simply connected domain $\Omega\subset\C$.
\[f(a)=\frac1{2\pi i}\int_\gamma\frac{f(z)}{z-a}\,dz.\]
Remind the proof of the mean value property for harmonic functions.
The proof essentially have a shrinking process using the homotopy and uses the boundedness of the difference quotient.
Higher order version: we can prove before the analyticity by interchange of diff and int.
\end{prb}
\begin{pf}

\end{pf}




\begin{prb}[Cauchy estimates]
\begin{parts}
\item If an entire function $f$ satisfies $|f(z)|\lesssim1+|z|^n$, then $f$ is a polynomial of degree at most $n$. In particular, the \emph{Liouville theorem} follows; a bounded entire function is constant.
\end{parts}
\end{prb}



\section{Power series}

\begin{prb}[Analyticity of holomorphic functions]
\[\sup_{z\in K}\left|\frac{f^{(k)}(z)}{k!}\right|^{\frac1{k+1}}<\infty.\]
\begin{parts}
\item
A real function on $I\subset\R$ is analytic if and only if it has an analytic extension on an open neighborhood $\Omega$ of $I$ in $\C$.
\end{parts}
\end{prb}




\begin{prb}[Identity theorem]
\end{prb}
identity theorem for harmonic: on an open set, but not on the real line, e.g.~$0$ and $y$

\begin{prb}[Open mapping theorem]
\end{prb}
inverse function if $n=1$
open mapping if $n\ge1$
Maximum principle
Schwarz lemma and description of automorphisms of the disk




\begin{prb}[Morera theorem]
The $C^1$ condition in the definition of holomorphic functions is necessary to apply the Stokes theorem when we prove the Cauchy theorem.
However, the $C^1$ condition can be dropped and the pointwise complex differentiability is sufficient to check a function is holomorphic.
Let $f\in C(\Omega,\C)$ on a domain $\Omega\subset\C$.
\begin{parts}
\item If for every point $a\in\Omega$ there is an open neighborhood $U$ of $a$ in $\Omega$ in which every affine triangle $\sigma:\Delta\to U$ satisfies $\int_{\partial\sigma}f(z)\,dz=0$, then $f$ is holomorphic. (Morera)
\item If $f$ is complex differentiable everywhere on $\Omega$, then it is holomorphic. (Goursat)
\end{parts}
\end{prb}
\begin{pf}
(a)
Let $U=\{z\in\Omega:|z-a|<\e\}$ for sufficiently small $\e$ in which every triangle $\sigma:\Delta\to U$ is integrated out by $f$.
If we define
\[F(z):=\int_0^zf(\zeta)\,d\zeta,\qquad z\in U,\]
then by the triangle condition, we have
\[F(z+h)-F(z)=\int_z^{z+h}f(\zeta)\,d\zeta.\]
We can show $F'(z)=f(z)$ by the continuity of $f$, so $F$ is holomorphic on $U$.
Therefore $f$ is holomorphic because it also has the power series representation as well as $F$.

(b)
We prove $\int_{\partial\sigma}f(z)\,dz=0$ for all affine triangle $\sigma:\Delta\to\Omega$.
Suppose not.
Then, there is a triangle $\sigma:\Delta\to\Omega$ such that $\int_\sigma f(z)\,dz\ne0$.
By subdivision, we have $\partial\sigma\simeq\sum_{i=1}^4\partial\sigma_i$ with $\diam\sigma_i\le\frac12\diam\sigma$, so there is $i$ such that
\[|\int_{\partial\sigma_i}f(z)\,dz|\ge\frac14|\int_\sigma f(z)\,dz|.\]
Then, we have a sequence of affine triangles $\sigma_n$ such that
\[|\int_{\partial\sigma_n}f(z)\,dz|\ge\frac1{4^n}|\int_\sigma f(z)\,dz|.\]
Take $a\in\Omega$ the limit point of the subdivision.
By the assumption, there is $\delta>0$ such that
\[|z-a|<\delta\quad\Rightarrow\quad\left|\frac{f(z)-f(a)}{z-a}-f'(a)\right|<\e,\]
so we see that
\[|\int_{\partial\sigma_n}f(z)\,dz|=|\int_{\partial\sigma_n}(f(z)-f(a)-f'(a)(z-a))\,dz|\le\e\sup_{z\in\partial\sigma_n}|z-a|\cdot\operatorname{length}(\partial\sigma_n)\lesssim\frac\e{4^n}.\]
The limit $\e\to0$ leads to a contradiction.
\end{pf}


\section{Harmonic functions on two dimensions}
Harmonic conjugate

\begin{prb}[Mean value property]
\[\frac1{2\pi}\int_0^{2\pi}f(re^{i\theta})(re^{i\theta})^{-k}\,d\theta
=\begin{cases}0&\text{ if }k<0\\\dfrac{f^{(k)}(0)}{k!}&\text{ if }k\ge0\end{cases}\]
for $r$ such that $f$ is defined on $\bar B_r$.
\end{prb}
\begin{prb}[Schwarz integral formula]
Let $f$ be a holomorphic function on the open unit disk $\D$.
If $h$ is another holomorphic function, then
\[f(a)=\frac1{2\pi}\int_{|z|=r}f(z)\left(\frac z{z-a}+zh(z)\right)\,\frac{dz}{iz}\]
for $0<r<1$.
Schwarz integral formula
\[f(a)=\frac1{2\pi}\int_0^{2\pi}\frac{re^{i\theta}+a}{re^{i\theta}-a}\Re f(re^{i\theta})\,d\theta+i\Im f(0).\]
\begin{parts}
\item Find the holomorhpic $h_a$ on an open neighborhood of $\D$ in terms of $a$ such that $|z|=1$ implies $\frac z{z-a}+zh_a(z)$ is real.
\item Poisson kernel.
\end{parts}
\end{prb}
\begin{pf}
\[h_a(z)=\]
\end{pf}




Maximum principle; Lindelöf principle,


\section{Polynomial approximatioin}

Mittag-Leffler theorem
\begin{prb}[Compact convergence of holomorphic functions]
\begin{parts}
\item injectivity preservation: Hurwitz theorem
\end{parts}
\end{prb}

Principal part
For a meromorphic function $f$, we say a polynomial $p$ without constant term is a \emph{principal part} of $f$ at $z_0$ if we have a partial fraction decomposition
\[f(z)=p\left(\frac1{z-z_0}\right)+h(z),\]
where $h(z)$ is holomorphic at $z_0$.
It is unique.
pre-assigned principal parts

Weierstrass factorization theorem
Infinite product

Runge's approximation
Mergelyan





\section*{Exercises}
\begin{prb}[Wirtinger derivatives]
\end{prb}
\begin{prb}[Branch of logarithm and $n$th root]
on simply connected domain
\end{prb}
\begin{prb}[Log$r$ on $\C\setminus\{0\}$]
harmonic function wihtout harmonic conjugate?
\end{prb}

\begin{prb}[Fundamental theorem of algebra]
Let $p\in\C[z]$ be a polynomial of degree $n$ such that
\[p(z)=\sum_{k=0}^nc_kz^k,\quad ㅊ_n\ne0.\]
\begin{parts}
\item $|p(z)|\lesssim|z|^n$.
\item There is $R>0$ such that $|p(z)|\gtrsim|z|^n$ for $|z|\ge R$.
\end{parts}
\end{prb}
\begin{pf}
(b)
We want to justify that the leading term $a_nz^n$ is dominant in the series $\sum_{k=0}^nc_kz^k$ when $|z|$ is sufficiently large.
Let $\e>0$.
Since $p(z)-c_nz^n$ is of degree at most $n-1$, we can take $R>0$ such that for $|z|\ge R$ we can control the relative error as
\[\left|\frac{p(z)-c_nz^n}{c_nz^n}\right|<\e,\]
which implies
\[|p(z)|\ge(1-\e)|c_n||z^n|.\]
\end{pf}





\section*{Problems}

\begin{enumerate}
\item If a holomorphic function has positive real parts on the open unit disk then $|f'(0)|\le2\Re f(0)$.
\item If at least one coefficient in the power series of a holomorphic function at each point is 0 then the function is a polynomial.
\item If a holomorphic function on a domain containing the closed unit disk is injective on the unit circle, then so is on the disk.
\item For a holomorphic function $f$ and every $z_0$ in the domain, there are $z_1\ne z_2$ such that $\frac{f(z_1)-f(z_2)}{z_1-z_2}=f'(z_0)$.
\item Let $f:\Omega\to\C$ be a holomorphic function on a domain. Then, $\bar{f(z)}=f(\bar z)$ if and only if $f(z)\in\R$ for $z\in\Omega\cap\R$.
\item For two linearly independent entire functions, one cannot dominate the other.
\item The uniform limit of injective holomorphic function is either constant or injective.
\item If the set of points in a domain $U\subset\C$ at which a sequence of bounded holomorphic functions converges has a limit point, then it compactly converges.
\item Find all entire functions $f$ satisfying $f(z)^2=f(z^2)$.
\item An entire function maps every unbounded sequence to an unbounded sequence is a polynomial.
\item If a holomorphic function satisfies $\Re f(z)\le1+|z|^2$, then $f$ is a polynomial at most degree two.
\item If $f(z)=\sum_{k=0}^\infty c_kz^k$ is a holomorphic function defined on the open unit disk satisfying $\sum_{k=2}^\infty k|c_k|\le|c_1|\ne0$, then $f$ is injective. (Grunsky coefficients)
\end{enumerate}



\chapter{Analytic continuation}

\section{Riemann surfaces}
Three perspectives:
We can see $\P^1$ as the moduli space of lines, $U_0\cup U_1$, and $\C\cup\{\infty\}$.


Runge: $\C[z]$ is dense in $\cO(\Omega)$ if $\Omega$ is simply connected.

Mergelyan: $\C[z]$ is dense in $\cA(\bar\Omega):=\cO(\Omega)\cap C(\bar\Omega)$.

transformation rule? gluing rule?

\begin{prb}[Riemann sphere]

\end{prb}

\begin{itemize}
\item analytic continuation by functional equation
\item analytic continuation by contour integral
\end{itemize}

\begin{prb}[Analytic continuation by contour integral]
For a not necessarily closed contour $\gamma$ on $\Omega$,
\[h(a):=\frac1{2\pi i}\int_\gamma\frac{f(z)}{z-a}\,dz,\qquad a\in\Omega\setminus\im\gamma\]
is a holomorphic function on $\Omega\setminus\im\gamma$.
For this, you can use either the power series or the Morera with Fubini.

If $f$ is holomorphic on the complement of a zero-length set(can we describe it with rectifiability?) in $\Omega$, then it is holomorphic. (Painlev\'e)
\end{prb}



\begin{prb}[Branch cut]
We can represent $f$ with any coordinate system(usually polar coordinates).

Define $f:\{re^{i\theta}:r>0,-\pi<\theta<\pi\}\to\C$ such that
\[f(re^{i\theta}):=\log r+i\theta.\]
Then, $e^{f(z)}=z$.
Define $f:\{x+iy:y\ne0\text{ or }-1<x<1\}\to\C$ such that
\[f(z):=\frac1{\sqrt{r_+r_-}}e^{i\frac{\theta_++\theta_-}2},\]
where $z-1=r_+e^{i\theta_+}$ and $z+1=r_-e^{i\theta_-}$.
Then, $f(z)$ is a branch of $1/\sqrt{z^2-1}$.
\end{prb}

Monodromy
Covering surfaces
Algebraic functions
Elliptic functions
Uniformization







\chapter{Zeros and poles}

\section{Isolated singularities}
\begin{prb}[Isolated singularities]
removable singularity, pole, essential singularity
\end{prb}
\begin{prb}[Laurent series expansion]
\end{prb}
\begin{prb}[Casorati-Weierstrass theorem]
\end{prb}
\begin{prb}[Picard's theorems]
\end{prb}


\section{Residue theorem}
\begin{prb}[Residue theorem]

\end{prb}


\begin{prb}[Unit circle substitution]
\[\int_0^{2\pi}\frac{dx}{1+a\cos x}=\frac{2\pi}{\sqrt{1-a^2}},\quad-1<a<1\]
\end{prb}



\begin{prb}[Semicircular contour]
We want to justify the following definite integral:
\[\int_0^\infty\frac{\cos x}{x^2+1}\,dx=\frac\pi{2e}.\]
This can be viewed as a special value of the characteristic function of the \emph{Cauchy distribution} in probability theory.
Define $f:\C\setminus\{\pm i\}\to\C$ and the \emph{semicircular contour} $\gamma=\gamma_1+\gamma_2$ for $R>0$ as follows:
\[f(z):=\frac{e^{iz}}{z^2+1},\qquad
\begin{cases}
\gamma_1(x):=x&\text{ for }x\in[-R,R],\\
\gamma_2(\theta):=Re^{i\theta}&\text{ for }\theta\in[0,\pi].
\end{cases}\]
\begin{parts}
\item We have
\[\sup_{R>0}\int_{\gamma_2}|e^{iz}|\,|dz|\le1.\]
This is called the \emph{Jordan lemma}.
\item
\[\lim_{R\to\infty}\int_{\gamma_i}f(z)\,dz=\begin{cases}
2\int_0^\infty\frac{\cos x}{x^2+1}\,dx&\text{ if }i=1\\
0&\text{ if }i=2
\end{cases}\]
\item
\[\lim_{R\to\infty}\int_\gamma f(z)\,dz=\frac\pi e.\]
\end{parts}
\end{prb}
\begin{pf}
(a)
Let $M_R=\max_{z\in \gamma_2}|h(z)|$.
Since $\sin\theta\ge\frac2\pi\theta$ for $0\le\theta\le\frac\pi2$, we have
\begin{align*}
\Bigl|\int_{\gamma_2}e^{iz}h(z)\,dz\Bigr|
&=\Bigl|\int_0^\pi e^{iRe^{i\theta}}h(Re^{i\theta})\,iRe^{i\theta}\,d\theta\Bigr|\\
&\le M_RR\int_0^\pi e^{-R\sin\theta}\,d\theta\\
&=2M_RR\int_0^{\frac\pi2}e^{-R\sin\theta}\,d\theta\\
&\le2M_RR\int_0^{\frac\pi2}e^{-R\frac2\pi\theta}\,d\theta\\\
&=\pi M_R(1-e^{-R}).
\end{align*}
So we are done because $\lim_{R\to\infty}M_R=0$.

(b)
For $i=1$, we have
\[\lim_{R\to\infty}\int_{\gamma_1}f(z)\,dz=\lim_{R\to\infty}\int_{-R}^Rf(x)\,dx=2\int_0^\infty f(x)\,dx\]
by the definition of improper integrals.
For $i=2$, it clearly follows from the part (a).

(c)
Note that for sufficiently large $R$, the function $f$ has only one pole at $z=i$ in the interior of $C$, which is simple; define $g:\operatorname{int}\gamma\to\C$ such that
\[f(z)=:\frac{g(z)}{(z-i)}=\frac{g(i)}{z-i}+\frac{g(z)-g(i)}{z-i}.\]
Then, by the residue theorem, we obtain
\[\int_\gamma f(z)\,dz=2\pi i\Res(f,i)=\frac\pi e\]
for sufficiently large $R$ such that $R>1$.
\end{pf}

\begin{prb}[Indented contour]
Indented contour is often used to compute the principal value of integrals.
Here we want to justify the \emph{Dirichlet integral} as an example:
\[\int_0^\infty\frac{\sin x}x\,dx=\frac\pi2.\]
Define $f:\C\setminus\{0\}\to\C$ and the \emph{indented contour} for $r,R>0$ as follows:
\[f(z)=\frac{e^{iz}}z,\qquad
\left\{
\begin{alignedat}{2}
\gamma_1&:x\mapsto x,&\quad&x\in[r,R],\\
\gamma_2&:\theta\mapsto Re^{i\theta},&&\theta\in[0,\pi],\\
\gamma_3&:x\mapsto x,&&x\in[-R,-r],\\
\gamma_4&:\theta\mapsto re^{\pi-\theta},&&\theta\in[0,\pi].
\end{alignedat}
\right.\]
The indented contour is effective when $f$ has a simple pole at zero.
\begin{parts}
\item
\[\lim_{\substack{R\to\infty\\r\to0}}\int_Cf(z)\,dz=\begin{cases}
0&\text{ if }\gamma=\gamma_1+\gamma_2+\gamma_3+\gamma_4\\
2i\int_0^\infty\frac{\sin x}x\,dx&\text{ if }\gamma=\gamma_1+\gamma_3\\
0&\text{ if }\gamma=\gamma_2\\
-\pi i&\text{ if }\gamma=\gamma_4.
\end{cases}\]
\end{parts}
\end{prb}
\begin{pf}

It follows from the Jordan lemma.


For $\gamma=\gamma_4$, since we have a partial fraction decomposition
\[f(z)=\frac1z+h(z),\qquad h(z):=\frac{e^{iz}-1}z,\]
where $h$ has a removable singularity at zero,
\[\int_{\gamma_4}f(z)\,dz=\int_{\gamma_4}\frac{dz}z+\int_{\gamma_4}h(z)\,dz\to-\pi i+0\]
as $r\to\infty$.
\end{pf}

\begin{prb}[Sector contour]
We want to justify the \emph{Fresnel integral}:
\[\int_0^\infty\cos x^2\,dx=\sqrt{\frac\pi8}.\]
Sector contour is also used to compute the Fourier transform of Gaussian function, which also contains a nonlinear polynomial in a exponential term.
Define $f:\C\setminus\{0\}\to\C$ and the \emph{circular sector contour} for $R>0$ as follows:
\[f(z)=e^{iz^2},\qquad
\left\{
\begin{alignedat}{2}
\gamma_1&:x\mapsto x,&\quad&x\in[0,R],\\
\gamma_2&:\theta\mapsto Re^{i\theta},&&\theta\in[0,\tfrac\pi4],\\
\gamma_3&:x\mapsto(R-x)e^{\frac\pi4i},&&x\in[0,R].
\end{alignedat}
\right.\]
\begin{parts}
\item
\end{parts}
\end{prb}
\begin{pf}
(b)

\end{pf}

\begin{prb}[Rectangular contour]
A rectangular contour is used for the Fourier transform of functions periodic along imaginary direction.
\[\int_0^\infty\frac{\sin x}{e^x-1}\,dx,\qquad\int_0^\infty\frac{\cos x}{\cosh x}\,dx\]
\end{prb}

\begin{prb}[Keyhole contour]
the \emph{keyhole contour} or the \emph{Hankel contour}

\[\int_0^\infty\frac{x^{a-1}}{1+x}=\frac\pi{\sin\pi a}\quad(0<a<1),\qquad\int_1^\infty\frac{dx}{x\sqrt{x^2-1}}\]
$\log z$ trick
\[\int_0^\infty\frac{dx}{1+x^3}\]
\end{prb}






\section{Argument principle}

\begin{prb}[Argument principle]\,
\begin{parts}
\item We have a partial fraction decomposition
\[\frac{f'(z)}{f(z)}=\frac{\ord_a(f)}{z-a}+h(z),\]
where $h$ is holomorphic at $a$.
\item
\[\frac1{2\pi i}\int_\gamma\frac{f'(z)}{f(z)}g(z)\,dz=\sum_a\ord_a(f)g(a).\]
\item Winding number
\end{parts}
\end{prb}
\begin{pf}
\[\frac{f'(z)}{f(z)}=\frac{\ord_a(f)}{z-a}+\frac{g'(z)}{g(z)},\]
where $g(z):=f(z)/(z-a)^{\ord_a(f)}$ is holomorphic at $a$.
\end{pf}


\begin{prb}[Rouch\'e theorem]
Let $f$ be a meromorphic function on $\Omega$.
\begin{parts}
\item
If $h:[0,1]\times\Omega\to\C$ is continuous, then 
\[\int_\gamma\frac{f'(z)}{f(z)}\,dz=\int_\gamma\frac{g'(z)}{g(z)}\,dz.\]
In particular, if $|g(z)|<|f(z)|$ on $z\in\gamma$, then
\[\int_\gamma\frac{f'(z)}{f(z)}\,dz=\int_\gamma\frac{f'(z)+g'(z)}{f(z)+g(z)}\,dz.\]
\end{parts}
\end{prb}

\section{Nevanlinna theory}

\begin{prb}[Poisson-Jensen formula]
\end{prb}

\begin{prb}[Nevanlinna functions]
Let $f$ be a meromorphic function on a neighborhood of the closed disk $\bar{B(0,r)}\subset\C$ and let $a\in\CP^1$.
We count the number of poles in the region $|z|\le r$, counting multiplicity, with the following function
\[n(r,a,f):=\sum_{|z|\le r}(\ord_z(f-a))^+,\qquad n(r,f):=n(r,\infty,f).\]
Note that $n(r,a,f)=n(r,(f-a)^{-1})$ and $n(0,f^{-1})-n(0,f)=\ord_0f$.
The \emph{Nevanlinna proximity function} is
\[m(r,f):=\frac1{2\pi}\int_0^{2\pi}\log^+|f(re^{i\theta})|\,d\theta.\]
The \emph{Nevanlinna counting function} is
\[N(r,f):=\int_0^r(n(t,f)-n(0,f))\,\frac{dt}t+n(0,f)\log r.\]
The \emph{Nevanlinna characteristic function} is
\[T(r,f):=m(r,f)+N(r,f).\]
\end{prb}


\begin{prb}[First fundamental theorem]
Jensen formula
\end{prb}

\begin{prb}[Second fundamental theorem]
\end{prb}

\begin{prb}[Ahlfors-Shimizu formulation]
Let $f$ be a meromorphic function on $\C$.
Consider the following uniform probability measure on the Riemann sphere
\[d\rho(w):=\frac{du\,dv}{\pi(1+|w|^2)^2},\qquad w=u+iv.\]
Define
\[A(r,f):=\frac1\pi\int_{|z|\le r}f^\#(z)^2\,dx\,dy=\int_{|z|\le r}f^*d\rho,\qquad f^\#(z):=\frac{|f'(z)|}{1+|f(z)|^2}.\]
The latter function $f^\#$ is called the \emph{spherical derivative} of $f$.
The \emph{Ahlfors-Shimizu characteristic function} and \emph{proximity function} are defined by
\[T_0(r,f):=\int_0^rA(t,f)\frac{dt}t,\qquad m_0(r,f):=\frac1{2\pi}\int_0^{2\pi}\log\sqrt{1+|f(re^{i\theta})|^2}\,d\theta.\]
\begin{parts}
\item $\int\log|f-w|\,d\rho(w)=\log\sqrt{1+|f|^2}$
\item We have
\[A(r,f)=\int n(r,a,f)\,d\rho(a)=n(r,f)+r\frac d{dr}m_0(r,f).\]
\item We have $T_0(r,f)=T(r,f)+O(1)$ as $r\to\infty$.
\end{parts}
\end{prb}
\begin{pf}
(b)

Let $F$ be the image of the set $\{z:|z|=r\}\cup\{z:f'(z)=0\}$ under $f$.
Since $F$ and $f^{-1}(F)$ are of measure zero, so we may assume $f:U\to f(U)$ is locally biholomorphic, where $U:=\{z:|z|\le r\}\setminus f^{-1}(F)$.
So we may define the degree of $f$, which is locally constant and coincides with $n(r,a,f)$.
So the first equality follows from
\[\int_{|z|\le r}f^*d\rho=\int_Uf^*d\rho=\int n(r,a,f)\,d\rho(a).\]


By the argument principle,
\[n(t,a,f)-n(t,f)=\frac1{2\pi i}\int_{|z|=t}\frac{f'(z)}{f(z)-a}\,dz,\]
and by
\[\frac1{2\pi}\int_0^{2\pi}\frac1{f(z)-re^{i\theta}}\,d\theta=\frac1{2\pi i}\int_{|w|=r}\frac1{f(z)}\left(\frac1{f(z)-w}+\frac1w\right)dw=\begin{cases}
1/f(z)&\text{ if }r<|f(z)|,\\0&\text{ if }r>|f(z)|
\end{cases}\]
for fixed $f(z)\in\C$ and $r>0$, we have
\begin{align*}
\int n(t,a,f)\,d\rho(a)-n(t,f)
&=\frac1{2\pi i}\int_{|z|=t}\int\frac{f'(z)}{f(z)-a}\,d\rho(a)\,dz\\
&=\frac1{2\pi i}\int_{|z|=t}\int_0^\infty\int_0^{2\pi}\frac{f'(z)}{f(z)-re^{i\theta}}\frac r{\pi(1+r^2)^2}\,d\theta\,dr\,dz\\
&=\frac1{2\pi i}\int_{|z|=t}\int_0^{|f(z)|}\frac{2\pi f'(z)}{f(z)}\frac r{\pi(1+r^2)^2}\,dr\,dz\\
&=\frac1{2\pi i}\int_{|z|=t}\frac{f'(z)\bar{f(z)}}{1+|f(z)|^2}\,dz.
\end{align*}
Also,
\begin{align*}
t\frac d{dt}m_0(t,f)
&=\frac t{2\pi}\int_0^{2\pi}\frac{d
\log\sqrt{1+|f(te^{i\theta})|^2}}{dt}\,d\theta\\
&=\frac t{2\pi}\int_0^{2\pi}\frac12\frac{f'(te^{i\theta})\bar{f(te^{i\theta})}+f(te^{i\theta})\bar{f'(te^{i\theta})}}{1+|f(te^{i\theta})|^2}e^i\theta\,d\theta\\
&=\frac1{2\pi i}\int_{|z|=t}\frac{f'(z)\bar{f(z)}}{1+|f(z)|^2}\,dz.
\end{align*}


(c)
Two solutions: one is $T_0(r,f)=N(r,f)+m_0(r,f)-m_0(0,f)$.
Another is using $T_0(r,f)=\int N(r,a,f)\,d\rho(a)$ and the first fundamental theorem.
\end{pf}

Applications of second fundamental theorem?
Borel directions and deficient values?

\section*{Exercises}
\begin{prb}[The second proof of the fundamental theorem of algebra]
by Rouch\'e.
\end{prb}
\begin{prb}[Laplace transforms]
\end{prb}
\begin{prb}[Gamma function]
Hankel representation
\end{prb}
\begin{prb}[Abel-Plana formula]
\end{prb}

Sokhotski-Plemelj theorem,
Kramers-Konig relations,
Titchmarsh theorem for Hilbert transform,
Phragm\'en-Lindel\"of principle,
Carlson's theorem

\section*{Problems}
\begin{enumerate}
\item We have $\int_0^{2\pi}\frac{d\theta}{1+\cos^2\theta}=\sqrt2\pi$.
\item Find the number of roots of $z^6+z+1=0$ in $\{x+iy\in\C:x>0,y>0\}$.
\item Find the number of roots of $z-e^{-z}=2$ in the right half plane.
\item If $f$ is an entire function such that $|f(z)|\le e^{|z|^\lambda}$, then $|\{z\in B(0,R):f(z)=0\}|\lesssim R^\lambda$.
\item There is no holomorphic function $f:\D\to\C$ such that $|f(z)|\to\infty$ for all sequences $z_n\in\D$ with $|z_n|\to1$.
\item If $f$ is a bounded holomorphic function defined on $\C\setminus E$, where $E\subset[0,1]$ is the Cantor set, then $f$ is constant.
\item Suppose a sequence of nowhere vanishing holomorphic functions $f_n$ on a domain $\Omega$ converges to a non-constant function $f$ uniformly on compact sets.
Then, $f$ is also nowhere vanishing. (Hurwitz)
\end{enumerate}






\part{Geometric function theory}

\chapter{}
\section{Conformal mappings}
\begin{prb}[Conformality of holomorphic maps]
$f'\ne0$ and $f'$ satisfies the Cauchy-Riemann
\end{prb}
\begin{prb}[M\"obius transform]
generators,
fixed points
\end{prb}
\begin{prb}[Blaschke factors]
\end{prb}

\begin{prb}[Normal family]
locally bounded, then compact (Montel)
\end{prb}

\begin{prb}[Schwarz lemma]
\end{prb}

\begin{prb}[Riemann mapping theorem]
Let $\Omega\subset\C$ be a simply connected domain such that $\Omega\ne\C$.
\[\cF=\{f:\Omega\to\D\mid f\text{ is injective and holomorphic, and }f(z_0)=0\}\]
\begin{parts}
\item There exists an injective holomorphic function $f:\Omega\to\D$.
\item If $0\in\Omega_1\subsetneq\D$, then there is a conformal mapping $h:\Omega_1\to\Omega_2$ such that $h(0)=0$ and $|h'(0)|>1$, where $0\in\Omega_2\subset\D$.
\item The supremum of $|f'(0)|$ is attained in $\cF$.
\item There exists a conformal mapping $f:\Omega\to\D$.
\end{parts}
\end{prb}


\section*{Exercises}
\begin{prb}[Special solution of Laplace' equation]
\end{prb}
\begin{prb}[Normal family for meromorphic functions]
\end{prb}

\section*{Problems}
\begin{enumerate}
\item Find a conformal mapping that maps the open unit disk onto $A:=\{\,z\in\C:\max\{|z|,|z-1|\}<1\,\}$.
\end{enumerate}




\chapter{Univalent functions}
\section{Bierbach conjecture}

\section{Riemann-Hilbert problem}
Hilbert transform
almost everywhere convergence, Hardy-Littlewood maximal function

\section{Quasi-conformal mappings}
Beltrami equations and Teichm\"uler theory?

\section{Exercises}
\begin{prb}[Carath\'eodory class]
Let $f$ be a holomorphic function on the open unit disk $\D$ such that $\Re f(z)>0$ for $z\in\D$ and $f(0)=1$. Show that $|f'(0)|\ge2$.
\end{prb}






\chapter{}






\part{Several complex variables}

\chapter{Complex analytic sheaves}


\section{Analytic spaces}

\begin{prb}[Complex model spaces]

Let $(X,\cO_X)$ be a ringed space and $A$ is a subset of $X$ that is the support of a coherent sheaf on $X$.
Then, we can show (really? I hope so.) that there exists a natural sheaf $\cO_A$ of rings on $X$ such that $A$ is the support of $\cO_A$ and there is locally an exact sequence of sheaves of rings on $X$
\[\cO_U^q\to\cO_U\to\cO_A|_U\to0.\]
The kernel of $\cO_X\to\cO_A$ is called the \emph{ideal sheaf} or the \emph{relation sheaf} of $A$ and denoted by $\cI_{(X,A)}$.

Note that we have a canonical ringed space $(\Omega,\cO_\Omega)$ of holomorphic functions on a domain $\Omega\subset\C^n$.
A \emph{complex model space} is a subset $A$ of some $\Omega$ that is the support of a coherent sheaf on $\Omega$.
\end{prb}
\begin{pf}

\end{pf}


\begin{prb}[Complex analytic spaces]
Let $(X,\cO_X)$ be a ringed space.
An \emph{analytic atlas} on $X$ is the family $\{\f_i\}$ of maps $\f_i:U_i\to\C^{n_i}$ which map open $U_i\subset X$ homeomorphically onto open $\f(U_i)\subset\C^{n_i}$, such that $\f_i$ and $\tau_{ij}=\f_j\f_i^{-1}$ induce the sheaf isomorphisms $\cO_{U_i}\to\cO_{\f(U_i)}$ and $\cO_{\f_i(U_i\cap U_j)}\to\cO_{\f_j(U_i\cap U_j)}$.
A \emph{complex analytic space} or briefly a \emph{complex space} is a ringed space $(X,\cO_X)$ together with an analytic atlas $\{\f_i\}$, satisfying an additional condition that $X$ is Hausdorff.
We do not have to assume the second countability of $X$ because the partition of unity does not play a role in complex analysis.
An \emph{analytic set} in a complex space $X$ is a subset $A$ that is the support of a coherent sheaf on $X$.
\end{prb}




\section{Oka coherence theorems}

$\cO_{\P^1}(\P^1)=0$
$\cM_{\P^1}(\P^1)=\C(z)$
$\C[z]=\cO_{\P^1}(\C)\cap\cM_{\P^1}(\P^1)$
$\Aut(\P^1)\cong\PSL(2,\C)$
$\Hom(\P^1,\P^1)=\C(z)\cup\{\infty\}$.



Four coherence theorems:
\begin{enumerate}
\item 
\item
\item
\item
\end{enumerate}

\begin{prb}
smooth->normal(integrally closed)->irreducible(integral domain)->reduced(no nilpotents)
\end{prb}

\begin{prb}[Reduced points]
R\"ucker nullstellensatz,
every section is realized as a family of functions,
sheaf map $f_*$ is uniquely lifted
\end{prb}

\begin{prb}[Weierstrass preparation theorem]
Consider $\cO'_0\subset\cO'_0[z_n]\subset\cO_0$.
Consider $B_\rho$, where $\rho=(\rho',\rho_n)\in\R_{>0}^n$.
Note $\cO_0=\bigcup_\rho B_\rho$.

A Weierstrass polynomial is a monic polynomial in $\cO'_0[z_n]$ such that $\frac{d^kw}{dz_n^k}(0,0)=0$ for all $k$.
We use the convention that the degree and order are with respect to $z_n$ by letting $z'=0$.
\begin{parts}
\item If $f,g\in\cO_0$, then there are unique $q\in\cO_0$ and $r\in\cO'_0[z_n]$ such that $\deg r<\ord g$ and $f=qg+r$.
\item If $g\in\cO_0$, then there is a unique Weierstrass polynomial $w\in\cO'_0[z_n]$ and $u\in\cO_0^\times$ such that $\deg w=\ord g$ and $g=uw$.
\end{parts}
\end{prb}

\begin{prb}[Weierstrass isomorphism theorem]
\end{prb}


\begin{prb}[First Oka coherence theorem]\,
Let $\cO:=\cO_{\C^n}$ and assume $\cO':=\cO_{\C^{n-1}}$ is coherent.
We prove that $\cO$ is coherent at the origin.
\begin{parts}
\item Let $0\ne f_0\in\cO_0$. Then, there is an open neighborhood $U\subset\C^n$ of the origin such that $f\in\cO(U)$ and $\cO_U\to\cO_U/f\cO_U$ is a split epi over $\cO_U$.
\item Let $0\ne f_0\in\cO_0$. Then, there is an open neighborhood $U\subset\C^n$ of the origin such that $f\in\cO(U)$ and $\cO_U/f\cO_U$ is coherent over $\cO_U$.
\end{parts}
\end{prb}
\begin{pf}
(a)
Note that $\cO$ is locally irreducible and has Hausdorff \'etale.

(b)
We may assume $f(0)=0$, i.e.~there is no constant term in the power series $f_0$, because otherwise it is clear from $\cO_U=f\cO_U$ for some $U$.
We may assume $f_0(0,z_n)\ne0$, i.e.~there is a monomial of $z_n$ in the power series $f_0$ by coordinate transform.
So, by the above assumptions, we have $\ord f_0>0$.
By the Weierstrass preparation theorem, there is a Weierstrass polynomial $w_0\in\cO'_0[z_n]$ such that $\deg w=\ord f$ and $f_0\cO_0=w_0\cO_0$.

Choose open $U'\subset\C^{n-1}$ such that $w_0$ has a representative $w\in\cO'(U')[z_n]$.
Then, we use the Weierstrass isomorphism theorem and the extension principle.


\end{pf}


\begin{prb}[Local rings on complex analytic spaces]\,
\begin{parts}
\item $\cO_{X,x}$ is isomorphic to a quotient of $\cO_{\C^n,0}$.
\item $\cO_{X,x}$ is local, noetherian, and henselian.
\item $\cO_{\C^n,0}$ is factorial.
\end{parts}
\end{prb}


\section{Levi problem}


\begin{prb}[Domains of holomorphy]
A domain $\Omega\subset\C^n$ is called a \emph{domain of holomorphy} if there is no domain $\tilde\Omega\subset\C^n$ such that $\Omega$ is a proper subset of $\tilde\Omega$ and $\cO(\tilde\Omega)\to\cO(\Omega)$ is surjective.
\begin{parts}
\item For a compact $K\subset\Omega$ such that $\Omega\setminus K$ is connected, $\cO(\Omega)\to\cO(\Omega\setminus K)$ is surjective. (Hartog extension theorem)
\item The union of increasing sequence of domains of holomorphy is a domain of holomorphy (Behnke-Stein theorem)
\end{parts}
\end{prb}



\begin{prb}[Holomorphically convex domains]
Let $\Omega\subset\C^n$ be a domain.
For compact $K\subset\Omega$, the \emph{holomorphically convex hull} in $\Omega$ is the set
\[\hat K_\Omega:=\{z\in\Omega:|f(z)|\le\|f\|_{C(K)}\text{ for }f\in\cO(\Omega)\}.\]
We say the domain $\Omega$ is \emph{holomorphically convex} if for every compact $K\subset\Omega$ the holomorphically convex hull $\hat K_\Omega$ is compact.
\begin{parts}
\item A polydisc, a convex domain is holomorphically convex.
\item $\Omega$ is holomorphically convex if and only if it is a domain of holomorphy if and only if $d(K,\partial\Omega)=d(\hat K_\Omega,\partial\Omega)$ for every compact $K\subset\Omega$(Cartan-Thullen theorem)
\end{parts}
\end{prb}

\begin{prb}[Plurisubharmonic functions]
Let $X$ be a complex analytic space.
An upper semi-continuous function $f:X\to\R\cup\{-\infty\}$ is said to be \emph{plurisubharmonic} if for every holomorphic $\f:\D\subset\C\to X$ the composition $f\circ\f$ is subharmonic.
\begin{parts}
\item If $\Omega$ is a domain of holomoprhy, then $-\log d$ is plurisubharmonic.
\end{parts}
\end{prb}

\begin{prb}[Pseudo-convex domains]
\end{prb}

\begin{prb}[Levi problem]
\end{prb}

Oka lemma?

\[\begin{tikzcd}[sep=small]
& \tab{domain of\\holomorphy} \ar{dr}\ar[<->, dashed]{dd}{\scriptsize\tab[l]{in $\C^n$\\Cartan-Thullen}} &&&\\
\text{Stein} \ar{ur}\ar{dr} && \tab{continuous\\exhaustive\\plurisubharmonic} \ar{rr} && \text{pseudo-convex} \\
& \tab{holomorphically\\convex} \ar{ur} &&&
\end{tikzcd}\]


\section{Cartan theorem}

Cartan's theorem B: if $\cF$ is a coherent sheaf on a Stein manifold $X$, then $H^p(X,\cF)=0$ for $p\ge1$.

Cousin problems in terms of sheaf cohomologies:
\begin{enumerate}
\item Characterize the image of $H^0(X,\cM)\to H^0(X,\cM/\cO)$.\\
It is a generalization of the Mittag-Leffler theorem for prescribed poles.
Consider an exact sequence
\[H^0(X,\cM)\to H^0(X,\cM/\cO)\to H^1(X,\cO)=0.\]
Then, the first Cousin problem is solved when $X$ is a Stein manifold.
\item Characterize the image of $H^0(X,\cM^\times)\to H^0(X,\cM^\times/\cO^\times)$.\\
It is a generalization of the Weierstrass theorem for prescribed zeros.
Consider an exact sequence
\[H^0(X,\cM^\times)\to H^0(X,\cM^\times/\cO^\times)\to H^1(X,\cO^\times).\]
The sheaf $\cM^\times/\cO^\times$ is the sheaf of Cariter divisors, and line bundles are classified by $H^1(X,\cO^\times)$.
Considering the exponential exact sequence, we also have an exact sequence
\[0=H^1(X,\cO)\to H^1(X,\cO^\times)\to H^2(X,\Z)\to H^2(X,\cO)=0.\]
Then, the second Cousin problem is solved when $X$ is a Stein manifold such that $H^2(X,\Z)=0$.
We can compute this by the first Chern class.
\end{enumerate}





\end{document}