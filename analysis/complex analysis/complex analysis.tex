\documentclass{../../large}
\usepackage{../../ikhanchoi}


\begin{document}
\title{Complex Analysis}
\author{Ikhan Choi}
\maketitle
\tableofcontents


\part{Holomorphic functions}


\chapter{Cauchy theory}
\section{Complex differentiability}

\begin{prb}[Holomorphic functions]
We call a connected open subset of $\C$ as a \emph{domain}.
definition as a multi-variable calculus, chain rule
\end{prb}

\begin{prb}[Cauchy-Riemann equation]
Cauchy-Riemann equation can be interpreted as several ways: the matrix representation of $df$ corresponds to a complex number via $x+iy\mapsto\mat{x&-y\\y&x}$, the closedness of the 1-form $f(z)\,dz$, 
\begin{parts}
\item For $f\in C^1(\Omega,\R^2)$, $f$ is holomorphic if and only if it satisfies the Cauchy-Riemann equation in $\Omega$. (Is the $C^1$ condition necessary?)
\end{parts}
\end{prb}


\begin{prb}[Contour integral]
Let $f:\Omega\to\C$ be a continuous function on a domain $\Omega\subset\C$ and let $\gamma:[a,b]\to\Omega$ be a $C^1$ curve.
The \emph{contour integral} of $f$ along the curve $\gamma$ is defined by
\[\int_\gamma f(z)\,dz:=\int_a^bf(\gamma(t))\gamma'(t)\,dt.\]
In the language of differential geometry, it is a special case of integration with the pullback form $\gamma^*(f(z)\,dz)$.
We can extend the definition of contour integral to \emph{piecewise $C^1$ curves} by considering it as a formal sum of $C^1$ curves, which will be meant by \emph{contours}.
\begin{parts}
\item The contour integral is independent of the choice of $\Omega$.
\item
\[\int_\gamma f'(z)\,dz=f(\gamma(b))-f(\gamma(a)).\]
\item
\[\int_{|z|=1}z^n\,dz=\begin{cases}2\pi i&\text{ if }n=-1,\\0&\text{ otherwise }.\end{cases}\]
\end{parts}
\end{prb}

\begin{prb}[Cauchy theorem]
The Cauchy integral is independent of the contour up to homotopy.
\end{prb}
\begin{pf}
The assumption $\gamma_0$ and $\gamma_1$ are homotopic means that $\gamma_0-\gamma_1$ is a boundary of a 2-chain.
We have to choose the homotopy on $[0,1]^2$ carefully so that it becomes $C^1$ except on the finite number of vertical and horizontal lines.
It can be realized if a continuous(!) homotopy of curves is contained in a simply connected domain, but I don't know how to approximate $C^1$ homotopy...
Then, the claim follows from the Stokes theorem.
The triangulation technique is used in the proof of the Stokes theorem.
\end{pf}


\begin{prb}[Cauchy integral formula]
Remind the proof of the mean value property for harmonic functions.
The proof essentially have a shrinking process and uses the boundedness of the difference quotient.
Higher order version: we can prove before the analyticity by interchange of diff and int.
\end{prb}




\begin{prb}[Cauchy estimates]
\begin{parts}
\item If an entire function $f$ satisfies $|f(x)|\lesssim1+|x|^n$, then $f$ is a polynomial of degree at most $n$. In particular, the \emph{Liouville theorem} follows; a bounded entire function is constant.
\end{parts}
\end{prb}



\section{Power series}

\begin{prb}[Analyticity of holomorphic functions]
A real function on $I\subset\R$ is analytic iff it has an analytic extension on an open neighborhood $\Omega$ of $I$ in $\C$.
\[\sup_{x\in K}\left|\frac{f^{(k)}(x)}{k!}\right|^{\frac1{k+1}}<\infty.\]
\end{prb}







normal form
and analogue of harmonic functions

\begin{prb}[Identity theorem]
\end{prb}
identity theorem for harmonic: on an open set, but not on the real line, e.g.~$0$ and $y$

\begin{prb}[Open mapping theorem]
\end{prb}
inverse function if $n=1$
open mapping if $n\ge1$
Maximum principle
Schwarz lemma and description of automorphisms of the disk




\begin{prb}[Morera and Goursat theorem]
The $C^1$ condition in the definition of holomorphic functions is necessary to apply the Stokes theorem when we prove the Cauchy theorem.
However, the $C^1$ condition can be dropped and the pointwise complex differentiability is sufficient to check a function is holomorphic.

Suppose $f:\Omega\to\C$ be a continuous function on a domain $\Omega$.
If for every point $z_0\in\Omega$ there is an open neighborhood of $z_0$ in $\Omega$ in which every triangle $T$ satisfies $\oint_{\partial T}f(z)\,dz=0$. (Morera)
\end{prb}
\begin{pf}
If we define
\[F(z):=\int_0^zf(\zeta)\,d\zeta,\]
then by the triangle condition, we have
\[F(z+h)-F(z)=\int_z^{z+h}f(\zeta)\,d\zeta.\]
We can show $F'(z)=f(z)$ by the continuity of $f$, and therefore $f$ is holomorphic because it also has the power series representation as well as $F$.
\end{pf}



\section*{Exercises}
\begin{prb}[Wirtinger derivatives]
\end{prb}
\begin{prb}[Branch of logarithm and $n$th root]
on simply connected domain
\end{prb}
\begin{prb}[Log$r$ on $\C\setminus\{0\}$]
harmonic function wihtout harmonic conjugate?
\end{prb}

\begin{prb}[Fundamental theorem of algebra]
Let $p\in\C[z]$ be a polynomial of degree $n$ such that
\[p(z)=\sum_{k=0}^na_kz^k,\quad a_n\ne0.\]
\begin{parts}
\item $|p(z)|\lesssim|z|^n$.
\item There is $R>0$ such that $|p(z)|\gtrsim|z|^n$ for $|z|\ge R$.
\end{parts}
\end{prb}
\begin{pf}
(b)
We want to justify that the leading term $a_nz^n$ is dominant in the series $\sum_{k=0}^na_kz^k$ when $|z|$ is sufficiently large.
Let $\e>0$.
Since $p(z)-a_nz^n$ is of degree at most $n-1$, we can take $R>0$ such that for $|z|\ge R$ we can control the relative error as
\[\left|\frac{p(z)-a_nz^n}{a_nz^n}\right|<\e,\]
which implies
\[|p(z)|\ge(1-\e)|a_n||z^n|.\]
\end{pf}


\section*{Problems}

\begin{enumerate}
\item If a holomorphic function has positive real parts on the open unit disk then $|f'(0)|\le2\Re f(0)$.
\item If at least one coefficient in the power series of a holomorphic function at each point is 0 then the function is a polynomial.
\item If a holomorphic function on a domain containing the closed unit disk is injective on the unit circle, then so is on the disk.
\item For a holomorphic function $f$ and every $z_0$ in the domain, there are $z_1\ne z_2$ such that $\frac{f(z_1)-f(z_2)}{z_1-z_2}=f'(z_0)$.
\item Let $f:\Omega\to\C$ be a holomorphic function on a domain. Then, $\bar{f(z)}=f(\bar z)$ if and only if $f(z)\in\R$ for $z\in\Omega\cap\R$.
\item For two linearly independent entire functions, one cannot dominate the other.
\item The uniform limit of injective holomorphic function is either constant or injective.
\item If the set of points in a domain $U\subset\C$ at which a sequence of bounded holomorphic functions converges has a limit point, then it compactly converges.
\item Find all entire functions $f$ satisfying $f(z)^2=f(z^2)$.
\item An entire function maps every unbounded sequence to an unbounded sequence is a polynomial.
\item If a holomorphic function satisfies $\Re f(z)\le1+|z|^2$, then $f$ is a polynomial at most degree two.
\item If $f(z)=\sum_{k=0}^\infty a_kz^k$ is a holomorphic function defined on the open unit disk satisfying $\sum_{k=2}^\infty k|a_k|\le|a_1|\ne0$, then $f$ is injective. (Grunsky coefficients)
\end{enumerate}







\chapter{Singularities}

\section{Classification of singularities}
\begin{prb}[Isolated singularities]
\end{prb}
\begin{prb}[Riemann removable singularity theorem]
\end{prb}
\begin{prb}[Laurent expansion at an isolated singularity]
\end{prb}
\begin{prb}[Casorati-Weierstrass theorem]
\end{prb}
\begin{prb}[Picard's theorems]
\end{prb}
Riemann sphere and meromorphic functions?

\section{Residue theorem}
\begin{prb}[Residue theorem]

\end{prb}


\begin{prb}[Unit circle substitution]
\[\int_0^{2\pi}\frac{dx}{1+a\cos x}=\frac{2\pi}{\sqrt{1-a^2}},\quad-1<a<1\]
\end{prb}



\begin{prb}[Semicircular contour]
We want to justify the following definite integral:
\[\int_0^\infty\frac{\cos x}{x^2+1}\,dx=\frac\pi{2e}.\]
This can be viewed as a special value of the characteristic function of the \emph{Cauchy distribution} in probability theory.
Define $f:\C\setminus\{\pm i\}\to\C$ and the \emph{semicircular contour} for $R>0$ as follows:
\[f(z)=\frac{e^{iz}}{z^2+1},\qquad
\left\{
\begin{alignedat}{2}
\gamma_1&:x\mapsto x,&\quad&x\in[-R,R],\\
\gamma_2&:\theta\mapsto Re^{i\theta},&&\theta\in[0,\pi].
\end{alignedat}
\right.\]
\begin{parts}
\item Let $h$ be a holomorphic function on a domain containing the arcs $\gamma_2$ for every large $R>0$.
If $h$ vanishes at infinity, then
\[\lim_{R\to\infty}\int_{\gamma_2}e^{iz}h(z)\,dz=0.\]
This is called the \emph{Jordan lemma}.
\item
\[\lim_{R\to\infty}\int_\gamma f(z)\,dz=\begin{cases}
\frac\pi e&\text{ if }\gamma=\gamma_1+\gamma_2\\
2\int_0^\infty\frac{\cos x}{x^2+1}\,dx&\text{ if }\gamma=\gamma_1\\
0&\text{ if }\gamma=\gamma_2
\end{cases}\]
\end{parts}
\end{prb}
\begin{pf}
(a)
Let $M_R=\max_{z\in \gamma_2}|h(z)|$.
Since $\sin\theta\ge\frac2\pi\theta$ for $0\le\theta\le\frac\pi2$, we have
\begin{align*}
\Bigl|\int_{C_2}e^{iz}h(z)\,dz\Bigr|
&=\Bigl|\int_0^\pi e^{iRe^{i\theta}}h(Re^{i\theta})\,iRe^{i\theta}\,d\theta\Bigr|\\
&\le M_RR\int_0^\pi e^{-R\sin\theta}\,d\theta\\
&=2M_RR\int_0^{\frac\pi2}e^{-R\sin\theta}\,d\theta\\
&\le2M_RR\int_0^{\frac\pi2}e^{-R\frac2\pi\theta}\,d\theta\\\
&=\pi M_R(1-e^{-R}).
\end{align*}
So we are done because $\lim_{R\to\infty}M_R=0$.

(b)
For $\gamma=\gamma_1+\gamma_2$, note that for sufficiently large $R$, the function $f$ has only one pole at $z=i$ in the interior of $C$, which is simple; define $g:\operatorname{int}\gamma\to\C$ such that
\[f(z)=:\frac{g(z)}{(z-i)}=\frac{g(i)}{z-i}+\frac{g(z)-g(i)}{z-i}.\]
Then, by the residue theorem, we obtain
\[\int_\gamma f(z)\,dz=\int_\gamma\frac{g(z)}{z-i}\,dz=2\pi i\cdot g(i)=\frac\pi e\]
for large $R$.

For $\gamma=\gamma_1$, we have
\[\lim_{R\to\infty}\int_{\gamma_1}f(z)\,dz=\lim_{R\to\infty}\int_{-R}^Rf(x)\,dx=2\int_0^\infty f(x)\,dx\]
by the definition of improper integrals.

For $\gamma=\gamma_2$, it clearly follows from the aprt (a).
\end{pf}

\begin{prb}[Indented contour]
Indented contour is often used to compute the principal value of integrals.
Here we want to justify the \emph{Dirichlet integral} as an example:
\[\int_0^\infty\frac{\sin x}x\,dx=\frac\pi2.\]
Define $f:\C\setminus\{0\}\to\C$ and the \emph{indented contour} for $r,R>0$ as follows:
\[f(z)=\frac{e^{iz}}z,\qquad
\left\{
\begin{alignedat}{2}
\gamma_1&:x\mapsto x,&\quad&x\in[r,R],\\
\gamma_2&:\theta\mapsto Re^{i\theta},&&\theta\in[0,\pi],\\
\gamma_3&:x\mapsto x,&&x\in[-R,-r],\\
\gamma_4&:\theta\mapsto re^{\pi-\theta},&&\theta\in[0,\pi].
\end{alignedat}
\right.\]
The indented contour is effective when $f$ has a simple pole at zero.
\begin{parts}
\item
\[\lim_{\substack{R\to\infty\\r\to0}}\int_Cf(z)\,dz=\begin{cases}
0&\text{ if }\gamma=\gamma_1+\gamma_2+\gamma_3+\gamma_4\\
2i\int_0^\infty\frac{\sin x}x\,dx&\text{ if }\gamma=\gamma_1+\gamma_3\\
0&\text{ if }\gamma=\gamma_2\\
-\pi i&\text{ if }\gamma=\gamma_4.
\end{cases}\]
\end{parts}
\end{prb}
\begin{pf}

It follows from the Jordan lemma.


For $\gamma=\gamma_4$, since we have a partial fraction decomposition
\[f(z)=\frac1z+h(z),\qquad h(z):=\frac{e^{iz}-1}z,\]
where $h$ has a removable singularity at zero,
\[\int_{\gamma_4}f(z)\,dz=\int_{\gamma_4}\frac{dz}z+\int_{\gamma_4}h(z)\,dz\to-\pi i+0\]
as $r\to\infty$.
\end{pf}

\begin{prb}[Sector contour]
We want to justify the \emph{Fresnel integral}:
\[\int_0^\infty\cos x^2\,dx=\sqrt{\frac\pi8}.\]
Sector contour is also used to compute the Fourier transform of Gaussian function, which also contains a nonlinear polynomial in a exponential term.
Define $f:\C\setminus\{0\}\to\C$ and the \emph{circular sector contour} for $R>0$ as follows:
\[f(z)=e^{iz^2},\qquad
\left\{
\begin{alignedat}{2}
\gamma_1&:x\mapsto x,&\quad&x\in[0,R],\\
\gamma_2&:\theta\mapsto Re^{i\theta},&&\theta\in[0,\tfrac\pi4],\\
\gamma_3&:x\mapsto(R-x)e^{\frac\pi4i},&&x\in[0,R].
\end{alignedat}
\right.\]
\begin{parts}
\item
\end{parts}
\end{prb}
\begin{pf}
(b)

\end{pf}

\begin{prb}[Rectangular contour]
A rectangular contour is used for the Fourier transform of functions periodic along imaginary direction.
\[\int_0^\infty\frac{\sin x}{e^x-1}\,dx,\qquad\int_0^\infty\frac{\cos x}{\cosh x}\,dx\]
\end{prb}

\begin{prb}[Keyhole contour]
the \emph{keyhole contour} or the \emph{Hankel contour}

\[\int_0^\infty\frac{x^{a-1}}{1+x}=\frac\pi{\sin\pi a}\quad(0<a<1),\qquad\int_1^\infty\frac{dx}{x\sqrt{x^2-1}}\]
$\log z$ trick
\[\int_0^\infty\frac{dx}{1+x^3}\]
\end{prb}






\section{Zeros and poles}
\begin{prb}[Argument principle]\,
\begin{parts}
\item We have a partial fraction decomposition
\[\frac{f'(z)}{f(z)}=\frac{\ord_a(f)}{z-a}+h(z),\]
where $h$ is holomorphic at $a$.
In particular, $f$ has either a zero or a pole at $a$ if and only if $f'(z)/f(z)$ has a simple pole at $a$.
\item
\[\int_\gamma\frac{f'(z)}{f(z)}\,dz=2\pi i(\text{number of zeros}-\text{number of poles}).\]
\item
\[\int_\gamma\frac{f'(z)}{f(z)}g(z)\,dz=2\pi i\sum_a\ord_a(f)g(a).\]
\item Winding number
\end{parts}
\end{prb}
\begin{pf}
\[\frac{f'(z)}{f(z)}=\frac{\ord_a(f)}{z-a}+\frac{g'(z)}{g(z)},\]
where $g(z):=f(z)/(z-a)^{\ord_a(f)}$ is holomorphic at $a$.
\end{pf}


\begin{prb}[Rouch\'e theorem]
Let $f$ be a meromorphic function on $\Omega$.
\begin{parts}
\item
If $h:[0,1]\times\Omega\to\C$ is continuous, then 
\[\int_\gamma\frac{f'(z)}{f(z)}\,dz=\int_\gamma\frac{g'(z)}{g(z)}\,dz.\]
In particular, if $|g(z)|<|f(z)|$ on $z\in\gamma$, then
\[\int_\gamma\frac{f'(z)}{f(z)}\,dz=\int_\gamma\frac{f'(z)+g'(z)}{f(z)+g(z)}\,dz.\]
\end{parts}
\end{prb}





\section*{Exercises}
\begin{prb}[The second proof of the fundamental theorem of algebra]
by Rouch\'e.
\end{prb}
\begin{prb}[Laplace transforms]
\end{prb}
\begin{prb}[Gamma function]
Hankel representation
\end{prb}
\begin{prb}[Abel-Plana formula]
\end{prb}

Sokhotski-Plemelj theorem,
Kramers-Konig relations,
Titchmarsh theorem for Hilbert transform,
Phragm\'en-Lindel\"of principle,
Carlson's theorem

\section*{Problems}
\begin{enumerate}
\item We have $\int_0^{2\pi}\frac{d\theta}{1+\cos^2\theta}=\sqrt2\pi$.
\item Find the number of roots of $z^6+z+1=0$ in $\{x+iy\in\C:x>0,y>0\}$.
\item Find the number of roots of $z-e^{-z}=2$ in the right half plane.
\item If $f$ is an entire function such that $|f(z)|\le e^{|z|^\lambda}$, then $|\{z\in B(0,R):f(z)=0\}|\lesssim R^\lambda$.
\item There is no holomorphic function $f:\D\to\C$ such that $|f(z)|\to\infty$ for all sequences $z_n\in\D$ with $|z_n|\to1$.
\item If $f$ is a bounded holomorphic function defined on $\C\setminus E$, where $E\subset[0,1]$ is the Cantor set, then $f$ is constant.
\item Suppose a sequence of nowhere vanishing holomorphic functions $f_n$ on a domain $\Omega$ converges to a non-constant function $f$ uniformly on compact sets.
Then, $f$ is also nowhere vanishing. (Hurwitz)
\end{enumerate}




\chapter{Polynomial approximation}
\section{Mittag-Leffler theorem}
\begin{prb}[Compact convergence of holomorphic functions]
\begin{parts}
\item injectivity preservation: Hurwitz theorem
\end{parts}
\end{prb}

\begin{prb}[Principal part]
For a meromorphic function $f$, we say a polynomial $p$ without constant term is a \emph{principal part} of $f$ at $z_0$ if we have a partial fraction decomposition
\[f(z)=p\left(\frac1{z-z_0}\right)+h(z),\]
where $h(z)$ is holomorphic at $z_0$.
It is unique.
pre-assigned principal parts
\end{prb}

\section{Weierstrass factorization theorem}
Infinite product

\section{Runge's approximation}
Mergelyan









\part{Geometric function theory}

\chapter{Conformal mappings}
\section{Riemann sphere and open unit disk}
\begin{prb}[Conformality of holomorphic maps]
$f'\ne0$ and $f'$ satisfies the Cauchy-Riemann
\end{prb}
\begin{prb}[M\"obius transform]
generators,
fixed points
\end{prb}
\begin{prb}[Blaschke factors]
\end{prb}

\section{Riemann mapping theorem}


\begin{prb}[Normal family]
locally bounded, then compact (Montel)
\end{prb}

\begin{prb}[Schwarz lemma]
\end{prb}

\begin{prb}[Riemann mapping theorem]
Let $\Omega\subset\C$ be a simply connected domain such that $\Omega\ne\C$.
\[\cF=\{f:\Omega\to\D\mid f\text{ is injective and holomorphic, and }f(z_0)=0\}\]
\begin{parts}
\item There exists an injective holomorphic function $f:\Omega\to\D$.
\item If $0\in\Omega_1\subsetneq\D$, then there is a conformal mapping $h:\Omega_1\to\Omega_2$ such that $h(0)=0$ and $|h'(0)|>1$, where $0\in\Omega_2\subset\D$.
\item The supremum of $|f'(0)|$ is attained in $\cF$.
\item There exists a conformal mapping $f:\Omega\to\D$.
\end{parts}
\end{prb}


\section*{Exercises}
\begin{prb}[Special solution of Laplace' equation]
\end{prb}
\begin{prb}[Normal family for meromorphic functions]
\end{prb}

\section*{Problems}
\begin{enumerate}
\item Find a conformal mapping that maps the open unit disk onto $A:=\{\,z\in\C:\max\{|z|,|z-1|\}<1\,\}$.
\end{enumerate}




\chapter{Univalent functions}
\section{Bierbach conjecture}
\section{Harmonic functions}
harmonic conjugates
conformal change

\begin{prb}[Mean value property]
\[\frac1{2\pi}\int_0^{2\pi}f(re^{i\theta})(re^{i\theta})^{-k}\,d\theta
=\begin{cases}0&\text{ if }k<0\\\dfrac{f^{(k)}(0)}{k!}&\text{ if }k\ge0\end{cases}\]
for $r$ such that $f$ is defined on $\bar B_r$.
\end{prb}
\begin{prb}[Poisson kernel]
Let $f$ be a holomorphic function on the open unit disk $\D$.
If $h$ is another holomorphic function, then
\[f(a)=\frac1{2\pi}\int_{|z|=r}f(z)\left(\frac z{z-a}+zh(z)\right)\,\frac{dz}{iz}\]
for $0<r<1$.
\begin{parts}
\item Find the holomorhpic $h$ on an open neighborhood of $\D$ in terms of $a$ such that $|z|=1$ implies $\frac z{z-a}+zh(z)$ is real.
\end{parts}
\end{prb}


\section{Exercises}
\begin{prb}[Carath\'eodory class]
Let $f$ be a holomorphic function on the open unit disk $\D$ such that $\Re f(z)>0$ for $z\in\D$ and $f(0)=1$. Show that $|f'(0)|\ge2$.
\end{prb}


\chapter{}

Maximum principle; Lindelöf principle,
Nevanlinna theory?

\section{Riemann-Hilbert problem}
Hilbert transform
almost everywhere convergence, Hardy-Littlewood maximal function

\section{Quasi-conformal mappings}
Beltrami equations and Teichm\"uler theory?











\part{Riemann surfaces}

\chapter{Analytic continuation}
\section{}
Three perspectives:
We can see $\CP^1$ with $\C^2\setminus\{(0,0)\}/\sim$ and $U_0\cup U_1$ and $\C\cup\{\infty\}$.

holomorphic functions and meromorphic functions $\cO_{\P^1}=0$, $\cM_{\P^1}=\C(z)$, $\Aut(\P^1)=\PSL(2,\C)$, $\Hom(\P^1)=\C(z)\cup\{\infty\}$

transformation rule? gluing rule?

\begin{prb}[Riemann sphere]

\end{prb}



\section{Branch cuts}
We can represent $f$ with any coordinate system(usually polar coordinates).

Define $f:\{re^{i\theta}:r>0,-\pi<\theta<\pi\}\to\C$ such that
\[f(re^{i\theta}):=\log r+i\theta.\]
Then, $e^{f(z)}=z$.
Define $f:\{x+iy:y\ne0\text{ or }-1<x<1\}\to\C$ such that
\[f(z):=\frac1{\sqrt{r_+r_-}}e^{i\frac{\theta_++\theta_-}2},\]
where $z-1=r_+e^{i\theta_+}$ and $z+1=r_-e^{i\theta_-}$.
Then, $f(z)$ is a branch of $1/\sqrt{z^2-1}$.


\section{Monodromy}
\section{Covering surfaces}
\section{Algebraic functions}
\section{Elliptic curves}

\chapter{Differential forms}

\chapter{Uniformization theorem}




\part{Several complex variables}

\chapter{}

\section{}
Cousin problems:
\begin{enumerate}
\item 
\item
\item
\end{enumerate}

Four coherence theorems:
\begin{enumerate}
\item 
\item
\item
\item
\end{enumerate}


\section{}

\begin{prb}[Sheaves and \'etale spaces]
A sheaf is a presheaf(contravariant functor) satisfying locality and gluing axiom.
However, there is an \'etale space characterization by Serre.
(stalks are Noetherian, regular local, and a UFD?
See the Weierstrass preparation theorem later.)
\end{prb}

\begin{prb}[Ringed spaces]
Let $(X,\cO_X)$ be a ringed space, a topological space $X$ together with a sheaf $\cO_X$ of rings.
We will also denote by $\cO_X$ the \'etale space of the sheaf $\cO_X$.
A sheaf $\cF_X$ of $\cO_X$-modules is called \emph{locally finite} or \emph{finite type} if there is an open cover $\{U_i\}$ of $X$ together with surjective ring homomorphisms $\cO_X(U_i)\twoheadrightarrow\cF_X(U_i)$ for all $i$.
The section space $\Gamma(U,\cO_X)$ will be also denoted by $\cO_X(U)$.
\end{prb}

\begin{prb}[Analytic sets and local model spaces]
For a domain $X\subset\C^d$, we have a canonical ringed space $(X,\cO_X)$ of holomorphic functions.
For an analytic set $A\subset X$, we can consider an exact sequence of sheaves of rings
\[0\to\cI_{(X,A)}\to\cO_X\to\cO_A\to0,\]
which can be admitted as the definition of $\cO_A$.
We can distinguish this exact sequence from the two ring homomorphisms given by the sheaf theoretic restriction
\[\cO_X(U)\leftarrow\cO_X(X)\to\cO_X(A),\]
where $\cO_X(A)$ is defined as the colimit by open neighborhoods of $A$.
By taking $\cI_{(X,A)}$ as an arbitrary ideal sheaf of finite type, we may define a complex model space as the quotient sheaf $\cO_X/\cI_{(X,A)}$, in which open subsets of a complex model space, also called local model spaces, coincide what we call analytic subsets of a domain, i.e.~we can recover the subset $A$ from $\cI_{(X,A)}$.
\end{prb}

\begin{prb}[Coherent sheaves]
Consider a quasi-coherent sheaf with an exact sequence of $\cO_X$-modules
\[\cO_X^m\to\cO_X^n\to\cF_X\to0.\]
The \emph{generating system} is the basis of $\cO_X^n$, and the \emph{relation sheaf} is the kernel of $\cO_X^n\to\cF_X$.
We say $\cF_X$ is \emph{coherent} if $\cF_X$ is of finite type and the relation sheaf is also of finite type.
\end{prb}


Prop 1: For an subsheaf of an coherent sheaf is coherent if and only if it is of finite type.

Prop 2: Finite direct sum of a coherent sheaf is coherent.


\begin{thm*}[2.2.12]
Let $X$ be a domain and $A\subset X$ is closed.
If the ideal sheaf $\cI_{(X,A)}$ is of finite type if and only if $A$ is analytic.
\end{thm*}
\begin{thm*}[2.2.17]
If $S$ is smooth analytic in a domain $X$, then the ideal sheaf $\cI_{X,S}$ is coherent. (2nd Oka coherence for smooth sets)
\end{thm*}




\end{document}