\documentclass{../../large}
\usepackage{../../ikhanchoi}

\DeclareMathOperator{\Map}{Map}

\begin{document}
\title{General Topology}
\author{Ikhan Choi}
\maketitle
\tableofcontents

\part{Topological spaces}


\chapter{Topologies}

\section{Topologies}

A \emph{directed set} is a preordered set in which every finite subset has a upper bound.
A \emph{downward directed set} is a preordered set in which every finite subset has a lower bound.
A \emph{filter} is a partially ordered set which is non-empty, downward directed, upward closed.


\begin{prb}[Filters]
Let $X$ be a set.
A \emph{filter base} on $X$ is a family of subsets of $X$ that is non-empty and downward directed.
We say a filter base $\cB$ is coarser than another filter base $\cB'$ if for every $U\in\cB$ there is $U'\in\cB'$ such that $U'\subset U$.
A \emph{filter} on $X$ is a filter base on $X$ that is upward closed.
For a filter base $\cB$ on $X$, the smallest filter $\cF$ on $X$ containing $\cB$ is said to be generated by $\cB$.
\begin{parts}
\item Let $\cF$ be the filter generated by a filter base $\cB$. Then, $U\in\cF$ if and only if there is $V\subset X$ such that $V\subset U$ and $V\in\cB$.
\item A filter base is a filter if and only if it is the maximal one in its equivalence class.
\item $\cB$ is coarser than $\cB'$ if and only if $\cF\subset\cF'$.
\end{parts}
\end{prb}

\begin{prb}[Topologies]
Let $X$ be a set.
A \emph{topological base} on $X$ is a family $\cB$ of subsets of $X$ such that the restriction $\cB_x:=\{U\in\cB:x\in U\}$ is a filter base for each $x\in X$.
We say a topological base $\cB$ is coarser than another topological base $\cB'$ if $\cB_x$ is coarser than $\cB'_x$ for each $x\in X$.
A \emph{topology} on $X$ is a topological base on $X$ that is closed under union and contains an empty set.
For a topological base $\cB$ on $X$, the smallest topology $\cT$ on $X$ containing $\cB$ is said to be generated by $\cB$.
\begin{parts}
\item Let $\cT$ be the topology generated by a topological base $\cB$. Then, $U\in\cT$ if and only if $U$ is the union of elements of $\cB$.
\item A topological base is a topology if and only if it is the maximal one in its equivalence class.
\item $\cB$ is coarser than $\cB'$ if and only if $\cT\subset\cT'$.
\end{parts}
\end{prb}

\section{Open sets and closed sets}
Interior, closure, and boundary


\section*{Exercises}

\begin{prb}
If $A^\circ\subset B$ and $B$ is closed, then $(A\cup B)^\circ\subset B$.
\end{prb}




\chapter{Topological spaces}
\section{Continuous maps}

\section{Fundamental constructions}
Subspace topology
Quotient topology
Product topology




\chapter{Nets and sequences}
\section{Nets and subnets}


\section{Ultrafilters and ultranets}

A filter is a non-empty downward directed and upward closed partially ordered set.
Note that every partially ordered set can be embedded in the power set of a set.
Let $\cF$ be a filter on a set $\cA$ and $\cA\to X:\alpha\mapsto x_\alpha$ be a function.
We say the function $\alpha\mapsto x_\alpha$ converges to $x_0$ if the neighborhood filter $\cF_{x_0}$ of $x_0$ is contained in the pushforward filter of $\cF$ by $x_\alpha$.

When we consider the case that $\cA$ is a directed set, i.e.~$x_\alpha$ is a net, then since a directed set $\cA$ has a natural filter generated by the filter base $\{\{\alpha\in\cA:\alpha\ge\alpha_0\}:\alpha_0\in\cA\}$, thus every net in $X$ derives a filter on $X$ by pushforward.

\begin{prb}[Ultrafilter convergence theorem]
\begin{parts}
\item $X$ is compact if and only if every ultrafilter on $X$ converges at least one point.
\item $X$ is Hausdorff if and only if every ultrafilter on $X$ converges at most one point.
\end{parts}
\end{prb}

\begin{prb}[Non-principal ultrafilter]
\begin{parts}
\item Every filter on a set $X$ is contained in an ultrafilter.
\item Every infinite set $X$
\end{parts}
\end{prb}

\section{Eventuality filters}
\section{Sequential spaces}



\begin{prb}[Sequential spaces]
Let $X$ be a topological space.
A subset $U$ is called \emph{sequentially open} in $X$ if every sequence that converges to a point in $U$ is eventually in $U$.
A \emph{sequential space} is a topological space in which every sequentially open set is open.
\begin{parts}
\item Every open set is sequentially open.
\item
\end{parts}
\end{prb}

\begin{prb}[Sequential continuity]
Let $f:X\to Y$ be a map between topological spaces.
\begin{parts}
\item If $X$ is sequential, then the sequential continuity implies the continuity of $f$.
\end{parts}
\end{prb}
\begin{pf}
Suppose $f$ is not continuous so that there is an open set $V$ in $Y$ such that $f^{-1}(V)$ is not open.
Since $X$ is sequential, $f^{-1}(Y)$ is not sequentially open.
Take a seuqence $x_n$ in $X\setminus f^{-1}(V)$ that converges to $x\in f^{-1}(V)$.
Then, $f(x_n)$ dose not converges since $f(x_n)\notin V$ and $f(x)\in V$, so $f$ is not sequentially continuous.
\end{pf}

\begin{prb}
\end{prb}


important: existence of convergent subnet

no convergent subsequence implies the discreteness(closedness) of the sequence
convergent sequence is relatively compact
there is a convergent subnet relatively compact if $X$ is locally compact
every infinite set is cofinal


if every subnet is a sequence, then the net is a sequence?
% 의미 있는 문젠가?


\part{Topological properties}
\chapter{Compactness}
\section{Compact spaces}

Does not depend on the ambient spaces
Tube lemma



A topological space is called $\sigma$-compact if it is the countable inductive limit of compact sets.
A topological space is called hemi-compact if the direct set of non-empty compact subsets has a cofinal sequence.

\section{Sequential compactness}
\section{Local compactness}
\section{Tychonoff theorem}
\section*{Exercises}
\begin{prb}
$\Q$ is $\sigma$-compact but not hemicompact.
\end{prb}
\chapter{Connectedness}
\section{Connected spaces}
\section{Path connectedness}
\section{Local connectedness}



\chapter{Separation axioms}
\section{Separation axioms}
subspace and product


\section{Normal spaces}
not for product

compact Hausdorff
order topology
metric topology?

\section{Urysohn lemma and Tietze extension}



\section*{Exercises}
\begin{prb}[Sorgenfrey's plane]
\end{prb}
\begin{prb}[Tychonoff plank]
\end{prb}
\begin{prb}[Moore plane]
\end{prb}
\begin{prb}[Rational sequence topology]
\end{prb}
\begin{prb}[Mr\'owka space]
\end{prb}



\part{Topological structures}

\chapter{Metric spaces}

\section{Equivalence of metrics}
\begin{prb}[Comparison of metrics]
Let $d_1$ and $d_2$ be metrics on a set $X$.
We say that $d_1$ is \emph{stronger than} $d_2$ (equivalently, $d_2$ is \emph{weaker than} $d_1$) or $d_1$ \emph{refines} $d_2$ if for any $x\in X$ and $\e>0$, there is $\delta>0$ such that
\[B_1(x,\delta)\subset B_2(x,\e),\]
where $B_1$ and $B_2$ refer to balls within the metrics $d_1$ and $d_2$ respectively.
\begin{parts}
\item
This refinement relation is a preorder.
\item
$d_1$ is stronger than $d_2$ if and only if every sequence that converges to $x\in X$ in $d_1$ converges to $x$ in $d_2$.
\item
$d_1$ is stronger than $d_2$ if and only if the identity map $\id:(X,d_1)\to(X,d_2)$ is continuous.
\end{parts}
\end{prb}
\begin{pf}
(a)
It is enough to show the transitivity.
Suppose there are three metric $d_1$, $d_2$, and $d_3$ on a set $X$ such that $d_1$ is stronger than $d_2$ and $d_2$ is stronger than $d_3$.
For $i=1,2,3$, let $B_i$ be a notation for the balls defined with the metric $d_i$.

Take $x\in X$ and $\e>0$ arbitrarily.
Then, we can find $\e'>0$ such that
\[B_2(x,\e')\subset B_3(x,\e).\]
Also, we can find $\delta>0$ such that
\[B_1(x,\delta)\subset B_2(x,\e').\]
Therefore, we have $B_1(x,\delta)\subset B_3(x,\e)$ which implies that $d_1$ refines $d_3$.

(b)
($\Rightarrow$)
Let $\{x_n\}_n$ be a sequence in $X$ that converges to $x$ in $d_1$.
By the assumption, for an arbitrary ball $B_2(x,\e)=\{y:d_2(x,y)<\e\}$, there is $\delta>0$ such that
\[B_1(x,\delta)\subset B_2(x,\e),\]
where $B_1(x,\delta)=\{y:d_1(x,y)<\delta\}$.
Since $\{x_n\}_n$ converges to $x$ in $d_1$, there is an integer $n_0$ such that
\[n>n_0\impl x_n\in B_1(x,\delta).\]
Combining them, we obtain an integer $n_0$ such that
\[n>n_0\impl x_n\in B_2(x,\e).\]
It means $\{x_n\}$ converges to $x$ in the metric $d_2$.

($\Leftarrow$)
We prove it by contradiction.
Assume that for some point $x\in X$ we can find $\e_0>0$ such that there is no $\delta>0$ satisfying $B_1(x,\delta)\subset B_2(x,\e_0)$.
In other words, at the point $x$, the difference set $B_1(x,\delta)\setminus B_2(x,\e_0)$ is not empty for every $\delta>0$.
Thus, we can choose $x_n$ to be a point such that
\[x_n\in B_1\left(x,\tfrac1n\right)\setminus B_2(x,\e_0)\]
for each positive integer $n$ by putting $\delta=\frac1n$.

We claim $\{x_n\}_n$ converges to $x$ in $d_1$ but not in $d_2$.
For $\e>0$, if we let $n_0=\ceil{\frac1\e}$ so that we have $\frac1{n_0}\le\e$, then
\[n>n_0\impl x_n\in B_1\left(x,\tfrac1n\right)\subset B_1(x,\e).\]
So $\{x_n\}_n$ converges to $x$ in $d_1$.
However in $d_2$, for $\e=\e_0$, we can find such $n_0$ like $d_1$ since
\[x_n\notin B_2(x,\e_0)\]
for every $n$.
Therefore, $\{x_n\}$ does not converges to $x$ in $d_2$.
\end{pf}

\begin{prb}[Equivalence of metrics]
Let $d_1$ and $d_2$ be metrics on a set $X$.
They are said to be \emph{(topologically) equivalent} if they refines each other.
\begin{parts}
\item
$d_1$ and $d_2$ are equivalent if for each $x$ in $X$ there exist two constants $C_1$ and $C_2$ such that $d_2(x,y)\le C_1d_1(x,y)$ and $d_1(x,y)\le C_2d_2(x,y)$ for all $y$ in $X$.
\item
$d_1$ and $d_2$ are equivalent if $d_2=f\circ d_1$ for a monotonically increasing $f:\R_{\ge0}\to\R_{\ge0}$ that is continuous at $0$.
\end{parts}
\end{prb}
\begin{pf}
(a)
Let $d_1$ and $d_2$ be metrics on a set $X$.
Suppose for each point $x$ there exists a constant $C$ which may depend on $x$ such that $d_2(x,y)\le Cd_1(x,y)$ for all $y\in Y$.
We will show $d_1$ is stronger than $d_2$.

(b)
For any ball $B_1(x,\e)$, we have a smaller ball
\[B_2(x,f(\e))\subset B_1(x,\e)\]
since $f(d(x,y))<f(\e)$ implies $d(x,y)<\e$.
Conversely, take an arbitrary ball $B_2(x,\e)$.
Since $f$ is continuous at 0, we can find $\delta>0$ such that
\[d(x,y)<\delta\impl f(d(x,y))<\e,\]
which implies $B_1(x,\delta)\subset B_2(x,\e)$.
\end{pf}






A metric can be viewed as a function that takes a sequence as input and returns whether the sequence converges or diverges.
That is, a metric acts like a criterion which decides convergence of sequences.
Take note on the fact that the sequence of real numbers defined by $x_n=\frac1n$ converges in standard metric but diverges in discrete metric.
Like this example, even for the same sequence on a same set, the convergence depends on the attached metrics.
What we are interested in is comparison of metrics and to find a proper relation structure.
If a sequence converges in a metric $d_2$ but diverges in another metric $d_1$, we would say $d_1$ has stronger rules to decide the convergence.
Refinement relation formalizes the idea.

Unlike metrics, there exist two different topologies that have same sequential convergence data.
For example, a sequence in an uncountable set with cocountable topology converges to a point if and only if it is eventually at the point, which is same with discrete topology.
This means the informations of sequence convergence are not sufficient to uniquely characterize a topology.
Instead, convergence data of generalized sequences also called nets, recover the whole topology.
For topologies having a property called the first countability, it is enough to consider only usual sequences in spite of nets.
What we did in this subsection is not useless because topology induced from metric is a typical example of first countable topologies.
These kinds of problems will be profoundly treated in Chapter 3.

One can ask some results for the equivalence of metrics characterized by a same set of continuous functions.
However, they are generally difficult problems: is it possible to recover the base space from a continuous function space or a path space?


Topologies are occasionally described by not a single but several metrics.
It provides a useful method to construct a metric or topology, which can be applied to a quite wide range of applications.
Specifically, in a conventional way, metrics can be summed or taken maximum to make another metric out of olds.
The following proposition can be easily generalized to an arbitrary finite number of metrics by mathematical induction.

\begin{prop}
Let $d_1$ and $d_2$ be metrics on a set $X$.
For a sequecne $\{x_n\}_n$ in $X$, the following statements are all equivalent:
\begin{parts}
\item it converges to $x$ in both $d_1$ and $d_2$,
\item it converges to $x$ in $d_1+d_2$,
\item it converges to $x$ in $\max\{d_1,d_2\}$.
\end{parts}
In particular, the metrics $d_1+d_2$ and $\max\{d_1,d_2\}$ are equivalent.
\end{prop}
\begin{pf}
We skip to prove $d_1+d_2$ and $\max\{d_1,d_2\}$ are metrics.

(b) or (c)$\Rightarrow$(a)
The inequalities $d_i\le d_1+d_2$ and $d_i\le\max\{d_1,d_2\}$ imply the desired results.

(a)$\Rightarrow$(b)
For $\e>0$, we may find positive integers $n_1$ and $n_2$ such that $n>n_1$ and $n>n_2$ imply $d_1(x_n,x)<\frac\e2$ and $d_2(x_n,x)<\frac\e2$ respectively.
If we define $n_0:=\max\{n_1,n_2\}$, then
\[n>n_0\impl d_1(x_n,x)+d_2(x_n,x)<\e.\]

(a)$\Rightarrow$(c)
Take $n_0$ as we did previously.
Then,
\[n>n_0\impl\max\{d_1(x_n,x),d_2(x_n,x)\}<\frac\e2<\e.\qedhere\]
\end{pf}

\begin{rmk}
In general, for any norm $\|\cdot\|$ on $\R^2$, the function $\|(d_1,d_2)\|$ defines another equivalent metric.
\end{rmk}

There is also a method for combining not only finite family of metrics, but also infinite family of metrics.
Since the sum of infinitely many positive numbers may diverges to infinity, we cannot sum the metrics directly.
The strategy is to ``bound'' the metrics.
We call a metric bounded when the range of metric function is bounded.

\begin{prop}
Every metric possesses an equivalent bounded metric.
\end{prop}
\begin{pf}
Let $d$ be a metric on a set.
Let $f$ be a bounded, monotonically increasing, and subadditive function on $\R_{\ge0}$ that is continuous at 0 and satisfies $f^{-1}(0)=\{0\}$.
The mostly used examples are
\[f(x)=\frac x{1+x}\quad\text{and}\quad f(x)=\min\{x,1\}.\]
Then, $f\circ d$ is a bounded metric equivalent to $d$ by Example 1.4.
\end{pf}

\begin{defn}
Let $d$ be a metric on a set $X$.
A \emph{standard bounded metric} means either metric
\[\min\{d,1\}\quad\text{or}\quad\frac d{d+1},\]
and we will denote it by $\hat d$.
\end{defn}

\begin{prop}
Let $\{d_i\}_{i\in\N}$ be a countable family of metrics on a set $X$.
For a sequence $\{x_n\}_n$ in $X$, the following statements are all equivalent:
\begin{parts}
\item it converges in $d_i$ for every $i$,
\item it converges in a metric
\[d(x,y):=\sum_{i\in\N}\,2^{-i}\hat d_i(x,y),\]
\item it converges in a metric
\[d'(x,y):=\sup_{i\in\N}\,i^{-1}\hat d_i(x,y).\]
\end{parts}
In particular, the metrics $d$ and $d'$ are equivalent.
\end{prop}
\begin{pf}
The functions $d$ and $d'$ in (b) and (c) are well-defined by the monotone convergence theorem and the least upper bound property.
We skip checking for them to satisfy the triangle inequality and be metrics.

(b) or (c)$\Rightarrow$(a)
We have ineuqalities $\hat d_i\le2^id$ and $\hat d_i\le id'$ for each $i$, so convergence in $d$ or $d'$ implies the convergence in each $\hat d_i$.
The equivalence of $\hat d_i$ and $d_i$ implies the desired result.

(a)$\Rightarrow$(b)
Suppose a sequence $\{x_n\}_n$ converges to a point $x$ in $d_i$ for every index $i$.
Take an arbitrary small ball $B(x,\e)=\{y:d(x,y)<\e\}$ with metric $d$.
By the assumption, we can find $n_i$ for each $i$ satisfying
\[n>n_i\impl\hat d_i(x_n,x)<\frac\e2.\]
Define $k:=\lceil1-\log_2\e\rceil$ so that we have $2^{-k}\le\frac\e2$.
With this $k$, define
\[n_0:=\max_{1\le i\le k}n_i.\]
If $n>n_0$, then
\begin{align*}
d(x_n,x)&=\sum_{i=1}^k2^{-i}\hat d_i(x_n,x)+\sum_{i=k+1}^\infty2^{-i}\hat d_i(x_n,x)\\
&<\sum_{i=1}^k2^{-i}\frac\e2+\sum_{i=k+1}^\infty2^{-i}\\
&<\frac\e2+2^{-k}\le\e,
\end{align*}
so $x_n$ converges to $x$ in the metric $d$.

(a)$\Rightarrow$(c)
Suppose a sequence $\{x_n\}_n$ converges to a point $x$ in each $d_i$, and take an arbitrary small ball $B(x,\e)=\{y:d(x,y)<\e\}$ with metric $d$.
By the assumption, we can find $n_i$ for each $i$ satisfying
\[n>n_i\impl\hat d_i(x_n,x)<\e.\]
Define $k:=\lceil\frac1\e\rceil$ so that we have $k^{-1}\le\e$.
With this $k$, define
\[n_0:=\max_{1\le i\le k}n_i.\]
If $n>n_0$, then
\[i^{-1}\hat d_i(x,y)\le\hat d_i(x,y)<\e\quad\text{for}\quad i\le k\]
and
\[i^{-1}\hat d_i(x,y)\le i^{-1}<k^{-1}\le\e\quad\text{for}\quad i>k\]
imply $d(x_n,x)<\e$, which means that $x_n$ converges to $x$ in the metric $d$.
\end{pf}

Combination of uncountably many metrics does not result in a single metric, but a topology which cannot be induced from a metric in general.
It will be discussed in the rest of the note.

\begin{rmk}
A metric
\[d''(x,y)=\sup_{i\in\N}\,d_i(x,y)\]
is not used because the convergence in this metric is a stronger condition than the convergence with respect to each metric $d_i$.
In other words, this metric generates a finer(stronger) topology than the topology generated by subbase of balls.
For example, the topology on $\R^\N$ generated by this metric defined with the projection pseudometrics is exactly what we often call the box topology.
\end{rmk}


We can also form a metric by summation of generalized metrics, called pseudometrics, which is defined by missing the nondegeneracy condition from the original definition of metric.

\begin{defn}
A function $\rho:X\times X\to\R_{\ge0}$ is called a \emph{pseudometric} if
\begin{parts}
\item $\rho(x,x)=0$ for all $x\in X$,
\item $\rho(x,y)=\rho(y,x)$ for all $x,y\in X$, \hfill(symmetry)
\item $\rho(x,z)\le \rho(x,y)+\rho(y,z)$ for all $x,y,z\in X$. \hfill(triangle inequality)
\end{parts}
\end{defn}

For pseudometrics, it is possible to duplicate every definition we studied in metric spaces: convergence of a sequence, continuity between a set endowed with a pseudometric, refinement and equivalence relations, and countable sum of bounded pseudometrics to make a new pseudometric.
Furthermore, every statement for metrics can be generalized to pseudometrics since we have not actually used the condition that $d(x,y)=0$ implies $x=y$.
In fact, we have a flaw that the limit of a convergent sequence may not be unique within a pseudometric.

\begin{ex}
Let $\rho(x,y)=\rho((x_1,x_2),(y_1,y_2))=|x_1-y_1|$ be a pseudometric on $\R^2$.
Consider a sequence $\{(\frac1n,0)\}_n$.
Since $(0,c)$ satisfies 
\[\rho((\tfrac1n,0),(0,c))=\tfrac1n\to0\quad\text{as}\quad n\to\infty\]
for any real number $c$, the sequence converges to $(x_1,x_2)$ if and only if $x_1=0$.
\end{ex}

Although sequences may have several limits in each pseudometric, the sum of a family of pseudometrics can allow the sequences to have at most one limit, only if the sum satisfies the axioms of a metric.

\begin{defn}
A family of pseudometrics $\{\rho_\alpha\}_\alpha$ on a set $X$ is said to \emph{separate points} if the condition 
\[\rho_\alpha(x,y)=0\quad\text{for all $\alpha$}\]
implies $x=y$.
\end{defn}
\begin{prop}
\begin{parts}
\item
A finite family of pseudometrics $\{\rho_i\}_{i=1}^N$ separates points if and only if the pseudometric $\rho:=\sum_{i=1}^N\rho_i$ is a metric.
\item
A countable family of pseudometrics $\{\rho_i\}_{i\in\N}$ separates points if and only if the pseudometric defined by
\[\rho:=\sum_{i\in\N}2^{-i}\tilde\rho_i\quad\text{or}\quad\sup_{i\in\N}i^{-1}\tilde\rho_i,\]
where $\tilde\rho_i$ is either $\min\{\rho_i,1\}$ or $\rho_i/(\rho_i+1)$, is a metric.
\end{parts}
\end{prop}





\section{Metrization theorems}

Urysohn metrization theorem: regular second countable

Paracompactness
Dieudonne's theorem: paracompact Hausdorff is normal.


\section{Polish spaces}
\begin{prb}[Embedding of separable metric spaces]
Urysohn's embedding
Hilbert cube
\end{prb}





\section*{Exercises}

\begin{prb}[Discrete metrics]
Let $X$ be a set, and define a metric as
\[d(x,y):=\begin{cases}0&,x=y\\1&,x\ne y\end{cases}.\]
This metric is called \emph{discrete}.
\begin{parts}
\item
The discrete metric is a strongest metric on $X$.
\item 
\item
A sequence $x_n$ converges to a point $x$ in a discrete metric space if and only if there is $n_0$ such that $n\ge n_0$ implies $x_n=x$.
\end{parts}
\end{prb}

\begin{prb}
\begin{parts}
\item
Let $d$ be a metric on a set $X$.
Let $f:\R_{\ge0}\to\R_{\ge0}$ be a function such that $f^{-1}(0)=\{0\}$.
If $f$ is monotonically increasing and subadditive, then $f\circ d$ satisfies the triangle inequality, hence is another metric on $X$.
Note that a function $f$ is called \emph{subadditive} if $f(x+y)\le f(x)+f(y)$ for all $x,y$ in the domain.
\item
Let $G=(V,E)$ be a connected graph.
Define $d:V\times V\to\Z_{\ge0}\subset\R_{\ge0}$ as the distance of two vertices; the length of shortest path connecting two vertices.
Then, $(V,d)$ is a metric space.
\item
Let $\cP(X)$ be the power set of a finite set $X$.
Define $d:\cP(X)\times\cP(X)\to\Z_{\ge0}\subset\R_{\ge0}$ as the cardinality of the symmetric difference; $d(A,B):=|(A-B)\cup(B-A)|$.
Then $(\cP(X),d)$ is a metric space.
\item
Let $C$ be the set of all compact subsets of $\R^d$.
Recall that a subset of $\R^d$ is compact if and only if it is closed and bounded.
Then, $d:C\times C\to\R_{\ge0}$ defined by
\[d(A,B):=\max\{\,\sup_{a\in A}\inf_{b\in B}\|a-b\|,\ \sup_{b\in B}\inf_{a\in A}\|a-b\|\,\}\]
is a metric on $C$.
It is a little special case of \emph{Hausdorff metric}.
\end{parts}
\end{prb}


\begin{prb}[Kuratowski embedding]
While every subset of a normed space is a metric space, we have a converse statement that every metric space is in fact realized as a subset of a normed space.
Let $X$ be a metric space, and denote by $C_b(X)$ the space of continuous and bounded real-valued functions on $X$ with uniform norm given by
\[\|f\|=\sup_{x\in X}|f(x)|.\]
Fix a point $p\in X$, which will serve as the origin.
\begin{parts}
\item Show that a map $\phi:X\to C_b(X)$ such that
\[[\phi(x)](t)=d(x,t)-d(p,t)\]
is well-defined.
\item Show that the map $\phi$ is an isometry; $d(x,y)=\|\phi(x)-\phi(y)\|$.
\end{parts}
\end{prb}

\begin{prb}[Equivalence of norms in finite dimension]
Let $V$ be a vector space of dimension $d$ over $\F=\R$ or $\C$.
Fix a basis $\{e_i\}_{i=1}^d$ on $V$ and let $x=\sum_{i=1}^dx_ie_i$ denote an arbitrary element of $V$.
We will prove all norms are equivalent to the standard Euclidean norm defined for this fixed basis:
\[\|x\|_2:=(\sum_{i=1}^d|x_i|^2)^{\frac12}.\]
With this standard norm any theorems studied in elementary analysis including the Bolzano-Weierstrass theorem are allowed to be applied.
Take a norm $\|\cdot\|$ on $V$ which may differ to $\|\cdot\|_2$.
\begin{parts}
\item Find a constant $C_2$ such that $\|x\|\le C_2\|x\|_2$ for all $x\in V$.
\item Show that if no constant $C$ satisfies $\|x\|_2\le C\|x\|$, then there exists a sequence $\{x_n\}_n$ such that $\|x_n\|_2=1$ and $\|x_n\|<\tfrac1n$.
\item Show that there exists a constant $C$ satisfies $\|x\|_2\le C\|x\|$.
\end{parts}
\end{prb}


\chapter{Uniform spaces}

\section{Uniformity}

\begin{prb}[Uniformity and entourages]
A uniformity is a generalization of a pseudometric.
An \emph{uniformity} on a topological space $X$ is a filter $\Phi$ on $X\times X$ such that
\begin{enumerate}[(i)]
\item if $E\in\Phi$, there is $F\in\Phi$ such that $\{(x,z):(x,y),(y,z)\in E\}\in\Phi$, \hfill(Triangle inequality)
\item if $E\in\Phi$, then $\{(y,x):(x,y)\in E\}\in\Phi$, \hfill(Symmetry)
\item if $E\in\Phi$, then $(x,x)\in E$ for every $x\in X$.
\end{enumerate}
For a uniformity $\Phi$ on $X$, an element of $\Phi$ is called an \emph{entourage}.
\begin{parts}
\item a
\end{parts}
\end{prb}




\section{Completely regular spaces}
uniformizability
Stone-\v Cech compactification

\section{Cauchy structure}
completion
compact uniform space is complete
relatively compact

\section*{Exercises}
\begin{prb}
Even for a first countable uniform space, completeness and sequential completeness are not equivalent.
\end{prb}
\begin{prb}
A uniformity is pseudometrizable if and only if it admits a countable fundamental system of entourages.
\end{prb}



\chapter{}





\part{Category of topological spaces}



\chapter{Compact-open topology}

\section{}

\begin{prb}[Compact-open topology]
Let $X$ and $Y$ be topological spaces.
The \emph{compact-open topology} is a topology on $C(X,Y)$ generated by
\[V(K,U):=\{f\in C(X,Y):f(K)\subset U\}\]
for compact $K\subset X$ and open $U\in Y$.
Suppose the function space $C(X,Y)$ carries the compact-open topology.
\end{prb}


\section{Compactly generated spaces}


\begin{prb}
Let $X,Y$, and $Z$ be topological spaces.
Consider the \emph{transpose} or the \emph{adjoint}
\[\alpha:C(X\times Y,Z)\to C(X,C(Y,Z)),\qquad(\alpha(f)(x))(y):=f(x,y).\]
Suppose all function spaces are endowed with the compact-open topology.
\begin{parts}
\item $\alpha$ is well-defined and injective.
\item $\alpha$ is continuous if $X$ is Hausdorff.
\item $\alpha$ is surjective if $Y$ is locally compact.
\item $\alpha$ is an embedding if $X$ and $Y$ are Hausdorff.
\item The evaluation $C(Y,Z)\times Y\to Z$ is continuous if $Y$ is locally compact.
\item The composition $C(X,Y)\times C(Y,Z)\to C(X,Z)$ is continuous if $X$ and $Y$ are locally compact.
\end{parts}
\end{prb}
\begin{pf}
(a)
Suppose $f\in C(X\times Y,Z)$.
We need to check the continuity of the function
\[\alpha(f)(x):Y\to Z:y\mapsto f(x,y)\]
for each $x\in X$, which is trivial, and the continuity of the function $\alpha(f):X\to C(Y,Z)$.
Take any compact $K\subset Y$ and open $W\subset Z$.
The inverse image of $V(K,W)$ under $\alpha(f)$ is the set
\[S:=\{x\in X:f(x,K)\subset W\}=\{x\in X:\{x\}\times K\subset f^{-1}(W)\}.\]
Pick any $x\in S$ and apply the tube lemma for the slice $\{x\}\times K$ to obtain an open neighborhood $U$ of $x$ such that $U\times K\subset S$, which implies $S$ is open.
The injectivity is clear.

(b)
Take compact $K\subset X$, compact $L\subset Y$, and open $W\subset Z$.
Then,
\[\alpha^{-1}(V(K,V(L,W)))=\{f\in C(X\times Y,Z):\forall x\in K\ \forall y\in L,\ f(x,y)\in W\}=V(K\times L,W)\]
is open.
Since $V(K,-)$ preserves the intersection, for any base element $B\subset C(Y,Z)$ the set $\alpha^{-1}(V(K,B))$ is also open.

(c)

\end{pf}


\begin{prb}[Exponential spaces]
Let $Y$ and $Z$ be topological spaces and consider a contravariant functor $C(-\times Y,Z):\mathbf{Top}^{\mathrm{op}}\to\mathbf{Set}$.
An \emph{exponential space} of $Y$ and $Z$ is a topological space $Z^Y$ such that there is a natural isomorphism $C(-\times Y,Z)\cong C(-,Z^Y)$.
We say a topological space $Y$ is \emph{exponentiable} if the exponential space $Z^Y$ exists for every topological space $Z$.
Suppose $Y$ is Hausdorff.
\begin{parts}
\item
$Z^Y$ exists if and only if the evaluation map $Z^Y\times Y\to Z$ is continuous.
\item
If $Y$ is locally compact, then $Y$ is exponentiable.
\item
If $Y$ is exponentiable, then $Y$ is locally compact.
Moreover, $Z^Y$ is homeomorphic to $C(Y,Z)$ with the compact-open topology for every topological space $Z$.
\end{parts}
\end{prb}
\begin{pf}
(a)
We claim $C(Y,Z)$ is the exponential space of topological spaces $Y$ and $Z$.

\end{pf}


\begin{prb}[Topology of compact convergence]
Let $X$ be a topological space and $Y$ be a topological and a uniform spaces.
\end{prb}

Topologies on $C_c(X)$ for LCH $X$: weaker to stronger
\begin{parts}
\item Topology of compact convergence: $\bar{C_c(X)}=C(X)$.
\item Topology of uniform convergence: $\bar{C_c(X)}=C_0(X)$.
\item Inductive topology: $\bar{C_c(X)}=C_c(X)$.
\end{parts}


\iffalse
Z^Y가 존재 iff Z^Y=C(Y,Z) as sets and there is a topology on C(Y,Z) such that the usual eval:C(Y,Z)xY→Z is continuous

Z^Y가 존재한다 가정하자. 그러면 보편원소의 정의에 따라

연속 f: XxY→Z가 있을 때마다

연속 g:X→Z^Y가 유일하게 존재해서

연속 e:Z^YxY→Z를 팩터 스루

X가 원포인트인 경우에 대해 위의 것을 적용하면

임의의 f:Y→Z에 대해 Z^Y의 원소 h가 유일하게 존재해서

모든 y에 대해 f(y)=e(h,y) 성립.

즉 set으로서의 함수 s:C(Y,Z)→Z^Y

s.t. f(y)=e(s(f),y) (*)이고 유일성에 의해

만약 f(y)=e(h,y)이면 h=s(f).

(*)에 따라 s는 단사.

아무 h in Z^Y를 뽑았을 때 e(h,y)를 잡으면 s는 전사

즉 Z^Y=C(Y,Z) as sets, e는 eval

e는 연속이어야 하므로 C(Y,Z)에는 충분히 위상이 있어야 한다

그 중 가장 약한 토폴로지가 존재해야 한다, minimum이 attain되어야 한다

드디어 eval이 연속이게 하는 조건(Y,Z,말고도 위상도)을 묻는 문제가 되었다.

요네다로 아예 자연동형인 걸 가정했을 때 이렇게 됐는데

자연변환이 좀더 약한 조건을 가진다면 어떻게 되나?

Aut(C(X))=Homeo(X) 왼쪽 ptnorm 오른쪽 co?

\fi

\chapter{Topological groups}

\section{Topological properties of groups}
left and right uniform structures
Birkhoff-Kakutani
metrizability

\section{Transformation groups}

\begin{prb}
Let $G$ be a topological group.
A \emph{left $G$-space} is a topological space $X$ together with a \emph{continuous left action} of $G$, a continuous map $G\times X\to X:(g,x)\mapsto gx$ such that $g(hx)=(gh)x$ and $ex=x$.
Equivalently we sometimes say $G$ is a \emph{transformation group} of $X$.
We may define right actions and right $G$-spaces similarly.
The left orbit space $G\backslash X$ is topologized with the quotient topology induced by the canonical surjection $X\to G\backslash X$.
We frequently write $X/G$ for the left orbit space if there is no confusion on the left-right issues.

$G$-equivariant map and $G$-homeomorphism

$G$-homotopy is a $G$-equivariant map $h:X\times[0,1]\to Y$ such that $h(x,i)=f_i(x)$ for $i\in\{0,1\}$.
\end{prb}

\begin{prb}
Suppose $\rho:G\times X\to X$ is a continuous left action, and $\bar\rho:G\to\Homeo(X)$ is its adjoint map.
Here we suppose $\Homeo(X)$ carries the compact-open topology.
\begin{parts}
\item The continuity of $\rho$ implies the continuity of $\bar\rho$, but the converse holds if $X$ is locally compact.
\item If $X$ is locally compact, then $\Homeo(X)\times X\to X$ is continuous.
\item If $X$ is locally compact, $\Homeo(X)\times\Homeo(X)\to\Homeo(X):(f,g)\mapsto g\circ f$ is continuous.
\item If $X$ is locally compact, locally connected, and Hausdorff, then $\Homeo(X)\to\Homeo(X):f\mapsto f^{-1}$ is continuous, hence $\Homeo(X)$ is a topological group.
\end{parts}
\end{prb}


The isometry group of a locally compact metric space with finite components is locally compact.


\begin{prb}
Let $H$ be a subgroup of a locally compact group $G$.
Then, $H$ is closed if and only if $H$ is locally compact.
(We always assume a topological group is Hausdorff.)
\end{prb}

\begin{itemize}
\item free action of a discrete group: the theory of covering spaces
\item transitive action: the theory of homogeneous spaces
\end{itemize}

classification of homogeneous space realization of spheres by compact connected Lie groups: Montgomery-Samuelson, Borel.





\begin{prb}[Proper actions]
\end{prb}




\chapter{}
























\newpage
\section*{Implications}

\begin{prb}[Metrizable spaces (1)]
Let $X$ be a regular Hausdorff space.
\begin{parts}
\item (Urysohn) second countable $\Rightarrow$ metrizable.
\item (Nagata-Smirnov) $\sigma$-locally countable base $\Leftrightarrow$ metrizable.
\item (Smirnov) paracompact locally metrizable $\Leftrightarrow$ metrizable.
\end{parts}
\end{prb}


\begin{prb}[Metrizable spaces (2)]
Let $X$ be a metrizable space.
\begin{parts}
\item (countability) separable $\Leftrightarrow$ second countable $\Leftrightarrow$ Lindel\"of.\\
(cf.~separable $\Leftarrow$ second countable $\Rightarrow$ Lindel\"of in general)
\end{parts}
\end{prb}


\begin{prb}[Locally compact spaces (2)]
Let $X$ be a locally compact Hausdorff space.
\begin{parts}
\item completely regular.
\item metrizable $\Leftrightarrow$ completely metrizable.
\item $\sigma$-compact $\Leftrightarrow$ Lindel\"of.\\
(cf.~$\sigma$-compact $\Rightarrow$ Lindel\"of in general)
\item $\sigma$-compact metrizable $\Leftrightarrow$ $C_0(X)$ is separable $\Leftrightarrow$ second countable $\Leftrightarrow$ Polish.
\end{parts}
\end{prb}

For a compact Hausdorff space $X$, $C(X)$ is separable if and only if $X$ is metrizable.
For a locally compact Hausdorff space $X$, $C_0(X)$ is separable if and only if $X$ is second countable.


\begin{prb}[Paracompact spaces (1)]
Let $X$ be a topological space.
\begin{parts}
\item metrizable $\Rightarrow$ fully normal $\Leftrightarrow$ paracompact. (Stone)
\item regular Lindel\"of $\Rightarrow$ paracompact.
\end{parts}
\end{prb}

\begin{prb}[Paracompact spaces (2)]
Let $X$ be a paracompact Hausdorff space.
\begin{parts}
\item normal. (Dieudonn\'e)
\item
\end{parts}
\end{prb}



\end{document}