\documentclass{../../large}
\usepackage{../../ikhanchoi}


\begin{document}
\title{Real Analysis}
\author{Ikhan Choi}
\maketitle
\tableofcontents

\part{Measur theory}


\chapter{Measure spaces}


\section{Measure spaces}



\begin{prb}[Boolean algebras and lattices]
A \emph{Boolean ring} is a possibly non-unital ring $\cR$ such that every element is idempotent, and a \emph{Boolean algebra} is a unital Boolean ring.
A \emph{Boolean lattice} is a complemented distributive lattice.
\begin{parts}
\item A Boolean ring is automatically commutative.
\item There is a one-to-one correspondence between boolean rings and boolean lattices.
\item a unital ring homomorphism is monotone.
\item a monotone bijection is a ring isomorphism.
homomorphism extension
\end{parts}
\end{prb}


For projection lattices:
- bounded lattices
- complemented lattices
- orthocomplemented lattices
- orthomodular lattices
- complete orthomodular lattices <- vN
- complete complemented distributive lattices
	= complete Boolean algebras <- CvN

For self-adjoint operators:
- ovs = directed partially ordered vector space <- C*
- finitely complete ovs
	= Riesz space <- CC*
- directed Dedekind complete ovs <- vN
- Dedekind complete ovs
	= Dedekind complete Riesz space <- CvN



(Dedekind means bounded here)


\begin{prb}[Complete lattices]
A partially ordered set is called
\begin{enumerate}[(i)]
\item \emph{finitely complete} if every non-empty finite subset admits the supremum and infimum,
\item \emph{$\sigma$-complete} if every non-empty countable subset admits the supremum and infimum,
\item \emph{complete} if every non-empty subset admits the supremum and the infimum,
\end{enumerate}
A \emph{lattice} is a finitely complete partially ordered set.
\begin{parts}
\item 
\end{parts}
\end{prb}
\begin{pf}

A directed complete, filtered complete, finitely complete partially ordered set is complete:

Suppose $\cA$ is directed complete, filtered complete, finitely complete partially ordered set. 
Let $\{a_i\}_{i\in I}$ be a family of elements of $\cA$.
Consider the directed set $\{\sup_{i\in J}a_i\}_{J\subset_{\mathrm{fin}}I}$, where $J$ runs through all finite subsets of $I$.
Since $\cA$ is directed complete, we can take $a$ the supremum of $\{\sup_{i\in J}a_i\}$.
We claim that $a$ is also the supremum of $\{a_i\}$.
By choosing $J=\{i\}$ for each $i$, we can see $a$ is an upper bound of $\{a_i\}$.
If $x$ is another upper bound of $\{a_i\}$, then $x$ is also an upper bound of $\{a_J\}$, so $a\le x$, the claim follows.
Similarly, $\{a_i\}$ also has the infimum, so $\cA$ is a complete lattice.

Now we prove $\cA$ is a Boolean algebra....
(Really?)
\end{pf}




\begin{prb}[Measurable spaces]
A \emph{measurable space} or a \emph{Borel space} is a set $X$ together with a $\sigma$-algebra $\cM$ on $X$, whose elements are called \emph{measurable}.
An \emph{enhanced measurable space} is a measurable space $(X,\cM)$ together with another $\sigma$-algebra $\cN$ on $X$ which is an ideal of $\cM$, whose elements are called \emph{negligible}.
The quotient Boolean algebra $\cM/\cN$ of an enhanced measurable space is called the \emph{measurable algebra} of the enhanced measurable space.
\begin{parts}
\item generated by a set.
\item countable and cocountable sets
\item Borel
\item Loomis-Sikorski representation
\end{parts}
\end{prb}
\begin{pf}

\end{pf}


\begin{prb}[Measure spaces]
A \emph{measure space} is a measurable space $(X,\cM)$ together with a \emph{measure}, which is defined as a set function $\mu:\cM\to[0,\infty]:\varnothing\mapsto0$ that is \emph{countably additive} in the sense that
\[\mu\Bigl(\bigsqcup_{n=1}^\infty E_n\Bigr)=\sum_{n=1}^\infty\mu(E_n),\qquad E_n\in\cM.\]
Here the squared cup notation reads the disjoint union.
If $(X,\cM,\mu)$ be a measure space, then the set $\cN:=\{N\in\cM:\mu(N)=0\}$ canonically defines an enhanced mesurable space $(X,\cM,\cN)$.
\begin{parts}
\item $\mu$ is monotone: if $E_1\subset E_2$ then $\mu(E_1)\le\mu(E_2)$ for $E_1,E_2\in\cM$.
\item $\mu$ is countably subadditive: for
\item $\mu$ is continuous from below, or Scott-continuous:
\item $\mu$ is continuous from above:
\end{parts}
\end{prb}
\begin{pf}

\end{pf}


\begin{prb}[Complete measure spaces]
An enhanced measurable space is called \emph{complete} if the $\sigma$-algebra of negligible sets is an ideal of the power set of the underlying set, and a measure space is called \emph{complete} if the canonically associated enhanced measurable space is complete.

For a measure space $(X,\cM,\mu)$ if we define
\[\tilde\cN:=\{P\cap N:P\in\cP(X),\ N\in\cN\},\qquad\tilde\cM:=\{E\triangle\tilde N:E\in\cM,\ \tilde N\in\tilde\cN\},\]
and a set function $\tilde\mu:\tilde\cM\to[0,\infty]$ such that
\[\tilde\mu(E\triangle\tilde N):=\mu(E),\qquad E\in\cM,\ \tilde N\in\tilde\cN,\]
then we have a complete measure space $(X,\tilde\cM,\tilde\mu)$, called the \emph{completion} of $(X,\cM,\mu)$.
\begin{parts}
\item There is a canonical ring isomorphism $\tilde\cM/\tilde\cN\to\cM/\cN$.
\end{parts}
\end{prb}


\begin{prb}[Localizable measure spaces]
For a measurable space $(X,\cM)$, we say a set function $\mu:\cM\to[0,\infty]:\varnothing\mapsto0$ is \emph{semi-finite} if $\mu(E):=\sup\{\mu(F):F\subset E,\ F\in\cM,\ \mu(F)<\infty\}$ for all $E\in\cM$.
A measure space $(X,\cM,\mu)$ is called
\begin{enumerate}[(i)]
\item \emph{localizable} if the measure is semi-finite and the measurable algebra is complete,
\item \emph{strictly localizable} or \emph{decomposable} if it admits a partition $F_i$ of measurable sets of finite measure such that a subset $E\subset X$ is measurable with $\mu(E)=\sum_i\mu(F_i\cap E)$ whenever $F_i\cap E$ is measurable for all $i$,
\item \emph{$\sigma$-finite} if it admits a countable cover of measurable sets of finite measure,
\item \emph{standard} if it is localizable and $L^1(X,\mu)$ is separable. (can we describe more naturally without introducing $L^1(X,\mu)$?)
\end{enumerate}
In most applications, non-standard measure spaces are rarely discussed.
\begin{parts}
\item A measure is semi-finite if and only if every measurable set $E$ of infinite measure has a measurable subset $F\subset E$ of positive finite measure.
\item A strictly localizable meausre space is localizable.
\item A $\sigma$-finite measure space is strictly localizable.
\item A standard measure space is $\sigma$-finite.
\end{parts}
\end{prb}


\begin{prb}[Measurable maps]
\end{prb}


\section{Carath\'eodory extension}

Important examples:
$\cA$ the set of boxes(Fubini) -> inner measures
$\cA$ the set of compact sets(Riesz-Markov-Kakutani) -> inner measures
$\cA$ the set of cylinder sets(Kolmogorov extension) -> outer measures



\begin{prb}[Outer measures]
Let $X$ be a set, and let $\mu_0:\cA\subset\cP(X)\to[0,\infty]:\varnothing\mapsto0$ be a set function.
Define $\mu^*:\cP(X)\to[0,\infty]$ such that
\[\mu^*(P):=\inf\left\{\,\sum_{i=1}^\infty\mu_0(A_i):P\subset\bigcup_{i=1}^\infty A_i,\ A_i\in\cA\,\right\},\qquad P\in\cP(X),\]
where we use the convention $\inf\varnothing=\infty$.
We can check $\mu^*$ is monotone and countably subadditive easily.

An \emph{outer measure} on a set $X$ is a set function $\cP(X)\to[0,\infty]:\varnothing\mapsto0$ that is monotone and countably subadditive.
\begin{parts}
\item
$\mu^*$ extends $\mu_0$ if $\mu_0$ satisfies the monotone countable subadditivity: we have
\[A_0\subset\bigcup_{i=1}^\infty A_i\quad\Rightarrow\quad\mu_0(A_0)\le\sum_{i=1}^\infty\mu_0(A_i),\qquad A_i\in\cA.\]
\item $\cA\subset\cM$
$\mu$ extends $\mu_0$ if $\mu_0$ satisfies the following property in addition: for $B,A\in\cA$ and any $\e>0$, there are $(C_j)_{j=1}^\infty,(D_j)_{j=1}^\infty\subset\cA$ such that
\[B\cap A\subset\bigcup_{j=1}^\infty C_j,\quad B\setminus A\subset\bigcup_{j=1}^\infty D_j,\quad\sum_{j=1}^\infty(\mu_0(C_j)+\mu_0(D_j))<\mu_0(B)+\e.\]
\end{parts}
\end{prb}
\begin{pf}
(a)
Fix $A\in\cA$.
Clearly $\mu^*(A)\le\mu_0(A)$.
For the opposite direction, we may assume $\mu^*(A)<\infty$.
By the finiteness of $\mu^*(A)$, for any $\e>0$ we have $(B_i)_{i=1}^\infty\subset\cA$ such that $A\subset\bigcup_{i=1}^\infty B_i$ and
\[\sum_{i=1}^\infty\mu_0(B_i)<\mu^*(A)+\e.\]
Therefore we have $\mu_0(A)<\mu^*(A)+\e$ by the assumption, and we get $\mu_0(A)\le\mu^*(A)$ by limiting $\e\to0$.

(b)
Fix $A\in\cA_0$.
It is enough to check the inequality $\mu^*(S\cap A)+\mu^*(S\setminus A)\le\mu^*(S)$ for $S\in\cP(\Omega)$ with $\mu^*(S)<\infty$.
By the finiteness of $\mu^*(S)$, we have $(B_i)_{i=1}^\infty\subset\cB$ such that $S\subset\bigcup_{i=1}^\infty B_i$.
From the condition, we have $B_i\cap A\subset\bigcup_{j=1}^\infty C_{i,j}$ and $B_i\setminus A\subset\bigcup_{j=1}^\infty D_{i,j}$ satisfying
\begin{align*}
\mu^*(S\cap A)+\mu^*(S\setminus A)
&\le\mu^*\Bigl(\bigcup_{j=1}^\infty(B_i\cap A)\Bigr)+\mu^*\Bigl(\bigcup_{j=1}^\infty(B_i\setminus A)\Bigr)\\
&\le\sum_{i,j=1}^\infty(\mu_0(C_{i,j})+\mu_0(D_{i,j}))\\
&\le\sum_{i=1}^\infty(\mu_0(B_i)+2^{-i}\e)\\
&<\mu^*(S)+\e.
\end{align*}
Therefore, $A$ is Carath\'eodory measurable relative to $\mu^*$, so the domain of $\mu$ contains the domain of $\mu_0$.
The values coincide by the part (a).

(c)

\end{pf}


\begin{prb}[Uniqueness of extension of measures]
The Carath\'eodory extension also provides a uniqueness result for measure extensions.
Let $\mu_0:\cM_0\to[0,\infty]:\varnothing\mapsto0$ be a set function, where $\varnothing\in\cM_0\subset\cP(\Omega)$ for a set $\Omega$.
We say $\mu_0$ is \emph{$\sigma$-finite} if there is a cover $\{B_i\}_{i=1}^\infty\subset\cM_0$ of $\Omega$ such that $\mu_0(B_i)<\infty$ for each $i$.

Let $\cM$ be a $\sigma$-algebra containing $\cM_0$.
Let $\mu$ be a measure on $\cM$, which extends $\mu_0$, given by the restriction of the outer measure $\mu^*$ associated to $\mu_0$.
Let $\nu$ be another measure on $\cM$ which extends $\mu_0$.
Let $E\in\cM$ and $\{E_i\}_{i=1}^\infty\subset\cM$.
\begin{parts}
\item $\nu(E)\le\mu(E)$.
\item $\nu(E_i)=\mu(E_i)$ implies $\nu\Bigl(\bigcup_{i=1}^\infty E_i\Bigr)=\mu\Bigl(\bigcup_{i=1}^\infty E_i\Bigr)$.
\item $\nu(E)=\mu(E)$ for $\mu(E)<\infty$.
\item $\nu(E)=\mu(E)$ for $\mu(E)=\infty$, if $\mu_0$ is $\sigma$-finite
\end{parts}
\end{prb}
\begin{pf}
(a)
We may assume $\mu(E)<\infty$.
By the definition of the outer measure, there is $\{B_i\}_{i=1}^\infty\subset\cM_0$ such that $E\subset\bigcup_{i=1}^\infty B_i$.
Also, whenever $E\subset\bigcup_{i=1}^\infty B_i$ we have
\[\nu(E)\le\nu\Bigl(\bigcup_{i=1}^\infty B_i\Bigr)\le\sum_{i=1}^\infty\nu(B_i)=\sum_{i=1}^\infty\mu_0(B_i)=\sum_{i=1}^\infty\mu(B_i),\]
hence $\nu(E)\le\mu(E)$.

(b)
In the light of the inclusion-exclusion principle, we have
\[\mu(E_i\cup E_j)=\mu(E_i)+\mu(E_j)-\mu(E_i\cap E_j)\le\nu(E_i)+\nu(E_j)-\nu(E_i\cap E_j)=\nu(E_i\cup E_j),\]
so that $\mu(E_i\cup E_j)=\nu(E_i\cap E_j)$.
Applying it inductively, we have for every $n$ that
\[\mu\Bigl(\bigcup_{i=1}^nB_i\Bigr)=\nu\Bigl(\bigcup_{i=1}^nB_i\Bigr),\]
and by limiting $n\to\infty$ the continuity from below gives
\[\mu\Bigl(\bigcup_{i=1}^\infty B_i\Bigr)=\nu\Bigl(\bigcup_{i=1}^\infty B_i\Bigr).\]

(c)
Because $\mu(E)<\infty$, for any $\e>0$ we have a sequence $(B_i)_{i=1}^\infty\subset\cM_0$ such that $E\subset\bigcup_{i=1}^\infty B_i$ and
\[\sum_{i=1}^\infty\mu_0(B_i)<\mu(E)+\e.\]
Applying the part (b) 
Then, we have
\[\mu(E)\le\mu\Bigl(\bigcup_{i=1}^\infty B_i\Bigr)=\nu\Bigl(\bigcup_{i=1}^\infty B_i\Bigr)=\nu\Bigl(\bigcup_{i=1}^\infty B_i\setminus E\Bigr)+\nu(E)\]
and
\[\nu\Bigl(\bigcup_{i=1}^\infty B_i\setminus E\Bigr)
\le\mu\Bigl(\bigcup_{i=1}^\infty B_i\setminus E\Bigr)
=\mu\Bigl(\bigcup_{i=1}^\infty B_i\Bigr)-\mu(E)
\le\sum_{i=1}^\infty\mu(B_i)-\mu(E)=\sum_{i=1}^\infty\rho(B_i)-\mu(E)<\e,\]
we get $\mu(E)<\nu(E)+\e$ and $\mu(E)\le\nu(E)$ by limiting $\e\to0$.

(d)
Let $\{B_i\}_{i=1}^\infty\subset\cM_0$ be a cover of $X$ such that $\mu_0(B_i)<\infty$.
Define $E_1:=B_1$ and $E_n:=B_n\setminus\bigcup_{i=1}^{n-1}B_i$ for $n\ge2$ so that $\{E_i\}_{i=1}^\infty$ is a pairwise disjoint cover of $X$ with
\[\mu(E\cap E_i)\le\mu(E_i)\le\mu(B_i)=\mu_0(B_i)<\infty\]
for each $i$, so we have by the part (c) that
\[\nu(E)=\sum_{i=1}^\infty\nu(E\cap E_i)=\sum_{i=1}^\infty\mu(E\cap E_i)=\mu(E).\qedhere\]
\end{pf}

\begin{prb}[Inner measures]
On a set $X$, let $\mu_0:\cA\subset\cP(X)\to[0,\infty):\varnothing\mapsto0$ be a set function, and $\cM$ be a $\sigma$-algebra on $X$ containing $\cA$ as a subfamily.
We want to investigate a sufficient condition for $\mu_0$ and $\cM$ in order to extend $\mu_0$ to a measure on $\cM$.
Define $\mu_*:\cP(X)\to[0,\infty]$ such that
\[\mu_*(P):=\sup\left\{\sum_{j=1}^n\mu_0(A_j):\bigsqcup_{j=1}^nA_j\subset P,\ A_i\in\cA\right\},\qquad P\in\cP(X).\]
We can check $\mu_*$ extends $\mu_0$ easily.
Our idea of construction is to restrict $\mu_*$ onto $\cM$ to obtain a measure.

An \emph{inner measure} on a set $X$ is a set function $\cP(X)\to[0,\infty]:\varnothing\mapsto0$ that is semi-finite, continuous from above, and super-additive.
Note that super-additivity implies the monotonicity.
When $\mu_*$ is an inner measure on $X$, a subset $\tilde E\subset X$ is called \emph{Carath\'eodory measurable} relative to $\mu_*$ if
\[\mu_*(P)=\mu_*(P\cap\tilde E)+\mu_*(P\cap\tilde E^c),\qquad P\in\cP(X),\]
where the equality holds in the extended real numbers $[0,\infty]$.
Let $\tilde\cM\subset\cP(X)$ be the set of all Carath\'eodory measurable subsets relative to $\mu_*$, and let $\tilde\mu:=\mu_*|_{\tilde\cM}$.
\begin{parts}
\item If , then $\cM\subset\tilde\cM$.
\item If $\mu_0$ is continuous from above for downward directed sequences, then $\mu_*$ is an inner measure.
\item If $\mu_*$ is an inner measure, then $(X,\tilde\cM,\tilde\mu)$ is a complete measure space.
\item If $\mu_0$ is continuous from above for downward directed nets, then $\mu$ is strictly localizable. (not yet proved, I think we have to use Haagerup's argument)
\item localizable extension is unique, and $(X,\tilde\cM,\tilde\mu)$ is the completion(maybe?) of $(X,\cM,\mu)$.
\end{parts}
\end{prb}


\begin{pf}
(a)

(b)
Clearly $\mu_*(\varnothing)=0$.
It is semi-finite since for any $P\in\cP(X)$ with $\mu_*(P)=\infty$, then the definition implies that there is a sequence $B_n\in\cP(X)$ such that $B_n\subset P$ and $0<\mu_*(B)<\infty$.
It is also super-additive because... clear.

To prove $\mu_*$ is continuous from above, take a sequence $P_n\downarrow P$ in $\cP(X)$ such that $\sup_n\mu_*(P_n)<\infty$.
For $\e>0$, we can take $B_n\in\cB$ inductively such that
\[B_n\subset P_n,\qquad \mu_*(P_n)<\mu(B_n)+\e,\qquad B_n\downarrow B\]
as follows: when
\[B_n\subset P_n,\qquad \mu_*(P_n)<\mu(B_n)+\left(1-\frac1{2^n}\right)\e,\]
we can take $B_{n+1}'$ such that 
\[B_{n+1}'\subset P_{n+1},\qquad \mu_*(P_{n+1})<\mu(B_{n+1}')+\frac\e{2^{n+1}},\]
then since $\mu_*$ is additive on $\cB$, if we let $B_{n+1}:=B_n\cap B_{n+1}'$, then we have $B_{n+1}\subset P_{n+1}$ and
\[\mu_*(P_{n+1})<\mu(B_{n+1}')+\frac\e{2^{n+1}}\le\mu(B_{n+1}'\cap B_n)+\mu(P_n\setminus B_n)+\frac\e{2^{n+1}}<\mu(B_{n+1})+\left(1-\frac1{2^{n+1}}\right)\e.\]
Then,
\[\mu_*(P_n)=\mu_*(B_n)+\mu_*(P_n\setminus B_n)\le\mu_*(B_n)+\mu_*(P_n)-\mu(B_n)\le\mu_*(B_n)+\e\downarrow\mu_*(B)+\e\le\mu_*(P)+\e,\]
so we are done.

(c)
If $E\in\tilde\cM$, then $E^c\in\tilde\cM$ since
\[\mu_*(P)=\mu_*(P\cap E^c)+\mu_*(P\cap E),\qquad P\in\cP(X),\]
and if $E_1,E_2\in\tilde\cM$, then $E_1\cup E_2\in\tilde\cM$ since
\begin{align*}
\mu_*(P)
&=\mu_*(P\cap E_1)+\mu_*(P\cap E_1^c)\\
&=\mu_*(P\cap E_1)+\mu_*(P\cap E_1^c\cap E_2)+\mu_*(P\cap E_1^c\cap E_2^c)\\
&=\mu_*(P\cap E_1\cap(E_1\cup E_2))+\mu_*(P\cap E_1^c\cap(E_1\cup E_2))+\mu_*(P\cap(E_1\cup E_2)^c)\\
&=\mu_*(P\cap(E_1\cup E_2))+\mu_*(P\cap(E_1\cup E_2)^c),\qquad P\in\cP(X).
\end{align*}
Thus, $\tilde\cM$ is an algebra on $X$.

Suppose $E_n\in\tilde\cM$ is a sequence such that $E_n\uparrow E$.
We claim $E\in\tilde\cM$.
Let $F$ be any subset of $P$ with $\mu_*(F)<\infty$.
By the Carath\'eodory measurability of $E_n$ and the inclusion $F\cap E_n\subset P\cap E$, we have
\begin{align*}
\mu_*(F)
&=\mu_*(F\cap E_n)+\mu_*(F\cap E_n^c)\\
&\le\mu_*(P\cap E)+\mu_*(F\cap E_n^c).
\end{align*}
By the continuity from above as $n\to\infty$ and the inclusion $F\cap E^c\subset P\cap E^c$, we have
\begin{align*}
\mu_*(F)
&\le\mu_*(P\cap E)+\mu_*(F\cap E^c)\\
&\le\mu_*(P\cap E)+\mu_*(P\cap E^c).
\end{align*}
By the semi-finiteness, the supremum on $F\subset P$ implies
\[\mu_*(P)\le\mu_*(P\cap E)+\mu_*(P\cap E^c).\]
Therefore, we have $E\in\tilde\cM$ by the super-additivity, which concludes that $\tilde\cM$ is a $\sigma$-algebra.


If $E_n$ is a sequence in $\tilde\cM$ such that $E_n\uparrow E$ in $\tilde\cM$, then for any subset $F$ of $E$ with $\mu_*(F)<\infty$, the continuity from above implies
\[\mu_*(F)=\mu_*(F\cap E_n)+\mu_*(F\setminus E_n)\le\mu_*(E_n)+\e,\]
and the semi-finiteness implies $\mu_*(E_n)\to\mu_*(E)$, so $(X,\tilde\cM,\tilde\mu)$ is a measure space.

Furthermore, the measure space is complete since if $N\in\cP(X)$ is contained in $E\in\tilde\cM$ satisfying $\mu_*(E)=0$, then the equality $\mu_*(P\cap E)=\mu_*(P\cap N)$ as zeros and $P\cap E^c\subset P\cap N^c$ imply
\[\mu_*(P)=\mu_*(P\cap E)+\mu_*(P\cap E^c)\le\mu_*(P\cap N)+\mu_*(P\cap N^c)\le0+\mu_*(P),\qquad P\in\cP(X),\]
hence we have $N\in\tilde\cM$.

(d)
Using the Zorn lemma, take a maximal family $\{F_i\}\subset\tilde\cM$ of mutually disjoint subsets of finite measures.
Suppose $F:=\bigcup_iF_i\ne X$.
For the partition, it suffices to show $F$ is Carath\'eodory measurable since its complement or its non-empty subset has finite inner measure by the semi-finiteness.

For $P\in\cP(X)$ such that $P\cap F$ covered by countably many $F_i$, then $P\setminus F$ is the intersection of countably many $P\setminus F_i$, so
\[\mu_*(P)=\mu_*(P\cap F_n)+\mu_*(P\setminus F_n)\le\mu_*(P\cap F)+\mu_*(P\setminus F_n)\downarrow\mu_*(P\cap F)+\mu_*(P\setminus F).\]

For general $P\in\cP(X)$,



Now suppose $F_i\cap E\in\tilde\cM$ for all $i$.
\[\mu_*(P)=\]
\end{pf}

To use inner measures, it is enough to prove the continuity from above by downward directed nets, and measurability.


\section{Measures on Euclidean spaces}

Cantor set

\begin{prb}[Borel $\sigma$-algebra]
\end{prb}

\begin{prb}[Distribution functions]
\begin{parts}
\item Let $a<b\in\R_{\pm\infty}$. There is one-to-one correspondence between right continuous non-decreasing functions $F:[a,b]\to\R$ such that $F(a)=0$, $F(b)=1$, and the probability Borel measures on $[a,b]$.
\item 
\end{parts}
\end{prb}
\begin{pf}
We may assume $a>-\infty$
Suppose $(a,b]\subset\bigcup_{i=1}^\infty(a_i,b_i]$.
Using the right-continuity of $F$, for arbitrary $\e>0$, take $\e_i$ such that $F(bi+\e_i)-F(b_i)<\e2^{-i}$ for each $i$.
Then, by the Heine-Borel, there is $n$ such that $[a+\e,b]\subset\bigcup_{i=1}^n(a_i,b_i+\e_i)$, and we have
\[F(b)-F(a+\e)\le\sum_{i=1}^n(F(b_i+\e_i)-F(a_i)).\]
By limiting $\e\to0$, we have what we desired.

\end{pf}

\begin{prb}[Helly selection theorem]
\end{prb}

\begin{prb}[Vitali set]
\end{prb}

\section{Descriptive set theory}


\section*{Exercises}

\begin{prb}[Cardinalities]
infinite $\sigma$-algebra is $\ge\fc$.

\end{prb}

\begin{prb}[Semi-rings and semi-algebras]
We will prove a simplified Carath\'eodory extension with respect to \emph{semi-rings} and \emph{semi-algebras}.
Let $\cM_0\subset\cP(\Omega)$ such that $\varnothing\in\cM_0$.
We say that $\cM_0$ is a semi-ring if it is closed under finite intersections, and each relative complement is a finite union of elements of $\cM_0$.
We say that $\cM_0$ is a semi-algebra

Let $\cM_0$ be a semi-ring of sets over $X$.
Suppose a set function $\mu_0:\cM_0\to[0,\infty]:\varnothing\mapsto0$ satisfies
\begin{enumerate}[(i)]
\item $\mu_0$ is \emph{disjointly countably subadditive}: we have
\[\rho\Bigl(\bigsqcup_{i=1}^\infty A_i\Bigr)\le\sum_{i=1}^\infty\rho(A_i)\]
for $(A_i)_{i=1}^\infty\subset\cM_0$,
\item $\mu_0$ is \emph{finitely additive}: we have
\[\rho(A_1\sqcup A_2)=\rho(A_1)+\rho(A_2)\]
for $A_1,A_2\in\cM_0$.
\end{enumerate}
A set function satisfying the above conditions are occasionally called a \emph{pre-measure}.
\begin{parts}
\item
\item 
\end{parts}
\end{prb}

\begin{prb}[Monotone class lemma]
A collection $\cC\subset\cP(\Omega)$ is called a \emph{monotone class} if it is closed under countable increasing unions and countable decreasing intersections.

Let $H$ be a vector space closed under bounded monotone convergence.
If $\spn\{1_A:A\in\cM\}\subset H$ then $B^\infty(\sigma(\cM)\subset H$.
\end{prb}






\begin{prb}[Steinhaus theorem]
Let $\lambda$ denote the Lebesgue measure on $\R$ and let $\E\subset\R$ be a Lebesgue measurable set with $\lambda(E)>0$.
\begin{parts}
\item For any $0<\alpha<1$, there is an interval $I=(a,b)$ such that $\lambda(E\cap I)>\alpha\lambda(I)$.
\item $E-E=\{x-y:x,y\in E\}$ contains an open interval containing zero.
\end{parts}
\begin{pf}
(a)
We may assum $\lambda(E)<\infty$.
Since $\lambda$ is outer measure and $\lambda(E)\ne0$, we have an open subset $U$ of $\R$ such that $\lambda(U)<\alpha^{-1}\lambda(E)$.
Because $U$ is a countable disjoint union of open intervals $U=\bigsqcup_{i=1}^\infty(a_i,b_i)$, we have
\[\sum_{i=1}^\infty\lambda((a_i,b_i))=\lambda(U)<\alpha^{-1}\lambda(E)=\alpha^{-1}\sum_{i=1}^n\lambda(E\cap(a_i,b_i)).\]
Therefore, there is $i$ such that $\alpha\lambda((a_i,b_i))<\lambda(E\cap(a_i,b_i))$.
\end{pf}
% convolution으로 푸는 방법: continuous approximation 이 레벨에선 무리인듯
\end{prb}


\begin{prb}[Measures from volume forms]
	
\end{prb}


\section*{Problems}
\begin{enumerate}
\item* Every Lebesgue measurable set in $\R$ of positive measure contains an arbitrarily long arithmetic progression.
\end{enumerate}

















\chapter{Integration}



\section{Monotone convergence theorem}


% Stein: Egorov $\to$ BCT $\to$ Fatou $\to$ MCT $\to$ L1<M\\
% Stein: BCT + L1<M $\to$ DCT\\
% Folland: MCT $\to$ Fatou $\to$ DCT $\to$ BCT



\begin{prb}
Let $(X,\mu)$ be a localizable measure space.

On the set of measurable functions, we can define an equivalent relation such that two functions are said to be equivalent if they are equal on a full set.
We define the set $\bar L(X,\mu)$ and $L(X,\mu)$ of all equivalence classes of measurable functions $X\to\bar\R$ and $X\to\R$ respectively.
The alphabet L is chosen from Lebesgue, and it is not a standard notation.

As the extended real numbers $\bar\R$, on $\bar L(X,\mu)$, algebraic structures are partially defined, but it is a complete lattice.

For $f:X\to\bar\R$ an almost everywhere finite measurable function, when we write $[f]$ to be the equivalence class of $f$, it is customary to write $f\in\bar L(X,\mu)$ by abusing notation if $[f]\in\bar L(X,\mu)$.



We want to consider algebraic structures and order structures on $\bar L(X,\mu)$ and $L(X,\mu)$.

For the order structure, we should not use the notion of almost everywhere suprema for generally uncountably many functions, and rather use the essential notion of essential suprema.
For $f_i\in\bar L(X,\mu)$, we say $f\in\bar L(X,\mu)$ is an \emph{essential supremum} of $f_i$ if it defines the supremum in $\bar L(X,\mu)$.

\begin{parts}
\item $\bar L(X,\mu)^+$ is a directed complete partially ordered set.
\item $L(X,\mu)$ is a Dedekind complete Riesz space.
\item $L(X,\mu)$ is a real commutative algebra.
\item For $f_i\in\bar L(X,\mu)$, if $\sup_if_i\in L(X,\mu)$, then $\sup_if_i$ is the supremum of $f_i$ in $L(X,\mu)$.
\item If a net $f_i\in L(X,\mu)$ is either countable or non-decreasing, and if $f\in L(X,\mu)$ satisfies $f=\sup_if_i$ almost everywhere, then $f=\sup_if_i$ is an essential supremum of $f_i$.
\end{parts}
\end{prb}




\begin{prb}[Monotone convergence theorem]
Let $(X,\mu)$ be a measure space.
Note that $\bar L(X,\mu)^+$ is a directed complete partially ordered set, equipped with the addition and the positive scalar multiplication.
A \emph{Lebesgue integral} can be defined as a map $I:\bar L(X,\mu)^+\to[0,\infty]$ such that
\begin{enumerate}[(i)]
\item it is additive and homogeneous in the sense that
\[I(f+g)=I(f)+I(g),\qquad I(af)=aI(f),\qquad f,g\in\bar L(X,\mu)^+,\ a\ge0,\]
\item it is continuous in the sense that $I(\sup f_i)=\sup_iI(f_i)$ for non-decreasing net $f_i\in\bar L(X,\mu)$,
\item $I(1_E)=\mu(E)$ for all measurable $E\subset X$.
\end{enumerate}

We show that it is uniquely determined by the assignment
\[f\mapsto\int f\,d\mu:=\sup_s\int s\,d\mu,\qquad f\in L_\loc^0(X,\mu)^+,\]
where $s$ runs through all non-negative simple functions dominated by $f$.
Let $f_i\in L_\loc^0(X,\mu)^+$ be a net.
\begin{parts}
\item The integral is uniquely well-defined.
\item If $f_i\uparrow f$ in $L_\loc^0(X,\mu)^+$, then $\int f_i\,d\mu\uparrow\int f\,d\mu$ in $[0,\infty]$.
\item If $f_i$ is non-decreasing and $\sup_i\int g\circ f_i\,d\mu<\infty$ for some $g\in L_\loc^0(X,\mu)$, then $f_i$ is bounded above in $L_\loc^0(X,\mu)$.
\end{parts}
\end{prb}
\begin{pf}
(a)
First we prove that $E\mapsto\int_Ef\,d\mu$ is a measure.
Let $E_n$ be a sequence of measurable sets such that $E_n\uparrow E$.
Since $s\,d\mu$ is a measure for each simple function $s:X\to\R_{\ge0}$ by the linearity of the integral for simple functions, we have
\[\sup_n\int_{E_n}g\,d\mu=\sup_n\sup_s\int_{E_n}s\,d\mu=\sup_s\sup_n\int_{E_n}s\,d\mu=\sup_s\int_Es\,d\mu=\int_Eg\,d\mu,\]
where $s$ runs through all non-negative simple functions dominated by $g$, so the claim follows.

Now we prove the original statement.
Fix $\e>0$.
The union of the non-decreasing net of measurable subsets $E_i:=\{x:f(x)<(1+\e)f_i(x)\}$ is a full set by the almost everywhere convergence $f_i\to f$.
Since
\[\int_{E_i}f\,d\mu\le(1+\e)\int_{E_i}f_j\,d\mu,\qquad j\succ i\]
so that
\[\int f\,d\mu=\sup_i\int_{E_i}f\,d\mu\le\sup_i\sup_j\,(1+\e)\int_{E_i}f_j\,d\mu=\sup_j\sup_i\,(1+\e)\int_{E_i}f_j\,d\mu=\sup_j\,(1+\e)\int f_j\,d\mu.\]
Since $\e>0$ is arbitrary, this completes the proof.


\end{pf}

\begin{prb}[Corollaries of monotone convergence theorem]
Fatou's lemma, linearity of the integral, $f\ge0$ and $\int f=0$ imply $f=0$ a.e.
\end{prb}




\begin{prb}[Lebesgue integral of complex-valued functions]
\end{prb}





\section{Lebesgue spaces}

$L^\infty$ is a Banach algebra naturally acts on $L^p$.

We have $L(X,\mu)=L_\loc^0(X,\mu)$ in the notation of Lebesgue spaces.

\begin{prb}
H\"older and Minkowski inequalities
\end{prb}

\begin{prb}[Completeness]
\end{prb}
\begin{pf}
Let $1\le p<\infty$.
Let $\sum_kf_k$ be an absolutely convergent series of $L^p(X,\mu)$ so that $\sum_{k=0}^\infty\|f_k\|_p<\infty$.
Then, the partial sum $\sum_{k=0}^n|f_k|$ is a non-decreasing sequence in $L_\loc^0(X,\mu)$ that is bounded above by
\[\left(\int\Bigl(\sum_{k=0}^n|f_k|\Bigr)^p\,d\mu\right)^{\frac1p}=\Bigl\|\sum_{k=0}^n|f_k|\Bigr\|_p\le\sum_{k=0}^n\|f_k\|_p<\sum_{k=0}^\infty\|f_k\|_p,\qquad n\ge0.\]

\[I((\Sigma^n)^p)\]

so the monotone convergence theorem implies $\sum_{k=0}^\infty|f_k|$ exists in $L_\loc^0(X,\mu)$.

$\sum_{k=0}^\infty f_k$ exists? in $L_\loc^0(X,\mu)$? in which sense?

$L^p$ convergence is done by dominated convergence?



\end{pf}


\begin{prb}[Radon-Nikodym theorem]
Let $(X,\mu)$ be a (localizable?) measure space.
Let $\nu$ be an absolutely continuous measure on $X$ with respect to $\mu$.
$\mu(E)=0$ implies $\nu(E)=0$.

Is $\nu$ localizable?
\begin{parts}
\item if $\mu$ and $\nu$ are finite.
\end{parts}
\end{prb}
\begin{pf}
(a)
Let
\[I:=\left\{i\in L_\loc^0(X,\mu)^+:\int_Ei\,d\mu\le\nu(E),\ E\subset X\text{ measurable}\right\}.\]
We can check that $I$ is directed because it is non-empty by containing zero and it admits the maximum operation, which means that it defines a net $f_i:=i$ for $i\in I$.
Then, $f_i$ is bounded above by $\sup_i\int f_i\,d\mu\le\nu(X)<\infty$.
Then, we have $f_i\uparrow f$ in $L_\loc^0(X,\mu)$ with $\int f_i\,d\mu\uparrow\int f\,d\mu$ by the monotone convergence theorem, so we can conclude $f\in I$.

Now we claim $f\,d\mu=\nu$.
Suppose not so that $\nu-f\,d\mu$ is non-zero.
By the finiteness of $\mu$, since there exists $\e>0$ such that $(\nu-f\,d\mu-\e\mu)_+\ne0$, we can take a measurable $E\subset X$ such that $\mu(E)>0$ and $\e\mu<\nu-f\,d\mu$ on $E$, the function $f+\e1_E$ in $L_\loc^0(X,\mu)$ belongs to $I$.
for all measurable sebset $E\subset X$.
Since $f$ is maximal, we should have $1_E=0$ almost everywhere with respect to $\mu$, which means $\mu(E)=0$.
The absolute continuity of $\nu$ implies $\nu(E)=0$, which is impossible to have the assumption $\int_Ef\,d\mu<\nu(E)$.
Therefore, the claim follows.

\end{pf}

\begin{prb}[Riesz representation theorem]
\end{prb}




\begin{prb}[Convolution?]
\end{prb}
\begin{prb}[Approximate identity?]
\end{prb}
\begin{prb}[Continuity of translation?]
\end{prb}







\section{Bounded convergence theorem}





\begin{prb}[Convergence in measure]
Let $(X,\mu)$ be a measure space and let $f_n,f_i:X\to\bar\R$ be a sequence and a net of almost everywhere finite measurable functions respectively.
We say $f_i$ converges to a measurable function $f:X\to\bar\R$ \emph{globally in measure} if for every $\e>0$
\[\lim_i\mu(\{x:|f_i(x)-f(x)|\ge\e\})=0,\]
and \emph{locally in measure} if it converges globally in measure on each measurable subspace $F\subset X$ of finite measure.
\begin{parts}
\item If $f_i\to f$ locally in measure, then $f$ is finite almost everywhere. ($L_\loc^0(X,\mu)$ is complete with respect to local measure topology?)
\item If $f_i\uparrow f$ in $L_\loc^0(X,\mu)$, then $f_i\to f$ locally in measure. (really?)
\item The local convergence in measure does not depend on the choice of faithful absolutely continuous measures.
\end{parts}
\end{prb}
\begin{pf}
\end{pf}


\begin{prb}[Bounded convergence theorem]
Let $(X,\mu)$ be a measure space.

If $f_i$ is a net of $L^\infty(X)$,

\begin{parts}
\item If $f_i\to f$ globally in measure, then $f_i\to f$ in $L^1$.
\item
\end{parts}
\end{prb}
\begin{pf}
We may assume $f=0$ almost everywhere.
For $\e>0$ and $\delta>0$, if we choose $i_0$ such that
\[\mu(\{x:|f_i(x)|\ge\e\})<\delta,\qquad i\succ i_0,\]
then taking limit superior for $i$ and taking limits $e,\delta\to0$ on
\[\int|f_i|\,d\mu=\int_{|f_i|\ge\e}|f_i(x)|\,d\mu(x)+\int_{|f_i|<\e}|f_i(x)|\,d\mu(x)\le\delta\sup_i\|f_i\|_\infty+\e\mu(X),\]
we obtain $\int|f_i|\,d\mu\to0$.

\end{pf}


\begin{prb}[Egorov theorem]
Egorov's theorem informally states that an almost everywhere convergent functional sequence is ``almost'' uniformly convergent.
Through this famous theorem, we introduce a convenient ``$\e/2^m$ argument'', occasionally used throughout measure theory to construct a measurable set having a special property.

Let $(X,\mu)$ be a finite measure space and let $f_n:X\to\R$ be a sequence of measurable functions such that $f_n\to f$ a.e.
For each positive integer $m$, which indexes the tolerance $1/m$, consider an increasing sequence of measurable subsets
\[E_n^m:=\bigcap_{i=n}^\infty\{x:|f_i(x)-f(x)|<\tfrac1m\}.\]
\begin{parts}
\item $E_n^m$ converges to a full set for each $m$.
\item For every $\e>0$ there is a measurable $K\subset X$ such that $\mu(X\setminus K)<\e$ and for each $m$ there is finite $n$ satisfying $K\subset E_n^m$.
\item For every $\e>0$ there is a measurable $K\subset X$ such that $\mu(X\setminus K)<\e$ and $f_n\to f$ uniformly on $K$.
\end{parts}
\end{prb}
\begin{pf}
(a)
Recall that the a.e. convergence $f_n\to f$ means that for every fixed $m$ the intersection
\[\bigcap_{n=1}^\infty(X\setminus E_n^m)=\limsup_n\{x:|f_n(x)-f(x)|\ge\tfrac1m\}\]
is a null set.
Since $\mu(X)<\infty$, it is equivalent to $E_n^m$ converges to a full set for each $m$ by the continuity from above.

(b)
For each $m$, we can find $n_m$ such that
\[\mu(X\setminus E_{n_m}^m)<\frac\e{2^m}.\]
If we define
\[K:=\bigcap_{m=1}^\infty E_{n_m}^m,\]
then it satisfies the second conclusion, and also have
\[\mu(X\setminus K)=\mu\Bigl(\bigcup_{m=1}^\infty(X\setminus E_{n_m}^m)\Bigr)\le\sum_{m=1}^\infty\mu(X\setminus E_{n_m}^m)<\sum_{m=1}^\infty\frac\e{2^m}=\e.\]


(c)
Fix $m>0$.
Since $n\ge n_m$ implies $K\subset E_{n_m}^m\subset E_n^m$, we have
\[n\ge n_m\quad\Rightarrow\quad\sup_{x\in K}|f_n(x)-f(x)|<\frac1m.\qedhere\]
\end{pf}







\begin{prb}[Almost everywhere convergence]
Let $(X,\mu)$ be a measure space.
Let $f_n:X\to\R$ be a sequence of measurable functions.
We say $f_n$ converges to a function $f:X\to\R$ \emph{alomst everywhere}, or simply \emph{a.e.}~if
\[\mu(\{x:\lim_{n\to\infty}f_n(x)\ne f(x)\})=0.\]
Let $f_n,f:X\to\R$ be a sequence of measurable functions.
Note that $x\in X$ satisfies $f_n(x)\to f(x)$ if and only if $x$ does not belong to
\[\bigcup_{\e>0}\bigcap_{n>0}\bigcup_{k\ge n}\{x:|f_k(x)-f(x)|\ge\e\}.\]
Each measurable set of the form $\{x:|f_n(x)-f(x)|\ge\e\}$ is sometimes called the \emph{tail event}, which comes from probability theory.



Conversely, every measurable extended real-valued function is a pointwise limit of simple functions.

\begin{parts}
\item If $f_n\to f$ almost everywhere, then $f$ is measurable.
\item $f_n\to f$ a.e. if and only if for each $\e>0$ we have
\[\mu(\{x:\limsup_{n\to\infty}|f_n(x)-f(x)|\ge\e\})=0.\]
\item $f_n\to f$ a.e. if and only if for each $\e>0$ we have
\[\mu(\limsup_{n\to\infty}\{x:|f_n(x)-f(x)|\ge\e\})=0.\]
\item $f_n\to f$ a.e. if for each $\e>0$ we have
\[\sum_{n=1}^\infty\mu(\{x:|f_n(x)-f(x)|\ge\e\})<\infty.\]

\item $f_n$ converges locally in measure to $f$ if and only if every subsequence has a further subsequence convergent to $f$ almost everywhere.
\end{parts}
\end{prb}
\begin{pf}
(c)
The set of divergence of the sequence $f_n$ is given by
\[\bigcup_{m=1}^\infty\bigcap_{n=1}^\infty\bigcup_{i=n}^\infty\{x:|f_i(x)-f(x)|\ge\tfrac1m\}=\bigcup_{m=1}^\infty\,\bigcap_{n=1}^\infty(X\setminus E_n^m).\]

(d)
Since
\[\mu\Bigl(\bigcup_{i=1}^\infty\{x:|f_i(x)-f(x)|\ge\e\}\Bigr)\le\sum_{i=1}^\infty\mu(\{x:|f_i(x)-f(x)|\ge\e\})<\infty,\]
we have by the continuity from above that
\begin{align*}
\mu(\limsup_{n\to\infty}\{x:|f_n(x)-f(x)|\ge\e\})
&=\mu\Bigl(\bigcap_{n=1}^\infty\bigcup_{i=n}^\infty\{x:|f_i(x)-f(x)|\ge\e\}\Bigr)\\
&=\lim_{n\to\infty}\mu\Bigl(\bigcup_{i=n}^\infty\{x:|f_i(x)-f(x)|\ge\e\}\Bigr)\\
&\le\lim_{n\to\infty}\sum_{i=n}^\infty\mu(\{x:|f_i(x)-f(x)|\ge\e\})
=0.\qedhere
\end{align*}


(e)
For each positive integer $k$, if we apply the definition of convergence in measure with $\e\ge\frac1k$, then we can see that there exists $n_k$ such that
\[\mu(\{x:|f_{n_k}(x)-f(x)|\ge\frac1k\})<\frac1{2^k}.\]
By the Borel-Cantelli lemma, we get
\[\mu(\limsup_{k\to\infty}\{x:|f_{n_k}(x)-f(x)|\ge\frac1k\})=0.\]
Then, for each $\e>0$
\begin{align*}
\limsup_{k\to\infty}\{x:|f_{n_k}(x)-f(x)|\ge\e\}
&=\bigcap_{k\ge\e^{-1}}\bigcup_{j\ge k}\{x:|f_{n_j}(x)-f(x)|\ge\e\}\\
&\subset\bigcap_{k\ge\e^{-1}}\bigcup_{j\ge k}\{x:|f_{n_j}(x)-f(x)|\ge\frac1k\}\\
&=\limsup_{k\to\infty}\{x:|f_{n_k}(x)-f(x)|\ge\frac1k\}
\end{align*}
is negligible, which means that $f_{n_k}\to f$ almost everywhere as $k\to\infty$.

\end{pf}







\section{Product measures}

\begin{prb}[Fubini-Tonelli theorem]
Let $(X,\mu)$ and $(Y,\nu)$ be localizable measure spaces.
\end{prb}


Lipschitz and differentiable transformations



\begin{prb}[Integrals on Euclidean spaces]
\end{prb}


\section*{Exercises}
\begin{prb}[Cauchy's functional equation]
Let $f:\R\to\R$ be a function.
Cauchy's functional equation refers to the equation $f(x+y)=f(x)+f(y)$, satisfied for all $x,y\in\R$.
Suppose $f$ satisfies the Cauchy functional equation.
We ask if $f$ is linear, that is $f(x)=ax$ for all $x\in\R$, where $a:=f(1)$.
\begin{parts}
\item $f(x)=ax$ for all $x\in\Q$, but there is a nonlinear solution of Cauchy's functional equation.
\item If $f$ is conitnuous at a point, then $f$ is linear.
\item If $f$ is Lebesgue measurable, then $f$ is linear.
\end{parts}
\end{prb}

\begin{prb}[Pointwise approximation by simple functions]
Let $(X,\mu)$ be a measure space and $X$ a metric space with Borel measurable structure.
By a \emph{simple function} we mean a measurable function $s:X\to X$ of finite image.
\begin{parts}
\item For each open set $U\subset X$ there is a sequence of open sets $U_i$ such that $U=\bigcup_iU_i$ and $\bar U_i\subset U$.
Let $f:X\to X$ be any function.
\item If $f$ is the pointwise limit of a sequence of measurable functions, then $f$ is measurable.
\item If $f$ is measurable, then $f$ is the pointwise limit of a sequence of simple functions, if $X$ is separable.
\item* The pointwise limit of a net of simple functions may not be measurable.
\end{parts}
\end{prb}
\begin{pf}

(b)
Suppose a sequence $(f_n)_n$ of measurable functions converges pointwisely to a function $f$.
For fixed open $U\subset X$ we claim
\[f^{-1}(U)=\bigcup_{i=1}^\infty\ \liminf_{n\to\infty}\ f_n^{-1}(U_i).\]
If it is true, then $f^{-1}(U)$ is the countable set operation of measurable sets $f_n^{-1}(U_i)$.
Let $U_i$ be the sequence associated to $U$ taken by the part (a).

($\subset$) If $\omega\in f^{-1}(U)$, then for some $i$ we have $f(\omega)\in U_i$, so $f_n(\omega)$ is eventually in $U_i$, thus we have $\omega\in\liminf_{n\to\infty}f_n^{-1}(U_i)$.

($\supset$) If $\omega\in\liminf_{n\to\infty}f_n^{-1}(U_i)$ for some $i$, then $f_n(\omega)$ is eventually in $U_i$, so $f(\omega)\in\bar U_i\subset U$, thus we have $\omega\in f^{-1}(U)$.

(c)
Suppose there is a increasing sequence of finite tagged partitions $\cP_n\subset\cB$ satisfying the following property: for each open-neighborhood pair $(x,U)$ there is $n$ and $i$ such that $P_{n,i}\in\cP_n$ and $x\in P_{n,i}\subset U$.
We denote the tags by $t_{n,i}\in P_{n,i}$ for each $P_{n,i}\in\cP_n$.
Define
\[s_n(\omega):=t_{n,i}\quad\text{for}\quad f(\omega)\in P_{n,i}.\]
To show $s_n(\omega)\to f(\omega)$, fix an open $f(\omega)\in U\subset X$.
Then, there is $n_0$ such that there is a sequence $(P_{n,i_n})_{n=n_0}^\infty$ satisfying $P_{n,i_n}\in\cP_n$ and $f(\omega)\in P_{n,i_n}\subset U$.
Then, for all $n\ge n_0$, we have for $f(\omega)\in P_{n,i_n}$ that $s_n(\omega)=t_{n,i_n}\in P_{n,i_n}\subset U$.

The existence of such sequence of partitions...

Another approach: mimicking Pettis measurability theorem.
\end{pf}

\begin{prb}
If the multiplication by $f\in L_\loc^0(X)$ stabilizes $L^1(X)$, then $f\in L^\infty$.
\end{prb}


\begin{prb}[Convergence of one-parameter family]
\end{prb}



If $\|f_n\|_{L^2([0,1])}\le C$ and $f_n\to f$ almost everywhere, then $f_n\to f$ weakly.

\[\lim_{n\to\infty}\int_0^1n^3x^2(1-x)^n\,dx=2\ne0=\int_0^1\lim_{n\to\infty}n^3x^2(1-x)^n\,dx.\]
\[\lim_{n\to\infty}\int_0^\infty n^2e^{-nx}\,dx=\infty\ne0=\int_0^\infty\lim_{n\to\infty}n^2e^{-nx}\,dx.\]




bounded almost everywhere convergent sequence converges weakly$^*$ in $L^p$, $1<p<\infty$.
When $p=1$..?
\begin{pf}
We may assume $(\Omega,\mu)$ is a probability space.
Suppose $f_n\to f$ almost everywhere on $\Omega$.
If we let $E_{\e,n}:=\bigcap_{i=n+1}^\infty\{\omega:|f_i(\omega)-f(\omega)|<\e\}$, then the almost everywhere convergence means that for every $\e>0$ we have $\mu(E_{\e,n})\uparrow1$.
Let $\mu(E_m)>1-\e2^{-m}$.
\end{pf}













\chapter{Topological measures}

\section{Locally compact Hausdorff spaces}

\begin{prb}[One-point compactification]
\end{prb}





\section{Riesz-Markov-Kakutani representation theorem}

\begin{prb}[Radon measures]
Let $X$ be a locally compact Hausdorff space.
A \emph{Radon measure} on $X$ is a locally finite and tight Borel measure on $X$.
In other words, for every $x\in X$ there is an open neighborhood $U$ of $x$ such that $\mu(U)<\infty$, and for every Borel $E\subset X$ we have $\mu(E):=\sup\{\mu(K):K\subset E,\ K\text{ compact}\}$.

\begin{parts}
\item A Radon measure is (strictly) localizable. (415A)
\item If $X$ is second countable, then a locally finite Borel measure is tight.
\item
\end{parts}
\end{prb}

\begin{prb}[Riesz-Markov-Kakutani representation theorem]
Let $X$ be a locally compact Hausdorff space.
We want to establish the following one-to-one correspondence:
\[\begin{array}{ccc}
\{\text{Radon measures on $X$}\} & \xrightarrow{\sim} & \{\text{positive linear functionals on $C_c(X)$}\}\\
\mu & \mapsto & (f\mapsto\int f\,d\mu).
\end{array}\]
\end{prb}
\begin{pf}
Let $I$ a positive linear functional on $C_c(X)$.
Let $\cK$ be the family of all compact sets in $X$ and define $\mu_0:\cK\to[0,\infty]$ such that
\[\mu_0(K):=\inf\{I(f):1_K\le f\in C_c(X)\},\qquad K\in\cK.\]
\end{pf}


\begin{prb}[Riesz-Markov-Kakutani representation theorem]
Let $X$ be a locally compact Hausdorff space.
We want to establish the following one-to-one correspondence:
\[\begin{array}{ccc}
\{\text{Radon measures on $X$}\} & \xrightarrow{\sim} & \{\text{positive linear functionals on $C_c(X)$}\}\\
\mu & \mapsto & (f\mapsto\int f\,d\mu).
\end{array}\]
Let $I$ a positive linear functional on $C_c(X)$.
Let $\cT$ be the set of all open subsets of $X$ and $\mu_0:\cT\to[0,\infty]$ a set function defined such that
\[\mu_0(U):=\sup\{I(f):f\in C_c(U,[0,1])\},\qquad U\in\cT.\]
Let $\mu^*:\cP(X)\to[0,\infty]$ be the associated outer measure defined by
\[\mu^*(S):=\inf\left\{\sum_{i=1}^\infty\mu_0(U_i):S\subset\bigcup_{i=1}^\infty U_i,\ U_i\in\cT\right\},\qquad S\in\cP(X),\]
and let $\mu:=\mu^*|_\cA$ be the restriction, where $\cA$ is the $\sigma$-algebra of Carath\'eodory measurable subsets relative to $\mu^*$.
\begin{parts}
\item $\mu^*$ extends $\mu_0$.
\item $\mu$ extends $\mu_0$.
\item $\mu$ is a finite Radon measure.
\item The correspondence is surjective.
\item The correspondence is injective.
\end{parts}
\end{prb}
\begin{pf}
(a)
It suffices to show that $\mu_0$ satisfies monotonically countably subadditive.
For an open set $U$ and a countable open cover $\{U_i\}$ of $U$, we claim that $\mu_0(U)\le\sum_{i=1}^\infty\mu_0(U_i)$.

Take any $f\in C_c(U,[0,1])$ and find a finite subcover $\{U_j\}$ of $\{U_i\}$ together with a partition of unitiy $\{\chi_j\}$ subordinate to the open cover $\{U_j\cap\supp f\}$.
Now we have $f\chi_j\in C_c(U_j,[0,1])$ for each $j$, because then the functional $I$ is linear so that it preserves finite sum, we have
\[I(f)=\sum_j I(f\chi_j)\le\sum_j\mu_0(U_j)\le\sum_{i=1}^\infty\mu_0(U_i).\]
Since $f$ is arbitrary, the claim follows.

(b)
We claim $\cT\subset\cA$.
It suffices to show $\mu^*(E\cap U)+\mu^*(E\setminus U)\le\mu^*(E)$ for any measurable $E$ and open $U$.
Take $\e>0$.
Since we may assume $\mu^*(E)<\infty$, there is a countable open cover $\{U_i\}_{i=1}^\infty$ of $E$ such that
\[\sum_{i=1}^\infty\mu_0(U_i)<\mu^*(E)+\frac\e3.\]
Take $f_i\in C_c(U_i\cap U,[0,1])$ such that
\[\mu_0(U_i\cap U)<I(f_i)+\frac13\cdot\frac\e{2^i},\]
and take $g_i\in C_c(U_i\setminus\supp f_i,[0,1])$ such that
\[\mu_0(U_i\setminus\supp f_i)<I(g_i)+\frac13\cdot\frac\e{2^i}.\]
Then, since $f_i+g_i\in C_c(U_i,[0,1])$, we have
\begin{align*}
\mu^*(E\cap U)+\mu^*(E\setminus U)
&\le\sum_{i=1}^\infty\mu_0(U_i\cap U)+\sum_{i=1}^\infty\mu_0(U_i\setminus U)\\
&<\sum_{i=1}^\infty I(f_i+g_i)+\frac\e3+\frac\e3\\
&<\sum_{i=1}^\infty\mu_0(U_i)+\frac23\e\\
&\le\mu^*(E)+\e.
\end{align*}
Limiting $\e\to0$, we get the desired inequality.

(c)
Since $\mu$ is a countably additive and $\cT$ is closed under union, we can rewrite
\[\mu^*(S)=\inf\{\mu_0(U):S\subset U\in\cT\},\qquad S\in\cP(X),\]
hence $\mu$ is outer regular.
Here now we claim for $f\in C_c(X,[0,1])$ and $0<a<1$ that
\[a\mu(f^{-1}((a,1]))\le I(f)\le\mu(\supp f).\]
If it is true, then the right inequality implies the inner regularity, and the left inequality together with the Urysohn lemma implies the local finiteness.

The right inequality directly follows from the definition of $\mu_0$ and the outer regularity
\[I(f)\le\inf\{\mu_0(U):\supp f\subset U\in\cT\}=\mu(\supp f).\]
For the left, if $h\in C_c(f^{-1}((a,1]),[0,1])$, then the inequality $ah\le f$ implies
\[a\mu(f^{-1}((a,1]))=a\mu_0(f^{-1}((a,1]))\le aI(h)\le I(f).\]

(d)
We will show $I(f)=\int f\,d\mu$ for $f\in C_c(X)$.
Since $C_c(X)$ is the linear span of $C_c(X,[0,1])$, we may assume $f\in C_c(X,[0,1])$.
For a fixed positive integer $n$ and for each index $1\le i\le n$, let $K_i:=f^{-1}([i/n,1])$ and define
\[f_i(x):=\begin{cases}\frac1n&\text{ if }x\in K_i,\\f(x)-\frac{i-1}n&\text{ if }x\in K_{i-1}\setminus K_i,\\0&\text{ if }x\in X\setminus K_{i-1},\end{cases}\]
where $K_0:=\supp f$.
Note that $f_i\in C_c(X,[0,n^{-1}])$ and $f=\sum_{i=1}^nf_i$.
For $1\le i\le n$ we have $\mu(K_i)<\infty$ because $K_i$ is compact subsets contained in a locally compact Hausdorff space $U:=f^{-1}((0,1])$.
By the previous claim and the property of integral, we have
\[\frac{\mu(K_i)}n\le I(f_i),\qquad\frac{\mu(K_i)}n\le\int f_i\,d\mu,\qquad1\le i\le n\]
and
\[I(f_i)\le\frac{\mu(K_{i-1})}n,\qquad\int f_i\,d\mu\le\frac{\mu(K_{i-1})}n,\qquad2\le i\le n.\]
Then, using the above inequalities and $\mu(K_n)\ge0$, we have
\[I(f)\le I(f_1)+\int f\,d\mu\quad\text{and}\quad\int f\,d\mu\le\int f_1\,d\mu+I(f).\]
Note that $f_1=\min\{f,n^{-1}\}$ is a sequence of functions indexed by $n$.
By the monotone convergence theorem, $\int f_1\,d\mu\to0$ as $n\to\infty$.
We now show $I(f_1)$ converges to zero.
If we let $U:=f^{-1}((0,1])$, then $U$ is locally compact and $f_1\in C_0(U)\subset C_c(X)$, and since a positive linear functional on $C_0(U)$ is bounded, we have $I(f_1)\le n^{-1}\|I\|\to0$ as $n\to\infty$.
($\mu(K_0)$ is possibly infinite if $X$ is not locally compact so that $\mu$ is not locally finite.)

(e)
Let $\mu$ and $\nu$ be finite Radon measures on $X$ such that
\[\int g\,d\mu=\int g\,d\nu\]
for all $g\in C(X)$.
Let $E$ be any measurable set.
Since $\mu+\nu$ is a finite Radon measure, and by the Luzin theorem, we have a closed set $F$ and $g\in C(X)$ with $0\le g\le1$ such that $1_E|_F=g|_F$ and $(\mu+\nu)(X\setminus F)<\e/2$.
Then,
\begin{align*}
|\mu(E)-\nu(E)|&=|\int1_E\,d\mu-\int1_E\,d\nu\,|\\
&\le\int_{X\setminus F}|1_E-g|\,d\mu+\int_{X\setminus F}|g-1_E|\,d\nu\\
&\le2\mu(X\setminus F)+2\nu(X\setminus F)<\e.
\end{align*}
By limiting $\e\to0$, we have $\mu(E)=\nu(E)$.
\end{pf}



\begin{prb}[Riesz-Markov-Kakutani representation theorem]
Let $X$ be a locally compact Hausdorff space, and $\cK$ be the set of all compact subsets of $X$.
For a positive linear functional $I$ on $C_c(X)$, define $\mu_0:\cK\to[0,\infty)$ such that
\[\mu_0(K):=\inf\{I(f):1_K\le f\in C_c(X)\},\qquad K\in\cK.\]
\begin{parts}
\item For given $I$, $\mu_0$ is uniquely extended to a Radon measure on $X$.
\item There is one-to-one correspondence between positive linear functionals $I$ on $C_c(X)$ and Radon measures $\mu$ on $X$, via the relation $I(f)=\int f\,d\mu$ for $f\in C_c(X)$.
\end{parts}
\end{prb}
\begin{pf}
(a)
To apply the Carath\'eodory extension for inner measures to show $\mu_0$ is uniquely extended to a Borel measure, it is enough to check $\mu_0$ is continuous from above for downward directed nets, and every open set is Carath\'eodory measurable.

Let $K_i$ be a net of compact subsets of $X$ such that $K_i\downarrow K$.
Take $\e>0$ and $f\in C_c(X)$ such that $1_K\le f$ and $I(f)<\mu_0(K)+\e$.
If none of $i$ satisfies $1_{K_i}\le(1+\e)f$ so that we have a net $x_i\in X$ such that $x_i\in K_i$ and $1>(1+\e)f(x_i)$, then we may assume $x_i\to x$ in $X$ still with $K_i\downarrow K$ by taking a subnet.
We have $x\in K$ because $x\in K_i$ for all $i$, but the continuity of $f$ implies $1\ge(1+\e)f(x)\ge(1+\e)1_K(x)$, which is a contradiction to $x\in K$.
Therefore, we can choose $i_0$ such that $1_{K_i}\le(1+\e)f$ for $i\succ i_0$.
Then, the limit for $i$ and $\e$ on the inequality
\[\mu_0(K_i)\le(1+\e)I(f)<(1+\e)(\mu_0(K)+\e)\]
gets the continuity $\mu_0(K_i)\downarrow\mu_0(K)$.

Now we show that every open set is Carath\'eodory measurable.
It suffices to prove that for any compact $K$ and open $U$ in $X$ we have
\[\mu_0(K)\le\mu_*(K\cap U)+\mu_0(K\setminus U).\]
For $0<\e<1$, take $f_2\in C_c(X)$ such that $1_{K_2}\le f_2$ and $I(f_2)<\mu_0(K_2)+\e$, where $K_2:=K\setminus U$, and take $f_1\in C_c(X)$ such that $1_{K_1}\le f_1$ and $I(f_1)<\mu_0(K_1)+\e$, where $K_1:=K\cap f_2^{-1}([0,1-\e])$.
Then, $(1-\e)1_K\le f_1+f_2$ implies that we have
\[(1-\e)\mu_0(K)\le I(f_1+f_2)=I(f_1)+I(f_2)<\mu_0(K_1)+\mu_0(K_2)+2\e\le\mu_*(K\cap U)+\mu_0(K\setminus U)+2\e.\]
As $\e\to0$, the desired inequality follows.

We can finally check that the constructed Borel measure is locally finite and tight by definition of $\mu_0$, so it completes the proof.

(b)
\end{pf}


\begin{prb}[Dual of continuous function spaces]

signed measure
Hahn, Jordan decomposition

\end{prb}




\section*{Fremlin}


A measure $\mu$ is called \emph{inner regular} if for every measurable $E$ we have
\[\mu(E)=\sup\{\mu(F):F\subset E,\ \text{$F$ closed}\},\]
and called \emph{tight} if for every measurable $E$ we have
\[\mu(E)=\sup\{\mu(K):K\subset E,\ \text{$K$ compact}\}.\]
Note that the inner regularity by Folland or Rudin, outer regular for Borel and inner regular for open, is in fact the tightness, the inner regularity with respect to compact sets.

On a Tychonoff space $X$, $\Prob(X)$ is defined as the set of tight Borel probability measures so that there is an embedding $\Prob(X)\to\Prob(\beta X)$ defined as the pushforward.

We can try to define a \emph{Radon measure} on a Hausdorff space $X$ as a locally finite tight Borel measure.
Then, how to deal with regularity on Polish spaces?

\subsection{}



A \emph{quasi-Radon measure} on a Hausdorff space is a measure which is complete, locally determined, $\tau$-additive, inner regular with respect to closed sets, and effectively locally finite.
A \emph{Radon measure} on a Hausdorff space is a measure which is complete, locally determined, locally finite, and tight.
By the completeness condition, it is not Borel in general.
\begin{itemize}
\item 415A A quasi-Radon measure is strictly localizble.
\item 416C For a locally finite quasi-Radon measure $\mu$, $\mu$ is Radon iff 
\item 416F A Borel measure on a Hausdorff space has a Radon extension if and only if it is locally finite and tight, and in this case the extension is unique.
\item 416G A locally finite quasi-Radon measure is Radon.
\end{itemize}




Riesz-Markov-Kakutani 436J and 436K
\begin{pf}
First we can show $I$ is smooth(I think it is equivalent to normality).
Since $X$ is locally compact, it is the coarsest topology for which $C_c$ is continuous, i.e.~Baire$=$Borel.
Also, $C_c$ is truncated Riesz subspace of $\R^X$.
So 436H implies there is a quasi-Radon measure $\mu$ such that $I(f)=\int f\,d\mu$ for $f\in C_c$, which is clearly locally finite.
By 416G, $\mu$ is Radon.
\end{pf}

\begin{itemize}
\item A Radon measure is tight.
\item A $\sigma$-finite Folland-Radon measure on a locally compact Hausdorff space is tight. Moreover, Folland-Radon and Fremlin-Radon coincides on $\sigma$-compact locally compact Hausdorff spaces.
\item A locally finite Borel measure on a locally compact Hausdorff and second countable space is tight.
\item A locally compact Hausdorff and second countable space is Polish.
\item A tight measure on a topological space is always inner regular with respect to closed sets, and the converse is true on where???
\end{itemize}

Definitions
\begin{itemize}
\item A measurable algebra is called \emph{localizable} if the essential union exists even for uncountable family of measurable sets.
\item A \emph{localizble measure} is a semi-finite measure on a localizable measurable algebra.
\item A \emph{strictly localizable measure} or \emph{decomposable measure} is a measure which admits a partition $\{F_i\}$ of $X$, called the decomposition, such that $F_i$ are finite measurable and $E\cap F_i\in\Sigma$ for all $F_i$ implies $E\in\Sigma$ and $\mu(E)=\sum_{i\in J}\mu(E\cap F_i)$.
\item A \emph{locally determined measure} is a semi-finite measure such that $E\cap F\in\Sigma$ for any $F\in\Sigma$ of finite measure implies $E\in\Sigma$.(I think it is more natural to say a enhanced measurable space is locally determined by a semi-finite measure)
\end{itemize}

Locally finite measures
\begin{itemize}
\item A $\sigma$-finite measure is strictly localizable.
\item A strictly localizable measure is localizable and locally determined.
\item A tight measure on a topological space is $\tau$-additive.
\item A locally finite measure on a topological space is finite on compact sets.
\item A locally finite measure on a Lindel\"of space is $\sigma$-finite.
\item A locally finite and tight measure is effectively locally finite.
\item A effectively locally finite(non-negligible set has an open set of finite measure whose intersection with it is non-negligible) measure on a topological space is semi-finite.
\item
\end{itemize}

\section*{Folland}
\begin{prb}[Regular Borel measures on locally compact metric spaces]
sss
\begin{parts}
\item $C_c(X)$ is dense in $L^p(\mu)$ for $1\le p<\infty$.
\item If $\mu$ is $\sigma$-finite, then for any $\e>0$ there is compact $K\subset X$ and continuous $g:X\to\R$ such that $f|_K=g|_K$ and $\mu(X\setminus K)<\e$.
\end{parts}
\end{prb}

\begin{prb}[Tightness and inner regularity]
\begin{parts}
\item
\end{parts}
\end{prb}

\begin{prb}[Regular Borel measures on metric spaces]
Let $\mu$ be a Borel measure on a metric space $X$.
We say $\mu$ is \emph{outer regular} if
\[\mu(E)=\inf\{\mu(U):E\subset U,\,U\text{ open}\},\]
and say $\mu$ is \emph{inner regular} if
\[\mu(E)=\sup\{\mu(F):F\subset E,\,F\text{ closed}\},\]
for every Borel subset $E\subset X$.
If $\mu$ is both outer and inner regular, we say $\mu$ is \emph{regular}.
\begin{parts}
\item Let $E$ be $\sigma$-finite. Then, $E$ is $\mu$-regular if and only if for any $\e>0$ there are open $U$ and closed $F$ such that $F\subset E\subset U$ and $\mu(U\setminus F)<\e$.
\item If $\mu$ is $\sigma$-finite, then the set of $\mu$-regular subsets is a $\sigma$-algebra. (may be extended?)
\item Every closed set is $G_\delta$.
\item Every finite Borel measure on $X$ is regular.
\end{parts}
\end{prb}
\begin{pf}
\end{pf}




\begin{prb}[Luzin's theorem]
Let $\mu$ be a regular Borel measure on a metric space $X$.
Let $f:X\to\R$ be a Borel measurable function.
Two proofs: direct and Egoroff.
% Important properties: NORMALITY for Tietze, and $\sigma$-FINITENESS for U,F squeezing.
\begin{parts}
\item If $E\subset X$ is $\sigma$-finite, then there is a continuous $g$ blabla
\item If $f$ vanishes outside a $\sigma$-finite set, then for any $\e>0$ there is a closed set $F\subset X$ such that $f|_F:F\to\R$ is continuous and $\mu(X\setminus F)<\e$.
\item If $f$ vanishes outside a $\sigma$-finite set, then for any $\e>0$ there is a closed set $F\subset X$ and continuous $g:X\to\R$ such that $f|_F=g|_F$ and $\mu(X\setminus F)<\e$.
\item If $f$ is further bounded, then $g$ also can be taken to be bounded.
\end{parts}
\end{prb}
\begin{pf}
(a)
Let $\e>0$ and suppose $E\subset X$ is measurable with $\mu(E)<\infty$.
Since $E$ is $\sigma$-finite, we have open $U$ and closed $F$ such that $F\subset E\subset U$ and $\mu(U\setminus F)<\e/2$.
By the Urysohn lemma, there is a continuous function $g:X\to[0,1]$ such that $g|_{U^c}=0$ and $g|_F=1$.
Then,
\[\int|1_E-g|\,d\mu=\int_{U\setminus F}|1_E-g|\,d\mu\le2\mu(U\setminus F)<\e.\]

(b)
Since $\R$ is second countable, we have a base $(V_n)_{n=1}^\infty$ of $\R$.
Since $\mu$ is $\sigma$-finite, for each $n$ we can take open $U_n$ and closed $F_n$ such that
\[F_n\subset f^{-1}(V_n)\subset U_n\]
and $\mu(U_n\setminus F_n)<\e/2^n$.
Define $F:=\left(\bigcup_{n=1}^\infty(U_n\setminus F_n)\right)^c$ so that $\mu(X\setminus F)<\e$ and $F$ is closed.
Then,
\begin{align*}
U_n\cap F
&=U_n\cap((U_n^c\cup F_n)\cap F)\\
&=(U_n\cap(U_n^c\cup F_n))\cap F\\
&=(\varnothing\cup(U_n\cap F_n))\cap F\\
&\subset F_n\cap F
\end{align*}
proves $f^{-1}(V_n)$ is open in $F$ for every $n$, hence the continuity of $f|_F$.
(In fact, we require that $X$ to be just a topological space.)

(b')
We can alternatively use the part (a) and the Egoroff theorem.
By the part (a), we can construct a sequence $(f_n)$ of continuous functions $X\to\R$ such that $f_n\to f$ in $L^1$.
By taking a subsequence, we may assume $f_n\to f$ pointwise.
Assuming $\mu$ is finite, by the Egorov theorem, there is a measurable $A\subset X$ such that $f_n\to f$ uniformly on $A$ and $\mu(X\setminus A)<\e/2$.
Since $\mu$ is inner regular, we have closed $F\subset A$ such that $\mu(A\setminus F)<\e/2$, so that we have $\mu(X\setminus F)<\e$.
Then, $f$ is continuous on $A$, and of course on $F$.

\end{pf}



\begin{prop}
A $\sigma$-finite Radon measure is regular.
\end{prop}
\begin{pf}
First we approximate Borel sets of finite measure, with compact sets.
Let $E$ be a Borel set with $\mu(E)<\infty$ and $U$ be an open set containing $E$.
By outer regularity, there is an open set $V\supset U-E$ such that
\[\mu(V)<\mu(U-E)+\frac\e2.\]
By inner regularity, there is a compact set $K\subset U$ such that
\[\mu(K)>\mu(U)-\frac\e2.\]
Then, we have a compact set $K-V\subset K-(U-E)\subset E$ such that
\begin{align*}
\mu(K-V)&\ge\mu(K)-\mu(V)\\
&>\left(\mu(U)-\frac\e2\right)-\left(\mu(U-E)+\frac\e2\right)\\
&\ge\mu(E)-\e.
\end{align*}
It implies that a Radon measure is inner regular on Borel sets of finite measures.

Suppose $E$ is a $\sigma$-finite Borel set so that $E=\bigcup_{n=1}^\infty E_n$ with $\mu(E_n)<\infty$.
We may assume $E_n$ are pairwise disjoint.
Let $K_n$ be a compact subset of $E_n$ such that
\[\mu(K_n)>\mu(E_n)-\frac\e{2^n},\]
and define $K=\bigcup_{n=1}^\infty K_n\subset E$.
Then,
\[\mu(K)=\sum_{n=1}^\infty\mu(K_n)>\sum_{n=1}^\infty\left(\mu(E_n)-\frac\e{2^n}\right)=\mu(E)-\e.\]
Therefore, a Radon measure is inner regular on all $\sigma$-finite Borel sets.
\end{pf}






Continuous functions in $L^p$ spaces
Approximate identity
density




% 1. most general -> Radon
% 2. sigma-finite -> regular measures
% 3. loc finite, metrizable(=2nd cntbl) LCH -> Borel measures
% 4. finite, metrizable -> Borel measures

% pde, prob, spectral theorem -> 4.
% Haar meas on Lie group, number theory -> 3.
% c* algebra, choquet -> compact but bad topology -> 2.
% Haar meas on LCG -> may not be sigma-finite -> 1.

% 이제 증명들이 어느 정도 수준에서 가능한 지를 보자
% C_c version

locally finite tight measure.


\begin{prb}[Radon measures]
\begin{parts}
\item A $\sigma$-finite Radon measure is regular.
\item If every open subset of $X$ is $\sigma$-compact, then a locally finite Borel measure is Radon.
\item $C_c(X)$ is dense in $L^p(\mu)$ for $1\le p<\infty$.
\end{parts}
\end{prb}



\section{Metric measures}

Hausdorff measure, surface measure, Brunn-Minkowski inequality









\part{Distribution theory}

\chapter{Space of distributions}
\section{Test functions}

\begin{prb}[Test functions]
Let $\Omega\subset\R^d$ be an open subset.
Let $K$ be a compact subset of $\Omega$.
Let $\cD(\Omega)$ be the space of all smooth functions whose support is contained in $K$, endowed with a locally convex topology given by semi-norms $f\mapsto\|\partial^\alpha f\|$.
Then, $\cD_K(\Omega)$ is a Fr\'echet space.

\end{prb}





\section{Product distributions}




\begin{prb}[Extension by double adjoint]
\end{prb}

\begin{prb}[Translation?]
\end{prb}
\begin{prb}[Schwartz product]
Let $\f\in\cD(\Omega)$.
Then, the multiplication by $\bar\f$ defines a bounded linear operator $\cD(\Omega)\to\cD(\Omega)$.
By taking adjoint, it is extended to $\cD'(\Omega)\to\cD'(\Omega)$.
The bilinear map $\cD(\Omega)\times\cD'(\Omega)\to\cD'(\Omega)$ is called the \emph{Schwartz product}.
\end{prb}



\section{Kernels}
\begin{prb}[Schur test]
\end{prb}
\begin{prb}[Young's inequality of integral operators]
\end{prb}










\chapter{Fourier transform}

\section{Fourier transform of integrable functions}
There are three conventions for the Fourier transform:
\[\cF f(\xi)=\hat f(\xi):=\begin{cases}
\int_{\R^d}e^{-ix\xi}f(x)\,dx,\\
(2\pi)^{-\frac d2}\int_{\R^d}e^{-ix\xi}f(x)\,dx,\\
\int_{\R^d}e^{-2\pi ix\xi}f(x)\,dx.
\end{cases}\]
We will accept the second one.


\begin{prb}[Riemann-Lebesgue lemma]
basic estimates
\[\|\hat f\|_{L^\infty(\R^d)}\le(2\pi)^{-\frac d2}\|f\|_{L^1(\R^d)},\qquad f\in L^1(\R^d).\]
\[\|\hat f\|_{L^1(\R^d)}\lesssim\|f\|_{W^{d+1,1}(\R^d)},\qquad f\in W^{d+1,1}(\R^d).\]
Lp extension
\end{prb}

\begin{prb}[Fourier inversion]
inversion theorem
Plancerel
\[\|\hat f\|_{L^2(\R^d)}=\|f\|_{L^2(\R^d)},\qquad f\in L^2(\R^d).\]
unitarity
\end{prb}

\begin{prb}[Properties]
If $D:=-i\partial$, then
\[\cF D_x\cF^*=M_\xi,\qquad D_\xi\cF=-\cF M_x.\]
\[\cF(fg)=(2\pi)^{-\frac d2}\hat f*\hat g.\]
\end{prb}




\begin{prb}[Approximation of identity]
Fej\'er, Poisson, box?
\end{prb}
\begin{prb}[Summability methods]
\end{prb}

\section{Fourier transform of distributions}
A routine of Fourier transform computation of $f\in\cS'$:
\begin{enumerate}
\item Choose a sequence $f_n\in L^1$ such that $f_n\to f$ in $\cS'$.
\item Write $\cF f_n$ in the integral form.
\item Compute the limit of $\cF f_n$ in $\cS'$.
\end{enumerate}
Since $g_n\to g$ pointwise implies $g_n\to g$ in $\cS'$ if the sequence $g_n$ is dominated by a locally integrable with polynomial growth, we can frequently check the pointwise limit instead of $\cS'$.

Methods: approximate identity, indented contour, imaginary shift, Feynman's trick

Examples: $e^{-\frac12x^2}$, $e^{\frac i2x^2}$, $\pv\frac1x$, $\sgn(x)$, $1$, $\delta(x)$, $\mathrm{sinc}(\frac x2)$, $1_{[-\frac12,\frac12]}$, $\frac1{1+x^2}$






\section{Fourier series}
\begin{prb}
\[\|\hat f\|_{\ell^1(\Z)}\lesssim\|f\|_{W^{1,1+\e}(\T)}.\]
\end{prb}

Inversion theorem is an approximation problem given by $\cF^*\cF=\lim_{n\to\infty}\cF_n^*\cF$.
The condition $\hat f\in\ell^1(\Z)$ is a condition just for defining $\cF^*\hat f$ wihtout using distribution theory, and it does not affect the inversion phenomena.
The approximation, in other words, can be seen as an extension method for $\cF^*:\ell^1(\Z)\to C(\T)$ on $c_0(\Z)$.
Note that $\cF_n^*$ on $c_0(\Z)$ cannot be bounded directly without distribution theory, but $\cF_n^*\cF$ on $L^p(\T)$ can be bounded well.

\begin{itemize}
\item If $\cF_n^*$ is the standard partial sum, then $\cF_n^*\cF$ is the Dirichlet kernel.
\item If $\cF_n^*$ is the Ces\`aro mean, then $\cF_n^*\cF$ is the Fej\'er kernel.
\item If $\cF_r^*$ is the Abel sum, then $\cF_r^*\cF$ is the Poisson kernel.
\item In Fourier transform, we often use the Gauss-Weierstrass kernel.
\end{itemize}

The injectivity of $\cF$ is not an easy problem, which comes from the inversion theorem.



\begin{prb}[Dirichlet kernel]
The \emph{Dirichlet kernel} is a sequence of functions on the unit circle $\T$ defined such that the Fourier series is 
\[D_n=\hat{1_{|k|\le n}},\quad\text{or equivalently,}\quad\hat{D_n}=1_{|k|\le n}.\]
This is because they are invariant under inverse, in other words, they are even.
\begin{parts}
\item
\[D_n(x)=\frac{\sin\frac{2n+1}2x}{\sin\frac12x}.\]
\item If $f\in\Lip(\bT)$, then $D_n*f\to f$ pointwisely as $n\to\infty$.
\item
\[\|D_n\|_{L^1(\bT)}\gtrsim\log n.\]
\end{parts}
\end{prb}

\begin{pf}
\begin{align*}
D_n(x)&=\sum_{k=-n}^ne^{ikx}\\
&=\frac{e^{i\frac{2n+1}2x}-e^{-i\frac{2n+1}2x}}{e^{i\frac12x}-e^{-i\frac12x}}\\
&=\frac{\sin\frac{2n+1}2x}{\sin\frac12x}.
\end{align*}

(c)
By (2) $\sin x\le x$ for $x\in[0,\pi/2]$, (3) change of variable,
\begin{align*}
\|D_n\|_{L^1(\bT)}
&=\frac1{2\pi}\int_{-\pi}^\pi|\frac{\sin\frac{2n+1}2x}{\sin\frac12x}|\,dx\\
&\ge\frac2\pi\int_0^\pi\frac{|\sin\frac{2n+1}2x|}x\,dx\\
&=\frac2\pi\int_0^{\frac{2n+1}2\pi}\frac{|\sin x|}x\,dx\\
&=\frac2\pi\sum_{k=0}^{2n}\int_{\frac k2\pi}^{\frac{k+1}2\pi}\frac{|\sin x|}x\,dx\\
&\ge\frac2\pi\sum_{k=0}^{2n}\int_0^{\frac12\pi}\frac{\sin x}{\frac{k+1}2\pi}\,dx\\
&\ge\frac4{\pi^2}\sum_{k=0}^{2n}\frac1{1+k}\\
&\ge\frac4{\pi^2}\log(2n+2).
\end{align*}
..?
\end{pf}

\begin{prb}[Fej\'er kernel]
The \emph{Fej\'er kernel} is

\begin{parts}
\item
\[K_n(x)=\frac1{n+1}\frac{\sin^2\frac{n+1}2x}{\sin^2\frac12x}.\]
\end{parts}
\end{prb}
\begin{pf}
Since
\begin{align*}
D_n(x)=
&=\frac{e^{i\frac{2n+1}2x}-e^{-i\frac{2n+1}2x}}{e^{i\frac12x}-e^{-i\frac12x}}\\
&=\frac{[e^{i\frac{2n+1}2x}-e^{-i\frac{2n+1}2x}][e^{i\frac12x}-e^{-i\frac12x}]}{[e^{i\frac12x}-e^{-i\frac12x}]^2}\\
&=\frac{[e^{i(n+1)x}+e^{-i(n+1)x}]-[e^{inx}+e^{-inx}]}{[e^{i\frac12x}-e^{-i\frac12x}]^2},
\end{align*}
by telescoping, we get
\begin{align*}
\sum_{k=0}^nD_k(x)
&=\frac{[e^{i(n+1)x}+e^{-i(n+1)x}]-[e^{i0x}+e^{-i0x}]}{[e^{i\frac12x}-e^{-i\frac12x}]^2}\\
&=\frac{[e^{i\frac{n+1}2x}-e^{-i\frac{n+1}2x}]^2}{[e^{i\frac12x}-e^{-i\frac12x}]^2}\\
&=\frac{\sin^2\frac{n+1}2x}{\sin^2\frac12x}.
\end{align*}
\end{pf}

Two important results from Fej\'er kernel:
\begin{enumerate}
\item If $f(x-)$, $f(x+)$ exist and $S_nf(x)$ converges, then $S_nf(x)\to\frac12(f(x-)+f(x+))$.
\item (If $f\in L^1(\bT)$, then $\sigma_nf\to f$ a.e.)

\item If $f\in L^1(\bT)$, then $S_nf\to f$ in $L^1$ and $L^2$.
\item If $f$ is continuous and $\hat{f}\in L^1(\Z)$, then $S_nf\to f$ uniformly.
\item Since $\sigma_nf$ is a trigonometric polynomial, the set of trigonometric polynomials are dense in $L^1(\bT)$ and $L^2(\bT)$.
\end{enumerate}


BV function: Dini, Jordan's criterion
\begin{prb}[Riemann localization principle]
\end{prb}












\section*{Exercises}


\begin{prb}[Gibbs phenomenon]
\end{prb}
\begin{prb}[Du Bois-Reymond function]
\end{prb}


\begin{prb}[Sampling theorem]
\[\cF1_{[-\frac12,\frac12]}(\xi)=\operatorname{sinc}(\xi/2)\]
$\operatorname{sinc}\in L^{1+\e}(\R)$.
\end{prb}
\begin{prb}[Gausssian function]
Gaussian function computation:
differential equation method, contour integral method, imaginary shift
\[\cF e^{-x^2}\]
\[\cF e^{-\frac12xQx}=\frac{e^{\frac{i\pi}4\sgn(Q)}}{|\det Q|^{\frac12}}e^{-\frac12xQ^{-1}x}.\]
\[\cF\operatorname{sech}{(2\pi)^{\frac12}\frac x2}=\operatorname{sech}{(2\pi)^{\frac12}\frac x2}.\]
\end{prb}
\begin{prb}
\[\cF1=(2\pi)^{\frac12}\delta\]
\[\cF x=(2\pi)^{\frac12}i\delta'\]
\end{prb}
\begin{prb}[Poisson summation formula]
\end{prb}
\begin{prb}[Uncertainty principle]
\end{prb}



\begin{prb}[Paley-Wiener theorem]
Let $f$ be an integrable compactly supported function.
Using the Morera to prove $\hat f$ is analytic.

\end{prb}

\begin{prb}
Let $f\in C^\infty(\R)$ and define $f_n(x):=\sum_{k=0}^{n-1}\frac1{k!}f^{(k)}(0)x^k$.
Suppose $f_n\to f$ pointwise.
\begin{parts}
\item $f_n$ never converge to $f$ in the space of tempered distributions. (Hint: use Borel's theorem to construct a Schwarz function)
\item $f_n$ converges to $f$ in the the space of distributions.
\end{parts}
\end{prb}


\begin{prb}[Multipole expansion]
Let $\rho$ be a compactly supported distribution on $\R^d$.
We want to investigate the limit behavior of $\rho(\e^{-1}x)$ as $\e\to0$.
More precisely, we want to compute an integer $k\ge d$ such that $\lim_{\e\to0+}\e^{-k}\rho(\e^{-1}x)$ defines a distribution supported at $\{0\}$, and the coefficients of derivatives of Dirac measures.

We need to introduce quantities called monopole, dipole, quadrapole, octupole, etc.
\begin{parts}
\item A distribution supported on $\{0\}$ is a linear combination of the Dirac measure and its derivatives.
\item 
\end{parts}
\end{prb}

\section*{Problems}
\begin{enumerate}
\item Find all $\alpha>0$ such that
\[\lim_{x\to\infty}x^{-\alpha}\int_0^xf(y)\,dy=0\]
for all $f\in L^3([0,\infty))$.
\end{enumerate}
















\chapter{Lebesgue differentiation theorem}


\section{Absolutely continuous functions}

The space of weakly differentiable functions with respect to all variables $=W_\loc^{1,1}$.

\begin{prb}[Product rule for weakly differentiable functions]
We want to show that if $u$, $v$, and $uv$ are weakly differentiable with respect to $x_i$, then $\pd_{x_i}(uv)=\pd_{x_i}uv+u\pd_{x_i}v$.
\begin{parts}
\item If $u$ is weakly differentiable with respect to $x_i$ and $v\in C^1$, then $\pd_{x_i}(uv)=\pd_{x_i}uv+u\pd_{x_i}v$.
\end{parts}
\end{prb}


\begin{prb}[Interchange of differentiation and integration]
Let $f:X_x\times X_y\to\R$ be such that $\pd_{x_i}f$ is well-defined. Suppose $f$ and $\pd_{x_i}f$ are locally integrable in $x$ and integrable $y$.

Then,
\[\pd_{x_i}\int f(x,y)\,dy=\int\pd_{x_i}f(x,y)\,dy.\]
\end{prb}


Do not think the Schwarz theorem as the condition for partial differentiation to commute.
We should understand like this: if $F$ is $C^2$ then the \emph{classical} partial differentiation commute, and if $F$ is not $C^2$ then the \emph{classical} partial derivatives of order two or more are \emph{meaningless} because it is not compatible with the generalized concept of differentiation.




\begin{parts}
\item $f$ is $\Lip_\loc$ iff $f'$ is $L_\loc^\infty$
\item $f$ is $\textrm{AC}_\loc$ iff $f'$ is $L_\loc^1$
\end{parts}
\begin{parts}
\item $f$ is $\Lip$ iff $f'$ is $L^\infty$
\item $f$ is $\textrm{AC}$ iff $f'$ is $L^1$
\item $f$ is $\textrm{BV}$ iff $f'$ is a finite regular Borel measure
\end{parts}

\begin{prb}[Absolute continuous measures]
\end{prb}

\begin{prb}[Absolute continuous functions]
\end{prb}



\section{Functions of bounded variation}


\section{Interpolations}
Lorentz spaces
Weak $L^p$ spaces

\begin{defn}
Let $f$ be a measurable function on a measure space $(X,\mu)$.
The \emph{distribution function} $\lambda_f:[0,\infty)\to [0,\infty)$ is defined as:
\[\lambda_f(\alpha):=\mu(\{x:|f(x)|>\alpha\})=\mu(|f|>\alpha).\]
\end{defn}

Do not use $\mu(\{x:|f(x)|\ge\alpha\})$.
The strict inequality implies the \emph{lower semi-continuity} of $\lambda_f$.


For $p>0$,
\begin{align*}
\|f\|_{L^p}^p
&=\int|f(x)|^p\,d\mu(x)\\
&=\int\int_0^{|f(x)|}p\alpha^{p-1}\,d\alpha\,d\mu(x)\\
&=\int_0^\infty\int_{|f(x)|>\alpha}p\alpha^{p-1}\,d\mu(x)\,d\alpha\\
&=p\int_0^\infty\left[\alpha\cdot\mu(|f|>\alpha)^\frac1p\right]^p\,\frac{d\alpha}\alpha.
\end{align*}

\begin{defn}
\[\|f\|_{L^{p,q}}^q:=p\int_0^\infty\left[\alpha\cdot\mu(|f|>\alpha)^\frac1p\right]^q\,\frac{d\alpha}\alpha.\]
Also,
\[\|f\|_{L^{p,\infty}}:=\sup_{0<\alpha<\infty}\left[\alpha\cdot\mu(|f|>\alpha)^\frac1p\right].\]
\end{defn}
\begin{thm}
For $p\ge1$ we have $\|f\|_{p,\infty}\le\|f\|_p$.
\end{thm}
\begin{pf}
By the Chebyshev inequality,
\[\sup_{0<\alpha<\infty}\left[\alpha^p\cdot\mu(|f|>\alpha)\right]\le\int_0^\infty p\alpha^{p-1}\cdot\mu(|f|>\alpha)\,d\alpha=\|f\|_{L^p}^p.\]

\end{pf}

\begin{prb}[Marcinkiewicz interpolation]
Let $X$ be a $\sigma$-finite measure space and $Y$ be a measure space.
Let
\[1<p_0<p<p_1<\infty.\]
If a sublinear operator $T\colon L^{p_0}(X)+L^{p_1}(X)\to M(Y)$ has two weak-type estimates
\[\|T\|_{L^{p_0}(X)\to L^{p_0,\infty}(Y)}<\infty\quad\text{and}\quad\|T\|_{L^{p_1}(X)\to L^{p_1,\infty}(Y)}<\infty,\]
then it has a strong-type estimate
\[\|T\|_{L^p(X)\to L^p(Y)}<\infty.\]
\end{prb}
\begin{pf}
Let $f\in L^p(X)$ and denote $f_h=\chi_{|f|>\alpha}f$ and $f_l=\chi_{|f|\le\alpha}f$.
It is easy to show $f_h\in L^{p_0}$ and $f_l\in L^{p_1}$.
Then,
\begin{align*} % 고치던 중
\|Tf\|_{L^p(Y)}^p&\sim\int\alpha^p\cdot\mu(|Tf|>\alpha)\,\frac{d\alpha}\alpha\\
&\lesssim\int\alpha^p\cdot\mu(|Tf_h|>\alpha)\,\frac{d\alpha}\alpha+\int\alpha^p\cdot\mu(|Tf_l|>\alpha)\,\frac{d\alpha}\alpha\\
&\le\int\alpha^p\cdot\frac1{\alpha^{p_0}}\|Tf_h\|_{L^{p_0,\infty}}^{p_0}\,\frac{d\alpha}\alpha+\int\alpha^p\cdot\frac1{\alpha^{q_1}}\|Tf_l\|_{L^{p_1,\infty}}^{p_1}\,\frac{d\alpha}\alpha\\
&\lesssim\int\alpha^{p-p_0}\|f_h\|_{p_0}^{p_0}\,\frac{d\alpha}\alpha+\int\alpha^{p-p_1}\|f_l\|_{p_1}^{p_1}\,\frac{d\alpha}\alpha\\
&\sim\|f\|_p^p.
\end{align*}
by (1) Fubini, (2) Sublinearlity, (3) Chebyshev, (4) Boundedness, (5) Fubini.
\end{pf}

\begin{prb}[Hadamard's three line lemma]
Let $f$ be a bounded holomorphic function on vertical unit strip $\{z:0<\Re z<1\}$ which is continuously extended to the boundary.
Then, for $0<\theta<1$ we have
\[\|f\|_{L^\infty(\Re=\theta)}\le\|f\|_{L^\infty(\Re=0)}^{1-\theta}\|f\|_{L^\infty(\Re=1)}^\theta.\]
\end{prb}
\begin{pf}
Fix $n$ and define
\[g_n(z):=\frac{f(z)}{\|f\|_{L^\infty(\Re=0)}^{1-z}\|f\|_{L^\infty(\Re=1)}^z}e^{-\frac{z(1-z)}n}.\]
Then,
\[|g_n(z)|\le e^{-\frac{(\Im z)^2}n}\]
for $z$ in the strip.
By the maximum principle,
\[|f(z)|\le\|f\|_{L^\infty(\Re=0)}^{1-\theta}\|f\|_{L^\infty(\Re=1)}^\theta e^{\frac{y^2}n}.\]
Letting $n\to\infty$, we are done.
\end{pf}



\begin{prb}[Riesz-Thorin interpolation]
Let $X,Y$ be $\sigma$-finite measure spaces.
Let
\[\frac1{p_\theta}=(1-\theta)\frac1{p_0}+\theta\frac1{p_1},\qquad\frac1{q_\theta}=(1-\theta)\frac1{q_0}+\theta\frac1{q_1}.\]
Then,
\[\|T\|_{p_\theta\to q_\theta}\le\|T\|_{p_0\to q_0}^{1-\theta}\|T\|_{p_1\to q_1}^\theta.\]
\end{prb}
\begin{pf}
Note that
\[\|T\|_{p_\theta\to q_\theta}=\sup_f\frac{\|Tf\|_{q_\theta}}{\|f\|_{p_\theta}}=\sup_{f,g}\frac{|\<Tf,g\>|}{\|f\|_{p_\theta}\|g\|_{q'_\theta}}.\]
Consider a holomorphic function
\[z\mapsto\<Tf_z,g_z\>=\int\bar{g_z(y)}Tf_z(y)\,dy,\]
where $f_z$ and $g_z$ are defined as
\[f_z=|f|^{\frac{p_\theta}{p_0}(1-z)+\frac{p_\theta}{p_1}z}\frac f{|f|}\]
so that we have $f_\theta=f$ and
\[\|f\|_{p_\theta}^{p_\theta}=\|f_z\|_{p_x}^{p_x}\]
for $\Re z=x$.

Then,
\[|\<Tf_z,g_z\>|\le\|T\|_{p_0\to q_0}\|f_z\|_{p_0}\|g_z\|_{q'_0}=\|T\|_{p_0\to q_0}\|f\|_{p_\theta}^{p_\theta/p_0}\|g\|_{q'_\theta}^{q'_\theta/q'_0}\]
for $\Re z=0$, and
\[|\<Tf_z,g_z\>|\le\|T\|_{p_1\to q_1}\|f_z\|_{p_1}\|g_z\|_{q'_1}=\|T\|_{p_1\to q_1}\|f\|_{p_\theta}^{p_\theta/p_1}\|g\|_{q'_\theta}^{q'_\theta/q'_1}\]
for $\Re z=1$.
By Hadamard's three line lemma, we have
\[|\<Tf_z,g_z\>|\le\|T\|_{p_0\to q_0}^{1-\theta}\|T\|_{p_1\to q_1}^\theta\|f\|_{p_\theta}\|g\|_{q'_\theta}\]
for $\Re z=\theta$.
Putting $z=\theta$ in the last inequality, we get the desired result.
\end{pf}




\section{Hardy-Littlewood maximal function}

Let $T_m$ be a net of linear operators.
It seems to have two possible definitions of maximal functions:
\[T^*f:=\sup_m|T_mf|\]
and
\[T^*f:=\sup_{m,\ \e:|\e(x)|=1}|T_m(\e f)|.\]

\begin{prb}[Hardy-Littlewood maximal function]
The Hardy-Littlewood maximal function is just the maximal function defined with the approximate identity by the box kernel.
\end{prb}

\begin{prb}[Weak type estimate]
\[\|Mf\|_{1,\infty}\le 3^d\|f\|_{L^1(X)}.\]
\begin{parts}
\item Proof by covering lemma.
\end{parts}
\end{prb}
\begin{pf}
(a)
By the inner regularity of $\mu$, there is a compact subset $K$ of $\{|Mf|>\lambda\}$ such that
\[\mu(K)>\mu(\{|Mf|>\lambda\})-\e.\]
For every $x\in K$, since $|Mf(x)|>\lambda$, we can choose an open ball $B_x$ such that
\[\frac1{\mu(B_x)}\int_{B_x}|f|>\lambda\]
if and only if
\[\mu(B_x)<\frac1\lambda\int_{B_x}|f|.\]
With these balls, extract a finite open cover $\{B_i\}_i$ of $K$.
Since the diameter of elements in this cover is clearly bounded, so the Vitali covering lemma can be applied to obtain a disjoint subcollection $\{B_k\}_k$ such that
\[K\subset\bigcup_iBi\subset\bigcup_k3B_k.\]
Therefore,
\[\mu(K)
\le\sum_k3^d\mu(B_k)
\le\frac{3^d}\lambda\sum_k\int_{B_k}|f|
\le\frac{3^d}\lambda\|f\|_1.\]
The disjointness is important in the last inequality which shows the constant does not depend on the number of $B_k$'s.
\end{pf}



\begin{prb}[Radially bounded approximate identity]
If an approximate identity $K_n$ is radially bounded, then its maximal function is dominated by the Hardy-Littlewood maximal function:
\[\sup_n|K_n*f(x)|\lesssim Mf(x)\]
for every $n$ and $x$, hence has a weak type estimate.
\end{prb}


\begin{prb}[Almost everywhere convergence of operators]
Suppose is $T_m$ is a sequence of linear operators such that the maximal function $T^*f$ is dominated by $Mf$.
If $f\in L^1(X)$ and $T_mg\to g$ pointwise for $g\in C(X)$, then $T_mf\to f$ a.e.
\end{prb}
\begin{pf}
Take $\e>0$ and $g\in C(X)$ such that $\|f-g\|_{L^1(X)}<\e$.
Since $T_mg(x)\to g(x)$ pointwise, we have
\begin{align*}
&\mu(\{x:\limsup_m|T_mf(x)-f(x)|>\lambda\})\\
&\qquad\le\mu(\{x:\limsup_m|T_mf(x)-T_mg(x)|>\tfrac\lambda2\})
+\mu(\{x:|g(x)-f(x)|>\tfrac\lambda2\})\\
&\qquad\le\mu(\{x:M(f-g)(x)>\tfrac\lambda2\})+\frac2\lambda\|f-g\|_{L^1(X)}\\
&\qquad\lesssim\frac1\lambda\e
\end{align*}
for every $\lambda>0$.
Limiting $\e\to0$, we get
\[\mu(\{x:\limsup_m|T_mf(x)-f(x)|>\lambda\})=0\]
for every $\lambda>0$, hence the continuity from below implies
\[\mu(\{x:\limsup_m|T_mf(x)-f(x)|>0\})=0.\]
\end{pf}


\begin{defn}
\[f^*(x):=\lim_{r\to0+}\frac1{\mu(B)}\int_B|f(y)-f(x)|\,dy.\]
\end{defn}
\begin{thm}[Lebesgue differentiation]
$f^*=0$ a.e.
\end{thm}
\begin{pf}
Note that $f^*\le Mf+|f|$ implies
\[\|f^*\|_{1,\infty}\le\|Mf\|_{1,\infty}+\|f\|_{1,\infty}\lesssim\|f\|_1.\]
Note that $g^*=0$ for $g\in C_c$.
Approximate using $f^*=(f-g)^*$.
\end{pf}

\section*{Exercises}
\begin{prb}[Doubling measure]

\end{prb}


\end{document}


