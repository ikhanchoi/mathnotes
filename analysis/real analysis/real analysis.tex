\documentclass{../../large}
\usepackage{../../ikhanchoi}


\begin{document}
\title{Real Analysis}
\author{Ikhan Choi}
\maketitle
\tableofcontents

\part{Measur theory}


\chapter{Algebra of sets}


\section{Boolean algebras}



\begin{prb}[Boolean algebras and Boolean lattices]
\end{prb}



\section{Complete Boolean algebras}


\begin{prb}[Complete Boolean algebras]
\end{prb}


\begin{prb}[Measurable spaces]
A \emph{measurable space} or a \emph{Borel space} is a set $X$ together with a $\sigma$-complete subalgebra $\cA$ of the power set $\cP(X)$, which is called a \emph{$\sigma$-algebra} on $X$.
Each element of $\cA$ is called \emph{measurable}.
We often omit $\cA$ to just write $X$ for $(X,\cA)$ if there is no confusion.
\begin{parts}
\item generated by a set.
\item countable and cocountable sets
\item Borel
\item Loomis-Sikorski representation
\end{parts}
\end{prb}



\section{Standard Borel spaces}

descriptive set theory




\chapter{Measures}

\section{Measure spaces}

$E\in\cM$ and $A,B\in\cM_0$ and $S\in\cP(\Omega)$.

\begin{prb}[Measure spaces]
Let $(\Omega,\cM)$ be a measurable space.
A \emph{measure} on $(\Omega,\cM)$ is a set function $\mu:\cM\to[0,\infty]:\varnothing\mapsto0$ that is \emph{countably additive} in the sense that
\[\mu\Bigl(\bigsqcup_{i=1}^\infty E_i\Bigr)=\sum_{i=1}^\infty\mu(E_i),\qquad(E_i)_{i=1}^\infty\subset\cM.\]
Here the squared cup notation reads the disjoint union.
A \emph{measure space} is a triple $(\Omega,\cM,\mu)$, where $\mu$ is a measure on $(\Omega,\cM)$.
Let $\mu$ be a measure on $\Omega$.
\begin{parts}
\item $\mu$ is monotone: for $E,F\in\cM$ if $E\subset F$ then $\mu(E)\le\mu(F)$.
\item $\mu$ is countably subadditive: for
\item $\mu$ is continuous from below:
\item $\mu$ is continuous from above:
\end{parts}
\end{prb}


\begin{prb}[Complete measure spaces]
Let $(\Omega,\cM,\mu)$ be a measure space.
A \emph{negligible set} or a \emph{null set} is a measurable set $N$ satisfying $\mu(N)=0$, and a \emph{full set} is a measurable set whose complement is negligible.

A \emph{complete measure} is a measure such that every subset of a null set is measurable.

For a predicate $P$ of points $\omega\in\Omega$, we say $P$ is true \emph{almost everywhere} or \emph{a.e.} on $Omega$ if there is a negligible set $N\subset\Omega$ such that $P(\omega)$ is true for all $\omega\in\Omega\setminus N$.
\end{prb}


\begin{prb}[$\sigma$-finite measure spaces]
Let $(\Omega,\cM,\mu)$ be a \emph{$\sigma$-finite measure space}, which means that there is a countable cover of meaasurable sets of finite measure.
In most cases of mathematics, non-$\sigma$-finite measure spaces are rarely discussed.
\begin{parts}
\item There is a finite measure $\nu$ on $(\Omega,\cM)$ such that $\mu$ and $\nu$ are mutually absolutely continuous.
\item $\mu$ is semi-finite and the corresponding $\sigma$-complete Boolean algebra is complete.
\end{parts}
\end{prb}


\section{Carath\'eodory extension}

\begin{prb}[Outer measures]
Let $\Omega$ be a set.
An \emph{outer measure} on $\Omega$ is a set function $\mu^*:\cP(\Omega)\to[0,\infty]:\varnothing\mapsto0$ which is monotone and countably subadditive.
\begin{enumerate}[(i)]
\item $\mu^*$ is \emph{monotone}: we have
\[S_1\subset S_2\quad\Rightarrow\quad\mu^*(S_1)\le\mu^*(S_2),\qquad S_1,S_2\in\cP(\Omega),\]
\item $\mu^*$ is \emph{countably subadditive}: we have
\[\mu^*\Bigl(\bigcup_{i=1}^\infty S_i\Bigr)\le\sum_{i=1}^\infty\mu^*(S_i),\qquad (S_i)_{i=1}^\infty\subset\cP(\Omega).\]
\end{enumerate}
Compairing the definition of measures, we can see the outer measures extend the domain to the power set, but loosen the countable additivity to monotone countable subadditivity.
\begin{parts}
\item A set function $\mu^*:\cP(\Omega)\to[0,\infty]:\varnothing\mapsto0$ is an outer measure if and only if $\mu^*$ is \emph{monotonically countably subadditive}:
\[S\subset\bigcup_{i=1}^\infty S_i\quad\Rightarrow\quad\mu^*(S)\le\sum_{i=1}^\infty\mu^*(S_i),\qquad S\in\cP(\Omega),\ (S_i)_{i=1}^\infty\subset\cP(\Omega).\]
\item
For any $\varnothing\in\cM_0\subset\cP(X)$, let $\mu_0:\cM_0\to[0,\infty]:\varnothing\mapsto0$ be a set function.
We can associate an outer measure $\mu^*:\cP(\Omega)\to[0,\infty]$ by defining as
\[\mu^*(S):=\inf\left\{\,\sum_{i=1}^\infty\mu_0(B_i):S\subset\bigcup_{i=1}^\infty B_i,\ B_i\in\cM_0\,\right\},\]
where we use the convention $\inf\varnothing=\infty$.
\end{parts}
\end{prb}
\begin{pf}
\end{pf}


\begin{prb}[Carath\'eodory measurable sets]
Let $\mu^*$ be an outer measure on a set $X$.
We want to construct a measure by restriction of $\mu^*$ on a properly defined $\sigma$-algebra.
A subset $E\subset X$ is called \emph{Carath\'eodory measurable} relative to $\mu^*$ if
\[\mu^*(S)=\mu^*(S\cap E)+\mu^*(S\setminus E)\]
for every $S\in\cP(X)$.
Let $\cM\subset\cP(X)$ be the set of all Carath\'eodory measurable subsets relative to $\mu^*$.
\begin{parts}
\item $\cM$ is an algebra and $\mu^*$ is finitely additive on $\cM$.
\item $\cM$ is a $\sigma$-algebra and $\mu^*$ is countably additive on $\cM$. That is, $\mu:=\mu^*|_\cM$ is a measure.
\item The measure $\mu$ is complete.
\end{parts}
\end{prb}
\begin{pf}
\end{pf}


\begin{prb}[Carath\'eodory extension theorem]
The Carath\'eodory extension is a construction method for a measure extending a given set function $\mu_0$ on $\cM_0\subset\cP(\Omega)$ for a set $\Omega$.
The idea is to restrict the outer measure $\mu^*$ associated to $\mu_0$ in order to obtain a measure $\mu$.
We want to find a sufficient condition for $\mu$ to be a measure on a $\sigma$-algebra containing $\cM_0$.

Let $\varnothing\in\cM_0\subset\cP(\Omega)$, and let $\mu_0:\cM_0\to[0,\infty]$ be a set function with $\mu_0(\varnothing)=0$.
Let $\mu^*:\cP(\Omega)\to[0,\infty]$ be the associated outer measure of $\mu_0$, and $\mu:\cM\to[0,\infty]$ the measure defined by the restriction of $\mu^*$ on Carath\'eodory measurable subsets.
\begin{parts}
\item
$\mu^*$ extends $\mu_0$ if $\mu_0$ satisfies the monotone countable subadditivity: we have
\[A\subset\bigcup_{i=1}^\infty B_i\quad\Rightarrow\quad\mu_0(A)\le\sum_{i=1}^\infty\mu_0(B_i),\qquad A\in\cM_0,\ (B_i)_{i=1}^\infty\subset\cM_0\]
\item
$\mu$ extends $\mu_0$ if $\mu_0$ satisfies the following property in addition: for $B,A\in\cM_0$ and any $\e>0$, there are $(C_j)_{j=1}^\infty,(D_j)_{j=1}^\infty\subset\cM_0$ such that
\[B\cap A\subset\bigcup_{j=1}^\infty C_j,\quad B\setminus A\subset\bigcup_{j=1}^\infty D_j,\quad\sum_{j=1}^\infty(\mu_0(C_j)+\mu_0(D_j))<\mu_0(B)+\e.\]
\end{parts}
\end{prb}
\begin{pf}
(a)
Fix $A\in\cM_0$.
Clearly $\mu^*(A)\le\mu_0(A)$.
For the opposite direction, we may assume $\mu^*(A)<\infty$.
By the finiteness of $\mu^*(A)$, for any $\e>0$ we have $(B_i)_{i=1}^\infty\subset\cM_0$ such that $A\subset\bigcup_{i=1}^\infty B_i$ and
\[\sum_{i=1}^\infty\mu_0(B_i)<\mu^*(A)+\e.\]
Therefore we have $\mu_0(A)<\mu^*(A)+\e$ by the assumption, and we get $\mu_0(A)\le\mu^*(A)$ by limiting $\e\to0$.

(b)
Fix $A\in\cA_0$.
It is enough to check the inequality $\mu^*(S\cap A)+\mu^*(S\setminus A)\le\mu^*(S)$ for $S\in\cP(\Omega)$ with $\mu^*(S)<\infty$.
By the finiteness of $\mu^*(S)$, we have $(B_i)_{i=1}^\infty\subset\cB$ such that $S\subset\bigcup_{i=1}^\infty B_i$.
From the condition, we have $B_i\cap A\subset\bigcup_{j=1}^\infty C_{i,j}$ and $B_i\setminus A\subset\bigcup_{j=1}^\infty D_{i,j}$ satisfying
\begin{align*}
\mu^*(S\cap A)+\mu^*(S\setminus A)
&\le\mu^*\Bigl(\bigcup_{j=1}^\infty(B_i\cap A)\Bigr)+\mu^*\Bigl(\bigcup_{j=1}^\infty(B_i\setminus A)\Bigr)\\
&\le\sum_{i,j=1}^\infty(\mu_0(C_{i,j})+\mu_0(D_{i,j}))\\
&\le\sum_{i=1}^\infty(\mu_0(B_i)+2^{-i}\e)\\
&<\mu^*(S)+\e.
\end{align*}
Therefore, $A$ is Carath\'eodory measurable relative to $\mu^*$, so the domain of $\mu$ contains the domain of $\mu_0$.
The values coincide by the part (a).
\end{pf}


\begin{prb}[Uniqueness of extension of measures]
The Carath\'eodory extension also provides a uniqueness result for measure extensions.
Let $\mu_0:\cM_0\to[0,\infty]:\varnothing\mapsto0$ be a set function, where $\varnothing\in\cM_0\subset\cP(\Omega)$ for a set $\Omega$.
We say $\mu_0$ is \emph{$\sigma$-finite} if there is a cover $\{B_i\}_{i=1}^\infty\subset\cM_0$ of $\Omega$ such that $\mu_0(B_i)<\infty$ for each $i$.

Let $\cM$ be a $\sigma$-algebra containing $\cM_0$.
Let $\mu$ be a measure on $\cM$, which extends $\mu_0$, given by the restriction of the outer measure $\mu^*$ associated to $\mu_0$.
Let $\nu$ be another measure on $\cM$ which extends $\mu_0$.
Let $E\in\cM$ and $\{E_i\}_{i=1}^\infty\subset\cM$.
\begin{parts}
\item $\nu(E)\le\mu(E)$.
\item $\nu(E_i)=\mu(E_i)$ implies $\nu\Bigl(\bigcup_{i=1}^\infty E_i\Bigr)=\mu\Bigl(\bigcup_{i=1}^\infty E_i\Bigr)$.
\item $\nu(E)=\mu(E)$ for $\mu(E)<\infty$.
\item $\nu(E)=\mu(E)$ for $\mu(E)=\infty$, if $\mu_0$ is $\sigma$-finite
\end{parts}
\end{prb}
\begin{pf}
(a)
We may assume $\mu(E)<\infty$.
By the definition of the outer measure, there is $\{B_i\}_{i=1}^\infty\subset\cM_0$ such that $E\subset\bigcup_{i=1}^\infty B_i$.
Also, whenever $E\subset\bigcup_{i=1}^\infty B_i$ we have
\[\nu(E)\le\nu\Bigl(\bigcup_{i=1}^\infty B_i\Bigr)\le\sum_{i=1}^\infty\nu(B_i)=\sum_{i=1}^\infty\mu_0(B_i)=\sum_{i=1}^\infty\mu(B_i),\]
hence $\nu(E)\le\mu(E)$.

(b)
In the light of the inclusion-exclusion principle, we have
\[\mu(E_i\cup E_j)=\mu(E_i)+\mu(E_j)-\mu(E_i\cap E_j)\le\nu(E_i)+\nu(E_j)-\nu(E_i\cap E_j)=\nu(E_i\cup E_j),\]
so that $\mu(E_i\cup E_j)=\nu(E_i\cap E_j)$.
Applying it inductively, we have for every $n$ that
\[\mu\Bigl(\bigcup_{i=1}^nB_i\Bigr)=\nu\Bigl(\bigcup_{i=1}^nB_i\Bigr),\]
and by limiting $n\to\infty$ the continuity from below gives
\[\mu\Bigl(\bigcup_{i=1}^\infty B_i\Bigr)=\nu\Bigl(\bigcup_{i=1}^\infty B_i\Bigr).\]

(c)
Because $\mu(E)<\infty$, for any $\e>0$ we have a sequence $(B_i)_{i=1}^\infty\subset\cM_0$ such that $E\subset\bigcup_{i=1}^\infty B_i$ and
\[\sum_{i=1}^\infty\mu_0(B_i)<\mu(E)+\e.\]
Applying the part (b) 
Then, we have
\[\mu(E)\le\mu\Bigl(\bigcup_{i=1}^\infty B_i\Bigr)=\nu\Bigl(\bigcup_{i=1}^\infty B_i\Bigr)=\nu\Bigl(\bigcup_{i=1}^\infty B_i\setminus E\Bigr)+\nu(E)\]
and
\[\nu\Bigl(\bigcup_{i=1}^\infty B_i\setminus E\Bigr)
\le\mu\Bigl(\bigcup_{i=1}^\infty B_i\setminus E\Bigr)
=\mu\Bigl(\bigcup_{i=1}^\infty B_i\Bigr)-\mu(E)
\le\sum_{i=1}^\infty\mu(B_i)-\mu(E)=\sum_{i=1}^\infty\rho(B_i)-\mu(E)<\e,\]
we get $\mu(E)<\nu(E)+\e$ and $\mu(E)\le\nu(E)$ by limiting $\e\to0$.

(d)
Let $\{B_i\}_{i=1}^\infty\subset\cM_0$ be a cover of $X$ such that $\mu_0(B_i)<\infty$.
Define $E_1:=B_1$ and $E_n:=B_n\setminus\bigcup_{i=1}^{n-1}B_i$ for $n\ge2$ so that $\{E_i\}_{i=1}^\infty$ is a pairwise disjoint cover of $X$ with
\[\mu(E\cap E_i)\le\mu(E_i)\le\mu(B_i)=\mu_0(B_i)<\infty\]
for each $i$, so we have by the part (c) that
\[\nu(E)=\sum_{i=1}^\infty\nu(E\cap E_i)=\sum_{i=1}^\infty\mu(E\cap E_i)=\mu(E).\qedhere\]
\end{pf}



\section{Measures on Euclidean spaces}

Cantor set

\begin{prb}[Borel $\sigma$-algebra]
\end{prb}

\begin{prb}[Distribution functions]
\begin{parts}
\item Let $a<b\in\R_{\pm\infty}$. There is one-to-one correspondence between right continuous non-decreasing functions $F:[a,b]\to\R$ such that $F(a)=0$, $F(b)=1$, and the probability Borel measures on $[a,b]$.
\item 
\end{parts}
\end{prb}
\begin{pf}
We may assume $a>-\infty$
Suppose $(a,b]\subset\bigcup_{i=1}^\infty(a_i,b_i]$.
Using the right-continuity of $F$, for arbitrary $\e>0$, take $\e_i$ such that $F(bi+\e_i)-F(b_i)<\e2^{-i}$ for each $i$.
Then, by the Heine-Borel, there is $n$ such that $[a+\e,b]\subset\bigcup_{i=1}^n(a_i,b_i+\e_i)$, and we have
\[F(b)-F(a+\e)\le\sum_{i=1}^n(F(b_i+\e_i)-F(a_i)).\]
By limiting $\e\to0$, we have what we desired.

\end{pf}

\begin{prb}[Helly selection theorem]
\end{prb}

\begin{prb}[Vitali set]
\end{prb}


\section{Hausdorff measures}

Hausdorff measure, surface measure, Brunn-Minkowski inequality

\section*{Exercises}

\begin{prb}[Cardinalities]
infinite $\sigma$-algebra is $\ge\fc$.

\end{prb}

\begin{prb}[Semi-rings and semi-algebras]
We will prove a simplified Carath\'eodory extension with respect to \emph{semi-rings} and \emph{semi-algebras}.
Let $\cM_0\subset\cP(\Omega)$ such that $\varnothing\in\cM_0$.
We say that $\cM_0$ is a semi-ring if it is closed under finite intersections, and each relative complement is a finite union of elements of $\cM_0$.
We say that $\cM_0$ is a semi-algebra

Let $\cM_0$ be a semi-ring of sets over $X$.
Suppose a set function $\mu_0:\cM_0\to[0,\infty]:\varnothing\mapsto0$ satisfies
\begin{enumerate}[(i)]
\item $\mu_0$ is \emph{disjointly countably subadditive}: we have
\[\rho\Bigl(\bigsqcup_{i=1}^\infty A_i\Bigr)\le\sum_{i=1}^\infty\rho(A_i)\]
for $(A_i)_{i=1}^\infty\subset\cM_0$,
\item $\mu_0$ is \emph{finitely additive}: we have
\[\rho(A_1\sqcup A_2)=\rho(A_1)+\rho(A_2)\]
for $A_1,A_2\in\cM_0$.
\end{enumerate}
A set function satisfying the above conditions are occasionally called a \emph{pre-measure}.
\begin{parts}
\item
\item 
\end{parts}
\end{prb}

\begin{prb}[Monotone class lemma]
A collection $\cC\subset\cP(\Omega)$ is called a \emph{monotone class} if it is closed under countable increasing unions and countable decreasing intersections.

Let $H$ be a vector space closed under bounded monotone convergence.
If $\spn\{1_A:A\in\cM\}\subset H$ then $B^\infty(\sigma(\cM)\subset H$.
\end{prb}






\begin{prb}[Steinhaus theorem]
Let $\lambda$ denote the Lebesgue measure on $\R$ and let $\E\subset\R$ be a Lebesgue measurable set with $\lambda(E)>0$.
\begin{parts}
\item For any $0<\alpha<1$, there is an interval $I=(a,b)$ such that $\lambda(E\cap I)>\alpha\lambda(I)$.
\item $E-E=\{x-y:x,y\in E\}$ contains an open interval containing zero.
\end{parts}
\begin{pf}
(a)
We may assum $\lambda(E)<\infty$.
Since $\lambda$ is outer measure and $\lambda(E)\ne0$, we have an open subset $U$ of $\R$ such that $\lambda(U)<\alpha^{-1}\lambda(E)$.
Because $U$ is a countable disjoint union of open intervals $U=\bigsqcup_{i=1}^\infty(a_i,b_i)$, we have
\[\sum_{i=1}^\infty\lambda((a_i,b_i))=\lambda(U)<\alpha^{-1}\lambda(E)=\alpha^{-1}\sum_{i=1}^n\lambda(E\cap(a_i,b_i)).\]
Therefore, there is $i$ such that $\alpha\lambda((a_i,b_i))<\lambda(E\cap(a_i,b_i))$.
\end{pf}
% convolution으로 푸는 방법: continuous approximation 이 레벨에선 무리인듯
\end{prb}


\begin{prb}[Measures from volume forms]
	
\end{prb}


\section*{Problems}
\begin{enumerate}
\item* Every Lebesgue measurable set in $\R$ of positive measure contains an arbitrarily long arithmetic progression.
\end{enumerate}

















\chapter{Lebesgue integral}



\section{Measurable functions}

simple function approximations, convergence in measure

\begin{prb}[Measurability of pointwise limits]

Conversely, every measurable extended real-valued function is a pointwise limit of simple functions.

\end{prb}
\begin{pf}
Let $f(x)=\lim_{n\to\infty}s_n(x)$.

\end{pf}



\begin{prb}[Almost everywhere convergence]
Let $(X,\mu)$ be a measure space and let $f_n:X\to\bar\R$ and $f:X\to\bar\R$ be measurable functions.
The set of convergence of the sequence $f_n$ is defined as the set
\[\{x\in X:\lim_{n\to\infty}f_n(x)=f(x)\},\]
and the set of divergence is defined as its complement.
We say $f_n$ converges to $f$ \emph{alomst everywhere} with respect to $\mu$ if the set of divergence is a null set in $\mu$.
We simply write
\[f_n\to f\text{ a.e.}\]
if $f_n$ converges to $f$ almost everywhere, and we frequently omit the measure $\mu$ if it has no confusion.
\begin{parts}
\item If $\mu$ is complete and, if $f_n\to f$ a.e., then $f$ is measurable.
\end{parts}
\end{prb}

\begin{prb}[Borel-Cantelli lemma]
Let $(X,\mu)$ be a measure space and let $f_n:X\to\bar\R$ and $f:X\to\bar\R$ be a sequence of measurable functions.
Note that the set of divergence is given by
\[\bigcup_{\e>0}\bigcap_{N\ge0}\bigcup_{n>N}\{x:|f_n(x)-f(x)|\ge\e\}.\]
Each measurable set of the form
\[\{x:|f_n(x)-f(x)|\ge\e\}\]
is sometimes called the \emph{tail event}, coined in probability theory.
\begin{parts}
\item $f_n\to f$ a.e. if and only if for each $\e>0$ we have
\[\mu(\{x:\limsup_{n\to\infty}|f_n(x)-f(x)|\ge\e\})=0.\]
\item $f_n\to f$ a.e. if and only if for each $\e>0$ we have
\[\mu(\limsup_{n\to\infty}\{x:|f_n(x)-f(x)|\ge\e\})=0.\]
\item $f_n\to f$ a.e. if for each $\e>0$ we have
\[\sum_{n=1}^\infty\mu(\{x:|f_n(x)-f(x)|\ge\e\})<\infty.\]
\end{parts}
\end{prb}
\begin{pf}
(b)
The set of divergence of the sequence $f_n$ is given by
\[\bigcup_{m=1}^\infty\bigcap_{n=1}^\infty\bigcup_{i=n}^\infty\{x:|f_i(x)-f(x)|\ge\tfrac1m\}=\bigcup_{m=1}^\infty\,\bigcap_{n=1}^\infty(X\setminus E_n^m).\]

(c)
Since
\[\mu\Bigl(\bigcup_{i=1}^\infty\{x:|f_i(x)-f(x)|\ge\e\}\Bigr)\le\sum_{i=1}^\infty\mu(\{x:|f_i(x)-f(x)|\ge\e\})<\infty,\]
we have by the continuity from above that
\begin{align*}
\mu(\limsup_{n\to\infty}\{x:|f_n(x)-f(x)|\ge\e\})
&=\mu\Bigl(\bigcap_{n=1}^\infty\bigcup_{i=n}^\infty\{x:|f_i(x)-f(x)|\ge\e\}\Bigr)\\
&=\lim_{n\to\infty}\mu\Bigl(\bigcup_{i=n}^\infty\{x:|f_i(x)-f(x)|\ge\e\}\Bigr)\\
&\le\lim_{n\to\infty}\sum_{i=n}^\infty\mu(\{x:|f_i(x)-f(x)|\ge\e\})
=0.\qedhere
\end{align*}
\end{pf}

\begin{prb}[Convergence in measure]
Let $(X,\mu)$ be a $\sigma$-finite measure space and let $f_n:X\to\bar\R$ be a sequence of measurable functions.
We say $f_n$ converges to a measurable function $f:X\to\bar\R$ \emph{in measure} if for each $\e>0$ we have
\[\lim_{n\to\infty}\mu(\{x:|f_n(x)-f(x)|\ge\e\})=0.\]
\begin{parts}
\item If $f_n$ is a non-decreasing sequence, then $f_n$ converges locally in measure.
\item If $f_n\to f$ locally in measure, then $f_n$ has a subsequence convergent to $f$ a.e.
\item If every subsequence of $f_n$ has a further subsequence convergent to $f$ a.e., then $f_n\to f$ locally in measure.
\end{parts}
\end{prb}
\begin{pf}
(b)
Since for each positive integer $k$ we have $\mu(\{x:|f_n(x)-f(x)|\ge\frac1k\})\to0$ as $n\to\infty$, there exists $n_k$ such that
\[\mu(\{x:|f_{n_k}(x)-f(x)|\ge\tfrac1k\})<\frac1{2^k}.\]
By the Borel-Cantelli lemma, we get
\[\mu(\limsup_{k\to\infty}\{x:|f_{n_k}(x)-f(x)|\ge\tfrac1k\})=0.\]
Then, for each $\e>0$,
\begin{align*}
\limsup_{k\to\infty}\{x:|f_{n_k}(x)-f(x)|\ge\e\}
&=\bigcap_{k=\lceil\e^{-1}\rceil}^\infty\bigcup_{j=k}^\infty\{x:|f_{n_j}(x)-f(x)|\ge\e\}\\
&\subset\bigcap_{k=\lceil\e^{-1}\rceil}^\infty\bigcup_{j=k}^\infty\{x:|f_{n_j}(x)-f(x)|\ge\tfrac1k\}\\
&=\limsup_{k\to\infty}\{x:|f_{n_k}(x)-f(x)|\ge\tfrac1k\}
\end{align*}
implies the limit superior of the tail events is a null set, hence $f_{n_k}\to f$ a.e.

(c)
\end{pf}

\begin{prb}[Egorov theorem]
Egorov's theorem informally states that an almost everywhere convergent functional sequence is ``almost'' uniformly convergent.
Through this famous theorem, we introduce a convenient ``$\e/2^m$ argument'', occasionally used throughout measure theory to construct a measurable set having a special property.

Let $(X,\mu)$ be a finite measure space and let $f_n:X\to\bar\R$ be a sequence of measurable functions such that $f_n\to f$ a.e.
For each positive integer $m$, which indexes the tolerance $1/m$, consider an increasing sequence of measurable subsets
\[E_n^m:=\bigcap_{i=n}^\infty\{x:|f_i(x)-f(x)|<\tfrac1m\}.\]
\begin{parts}
\item $E_n^m$ converges to a full set for each $m$.
\item For every $\e>0$ there is a measurable $K\subset X$ such that $\mu(X\setminus K)<\e$ and for each $m$ there is finite $n$ satisfying $K\subset E_n^m$.
\item For every $\e>0$ there is a measurable $K\subset X$ such that $\mu(X\setminus K)<\e$ and $f_n\to f$ uniformly on $K$.
\end{parts}
\end{prb}
\begin{pf}
(a)
Recall that the a.e. convergence $f_n\to f$ means that for every fixed $m$ the intersection
\[\bigcap_{n=1}^\infty(X\setminus E_n^m)=\limsup_n\{x:|f_n(x)-f(x)|\ge\tfrac1m\}\]
is a null set.
Since $\mu(X)<\infty$, it is equivalent to $E_n^m$ converges to a full set for each $m$ by the continuity from above.

(b)
For each $m$, we can find $n_m$ such that
\[\mu(X\setminus E_{n_m}^m)<\frac\e{2^m}.\]
If we define
\[K:=\bigcap_{m=1}^\infty E_{n_m}^m,\]
then it satisfies the second conclusion, and also have
\[\mu(X\setminus K)=\mu\Bigl(\bigcup_{m=1}^\infty(X\setminus E_{n_m}^m)\Bigr)\le\sum_{m=1}^\infty\mu(X\setminus E_{n_m}^m)<\sum_{m=1}^\infty\frac\e{2^m}=\e.\]


(c)
Fix $m>0$.
Since $n\ge n_m$ implies $K\subset E_{n_m}^m\subset E_n^m$, we have
\[n\ge n_m\quad\Rightarrow\quad\sup_{x\in K}|f_n(x)-f(x)|<\frac1m.\qedhere\]
\end{pf}







\section{Convergence theorems}

% Stein: Egorov $\to$ BCT $\to$ Fatou $\to$ MCT $\to$ L1<M\\
% Stein: BCT + L1<M $\to$ DCT\\
% Folland: MCT $\to$ Fatou $\to$ DCT $\to$ BCT

\begin{prb}[Lebesgue integral of non-negative functions]
Let $(X,\mu)$ be a measure space.
Let $f:X\to[0,\infty)$ be a measurable function.
The \emph{Lebesgue integral} of $f$ is defined by
\[\int f\,d\mu:=\sup\left\{\int s\,d\mu:0\le s\le f,\ \text{$s$ simple}\right\}\]
\end{prb}

\begin{prb}[Monotone convergence theorem]
Let $(X,\mu)$ be a measure space.
Let $(f_n)$ be a non-decreasing sequence of measurable functions $X\to[0,\infty)$.
\begin{parts}
\item $E\mapsto\int_Ef\,d\mu$ is a measure.
\item $\int\sup_nf_n\,d\mu=\sup_n\int f_n\,d\mu$.
\end{parts}
\end{prb}
\begin{pf}
(a)
The map $E\mapsto\int_Ef\,d\mu$ is a measure if $f$ is simple, from the linearity of the integral for simple functions.
For $E_n\uparrow E$, we want to show the continuity from below, $\tint_{E_n}f\to\tint_Ef$.
Take $\e>0$.
We introduce a continuous bijection $\beta:[0,\infty]\to[0,1]:t\mapsto t/(1+t)$ to avoid dividing the cases for infinity.
By the definition of the Lebesgue integral, we have a simple function $s$ such that $0\le s\le f$ and
\[\beta(\tint_Ef)-\beta(\tint_Es)<\e,\]
whether or not $\int_Ef$ diverges.
Then,
\begin{align*}
\beta(\tint_Ef)-\beta(\tint_{E_n}f)
&=[\beta(\tint_Ef)-\beta(\tint_Es)]
+[\beta(\tint_Es)-\beta(\tint_{E_n}s)]
+[\beta(\tint_{E_n}s)-\beta(\tint_{E_n}f)]\\
&<\e+[\beta(\tint_Es)-\beta(\tint_{E_n}s)]+0
\xrightarrow{n\to\infty}\e.
\end{align*}
%\begin{alignat*}{4}
%\beta(\tint_Ef)-\beta(\tint_{E_n}f)
%&=[\beta(\tint_Ef)-\beta(\tint_Es)]
%&&+[\beta(\tint_Es)-\beta(\tint_{E_n}s)]
%&&+[\beta(\tint_{E_n}s)-\beta(\tint_{E_n}f)]\\
%&\ccol{<}{\e}
%&&+[\beta(\tint_Es)-\beta(\tint_{E_n}s)]
%&&\ccol{+}{0}\\
%&\xrightarrow{n\to\infty}\e
%\end{alignat*}
We are done by letting $\e\to0$.

(b)
For any $\e>0$ let $E_n:=\{x:f(x)<(1+\e)f_n(x)\}$, which converges to a full set because $f_n\to f$ a.e.
Since $f$ is a measure, we can choose $N$ such that
\[\beta(\tint_Ef)-\beta(\tint_{E_N}f)<\e.\]
With this $N$, we have
\[\beta(\tint_{E_N}f)
\le\beta((1+\e)\tint_{E_N}f_n)
\le(1+\e)\beta(\tint_{E_N}f_n)
\le\beta(\tint_{E_N}f_n)+\e,\qquad n>N.\]
Then, we have for $n>N$ that
\begin{align*}
\beta(\tint_Ef)-\beta(\tint_Ef_n)
&=[\beta(\tint_Ef)-\beta(\tint_{E_N}f)]+[\beta(\tint_{E_N}f)-\beta(\tint_{E_N}f_n)]+[\beta(\tint_{E_N}f_n)-\beta(\tint_{E}f_n)]\\
&<\e+\e+0,
\end{align*}
so we are done by letting $n\to\infty$ and $\e\to0$.
\end{pf}

\begin{prb}[Corollaries of monotone convergence theorem]
Fatou's lemma, linearity of the integral, $f\ge0$ and $\int f=0$ imply $f=0$ a.e.
\end{prb}




\begin{prb}[Lebesgue integral of complex-valued functions]
\end{prb}

\begin{prb}[Bounded convergence theorem]
Semifinite measures

Let $f_n$ be a sequence of measurable functions such that $\sup_n\sup_x|f_n(x)|\le1$ and $f_n\to f$ locally in measure.
\begin{parts}
\item
\[\sup_{g\le f}\int g\,d\mu=\int f\,d\mu\]
where $g$ runs through bounded measurable functions.
\item
\end{parts}
\end{prb}


\section{Product measures}

\begin{prb}[Fubini-Tonelli theorem]
Lebesgue measure on Euclidean spaces
\end{prb}


Lipschitz and differentiable transformations



\section{Integrals on Euclidean spaces}




\section*{Exercises}
\begin{prb}[Cauchy's functional equation]
Let $f:\R\to\R$ be a function.
Cauchy's functional equation refers to the equation $f(x+y)=f(x)+f(y)$, satisfied for all $x,y\in\R$.
Suppose $f$ satisfies the Cauchy functional equation.
We ask if $f$ is linear, that is $f(x)=ax$ for all $x\in\R$, where $a:=f(1)$.
\begin{parts}
\item $f(x)=ax$ for all $x\in\Q$, but there is a nonlinear solution of Cauchy's functional equation.
\item If $f$ is conitnuous at a point, then $f$ is linear.
\item If $f$ is Lebesgue measurable, then $f$ is linear.
\end{parts}
\end{prb}

\begin{prb}[Pointwise approximation by simple functions]
Let $(X,\mu)$ be a measure space and $X$ a metric space with Borel measurable structure.
By a \emph{simple function} we mean a measurable function $s:X\to X$ of finite image.
\begin{parts}
\item For each open set $U\subset X$ there is a sequence of open sets $U_i$ such that $U=\bigcup_iU_i$ and $\bar U_i\subset U$.
Let $f:X\to X$ be any function.
\item If $f$ is the pointwise limit of a sequence of measurable functions, then $f$ is measurable.
\item If $f$ is measurable, then $f$ is the pointwise limit of a sequence of simple functions, if $X$ is separable.
\item* The pointwise limit of a net of simple functions may not be measurable.
\end{parts}
\end{prb}
\begin{pf}

(b)
Suppose a sequence $(f_n)_n$ of measurable functions converges pointwisely to a function $f$.
For fixed open $U\subset X$ we claim
\[f^{-1}(U)=\bigcup_{i=1}^\infty\ \liminf_{n\to\infty}\ f_n^{-1}(U_i).\]
If it is true, then $f^{-1}(U)$ is the countable set operation of measurable sets $f_n^{-1}(U_i)$.
Let $U_i$ be the sequence associated to $U$ taken by the part (a).

($\subset$) If $\omega\in f^{-1}(U)$, then for some $i$ we have $f(\omega)\in U_i$, so $f_n(\omega)$ is eventually in $U_i$, thus we have $\omega\in\liminf_{n\to\infty}f_n^{-1}(U_i)$.

($\supset$) If $\omega\in\liminf_{n\to\infty}f_n^{-1}(U_i)$ for some $i$, then $f_n(\omega)$ is eventually in $U_i$, so $f(\omega)\in\bar U_i\subset U$, thus we have $\omega\in f^{-1}(U)$.

(c)
Suppose there is a increasing sequence of finite tagged partitions $\cP_n\subset\cB$ satisfying the following property: for each open-neighborhood pair $(x,U)$ there is $n$ and $i$ such that $P_{n,i}\in\cP_n$ and $x\in P_{n,i}\subset U$.
We denote the tags by $t_{n,i}\in P_{n,i}$ for each $P_{n,i}\in\cP_n$.
Define
\[s_n(\omega):=t_{n,i}\quad\text{for}\quad f(\omega)\in P_{n,i}.\]
To show $s_n(\omega)\to f(\omega)$, fix an open $f(\omega)\in U\subset X$.
Then, there is $n_0$ such that there is a sequence $(P_{n,i_n})_{n=n_0}^\infty$ satisfying $P_{n,i_n}\in\cP_n$ and $f(\omega)\in P_{n,i_n}\subset U$.
Then, for all $n\ge n_0$, we have for $f(\omega)\in P_{n,i_n}$ that $s_n(\omega)=t_{n,i_n}\in P_{n,i_n}\subset U$.

The existence of such sequence of partitions...

Another approach: mimicking Pettis measurability theorem.
\end{pf}




\begin{prb}[Convergence of one-parameter family]
\end{prb}



If $\|f_n\|_{L^2([0,1])}\le C$ and $f_n\to f$ almost everywhere, then $f_n\to f$ weakly.

\[\lim_{n\to\infty}\int_0^1n^3x^2(1-x)^n\,dx=2\ne0=\int_0^1\lim_{n\to\infty}n^3x^2(1-x)^n\,dx.\]
\[\lim_{n\to\infty}\int_0^\infty n^2e^{-nx}\,dx=\infty\ne0=\int_0^\infty\lim_{n\to\infty}n^2e^{-nx}\,dx.\]










\part{Function spaces}


\chapter{Lebesgue spaces}
\section{}

\begin{prb}[H\"older inequality]
\end{prb}
\begin{pf}
\[\int fg\le C^p\int\frac{|f|^p}p+\frac1{C^q}\int\frac{|g|^q}q\]
Take $C$ such that
\[C^p\int\frac{|f|^p}p=\frac1{C^q}\int\frac{|g|^q}q.\]
Then,
\[C^p\int\frac{|f|^p}p+\frac1{C^q}\int\frac{|g|^q}q=2p^{-\frac1p}q^{-\frac1q}\Bigl(\int|f|^p\Bigr)^{\frac1p}\Bigl(\int|g|^p\Bigr)^{\frac1q}.\]
Note that we can show that $1\le2p^{-\frac1p}q^{-\frac1q}\le2$ and the minimum is attained only if $p=q=2$, so this method does not provide the sharpest constant.
\end{pf}



\section{Convolutions}

\begin{prb}[Convolution?]
\end{prb}
\begin{prb}[Approximate identity?]
\end{prb}
\begin{prb}[Continuity of translation?]
\end{prb}



\section{Interpolations}

Lorentz spaces
Weak $L^p$ spaces

\begin{defn}
Let $f$ be a measurable function on a measure space $(X,\mu)$.
The \emph{distribution function} $\lambda_f:[0,\infty)\to [0,\infty)$ is defined as:
\[\lambda_f(\alpha):=\mu(\{x:|f(x)|>\alpha\})=\mu(|f|>\alpha).\]
\end{defn}

Do not use $\mu(\{x:|f(x)|\ge\alpha\})$.
The strict inequality implies the \emph{lower semi-continuity} of $\lambda_f$.


For $p>0$,
\begin{align*}
\|f\|_{L^p}^p
&=\int|f(x)|^p\,d\mu(x)\\
&=\int\int_0^{|f(x)|}p\alpha^{p-1}\,d\alpha\,d\mu(x)\\
&=\int_0^\infty\int_{|f(x)|>\alpha}p\alpha^{p-1}\,d\mu(x)\,d\alpha\\
&=p\int_0^\infty\left[\alpha\cdot\mu(|f|>\alpha)^\frac1p\right]^p\,\frac{d\alpha}\alpha.
\end{align*}

\begin{defn}
\[\|f\|_{L^{p,q}}^q:=p\int_0^\infty\left[\alpha\cdot\mu(|f|>\alpha)^\frac1p\right]^q\,\frac{d\alpha}\alpha.\]
Also,
\[\|f\|_{L^{p,\infty}}:=\sup_{0<\alpha<\infty}\left[\alpha\cdot\mu(|f|>\alpha)^\frac1p\right].\]
\end{defn}
\begin{thm}
For $p\ge1$ we have $\|f\|_{p,\infty}\le\|f\|_p$.
\end{thm}
\begin{pf}
By the Chebyshev inequality,
\[\sup_{0<\alpha<\infty}\left[\alpha^p\cdot\mu(|f|>\alpha)\right]\le\int_0^\infty p\alpha^{p-1}\cdot\mu(|f|>\alpha)\,d\alpha=\|f\|_{L^p}^p.\]

\end{pf}

\begin{prb}[Marcinkiewicz interpolation]
Let $X$ be a $\sigma$-finite measure space and $Y$ be a measure space.
Let
\[1<p_0<p<p_1<\infty.\]
If a sublinear operator $T\colon L^{p_0}(X)+L^{p_1}(X)\to M(Y)$ has two weak-type estimates
\[\|T\|_{L^{p_0}(X)\to L^{p_0,\infty}(Y)}<\infty\quad\text{and}\quad\|T\|_{L^{p_1}(X)\to L^{p_1,\infty}(Y)}<\infty,\]
then it has a strong-type estimate
\[\|T\|_{L^p(X)\to L^p(Y)}<\infty.\]
\end{prb}
\begin{pf}
Let $f\in L^p(X)$ and denote $f_h=\chi_{|f|>\alpha}f$ and $f_l=\chi_{|f|\le\alpha}f$.
It is easy to show $f_h\in L^{p_0}$ and $f_l\in L^{p_1}$.
Then,
\begin{align*} % 고치던 중
\|Tf\|_{L^p(Y)}^p&\sim\int\alpha^p\cdot\mu(|Tf|>\alpha)\,\frac{d\alpha}\alpha\\
&\lesssim\int\alpha^p\cdot\mu(|Tf_h|>\alpha)\,\frac{d\alpha}\alpha+\int\alpha^p\cdot\mu(|Tf_l|>\alpha)\,\frac{d\alpha}\alpha\\
&\le\int\alpha^p\cdot\frac1{\alpha^{p_0}}\|Tf_h\|_{L^{p_0,\infty}}^{p_0}\,\frac{d\alpha}\alpha+\int\alpha^p\cdot\frac1{\alpha^{q_1}}\|Tf_l\|_{L^{p_1,\infty}}^{p_1}\,\frac{d\alpha}\alpha\\
&\lesssim\int\alpha^{p-p_0}\|f_h\|_{p_0}^{p_0}\,\frac{d\alpha}\alpha+\int\alpha^{p-p_1}\|f_l\|_{p_1}^{p_1}\,\frac{d\alpha}\alpha\\
&\sim\|f\|_p^p.
\end{align*}
by (1) Fubini, (2) Sublinearlity, (3) Chebyshev, (4) Boundedness, (5) Fubini.
\end{pf}

\begin{prb}[Hadamard's three line lemma]
Let $f$ be a bounded holomorphic function on vertical unit strip $\{z:0<\Re z<1\}$ which is continuously extended to the boundary.
Then, for $0<\theta<1$ we have
\[\|f\|_{L^\infty(\Re=\theta)}\le\|f\|_{L^\infty(\Re=0)}^{1-\theta}\|f\|_{L^\infty(\Re=1)}^\theta.\]
\end{prb}
\begin{pf}
Fix $n$ and define
\[g_n(z):=\frac{f(z)}{\|f\|_{L^\infty(\Re=0)}^{1-z}\|f\|_{L^\infty(\Re=1)}^z}e^{-\frac{z(1-z)}n}.\]
Then,
\[|g_n(z)|\le e^{-\frac{(\Im z)^2}n}\]
for $z$ in the strip.
By the maximum principle,
\[|f(z)|\le\|f\|_{L^\infty(\Re=0)}^{1-\theta}\|f\|_{L^\infty(\Re=1)}^\theta e^{\frac{y^2}n}.\]
Letting $n\to\infty$, we are done.
\end{pf}



\begin{prb}[Riesz-Thorin interpolation]
Let $X,Y$ be $\sigma$-finite measure spaces.
Let
\[\frac1{p_\theta}=(1-\theta)\frac1{p_0}+\theta\frac1{p_1},\qquad\frac1{q_\theta}=(1-\theta)\frac1{q_0}+\theta\frac1{q_1}.\]
Then,
\[\|T\|_{p_\theta\to q_\theta}\le\|T\|_{p_0\to q_0}^{1-\theta}\|T\|_{p_1\to q_1}^\theta.\]
\end{prb}
\begin{pf}
Note that
\[\|T\|_{p_\theta\to q_\theta}=\sup_f\frac{\|Tf\|_{q_\theta}}{\|f\|_{p_\theta}}=\sup_{f,g}\frac{|\<Tf,g\>|}{\|f\|_{p_\theta}\|g\|_{q'_\theta}}.\]
Consider a holomorphic function
\[z\mapsto\<Tf_z,g_z\>=\int\bar{g_z(y)}Tf_z(y)\,dy,\]
where $f_z$ and $g_z$ are defined as
\[f_z=|f|^{\frac{p_\theta}{p_0}(1-z)+\frac{p_\theta}{p_1}z}\frac f{|f|}\]
so that we have $f_\theta=f$ and
\[\|f\|_{p_\theta}^{p_\theta}=\|f_z\|_{p_x}^{p_x}\]
for $\Re z=x$.

Then,
\[|\<Tf_z,g_z\>|\le\|T\|_{p_0\to q_0}\|f_z\|_{p_0}\|g_z\|_{q'_0}=\|T\|_{p_0\to q_0}\|f\|_{p_\theta}^{p_\theta/p_0}\|g\|_{q'_\theta}^{q'_\theta/q'_0}\]
for $\Re z=0$, and
\[|\<Tf_z,g_z\>|\le\|T\|_{p_1\to q_1}\|f_z\|_{p_1}\|g_z\|_{q'_1}=\|T\|_{p_1\to q_1}\|f\|_{p_\theta}^{p_\theta/p_1}\|g\|_{q'_\theta}^{q'_\theta/q'_1}\]
for $\Re z=1$.
By Hadamard's three line lemma, we have
\[|\<Tf_z,g_z\>|\le\|T\|_{p_0\to q_0}^{1-\theta}\|T\|_{p_1\to q_1}^\theta\|f\|_{p_\theta}\|g\|_{q'_\theta}\]
for $\Re z=\theta$.
Putting $z=\theta$ in the last inequality, we get the desired result.
\end{pf}






\chapter{Topological measures}

\section{Borel measures}



\section{Locally compact spaces}



\begin{prb}[One-point compactification]
\end{prb}

\section{Locally finite measures}

\begin{prb}[Regular Borel measures on locally compact metric spaces]
sss
\begin{parts}
\item $C_c(X)$ is dense in $L^p(\mu)$ for $1\le p<\infty$.
\item If $\mu$ is $\sigma$-finite, then for any $\e>0$ there is compact $K\subset X$ and continuous $g:X\to\R$ such that $f|_K=g|_K$ and $\mu(X\setminus K)<\e$.
\end{parts}
\end{prb}


\begin{prb}[Tightness and inner regularity]
\begin{parts}
\item
\end{parts}
\end{prb}

\begin{prb}[Regular Borel measures on metric spaces]
Let $\mu$ be a Borel measure on a metric space $X$.
We say $\mu$ is \emph{outer regular} if
\[\mu(E)=\inf\{\mu(U):E\subset U,\,U\text{ open}\},\]
and say $\mu$ is \emph{inner regular} if
\[\mu(E)=\sup\{\mu(F):F\subset E,\,F\text{ closed}\},\]
for every Borel subset $E\subset X$.
If $\mu$ is both outer and inner regular, we say $\mu$ is \emph{regular}.
\begin{parts}
\item Let $E$ be $\sigma$-finite. Then, $E$ is $\mu$-regular if and only if for any $\e>0$ there are open $U$ and closed $F$ such that $F\subset E\subset U$ and $\mu(U\setminus F)<\e$.
\item If $\mu$ is $\sigma$-finite, then the set of $\mu$-regular subsets is a $\sigma$-algebra. (may be extended?)
\item Every closed set is $G_\delta$.
\item Every finite Borel measure on $X$ is regular.
\end{parts}
\end{prb}
\begin{pf}
\end{pf}




\begin{prb}[Luzin's theorem]
Let $\mu$ be a regular Borel measure on a metric space $X$.
Let $f:X\to\R$ be a Borel measurable function.
Two proofs: direct and Egoroff.
% Important properties: NORMALITY for Tietze, and $\sigma$-FINITENESS for U,F squeezing.
\begin{parts}
\item If $E\subset X$ is $\sigma$-finite, then there is a continuous $g$ blabla
\item If $f$ vanishes outside a $\sigma$-finite set, then for any $\e>0$ there is a closed set $F\subset X$ such that $f|_F:F\to\R$ is continuous and $\mu(X\setminus F)<\e$.
\item If $f$ vanishes outside a $\sigma$-finite set, then for any $\e>0$ there is a closed set $F\subset X$ and continuous $g:X\to\R$ such that $f|_F=g|_F$ and $\mu(X\setminus F)<\e$.
\item If $f$ is further bounded, then $g$ also can be taken to be bounded.
\end{parts}
\end{prb}
\begin{pf}
(a)
Let $\e>0$ and suppose $E\subset X$ is measurable with $\mu(E)<\infty$.
Since $E$ is $\sigma$-finite, we have open $U$ and closed $F$ such that $F\subset E\subset U$ and $\mu(U\setminus F)<\e/2$.
By the Urysohn lemma, there is a continuous function $g:X\to[0,1]$ such that $g|_{U^c}=0$ and $g|_F=1$.
Then,
\[\int|1_E-g|\,d\mu=\int_{U\setminus F}|1_E-g|\,d\mu\le2\mu(U\setminus F)<\e.\]

(b)
Since $\R$ is second countable, we have a base $(V_n)_{n=1}^\infty$ of $\R$.
Since $\mu$ is $\sigma$-finite, for each $n$ we can take open $U_n$ and closed $F_n$ such that
\[F_n\subset f^{-1}(V_n)\subset U_n\]
and $\mu(U_n\setminus F_n)<\e/2^n$.
Define $F:=\left(\bigcup_{n=1}^\infty(U_n\setminus F_n)\right)^c$ so that $\mu(X\setminus F)<\e$ and $F$ is closed.
Then,
\begin{align*}
U_n\cap F
&=U_n\cap((U_n^c\cup F_n)\cap F)\\
&=(U_n\cap(U_n^c\cup F_n))\cap F\\
&=(\varnothing\cup(U_n\cap F_n))\cap F\\
&\subset F_n\cap F
\end{align*}
proves $f^{-1}(V_n)$ is open in $F$ for every $n$, hence the continuity of $f|_F$.
(In fact, we require that $X$ to be just a topological space.)

(b')
We can alternatively use the part (a) and the Egoroff theorem.
By the part (a), we can construct a sequence $(f_n)$ of continuous functions $X\to\R$ such that $f_n\to f$ in $L^1$.
By taking a subsequence, we may assume $f_n\to f$ pointwise.
Assuming $\mu$ is finite, by the Egorov theorem, there is a measurable $A\subset X$ such that $f_n\to f$ uniformly on $A$ and $\mu(X\setminus A)<\e/2$.
Since $\mu$ is inner regular, we have closed $F\subset A$ such that $\mu(A\setminus F)<\e/2$, so that we have $\mu(X\setminus F)<\e$.
Then, $f$ is continuous on $A$, and of course on $F$.

\end{pf}



\begin{prop}
A $\sigma$-finite Radon measure is regular.
\end{prop}
\begin{pf}
First we approximate Borel sets of finite measure, with compact sets.
Let $E$ be a Borel set with $\mu(E)<\infty$ and $U$ be an open set containing $E$.
By outer regularity, there is an open set $V\supset U-E$ such that
\[\mu(V)<\mu(U-E)+\frac\e2.\]
By inner regularity, there is a compact set $K\subset U$ such that
\[\mu(K)>\mu(U)-\frac\e2.\]
Then, we have a compact set $K-V\subset K-(U-E)\subset E$ such that
\begin{align*}
\mu(K-V)&\ge\mu(K)-\mu(V)\\
&>\left(\mu(U)-\frac\e2\right)-\left(\mu(U-E)+\frac\e2\right)\\
&\ge\mu(E)-\e.
\end{align*}
It implies that a Radon measure is inner regular on Borel sets of finite measures.

Suppose $E$ is a $\sigma$-finite Borel set so that $E=\bigcup_{n=1}^\infty E_n$ with $\mu(E_n)<\infty$.
We may assume $E_n$ are pairwise disjoint.
Let $K_n$ be a compact subset of $E_n$ such that
\[\mu(K_n)>\mu(E_n)-\frac\e{2^n},\]
and define $K=\bigcup_{n=1}^\infty K_n\subset E$.
Then,
\[\mu(K)=\sum_{n=1}^\infty\mu(K_n)>\sum_{n=1}^\infty\left(\mu(E_n)-\frac\e{2^n}\right)=\mu(E)-\e.\]
Therefore, a Radon measure is inner regular on all $\sigma$-finite Borel sets.
\end{pf}

\section{Continuous functions in $L^p$ spaces}

Approximate identity
density












\chapter{Dual spaces}


\section{Dual of Lebesgue spaces}



Radon-Nikodym theorem

An integrable function as a measure

$\sigma$-finite measures

\section{Riesz-Markov-Kakutani representation theorem}

% 1. most general -> Radon
% 2. sigma-finite -> regular measures
% 3. loc finite, metrizable(=2nd cntbl) LCH -> Borel measures
% 4. finite, metrizable -> Borel measures

% pde, prob, spectral theorem -> 4.
% Haar meas on Lie group, number theory -> 3.
% c* algebra, choquet -> compact but bad topology -> 2.
% Haar meas on LCG -> may not be sigma-finite -> 1.

% 이제 증명들이 어느 정도 수준에서 가능한 지를 보자
% C_c version

locally finite tight measure.


\begin{prb}[Radon measures]
\begin{parts}
\item A $\sigma$-finite Radon measure is regular.
\item If every open subset of $X$ is $\sigma$-compact, then a locally finite Borel measure is Radon.
\item $C_c(X)$ is dense in $L^p(\mu)$ for $1\le p<\infty$.
\end{parts}
\end{prb}

\begin{prb}[Riesz-Markov-Kakutani representation theorem for $C_0(X)$]
Let $X$ be a locally compact Hausdorff space.
We want to establish the following one-to-one correspondence:
\[\begin{array}{ccc}
\{\text{finite Radon measures on $X$}\} & \xrightarrow{\sim} & \{\text{positive linear functionals on $C_0(X)$}\}\\
\mu & \mapsto & (f\mapsto\int f\,d\mu).
\end{array}\]
Let $I$ a positive linear functional on $C_0(X)$.
Let $\cT$ be the set of all open subsets of $X$ and $\mu_0:\cT\to[0,\infty]$ a set function defined such that
\[\mu_0(U):=\sup\{I(f):f\in C_c(U,[0,1])\},\qquad U\in\cT.\]
Let $\mu^*:\cP(X)\to[0,\infty]$ be the associated outer measure defined by
\[\mu^*(S):=\inf\left\{\sum_{i=1}^\infty\mu_0(U_i):S\subset\bigcup_{i=1}^\infty U_i,\ U_i\in\cT\right\},\qquad S\in\cP(X),\]
and let $\mu:=\mu^*|_\cA$ be the restriction, where $\cA$ is the $\sigma$-algebra of Carath\'eodory measurable subsets relative to $\mu^*$.
\begin{parts}
\item $\mu^*$ extends $\mu_0$.
\item $\mu$ extends $\mu_0$.
\item $\mu$ is a finite Radon measure.
\item The correspondence is surjective.
\item The correspondence is injective.
\end{parts}
\end{prb}
\begin{pf}
(a)
It suffices to show that $\mu_0$ satisfies monotonically countably subadditive.
For an open set $U$ and a countable open cover $\{U_i\}_{i=1}^\infty$ of $U$ we claim that $\rho(U)\le\sum_{i=1}^\infty\rho(U_i)$.

Take any $f\in C_c(U,[0,1])$ and find a finite subcover $\{U_{i_k}\}_{k=1}^n$ of $\{U_i\}$ together with a partition of unitiy $\{\chi_{i_k}\}$ subordinate to the open cover $\{U_{i_k}\cap\supp f\}_k$.
Now we have $f\chi_{i_k}\in C_c(U_{i_k},[0,1])$ for each $k$, because then $I$ is linear so that it preserves finite sum, we have
\[I(f)=\sum_{k=1}^n I(f\chi_{i_k})\le\sum_{k=1}^n\mu_0(U_{i_k})\le\sum_{i=1}^\infty\mu_0(U_i).\]
Since $f$ is arbitrary, we are done.

(b)
We claim $\cT\subset\cA$.
It suffices to show $\mu^*(E\cap U)+\mu^*(E\setminus U)\le\mu^*(E)$ for any measurable $E$ and open $U$.
Take $\e>0$.
Since we may assume $\mu^*(E)<\infty$, there is a countable open cover $\{U_i\}_{i=1}^\infty$ of $E$ such that
\[\sum_{i=1}^\infty\mu_0(U_i)<\mu^*(E)+\frac\e3.\]
Take $f_i\in C_c(U_i\cap U,[0,1])$ such that
\[\mu_0(U_i\cap U)<I(f_i)+\frac13\cdot\frac\e{2^i},\]
and take $g_i\in C_c(U_i\setminus\supp f_i,[0,1])$ such that
\[\mu_0(U_i\setminus\supp f_i)<I(g_i)+\frac13\cdot\frac\e{2^i}.\]
Then, since $f_i+g_i\in C_c(U_i,[0,1])$, we have
\begin{align*}
\mu^*(E\cap U)+\mu^*(E\setminus U)
&\le\sum_{i=1}^\infty\mu_0(U_i\cap U)+\sum_{i=1}^\infty\mu_0(U_i\setminus U)\\
&<\sum_{i=1}^\infty I(f_i+g_i)+\frac\e3+\frac\e3\\
&<\sum_{i=1}^\infty\mu_0(U_i)+\frac23\e\\
&\le\mu^*(E)+\e.
\end{align*}
Limiting $\e\to0$, we get the desired inequality.

(c)
Since $\mu$ is a countably additive and $\cT$ is closed under union, we can rewrite
\[\mu^*(S)=\inf\{\mu_0(U):S\subset U\in\cT\},\qquad S\in\cP(X),\]
hence $\mu$ is outer regular.
Here now we claim for $f\in C_c(X,[0,1])$ and $0<a<1$ that
\[a\mu(f^{-1}((a,1]))\le I(f)\le\mu(\supp f).\]
If it is true, then the right inequality implies the inner regularity, and the left inequality together with the Urysohn lemma implies the local finiteness.

The right inequality directly follows from the definition of $\mu_0$ and the outer regularity
\[I(f)\le\inf\{\mu_0(U):\supp f\subset U\in\cT\}=\mu(\supp f).\]
For the left, if $h\in C_c(f^{-1}((a,1]),[0,1])$, then the inequality $ah\le f$ implies
\[a\mu(f^{-1}((a,1]))=a\mu_0(f^{-1}((a,1]))\le aI(h)\le I(f).\]

(d)
We will show $I(f)=\int f\,d\mu$ for $f\in C_c(X)$.
Since $C_c(X)$ is the linear span of $C_c(X,[0,1])$, we may assume $f\in C_c(X,[0,1])$.
For a fixed positive integer $n$ and for each index $1\le i\le n$, let $K_i:=f^{-1}([i/n,1])$ and define
\[f_i(x):=\begin{cases}\frac1n&\text{ if }x\in K_i,\\f(x)-\frac{i-1}n&\text{ if }x\in K_{i-1}\setminus K_i,\\0&\text{ if }x\in X\setminus K_{i-1},\end{cases}\]
where $K_0:=\supp f$.
Note that $f_i\in C_c(X,[0,n^{-1}])$ and $f=\sum_{i=1}^nf_i$.
For $1\le i\le n$ we have $\mu(K_i)<\infty$ because $K_i$ is compact subsets contained in a locally compact Hausdorff space $U:=f^{-1}((0,1])$.
By the previous claim and the property of integral, we have
\[\frac{\mu(K_i)}n\le I(f_i),\qquad\frac{\mu(K_i)}n\le\int f_i\,d\mu,\qquad1\le i\le n\]
and
\[I(f_i)\le\frac{\mu(K_{i-1})}n,\qquad\int f_i\,d\mu\le\frac{\mu(K_{i-1})}n,\qquad2\le i\le n.\]
Then, using the above inequalities and $\mu(K_n)\ge0$, we have
\[I(f)\le I(f_1)+\int f\,d\mu\quad\text{and}\quad\int f\,d\mu\le\int f_1\,d\mu+I(f).\]
Note that $f_1=\min\{f,n^{-1}\}$ is a sequence of functions indexed by $n$.
By the monotone convergence theorem, $\int f_1\,d\mu\to0$ as $n\to\infty$.
We now show $I(f_1)$ converges to zero.
If we let $U:=f^{-1}((0,1])$, then $U$ is locally compact and $f_1\in C_0(U)\subset C_c(X)$, and since a positive linear functional on $C_0(U)$ is bounded, we have $I(f_1)\le n^{-1}\|I\|\to0$ as $n\to\infty$.
($\mu(K_0)$ is possibly infinite if $X$ is not locally compact so that $\mu$ is not locally finite.)

(e)
Let $\mu$ and $\nu$ be finite Radon measures on $X$ such that
\[\int g\,d\mu=\int g\,d\nu\]
for all $g\in C(X)$.
Let $E$ be any measurable set.
Since $\mu+\nu$ is a finite Radon measure, and by the Luzin theorem, we have a closed set $F$ and $g\in C(X)$ with $0\le g\le1$ such that $1_E|_F=g|_F$ and $(\mu+\nu)(X\setminus F)<\e/2$.
Then,
\begin{align*}
|\mu(E)-\nu(E)|&=|\int1_E\,d\mu-\int1_E\,d\nu\,|\\
&\le\int_{X\setminus F}|1_E-g|\,d\mu+\int_{X\setminus F}|g-1_E|\,d\nu\\
&\le2\mu(X\setminus F)+2\nu(X\setminus F)<\e.
\end{align*}
By limiting $\e\to0$, we have $\mu(E)=\nu(E)$.
\end{pf}


\begin{prb}[Dual of continuous function spaces]
\end{prb}




\section*{Fremlin}


A measure $\mu$ is called \emph{inner regular} if for every measurable $E$ we have
\[\mu(E)=\sup\{\mu(F):F\subset E,\ \text{$F$ closed}\},\]
and called \emph{tight} if for every measurable $E$ we have
\[\mu(E)=\sup\{\mu(K):K\subset E,\ \text{$K$ compact}\}.\]
Note that the inner regularity by Folland or Rudin, outer regular for Borel and inner regular for open, is in fact the tightness, the inner regularity with respect to compact sets.

On a Tychonoff space $X$, $\Prob(X)$ is defined as the set of tight Borel probability measures so that there is an embedding $\Prob(X)\to\Prob(\beta X)$ defined as the pushforward.

We can try to define a \emph{Radon measure} on a Hausdorff space $X$ as a locally finite tight Borel measure.
Then, how to deal with regularity on Polish spaces?

\subsection{}



A \emph{quasi-Radon measure} on a Hausdorff space is a measure which is complete, locally determined, $\tau$-additive, inner regular with respect to closed sets, and effectively locally finite.
A \emph{Radon measure} on a Hausdorff space is a measure which is complete, locally determined, locally finite, and tight.
By the completeness condition, it is not Borel in general.
\begin{itemize}
\item 415A A quasi-Radon measure is strictly localizble.
\item 416C For a locally finite quasi-Radon measure $\mu$, $\mu$ is Radon iff 
\item 416F A Borel measure on a Hausdorff space has a Radon extension if and only if it is locally finite and tight, and in this case the extension is unique.
\item 416G A locally finite quasi-Radon measure is Radon.
\end{itemize}




Riesz-Markov-Kakutani 436J and 436K
\begin{pf}
First we can show $I$ is smooth(I think it is equivalent to normality).
Since $X$ is locally compact, it is the coarsest topology for which $C_c$ is continuous, i.e.~Baire$=$Borel.
Also, $C_c$ is truncated Riesz subspace of $\R^X$.
So 436H implies there is a quasi-Radon measure $\mu$ such that $I(f)=\int f\,d\mu$ for $f\in C_c$, which is clearly locally finite.
By 416G, $\mu$ is Radon.
\end{pf}

\begin{itemize}
\item A Radon measure is tight.
\item A $\sigma$-finite Folland-Radon measure on a locally compact Hausdorff space is tight. Moreover, Folland-Radon and Fremlin-Radon coincides on $\sigma$-compact locally compact Hausdorff spaces.
\item A locally finite Borel measure on a locally compact Hausdorff and second countable space is tight.
\item A locally compact Hausdorff and second countable space is Polish.
\item A tight measure on a topological space is always inner regular with respect to closed sets, and the converse is true on where???
\end{itemize}

Definitions
\begin{itemize}
\item A measurable algebra is called \emph{localizable} if the essential union exists even for uncountable family of measurable sets.
\item A \emph{localizble measure} is a semi-finite measure on a localizable measurable algebra.
\item A \emph{strictly localizable measure} or \emph{decomposable measure} is a measure which admits a partition $\{F_i\}$ of $X$, called the decomposition, such that $F_i$ are finite measurable and $E\cap F_i\in\Sigma$ for all $F_i$ implies $E\in\Sigma$ and $\mu(E)=\sum_{i\in J}\mu(E\cap F_i)$.
\item A \emph{locally determined measure} is a semi-finite measure such that $E\cap F\in\Sigma$ for any $F\in\Sigma$ of finite measure implies $E\in\Sigma$.(I think it is more natural to say a enhanced measurable space is locally determined by a semi-finite measure)
\end{itemize}

Locally finite measures
\begin{itemize}
\item A $\sigma$-finite measure is strictly localizable.
\item A strictly localizable measure is localizable and locally determined.
\item A tight measure on a topological space is $\tau$-additive.
\item A locally finite measure on a topological space is finite on compact sets.
\item A locally finite measure on a Lindel\"of space is $\sigma$-finite.
\item A locally finite and tight measure is effectively locally finite.
\item A effectively locally finite(non-negligible set has an open set of finite measure whose intersection with it is non-negligible) measure on a topological space is semi-finite.
\item
\end{itemize}




\section{Dual of continuous function spaces}

signed measure
Hahn, Jordan decomposition







\part{Distribution theory}
\chapter{Test functions}

Let $\Omega$ be a hemi-compact Hausdorff space.

\[C^\infty(\Omega)\]
is a Fr\'echet space if and only if $\Omega$ is hemi-compact.

\[C_K^\infty(\Omega)\]

\chapter{Distributions}


\chapter{Linear operators}

\section{Boundedness}

Translation and multiplication operators

\begin{prb}[Bitranspose extension]
\end{prb}


\section{Kernels}
\begin{prb}[Schur test]
\end{prb}
\begin{prb}[Young's inequality of integral operators]
\end{prb}

\section{Convolution}
\begin{prb}[Approximation of identity]
Fej\'er, Poisson, box?
\end{prb}
\begin{prb}[Summability methods]
\end{prb}











\part{Fundamental theorem of calculus}

\chapter{}

\section{Absolutely continuous functions}

The space of weakly differentiable functions with respect to all variables $=W_\loc^{1,1}$.

\begin{prb}[Product rule for weakly differentiable functions]
We want to show that if $u$, $v$, and $uv$ are weakly differentiable with respect to $x_i$, then $\pd_{x_i}(uv)=\pd_{x_i}uv+u\pd_{x_i}v$.
\begin{parts}
\item If $u$ is weakly differentiable with respect to $x_i$ and $v\in C^1$, then $\pd_{x_i}(uv)=\pd_{x_i}uv+u\pd_{x_i}v$.
\end{parts}
\end{prb}


\begin{prb}[Interchange of differentiation and integration]
Let $f:X_x\times X_y\to\R$ be such that $\pd_{x_i}f$ is well-defined. Suppose $f$ and $\pd_{x_i}f$ are locally integrable in $x$ and integrable $y$.

Then,
\[\pd_{x_i}\int f(x,y)\,dy=\int\pd_{x_i}f(x,y)\,dy.\]
\end{prb}


Do not think the Schwarz theorem as the condition for partial differentiation to commute.
We should understand like this: if $F$ is $C^2$ then the \emph{classical} partial differentiation commute, and if $F$ is not $C^2$ then the \emph{classical} partial derivatives of order two or more are \emph{meaningless} because it is not compatible with the generalized concept of differentiation.




\begin{parts}
\item $f$ is $\Lip_\loc$ iff $f'$ is $L_\loc^\infty$
\item $f$ is $\textrm{AC}_\loc$ iff $f'$ is $L_\loc^1$
\end{parts}
\begin{parts}
\item $f$ is $\Lip$ iff $f'$ is $L^\infty$
\item $f$ is $\textrm{AC}$ iff $f'$ is $L^1$
\item $f$ is $\textrm{BV}$ iff $f'$ is a finite regular Borel measure
\end{parts}

\begin{prb}[Absolute continuous measures]
\end{prb}

\begin{prb}[Absolute continuous functions]
\end{prb}



\section{Functions of bounded variation}




\chapter{Lebesgue differentiation theorem}

\section{Hardy-Littlewood maximal function}

Let $T_m$ be a net of linear operators.
It seems to have two possible definitions of maximal functions:
\[T^*f:=\sup_m|T_mf|\]
and
\[T^*f:=\sup_{m,\ \e:|\e(x)|=1}|T_m(\e f)|.\]

\begin{prb}[Hardy-Littlewood maximal function]
The Hardy-Littlewood maximal function is just the maximal function defined with the approximate identity by the box kernel.
\end{prb}

\begin{prb}[Weak type estimate]
\[\|Mf\|_{1,\infty}\le 3^d\|f\|_{L^1(X)}.\]
\begin{parts}
\item Proof by covering lemma.
\end{parts}
\end{prb}
\begin{pf}
(a)
By the inner regularity of $\mu$, there is a compact subset $K$ of $\{|Mf|>\lambda\}$ such that
\[\mu(K)>\mu(\{|Mf|>\lambda\})-\e.\]
For every $x\in K$, since $|Mf(x)|>\lambda$, we can choose an open ball $B_x$ such that
\[\frac1{\mu(B_x)}\int_{B_x}|f|>\lambda\]
if and only if
\[\mu(B_x)<\frac1\lambda\int_{B_x}|f|.\]
With these balls, extract a finite open cover $\{B_i\}_i$ of $K$.
Since the diameter of elements in this cover is clearly bounded, so the Vitali covering lemma can be applied to obtain a disjoint subcollection $\{B_k\}_k$ such that
\[K\subset\bigcup_iBi\subset\bigcup_k3B_k.\]
Therefore,
\[\mu(K)
\le\sum_k3^d\mu(B_k)
\le\frac{3^d}\lambda\sum_k\int_{B_k}|f|
\le\frac{3^d}\lambda\|f\|_1.\]
The disjointness is important in the last inequality which shows the constant does not depend on the number of $B_k$'s.
\end{pf}



\begin{prb}[Radially bounded approximate identity]
If an approximate identity $K_n$ is radially bounded, then its maximal function is dominated by the Hardy-Littlewood maximal function:
\[\sup_n|K_n*f(x)|\lesssim Mf(x)\]
for every $n$ and $x$, hence has a weak type estimate.
\end{prb}


\begin{prb}[Almost everywhere convergence of operators]
Suppose is $T_m$ is a sequence of linear operators such that the maximal function $T^*f$ is dominated by $Mf$.
If $f\in L^1(X)$ and $T_mg\to g$ pointwise for $g\in C(X)$, then $T_mf\to f$ a.e.
\end{prb}
\begin{pf}
Take $\e>0$ and $g\in C(X)$ such that $\|f-g\|_{L^1(X)}<\e$.
Since $T_mg(x)\to g(x)$ pointwise, we have
\begin{align*}
&\mu(\{x:\limsup_m|T_mf(x)-f(x)|>\lambda\})\\
&\qquad\le\mu(\{x:\limsup_m|T_mf(x)-T_mg(x)|>\tfrac\lambda2\})
+\mu(\{x:|g(x)-f(x)|>\tfrac\lambda2\})\\
&\qquad\le\mu(\{x:M(f-g)(x)>\tfrac\lambda2\})+\frac2\lambda\|f-g\|_{L^1(X)}\\
&\qquad\lesssim\frac1\lambda\e
\end{align*}
for every $\lambda>0$.
Limiting $\e\to0$, we get
\[\mu(\{x:\limsup_m|T_mf(x)-f(x)|>\lambda\})=0\]
for every $\lambda>0$, hence the continuity from below implies
\[\mu(\{x:\limsup_m|T_mf(x)-f(x)|>0\})=0.\]
\end{pf}


\begin{defn}
\[f^*(x):=\lim_{r\to0+}\frac1{\mu(B)}\int_B|f(y)-f(x)|\,dy.\]
\end{defn}
\begin{thm}[Lebesgue differentiation]
$f^*=0$ a.e.
\end{thm}
\begin{pf}
Note that $f^*\le Mf+|f|$ implies
\[\|f^*\|_{1,\infty}\le\|Mf\|_{1,\infty}+\|f\|_{1,\infty}\lesssim\|f\|_1.\]
Note that $g^*=0$ for $g\in C_c$.
Approximate using $f^*=(f-g)^*$.
\end{pf}

\section*{Exercises}
\begin{prb}[Doubling measure]

\end{prb}


\end{document}


