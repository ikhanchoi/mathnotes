\documentclass[a4paper]{article}
\usepackage{amsmath,amsthm,enumitem}
\usepackage[margin=3cm]{geometry}
\usepackage[T1]{fontenc}
\usepackage[bitstream-charter,cal]{mathdesign}
\linespread{1.15}
\usepackage{luatexja}
\theoremstyle{definition}
\newtheorem{prb}{}
\renewcommand{\theprb}{問\arabic{prb}}

\renewcommand{\Re}{\operatorname{Re}}
\renewcommand{\Im}{\operatorname{Im}}
\renewcommand{\H}{\mathbb{H}}
\newcommand{\C}{\mathbb{C}}
\newcommand{\R}{\mathbb{R}}
\newcommand{\Z}{\mathbb{Z}}
\newcommand{\e}{\varepsilon}
\newcommand{\f}{\varphi}
\newcommand{\PSL}{\operatorname{PSL}}
\renewcommand{\tilde}{\widetilde}
\renewcommand{\bar}{\overline}

\begin{document}

\title{複素解析学I演習2023年}
\date{}
\maketitle


\begin{prb}[フックス群としてのモジュラー群]
複素数体$\C$の部分集合$A$に対して、成分$a,b,c,d$が$A$の元で$ad-bc=1$を満たす一次分数変換$f(z)=(az+b)/(cz+d)$の集合を$\PSL(2,A)$と書く.
特に$\PSL(2,\Z)$を\emph{モジュラー群}と呼ぶ.
上半平面$\H:=\{z\in\C:\Im z>0\}$の部分集合$D:=\{z\in\H:|z|>1,\ |\Re z|<\frac12\}$を定義する.
\begin{enumerate}[label=(\arabic*)]
\item $\PSL(2,\R)$の元$f$は全単射写像$\H\to\H$を定義することを示せ.
\item $\PSL(2,\Z)$は$S(z):=-1/z$と$T(z):=z+1$によって生成されることを示せ.つまり、全ての元が$S^{\pm1}$と$T^{\pm1}$の有限回の合成として表れることを示せ.
\item 集合$D$は$\PSL(2,\Z)$の\emph{基本領域}であることを示せ.つまり、次の二つが成り立つことを示せ:
\begin{enumerate}[label=(\alph*)]
\item 任意の点$z\in\H$に対して$f(z)\in\bar D$を満たす$f\in\PSL(2,\Z)$が少なくとも一つ存在する.
\item 任意の点$z\in\H$に対して$f(z)\in D$を満たす$f\in\PSL(2,\Z)$が多くとも一つ存在する.
\end{enumerate}
\item $\PSL(2,\Z)$は$\H$に\emph{真性不連続に作用}することを示せ.つまり、任意の点$z\in\H$に対して軌道$\{f(z):f\in\PSL(2,\Z)\}$が離散集合であることを示せ.
\end{enumerate}
\end{prb}


\begin{prb}[カラテオドリ級関数集合の極点]
開単位円板上で定義された正則関数$f$が$f(0)=1$を満たすとする.
もし任意の$|z|<1$を満たす複素数$z$に対して$\Re f(z)>0$ならば、$f$を\emph{カラテオドリ級}の関数という.
関数$f$が冪級数展開$f(z)=1+2\sum_{k=1}^\infty c_kz^k$を持つとする.
\begin{enumerate}[label=(\arabic*)]
\item 正の整数$k$と実数$0<r<1$に対して次の式を示せ:
\[c_kr^k=\frac1{2\pi}\int_0^{2\pi}\Re f(re^{i\theta})e^{-ik\theta}\,d\theta.\]
\item 次の二つの条件が同値であることを示せ:
\begin{enumerate}[label=(\alph*)]
\item 関数$f$がカラテオドリ級である.
\item 任意の正の整数$n$に対して点$(c_1,\cdots,c_n)\in\C^n$は$\theta\in[0,2\pi)$によって媒介変数表示された曲線$(e^{-i\theta},\cdots,e^{-in\theta})\in\C^n$の凸包絡の元である.
\end{enumerate}
\end{enumerate}
\end{prb}


\begin{prb}[アールフォルス・清水標数]
複素平面上の有理型関数$f$を考える.
次のように$r\ge0$に対する関数$A(\cdot,f)$を定義する:
\[A(r,f):=\frac1\pi\int_{\sqrt{x^2+y^2}\le r}f^\#(x+iy)^2\,dx\,dy,\qquad\text{ただし、}\ f^\#(z):=\frac{|f'(z)|}{1+|f(z)|^2},\quad z\in\C.\]
関数$f^\#$を$f$の\emph{球面導関数}と呼ぶ.
\begin{enumerate}[label=(\arabic*)]
\item 任意の点$(x,y)\in\R^2$に対して、
\[\frac1\pi\,f^\#(x+iy)^2=\frac{\partial Q}{\partial x}(x,y)-\frac{\partial P}{\partial y}(x,y)\]
を満たす実平面$\R^2$上の実関数$P$と$Q$を求め、関数$K(x,y):=1+|f(x+iy)|^2$を用いて表せ.
\item グリーンの定理と偏角の原理を用いて$r\ge0$に対して次の式が成り立つことを示せ:
\[\int_0^rA(t,f)\frac{dt}t=\int_0^rn(t,f)\frac{dt}t+\frac1{2\pi}\int_0^{2\pi}\log\sqrt{1+|f(re^{i\theta})|^2}\,d\theta-\log\sqrt{1+|f(0)|^2}.\]
ただし、$n(r,f)$は閉円板$\bar{B(0,r)}$内にある重複度を込めて数えた$f$の極の数である.
左辺の関数を$f$の\emph{アールフォルス・清水標数}と呼ぶ.
\item 球面導関数$f^\#$が有界ならば、ある定数$C>0$が存在して、全ての$z\in\C$に対して$|f(z)|\le Ce^{|z|^2}$であることを示せ.特に、$f$は$\C$全体上正則である.
\end{enumerate}
\end{prb}

\begin{prb}[四分円上のディリクレ問題]
領域$\Omega:=\{(x,y)\in\R^2:x^2+y^2<1,\ x>0,\ y>0\}$上に定義された調和関数$u\in C^2(\Omega,\R)$が次の境界値条件を満たすとする:
各点$(x_0,y_0)\in\partial\Omega$に対して
\[\lim_{(x,y)\to(x_0,y_0)}u(x,y)=\begin{cases}
0&\text{ if }y_0>0,\\
1&\text{ if }y_0=0\text{ and }0<x_0<1.
\end{cases}\]
\begin{enumerate}[label=(\arabic*)]
\item 反射原理を用いて$u$は領域$\tilde\Omega:=\{(x,y)\in\R^2:x^2+y^2<1,\ x>0\}$上の調和関数$\tilde u\in C^2(\tilde\Omega,\R)$に拡張されることを示せ.
\item 適切な等角変換とポアソン積分を用いて$u$を求めよ.
\end{enumerate}
\end{prb}


\newpage

\begin{proof}[Solution of 1]

(3)
(a)
Let $z_0\in\H$.
We may assume $\Re z_0\in[-\frac12,\frac12)$.
For $z\in\H$ satisfying $\Re z\in[-\frac12,\frac12)$, if we define $f_z:=T^{-\lfloor\Re Sz+\frac12\rfloor}S$, then $\Re f_z(z)\in[-\frac12,\frac12)$.
Define a sequence $z_n$ inductively by $z_n:=f_{z_{n-1}}(z_{n-1})$ for $n\ge1$.
Then, $\Re z_n\in[-\frac12,\frac12)$ for all $n$.
Since
\[\Im z_n=\frac{\Im z_{n-1}}{(\Re z_{n-1})^2+(\Im z_{n-1})^2}\ge g(\Im z_{n-1}),\]
where $g(y):=4y/(1+4y^2)$, since $g^n(y)\uparrow\frac{\sqrt3}2$ for $0<y<\frac{\sqrt3}2$, so there is $n$ such that
\[-\frac12\le\Re z_n<\frac12,\qquad\Im z_n>\frac{\sqrt3}4.\]
If $|z_n|\ge1$, then we are done, so assume $|z_n|<1$.
Now we have three possibilities: $|z_n-1|<1$, $|z_n+1|<1$, or $\min\{|z_n-1|,|z_n+1|\}\ge1$.
For each case, we can check that $T^{-1}Sz_n$, $TSz_n$, $Sz_n$ is contained in $D$, respectively.

(b)
Let $w=(az+b)/(cz+d)$.
It suffices to show $c=0$.
Suppose $c\ne0$.
Let $n$ be an integer such that $|n-\frac ac|\le\frac12$.
Note that $|z-m|>1$ and $|w-m|>1$ for every integer $m$.
Write
\[1<|w-n|=\left|\frac{az+b}{cz+d}-n\right|\le\left|\frac1{c(cz+d)}\right|+\left|n-\frac ac\right|.\]
If $|c|\ge2$, then $|c(cz+d)|\ge4\Im z>2\sqrt3$
leads a contradiction.
If $|c|=1$, say $c=1$, then $|n-a|\le\frac12$ implies $|n-\frac ac|=0$ and $|c(cz+d)|=|z+d|>1$ leads a contradiction.
Thus, $c=0$, and we are done.

(4)
Clear from (3).
\end{proof}


\newpage
\begin{proof}[Solution of 2]
(1)
Suppose $k>0$ first.
The Cauchy integral formula writes
\begin{align*}
2c_kk!=\pd[k]{f}{z}(0)=\frac{k!}{2\pi i}\int_{|z|=r}\frac{f(z)}{z^{k+1}}\,dz=\frac{k!}{2\pi}\int_0^{2\pi}\frac{f(re^{i\theta})}{(re^{i\theta})^k}\,d\theta,
\end{align*}
and it implies
\[2c_kr^k=\frac1{2\pi}\int_0^{2\pi}f(re^{i\theta})e^{-ik\theta}\,d\theta.\]
Since $f(z)\,z^k$ is analytic, the Cauchy theorem can be applied to get
\[0=\frac1{2\pi i}\int_{|z|=r}f(z)\,z^k\,dz=\frac1{2\pi}\int_0^{2\pi}f(re^{i\theta})r^ke^{ik\theta}\,d\theta,\]
and it implies
\[0=\frac1{2\pi}\int_0^{2\pi}\bar{f(re^{i\theta})}e^{-ik\theta}\,d\theta.\]
By combining the above two equations, we obtain the formula.
For $k=0$, applying the Cauchy theorem for $f$, we have
\[c_0=f(0)=\frac1{2\pi i}\int_{|z|=r}\frac{f(z)}z\,dz=\frac1{2\pi}\int_0^{2\pi}\Re f(re^{i\theta})\,d\theta.\]

Alternatively, we can show the same result using the orthogonal relation of complex exponential functions.
An easy computation shows the identity
\begin{align*}
\Re f(re^{i\theta})
&=\frac12[f(re^{i\theta})+\bar{f(re^{i\theta})}]\\
&=\frac12\left[\left(1+\sum_{k=1}^\infty2c_k(re^{i\theta})^k\right)+\bar{\left(1+\sum_{k=1}^\infty2c_k(re^{i\theta})^k\right)}\right]\\
&=\frac12\left[\left(1+\sum_{k=1}^\infty2c_kr^ke^{ik\theta}\right)+\left(1+\sum_{k=1}^\infty2\bar{c_k}r^ke^{-ik\theta}\right)\right]\\
&=\sum_{k=-\infty}^\infty c_kr^{|k|}e^{ik\theta}.
\end{align*}
From the uniform convergence of the power series on the compact set $\{z:|z|\le(r+1)/2\}$, it follows that
\[\frac1{2\pi}\int_0^{2\pi}\Re f(re^{i\theta})e^{-ik\theta}\,d\theta=\sum_{l=-\infty}^{\infty}c_lr^{|l|}\frac1{2\pi}\int_0^{2\pi}e^{il\theta}e^{-ik\theta}\,d\theta=\sum_{l=-\infty}^{\infty}c_lr^{|l|}\delta_{kl}=c_kr^{|k|}.\]

(2)
(b)$\Rightarrow$(a)
Denote by $K_n$ the convex hull of the curve $\theta\mapsto(e^{-i\theta},\cdots,e^{-in\theta})\in\C^n$.
Suppose first that $(c_1,\cdots,c_n)\in K_n$.
For each $n$, there exists a finite sequence of pairs $(\lambda_{n,j},\theta_{n,j})_j$ having the following convex combination
\[(c_1,\cdots,c_n)=\sum_j\lambda_{n,j}(e^{-i\theta_{n,j}},\cdots,e^{-in\theta_{n,j}})\]
with coefficients $\lambda_{n,j}\ge0$ such that $\sum_j\lambda_{n,j}=1$.
Define
\[f_n(z):=\sum_j\lambda_{n,j}\frac{e^{i\theta_{n,j}}+z}{e^{i\theta_{n,j}}-z},\]
which has positive real part on $|z|<1$ because $\Re(e^{i\theta_{n,j}}+z)/(e^{i\theta_{n,j}}-z)>0$ for $|z|<1$.
Then,
\begin{align*}
f_n(z)
&=\sum_j\lambda_{n,j}(1+\sum_{k=1}^\infty2e^{-ik\theta_{n,j}}z^k)=1+\sum_{k=1}^n2c_kz^k+\sum_{k=n+1}^\infty\left(\sum_j2\lambda_{n,j}e^{-ik\theta_{n,j}}\right)z^k
\end{align*}
implies
\begin{align*}
|f_n(z)-f(z)|
&=\left|\sum_{k=n+1}^\infty\left(\sum_j2\lambda_{n,j}e^{-ik\theta_{n,j}}\right)z^k-\sum_{k=n+1}^\infty2c_kz^k\right|\\
&\le\sum_{k=n+1}^\infty\left|\left(\sum_j2\lambda_{n,j}e^{-ik\theta_{n,j}}\right)-2c_k\right||z|^k\le\sum_{k=n+1}^\infty4|z|^k
\end{align*}
converges to zero for $|z|<1$.
Therefore, $f$ has a non-negative real part on the open unit disk.
The non-negativity can be strengthened to positivity by the open mapping theorem so that $f$ belongs to the Carath\'eodory class.

(a)$\Rightarrow$(b)
Conversely, suppose that $f$ is in the Carath\'eodory class.
Let $(\gamma_1,\cdots,\gamma_n)$ be any point on the surface $\partial K_n$ of $K_n$ and $S$ any supporting hyperplane of $K_n$ tangent at $(\gamma_1,\cdots,\gamma_n)$.
Let $(u_1,\cdots,u_n)$ be the outward unit normal vector of the supporting hyperplane $S$.
Note that this unit normal vector is uniquely determined for the hyperplane with respect to the induced real inner product structure on the real $2n$-dimensional space $\C^n$ given by
\[\langle(z_1,\cdots,z_n),(w_1,\cdots,w_n)\rangle=\sum_{k=1}^n(\Re z_k\Re w_k+\Im z_k\Im w_k)=\Re\sum_{k=1}^nz_k\bar w_k.\]
Then, $\sum_{k=1}^n|u_k|^2=1$ and further that the maximum
\[M:=\max_{(x_1,\cdots,x_n)\in K_n}\ \Re\sum_{k=1}^nx_k\bar u_k>0\]
is attained at $(\gamma_1,\cdots,\gamma_n)$.
Our goal is to verify the bound
\[\Re\sum_{k=1}^nc_k\bar u_k\le M,\]
which implies that $(c_1,\cdots,c_n)$ is contained in every half space tangent to $K_n$ so that we finally obtain $(c_1,\cdots,c_n)\in K_n$.

Since for any $\theta\in[0,2\pi)$ the point $(e^{-i\theta},\cdots,e^{-in\theta})$ is in $K_n$ so that
\[\Re\sum_{k=1}^ne^{-ik\theta}\bar u_k\le M,\]
we have for arbitrarily small $\e>0$ that
\[\Re\sum_{k=1}^n\frac1{r^k}e^{-ik\theta}\bar u_k\le M+\e\]
for any $0<r<1$ sufficiently close to $1$, thus we can write
\begin{align*}
\Re\sum_{k=1}^nc_k\bar u_k
&=\Re\sum_{k=1}^n\frac1{2\pi r^k}\int_0^{2\pi}\Re f(re^{i\theta})e^{-ik\theta}\bar u_k\,d\theta\\
&=\frac1{2\pi}\int_0^{2\pi}\Re f(re^{i\theta})\Re\sum_{k=1}^n\frac1{r^k}e^{-ik\theta}\bar u_k\,d\theta\\
&\le\frac1{2\pi}\int_0^{2\pi}\Re f(re^{i\theta})\,d\theta\cdot(M+\e)\\
&=M+\e
\end{align*}
thanks to the positivity of $\Re f$, and by limiting $r\to1$ from left we get the desired bound.
\end{proof}


\newpage
\begin{proof}[Solution of 3]
(1)
\[\frac{du\wedge dv}{\pi(1+u^2+v^2)^2}=d\left(-\frac v{2\pi(1+u^2+v^2)}\,du+\frac u{2\pi(1+u^2+v^2)}\,dv\right)\]
\[P=-\frac{K_y}{4\pi K},\qquad Q=\frac{K_x}{4\pi K}.\]

(2)

(3)
Since every Taylor coefficient of the log function is real, we have
\[\Re\log f(z)=\frac12(\log f(z)+\log\bar{f(z)})=\log|f(z)|.\]
Take $a\in\C$ and let $r:=2|a|$.
By the Schwarz integral formula,
\begin{align*}
\log|f(a)|=\Re\log f(a)&=\frac1{2\pi}\int_0^{2\pi}\Re\frac{re^{i\theta}+a}{re^{i\theta}-a}\Re\log f(re^{i\theta})\,d\theta\\
&\le\frac1{2\pi}\int_0^{2\pi}\left|\frac{re^{i\theta}+a}{re^{i\theta}-a}\right|\log|f(re^{i\theta})|\,d\theta\\
&\le\frac1{2\pi}\int_0^{2\pi}3\log\sqrt{1+|f(re^{i\theta})|^2}\,d\theta\\
&\le\int_0^rA(t,f)\frac{dt}t\lesssim\int_0^rt^2\frac{dt}t\lesssim|a|^2.\qedhere
\end{align*}


\end{proof}


\newpage
\begin{proof}[Solution of 4]
(1)
$(x_0,y_0)\in\partial\tilde\Omega$
\[\lim_{(x,y)\to(x_0,y_0)}\tilde u(x,y)=\begin{cases}
0&\text{ if }y_0>0,\\
2&\text{ if }y_0<0.
\end{cases}\]

(2)
$\tilde\Omega$ is conformally mapped onto the upper half plane by
\[\f:z\mapsto\left(\frac{z+i}{iz+1}\right)^2.\]

(3)
We can compute
\[|\f(x+iy)|^2=\left(\frac{x^2+(y+1)^2}{x^2+(y-1)^2}\right)^2,\qquad\Im\f(x+iy)=\frac{4x(1-x^2-y^2)}{(x^2+(y-1)^2)^2}.\]
For $x^2+y^2>1$ the Poisson kernel gives that
\begin{align*}
U(x,y)
&=\frac2\pi\int_{-1}^1\frac y{(x-t)^2+y^2}\,dt\\
&=\frac2\pi\left(\tan^{-1}\frac{1-x}y+\tan^{-1}\frac{1+x}y\right)\\
&=\frac2\pi\tan^{-1}\frac{2y}{x^2+y^2-1}.
\end{align*}
\[u(x,y)=U(\Re\f(x+iy),\Im\f(x+iy)).\]
Thus we have
\[u(x,y)=\frac2\pi\tan^{-1}\frac{x(1-x^2-y^2)}{y(1+x^2+y^2)}.\]
\end{proof}

\end{document}