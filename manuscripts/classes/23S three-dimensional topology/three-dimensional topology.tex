\documentclass{../../../small}
\usepackage{../../../ikhanchoi}

\DeclareMathOperator{\Int}{Int}

\begin{document}
\title{Three-dimensional Topology}
\author{Ikhan Choi\\Lectured by Takahiro Kitayama\\University of Tokyo, Spring 2023}
\maketitle
\tableofcontents

\newpage
\section{Day 1: April 11}

Plan:
\begin{enumerate}
\item Fundamental groups of manifolds
\item Examples and constructions
\item Prime decomposition
\item Loop and sphere theorems
\item Haken manifolds
\item Seifert manifolds
\item JSJ composition
\item Geometrization
\item Virtually special theorems
\end{enumerate}

\noindent References:
\begin{enumerate}
\item J. Hempel, 3-manifolds
\item W. Jaco, Lectures on three-manifold topology
\item B. Martelli, An introduction to geometric topology
\item Morimoto, An introduction to three-dimensional manifolds (Japanses)
\end{enumerate}

\noindent Grading:
Submit a report for any three among the exercises given in the lecture (ITC-LMS Kadai).
Cancellation of class: 5/2, 7/11(maybe)

\noindent Convention:
\begin{itemize}
\item manifold = connected compact orientable smooth manifold
\item surface = connected compact orientable smooth 2-dimensional manifold
\item tub nbd, isotopy extension, transversality, triangulation,$\cdots$
\end{itemize}

\subsection*{1. Fundamental group}
\subsection{Fundamental groups of higher dimensional manifolds}
\begin{prop}
Let $\pi$ be a finitely presented group.
Then, for every $d\ge4$ there is a $d$-manifold $X$ such that $\pi_1(X)\cong\pi$.
\end{prop}
\begin{pf}
Let $\pi=\<x_1,\cdots,x_n\mid r_1,\cdots r_m\>$ be given.
If $Y=(S^1\times S^{d-1})^{\#n}$, then $\pi_1(Y)\cong\<x_1,\cdots\,x_n\>$ by the van Kampen theorem.
Let
\[Z=Y\setminus(\coprod_{i=1}^m\nu(l_i)),\]
where $l_i\subset Y$ is embedded loops representing $r_i$ and $\nu$ denotes the open tubular neighborhood.
Then, $\pd Z=\coprod_{i=1}^ml_i\times S^{d-2}$.
Since $d\ge4$, any loops and disks can be pushed off $l_1\cdots,l_n$, we have an isomorphism $\pi_1(Z)\to\pi_1(Y)$.
Then, if we let
\[X=Z\cup_{\pd}(\coprod_{i=1}^mD^2\times S^{d-2}),\] then $l_i\times*=\pd(D^2\times*)$, we have $\pi_1(X)\cong\pi_1(Y)/\<[l_1],\cdots,[l_m]\>\cong\pi$.
\end{pf}

\subsection{Surfaces and their groups}

\begin{thm}[Rad\'o, Whitehead]
Every topological surface admits a unique smooth and PL structure.
\end{thm}

\begin{thm}
Every surface is diffeomorphic to only one of $\Sigma_{g,b}$, where $\Sigma_{g,b}=(T^2)^{\#g}\#(D^2)^{\#b}$.
\end{thm}

\begin{cor}
$S=S^2,T^2,D^2$ are prime, that is, $S=S_1\#S_2$ implies $S_i\approx S^2$ for $i=1$ or $i=2$.
\end{cor}

\begin{rmk}
If the orientability is reduced out, then $\RP^2$ is prime.
Also note that $T^2\#\RP^2\approx(\RP^2)^{\#3}$.
\end{rmk}

\begin{thm}[Uniformization]
For every surface $S\ne D^2$, its interior admits a complete Riemannian metric of constant curvature
\[\begin{cases}1,&\chi(S)>0\\0,&\chi(S)=0\\-1,&\chi(S)<0\end{cases}\]
with universal covering $S^2$, $\R^2$, $\H^2$, respectively.
\end{thm}

The hyperbolic plane is $\H^2=\{(x,y)\in\R^2:y>0\}$ with the Riemannian metric $ds^2=(dx^2+dy^2)/y^2$, and $\Isom^+(\H^2)=\PSL_2(\R)$.

\begin{prop}
If a surface $S$ has $\chi(S)<0$, then there is a discrete group $\Gamma\le\PSL_2(\R)$ such that $S\approx\H^2/\Gamma$.
In particular, $\pi_1(S)$ is isomorphic to $\Gamma$.
\end{prop}
We have
\[\pi_1(\Sigma_{g,b})\cong F_{2g+b-1}\quad\text{ and }\quad\pi_1(\Sigma_g)\cong\<a_1,b_1,\cdots,a_g,b_g\mid[a_1,b_1],\cdots,[a_g,b_g]\>.\]

\begin{prop}
$\pi_1(\Sigma_g)$ is torsion free.
\end{prop}

\textbf{Exercise 1.} Prove Proposition 1.7.


\begin{thm}[Newman]
$\pi_1(\Sigma_{g,b})$ is linear over $\Z$, that is, is isomorphic to a subgroup of $\GL_n(\R)$.
For example, we can check $F_n\hookrightarrow F_2\hookrightarrow \SL_2(\Z)$ according to the pingpong lemma.
\end{thm}


Over $\R$, we may embed $\pi_1(S)\hookrightarrow\PSL_2(\Z)\cong\SO_{1,2}^+(\R)<\GL_3(\R)$ if $\chi(S)<0$,

\begin{defn*}
A group $\pi$ is called residually finite(RF) if for every $1\ne\gamma\in\pi$ there is a group homomorphism $\f:\pi\to G$ to a finite group $G$ such that $\f(\gamma)\ne1$.
A subgroup $\pi'<\pi$ is called separable if there is a group homomorphism $\f:\pi\to G$ to a finite group $G$ such that $\f(\gamma)\notin\f(\pi')$.
In particular, $\pi$ is residually finite if the trivial subgroup is separable in $\pi$.
A group $\pi$ is called locally extended residually finite(LERF) if every finitely generated subgroup of $\pi$ is separable.
\end{defn*}

\begin{thm}[Mal'cev]
Every finitely generated linear group over a field is residually finite.
\end{thm}

Over $\Z$, if $1\ne(a_{ij})\in\pi<\GL_n(\Z)$ given, then for $m>\max_{i,j}|a_{ij}|$ if we let $\f_m:\pi\hookrightarrow\GL_n(\Z)\twoheadrightarrow\GL_n(\Z/m\Z)$, then $\f_m((a_{ij}))\ne1$.

\begin{thm}[Scott]
$\pi_1(\Sigma_{g,b})$ is LERF.
\end{thm}


\section{Day 2: April 18}

\subsection*{2. Examples and constructions of 3-manifolds}

\begin{thm}[Moise]
Every topological 3-manifold(not neccesarily compact, connected, orientable) admits a unique smooth and PL structure.
\end{thm}

\subsection{Spherical manifolds}

Recall
\begin{align*}
S^3:&=\{x\in\R^4:|x|=1\}\\
&=\{(z,w)\in\C^2:|z|^2+|w|^2=1\}\\
&=\{a+bi+cj+dk\in\H:a^2+b^2+c^2+d^2=1\}.
\end{align*}
\subsubsection*{Lens spaces}
Let $p,q\in\Z$, $p>0$, $(p,q)=1$.
Then, $\Z/p\Z=\<\zeta=\exp(2\pi\sqrt{-1}/p)\>$ acts on $S^3$ such that $\zeta\cdot(z,w)=(\zeta z,\zeta^qw)$.
Then, the Lens spaces are defined as
\[L(p,q):=S^3/(\Z/p\Z)\quad\text{ with }\quad\pi_1(L(p,q))=\Z/p\Z.\]
For example, $L(1,1)=S^3$ and $L(2,1)=\RP^3$.


\begin{thm}[Reidemeister]\,
\begin{parts}
\item $L(p,q)\simeq L(p,q')$ (homotopy equiv) if and only if there is $a\in\Z$ such that $qq'\equiv\pm a^2\pmod p$.
\item $L(p,q)\approx L(p,q')$ (diffeo) if and only if $q'\equiv\pm q^{\pm1}\pmod p$.
\end{parts}
\end{thm}

For example, $L(7,1)\simeq L(7,2)$ since $1\cdot2\equiv 3^2\pmod7$, but $L(7,1)\approx L(7,2)$ since $2\not\equiv\pm1\pmod7$.
\begin{pf}[Sketch of ($\Leftarrow$)]
Direct construction.
(a) With the linking form $H_1(L)\times H_1(L)\to\Q/\Z$.
(b) Reidemeister torsion. 
\end{pf}

\subsubsection*{General quotients}
A spherical manifold is the orbit space $S^3/\Gamma$, where $\Gamma$ is a finite subgroup of $\SO(4)$ and $\Gamma\curvearrowleft S^3$ freely.

\begin{ex*}
With an action $\<-1,i,j,k\>\curvearrowleft S^3$, we obtain the prism manifold.
\end{ex*}
\begin{ex*}
With an action of the binary icosahedral group $\Gamma=\Z/2\Z\rtimes A_5$ on $S^3$, we obtain the Poincar\'e sphere.
We have $H_*(S^3/\Gamma)\cong H_*(S^3)$.
If we take 3/10 turn instead of 1/10, we have the Seifert-Weber space.
\end{ex*}

\subsection{Fibered manifolds}


\subsubsection*{Twisted bundles}
\[N_{g,b}=(\RP^2)^{\#g}\#(D^2)^{\#b}.\]

Let $D$ be a polygon with oriented sides $a_1,a_1',a_2,a_2',\cdots,a_g,a_g'$.
\[N_g\tilde\times[0,1]:=D\times[0,1]/\sim,\] where $(x,t)\sim(x',1-t)$ for $x\in a_i$, $x'\in a_i'$, $t\in[0,1]$ with $[x]=[x']\in N_g$, and it is orientable.
\[N_g\tilde\times S^1:=N_g\tilde\times[0,1]/(x,0)\sim(x,1),\ x\in N_g\]
\[N_{g,b}\tilde\times S^1:=N_g\tilde\times S^1\setminus\nu(b\text{ fibers}).\]

\textbf{Exercise 2.} Show the following:
\begin{parts}
\item $\RP^2\tilde[0,1]\approx\RP^3\setminus\text{(open ball)}$.
\item $\RP^2\tilde\times S^1\approx\RP^3\#\RP^3$.
\item $N_{1,1}\tilde\times S^1\approx N_2\tilde\times[0,1]$.
\end{parts}

\subsubsection*{Mapping tori}
The mapping class group is
\[\cM_{g,b}:=\mathrm{Diff}^+(\Sigma_{g,b},\pd\Sigma_{g,b})/\text{isotopy relative to }\pd.\]

\begin{thm}[Dehn, Lickorish]
$\cM_{g,b}$ is finitely generated by Dehn twists.
\end{thm}
For examples, $\cM_0=\cM_{0,1}=1$ by the Alexander trick, and $\cM_{0,2}\cong\Z$, $\cM_1=\cM_{1,1}\cong\SL_2(\Z)$.

Let $\f\in\cM_{g,b}$.
Then, a mapping torus is defined by
\[M_\f:=\Sigma_{g,b}\times[0,1]/(\f(x),0)\sim(x,1).\]

\subsection{Heegaard decomposition}

A manifold with a boundary
\[H_g:=D^3\cup(D^2\times[0,1])^{\sqcup g}\]
is called the handle body with genus $g$.
Then, $\pi_1(H_g)\cong F_g$ and $\pd H_g\approx\Sigma_g$.
Let $\f:\pd H_g\to\pd H_g$ be an orientation-preserving diffeomorphism, i.e. an element of the mapping class group $\cM_g$.
If a 3-manifold $M$ satisfies
\[M\approx H_g\approx H_g,\]
then the right-hand side is called the Heegaard decomposition(splitting) of $M$.
\begin{prop}
Every closed 3-manifold admitting a Heegaard decomposition of genus 0 is diffeomorphic to $S^3$.
\end{prop}
\begin{pf}
$\cM_0=1$.
\end{pf}
\begin{prop}
Every closed 3-manifold admitting a Heegaard decomposition of genus 1 is diffeomorphic to $S^3$, $S^2\times$, or $L(p,q)$.
\end{prop}
\textbf{Exercise 3.} Prove the above proposition.
\begin{thm}
Every closed 3-manifold $M$ admits a Heegaard decomposition along some $\Sigma_g$.
\end{thm}
\begin{pf}
Pick a triangulation $T$ of $M$.
Then, $H:=\bar{\nu(T^{(1)})}$ is a handlebody.
Then, $H':=M\setminus\mathrm{Int}H$ is also a handlebody.
Since $M$ is orientable, so are $H$ and $H'$, thus we are done.
\end{pf}
There is another proof using Morse theory.

\begin{cor}
The fundamental group of every closed 3-manifold admits a finite presentation of deficiency 0, i.e. the number of generators is equal to the number of relations.
\end{cor}
\begin{pf}
Apply the van Kampen theorem to
\[M=H_g\cup H_g=H_g\cup(D^2\times[-\e,\e])^{\sqcup g}\cup D^3.\qedhere\]
\end{pf}

\subsection{Dehn surgery}
Let $L$ be a link.
The link exterior is the set $E_L=S^3\setminus\nu(L)$.

\begin{prop}
Let $M$ be a 3-manifold, and $T\subset\pd M$ a torus component.
Let $h:\pd(D^2\times S^1)\to T$ be a diffeomorphism.
Then, $M\cup_h(D^2\times S^1)$ is determined only by $\pm[h(\pd D^2\times*)]\in H_1(T)$.
\end{prop}
\begin{pf}
Write
\[M\cup_h(D^2\times S^1)=M\cup_h(D^2\times(-\e,\e))\cup D^3.\qedhere\]
\end{pf}
For a knot $K$, there are two generators $\mu$ and $\lambda$, called the meridian and the longitude, of $H_1(\pd E_K)$ such that $\ker(H_1(\pd E_K)\to H_1(E_K)=\Z\mu)$ is generated by $\lambda$.

\textbf{Exercise 4.} Show that $L(p,q)$ with $p\ne0$ and $S^2\times S^1$ are obtained by the $p/q$-Dehn surgery along the unknot.

\begin{thm}[Lickorish-Wallace]
Every closed 3-manifold can be obtained by an (integral)-Dehn surgery along some link in $S^3$.
\end{thm}
\begin{pf}[Sketch]
Heegaard decomposition and $\cM_g=\<\text{Dehn twists}\>$.
Each Dehn twist realizes the Dehn surgery steps.
\end{pf}

\newpage
\section{Day 3: April 25}
\subsection*{3. Prime decomposition}

\subsection{Alexander's theorem}
\begin{thm}
Every (smooth) embedding $S^2\subset\R^3$ bounds some (smooth) embedding $D^3\subset\R^3$.
\end{thm}
\begin{rmk*}
The above theorem does not hold in the category of topological spaces.
Alexander's horned sphere is one of the counterexamples.

If $\R^d$ for $d\ge5$, then more complicated result such as h-cobordism theorem must be used to obtain the same conclusion.
\end{rmk*}
\begin{pf}[Sketch]
Isotope such a sphere $S$ so that the coordinate $z:S\to\R$ is a Morse function.
Assume that for all $p\ne q\in\mathrm{Crit}(z)$, then $z(p)\ne z(q)$.
We use induction on $(m,n)$, where $m$ is the number of saddles and
\[n:=\min\{\#\pi_0(S\cap z^{-1}(r)):\text{$r$ is a regular value s.t. $z(p)<r<z(q)$ for some saddles $p,q$}\}.\]
Note that $\#(\text{minima of $z$})-m+\#(\text{maxima of $z$})=\chi(S)=2$.

For the case $m=0$ so that there are only one minimum and maximum, then we can construct a ball by applying the Jordan-Sch\"onflies theorem to each level.

For the case $m=1$, then only four types appear: a jelly bean, a red blood cell, and their upside down versions.
Apply the Jordan Sch\"onflies again.

For the case $m\ge2$, let $r$ be a regular value realizing the value of $n$.
Let $D$ be union of the closure of the interior of the innermost circles of $S\cap z^{-1}(r)$.
Replace $S$ by $(S\setminus\pd D\times(-\e,\e))\cup(D\times\{-\e,\e\})$.
Then, each connected component has lower $(m,n)$ so that it bounds a ball.
Attaching all balls bounded by the components with balls $D\times[-\e,\e]$, $S$ also bounds a ball.
\end{pf}

\subsection{Irreducible manifolds}

The connected sum is defined as
\[M\# N:=(M\setminus(\text{open ball}))\cup_\partial(N\setminus(\text{open ball})).\]
\begin{prop}\,
\begin{parts}
\item $M\# N\approx N\# M$.
\item $(M_1\# M_2)\# M_3\approx M_1\#(M_2\# M_3)$.
\item $M\# S^3\approx M$.
\end{parts}
\end{prop}

We say a manifold $M$ is \emph{prime} if $M=N_1\# N_2$ implies $N_1\approx S^3$ or $N_2\approx S^3$.
We say a 3-manifold $M$ is \emph{irreducible} if every embedding $S^2\subset M$ bounds some embedding $D^3\subset M$.
In other words, in a prime manifold every separating sphere bounds a ball, in an irreducible manifold every sphere bounds a ball.

\begin{cor}
Every irreducible 3-manifold is prime.
\end{cor}
\begin{cor}
By Theorem 3.1, $S^3$ is irreducible.
\end{cor}
\begin{thm}\,
\begin{parts}
\item $S^2\times S^1$ is prime, but is not irreducible.
\item Every closed prime 3-manifold which is not irreducible is diffeomorphic to $S^2\times S^1$.
\end{parts}
\end{thm}
\begin{pf}
(a)
A sphere $S^2\times *$ cannot bound any $D^3\subset S^2\times S^1$ because $[S^2\times *]\ne0\in H_2(S^2\times S^1)$, so $S^2\times S^1$ is not irreducible.

Suppose $S^2\times S^1=N_1\# N_2$.
Since $\pi_1(N_1)*\pi_1(N_2)\cong\Z$, one of $\pi_1(N_1)$ or $\pi_1(N_1)$ is trivial.
Assume $\pi_1(N_1)$ is trivial and let $B:=N_1\setminus(\text{open ball})$.
Since $B$ is also simply connected, it lifts diffeomorphically into the universal cover $S^2\times\R$ of $S^2\times S^1$.
Because $S^2\times\R\approx\R^3\setminus\{0\}$, we have an embedding $B\subset\R^3$.
Because $\partial B\approx S^2$, by Theorem 3.1 we have $B\approx D^3$, so $N_1\approx S^3$.

(b)
If every sphere in $M$ is separating, then it has to be irreducible since $M$ is prime, so such $M$ contains a nonseparating sphere $S$.
Let $\gamma$ be an arc connecting the inside and the outside of $\pd\nu(S)$.
If we let $M':=\bar{\nu(S)}\cup\bar{\nu(\gamma)}$, then $\pd M'\approx S^2\#(S^1\times I)\#S^2\approx S^2$ is a separating sphere.
Since $M\setminus\mathrm{Int}M'\approx D^3$ because $M'$ is not simply connected and $M$ is prime, $M$ is diffeomorphic to $S^2\times S^1$.
\end{pf}

\begin{prop}
If a covering space $\tilde M$ of $M$ is irreducible, then so is $M$.
\end{prop}

\noindent\textbf{Exercise 4.} Prove Proposition 3.6.
\begin{rmk*}
The converse of Proposition 3.6 is also known to be true.
\end{rmk*}

\subsection{Normal surfaces}

Fix a 3-manifold $M$ and its triangulation $T$.
A (possibly disconnected) subsurface $S\subset M$ is called a \emph{normla surface} with respect to $T$ if $S$ is a union of \emph{normal disks}, defined as seven types of disks in a given tetrahedron: four triangles and three quadrilaterals.

\begin{prop}
Every (possibly disconnected) subsurface $S\subset M$ becomes a normal surface with respect to $T$ by isotopies and the following operations:
\begin{enumerate}[(i)]
\item Replace $S$ by $(S\setminus\partial D\times(-\e,\e))\cup(D\times\{\pm\e\})$ for a disk $D$ satisfying $D\cap(S\cup\partial M)=\partial D$.
\item Remove a components of $S$ contained in a ball in $M$.
\end{enumerate}
\end{prop}
\begin{pf}
Isotope $S$ so that $S\pitchfork T$.
It is sufficient to realize the following:
\begin{parts}
\item For every tetrahedron $\Delta^{(3)}\subset T^{(3)}$, $S\cap\Delta^{(3)}=\coprod(\text{disks})$.
\item For every disk component $D$ in (a) and for every edge $\Delta^{(1)}\in T^{(1)}$, $\#(D\cap\Delta^{(1)})\le1$.
\item For every triangle $\Delta^{(2)}\subset T^{(2)}$, $S\cap\Delta^{(2)}=\coprod(\text{arcs})$.
\end{parts}

For (a), if there is a non-disk connected component of $S\cap\Delta^{(3)}$, perform (i) along innermost loops in $S\cap\partial\Delta^{(3)}$.
Then, perform (ii) for closed components of $S$ contained in $\Delta^{(3)}$.

For (b), if there is a disk component of $S\cap\Delta^{(3)}$ which intersects an edge $\Delta^{(1)}$ more than twice, then for an inner most pair of two points connected by an arc in $D$, push the arc to the outside $\Delta^{(3)}$ with an ambient isotopy.

For (c)
\end{pf}


\newpage
\section{Day 4: May 9}
\setcounter{section}{3}
\setcounter{subsection}{3}
\setcounter{thm}{7}

\subsection{Prime decomposition theorem}
\begin{prop}
Let $T$ be a triangulation of $M$ and let $S\subset M$ be a normal surface with respect to $T$.
If no component of $M\setminus\nu(S)$ is a trivial $[0,1]$-bundle, then
\[\#\pi_0(S)\le6\cdot\#T^{(3)}+\rk H_2(M,\partial M;\Z/2\Z).\]
\end{prop}
\begin{pf}
Let $M_0$ be a component of $M\setminus\nu(S)$ such that for each tetrahedron $\Delta^{(3)}$ the intersection $M_0\cap\Delta^{(3)}$ is a trivial $[0,1]$-bundle over a normal disk $D_0(\Delta^{(3)})\subset\Delta^{(3)}$.
Let $S'=(S\setminus\sqcup\partial M_0)\sqcup(\sqcup D_0)$.
Then,
\[\#\pi_0(S)=\#\pi_0(S')=\rk H_2(S',\partial S';\Z/2\Z),\]
and since
\[H_3(M,\partial M\cup S';\Z/2\Z)\to H_2(S',\partial S;\Z/2\Z)\to H_2(M,\partial M;\Z/2\Z)\]
is exact, we have
\[\rk H_2(S',\partial S';\Z/2\Z)\le\rk H_3(M,\partial M\cup S';\Z/2\Z)+\rk H_2(M,\partial M;\Z/2\Z).\]
If we let $M':=(M\setminus\nu(S))\setminus\sqcup M_0$, then by the excision theorem,
\[\rk H_3(M,\partial M;\Z/2\Z)=\rk H_3(M',\partial M';\Z/2\Z)=\pi_0(M').\]
Every component of $M'$ is not a trivial $[0,1]$-bundle over a normal disk in some tetrahedron in $T$.
The number of such non-trivial components of the complement $\Delta^{(3)}\setminus\nu(S)$ is at most 6.
\end{pf}



\begin{thm}[Prime decomposition theorem; Kneser, Milnor]
Every 3-manifold $M\ne S^3$ admits a decomposition
\[M=N_1\#\cdots\# N_n,\quad N_i:\text{ prime }\ne S^3.\]
Moreover, such a decomposition is unique up to diffeomorphisms and permutations.
\end{thm}
\begin{pf}
(Existence)
If $M$ contains a non-separating sphere, then there is $M'$ such that $M=M'\#(S^2\times S^1)$ as in the proof of Proposition 3.5.
Since we have $H_(M)=H_1(M')\oplus\Z$, the number of such $S^2\times S^1$ factors is finite, so by cutting out all of them, we may assume that every sphere in $M$ is separating.

Let $S$ be an arbitrary union of spheres in $M$ such that
\[\text{no components of $M\setminus\nu(S)$ is diffeomorphic to $S^3\setminus(\text{balls})$}.\tag{\dagger}\]
It suffices to show that $\pi_0(S)$ is finite.
Let $T$ be a triangulation of $M$ and $S'\subset M$ be a normal surface with respect to $T$ obtained from $S$ as Proposition 3.7.
Suppose that a component $S_0\subset S$ is contained in a ball.
Then by the Alexander theorem such an innermost component would bound a ball, which contradicts to (\dagger), hence the operation (ii) in Proposition 3.7 does not occur when we construct $S'$.
Each operation (i) splits a component $S_0\subset S$ into two spheres $S_1$ and $S_2$, and one of them satisfies (\dagger).
Therefore, there exists a union of components $S''\subset S'$ satisfying the same condition with $S$ with $\pi_0(S'')=\pi_0(S)$, which is bounded by Proposition 3.8.

(Uniqueness)
Suppose
\[P_1\#\cdots\#P_m\#(S^2\times S^1)^{\#k}=Q_1\#\cdots\#Q_n\#(S^2\times S^1)^{\#l},\]
where $P_i$ and $Q_j$ are irreducible but not $S^3$.
It suffices to find a union of spheres $T\subset M$ such that
\begin{enumerate}
\item[(P)]$M\setminus\nu(T)=\coprod_{i=1}^m(P_i\setminus(\text{balls}))\sqcup(\coprod S^3\setminus(\text{balls}))$,
\item[(Q)]$M\setminus\nu(T)=\coprod_{j=1}^n(Q_j\setminus(\text{balls}))\sqcup(\coprod S^3\setminus(\text{balls}))$.
\end{enumerate}
Pick unions of spheres $R,S\subset M$ such that $R$ satisfies (P), $S$ satisfies (Q), and $R\pitchfork S$.
If $R\cap S=\varnothing$, then let $T:=R\sqcup S$.
If not, pick components $R_0\subset R$ and $S_0\subset S$ such that $R_0\cap S_0\ne\varnothing$.
Split $R_0$ along an innermost disk in $S_0$ into two spheres $R_1$ and $R_2$, and replace $R$ by $R':=(R\setminus R_0)\sqcup R_1\sqcup R_2$.
Then, $R'$ satisfies (P) and $\#\pi_0(R'\cap S)<\#\pi_0(R\cap S)$ so that such operations reduce to the case of $R\cap S=\varnothing$.
\end{pf}


\noindent\textbf{Exercise 5.} Show that the existence of the prime decomposition theorem follows from the Poincar\'e conjecture and Grushko's theorem: $\rk G*H=\rk G+\rk H$, where the rank is defined to be the minimum number of generators.

\begin{thm}[Kneser conjecture, proved by Stallings]
Let $M$ be an oriented 3-manifold such that $\pi_1(M)=G_1*G_2$.
Then, there are oriented 3-manifolds $N_1$ and $N_2$ such that $M=N_1\#N_2$ and $\pi_1(N_i)\cong G_i$ for $i\in\{1,2\}$.
\end{thm}
For this theorem, it suffices to study irreducible 3-manifolds.




\setcounter{section}{4}
\setcounter{subsection}{0}
\setcounter{thm}{0}
\subsection*{4. Loop and sphere theorems}
\subsection{Loop theorem}

\begin{thm}[Loop theorem]
For a subsurface $S\subset M$, there is a continuous embedding $f:(D^2,\partial D^2)\to(M,S)$ such that $f|_{\partial D^2}\ne\text{null homotopic in }S$.
In particular, we may assume $f$ is smooth.
\end{thm}

\begin{rmk*}
The loop theorem holds also for noncompact or nonorientable 3-manifolds.
\end{rmk*}

\begin{cor}[Dehn's lemma]
Every loop in $\partial M$ null homotopic in $M$ bounds some embedded disk in $M$.
\end{cor}
\begin{pf}
Apply the loop theorem for the tubular neighborhood of the loop in $\partial M$.
\end{pf}

\begin{thm}
Let $K\subset S^3$ be a knot and $E_K:=S^3\setminus\nu(K)$.
Then, $K$ is unknot if and only if $\pi_1(E_k)\cong\Z$.
\end{thm}
\begin{pf}
($\Rightarrow$)
Clear.

($\Leftarrow$)
Take a longitude $\lambda\in\pi(\partial E_K)=H_1(\partial E_K)$ such that
\[\lambda\in\ker(\pi_1(\partial E_K)\to H_1(E_K))\subset\iota_*^{-1}([\pi_1(E_K),\pi_1(E_K)]),\]
where $\iota:\partial E_K\hookrightarrow E_K$.
Then, $\pi_1(E_K)\cong\Z$ implies $\lambda=1$, and apply Dehn's lemma.
\end{pf}

\textbf{Exercise 6.} Let $M$ be an irreducible 3-manifold and $T\subset M$ a torus.
Show that the followins are equivalent:
\begin{enumerate}[(i)]
\item $\pi_1(T)\to\pi_1(M)$ is not injective.
\item $T$ bounds some embedded solid torus $D^2\times S^1$ in $M$ or $T$ is contained in some embedded ball in $M$.
\end{enumerate}

\newpage
\section{Day 5: May 16}
\setcounter{section}{4}
\begin{thm}[Loop theorem, Papakyriakopoulos]
For a subsurface $S$ of 3-manifold $M$, if there is a continuous map $f:(D^2,\partial D^2)\to(M,S)$ such that $f|_{\partial D^2}$ is not null-homotopic, then there is a smooth embedding $g:(D^2,\partial D^2)\to(M,S)$ such that $g|_{\partial D^2}$ is not null-homotopic.
\end{thm}
\begin{pf}
Step 1: \emph{Tower construction}.
Pick triangulations of $(M,S)$ and $D^2$.


Let $f_0:(D^2,\partial D)\to(M,S)$ be a simplicial map such that $f_0\simeq f$.
Define $V_0\subset M$ to be a 3-submanifold deformation retracting to $f_0(D^2)$ and $S_0\subset\partial V_0$ a component of $S\cap V_0$ containing $f_0(\partial D^2)$.

Given the $i$-th level, for any double covering $p_{i+1}:M_{i+1}\to V_i$, if it exists, let $f_{i+1}:(D^2,\partial D^2)\to(M_{i+1},p_{i+1}^{-1}(S_i))$ be a lift of $f_i$.
Similarly as the case $i+1=0$, define $V_{i+1}\subset M_{i+1}$ to be a 3-manifold deformation retracting to $f_{i+1}(D^2)$ and $S_{i+1}\subset\partial V_{i+1}$ the component of $p_{i+1}^{-1}(S_i)\cap V_{i+1}$ containing $f_{i+1}(\partial D^2)$.

\begin{cd}
&V_n \ar[hook]{r}& M_n \ar{ld}{p_n}&&
&S_n\ar[hook]{r}\ar{d}&\partial V_n\\
&\vdots\,&\vdots\ar{ld}{p_2}&&
&\vdots\ar{d}&\\
&V_1 \ar[hook]{r}& M_1 \ar{ld}{p_1}&&
&S_1\ar[hook]{r}\ar{d}&\partial V_1\\
D^2\ar{r}{f_0}\ar{ur}{f_1}\ar{uuur}{f_n}&V_0 \ar[hook]{r}& M_0=M&&
\partial D^2\ar{r}\ar{ur}\ar{uuur}&S_0\ar[hook]{r}&\partial V_0
\end{cd}

Since
\[\text{\#of simplices of $f_0(D^2)$}<\text{\#of simplices of $f_1(D^2)$}<\cdots\le\text{\#of simplices of $D^2$},\]
there is $n$ such that $V_n$ has no double cover.
If we let $N_i:=\ker(\pi_1(S_i)\to\pi_1(S))$, then $f_{i*}([\partial D^2])\notin N_i$ implies that $N_i\ne\pi_1(S_i)$ for each $i$.

Step 2: \emph{Embedding on the top}.
Consider the following exact sequence:
\[H_2(V_n,\partial V_n;\Z/2\Z)\to H_1(\partial V_n;\Z/2\Z)\to H_1(V_n;\Z/2\Z).\]
Since $V_n$ has no double cover, the Poincar\'e duality writes
\[H_2(V_n,\partial V_n;\Z/2\Z)\cong H^1(V_n;\Z/2\Z)\cong\Hom(\pi_1(V_n),\Z/2\Z)=0.\]
Also we have
\[H_1(V_n;\Z/2\Z)\cong H_1(V_n)\otimes\Z/2\Z=0,\]
which implies $H_1(\partial V_n;\Z/2\Z)=0$ so that every component of $\partial V_n$ is a sphere.

Note that $S_n\approx S^2\setminus(\text{disks})$ and $\pi_1(S_n)$ is generated by components of $\partial S_n$.
Since $N_n\ne\pi_1(S_n)$, there is a component $l\subset\partial S_n$ such that $[l]\notin N_n$.
Pick

Step 3: \emph{Descending the tower}.
We shall show that given an embedding $g_i:(D^2,\partial D^2)\to(V_i,S_i)$ such that $g_{i*}([\partial D^2])\notin N_i$, there is such an embedding $g_{i-1}$.
Then, $g_0$ is the desired embedding.

Isotope $p_i\circ g_i$ so that self-intersections are simple double curves.
These are modified by the following operations which give $g_{i-1}$.


Case 2 double arces
Therse are cancelled in two ways and at least one of the results satisfies $g_{i-1}([\partial D^2])\notin N_{i-1}$.
\end{pf}


Let $X$ be a path-connected space and $x_0\in X$.
Then,
\[\pi_n(X):=\pi_n(X,x_0)=\{f:(S^n,*)\to (X,x_0)\}/\text{homotopy fixing base points}.\]
For a covering $p:\tilde X\to X$, the induced homomorphism $p_*:\pi_n(\tilde X)\to\pi_n(X)$ is an isomorphism for $n\ge2$.
We will use the following two theorems without proofs
\begin{thm*}[Whitehead theorem]
Let $X,Y$ be CW-complexes.
If $\f:X\to Y$ satisfies $\f_*:\pi_n(X)\to\pi_n(Y)$ is an isomorphism for each $n$, then $\f$ is a homotopy equivalence.
\end{thm*}
\begin{thm*}[Hurewicz theorem]
For, $h_n:\pi_n(X)\to H_n(X):[f]\mapsto f_*([S_n])$, $h_1$ is the abelianization, and $h_{n+1}$ is an isomorphism if $\pi_i(X)=0$ for $1\le i\le n$.
\end{thm*}
\setcounter{thm}{3}
\begin{thm}[Sphere theorem, Papakyriakopoulos]
For a 3-manifold $M$ with $\pi_2(M)\ne0$, there exists an embedding $f:S^2\to M$ such that $[f]\ne0$.
\end{thm}
\begin{cor}
If $M$ is irreducible, then $\pi_2(M)=0$.
\end{cor}
\begin{thm}
Let $M$ be an irreducible 3-manifold with infinite fundamental group.
\begin{parts}
\item $M$ is aspherical, i.e.~$\pi_n(M)=0$ for $n\ge2$.
\item $\pi_1(M)$ is torsion-free.
\end{parts}
It is well-known that a manifold $M$ is aspherical if and only if $M\simeq K(\pi_1(M),1)$.
\end{thm}

\begin{lem}
Let $m\ge2$.
Just use the identity $H_*(\pi)=H_*(K(\pi,1))$ as the definition of group homology.
Then,
\[H_n(\Z/m\Z)=\begin{cases}\Z&n=0,\\\Z/m\Z&n:\text{ odd,}\\0&\text{ otherwise.}\end{cases}\]
In fact, $K(\Z/m\Z,1)\simeq S^\infty/(\Z/m\Z)$.
\end{lem}
\begin{pf}[Proof of Theorem 4.6]
(a)
By Corollary 4.5, we have $\pi_2(M)=0$.
If we let $\tilde M$ be the universal cover of $M$, which is not compact because $\pi_1(M)$ is infinite, so that we have
\[\pi_3(M)=\pi_3(\tilde M)=H_3(\tilde M)=0.\]
Inductively, and since $\dim M=3$, we have for $n\ge4$ that
\[\pi_n(M)=\pi_n(\tilde M)=H_n(\tilde M)=0.\]

(b)
Suppose that there is a non-trivial finite cyclic subgroup $C\le\pi_1(M)$.
If we let $\hat M$ be the cover of $M$ corresponding to $C$, then $\pi_1(\hat M)=C$.
It follows from (i) that $\hat M=K(C,1)$.
By Lemma 4.7, we have $H_n(\hat M)\ne0$ for odd $n$, but $\dim\hat M=3$ implies that $H_n(\hat M)=0$ for $n\ge4$.
\end{pf}
\begin{thm}
The only abelian groups appearing as the fundamental group of closed (orientable) 3-manifolds are $1,\Z,\Z^3,\Z/n\Z$.
\end{thm}
\begin{rmk*}
If we also consider non-orientable 3-manifolds, then $\Z\times\Z/2\Z$ also appears as $\pi_1(\RP^2\times S^1)$.
\end{rmk*}

\noindent\textbf{Exercise 7.}
Show that the only abelian groups admitting presentations of deficiency zero are
\[1,\Z,\Z^2,\Z^3,\Z/n\Z,\Z\times\Z/n\Z.\]
(Hint: use Hopf's formula $H_2(\pi)=R\cap[F,F]/[F,R]$, where $\pi=F/R$.)


\begin{pf}[Proof of Theorem 4.8]
By Corollary 2.7 and Exercise 7, it suffices to show $\pi_1(M)\not\cong\Z^2,\Z\times\Z/n\Z$.
By the prime decomposition theorem, we may assume that $M$ is irreducible.
By the part (b) of Theorem 4.6, $\pi_1(M)\not\cong\Z\times\Z/n\Z$.
Now suppose that $\pi_1(\Z)\cong\Z^2$.

Since $M$ is aspherical by Theorem 4.6, $M\cong K(\Z^2,1)\simeq\T^2$.
However, $\Z=H^3(M)\cong H^3(\T^2)=0$ leads to a contradiction.
\end{pf}


\newpage
\setcounter{section}{5}
\section{Day 6: May 23}
\setcounter{section}{5}
\subsection*{5. Haken manifolds}
In the following, we only consider \emph{irreducible} 3-manifolds, i.e. every embedded sphere bounds a ball.

\subsection{Essential surfaces}
When we say subsurfaces $S$, we always assume $S\cap\partial M=\partial S$.

A subsurface $S\subset M$ which is not a sphere is called \emph{incompressible} if there is no (non-trivial) compressing disk.
A \emph{compressing disk} of a subsurface $S$ is a disk $D\subset M$ such that $D\cap S=\partial D$ and $\partial D$ bounds no disk in $S$.
The second condition intuitively says that the compression does not generate spheres so that the disk compresses $S$ non-trivially.

A subsurface $S\subset M$ which is not a sphere is called \emph{boundary incompressible} if there is no (non-trivial) boundary compressing disks.
A \emph{boundary compressing disk} of $S$ is a disk $D$ such that arcs $\gamma=D\cap S$ and $\gamma'=D\cap\partial M$ form the boundary $\partial D=\gamma\cup_\partial\gamma'$ and $\gamma\cup_\partial\gamma''$ does not bound a disk in $S$ for any arc $\gamma''\subset\partial S$.

A surface $S$ is called \emph{boundary parallel} if $S$ can be isotoped into $\partial M$.

A (possibly disconnected) subsurface $S\subset M$ is called an \emph{essential surface} if every component $S_0$ of $S$ is incompressible, boundary incompressible, and not boundary parallel. ($S_0$ must not be a sphere)

\begin{prop}
For a subsurface $S\subset M$, $S$ is incompressible if and only if $\pi_1(S)\to\pi_1(M)$ is injective.
\end{prop}
\begin{pf}
The loop theorem.
\end{pf}
\begin{prop}
For nonzero $\alpha\in H_2(M,\partial M)$, there is an essential surface $S\subset M$ with $[S]=\alpha$.
\end{prop}

Let $S$ be a possibly disconnected subsurface of $M$.
We define a complexity $c(S):=\sum_i(2-\chi(S_i))^2$, where $S_i$ are components of $S$.

\begin{lem}
If $S'$ is obtained from $S$ by the surgery along (boundary) compressing disks, then $c(S')<c(S)$.
\end{lem}
\begin{pf}
We may assume $S$ is connected.
Note that every component has $\chi<2$ since we are not allowing spheres after surgery.
If $S'$ is connected, then since $\chi(S')=\chi(S)+2$, we are done.
If $S'=S_1\sqcup S_2$, then since $\chi(S_1)+\chi(S_2)=\chi(S')=\chi(S)+2$, we are done.
\end{pf}

\begin{pf}[Proof of Proposition 5.2]
Pick a (possibly disconnected) subsurface $S_0\subset M$ with $[S_0]=\alpha$.
(Choose $f:M\to S^1$ so that $[f]\in[M,S^1]=H^1(M)$ is equal to the Poincar\'e dual of $\alpha$ and let $S_0:=f^{-1}(\text{a regular value}$)
Perform the surgery $S_0$ along (boundary) compressing disks until every components of $S_0$ becomes (boundary) incompressible.
By Lemma 5.3, this operation stops, and before and after of the compression are homologous.
Removing all spheres and boundary parallel components of $S_0$, now we have a desired subsurface $S$ with $[S]=[S_0]=\alpha$.
Since $\alpha\ne0$, $S\ne\varnothing$.
\end{pf}


A 3-manifold is a \emph{Haken manifold} if it admits an essential surface in $M$.

\begin{cor}
If $b_1(M)>0$, then $M$ is Haken.
\end{cor}
\begin{pf}
It follows from Proposition 5.2.
\end{pf}
\begin{cor}
If there is a non-sphere component of $\partial M$, then $M$ is Haken.
\end{cor}
\begin{pf}
Since $H_1(\partial M)$ contains an element of infinite order, the exact sequence
\[H^1(M)\cong H_2(M,\partial M)\to H_1(\partial M)\to H_1(M)\]
implies $b_1(M)>0$.
\end{pf}


\begin{thm}[Thurston]
For the figure eight knot, the Dehn filling by a rational number $r=p/q$ is Haken if and only if $r=0,\pm4$.
\end{thm}

\subsection{Haken hierarchy}

\begin{prop}
Let $S\subset M$ be an essential surface.
Let $M'=M\setminus\nu(S)$.
Then, (a) $M'$ is irreducible and (b) every closed incompressible surface in $M'$ is incompressible also in $M$.
\end{prop}
\begin{pf}
(a)
Let $S'\subset M'$ be arbitrary sphere.
Since $M$ is irreducible, $S'$ bounds a ball $B\subset M$.
Then, we have $S\subset B$ or $S\cap B=\varnothing$.
Since $B$ contains no closed incompressible surface, $B\cap S=\varnothing$, so $B\subset M'$.

(b)
Suppose that there is a closed incompressible surface $S'\subset M'$ not incompressible in $M$.
Let $D\subset M$ be a compressing disk for $S'$ such that $D\pitchfork S$ and $\#\pi_0(D\cap S)$ is minimal.
Since $S$ is essential, an innermost loop in $S\cap D\subset S$ bounds a disk $D'\subset S$ (if we take any compressible disk of $S$ in $M$ whose boundary is $S\cap D$, then the assumption $S$ is essential implies that the compression gives a sphere; we can take a disk $D'$).
Since $M$ is irreducible, $D\cup D'$ bounds a ball in $M$, so $D$ can be isotoped such that $\#\pi_0(D\cap S)$ would reduce, which is a contradiction.
\end{pf}

\begin{prop}
If $\partial M\ne\varnothing$ and $M$ contains no closed incompressible surface, then $M$ is a handlebody.
\begin{pf}
Cut $M$ along essential disks into $\coprod_iM_i$ until every component of $\partial M_i$ is incompressible.
Suppose that there is $i_0$ such that $M_{i_0}\not\approx D^3$.
By Proposition 5.7(a), every component of $\partial M_{i_0}$ is not a sphere.
By Proposition 5.7(b), the componenets of $\partial M_{i_0}$ are closed incompressible surfaces in $M$ which contradicts the assumption.
Therefore, for every $i$ we have $M_i\approx D^3$, $M$ is a handlebody.
\end{pf}
\end{prop}

\begin{thm}[Haken hierarchy]
For every Haken manifold $M$, there is a sequence of irreducible (possibly disconnected) 3-manifolds $M=M_0,M_1,\cdots,M_n$ such that
\begin{parts}
	\item $M_{i+1}=M_i\setminus\nu(S_i)$ for some essential surface $S_i\subset M_i$
	\item $M_n\approx \coprod D^3$.
\end{parts}
\end{thm}

The strongest feature of the Haken hierarchy is the possibility of using induction.

\begin{thm}[Waldhausen]
Two closed Haken manifolds with isomorphic fundamental groups are diffeomorphic.
\end{thm}
\begin{pf}
Induction on the Haken hierarchy.
\end{pf}

\begin{rmk*}
When $\partial M\ne\varnothing$, the theorem also holds under a certain additional condition on isomorphisms on $\pi_1$.
\end{rmk*}


\newpage
\setcounter{section}{6}
\section{Day 7: June 6}
\setcounter{section}{5}

\noindent\textbf{Exercise 8.} Show that every connected essential surface in $\Sigma_g\times[0,1]$ ($g\ge1$) is isotopic to an anulus $\gamma\times[0,1]$ for some simple closed curve $\gamma$ bounding no disk in $\Sigma_g$.
\setcounter{thm}{10}
\begin{lem}
Let $M$ be an irreducible 3-manifold.
Let $\partial M=\coprod_iR_i$, where $R_i$ are components which are not diffeomorphic to sphere.
Then, there is $\alpha\in H_2(M,\partial M)$ such that for every $i$ we have $\partial\alpha|_{R_i}\ne0\in H_1(R_i)$.
\end{lem}

\noindent\textbf{Exercise 9.} Prove Lemma 5.11.

\begin{pf}[Proof of Theorem 5.9]
Note that the theorem holds for handlebodies $H_g$.
We may assume $M\not\approx H_g$.
By Proposition 5.8, $M$ contains a union of closed incompressible surfaces $S$.
By Proposition 3.7 and 3.8, $S$ is isotopic to a normal surface with respect to some triangulation of $M$, and if no component of $M\setminus\nu(S)$ is a trivial $[0,1]$-bundle, then $\pi_0(S)$ is bounded.

Take $S_0\subset M$ with maximal fundamental group.
Let $M_1=M\setminus\nu(S_0)$.
Let $R_i$ are components of $\partial M_1$ such that $R_i\not\approx S^2$.
By Corollary 5.5 and Proposition 5.7(a), $M_1$ is Haken.
By Lemma 5.11, pick $\alpha\in H_2(M,\partial M)$ such that $\partial\alpha|_{R_i}\ne0\in H_1(R_i)$.
By Proposition 5.2, pick an essential surface $S_1\subset M$ with $[S_1]=\alpha$.
Since $\pi_0(S_0)$ is maximal, there is a component $M'$ of $M\setminus(S_0\sqcup S')$ is a trivial $[0,1]$-bundle.
Since $\partial\alpha|_{R_i}=[S_1\cap R_i]\ne0$, some component of $S_1$ intersects only one component of $\partial M'$.
However, there is not such an essential in $M'\approx S'\times[0,1]$.
Then, constradiction(Exercise 8).
\end{pf}


\setcounter{section}{6}
\subsection*{6. Seifert manifolds}
\subsection{Seifert fibrations}
\setcounter{thm}{0}

Let $S$ be a (possibly non-orientable) surface.
Let $p_i,q_i\in\Z$, $p_i\ge1$, $(p_i,q_i)=1$, $1\le i\le k$.

If orientable, then $\times^{(\sim)}=\times$, and if non-orientable, then $\times^{(\sim)}=\tilde\times$, the twisted direct product.

\[M(S;(p_1,q_1),\cdots,(p_k,q_k)):=\left((S\setminus\coprod_{i=1}^kD_i)\times^{(\sim)}S^1\right)\cup_{h_1,\cdots,h_k}\coprod_{i=1}^k(D^2\times S^1),\]
where $h_i:\partial(D^2\times S^1)\to\partial(D_i\times S^1)$ is a diffeomorphism such that
\[[h(\partial D^2\times\{*\})]=p_i[\partial D_i\times\{*\}]+q_i[\{*\}\times S^1]\in H_1(\partial D_i\times S^1).\]

For example,
\begin{itemize}
\item $M(S;(1,0))\approx S\times^{(\sim)}S^1$.
\item $M(D^2;(p,q))\approx D^2\times S^1$.
\item $M(S^2;(p,q))\approx L(q,p)$.
\item $M(S^2;(p_1,q_1),(p_2,q_2))\approx L(p_1q_2+p_2q_1,q_2r+p_2s$, where $p_1s-q_r=\pm1$.
\item $M(D^2;(p,q))=D^2\times[0,1]/(r,\theta,1)\sim(r,\theta+\frac{2\pi q}p,0)$.
\end{itemize}
Note that we have $M(D^2;(p,q))\to D^2/(r,\theta)\sim(r,\theta+\frac{2\pi n}p)$ for $n\in\Z$.
As a generallization of this, we have the \emph{Seifert fibration}
\[M(S;(p_1,q_1),\cdots,(p_k,q_k))\to S(p_1,\cdots,p_k),\]
where $S(p_1,\cdots,p_k)$ is an orbifold constructed from $S$ with cone points of order $p_1,\cdots,p_k$.
A \emph{singular fiber} is the fiber of a cone point of order $\ge2$.
\begin{rmk*}
Every Seifert fibration without any singular fiber is a $S^1$-bundle over $S$.
\end{rmk*}

\begin{prop}
Every $S^1$-bundle over $S$ with orientable total space $M$ is a Seifert fibration.
\end{prop}
\begin{pf}
If $\partial S\ne\varnothing$, then $M\cong S\times^{(\sim)}S^1$.
If $\partial S=\varnothing$, then $M\setminus\nu(\text{fiber})\cong(S\setminus(\text{open disk}))\times^{(\sim)}S^1$.
Thus there is $e\in\Z$ such that $M\cong M(S;(1,e))$.
\end{pf}

\noindent\textbf{Exercise 10.} Show that if $\partial S=\varnothing$ and $e,e'\in\Z$, then $M(S;(1,e))\cong M(S;(1,e'))$ if and only if $e=\pm e'$.

Let $\pi:M(S;(p_1,q_1),\cdots,(p_k,q_k))\to S(p_1,\cdots,p_k)$ be a Seifert fibration.
The \emph{Euler number} of $\pi$ is
\[e(\pi):=\sum_{i=1}^k\frac{q_i}{p_i},\]
which is set to sit in $\Q$ if $\partial S=\varnothing$, and in $\Q/\Z$ if $\partial S\ne\varnothing$.


\begin{thm}
Two Seifert fibrations
\[\pi:M(S;(p_1,q_1),\cdots,(p_k,q_k))\to S(p_1,\cdots,p_k)\]
and
\[\pi':M(S;(p_1,q'_1),\cdots,(p_k,q'_k))\to S(p_1,\cdots,p_k)\]
with $p_i\ge2$ for all $i$ are isomorphic by an orientation preserving diffeomorphism if and only if the following two conditions hold:
\begin{enumerate}[(i)]
\item $q_i\equiv q_i'\pmod{p_i}$ for all $i$,
\item $e(\pi)=e(\pi')$.
\end{enumerate}
\end{thm}
\begin{rmk*}
We write $M(S;(p_1,-q_1),\cdots,(p_k,-q_k))=-M(S;(p_1,q_1),\cdots,(p_k,q_k))$. 
\end{rmk*}
This theorem follows from the following proposition.
\begin{prop}
Isomorphism by an orientation preserving diffeomorphism is obtained by a finite number of the following moves:
\begin{enumerate}[(i)]
\item $(p_i,q_i),(p_j,q_j)\leftrightarrow(p_i,q_i+p_i),(p_j,q_j-p_j)$,
\item $(p_i,q_i)\leftrightarrow(p_i,q_i\pm p_i)$ if $\partial S\ne\varnothing$.
\item $(1,0)\leftrightarrow\varnothing$.
\end{enumerate}
\end{prop}
\begin{pf}
First we prove the moves generate orientation preserving diffeomorphisms.
Let $A\subset M$ be a fibered annulus connecting $M(D^2;(p_i,q_i))$ and $M(D^2;(p_j,q_j))$(or resp.~$\partial M$).
Then, moves (i) (and resp.!(ii)) correspond to the 3-dimensional version of twist along $A$.
The move (iii) inserts or eliminates $M(D^2,(1,0))\approx D^2\times S^1$.

Now we show the converse.
First we can reduce the regular fibers $p_i=1$ with three kinds of moves.

Case (a): no singular fiber.
If $\partial S=\varnothing$, then we have $M(S;(1,e))$ with the same $e$(Exercise 10).
Otherwise $\partial S\ne\varnothing$, then we have $S\times^{(\sim)}S^1$.

Case (b): singular fiber.
We have $p_i\ge2$ for all $i$.
Let $\gamma_1,\cdots,\gamma_n\subset S$ disjoint arcs connecting cone points or $\partial S$, cutting $S$ into a disk $D$.
Note that
\[M(S;(p_1,q_1),\cdots,(p_k,q_k))=\coprod_{i=1}^kM(D^2;(p_i,q_i))\cup\coprod_{j=1}^n\bar{\nu(A_j)}\cup D^2\times S^1,\]
where $A_j:=\gamma_j\times S^1$.
Every isomorphism by an orientation preserving diffeomorphism is determined by mapping classes on $\bar{\nu(A_j)}$ since it sends singular fibers to themselves.
Such a mapping class is some product of the twist along $A_j$, which corresponds the moves (i) or (ii).
\end{pf}

\newpage
\setcounter{section}{7}
\section{Day 8: June 13}
\setcounter{section}{6}
\setcounter{subsection}{1}
\subsection{Classifications}
\setcounter{thm}{3}


Let $O$ and $\tilde O$ be 2-orbifolds with cone points.
A map $p:\tilde O\to O$ is called a \emph{covering} if for each $x\in O$has an open neighborhood $U$ such that $p^{-1}(U)=\coprod_i U_i$, where $p|_{U_i}:U_i\to U$ is diffeomorphically interpreted as $\R^2/(\Z/m\Z)\to\R^2/(\Z/mn\Z)$ for some $m,n\ge1$.

The Euler characteristic is defined by
\[\chi(S(p_1,\cdots,p_k)):=\chi(S)-\sum_{i=1}^k(1-\frac1{p_i})\in\Q.\]
Then we have $\chi(\tilde O)=\deg p\chi(O)$, where $\deg p=\# p^{-1}(x)$ for a non-singular $x\in O$.

\begin{lem}
Every 2-orbifol except $S^2(p)$ and $S^2(p_1,p_2)$ with $p_1\ne p_2$ is finitely covered by some (usual) surface.
\end{lem}

\begin{prop}
Every Seifert manifold $M$ is finitely covered by $S^3$, $S^2\times S^1$, $S^1$-bundles over $\Sigma_g$ ($g\ge1$), or $\Sigma_{g,b}\times S^1$ ($b\ge1$).
\end{prop}
\begin{pf}
Let $M\to S$ be a Seifert fibration.
If $S=S^2(p)$ or $S^2(p_1,p_2)$, then $M$ is a lens space or diffeomorhpic to $S^2\times S^1$.
Otherwise, by Lemma 6.4, $S$ is finitely covered by some surface $S'$.
Pulling back $M$ along $S'\to S$, we have a $S^1$-bundle $M'\to S'$.
If $S'=S^2$, then $M'$ is a lens space or diffeomorphic to $S^2\times S^1$.
Otherwise, $M'$ is a $S^1$-bundle over $\Sigma_g$ with $g\ge1$, or diffeomorphic to $\Sigma_{g,b}\tilde\times S^1$ with $b\ne0$.
\end{pf}
\begin{prop}
Every Seifert manifold finitely covered by $S^2\times S^1$ is diffeomorphic to $S^2\times S^1$ or $\RP^2\tilde\times S^1$.
\end{prop}

\noindent\textbf{Exercise 11.} Prove Proposition 6.6.

\begin{thm}
Every Seifert manifold not diffeomorphic to $S^2\times S^1$ nor $\RP^2\tilde\times S^1$ is irreducible.
\end{thm}
\begin{pf}
By Proposition 6.5 and 6.6, the interior of such a 3-manifold has a universal cover diffeomorphic to $S^3$ or $\R^3$.
Therefore, the theorem follows from Theorem 3.1 and Proposition 3.6.
\end{pf}

A subsurface $\Sigma$ of a Seifert manifold $M$ is called \emph{horizontal} if every fiber transversally intersects $\Sigma$, and \emph{vertical} if it is a union of regular fibers(?).
Then, every horizontal subsurface is a finite cover of the base orbifold, and every vertical subsurface is diffeomorphic either to a torus or an annulus.

\begin{prop}
Every connected essential surface in an irreducible Seifert manifold is isotopic to a horizontal or vertical subsurface.
\end{prop}
\begin{pf}
Let $M\to S$ be a Seifert fibration with an irreducible $M$.
Let $C_1,\cdots,C_k$ be singular fibers.
If there is no, pick one regular fiber as $C_1$.
Let $\gamma_1,\cdots,\gamma_n\subset S$ be distinct arcs connecting cone points or $\partial S$, and cutting $S$ into a disk $D$.
Let $A_j:=\gamma_j\times S^1$ be annuli, then we have
\[M=\coprod_{i=1}^k\bar{\nu(C_i)}\cup\coprod_{j=1}^n\bar{\nu(A_j)}\cup(D\times S^1).\]

Let $\Sigma$ be a connected essential surface and isotope $\Sigma$ so that
\begin{itemize}
\item for each $i$ we have $\Sigma\cap\bar{\nu(C_i)}=\coprod_p(\text{disk})\times\{p\}$, $p\in C_i$,
\item for each $j$ we have $\Sigma\pitchfork A_j$,
\item among these satisfying the above two conditions, $(\sum_{i=1}^k\#\Sigma\cap C_i,\sum_{j=1}^n\#\pi_0(\Sigma\cap A_j))$ is minimal.
\end{itemize}
It suffices to show the following:
\begin{parts}
\item Every component of $\Sigma\cap D\times S^1$ is isotopic to a horizontal disk or a vertical annulus.
\item Every component of $\Sigma\cap A_j$ is a horizontal arc or a vertical loop.
\end{parts}

(a)
We can check that the desired statement holds if and only if every component of $\Sigma\cap D\times S^1$ is incompressible in $D\times S^1$.
Suppose there is a compressing disk $D'\subset D\times S^1$ for $\Sigma\cap D\times S^1$.
Since $\Sigma$ is essential in $M$, $\partial D'$ bounds a disk $D''\subset\Sigma$.
Since $M$ is irreducible, the sphere $D'\cup D''$ bounds some ball in $M$.
We can isotope such a ball and it would reduce $\sum_{i=1}^k\#\Sigma\cap C_i$ or $\sum_{j=1}^n\#\pi_0(\Sigma\cap A_j)$, which is a contradiction, hence every component of $\Sigma\cap D\times S^1$ is incompressible.

(b)
The same argument shows that for each $j$ there is no loop in $\Sigma\cap A_j$ bounding a disk in $A_j$.
If there is an arc $\gamma$ in $\Sigma\cap A_j$ bounding a disk, the since $\Sigma$ is boundary incomporessible, $\partial\gamma$ is adjacent to some $C_i$.
An isotopy pushing the arc toward $C_i$ along such a disk would reduce $\sum_{i=1}^k\#\Sigma\cap C_i$, which is a contradiction, so the conclusion follows.
\end{pf}

\begin{prop}
Let $M$ is an irreducible Seifert manifold.
\begin{parts}
\item Every horizontal subsurface in $M$ is essential.
\item A vertical subsurface in $M$ is essential if its projection in the base orbifold is not the following:
\begin{enumerate}[(i)]
\item a loop bounding a disk with at most one cone point,
\item a boundary parallel curve,
\end{enumerate}
\end{parts}
\end{prop}

\begin{pf}[Sketch]
(a)
Let $\Sigma\subset M$ be a horizontal subsurface.
It is easy to see that $\Sigma$ is not boundary parallel.
Since $\Sigma$ is a finite cover of the base manifold $S$, then we have $\pi_1(\Sigma)\hookrightarrow\pi_1(S)\hookrightarrow\pi_1(M)$, so $\Sigma$ is incompressible in $M$.
The same argument shows that $\Sigma\cup_\partial\Sigma$ in a closed manifold $M\cup_\partial M$ is incompressible, which implies $\Sigma$ is boundary incompressible.

(b)
Let $\Sigma\subset M$ be a vertial subsurface satisgying the conditions.
By the assumption (ii), $\Sigma$ is not boundary parallel.
Components of $M\setminus\nu(\Sigma)$ are Seifert manifolds, suppose there is an essential disk in $M\setminus\nu(\Sigma)$.
By Proposition 6.8, $M\setminus\nu(\Sigma)$ contains a horizontal disk $D$, and $D$ is a finite cover of the base orbifold, which is diffeomorphic to $D^2$ or $D^2(p)$.
\end{pf}

\begin{thm}
Two Seifert manifolds are diffeomorphic if and only if their Seifert fibrations are isomorphic, except the following:
\begin{enumerate}[(i)]
\item $M(D^2;(p,q))\approx D^2\times S^1$.
\item $M(D^2;(2,1),(2,1))\approx N_{1,1}\tilde\times S^1$.
\item $M(S^2;(p_1,q_1),(p_2,q_2))\approx S^3,S^2\times S^1,\text{ or }L(p,q)$.
\item $M(S^2;(p,q),(2,1),(2,-1))\approx M(\RP^2;(q,p))$.
\item $M(S^2;(2,1),(2,1),(2,-1),(2,-1))\approx N_2\tilde\times S^1$.
\end{enumerate}
\end{thm}
\begin{thm}
Two closed Seifert manifold with isomorphic fundamental groups are diffeomorphic or both are lens spaces.
\end{thm}

\begin{thm}[Casson-Junreis, Gabai]
For an irreducible manifold $M$ with infinite $\pi_1(M)$.
Then, $M$ is Seifert if and only if $\pi_1(M)$ has a infinite cyclic normal subgroup.
\end{thm}



\newpage
\setcounter{section}{8}
\section{Day 9: June 20}
\setcounter{section}{7}
\subsection*{7. JSJ decomposition}
Today, we assume that $M$ is an irreducible 3-manifold with $\partial M=\varnothing$ or $\coprod T^2$.
We say $M$ is \emph{atoroidal} if there is no essential torus in $M$.

\begin{thm}[JSJ(Jaco-Shalen, Johanson) decomposition]
There is a (possibly empty) essential surface $S\subset M$ consisting of tori such that every component of $M\setminus\nu(S)$ is atoroidal or a Seifert manifold.
Moreover, such $S$ with minimal $\#\pi_0(S)$ is unique up to isotopy.
\end{thm}

\subsection{Existence}
\begin{prop}
There is a (possibly empty) essential surface $S\subset M$ consisting of tori such that every component of $M\setminus\nu(S)$ is atoroidal.
\end{prop}
\begin{pf}
If $S'\subset M$ is an essential surface consisting of tori such that no component of $M\setminus\nu(S)$ is a trivial $[0,1]$-bundle, then by Proposition 3.7, $S'$ is isotopic to a normal surface with respect to some triangulation of $M$, and by Proposition 3.8, $\#\pi_0(S')$ is bounded.

Take $S\subset M$ to be such a surface $S'$ conditioned as above with maximal $\#\pi_0(S)$.
Suppose that there is a component of $M\setminus\nu(S)$ contains an essential torus $T$.
Then, no component of $M\setminus\nu(S\cup T)$ is a trivial $[0,1]$-bundle and $\#(S\cup T)=\#\pi_0(S)+1$, which is a contradiction.
\end{pf}

\subsection{Uniqueness}

\begin{lem}
Every atoroidal manifold $M$ containing an essential annulus is a Seifert manifold.
\end{lem}
\begin{lem}
Every annulus in a Seifert manifold $M$ is isotopic to a vertical subsurface, if neccessay after we change Seifert fibrations.
\end{lem}
\begin{pf}
Take $S\subset M$ to be such a surface $S'$ condition as in the proof of existence, with maximal $\#\pi_0(S)$.
Suppose that there is a component of $M\setminus\nu(S)$ contains an essential torus $T$.
Then, no component of $M\setminus\nu(S\cap T)$ is a trivial $[0,1]$-bundle and $\#\pi_0(S\cup T)=\#\pi_0(S)+1$.
It is a contradiction.
\end{pf}

\begin{pf}[Proof of uniqueness]
Let $S,S'\subset M$ be \emph{such} essential surfaces consisting of tori.
Isotope $S$ and $S'$ so that they are transversal and $\#(S\cap S')$ is minimal.
In the following we show:
\begin{parts}
\item $S\cap S'=\varnothing$,
\item there is a pair of parallel components in $S$ and $S'$.
\end{parts}
If so, cutting $M$ along one of such parallel tori, we have the case with less $\#\pi_0(S)$.
Therefore, the induction on $\#\pi_0(S)$ shows the uniqueness.

(a)
Suppose that $S\cap S'\ne\varnothing$.
Since $M$ is irreducible, and since $S$ and $S'$ are essential, the standard argument shows that every component of $S\cap S'$ is a loop cutting torus components into annuli.
Pick a torus component $T\subset S$ such that $T\cap S'\ne\varnothing$.
Let $N$ be the union of one or two components of $M\setminus\nu(S)$ attaching $\bar{\nu(T)}$.
Every annulus $A\subset N$ which is a part of $S'$ is essential.
In particular, $A$ is not boundary parallel.
Otherwise, an isotopy would reduce $\#\pi_0(S\cap S')$.
By Lemma 7.3, $N$ is a Seifert manifold.
By Lemma 7.4, we may assume that every annulus is a vertical subsurface.
The fibration of $A$ induces one of $T$, which agrees with that of $N$.
Therefore, $N\cup\bar{\nu(T)}$ becomes a bigger Seifert manifold, so $S\setminus T$ would satisfy the condition.

(b)
Suppose that no pair of torus components in $S$ and $S'$ is parallel.
Then, every torus in $S$ (resp.~$S'$) is contained in a Seifert component of $M\setminus\nu(S')$ (resp.~$M\setminus\nu(S)$), and it is a vertical subsurface by Proposition 6.8.
Every atoroidal component of $M\setminus\nu(S)$ (resp.~$M\setminus\nu(S')$) must be contained in a Seifert component of $M\setminus\nu(S')$ (resp.~$M\setminus(S)$), and it is also a Seifert manifold.
Then, there are components $N\subset M\setminus\nu(S)$ and $N'\subset M\setminus\nu(S')$ such that $N\cap N'\ne\varnothing$ and any Seifert fibrations of $N$ and $N'$ are different in $N\cap N'$, since all Seifert fibrations would agree and $M$ would be a Seifert manifold otherwise.
By Theorem 6.2 and 6.10, $N\cap N'\approx M(D^2;(2,1),(2,1))\approx N_{1,1}\tilde\times S^1$.
Since $\partial(N\cap N')$ is connected, we have $N=N\cap N'\subset N'$ or $N'=N\cap N'\subset N$, and this implies that the Seifert fibration of the small one could be replaced by that of the bigger one, a contradiction.
\end{pf}

\begin{pf}[Proof of Lemma 7.3]
Let $M$ be atoroidal and $A\subset M$ be an essential annulus.
We have the following cases:
\begin{parts}
\item $\partial A$ lies in two tori $T_1,T_2\subset\partial M$.
\item $\partial A$ lies in one torus $T_0\subset\partial M$:
(b1) $A$ is twisted, (b2) $A$ is untwisted.
\end{parts}
If we let $V:=\nu(A\cup T_1\cup T_2)$ or $\nu(A\cup T_0)$, then $V\approx\Sigma_{0,3}\times S^1$ for (a) and (b1), and $V\approx N_{1,2}\tilde\times S^1$ for (b2).
Since $M$ is atoroidal, $\partial V\setminus\partial M$ is compressible or boundary parallel.
Such a compressible torus bounds $D^2\times S^1$(Exercise 6).
Then, $M\approx V$ or $V\cup(\coprod D^2\times S^1)$.
Since such $D^2\times S^1$ is attached so that $\partial D\times\{*\}$ is not isotopic to a fiber of $V$ because $A$ is incompressible, we have $M$ a Seifert manifold.
\end{pf}

\begin{pf}[Sketch of Lemma 7.4]
Let $M$ be a Seifert manifold and $A\subset M$ be an essential annulus.
By Proposition 6.8, $A$ is isotopic to a horizontal or vertical subsurface.
Suppose that $A$ is a horizontal annulus.
Then, $A$ is a finite cover of the base manifold $O$.
Since $\chi(O)=\chi(A)=0$, $O\approx\Sigma_{0,2},\ N_{1,1}$, or $D^2(2,2)$, so $M\approx\Sigma_{0,2}\times S^1,\ N_{1,1}\tilde\times S^1$, or $M(D^2;(2,1),(2,1))$.
Since we have $\Sigma_{0,2}\times S^1\approx[0,1]\times S^1\times S^1$ and $N_{1,1}\tilde\times S^1\approx M(D^2;(2,1),(2,1))$ with different Seifert fibrations, $A$ becomes a vertical subsurface if we swap Seifert fibrations of $M$.
\end{pf}


\newpage
\setcounter{section}{9}
\section{Day 10: June 27}
\setcounter{section}{8}

\subsection*{8. Geometrization conjecture}

\subsection{Hyperbolic manifolds}
Suppose $M$ is a three-manifold with $\partial M=\varnothing$ or $\coprod T^2$.
We say $M$ is \emph{hyperbolic} if the interior $\Int M$ admits a complete finite-volume Riemannian metric such that it is locally isometric to $\H^3$, or equivalently, there is a discrete subgroup $\Gamma$ of $\PSL(2,\C)$ such that $\Int M\approx\H^3/\Gamma$ with finite volume.

\begin{thm}[Thurston]
For the figure-eight knot, the Dehn surgery is hyperbolic if and only if $r\ne0,\pm1,\pm2,\pm3,\pm4$.
\end{thm}

Alexander's theorem(Thm 3.1) and Prop 3.6 imply:
\begin{prop}
Every hyperbolic manifold is irreducible.
\end{prop}

\begin{prop}
If $M$ is hyperbolic such that $\pi_1(M)$ has a subgroup $\pi'$ isomorphic to $\Z^2$, then there are a torus component $T\subset\partial M$ and a loop $\gamma\in\pi_1(M)$ such that $\gamma\pi'\gamma^{-1}$ is a subgroup of the image of $\pi_1(T)\to\pi_1(M)$.
\end{prop}
Note that if $\Gamma<\PSL(2,\C)$ is isomorphic to $\Z^2$, then $\Gamma$ is conjugate to $\mat{\pm1&\Z+\omega\Z\\0&\pm1}$ for some $\omega\in\C\setminus\R$, and $\H^3/\Gamma\approx\C/(\Z+\omega\Z)\times(0,\infty)\approx T^2\times\R$.

\begin{cor}
Every hyperbolic manifold is atoroidal.
\end{cor}

\begin{thm}[Rigidity, Mostow-Prasad]
Let $M,N$ be hyperbolic three-manifolds.
Then, every isomorphism $\pi_1(M)\to\pi_1(N)$ is induced by a unique isometry $M\to N$.
\end{thm}
The rigidity implies that every geometric invariant(e.g.~volume) is a topological invariant.

\begin{prop}
The fundamental group of every hyperbolic three-manifold is linear over $\C$.
\end{prop}
\begin{pf}
Consider $\pi_1(M)$ as a subgroup of $\PSL(2,\C)$, whose adjoint action gives rise to an embedding $\PSL(2,\C)\hookrightarrow\SL(3,\C)$.
\end{pf}
Mal'cev's theorem(Theorem 1.9) implies:
\begin{cor}
The fundamental group of every hyperbolic three-manifold is residually finite.
\end{cor}


\subsection{Geometrization conjecture}
\begin{thm}[Geometrization conjecture, Perelman 02,03]
Each component of the JSJ decomposition is either hyperbolic or Seifert.
\end{thm}

Ricci flow with surgery:
\[\frac{\partial g(t)}{\partial t}=-2\mathrm{Ric}(g(t)).\]
If the conjecture holds for sufficiently large $t$, then it holds for every $t$.
For large $t$, we can do the thick-thin decomposition of $M$.
The thick part, which always exists, has a hyperbolic structure, and the boundary of the decomposition is essential.
If there is a thin part, then $M$ is Haken, for which the conjecture is true from the result by Thurston in 1982.

Let $X$ be a simply connected homogeneous Riemannian manifold.
We say $M$ is said to have a \emph{geometric structure} modeled on $X$ if $\Int M$ admits a complete finite-volume Riemannian metric such that it is locally isometric to $X$.

$S^3,\R^3,\H^3,S^2\times\R,\H^2\times\R$.
\[Nil:=\{\mat{1&x&y\\0&1&z\\0&0&1}:x,y,z\in\R\},\quad ds^2=dx^2+dy^2+(dz-xdy)^2.\]
\[Sol:=\R^2\rtimes\R\text{ with }\R\curvearrowleft\R^2\ \mat{e^z&0\\0&e^{-z}}\text{ for }z\in\R,\quad ds^2=e^{2z}dx^2+e^{-2z}dy^2+dz^2.\]
\[\tilde{\SL(2,\R)}=\tilde{UT\H^2}=\tilde{\PSL(2,\R)}\text{ as Riemannian manifolds.}\]


\begin{thm}[Geometrization conjecture]
There is an (possibly empty) essential surface $S\subset M$ consisting of tori such that each component of $M\setminus\nu(S)$ has a geometric structure modeled on one of the eight model geometries.
Moreover, such $S$ with minimal $\#\pi_0(S)$ is unique up to isotopy.
\end{thm}
\begin{rmk*}
There is a $Sol$-manifold admitting a nontrivial JSJ decomposition.
\end{rmk*}
\begin{prop}
If $M$ has a geometric structure modeled on except $\H^3$ and $Sol$, then $M$ is Seifert.
\end{prop}
\begin{prop}
Every $Sol$-manifold $M$ is finitely covered by a $T^2$-bundle over $S^1$ with Anosov monodromy $\mat{\lambda&0\\0&\lambda^{-1}}$ for $\lambda\in\R\setminus\{\pm1\}$.
\end{prop}

\subsection{Some results}
\begin{thm}[Elliptization conjecture]
Every closed three-manifold with finite $\pi_1$ is spherical.
\end{thm}
\begin{pf}
It follows from the geometrization conjecture and the analysis of Seifert manifolds.
\end{pf}
\begin{thm}[Poincar\'e conjecture]
Every simply connected closed 3-manifold is diffeomorphic to $S^3$.
\end{thm}
\begin{thm}[Hyperbolization conjecture]
Every closed irreducible atoroidal three-manifold with infinite $\pi_1$ is hyperbolic.
\end{thm}
\begin{thm}
Two closed prime three-manifolds with isomorphic $\pi_1$ are diffeomorphic, or both are lens spaces.
\end{thm}
\begin{pf}
Let $M,M'$ be such three-manifolds.
By the geometrization conjecture, it suffices to consider the following three cases:
\begin{parts}
\item $M$ and $M'$ are hyperbolic.
\item $M$ or $M'$ is Seifert.
\item $M$ or $M'$ contains an essential torus(it has a non-trivial JSJ decomposition).
\end{parts}

For the case (a), the theorem follows from Mostow's rigidity theorem.

For the case (b), if $\pi_1$ is finite, then $M$ and $M'$ are irreducible atoroidal so that they are Seifert, and if $\pi_1$ is infinite, then it has a normal infinite cyclic subgroup so that they are Seifert.
Then, the theorem follows from Theorem 6.11.

For the case (c), we may assume $M$ and $M'$ are not Seifert by (b).
Since it has $\Z^2$ as a subgroup, by Proposition 8.3, $M$ and $M'$ are not hyperbolic.
By the geometrization conjecture, $M$ and $M'$ are Haken.
Then, the theorem follows from Waldhausen's theorem(Theorem 5.10).
\end{pf}

\begin{thm}
The fundamental group of every three-manifold is residually finite.
\end{thm}
The fact that surface groups are residually finite, and we have its generalization to Seifert manifolds, and Corollary 8.7 with the geometrization conjecture imply the theorem.


Next lecture: Virtually Haken conjecture.
\end{document}