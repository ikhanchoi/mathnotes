\documentclass{../crs}
\usepackage{../../ikany}

\title{Analysis 3 : Measure Theory}

\begin{document}
\maketitle
\tableofcontents

\chapter{Carath\'eodory's theory}

\section{$\sigma$-algebras}
% \texorpdfstring{$\sigma$}{Lg}-algebras
\begin{defn}
Let $X$ be a set.
A \emph{ring of sets} is a family of subsets of $X$ that is closed under finite union and finite relative complement; in other words, $\cR\subset\cP(X)$ is called a ring of sets if the following two conditions are satisfied:
\begin{cond}
\item if $A,B\in\cR$, then $A\cup B\in\cR$,
\item if $A,B\in\cR$, then $A\setminus B\in\cR$.
\end{cond}
\end{defn}

\begin{prop}
Let $X$ be a set and $\cR\in\cP(X)$.
Then, the followings are equivalent:
\begin{cond}
\item $\cR$ is a ring of sets,
\item $\cR$ is closed under symmetric difference and finite intersection,
\item $\cR$ is a ring,
\item $\cR$ is a Boolean ring.
\end{cond}
For the ring structure, we take the symmetric difference as addition and the intersection as multiplication.
\end{prop}

\begin{prop}
A ring of sets is a distributive lattice.
\end{prop}


If a ring of sets contains a multiplicative identity, the entire set, then we call the ring of sets as follows:
\begin{defn}
An \emph{algebra of sets} is a ring of sets with the entire set.
\end{defn}
\begin{prop}
Let $X$ be a set and $\cR\in\cP(X)$.
Then, the followings are equivalent:
\begin{cond}
\item $\cR$ is an algebra of sets,
\item $\cR$ is closed under finite union, finite intersection, and complement,
\item $\cR$ is a Boolean algebra.
\end{cond}
\end{prop}
An algebra of sets is sometimes called a field of sets.



\section{Carath\'eodory's extension theorem}
\begin{thm}[Carath\'eodory's extension theorem]
Let $\cR$ be a ring of sets over $X$.
Let $\sigma(\cR)$ be the $\sigma$-algebra generated by $\cR$.
A set function $\mu:\cR\to[0,\oo]$ is extended to a measure on $\sigma(\cR)$ if and only if it is a premeasure.
\end{thm}



\chapter{}





\chapter{Topological measures}

\section{Descriptive set theory}





\section{Borel measures}





\chapter{Hmmmm}

\subsection{Convergence in measure}
Since $\{f_n(x)\}_n$ diverges if and only if
\[\exists k>0,\quad\forall n_0>0.\quad\exists n>n_0:\quad |f_n(x)-f(x)|>n^{-1},\]
we have
\begin{align*}
\{x:\{f_n(x)\}_n\text{ diverges}\}&=\bigcup_{k>0}\bigcap_{n_0>0}\bigcup_{n>n_0}\{x:|f_n(x)-f(x)|>n^{-1}\}\\
&=\bigcup_{k>0}\limsup_n\{x:|f_n(x)-f(x)|>n^{-1}\}.
\end{align*}
Since for every $k$
\[\limsup_n\{x:|f_n(x)-f(x)|>k^{-1}\}\subset\limsup_n\{x:|f_n(x)-f(x)|>n^{-1}\},\]
we have
\[\{x:\{f_n(x)\}_n\text{ diverges}\}\subset\limsup_n\{x:|f_n(x)-f(x)|>n^{-1}\}.\]




\begin{thm}
Let $f_n$ be a sequence of measurable functions on a measure space $(X,\mu)$.
If $f_n$ converges to $f$ in measure, then $f_n$ has a subsequence that converges to $f$ $\mu$-a.e.
\end{thm}
\begin{pf}
Since $d_{f_n-f}(1/k)\to0$ as $n\to\infty$, we can extract a subsequence $f_{n_k}$ such that
\[\mu(\{x:|f_{n_k}(x)-f(x)|>k^{-1}\})>2^{-k}.\]
Since
\[\sum_{k=1}^\infty\mu(\{x:|f_{n_k}(x)-f(x)|>k^{-1}\})<\infty,\]
by the Borel-Canteli lemma, we get
\[\mu(\limsup_k\{x:|f_{n_k}(x)-f(x)|>k^{-1}\})=0.\]
Therefore, $f_{n_k}$ converges $\mu$-a.e.
\end{pf}




\end{document}