\documentclass{../../large}
\usepackage{../../ikhanchoi}


\begin{document}
\title{Algebraic Structures}
\author{Ikhan Choi}
\maketitle
\tableofcontents

\part{Groups}
% Homomorphism의 construction에만
% group의 활용에 집중, group의 구조에는 관심을 끈다
% orderwise, lagrange, orbit-stabilizer 외에 카운팅은 하지 않는다
% stabilizer 계산은 해야지
% 부분군 래티스 그리는 정도까지는 한다
\chapter{Groups}
\section{Definition of groups}

\begin{prb}[Binary operation]
Let $A$ be a set.
A \emph{binary operation} on $A$ is a function $\cdot:A\times A\to A$.
A binary operation on $A$ is called to satisfy
\begin{enumerate}[(i)]
\item the \emph{associativity} if for every $a,b,c\in A$ we have
\[(a\cdot b)\cdot c=a\cdot(b\cdot c),\]
\item the \emph{existence of identity} if there exists $e\in A$ such that for every $a\in A$ we have
\[a\cdot e=e\cdot a=a,\]
\item the \emph{existence of inverses} if satisfies (ii) and for every $a\in A$ there is $x\in A$ such that
\[a\cdot x=x\cdot a=e,\]
\item the \emph{commutativity} if for every $a,b\in A$ we have
\[a\cdot b= b\cdot a.\]
\end{enumerate}
A \emph{monoid}, \emph{group}, and \emph{abelian group} is an ordered pair $(A,\cdot)$ of a set $A$ and a binary operation $\cdot:A\times A\to A$ satisfying the first two, three, and four of the above conditions, respectively.
An accompanying binary operation is called a \emph{group structure} if it defines a group, that is, it satisfies (i), (ii), and (iii).
\begin{parts}
\item $(\N,+)$ is not a monoid, and $(\N,\times)$ is a monoid.
\item $(\Z,+)$ is a group, and $(\Z,\times)$ is a monoid.
\item $(\Q,+)$ is a group, and $(\Q\setminus\{0\},\times)$ is also a group.
\item The set of all invertible $2\times2$ real matrices forms a group with multiplication, which is not abelian.
\end{parts}
\end{prb}

\begin{prb}[Properties of a group structure]
We say a group is \emph{additive} if we use the symbol $+$ for the group structure, and \emph{multiplicative} if we use the symbol $\cdot$ or omit the symbol for the group structure.
\begin{parts}
\item For $g_1,\cdots,g_n\in G$, the value of $g_1\cdots g_n$ is well-defined independently of how the expression is bracketed.
\item The identity of $G$ and the inverses of each element $g\in G$ are unique.
\item $(g^{-1})^{-1}$ and $(gh)^{-1}=h^{-1}g^{-1}$ for all $g,h\in G$.
\item The left and right ancellation laws.
\end{parts}
\end{prb}

\begin{prb}[Group table]
\end{prb}

\section{Homomorphisms}
homomorphisms, image, kernel, preimage
isomorphism

\section{Subgroups}

\begin{prb}[Subgroups]
\end{prb}
\begin{prb}[Lagrange theorem]
cosets, index
\end{prb}
\begin{prb}[Subgroup lattice]
\end{prb}

generators


\section{Quotient groups}
\begin{prb}[Normal subgroups]
\end{prb}
\begin{prb}[Isomorphism theorems]
\end{prb}



\section*{Exercises}
\begin{prb}[Direct sum and direct product]
\end{prb}

\begin{prb}[Automorphism groups]
\end{prb}







\chapter{Examples of groups}



\section{Cyclic groups}
\begin{prb}[Orders]
\end{prb}

cyclic groups

\section{Dihedral and Dicyclic groups}
\begin{prb}[Dihedral groups]
\end{prb}
\begin{prb}[Dicyclic groups]
\end{prb}
\begin{prb}[Quoternion group]
\end{prb}

\section{Symmetric and alternating groups}

sign homomorphism
generators, transpositions
cycle type

\section{Matrix groups}
general, special







\chapter{Group actions}

\section{Representations}

Let $G$ be a group and $X$ be a set.
A \emph{left action} of $G$ on $X$ is a function $G\times X\to X:(g,x)\to gx$ such that $g(hx)=(gh)x$ and $ex=x$.
A \emph{left $G$-set} is a set $X$ together with a left action of $G$ on $X$.
We may define right actions and right $G$-sets similarly.

effective, free, transitive actions.
The orbit spaces of a left $G$-set $X$ is a set $G\backslash$ of orbits.
When we do not have to emphasize the $G$-space is left, that is we do not deal with both left and right actions simultaneously, we often write the orbit space just by $X/G$.

Let $H$ be a subgroup of $G$.
A left coset is an element of the orbit space of the right action $G\times H\to G$ of $H$ on $G$ given by the right multiplication.
Here we can define a left multiplication action of $G$ on $G/H$, which is transitive.

\section{Orbits and stabilizers}
Invariants on orbit space.

\begin{prb}[Orbit-stabilizer theorem]
The size of orbits.
The number of orbits.
The class equation.
\end{prb}


\begin{prb}[Transitive actions]
\begin{parts}
\item Stabilizers are all isomorphic.
\end{parts}
\end{prb}

\begin{prb}[Free actions]
no fixed point,
trivial stabilizer for any point,
every orbit has 1-1 correspondence to group
\end{prb}

\section{Action by left multiplication}

\section{Action by conjugation}
\begin{prb}[Centralizers and normalizers]
\end{prb}

\begin{prb}[Conjugacy classes of elements]
\end{prb}

\begin{prb}[Conjugacy classes of subgroups]
\end{prb}

H has index n  : G can act on Sym(G/H) : left mul
K normalizes H : K -> NG(H) -> NG(H)/H  with ker = KnH
K normalizes H : K -> NG(H) -> Aut(H)  with ker = CG(H)







\section*{Exercises}

\section*{Problems}

\begin{enumerate}
\item Show that a group of order $2p$ for a prime $p$ has exactly two isomorphic types.
\item Let $G$ be a finite group of order $n$ and $p$ the smallest prime divisor of $n$. Show that a subgroup of $G$ of index $p$ is normal in $G$.
\item Show that a finite group $G$ satisfying $\sum_{g\in G}\ord(g)\le2n$ is abelian.
\item Find all homomorphic images of $A_4$ up to isomorphism.
\item For a prime $p$, find the number of subgroups of $Z_{p^2}\times Z_{p^3}$ of order $p^2$.
\item Let $G$ be a finite group. If $G/Z(G)$ is cylic, then $G$ is abelian.
\item Let $G$ be a finite group. If the cube map $x\mapsto x^3$ is a surjective endomorhpism, then $G$ is abelian.
\item Show that if $|G|=p^2$ for a prime $p$, then a group $G$ is abelian.
\item Show that the order of a group with only on automorphism is at most two.
\end{enumerate}










\part{Rings}
\chapter{Ideals}
\section{Definitions of rings and ideals}
\begin{prb}[Definition of rings]
A \emph{ring} is an abelian group $R=(R,+)$ together with a \emph{multiplication} $\times:R\times R\to R$ which satisfies the associativity law, such that the following compatibility condition holds: the \emph{distributive laws}:
\[r\times(s+t)=(r\times s)+(r\times t),\qquad(s+t)\times r=(s\times r)+(t\times r),\qquad r,s,t\in R.\]
We usually omit the cross symbol to write $r\times s$ as $rs$.

We are usually concerned with \emph{commutative unital} rings, that is, rings whose multiplication is commutative and admits a multiplicative identity.
The additive and multiplicative identities are usually denoted by $0$ and $1$ and called the \emph{zero} and the \emph{unity} respectively.
\end{prb}


\begin{prb}[Definition of ideals]
Let $R$ be a commutative unital ring.
\end{prb}

\begin{prb}[Quotient rings]
\end{prb}
\begin{prb}[Isomorphism theorems]
\end{prb}


\section{Maximal and prime ideals}
fields and integral domains
existence by Zorn's lemma

\section{Operations on ideals}

\section*{Exercises}
size of units, the number of ideals








\chapter{Integral domains}
\section{Unique factorization domains}
\section{Principal ideal domains}

\begin{prb}
In PID $R$,
\begin{parts}
\item every irreducible element is prime, \hfill(Euclid's lemma)
\item every two elements has greatest common divisor, \hfill(existence of gcd)
\item the gcd is given as a $R$-linear combination, \hfill(B\'zout's identity)
\item factorization into primes is unique up to permutation, \hfill(UFD)
\item every prime ideal is maximal. \hfill(Krull dimension 1)
\end{parts}
\end{prb}


\section{Noetherian rings}

\section*{Exercises}
\section*{Problems}
\begin{enumerate}
\item Show that a finite integral domain is a field.
\item Show that every ring of order $p^2$ for a prime $p$ is commutative.
\item Show that a semiring with multiplicative identity and cancellative addtion has commutative addition.
\item Show that the complement of a saturated monoid in a commutative ring is a union of prime ideals.
\end{enumerate}


\section*{Exercises}
\begin{prb}[Primitive roots]
We find all $n$ such that $(\Z/n\Z)^\times$ is cyclic.
\end{prb}





\chapter{Polynomial rings}
\section{Irreducible polynomials}
relation to maximal ideals
Irreducibles over several fields
\begin{prb}[Gauss lemma]
\end{prb}
\begin{prb}[Eisenstein criterion]
\end{prb}

\section{Polynomial rings over a field}
\begin{prb}[Euclidean algorithm for polynoimals]
\end{prb}
\begin{prb}[Polynomial rings over UFD]
\end{prb}
\begin{prb}[Hilbert's basis theorem]
\end{prb}

maximal ideals and monic irreducibles






\part{Modules}



\chapter{Modules}
\section{Modules}

\begin{prb}[Definition of modules]
Let $R$ be a (possibly non-commutative and possibly non-unital) ring.
A \emph{left $R$-module} is an abelian group $(M,+)$ together with a ring homomorphism $\alpha:R\to\End_\Z(M)$, where $\End_\Z(M)$ denotes the group endomorphisms on $M$.
We assume convectionally that $\alpha$ is unital if $R$ is unital.
The homomorphism $\alpha$ is called the \emph{left action} and the operation $\cdot:R\times M\to M$ defined by $r\cdot m:=\alpha(r)(m)$ is called the \emph{scalar multiplication}.
We usually omit the dot to denote it by $rm$.
\begin{parts}
\item If $R$ is commutative, then
\end{parts}
\end{prb}

submodules
quotient modules
isomorphism theorems


\section{Algebras}
\begin{prb}[Definition of algebras]
Let $R$ be a commutative unital ring.
An \emph{associative $R$-algebra} is a ring $A$ together with a unital ring homomoprhism $\alpha:R\to Z(\tilde A)\subset\End_\Z(A)$.
Although there are some important examples of \emph{non-associative} algebras in which the associativity of multiplication is dropped, in most cases we will assume that an $R$-algebra is associative.
\begin{parts}
\item The set of matrices $M_n(R)$ over a ring $R$ is a unital $R$-algebra.
\item The set of quaternions $\H$ is an $\R$-algebra.
\end{parts}
\end{prb}




\section{Free modules}
generators, cyclic
direct sum
free modules

\section{Tensor products}





\chapter{Exact sequences}
\section{}
injective modules
projective modules
flat modules
endomorphism algebra
Tor and Ext






\chapter{Modules over principal ideal domains}
\section{Structure theorem of finitely generated modules}

\begin{prb}[Torsion modules]
Let $R$ be a commutative unital ring.
An $R$-module $M$ is called a \emph{torsion} module if
for every element $m\in M$ there is $r\in R$ such that $rm=0$.
\end{prb}

\begin{prb}[Cyclic modules]
Let $R$ be a commutative unital ring.
An $R$-module $M$ is said to be \emph{cyclic} if it is generated by one element.
\begin{parts}
\item A cyclic $R$-module is isomorphic to a quotient of $R$.
\item A cyclic $R$-module is torsion-free if and only if it is isomorphic to $R$.
\end{parts}
\end{prb}

\begin{prb}
Let $R$ be a principal ideal domain.
A submodule of a finite-rank free module is also a finite-rank free module.
Two ways to take the basis imply the existence of invariant factors and elementary divisors.
\end{prb}

\begin{prb}[Structure theorem of finitely generated modules]
Let $R$ be a principal ideal domain and $M$ be a finitely generated $R$-module.
If we know the ideal structure of a PID $R$, then we can classify all finitely generated modules over $R$.
\begin{parts}
\item $M$ is isomorphic to the direct sum of cyclic $R$-modules.
\item existence and uniqueness: invariant factors
\item existence and uniqueness: elementary divisors
\end{parts}
\end{prb}

\[
(\Z/2\Z)\oplus(\Z/4\Z)\oplus(\Z/12\Z)\oplus(\Z/48\Z)
\Leftrightarrow
\begin{tabular}{c|cccc}
& 2 & 4 & 12 & 48 \\\hline
2 & 2 & 4 & 4 & 16 \\
3 & 0 & 0 & 3 & 3
\end{tabular}
\]
\[
(\Z/2\Z)\oplus(\Z/2^2\Z)^2\oplus(\Z/2^4\Z)\oplus(\Z/3\Z)^2
\Leftrightarrow
\begin{tabular}{c|cccc}
$p\setminus e$ & 1 & 2 & 3 & 4 \\\hline
2 & 1 & 2 & 0 & 1 \\
3 & 2 & 0 & 0 & 0
\end{tabular}
\]




\part{Vector spaces}


\chapter{Duality}
\section{Linear functionals}

\begin{prb}[Double dual space]
\end{prb}


\section{Bilinear and sesquilinear forms}

\begin{prb}[Polarization identity]
\begin{parts}
\item Let $F$ be a field of characteristic not $2$. If $\<-,-\>$ is a symmetric bilinear form, then
\[\<x,y\>=\frac12(\|x+y\|^2-\|x\|^2-\|y\|^2).\]
\item Let $F=\C$. If $\<-,-\>$ is a sesquilinear form, then
\[\<x,y\>=\frac14\sum_{k=0}^3i^k\|x+i^ky\|^2.\]
\item isometry check
\end{parts}
\end{prb}

\begin{prb}[Cauchy-Schwarz inequality]
\begin{parts}
\item Let $F=\R$. If $\<-,-\>$ is a positive semi-definite symmetric bilinear form, then
\item Let $F=\C$. If $\<-,-\>$ is a positive semi-definite Hermitian form, then
\end{parts}
\end{prb}

\begin{prb}[Dual space identification]
Let $\<-,-\>$ be a non-degenerate bilinear form
\end{prb}


\section{Adjoint}
\begin{prb}[Adjoint linear transforms]
\end{prb}








\chapter{Normal forms}
\section{Rational canonical form}
\begin{prb}[Rational canonical form]
Let $F$ be a field.
Invariant factor form
\begin{parts}
\item There is a one-to-one correspondence between the similarity classes of square matrices over $F$ and the isomorphism classes of finitely generated $F[x]$-modules.
\item Every finitely generated $F[x]$-module is a direct sum of cylic torsion $F[x]$-modules, i.e.~no free submodules.
\item Every cyclic torsion $F[x]$-module $V\cong R/(a)$ can be represented by the associated companion matrix $C_a$, constructed by the coefficients of $a$.
\end{parts}
\end{prb}

For $A\in M_n(F)$, the minimal polynomial $m_A(x)$ can be defined by the generator of the annihilator of the associated $F[x]$-module $(V,A)$.
The minimal polynomial is the largest invariant factor of $(V,A)$.
For each invariant factor $a_i$, we can construct a companion matrix with its coefficients.



\section{Jordan normal form}


\section{Conjugation action}

\begin{prb}[Commuting matrices]
\end{prb}

\section{Spectral theorems}

\section*{Exercises}

\begin{prb}[Conjugacy classes of $\GL_2(\F_p)$]
The conjugacy classes are classified by normal forms.
There are four cases: for some $a$ and $b$ in $\F_p$,
\begin{parts}
\item $\mat{a&0\\0&b}$: $\binom{p-1}2$ classes of size $\frac{|G|}{(p-1)^2}=p(p+1)$.
\item $\mat{a&0\\0&a}$: $p-1$ classes of size $1$.
\item $\mat{a&1\\0&a}$: $p-1$ classes of size $\frac{|G|}{p(p-1)}=p^2-1$.
\item otherwise, the eigenvalues are in $\F_{p^2}\setminus\F_p$.
In this case, the number of conjugacy classes is same as the number of monic irreducible qudratic polynomials over $\F_p$; $\frac{|\F_{p^2}|-|\F_p|}2=\frac{p(p-1)}2$ classes.
Their size is $\frac{p(p-1)}2$.
\end{parts}
\end{prb}

\begin{prb}[Conjugacy classes of $\GL_3(\F_p)$]
There are eight types of invariant factors:
\[(x-a)(x-b)(x-c),\ (x-a)^2(x-b),\ (x-a)^3,\ (x^2+ax+b)(x-c),\ (x^3+ax^2+bx+c),\]
\[(x-a)\mid(x-a)(x-b),\ (x-a)\mid(x-a)^2,\ (x-a)\mid(x-a)\mid(x-a)\]
\end{prb}

Show that a square matrix $A$ over $\F_p$ satisfying $A^p=A$ is diagonalizable.



\chapter{Tensor algebras}
\section{Graded and filtered algebras}
\section{Exterior algebras}
\begin{prb}[Determinants]
\end{prb}
\section{Symmetric algebras}





\end{document}