\documentclass{../../large}
\usepackage{../../ikhanchoi}

\begin{document}
\title{Mathematical Logic}
\author{Ikhan Choi}
\maketitle
\tableofcontents


\part{Set theory}
\chapter{First-order logic}
\section{First-order logic}
\begin{prb}[Meta-mathematics]
Most of modern usual mathematics is based on the informal $\mathrm{ZC}^-$, the Zermelo-Fraenkel set theory with choice but removed the axiom of regularity and the axiom schema of replacement.
This informal theory is undefinable in a mathematical rigorous sense but appreciated as the bottom assumption, believing in our innately shared and experimentally established ability of reasoning.
It can be regarded as a kind of statement of faith for the assumption on which every informal finite reasoning in usual mathematics relies.
Now then, we can study formal theories on logical systems and $\mathrm{ZFC}$ itself as an interesting example of first-order formal theories via the informal $\mathrm{ZC^-}$; we can prove (meta-)theorems on the formal $\mathrm{ZFC}$.
\end{prb}

\begin{prb}[Languages]
In general, a \emph{language} is a subset $\cL$ of the Kleene star $\Sigma^*$ of a set $\Sigma$ called an \emph{alphabet}.
Recall that the \emph{Kleene star} $\Sigma^*$ of a given set $\Sigma$ is another name of the free monoid generated by the set.
An element of a given alphabet is called a \emph{symbol} or a \emph{letter}, and an element of a language is called a \emph{formula}.
A language is usually given by an alphabet together with a finite set of formula-formation rules, called a grammar.
To stress that each formula satisfies the formation rule, a formula of a language given by a grammar is also called a \emph{well-formed formula}.
\end{prb}

\begin{prb}[Signatures]
A \emph{signature}, also called a \emph{lexicon} or a \emph{vocabulary}, is a datum consisting of
\begin{enumerate}[(i)]
\item a non-empty countable set $\cS$ of \emph{sorts},
\item a non-empty set of symbols partitioned into the following two disjoint sets:
\begin{enumerate}[(\text{ii}.i)]
\item a set $\cF$ of \emph{function symbols} together with functions $\mathrm{dom}:\cF\to\cS^*$ and $\mathrm{cod}:\cF\to\cS$ called the \emph{domain function}, and \emph{codomain function},
\item a set $\cP$ of \emph{predicate} or \emph{relation symbols} together with a function $\mathrm{ar}:\cP\to\cS^*$ called the \emph{arity function}.
\end{enumerate}
\end{enumerate}
A function symbol $f\in\cF$ is called a \emph{constant symbol} if $\dom(f)=0$, and a predicate symbol $p\in\cP$ is called a \emph{truth value} if $\mathrm{ar}(p)=0$.
If $\cS$ has only one or two elements, then the signature is called \emph{single-sorted} or \emph{double-sorted} respectively.
For instance, the signatures for groups and rings are single-sorted, and the signatures for modules and vector spaces are double-sorted.

\end{prb}


\begin{prb}[Syntax of first-order logic]
A \emph{first-order language} is a language $\cL$, determined by a signature $(\cS,\cF,\cP)$, with an alphabet and a grammar defined as follows.
The alphabet is given by the disjoint union of the following sets:
\begin{enumerate}[(i)]
\item the set of \emph{logical symbols}, which is the disjoint union
\[(\bigcup_{s\in\cS}\cV_s)\cup\{=\}\cup\{\neg,\wedge,\vee,\to,\leftrightarrow\}\cup\{\forall,\exists\},\]
where $\cV_s$ is any countably infinite set of \emph{variables} of sort $s$,
\item the set of \emph{non-logical symbols}, which is the disjoint union $\cF\cup\cP$.
\end{enumerate}
The grammar is given as follows:
\begin{enumerate}[(i)]
\item the set $\cT_s$ of \emph{terms} of a sort $s$ are defined recursively such that
\begin{enumerate}[(\text{i}.i)]
\item $v\in\cT_s$ for $v\in\cV_s$,
\item $f(\bar t)\in\cT_s$ for $f\in\cF_{\bar s\to s}$ and $\bar t\in\cT_{\bar s}$
\end{enumerate}
\item the set $\cL$ of \emph{formulas} are defined such that
\begin{enumerate}[(\text{iii}.i)]
\item $=tt'\in\cL$ for $t,t'\in\cT_s$,
\item $p(\bar t)\in\cL$ for $p\in\cP_{\bar s}$ and a finite sequence $\bar t\in\cT_{\bar s}$,
\item $\neg\f,\wedge\f\psi,\vee\f\psi,\to\f\psi,\leftrightarrow\f\psi\in\cL$ for $\f,\psi\in\cL$,
\item $\forall v\f,\exists v\f\in\cL$ for $v\in\cV$ and $\f\in\cL$.
\end{enumerate}
\end{enumerate}
Because a first-order language is totally determined by a signature, some authors simply define or identify a first-order language with a signature.

(notation $\bar s$, $\bar t$, $\cV_s$, $\cV_{\bar s}$, $\cF_{\bar s\to s}$, $\cP_s$, paranthese, $\f(v)$)

Polish notation and readability.


free variables and substitution

A \emph{sentence} or a \emph{closed formula} is a formula over $\cL$ not having any free variables.
A \emph{theory} over a first-order language $\cL$ is simply a set consisting of some sentences over $\cL$, which we designate each sen to be true.
Let $\cL$ be a first-order language with the signature $(\cS,\cF,\cP)$.
\begin{parts}
\item In our definition, we always have $|\cL|=\aleph_0+|\cF|+|\cP|$.
\item 
\end{parts}
\end{prb}



\begin{prb}[Semantics of first-order logic]
Let $\cL$ be a first-order language with the signature $(\cS,\cF,\cP)$.
A \emph{structure} $M$ over $\cL$ is a family of non-empty sets $\{M_s\}_{s\in\cS}$ together with functions $f_M:M_{\bar s}\to M_s$ and relations $p_M\subset M_{\bar s}$ assgined to each symbol of $f\in\cF_{\bar s\to s}$ and $p\in\cP_{\bar s}$.
The datum of the functions $f_M:M_{\bar s}\to M_s$ and relations $p_M\subset M_{\bar s}$ is called an \emph{interpretations} in $M$ of the function symbols $f$ and predicate symbols $p$.
Given an $\cL$-structure $M$, letting a variable be an identity function and applying the composition of the interpretations of function symbols repeatedly, we can also interpret a term $t\in\cT_s$ of free variables $\bar v\in\cV_{\bar s}$ as a function $t_M:M_{\bar s}\to M_s$.


Let $M$ be an $\cL$-structure.
For an $\cL$-formula $\f$ of free variables $\bar v\in\cV_{\bar s}$, and for elements $\bar a\in M_{\bar s}$, we say $M$ \emph{satisfies} $\f(\bar a/\bar v)$ and write $M\vDash\f(\bar a/\bar v)$ if
\begin{enumerate}[(i)]
\item $t_M(\bar a/\bar v)=t'_M(\bar a/\bar v)$, when $\f$ is $t=t'$ for $t,t'\in\cT_s$ of free variables $\bar v$,
\item $\bar t_M(\bar a/\bar v)\in p_M$, when $\f$ is $p(\bar t)$ for $p\in\cP_{\bar s}$ and $\bar t\in\cT_{\bar s}$ of free variables $\bar v$,
\item $M\not\vDash\psi(\bar a/\bar v)$, when $\f$ is $\neg\psi$, where $\psi$ is an $\cL$-formula of free variables $\bar v$,
\item $M\vDash\psi(\bar a/\bar v)$ and $M\vDash\psi'(\bar x)$, when $\f$ is $\psi\wedge\psi'$, where $\psi$ and $\psi'$ are $\cL$-formulas of free variables $\bar v$,
\item $M\vDash\psi(\bar a'/\bar v',a/v,\bar a''/\bar v'')$ for some $a\in M_s$ and $(\bar a',\bar a'')=\bar a$, when $\f$ is $\exists v\psi$, where $\psi$ is an $\cL$-formula of free variables $(\bar v',v,\bar v'')$ with $(\bar v',\bar v'')=\bar v$.
\end{enumerate}
\end{prb}

\begin{prb}[Deduction in first-order logic]
\end{prb}



Let $M$ be a structure over a first-order language $\cL$.
A relation $p_M$ on $M_{\bar s}$ is called \emph{definable} if there is an $\cL$-formula $\f$ and a constant $\bar c\in M_{\bar s'}$ such that $p=\{\bar a\in M_{\bar s}:\f(\bar a/\bar v,\bar c/\bar w)\}$.


\section{Compactness and completeness}

proofs: finite, sound, decidable.

A first-order language $\cL$ is called \emph{recursive} or \emph{decidable} if there is an algorithm that,

\emph{recursively enumerable} if there is an algorithm that, when given an $\cL$-sentence $\f$ as input, halts accepting if $T\vdash\f$, and does not halt if $T\not\vdash\f$.


\begin{prb}[Henkin constructions]
Let $\cL$ be a single-sorted first-order theory, and let $\cT$ be an $\cL$-theory.

The completeness theorem clearly implies the compactness theorem.
We give a proof of the compactness theorem that does not use any proof system.

We will say an $\cL$-theory $\cT$ has \emph{witnesses} if for every $\cL$-formula $\f$ with a single free variable $v\in\cL$ there is a constant symbol $c\in\cL$ such that $\cT_0\vDash(\exists v\f(v))\to\f(c)$ for a finite subset $\cT_0\subset\cT$.
Note that the witness property of $\cT$ implies that the set $\cC:=\cF_0$ of constant symbols is non-empty.

\begin{parts}
\item If $\cT$ is finitely satisfiable and maximal with witnesses, then it is satisfiable by a model of a cardinality $\kappa\le|\cC|$.
\item witnessing extension of languages
\item maximal extension of theories (extension of theories preserves the property of having witnesses)
\item If $\cT$ is finitely satisfiable, then it is satisfiable by a model of any cardiality $\kappa\ge|\cL|$.
\end{parts}
\end{prb}
\begin{pf}
(a)
To construct a model, we need to define a set $M$ with interpretations $f:M^n\to M$ and $p\subset M^n$ for each $f\in\cF_n$ and $p\in\cP_n$, where $n\in\N$ is fixed.
Define the underlying set $M:=\cC/\sim$, where an equivalence relation on $\cC$ is defined such that $c\sim d$ if and only if $\cT_0\vDash c=d$ for a finite subset $\cT_0\subset\cT$.
For each $c\in\cC$, we have a trivial interpretation $c_M:=[c]$ defined by the equivalence class of $c$ in $M$.

For each $p\in\cP_n$, we interpret it by
\[p_M:=\{\bar c_M\in M^n:p(\bar c)\in\cT\}.\]
To check the well-definedness, letting $\bar c_M=\bar c'_M$ and assuming $\bar c_M\in p_M$, we show $\bar c'_M\in p_M$.
Take representatives $\bar c,\bar c'\in\cC^n$ such that $p(\bar c)\in\cT$ and $\cT_0\vDash\bar c=\bar c'$ for a finite subset $\cT_0\subset\cT$.
If $p(\bar c')\notin\cT$, then we have $\neg p(\bar c')\in\cT$ by the maximality of $\cT$ and a contradiction $\{p(c),\neg p(c')\}\cup\cT_0\vDash\{\bar c=\bar c',\bar c\ne\bar c'\}$ to the finite satisfiability of $\cT$, so we have $p(\bar c')\in\cT$ and $\bar c'_M\in p_M$.

For each $f\in\cF_n$ with $n\ge1$, we interpret it by
\[f_M:=\{(\bar c_M,c_M)\in M^{n+1}:f(\bar c)=c\in\cT\}.\]
First, we claim that for $\bar c_M\in M^n$ there is $c_M\in M$ such that $(\bar c_M,c_M)\in f_M$.
If we choose the representative $\bar c\in\cC^n$ for $\bar c_M$, then since $\varnothing\vDash\exists v(f(\bar c)=v)$ by definition of interpretation of function symbols, and since there is $c\in\cC$ such that $\cT_0\vDash\exists v(f(\bar c)=v)\to f(\bar c)=c$ for a finite subset $\cT_0\subset\cT$ by the witness property of $\cT$, we have $\cT_0\vDash f(\bar c)=c$.
If $f(\bar c)=c\notin\cT$, then we have $f(\bar c)\ne c\in\cT$ by the maximality and a contradiction $\{f(\bar c)\ne c\}\cup\cT_0\vDash\{f(\bar c)\ne c,f(\bar c)=c\}$ to the finite satisfiability of $\cT$, so we get $f(\bar c)=c\in T$ and $(\bar c_M,c_M)\in f_M$.
Second, we claim that $(\bar c_M,c_M)\in f_M$ and $(\bar c_M,c_M')\in f_M$ imply $c_M=c_M'$.
Choose the representatives $(\bar c,c)\in\cC^{n+1}$ such that $f(\bar c)=c\in\cT$ and $(\bar c',c')\in\cC^{n+1}$ such that $f(\bar c')=c'\in\cT$ and $\bar c_M=\bar c_M'$.
Since $\cT_0\vDash\bar c=\bar c'$ for a finite subset $\cT_0\subset\cT$ by definition of $\sim$, we have $\{f(\bar c)=c,f(\bar c')=c'\}\cup\cT_0\vDash c=c'$, so $c_M=c_M'$ by definition of $\sim$ again.
By the above two claims, the interpretation $f_M:M^n\to M$ is a well-defined function.

Now we have an $\cL$-structure $M$ with $|M|\le|\cC|$, so it is enough to prove $M\vDash\f$ if $\f\in\cT$.



For any term $t$ of free variables $\bar v\in\cV^n$ and for any $(\bar a,a)\in M^{n+1}$, we will show that $t_M(\bar a/\bar v)=a$ if and only if there exists $(\bar c,c)\in\cC^{n+1}$ such that $t(\bar c)=c\in\cT$.

Now we prove $M\vDash\f$ for $\f\in\cT$ by the induction on the logical complexity.
We need to divide the cases into the five situations.

If $\f\in\cT$ is $t(\bar c)=t'(\bar c)$

If $\f\in\cT$ is $p(t(\bar c))$

If $\f\in\cT$ is $\neg\psi$,

If $\f\in\cT$ is $\psi\wedge\psi'$,

If $\f\in\cT$ is $\exists v\psi$, where $\psi$ is an $\cL$-formula of a single free variable $v$, then there is $c\in\cC$ such that $\cT_0\vDash\f\to\psi(c)$ for a finite subset $\cT_0\subset\cT$ because $\cT$ has witnesses.
If $\psi(c)\notin\cT$, then we have $\neg\psi(c)\in\cT$ by the maximality and a contradiction $\{\neg\psi(c),\f\}\cup\cT_0\vDash\{\neg\psi(c),\psi(c)\}$ to the finite satisfiability of $\cT$, so $\psi(c)\in\cT$.
By the inductive assumption, we have $M\vDash\psi(c)$.
Since $M\vDash\f$ is equivalent to the existence of $a\in M$ such that $M\vDash\psi(a/v)$, which is true by taking $a:=c_M$.
\end{pf}

\section{Peano arithmetic}

\begin{prb}[Peano arithmetic]
Let $\cL$ be a single-sorted first-order language with the set of non-logical symbols $\{+,\cdot,s,0\}$, where $+$ and $\cdot$ are binary function symbols, $s$ is a unary function symbol, and $0$ is a constant.
The \emph{first-order Peano arithmetic} is a theory $\mathrm{PA}$ over $\cL$ language consisting of the following sentences:
\begin{enumerate}[(i)]
\item $\forall x(s(x)\ne0)$,
\item $\forall x\exists y(x\ne0\to s(y)=x)$,
\item[(ii$'$)] $\forall x\forall y(s(x)=s(y)\to x=y)$,
\item $\forall x(x+0=x)$,
\item $\forall x\forall y(x+s(y)=s(x+y))$,
\item $\forall x(x\cdot0=0)$,
\item $\forall x\forall y(x\cdot s(y)=x\cdot y+x)$,
\item $\forall\bar t\forall x\forall y((\f(\bar t,0)\wedge(\f(\bar t,x)\to\f(\bar t,s(x))))\to\f(\bar t,y))$ for each formula $\f\in\cL$.
\end{enumerate}
A model of the first-order Peano arithmetic is also called a model of the \emph{natural numbers}.
We can ask the classification of models for the natural numbers.

The standard model of the natural numbers cannot be characterized by $\mathrm{PA}$, since definable formulas $\f\in\cL$ are at most countable.

\end{prb}



\section{Zermel-Fraenkel axioms with choice}


\begin{prb}[Zermelo-Frankel set theory with choice]
Let $\cL$ be a single-sorted first-order language with a single non-logical symbol $\in$ that is a binary relation.
The \emph{Zermelo-Fraenkel set theory with choice} is a theory $\mathrm{ZFC}$ over $\cL$ consisting of the following sentences:
\begin{enumerate}[(i)]
\item $\forall x\forall y(\forall z(z\in x\leftrightarrow z\in y)\to x=y)$,\hfill(extensionality)
\item $\forall x(x\ne\varnothing\to\exists y(y\in x\wedge y\cap x=\varnothing))$,\hfill(regularity)
\item $\forall\bar t\forall x\exists y\forall z(z\in y\leftrightarrow(z\in x\wedge\f(\bar t,z)))$ for each formula $\f\in\cL$,\hfill(comprehension)
\item $\forall\bar t\forall x(\forall z(z\in x\to\exists!w\f(\bar t,z,w))\to\exists y\forall z(z\in x\to\exists w(w\in y\wedge\f(\bar t,z,w))))$ for each formula $\f\in\cL$,\hfill(replacement)
\item $\forall x\forall y\exists z(x\in z\wedge y\in z)$,\hfill(pairing)
\item $\forall x\exists y\forall z\forall w(w\in z\wedge z\in x\to w\in y)$,\hfill(union)
\item $\forall x\exists y\forall z(z\subset x\to z\in y)$\hfill(power set)
\item $\exists x(\varnothing\in x\wedge\forall y(y\in x\to(s(y)\in x)))$ \hfill(infinity)
\item $\forall x(\forall z\forall w(z\in x\wedge w\in x\wedge z\ne w\to z\cap w=\varnothing)\to\exists y\forall z(z\in x\to(|y\cap z|=1)))$ \hfill(choice)
\end{enumerate}

Fraenkel added the replacement axiom.

\end{prb}

\begin{prb}[Classes]

\end{prb}


\begin{prb}[]
An \emph{ordinal} or a \emph{von Neumann ordinal} is a set that is strictly well-ordered  with respect to the set membership, and transitive.

A countable ordinal is corresponded to trees with finite leaves.

An ordinal is called a \emph{successor ordinal} if it can be written as $\omega+1$ for some ordinal $\omega$.
It an ordinal is not a successor, then it is called a \emph{limit ordinal}.
\end{prb}




\chapter{Consistency}

\chapter{Large cardinals}





\part{Model theory}

\chapter{}

satisfiable theory
complete theory
$\kappa$-categorical theory

For a first-order theory $T$, the completeness theorem says that a sentence is provable in $T$ if and only if it is true in every model of $T$.
The G\"odel incompleteness theorem states that there exists a sentence in the first-order Peano arithmetic which is true in the standard model but false in another model.



\begin{prb}[Henkin construction]
Fix a signature $\Sigma$.
We say a theory $T$ has the \emph{witness property} if for every formula $\f$ with a single free variable there exists a constant $c$ such that $T\vDash(\exists x\ \f(x))\to\f(c)$.
If a theory is maximal, finitely satisfible, and has the witness property, then it has a model $M$ such that $|M|\le\kappa$ for every cardinal $\kappa\ge|\mathrm{Const}(\Sigma)|$.
\end{prb}

\begin{prb}[Vaught test]
\end{prb}

\begin{prb}[L\"owenheim-Skolem theorem]
\end{prb}


\part{Proof theory}
\chapter{Proof calculi}

Hilbert calculus

Gentzen natural calculus

Gentzen sequent calculus


\part{Recursion theory}
\chapter{}
Let $D$ be a countably infinite set.



We say a function is \emph{computable} if there is an informal algorithm 

We say a relation is \emph{decidable} if there is an informal algorithm



For a relation $R$ on a set $A$ of arity $n$.
We say $R$ is $\Delta_0$ if there is a $\Delta_0$ formula $\f$ with $n$ free variables such that $R=\{\bar a:A\vDash\f(\bar a)\}$.

$\Delta_0$: $\forall x\ x\in y\to\f$ and $\exists x\ x\in y\wedge\f$..



definition of Turing machines.

A \emph{Turing machine} is a triple of a finite set $S$ of states, a finite set $\Sigma$ of symbols, and a transition function $\delta:S\times\Sigma\to S\times\Sigma\times\{L,R\}$.



\end{document}