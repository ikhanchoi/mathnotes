\documentclass{../../large}
\usepackage{../../ikhanchoi}


\begin{document}
\title{Number Theory}
\author{Ikhan Choi}
\maketitle
\tableofcontents

\part{Quadratic reciprocity}

\chapter{Congruence}


\chapter{Quadratic residue}

\section{Legendre symbol}
\begin{prb}[Supplenetal cases of Legendre symbol for odd primes]
\[\left(\frac{-1}p\right)=(-1)^{\frac{p-1}2}\quad(p\ne2).\]
\[\left(\frac2p\right)=1\text{ if and only if }p\equiv\pm1\pmod8\quad(p\ne2).\]
\[\left(\frac3p\right)=1\text{ if and only if }p\equiv\pm1\pmod{12}\quad(p\ne2).\]
\[\left(\frac5p\right)=1\text{ if and only if }p\equiv\pm1\pmod5\quad(p\ne2).\]
\end{prb}
\begin{prb}
\[x^2\equiv0,1\pmod{3,4}\]
\[x^2\equiv0,1,4\pmod{5,8}\]
\[x^2\equiv0,1,3,4\pmod{6}\]
\[x^2\equiv0,1,2,4\pmod{7}\]
\[x^2\equiv0,1,4,7\pmod{9}\]
\[x^2\equiv0,1,4,9\pmod{12}\]
\end{prb}



\section{Gauss sum}

\begin{prb}[Quadratic Gauss sum]
Let $p$ be an odd prime and $a$ an integer.
The \emph{quadratic Gauss} is
\[g(a;p):=\sum_{n=0}^{p-1}\zeta_p^{an^2},\]
where $\zeta:=e^{2\pi i/p}$ is a primitive $p$th root of unity.
\end{prb}
\begin{pf}


\end{pf}




\section{Proofs of quadratic reciprocity}
\begin{prb}[Eisenstein's proof]

\end{prb}

\section*{Exercises}
\begin{prb}[Dirichlet theorems by quadratic reciprocity]
\begin{parts}
\item For $f(x)\in\Z[x]$, there exist infinitely many primes $p$ such that $p\mid f(x)$ for some $x$.
\item There are infinitely many primes $p$ such that $p\equiv1\pmod4$.
\end{parts}
\end{prb}

\begin{prb}
$y^2=f(x)$

Higher order sides: At least a prime divisor of $f$ with a congruence (e.g. $4k+3$)
Quantratic sides: Every prime divisor of $f$ must satisfy a congruence (e.g. $4k+1$)
\end{prb}


\begin{prb}[Primes of the form $x^2-ny^2$]
(It is a very important problem in listing primes in $\cO_K$)
(Want to describe the surjective homomorphism $\Spec\Z[i]\to\Spec\Z$)
\end{prb}



\section*{Problems}

\begin{enumerate}
\item Show that if $\frac{x^2+y^2+z^2}{xy+yz+zx}$ is an integer, then it is not divided by three.
\item There is no non-trivial integral solution of $x^4-y^4=z^2$.
\end{enumerate}




\chapter{Binary quadratic forms}
\section{Reduced forms}
\section{Indefinite forms}

\section{Ideal class group}

\begin{prb}[Heegner number]
There are only nine numbers
\[-1,-2,-3,-7,-11,-19,-43,-67,-163.\]
\end{prb}

\section*{Exercises}
\begin{prb}[Mordell equation with no solutions]
\begin{parts}
\item $y^2=x^3+7$ has no integral solutions.
\end{parts}
\end{prb}

\begin{prb}[Mordell equation with solutions]
\begin{parts}
\item $y^2=x^3-2$ has only two solutions.
\end{parts}
\end{prb}









\part{Multiplicative number theory}
\chapter{Arithmetic functions}
\chapter{Dirichlet's theorem}
\chapter{Prime number theorem}




\part{Quadratic Diophantine equations}
\chapter{Pell's equation}
\section{Continued fraction}
Diophantine approximation, Thue theorem

\chapter{$p$-adic numbers}
\section{Hensel lemma}

\chapter{Local-global principle}
\section{Hasse-Minkowski theorem}




\part{Elliptic curves}
\chapter{Elliptic curves over $\C$}
\chapter{Elliptic curves over $\Q$}
\section{Finitely generatedness}
Mordell-Weil, Mazur torsion
\section{Integral solutions}
Nagell-Lutz, Siegel, Baker's bound
\chapter{Elliptic curves over $\F_p$}

\end{document}