\documentclass{../../large}
\usepackage{../../ikhanchoi}



\begin{document}
\title{Representation Theory}
\author{Ikhan Choi}
\maketitle
\tableofcontents

\part{Finite group representations}
\chapter{Character theory}

\section{Irreducible representations}
\begin{prb}[Definition of group representations]
\end{prb}

\begin{prb}[Intertwining maps]
\end{prb}

\begin{prb}[Subrepresentations]
We say \emph{invariant} or \emph{stable}
\end{prb}

\begin{prb}[Irreducible representations]
indecomposable and irreducible
\end{prb}

\begin{prb}[Maschke's theorem]
Let $G$ be a finite group and $k$ be a field.
Suppose the characteristic of $k$ does not divide $|G|$.
Let $V$ be a finite-dimensional representation of $G$ over $k$.
\begin{parts}
\item Every invariant subspace $W$ of $V$ has a complement $W'$ in $V$ that is also invariant.
\item $V$ is isomorphic to the direct sum of irreducible representations of $G$ over $k$.
\item If $k=\R$ or $\C$, then $V$ admits an inner product such that $W\perp W'$ and $\rho_V(g)$ is unitary for all $g\in G$.
\end{parts}
\end{prb}

\begin{prb}[Schur's lemma]
Let $G$ be a group and $k$ be a field.
Let $V$ and $W$ be irreducible representations of $G$ over $k$.
Let $\psi:V\to W$ be an intertwining map.
\begin{parts}
\item If $V\not\cong W$, then $\psi=0$.
\item If $V\cong W$, then $\psi$ is an isomorphism.
\item If $k$ is algebraically closed and $\dim V<\infty$, then every intertwining map $\psi:V\to V$ is a homothety.
\end{parts}
\end{prb}




\section{Group algebra}


\begin{prb}[Modules and representations]
ring <-> group
module <-> representation
finitely generated <-> finite dimensional
\end{prb}

\begin{prb}[Wedderburn's theorem]
central idempotents
dimension computation
\end{prb}

\begin{prb}[Group algebra]
regular representation
$k[G]$-module and $G$-representation correspondence
\begin{parts}
\item $\C[G]$ is the direct sum of all irreducible representations.
\item $|G|=\sum_{[V]\in\hat G}(\dim V)^2$.
\end{parts}
\end{prb}

\begin{prb}
The number of irreducible representations and the number of conjugacy classes
double counting on $Z(\C[G])$.
\end{prb}



\section{Characters}




\begin{prb}[Space of class functions]
Ring and inner product structure on the space of class functions.
\begin{parts}
\item $\dim\hom_G(V,W)=\<\chi_V,\chi_W\>$.
\item Irreducible characters form an orthonormal basis of the space of class functions.
\end{parts}
\end{prb}

\begin{prb}[Characters classify representations]
Let $G$ be a finite group and let $\mathbf{Rep}(G)$ be the category of finite-dimensional representations of $G$ over $\C$.
\[\Tr:\mathbf{Rep}(G)\to\{\text{finite sum of irreducible characters}\}\]
surjectivity: trivial
injectivity:
	Suppose two characters are equal.
	Maschke -> all characters are sum of irreducible characters
	Schur -> orthogonality, so the coefficients are all equal
	irreducible-factor-wisely construct an isomoprhism.
\end{prb}



\begin{prb}[Character table]
computation of matrix elements by character table
abelian group, 1dim rep lifting
\begin{center}
$\begin{array}{c|ccc}
S^3&e&(12)&(123)\\\hline
1&1&1&1\\
\e&1&-1&1\\
\rho&2&0&-1
\end{array}$
\end{center}
\end{prb}






the dual inner product: conjugacy check
relation to normal subgroups
center of rep




algebraic integer
dim of irrep divides group order
burnside pq theorem





\chapter{Classification of representations}
\section{Symmetric groups}
young tableux

\section{Linear groups over finite fields}
GL2 and SL2 over finite fields

\section{Induced representations}
induction and restriction of reps (from and to subgroup)
frobenius reciprocity, mackey theory





tensoring, complex, real
symmetric, exterior


\chapter{Brauer theory}












\part{Lie algebras}
\chapter{Semisimple Lie algebras}
\section{Linear Lie algebras}

group acts on an algebra $A$(e.g. $\End(V)$).
then its group algebra acts on $A$.
Lie algebra acts on $A$, and this Lie algebra information is enough to recover the group action.
Geometric meaning of Lie algebra action?

Lie algebra can only considered as a quantization of Poisson bracket.
How can the Poisson bracket embodies the group action?






Following Humphrey's book, let $L$ be always finite dimensional Lie algebra unless stated.

\begin{prb}
Every associative algebra is a Lie algebra, where the Lie bracket is given by the commutator.
For a Lie algebra, we are 

Intuitions of subalgebras, ideals, derivations.
Intuitions of solvable, nilpotent, and semisimple Lie algebras.
Constructing representations, trace forms,

The \emph{general linear Lie algebra} $\fgl(V)$ is just $\End(V)$ with a Lie bracket $[x,y]:=xy-yx$.
\end{prb}

\begin{prb}[Derivations]
Let $L$ be a Lie algebra.
A \emph{derivation} of $L$ is a linear map $\delta:L\to L$ such that \[\delta([x,y])=[\delta(x),y)]+[x,\delta(y)]\]
for all $x,y\in L$.
The set of derivations $\Der(L)$ of $L$ is a subalgebra of $\fgl(L)$, and we have the \emph{adjoint representation} $L\to\Der(L)\le\fgl(L)$ of $L$.
If $I$ is an ideal, then we have a faithful representation $\ad:L\to\ad L\le\Der(I)\le\fgl(I)$.
\end{prb}

\begin{prb}[Inner derivations and automorphisms]
Let $L$ be a Lie algebra.

The linear map $\ad x=[x,-]:L\to L$ for $x\in L$ is derivation, and derivation of this form is called \emph{inner}, and they form an ideal of $\Der(L)$.

Automorphisms of the form $\exp(\ad x)$ with nilpotent $\ad x$ generates a normal subgroup of $\Aut(L)$, and each generator is called \emph{inner automorphisms}.
\end{prb}







\begin{prb}[Solvable Lie algebras]
Let $L$ be a Lie algebra.
If the \emph{derived series} $L^{(0)}=L$, $L^{(n)}=[L^{(n-1)},L^{(n-1)}]$ eventually vanishes, then we call $L$ \emph{solvable}.

If $L$ is solvable, then its subalgebras and quotient algebras are all solvable.
If $I$ is a solvable ideal of $L$ such that $L/I$ is solvable, then $L$ is solvable.
The sum of two solvable ideals is also solvable.
\end{prb}

\begin{prb}[Nilpotent Lie algebras]
Let $L$ be a Lie algebra.
If the \emph{lower central series} $L^0=L$, $L^n=[L,L^{n-1}]$ eventually vanishes, then we call $L$ \emph{nilpotent}.
It is a stronger notion than solvability.

If $L$ is nilpotent, then its subalgebras and quotient algebras are all nilpotent.
If $L/Z(L)\cong\ad(L)\subset\fgl(L)$ is nilpotent, then $L$ is nilpotent.
If $L$ is non-zero and nipotent, then $Z(L)$ is non-trivial.
\end{prb}

\begin{prb}[Engel's theorem]
.
\begin{parts}
\item A linear Lie algebra $L\subset\fgl(V)$ consists of nilpotent endomorphisms if and only if $L\subset\fn(V)$ for a certain basis of $V$.
\item An abstract Lie algebra $L$ is nilpotent if and only if $\ad(L)$ consists of nilpotent endomorphisms.
\item If $L\subset\fgl(V)$ is nilpotent in $\End(V)$, then there is a \emph{common eigenvector} $v\in V$ such that $[L,v]=0$, i.e. there is a flag $V_i$ such that $xV_i\subset V_{i-1}$...?
\end{parts}
\end{prb}
\begin{pf}



Let $L$ be an ad-nilpotent Lie algebra.
Then, every element of $\ad L\subset\fgl(L)$ is a nilpotent endomorphism, so there is $x\in L$ such that $[L,x]=0$, which implies $Z(L)\ne0$.
Since $L/Z(L)$ is also ad-nilpotent, and by induction on dimension, $L/Z(L)$ is nilpotent.
Therefore, $L$ is nilpotent.
\end{pf}


\begin{prb}[Lie's theorem]
Let $\F$ have characteristic zero and be algebraically closed.
\begin{parts}
\item A linear Lie algebra $L\subset\fgl(V)$ is solvable if and only if $L\subset\ft(V)$ for a certain basis of $V$.
\item If $L$ is solvable, then there is a flag $V_i$ such that $xV_i\subset V_i$.
\item Let $L$ be an abstract Lie algebra. $L$ is solvable if and only if $[L,L]$ is nilpotent.
\item Every finite-dimensional irreduciable representation of a solvable Lie algebra is one-dimensional.
\end{parts}
\end{prb}
\begin{pf}
Use induction on dimension.
Since $L/[L,L]$ is a non-trivial commutative Lie algebra, in which every subspace is an ideal, we can show the existence of an ideal $K$ of $L$ with codimension one by pullback.

By the induction assumption, we have a common eigenvector in $V$ for $K$ so that we have the ``eigenvalue'' linear functional $\kappa:K\to\F$ such that the ``eigenspace'' of $\kappa$ as
\[V_\kappa:=\{v\in V:xv=\kappa(x)v\text{ for }x\in K\}\]
is non-trivial.

Let $L=K+\F z$ with $z\in\fgl(V)$.
If $V_\kappa$ is invariant by $L$, then $V_\kappa$ contains an eigenvector of $z$ by the fact that $\F$ is algebraically closed, so we can extend $\kappa$ to obtain $\lambda:L\to\F$ such that $(V_\kappa)_\lambda$ is non-trivial.

We now show that $V_\kappa$ is invariant by $L$.
Let $v\in V_\kappa$ and $x\in L$.
Since
\[yxv-\lambda(y)xv=yxv-xyv=[y,x]v=\lambda([y,x])v\]
for $y\in K$, we have to show $\lambda([y,x])=0$.
Take $n$ to be largest such that $v,\cdots,x^{n-1}v$ are linearly independent.
Since $[x,y]$ is upper triangular matrix relative to the basis $v,\cdots,x^{n-1}v$ and the diagonal entries are $\lambda([x,y])$.
Since the trace of $[x,y]$ must be zero, we have $\lambda([x,y])=0$ because $\F$ has characteristic zero.

\end{pf}

\begin{prb}



\end{prb}

There is a linear functional $\lambda:L\to\F$ such that $\lambda|_{[L,L]}=0$ and $V_\lambda$ is non-trivial.
$V_\kappa$


For a representation $\pi:\fg\to\fgl(V)$, then a weight of $V$ is a linear functional $\lambda:\pi(\fh)\to\F$ such that the weight space $V_\lambda$ is non-trivial.



\section{Semisimple Lie algebras}


\begin{prb}
Therefore, $L$ admits a unique maximal solvable ideal, called \emph{radical}.
If the radical is trivial, then we say $L$ is \emph{semisimple}.
Since the center is a solvable ideal, the center of a semisimple Lie algebra is trivial.
\begin{parts}
\item A canonical example of a solvable Lie algebra is the Lie algbera of upper triangular matrices.
\item The radical of $\fgl(n,\F)$ is $\fsl(n,\F)$.($\F$ cahracteristic zero?) Upper triangular matrices do not form an ideal of $\fgl(n,\F)$.
\item $[\ft,\ft]=\fn$, $\ft=\fd\otimes\fn$. $\ft$ is a solvable subalgebra of $\fgl$, but not a solvable ideal.
\item $\fsl(n,\F)$ is simple if $\char\F=0$.
\end{parts}
\end{prb}

\begin{prb}[Jordan-Chevalley decomposition]
Let $\char\F$ be arbitrary.
We say $x\in\End(V)$ is called \emph{semisimple} if the roots of its minimal polynomial are all distinct.
If $\F$ is algebraically closed, $x\in\End(V)$ is semisimple if and only if it is diagonalizable.
Let $x\in\End V$.
Even if $\F$ is not algebraically closed, we have a generalization of Jordan decomposition as follows:
\begin{parts}
\item There exist unique $x_s,x_n\in\End V$ such that $x=x_s+x_n$ and $x_s$ semisimple, $x_n$ nilpotent.
\item $x_s$ and $x_n$ are polynomials in $x$.
\item If $x$ maps $B$ to $A$, then $x_s$ and $x_n$ also map $B$ to $A$ for subspaces $A\le B\le V$.
\end{parts}
\end{prb}

\begin{prb}[Cartan's criterion]
We will show a powerful criterion for solvability.
\begin{parts}
\item Let $A\subset B$ be two subspaces of $\fgl(V)$, $V$ finite dimensional. Let
\[M:=\{x\in\fgl(V):[x,B]\subset A\}.\]
If $x\in M$ satisfies $\Tr(xy)=0$ for all $y\in M$, then $x$ is nilpotent.
\item Let $L\subset\fgl(V)$, $V$ finite dimensional. If $\Tr(xy)=0$ for all $x\in[L,L]$ and $y\in L$, then $L$ is solvable.
\end{parts}
\end{prb}

\begin{prb}[Killing forms]
Let $L$ be a Lie algebra.
\[\kappa(x,y):=\Tr(\ad x\ad y)\]
is a symmetric bilinear form on $L$, which is called the \emph{Killing form} on $L$, i.e. it is the trace form for the adjoint representation.
\begin{parts}
\item On an ideal $I\subset L$, the Killing form is given by restriction.
\item The kernel of $\kappa$ is contained in the radical of $L$, and triviality is equivalent; $L$ is semisimple if and only if $L$ is non-degenerate. (Here we use Cartan's criterion)
\item If $L$ is semisimple, then it is the direct sum of simple ideals.
\item If $L$ is semisimple, then every derivation is inner.
\item If $L$ is semisimple, then $L=[L,L]$ and every subalgebras and quotients are semisimple.
\end{parts}
\end{prb}

Levi decomposition


\begin{prb}[Casimir element]
For a faithful representation $\f:L\to\fgl(V)$, we can associate a non-degenerate trace form since $L$ is semisimple.
Then, the \emph{Casimir element} of the representation $\f$ is $C_\f:=\sum_i\f(x_i)\f(y_i)\in\End(V)$ where $i$ runs over dual bases relative to the trace form.
\end{prb}

\begin{prb}[Weyl's theorem]
Finite dimensional representation of a semisimple Lie algebra is completely reducible.
Preservation of Jordan decomposition.
\end{prb}

\begin{prb}[Toral subalgebras]
Cartan subalgebra uniqueness? (conjugacy theorem)
\end{prb}




\chapter{Root systems}
root space decomposition
integrality
Weyl group

Coxeter graph
Dynkin diagram
Real forms

Isomorphism theorem

Existence theorem
Universal enveloping algebra
PBW theorem
Verma module




\chapter{Representations of Lie algebras}
\section{Representations of $\fsl(2,\C)$}
\begin{prb}[Pauli matrices]
Pauli matrices are
\[\sigma_1=\mat{0&1\\1&0},\quad\sigma_2=\mat{0&-i\\i&0},\quad\sigma_3=\mat{1&0\\0&-1}.\]
\begin{parts}
\item $\{\sigma_1,\sigma_2,\sigma_3\}$ is a basis of complex Lie algebra $\fsl(2,\C)$, and $\{i\sigma_1,i\sigma_2,i\sigma_3\}$ is a basis of real Lie algebra $\fso(3)$.
\item For a unit vector $n=(n_1,n_2,n_3)\in\R^3$, $n_1\sigma_1+n_2\sigma_2+n_3\sigma_3$ has eigenvalues $\pm1$.
\end{parts}
\end{prb}
\section{Highest weight theorem}

\section{Multiplicity formulas}

\section*{Exercises}
\begin{prb}[Triplets and quadraplets]
Let $(\pi_2,V_2)$ be the irreducible representation of $\fsl(2,\C)$ of degree two.
Consider $V_2\otimes V_2$.
Cartan element $S_z$.
$V_2^{\otimes3}$.
\end{prb}
\begin{prb}[Casimir element]
Casimir element decomposes a representation into irreducible representations.
\end{prb}








\part{Lie groups}
\chapter{Lie correspondence}
\section{Exponential map}
\begin{prb}[Exponential map]
\end{prb}
\begin{prb}[Surjectivity of exponential map]
\end{prb}

\begin{prb}[Lie functor]
\end{prb}


\section{Lie's second theorem}
\begin{prb}[Derivative of the exponential map]
Let $G$ be a Lie group.
\begin{parts}
\item
\[\dd{s}\exp(sX)=\exp(sX)X\]
for $s\in\R$ and $X\in\fg$.
\item
\[\pd{s}\]
\end{parts}
\end{prb}

\begin{prb}[Baker-Campbell-Hausdorff formula]
Let $G$ be a Lie group.
Let $X,Y\in\fg$ such that $\exp(X)\exp(Y)$
Define
\[Z(t):=\log(\exp(X)\exp(tY))\]
\end{prb}

\begin{prb}[]
\begin{parts}
\item The Lie functor
\[\mathrm{Lie}:\mathrm{LieGrp}_{simple}\to\mathrm{LieAlg}_\R\]
is fully faithful.
\end{parts}
\end{prb}


\section{Lie's third theorem}
\begin{prb}[Ado's theorem]
\end{prb}

\begin{prb}[Lie's third theorem]
Also called the Cartan-Lie theorem.
\begin{parts}
\item The Lie functor
\[\mathrm{Lie}:\mathrm{LieGrp}_{simple}\to\mathrm{LieAlg}_\R\]
is essentially surjective.
\end{parts}
\end{prb}

\section{Fundamental groups of Lie groups}



\chapter{Compact Lie groups}
\section{Special orthogonal groups}
\section{Special unitary groups}
\section{Symplectic groups}

\section*{Exercises}
\begin{prb}[Lorentz group]
$\SL(2,\C)\to\SO^+(1,3)$
\begin{parts}
\item $O(1,3)$ has four components and $SO^+(1,3)$ is the identity component. Orthochronous $O^+(1,3)$, proper $SO(1,3)$.
\end{parts}
\end{prb}




\chapter{Representations of Lie groups}
\section{Peter-Weyl theorem}
\section{Spin representations}
Clifford algebra






\part{Hopf algebras}
\chapter{}
\chapter{Quantum groups}



\end{document}