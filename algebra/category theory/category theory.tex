\documentclass{../../large}
\usepackage{../../ikhanchoi}

\newcommand{\Set}{\mathbf{Set}}
\newcommand{\Mon}{\mathbf{Mon}}
\newcommand{\CMon}{\mathbf{CMon}}
\newcommand{\Grp}{\mathbf{Grp}}
\newcommand{\Ab}{\mathbf{Ab}}
\newcommand{\Ring}{\mathbf{Ring}}
\newcommand{\Top}{\mathbf{Top}}

\newcommand{\op}{\mathrm{op}}
\newcommand{\coim}{\operatorname{coim}}

\begin{document}
\title{Category Theory}
\author{Ikhan Choi}
\maketitle
\tableofcontents


\part{}

\chapter{Categories}
set theoretical issues
duality
morphisms
	monic
\section{Functors}
full, faithful
natural transformations and equivalence
2-category






\chapter{Universal property}

\section{Construction}
products, equalizers, pullbacks

\section{Representable functors}

\begin{prb}[Yoneda lemma]
Let $F:\cC\to\Set$ be a functor from a locally small category $\cC$.
Fix $c\in\Ob(\cC)$.
 we can define a function
\[\Nat(\Hom(c,-),F)\to F(c).\]

A \emph{representation} of $F$ is a pair $(c,\eta)$ of an object $c\in\cC$ and a natural isomorphism $\eta:\Hom(c,-)\to F$.

A \emph{universal element} of $F$ is a pair $(c,x)$ with $x\in F(c)$ such that for any pair $(d,y)$ with $y\in F(d)$ there is a unique morphism $f\in\Hom(c,d)$ satisfying $F(f):x\mapsto y$.

\begin{parts}
\item
\end{parts}
\end{prb}


\begin{pf}
(a)
Consider the diagram
\[\begin{tikzcd}[row sep=small, column sep=small]
\Hom(c,c) \ar{rrrr}{\eta_c} \ar{dddd} &&&& F(c) \ar{dddd}{F(f)}  \\
& \id_c \ar[symbol=\in]{ul} \ar[mapsto]{rr} \ar[mapsto]{dd} && x:=\eta_c(\id_c) \ar[symbol=\in]{ur} \ar[mapsto]{dd} & \\
\,&\,&\,&\,&\,\\
& f \ar[symbol=\in]{dl} \ar[mapsto]{rr} && \eta_d(f):=F(f)(x) \ar[symbol=\in]{dr} & \\
\Hom(c,d) \ar[swap]{rrrr}{\eta_d} &&&& F(d).
\end{tikzcd}\]
For a natural transformation $\eta:\Hom(c,-)\to F$, define $x:=\eta_c(\id_c)$ in $F(c)$.
For $x\in F(c)$, conversely, define a $\eta_d:\Hom(c,d)\to F(d)$ by $\eta_d(f):=F(f)(x)$ for $d\in\Ob(\cC)$ and $f\in\Hom(c,d)$.
Then, the collection $\eta=\{\eta_d:d\in\Ob(\cC)\}$ provides a natural transformation because for each $g\in\Hom(d,e)$ we can check the diagram
\[\begin{tikzcd}
\Hom(c,d) \ar{r}{\eta_d}\ar[swap]{d}{g\circ-} & F(d) \ar{d}{F(g)} \\
\Hom(c,e) \ar{r}{\eta_e} & F(e)
\end{tikzcd}\]
commutes from
\[F(g)(\eta_d(f))=F(g)(F(f)(x))=F(g\circ f)(x)=\eta_e(g\circ f),\qquad f\in\Hom(c,d).\]
The correspondences $\eta\mapsto x$ and $x\mapsto\eta$ are inverses of each other, hence the bijection.
\end{pf}




\chapter{Limits}
preservation, reflection, creation
completeness
functoriality









\part{}


\chapter{}
\section{Adjunctions}

\section{Monads}

\section{Kan extensions}



\chapter{Monoidal categories}
closed, symmetric, cartesian
coherence theorem, closure theorem

\section{Enriched categories}

\begin{prb}[Pointed category]
A \emph{pointed category} is a category with a zero object.
\begin{parts}
\item A category is $\Set_*$-enriched if and only if it admits a zero morphism.
\item Every pointed category is $\Set_*$-enriched.
\end{parts}
\end{prb}

\chapter{Abelian categories}
\section{Regular and exact categories}

split, regular, strong
effective, normal, strict

A kernel pair of a morphism $f$ is the pullback of $(f,f)$.

A category is called \emph{regular} if every kernel pair admits a coequalizer.

\begin{prb}
A regular category is called \emph{exact} if every equivalence relation is given by a kernel pair.
\begin{parts}
\item
\end{parts}
\end{prb}

The category $\Grp$ is regular but not coregular, since there is a monomorphism which is not regular.


\section{Additive and abelian categories}

\begin{prb}[Pre-additive categories]
A \emph{pre-additive category} is another name of $\Ab$-enriched category.
\begin{parts}
\item a
\end{parts}
\end{prb}

\begin{prb}[Semi-additive cateogries]
A \emph{semi-additive category} is a category with binary biproducts.
\begin{parts}
\item A category is semiadditive if and only if it is pointed $\CMon$-enriched.
\end{parts}
\end{prb}


\begin{prb}[Additive categories]
\begin{parts}
\item additive completion by formally adjoining finite biproducts.
\item additive structures on a semi-additive category is unique.
\end{parts}
\end{prb}

The notion of kernels and cokernels can be defined in a $\Set_*$-enriched category.


\begin{prb}[Pre-abelian categories]
A \emph{pre-abelian category} is an additive category in which every morphism has the kernel and cokernel.
Equivalently, it is a finitely bicomplete pre-additive category.
\begin{parts}
\item 
\end{parts}
\end{prb}

\begin{prb}[Semi-abelian categories in the sense of Jenelidze-M\'arkin-Tholen]
A pointed, Baar-exact, protomodular, with binary coproudcts.
\begin{parts}
\item short five lemma, $3\times3$ lemma, snake lemma, noehter isomorphism hold.
\item long exact homology sequence
\item Every semi-abelian category is exact.
\item Every semi-abelian category is finitely bicomplete.
\item In general, a semi-abelian category is not pre-additive nor semi-additive.
\end{parts}
\end{prb}

\begin{prb}[Abelian categories]
We say $\cC$ is \emph{abelian} if every morphism has the kernel and cokernel, and every mono and epi is normal.
Roughly, an abelian category is a $\Ab$-enriched category such that it is finitely bicomplete and the first isomorphism holds.
\begin{parts}
\item A category is abelian if and only if it is additive and exact.
\end{parts}
\end{prb}


\begin{prb}[Freyd-Mitchell embedding]
\end{prb}


\[\begin{tikzcd}[math mode=false]
abelian\ar{r}\ar{d} & pre-abelian \ar{r} & additive \ar{r}\ar{d} & pre-additive\\
semi-abelian (JMT) && semi-additive & \,
\end{tikzcd}\]
\begin{itemize}
\item Pre-abelian: abelian topological groups, Banach spaces, Fr\'echet spaces.
\item Semi-abelian: groups, non-unital algebras, Lie algebras, C$^*$-algebras, compact Hausdorff (profinite) spaces.
\item Additive: projective modules
\end{itemize}





The first isomorphism theorem states that $\coim\to\im$ is an isomorphism.
The normal subobjects and the first isomorphism theorem is generalized in the context of protomodular categories.
The cokernel may not be defined.
The category of unital rings is not semi-abelian but protomodular.
\begin{itemize}
\item A \emph{protomodular category}
\item A \emph{homological category} is a pointed regular protomodular category. (five, nine, snake, long exact sequence, noether isomorphism)
\item A \emph{semi-abelian category} is a homological category that is Barr-exact and finite coproducts(free products).
\end{itemize}




\chapter{}
site, topos
$(\infty,1)$-category


\end{document}