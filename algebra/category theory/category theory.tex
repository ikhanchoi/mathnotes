\documentclass{../../large}
\usepackage{../../ikhanchoi}

\newcommand{\Set}{\mathbf{Set}}
\newcommand{\Mon}{\mathbf{Mon}}
\newcommand{\CMon}{\mathbf{CMon}}
\newcommand{\Grp}{\mathbf{Grp}}
\newcommand{\Ab}{\mathbf{Ab}}
\newcommand{\Ring}{\mathbf{Ring}}
\newcommand{\Mod}{\mathbf{Mod}}
\newcommand{\Top}{\mathbf{Top}}

\begin{document}
\title{Category Theory}
\author{Ikhan Choi}
\maketitle
\tableofcontents


\part{}

\chapter{}
set theoretical issues
duality
morphisms
	monic
\section{Functors}
full, faithful
natural transformations and equivalence
2-category


\chapter{Universal properties}
products, equalizers, pullbacks
representability and yoneda
\chapter{Limits}
preservation, reflection, creation
completeness
functoriality

\chapter{}
\section{Adjunctions}

\section{Monads}

\section{Kan extensions}



\chapter{Monoidal categories}
closed, symmetric, cartesian
coherence theorem, closure theorem

\section{Enriched categories}

\begin{prb}[Pointed category]
A \emph{pointed category} is a category with a zero object.
\begin{parts}
\item A category is $\Set_*$-enriched if and only if it admits a zero morphism.
\item Every pointed category is $\Set_*$-enriched.
\end{parts}
\end{prb}

\chapter{Abelian categories}
\section{Regular and exact categories}

split, regular, strong
effective, normal, strict

A kernel pair of a morphism $f$ is the pullback of $(f,f)$.

A category is called \emph{regular} if every kernel pair admits a coequalizer.

\begin{prb}
A regular category is called \emph{exact} if every equivalence relation is given by a kernel pair.
\begin{parts}
\end{parts}
\end{prb}

The category $\Grp$ is regular but not coregular, since there is a monomorphism which is not regular.


\section{Additive and abelian categories}

\begin{prb}[Pre-additive categories]
A \emph{pre-additive category} is another name of $\Ab$-enriched category.
\begin{parts}
\item a
\end{parts}
\end{prb}

\begin{prb}[Semi-additive cateogries]
A \emph{semi-additive category} is a category with binary biproducts.
\begin{parts}
\item A category is semiadditive if and only if it is pointed $\CMon$-enriched.
\end{parts}
\end{prb}


\begin{prb}[Additive categories]
\begin{parts}
\item additive completion by formally adjoining finite biproducts.
\item additive structures on a semi-additive category is unique.
\end{parts}
\end{prb}

The notion of kernels and cokernels can be defined in a $\Set_*$-enriched category.


\begin{prb}[Pre-abelian categories]
A \emph{pre-abelian category} is an additive category in which every morphism has the kernel and cokernel.
Equivalently, it is a finitely bicomplete pre-additive category.
\begin{parts}
\item 
\end{parts}
\end{prb}

\begin{prb}[Semi-abelian categories in the sense of Jenelidze-M\'arkin-Tholen]
A pointed, Baar-exact, protomodular, with binary coproudcts.
\begin{parts}
\item short five lemma, $3\times3$ lemma, snake lemma, noehter isomorphism hold.
\item long exact homology sequence
\item Every semi-abelian category is exact.
\item Every semi-abelian category is finitely bicomplete.
\item In general, a semi-abelian category is not pre-additive nor semi-additive.
\end{parts}
\end{prb}

\begin{prb}[Abelian categories]
We say $\cC$ is \emph{abelian} if every morphism has the kernel and cokernel, and every mono and epi is normal.
\begin{parts}
\item A category is abelian if and only if it is additive and exact.
\end{parts}
\end{prb}


\begin{prb}[Freyd-Mitchell embedding]
\end{prb}


\[\begin{tikzcd}[math mode=false]
abelian\ar{r}\ar{d} & pre-abelian \ar{r} & additive \ar{r}\ar{d} & pre-additive\\
semi-abelian (JMT) && semi-additive & \,
\end{tikzcd}\]
\begin{itemize}
\item Pre-abelian: abelian topological groups, Banach spaces, Fr\'echet spaces.
\item Semi-abelian: groups, non-unital algebras, Lie algebras, C$^*$-algebras, compact Hausdorff (profinite) spaces.
\item Additive: projective modules
\end{itemize}



\chapter{}
site, topos
$(\infty,1)$-category


\end{document}