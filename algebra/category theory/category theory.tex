\documentclass{../../large}
\usepackage{../../ikhanchoi}

\newcommand{\Mon}{\mathrm{Mon}}
\newcommand{\CMon}{\mathrm{CMon}}
\newcommand{\Ring}{\mathrm{Ring}}


\begin{document}
\title{Category Theory}
\author{Ikhan Choi}
\maketitle
\tableofcontents


\part{}

\chapter{Categories}
set theoretical issues
morphisms
	monic

\section{Functors}
fully faithful, essentially surjective
natural transformations and equivalence
2-category


\section{Categorical constructions}
opposite category
product category
disjoint union category
comma category(slice category, morphism category)


\section{Representable functors}

Yoneda embedding gives fully faithful functors $h:\cC\to\mathrm{PSh}(\cC)$ and $k:\cC^{\mathrm{op}}\to\mathrm{coPSh}(\cC)$.
A presheaf $\cC^{\mathrm{op}}\to\mathrm{Set}$ is representable if and only if it is essentially contained in the image of the Yoneda embedding.

\begin{prb}[Yoneda lemma]
Let $F:\cC\to\Set$ be a functor from a locally small category $\cC$.
Fix $c\in\Ob(\cC)$.
 we can define a function
\[\Nat(\Hom(c,-),F)\to F(c).\]

A \emph{representation} of $F$ is a pair $(c,\eta)$ of an object $c\in\cC$ and a natural isomorphism $\eta:\Hom(c,-)\to F$.

A \emph{universal element} of $F$ is a pair $(c,x)$ with $x\in F(c)$ such that for any pair $(d,y)$ with $y\in F(d)$ there is a unique morphism $f\in\Hom(c,d)$ satisfying $F(f):x\mapsto y$.

\begin{parts}
\item
\end{parts}
\end{prb}


\begin{pf}
(a)
Consider the diagram
\[\begin{tikzcd}[row sep=small, column sep=small]
\Hom(c,c) \ar{rrrr}{\eta_c} \ar{dddd} &&&& F(c) \ar{dddd}{F(f)}  \\
& \id_c \ar[symbol=\in]{ul} \ar[mapsto]{rr} \ar[mapsto]{dd} && x:=\eta_c(\id_c) \ar[symbol=\in]{ur} \ar[mapsto]{dd} & \\
\,&\,&\,&\,&\,\\
& f \ar[symbol=\in]{dl} \ar[mapsto]{rr} && \eta_d(f):=F(f)(x) \ar[symbol=\in]{dr} & \\
\Hom(c,d) \ar[swap]{rrrr}{\eta_d} &&&& F(d).
\end{tikzcd}\]
For a natural transformation $\eta:\Hom(c,-)\to F$, define $x:=\eta_c(\id_c)$ in $F(c)$.
For $x\in F(c)$, conversely, define a $\eta_d:\Hom(c,d)\to F(d)$ by $\eta_d(f):=F(f)(x)$ for $d\in\Ob(\cC)$ and $f\in\Hom(c,d)$.
Then, the collection $\eta=\{\eta_d:d\in\Ob(\cC)\}$ provides a natural transformation because for each $g\in\Hom(d,e)$ we can check the diagram
\[\begin{tikzcd}
\Hom(c,d) \ar{r}{\eta_d}\ar[swap]{d}{g\circ-} & F(d) \ar{d}{F(g)} \\
\Hom(c,e) \ar{r}{\eta_e} & F(e)
\end{tikzcd}\]
commutes from
\[F(g)(\eta_d(f))=F(g)(F(f)(x))=F(g\circ f)(x)=\eta_e(g\circ f),\qquad f\in\Hom(c,d).\]
The correspondences $\eta\mapsto x$ and $x\mapsto\eta$ are inverses of each other, hence the bijection.
\end{pf}



\chapter{Limits}

\section{Construction}
products, equalizers, pullbacks



preservation, reflection, creation
completeness
functoriality



limit-preserving
filtered limit-preserving
product-preservig
mono-preserving


filtered limits


\chapter{}
\section{Adjunctions}

\section{Monads}

\section{Kan extensions}


\part{}


\chapter{}



\chapter{Abelian categories}
\section{Regular and exact categories}

split, regular, strong
effective, normal, strict

A kernel pair of a morphism $f$ is the pullback of $(f,f)$.

A category is called \emph{regular} if every kernel pair admits a coequalizer.

\begin{prb}
A regular category is called \emph{exact} if every equivalence relation is given by a kernel pair.
\begin{parts}
\item
\end{parts}
\end{prb}

The category $\Grp$ is regular but not coregular, since there is a monomorphism which is not regular.








\section{Additive and abelian categories}



\begin{prb}[Additive categories]
A \emph{pre-additive category} is an $\Ab$-enriched category.
A \emph{semi-additive category} is one of the followings:
\begin{enumerate}[(i)]
\item a pointed $\CMon$-enriched category.
\item a category with finite biproducts.
\end{enumerate}
An \emph{additive category} is one of the followings:
\begin{enumerate}[(i)]
\item a pointed $\Ab$-enriched category.
\item $\Ab$-enriched category with finite biproducts.
\end{enumerate}
\begin{parts}
\item additive completion by formally adjoining finite biproducts.
\item additive structures on a semi-additive category is unique.
\end{parts}
\end{prb}

The notion of kernels and cokernels can be defined in a $\Set_*$-enriched category.
In additive category, we have a natural $\Set_*$-enrichment.


\begin{prb}[Pre-abelian categories]
A \emph{pre-abelian category} is one of the followings:
\begin{enumerate}[(i)]
\item an additive category in which every morphism has the kernel and cokernel.
\item a finitely bicomplete pre-additive category.
\end{enumerate}
\begin{parts}
\item 
\end{parts}
\end{prb}

\begin{prb}[Semi-abelian categories in the sense of Jenelidze-M\'arkin-Tholen]
A pointed, Baar-exact, protomodular, with binary coproudcts.
\begin{parts}
\item short five lemma, $3\times3$ lemma, snake lemma, noehter isomorphism hold.
\item long exact homology sequence
\item Every semi-abelian category is exact.
\item Every semi-abelian category is finitely bicomplete.
\item In general, a semi-abelian category is not pre-additive nor semi-additive.
\end{parts}
\end{prb}

\begin{prb}[Abelian categories]
An \emph{abelian category} is a $\Ab$-enriched category which is finitely bicomplete and satisfies the first isomorphism theorem.

PULLBACK/PUSHOUT, ZERO, EXACT!
\begin{parts}
\item A category is abelian if and only if it is additive and exact.
\end{parts}
\end{prb}


\begin{prb}[Freyd-Mitchell embedding]
\end{prb}


\[\begin{tikzcd}[math mode=false]
abelian\ar{r}\ar{d} & pre-abelian \ar{r} & additive \ar{r}\ar{d} & pre-additive\\
semi-abelian (JMT) && semi-additive & \,
\end{tikzcd}\]
\begin{itemize}
\item Pre-abelian: abelian topological groups, Banach spaces, Fr\'echet spaces.
\item Semi-abelian: groups, non-unital algebras, Lie algebras, C$^*$-algebras, compact Hausdorff (profinite) spaces.
\item Additive: projective modules
\end{itemize}





The first isomorphism theorem states that $\coim\to\im$ is an isomorphism.
The normal subobjects and the first isomorphism theorem is generalized in the context of protomodular categories.
The cokernel may not be defined.
The category of unital rings is not semi-abelian but protomodular.
\begin{itemize}
\item A \emph{protomodular category}
\item A \emph{homological category} is a pointed regular protomodular category. (five, nine, snake, long exact sequence, noether isomorphism)
\item A \emph{semi-abelian category} is a homological category that is Barr-exact and finite coproducts(free products).
\end{itemize}




\section{}

How to manipulate triangles and squares!
We use the following definitions: a triangle is a complex of two morphisms, and a square is a commutative square.
A square is just a triangle $A\to B\oplus C\to D$.

taking kernels

taking pullbacks

commutativity and exactness are stable under epi-mono factorization.

\[\begin{tikzcd}[sep=small]
&& B\rar\dar & C \rar & 0\\
0 \rar & A' \rar & B' &&
\end{tikzcd}\]

\[\begin{tikzcd}[sep=small]
&&& KC \dar &\\
&& B\rar\dar & C \rar & 0\\
0 \rar & A' \rar & B' &&
\end{tikzcd}\]



\chapter{Tensor categories}



\section{Monoidal categories}
closed, symmetric, cartesian
coherence theorem, closure theorem


\begin{prb}[Monoidal categories]
A \emph{monoidal category} is a category $\cC$ equipped with a bifunctor $\otimes:\cC\times\cC\to\cC$ such that
\begin{enumerate}[(i)]
\item for each triple $A,B,C\in\cC$ there is an isomorphism $\alpha_{A,B,C}:(A\otimes B)\otimes C\to A\otimes(B\otimes C)$ called the \emph{associator}, satisfying the pentagon identity
\[\begin{tikzcd}
((A\otimes B)\otimes C)\otimes D) \rar{\alpha_{A\otimes B,C,D}} \dar{\alpha_{A,B,C}\otimes\id_D} &
(A\otimes B)\otimes(C\otimes D) \rar{\alpha_{A,B,C\otimes D}} &
A\otimes(B\otimes(C\otimes D))\\
(A\otimes(B\otimes C))\otimes D \ar{rr}{\alpha_{A,B\otimes C,D}} &&
A\otimes((B\otimes C)\otimes D) \uar{\id_A\otimes\alpha_{B,C,D}}
\end{tikzcd}\]
commutes for each $A,B,C,D\in\cC$.
\item there is a specified object $I\in\cC$ called the \emph{unit object}, and for each $A\in\cC$ there are isomorphisms $\lambda_A:I\otimes A\to A$ and $\rho_A:A\otimes I\to A$ called the \emph{left unitor} and the \emph{right unitor}, satisfying the triangle identity
\[\begin{tikzcd}[column sep=tiny]
(A\otimes I)\otimes B \ar{rr}{\alpha_{A,I,B}} \ar[swap]{dr}{\rho_A\otimes\id_B} &&
A\otimes(I\otimes B) \ar{dl}{\id_A\otimes\lambda_B}\\
& A\otimes B &
\end{tikzcd}\]
commutes for each $A,B\in\cC$.
\end{enumerate}
We say a monoidal category is \emph{strict} if the associators and unitors are all identity morphisms.
A \emph{cartesian} monoidal category is a monoidal category whose monoidal structure $\otimes$ is given by the categorical product.
\end{prb}


\begin{prb}[Coherence theorem]
Let $\cC$ be a monoidal category.
\begin{parts}
\item
\[\begin{tikzcd}[column sep=tiny]
(I\otimes A)\otimes B \ar{rr}{\alpha_{I,A,B}} \ar[swap]{dr}{\lambda_A\otimes\id_B} &&
I\otimes(A\otimes B) \ar{dl}{\lambda_{A\otimes B}}\\
& A\otimes B &
\end{tikzcd}\]
\item $\lambda_I=\rho_I$
\item The endomorphism monoid $\End(I)$ is commutative.
\item $I$ is unique up to unique isomoprhism.
\end{parts}
\end{prb}


\begin{prb}[Monoidal functors]
coherence maps
lax, strong, strict
\end{prb}




\begin{prb}[Enriched categories]
Let $\cM$ be a monoidal category.
A category $\cC$ is said to be \emph{enriched} over $\cM$ if for each $A,B\in\cC$ there is $\Hom(A,B)\in\cM$ such that
\begin{enumerate}[(i)]
\item for each $A,B,C\in\cC$ there is a morphism $\circ_{A,B,C}:\Hom(A,B)\otimes\Hom(B,C)\to\Hom(A,C)$, satisfying
\[\begin{tikzcd}[column sep=huge]
\Hom(A,B)\otimes\Hom(B,C)\otimes\Hom(C,D) \rar{\id_{\Hom(A,B)}\otimes\circ_{B,C,D}} \dar[swap]{\circ_{A,B,C}\otimes\id_{\Hom(C,D)}} &
\Hom(A,B)\otimes\Hom(B,D) \dar{\circ_{A,B,D}} \\
\Hom(A,C)\otimes\Hom(C,D) \rar{\circ_{A,C,D}} &
\Hom(A,D)
\end{tikzcd}\]
commutes for each $A,B,C,D\in\cC$.
\item for each $A\in\cC$ there is a morphism $\id_A:I\to\Hom(A,A)$, satisfying
\[\begin{tikzcd}
I\otimes\Hom(A,B) \dar[swap]{\id_A\otimes\id_{\Hom_{A,B}}} \ar{dr}{\lambda_{\Hom(A,B)}} &&
\Hom(A,B)\otimes I \ar[swap]{dl}{\rho_{\Hom(A,B)}} \dar{\id_{\Hom(A,B)}\otimes\id_B}\\
\Hom(A,A)\otimes\Hom(A,B) \rar[swap]{\circ_{A,A,B}} &
\Hom(A,B)  &
\Hom(A,B)\otimes\Hom(B,B) \lar{\circ_{A,B,B}}
\end{tikzcd}\]
\end{enumerate}
\end{prb}

\begin{prb}[Pointed category]
A \emph{pointed category} is a category with a zero object.
\begin{parts}
\item A category is $\Set_*$-enriched if and only if it admits a zero morphism.
\item Every pointed category is $\Set_*$-enriched.
\end{parts}
\end{prb}

rigid?

\section{Braided and ribbon categories}





\section{Internalization}


\section{Tensor and fusion categories}









\end{document}