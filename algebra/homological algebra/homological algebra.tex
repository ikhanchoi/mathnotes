\documentclass{../../large}
\usepackage{../../ikhanchoi}

\newcommand{\Cof}{\mathrm{Cof}}
\newcommand{\Fib}{\mathrm{Fib}}

\begin{document}
\title{Homological Algebra}
\author{Ikhan Choi}
\maketitle
\tableofcontents







\part{Derived categories}

\chapter{Derived functors}



\chapter{Differential graded categories}

\section{Chain complexes}

\begin{prb}
Let $\cA$ and $\cB$ be abelian categories and suppose $\cA$ has enough injectives, that is, every object $A\in\cA$ admits a monomorphism $A\to I$ for an injective object $I$.
Let $\cF:\cA\to\cB$ be a left-exact functor.

\end{prb}


derived category of differential graded category.



\section{Triangulated categories}


\begin{prb}[Triangulated categories]
A \emph{triangulated category} is an additive functor $\cD$ together with a translation functor $\cD\to\cD:X\mapsto X[1]$, which is an equivalence of categories, and a collection of distinguished triangles 

\end{prb}




\chapter{}








\part{Homotopical algebra}

\chapter{Model categories}

\section{Model structures}

\begin{prb}[Model structures]
Let $\cC$ be a category.
We say a pair $(\cA,\cB)$ of subcategories of $\cC$ is a \emph{functorial weak factorization system} if
\begin{enumerate}[(i)]
\item for $i\in\Mor(\cA)$ and $p\in\Mor(\cB)$ there exists $h\in\Mor(\cC)$ such that the following commutes:
\[\begin{tikzcd}
A \dar[>->,swap]{i} \rar{f} & X \dar[->>]{p} \\
B \rar[swap]{g} \ar[dashed]{ur}{h} & Y.
\end{tikzcd}\tag{lifting}\]
\item there are functors $\alpha:\cC\to\cA$ and $\beta:\cC\to\cB$ such that for $f\in\Mor(\cC)$ the following commutes:
\[\begin{tikzcd}[sep=small]
A \ar{rr}{f} \ar[swap]{dr}{\alpha(f)} && B \\
& M \ar[swap]{ur}{\beta(f)} &
\end{tikzcd}\tag{factorization}\]
\end{enumerate}



Following the definition of Hovey, a \emph{model structure} on $\cC$ is a three subcategories of $\cC$ called \emph{weak equivalences}, \emph{cofibrations}, and \emph{fibrations} such that
\begin{enumerate}[(i)]
\item the weak equivalences satisfy the two-out-of-three law,
\item cofibrations and acyclic fibrations form a functorial weak factorization system,
\item acyclic cofibrations and fibrations form a functorial weak factorization system.
\end{enumerate}
We denote by $\cW$ the subcategory of weak equivalences.
A \emph{model category} is a category with small limits and colimits equipped with a model structure.

cofibrant and fibrant replacements.

\begin{parts}
\item retract closedness
\item
\end{parts}
\end{prb}










\begin{prb}[Homotopy category of a model category]
left homotopy and right homotopy, cofibrant-fibrant objects.
\end{prb}

\begin{prb}[Derived categories and derived functors]
For a functor $F:\cC\to\cD$ between model categories, a \emph{left derived functor} $LF:h\cC\to\cD$ is defined as the right Kan extension of $F$ with respect to $\cC\to h\cC$.
It may not exist in general, but there is an equivalent condition which can be easily investigated.
\end{prb}




\section{Quillen functors}



\section{Examples of model structures}

\begin{prb}[Model structures on chain complexes]
projective and Hurewicz(chain homotopy) model structures on non-negative chain complexes
\end{prb}

\begin{prb}[Model structures on topological spaces]
Serre and Hurewicz model structures
\end{prb}




monoidal, simplicial, pointed, stable model categories



\chapter{Simplicial categories}

\section{Simplicial sets}

Simplicial methods convert a differential graded category to a simplicial category via the Dold-Kan correspondence, so that a model structure on the differential graded category becomes simplicial.
In a simplicial model category we can expand simplicial resolutions explicitly.

\begin{prb}[Simplicial sets]
\end{prb}

\begin{prb}[Simplicial complexes]
A \emph{simplicial complex} is a set $K$ of non-empty finite subsets of a set $V$ which is closed under subsets.
If $V$ is linearly ordered, then we say $K$ is ordered.
To every ordered simplicial complex one can associate a simplicial setas follows.
Let $K_n$ be the set of all ordered tuples $(v_0,\cdots,v_n)$ such that $v_0\le\cdots\le v_n$ and $\{v_0,\cdots,v_n\}\in K$.
Then, for each morphism $\alpha:[n]\to[m]$ in $\Delta$, we can define $\alpha^*:K_m\to K_n$.

\end{prb}


\begin{prb}[Dold-Kan correspondence]
The Dold-Kan correspondence states that $\cA_\Delta\to\Ch_{\ge0}(\cA)$ is a categorical equivalence for an abelian category $\cA$.

Two descriptions for normalized Moore complexes:
\[0\to N_\bullet(A)\to C_\bullet(A)\to D_\bullet(A)\to0.\]


Eilenberg-Maclane functor $K:\mathrm{Ch}(\Z)\to\mathrm{sAb}$ as the right adjoint for the functor $N_\bullet$.
\end{prb}



\[\begin{tikzcd}
\mathrm{Top} \rar{\mathrm{Sing}} &
\mathrm{sSet} \rar{R[\cdot]} &
\mathrm{sMod}_R \rar{C_\bullet\text{ or }N_\bullet} &
\mathrm{Ch}(R) \rar{H_n} &
\mathrm{Mod}_R
\end{tikzcd}\]




\section{Simplicial model categories}
\begin{prb}[Model structures on simplicial sets]
Kan and Joyal model structures


Via the Dold-Kan correspondence $\cA_\Delta\cong\Ch_{\ge0}(\cA)$, the Kan model structure corresponds to the projective model structures
\end{prb}




\chapter{Infinity categories}
\section{Simplicial sets}

Two representative examples: nerves and Kan complexes

infinity categories as simplicially enriched categories

\begin{prb}[Nerves]
For an ordinary category as a nerve, two morphisms are homotopic only if they are identical.
\end{prb}

\begin{prb}[Kan complexes]
A geometric model for infinity groupoids.
In a Kan complex, including Sing of a topological space, every morphism is invertible up to homotopy.

Infinity groupoids are usually considered as ``spaces''.
\end{prb}




\section{Kan complexes}

The \emph{infinity category of spaces}, denoted by $\mathrm{Spc}$, is defined as the homotopy-coherent nerve of the category $\mathrm{Kan}$ of Kan complexes.

\section{Stable infinity categories}

examples of stable infinity category: the infinity category of spectra, the dervied category of an abelian category

\begin{prb}
A \emph{stable infinity category} is an infinity category such that
\begin{enumerate}[(i)]
\item there is a zero object,
\item every morphism admits a fiber and cofiber,
\item a triangle is a fiber sequence if and only if it is a cofiber sequence.
\end{enumerate}
It is known that its homotopy category is tricngulated.
\end{prb}

\begin{prb}[Triangulated categories]
\end{prb}

\begin{prb}[Differential graded category]
\end{prb}






\end{document}













제가 
하지만 만약 이런 좋은 환경에서 공부하지 못했더라면 이렇게 빨리 회의감을 느끼는 날이 오지는 못했을 것 같다는 생각이 듧니다.
저는 다시 묻습니다. 무엇을 공부의 원동력으로 삼아야 할까요?
과거의 제가 남을 이기기 위해 공부하기 시작했다면, 이제는 


뛰어난 사람들과 뒤처지는 사람들을 나와 비교하지 않게 되고 싶다.
많은 사람들을 사랑할 수 있게 되고 싶다.
싫은 소리를 하는 방법에 대해 알게 되고 싶다.
인생의 유의미하다고 생각되는 일을 같이 해나갈 소중한 사람들을 만나고 싶다.
내 스스로가 어떤 위치에 있는지 자꾸 자기검열을 하게 된다.
