\documentclass{../../large}
\usepackage{../../ikhanchoi}

\newcommand{\Cof}{\mathrm{Cof}}
\newcommand{\Fib}{\mathrm{Fib}}

\begin{document}
\title{Homological Algebra}
\author{Ikhan Choi}
\maketitle
\tableofcontents


\part{Abelian categories}





\chapter{Category of modules}

A left $R$-module $P$ is projective if and only if the left exact functor $\Hom_R(P,-)$ is exact.

A left $R$-module $I$ is injective if and only if the left exact contravariant functor $\Hom_R(-,I)$ is exact.


projective
\begin{itemize}
\item direct sum of projectives is projective
    \\(lem) if free, then projective
\item PID: projective iff free (note sub of free is free in PID)
\item projective iff direct summand of a free
\item every module is a quotient of a free module
\end{itemize}

injective
\begin{itemize}
\item direct product of injectives is injective
    \\(lem) $M$ injective iff $\Hom_R(R,M)\to\Hom_R(I,M)$ surj
\item PID: injective iff divisible ($\cdots a:M\to M$ surj)
    \\(lem) $\Hom_Z(R,M)$ is injective if $M$ is injective $\Z$-module
\item every module is embedded in injective
\end{itemize}

flat
\begin{itemize}
\item PID: flat iff ($\cdot a:M\to M$ inj)
\item $M$ flat iff $\Hom(M,\Q/\Z)$ is injective
\item $M$ flat iff $I\otimes M\to R\otimes M$ inj
\item if projective, then flat
\end{itemize}

continuity of functors

\begin{prb}[Tor functor]
Let $R$ be a ring and $M$ be a left $R$-module.
We define the \emph{Tor functor} as the left derived functor of the right exact functor $-\otimes_RM:\Mod\text{-}R\to\Ab$
\[\Tor_n^R(N,M):=H_n(P_\bullet\otimes_RM),\]
where $P_\bullet$ is a projective resolution of a right $R$-module $N$.
\begin{parts}
\item In fact, the Tor functor may be defined by the left derived functor of the right exact functor $M\otimes_R-:R\text{-}\Mod\to\Ab$ for a right $R$-module $M$.
\item In fact, only for Tor functors, we may only assume $P_\bullet$ is a flat resolution. (Flat resolution lemma)
\end{parts}
\end{prb}

\begin{prb}[Ext functor]
Let $R$ be a ring and $M$ be a left $R$-module.
We define the \emph{Ext functor} as the right derived functor of left exact functor $\Hom_R(M,-)$
\[\Ext_R^n(M,N):=H^n(M,I^\bullet),\]
where $I^\bullet$ is an injective resolution of $N$.
\begin{parts}
\item In fact, the Ext functor may be defined by the right derived functor of the left exact contravariant functor $\Hom(-,M)$.
\end{parts}
\end{prb}

long exact seuqence


\begin{prb}[Universal coefficient theorem]
Let $R$ be a ring.
Let $C_\bullet$ be a chain complex of flat right $R$-modules and $M$ be a left $R$-module.
\end{prb}
\begin{pf}
We first prove the K\"unneth formula.
Note that modules in $Z_\bullet$ and $B_\bullet$ are also flat.
We start from that we have a short exact sequence of chain complexes
\[0\to Z_\bullet\to C_\bullet\to B_{\bullet-1}\to0.\]
We have a short exact sequence of chain complexes
\[\Tor_1^R(B_{\bullet-1},M)\to Z_\bullet\otimes_RM\to C_\bullet\otimes_RM\to B_{\bullet-1}\otimes_RM\to0.\]
Since modules in $B_{\bullet-1}$ are flat so that $\Tor_1^R(B_{\bullet-1},M)=0$, we have a short exact sequence of chain complexes
\[0\to Z_\bullet\otimes_RM\to C_\bullet\otimes_RM\to B_{\bullet-1}\otimes_RM\to0.\]
Since $H_n(C_{\bullet-1})=H_{n-1}(C_\bullet)$ for any chain complex $C$, we have a long exact sequence
\[H_n(B_\bullet\otimes_RM)\to H_n(Z_\bullet\otimes_RM)\to H_n(C_\bullet\otimes_RM)\to H_{n-1}(B_\bullet\otimes_RM)\to H_{n-1}(Z_\bullet\otimes_RM).\]
Since every morphism in $B_\bullet$ and $Z_\bullet$ is zero, we have an exact sequence
\[B_n\otimes_RM\xrightarrow{f_n}Z_n\otimes_RM\to H_n(C_\bullet\otimes_RM)\to B_{n-1}\otimes_RM\xrightarrow{f_{n-1}}Z_{n-1}\otimes_RM.\]
Therefore, we have a short exact sequence
\[0\to\coker f_n\to H_n(C_\bullet\otimes_RM)\to\ker f_{n-1}\to0.\]

Since
\[0\to B_n\to Z_n\to H_n(C_\bullet)\to0\]
is a flat resolution of $H_n(C_\bullet)$, by the flat resolution lemma, we have a long exact sequence
\[\Tor_1^R(Z_n,M)\to\Tor_1^R(H_n(C_\bullet),M)\to B_n\otimes_RM\xrightarrow{f_n}Z_n\otimes_RM\to H_n(C_\bullet)\otimes_RM\to0.\]
Since $Z_n$ is flat so that $\Tor_1^R(Z_n,M)=0$, we have
\[\coker f_n=H_n(C_\bullet)\otimes_RM,\quad\ker f_n=\Tor_1^R(H_n(C_\bullet),M).\]
Therefore, we have an exact sequence
\[0\to H_n(C_\bullet)\otimes_RM\to H_n(C_\bullet\otimes_RM)\to\Tor_1^R(H_{n-1}(C_\bullet),M)\to0.\]

Universal coefficient theorem states that if $R$ is a PID, then the K\"unneth formula splits non-canonically.
\end{pf}




\chapter{}

\[\begin{tikzcd}[sep=small]
K \rar & A \rar\dar[->>] & B \rar\dar & 0\\
K' \rar & A' \rar & B' \rar & 0\\
\end{tikzcd}\]
\begin{parts}
\item If $A\to A'$ is monic, then $K\to K'$ is monic.
\item If $B\to B'$ is monic, then $K\to K'$ is epic.
\end{parts}


\chapter{Cohomology of algberas}

\section{Group cohomology}

The category of $G$-modules can be identified with the category of $\Z[G]$-modules, which is abelian.


Let $M$ be a $G$-module.
The \emph{invariant submodule} of $M$ is denoted by $M^G$.
Sending $M$ to $M^G$ yields a functor $\mathrm{Grp}\to\mathrm{Ab}$, which is left exact but not right exact in general.
Then we can consider the right derived functor to define cohomology groups.
Let us do this concretely.


Let $M$ be a $G$-module.
Define $C^n(G,M)$ be the abelian group of all functions $G^n\to M$.
The coboundary homomorphism $d:C^n(G,M)\to C^{n+1}(G,M)$ is defined such that
\[d\f(g_1,\cdots,g_{n+1}):=g_1\f(g_2,\cdots,g_{n+1})+\sum_{i=1}^n(-1)^i\f(g_1,\cdots,g_{i-1},g_ig_{i+1},g_{i+2},\cdots,g_{n+1})+(-1)^{n+1}\f(g_1,\cdots,g_n).\]

\[H^0(G,M)=M^G=\Hom_{\Z[G]}(\Z,M).\]
For $x\in C^0(G,M)=M$, $dx(g)=gx-x$.
For $\f\in C^1(G,M)$, $d\f(g,h)=g\f(h)-\f(gh)+\f(g)$.



\part{Derived categories}

\chapter{Derived categories}

\section{Differential graded categories}



\begin{prb}
Let $\cA$ and $\cB$ be abelian categories and suppose $\cA$ has enough injectives, that is, every object $A\in\cA$ admits a monomorphism $A\to I$ for an injective object $I$.
Let $\cF:\cA\to\cB$ be a left-exact functor.

\end{prb}


derived category of differential graded category.



\section{Triangulated categories}


\begin{prb}[Triangulated categories]
A \emph{triangulated category} is an additive functor $\cD$ together with a translation functor $\cD\to\cD:X\mapsto X[1]$, which is an equivalence of categories, and a collection of distinguished triangles 

\end{prb}






\part{Homotopical algebra}

\chapter{Model categories}


\begin{prb}[Model structures]
Let $\cC$ be a category.
Following the definition of Hovey, a \emph{model structure} on $\cC$ is a three subcategories of $\cC$ called \emph{weak equivalences}, \emph{cofibrations}, and \emph{fibrations} such that
\begin{enumerate}[(i)]
\item the weak equivalences satisfy the two-out-of-three law,
\item cofibrations and acyclic fibrations form a functorial weak factorization system,
\item acyclic cofibrations and fibrations form a functorial weak factorization system.
\end{enumerate}
We denote by $\cW$ the subcategory of weak equivalences is denoted by.
\begin{parts}
\item retract closedness
\item
\end{parts}
\end{prb}


Serre model structure and Hurewicz model structure on $\mathrm{Top}$.
















\chapter{Infinity categories}
\section{Simplicial sets}

Two representative examples: nerves and Kan complexes

infinity categories as simplicially enriched categories

\begin{prb}[Nerves]
For an ordinary category as a nerve, two morphisms are homotopic only if they are identical.
\end{prb}

\begin{prb}[Kan complexes]
A geometric model for infinity groupoids.
In a Kan complex, including Sing of a topological space, every morphism is invertible up to homotopy.

Infinity groupoids are usually considered as ``spaces''.
\end{prb}

\begin{prb}[Dold-Kan correspondence]
\[\begin{tikzcd}
\mathrm{Top} \rar{\mathrm{Sing}} &
\mathrm{sSet} \rar{\Z[\cdot]} &
\mathrm{sAb} \rar{C_\bullet\text{ or }N_\bullet} &
\mathrm{Ch}(\Z) \rar{H_n} &
\mathrm{Ab}
\end{tikzcd}\]

Two descriptions for normalized Moore complexes:
\[0\to N_\bullet(A)\to C_\bullet(A)\to D_\bullet(A)\to0.\]

Eilenberg-Maclane functor $K:\mathrm{Ch}(\Z)\to\mathrm{sAb}$ as the right adjoint for the functor $N_\bullet$.
\end{prb}





\section{Kan complexes}

The \emph{infinity category of spaces}, denoted by $\mathrm{Spc}$, is defined as the homotopy-coherent nerve of the category $\mathrm{Kan}$ of Kan complexes.

\section{Stable infinity categories}

examples of stable infinity category: the infinity category of spectra, the dervied category of an abelian category

\begin{prb}
A \emph{stable infinity category} is an infinity category such that
\begin{enumerate}[(i)]
\item there is a zero object,
\item every morphism admits a fiber and cofiber,
\item a triangle is a fiber sequence if and only if it is a cofiber sequence.
\end{enumerate}
It is known that its homotopy category is tricngulated.
\end{prb}

\begin{prb}[Triangulated categories]
\end{prb}

\begin{prb}[Differential graded category]
\end{prb}






\end{document}













제가 
하지만 만약 이런 좋은 환경에서 공부하지 못했더라면 이렇게 빨리 회의감을 느끼는 날이 오지는 못했을 것 같다는 생각이 듧니다.
저는 다시 묻습니다. 무엇을 공부의 원동력으로 삼아야 할까요?
과거의 제가 남을 이기기 위해 공부하기 시작했다면, 이제는 


뛰어난 사람들과 뒤처지는 사람들을 나와 비교하지 않게 되고 싶다.
많은 사람들을 사랑할 수 있게 되고 싶다.
싫은 소리를 하는 방법에 대해 알게 되고 싶다.
인생의 유의미하다고 생각되는 일을 같이 해나갈 소중한 사람들을 만나고 싶다.
내 스스로가 어떤 위치에 있는지 자꾸 자기검열을 하게 된다.
