\documentclass{../../large}
\usepackage{../../ikhanchoi}

\newcommand{\Ab}{\mathbf{Ab}}
\newcommand{\Mod}{\mathbf{Mod}}

\begin{document}
\title{Homological Algebra}
\author{Ikhan Choi}
\maketitle
\tableofcontents


\part{}

\chapter{}

A left $R$-module $P$ is projective if and only if the left exact functor $\Hom_R(P,-)$ is exact.

A left $R$-module $I$ is injective if and only if the left exact contravariant functor $\Hom_R(-,I)$ is exact.


\begin{prb}[Tor functor]
Let $R$ be a ring and $M$ be a left $R$-module.
We define the \emph{Tor functor} as the left derived functor of the right exact functor $-\otimes_RM:\Mod\text{-}R\to\Ab$
\[\Tor_n^R(N,M):=H_n(P_\bullet\otimes_RM),\]
where $P_\bullet$ is a projective resolution of a right $R$-module $N$.
\begin{parts}
\item In fact, the Tor functor may be defined by the left derived functor of the right exact functor $M\otimes_R-:R\text{-}\Mod\to\Ab$ for a right $R$-module $M$.
\item In fact, only for Tor functors, we may only assume $P_\bullet$ is a flat resolution. (Flat resolution lemma)
\end{parts}
\end{prb}

\begin{prb}[Ext functor]
Let $R$ be a ring and $M$ be a left $R$-module.
We define the \emph{Ext functor} as the right derived functor of left exact functor $\Hom_R(M,-)$
\[\Ext_R^n(M,N):=H^n(M,I^\bullet),\]
where $I^\bullet$ is an injective resolution of $N$.
\begin{parts}
\item In fact, the Ext functor may be defined by the right derived functor of the left exact contravariant functor $\Hom(-,M)$.
\end{parts}
\end{prb}

long exact seuqence


\begin{prb}[Universal coefficient theorem]
Let $R$ be a ring.
Let $C_\bullet$ be a chain complex of flat right $R$-modules and $M$ be a left $R$-module.
\end{prb}
\begin{pf}
We first prove the K\"unneth formula.
Note that modules in $Z_\bullet$ and $B_\bullet$ are also flat.
We start from that we have a short exact sequence of chain complexes
\[0\to Z_\bullet\to C_\bullet\to B_{\bullet-1}\to0.\]
We have a short exact sequence of chain complexes
\[\Tor_1^R(B_{\bullet-1},M)\to Z_\bullet\otimes_RM\to C_\bullet\otimes_RM\to B_{\bullet-1}\otimes_RM\to0.\]
Since modules in $B_{\bullet-1}$ are flat so that $\Tor_1^R(B_{\bullet-1},M)=0$, we have a short exact sequence of chain complexes
\[0\to Z_\bullet\otimes_RM\to C_\bullet\otimes_RM\to B_{\bullet-1}\otimes_RM\to0.\]
Since $H_n(C_{\bullet-1})=H_{n-1}(C_\bullet)$ for any chain complex $C$, we have a long exact sequence
\[H_n(B_\bullet\otimes_RM)\to H_n(Z_\bullet\otimes_RM)\to H_n(C_\bullet\otimes_RM)\to H_{n-1}(B_\bullet\otimes_RM)\to H_{n-1}(Z_\bullet\otimes_RM).\]
Since every morphism in $B_\bullet$ and $Z_\bullet$ is zero, we have an exact sequence
\[B_n\otimes_RM\xrightarrow{f_n}Z_n\otimes_RM\to H_n(C_\bullet\otimes_RM)\to B_{n-1}\otimes_RM\xrightarrow{f_{n-1}}Z_{n-1}\otimes_RM.\]
Therefore, we have a short exact sequence
\[0\to\coker f_n\to H_n(C_\bullet\otimes_RM)\to\ker f_{n-1}\to0.\]

Since
\[0\to B_n\to Z_n\to H_n(C_\bullet)\to0\]
is a flat resolution of $H_n(C_\bullet)$, by the flat resolution lemma, we have a long exact sequence
\[\Tor_1^R(Z_n,M)\to\Tor_1^R(H_n(C_\bullet),M)\to B_n\otimes_RM\xrightarrow{f_n}Z_n\otimes_RM\to H_n(C_\bullet)\otimes_RM\to0.\]
Since $Z_n$ is flat so that $\Tor_1^R(Z_n,M)=0$, we have
\[\coker f_n=H_n(C_\bullet)\otimes_RM,\quad\ker f_n=\Tor_1^R(H_n(C_\bullet),M).\]
Therefore, we have an exact sequence
\[0\to H_n(C_\bullet)\otimes_RM\to H_n(C_\bullet\otimes_RM)\to\Tor_1^R(H_{n-1}(C_\bullet),M)\to0.\]

Universal coefficient theorem states that if $R$ is a PID, then the K\"unneth formula splits non-canonically.
\end{pf}







\end{document}



projective
  direct sum of projectives is projective
  	(lem) free => projective
  PID: projective iff free (note sub of free is free in PID)
  projective iff direct summand of a free
  every module is a quotient of a free module

injective
  direct product of injectives is injective
  	(lem) M injective iff Hom_R(R,M)->Hom_R(I,M) surj
  PID: injective iff divisible (.a:M->M surj)
  	(lem) Hom_Z(R,M) is injective if M is injective Z-module
  every module is embedded in injective

flat
  PID: flat iff (.a:M->M inj)
  M flat iff Hom(M,Q/Z) is injective
  M flat iff I⊗M->R⊗M inj
  if projective then flat

continuity of functors