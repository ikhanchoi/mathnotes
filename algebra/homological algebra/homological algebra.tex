\documentclass{../../large}
\usepackage{../../ikhanchoi}

\newcommand{\Cof}{\mathrm{Cof}}
\newcommand{\Fib}{\mathrm{Fib}}
\newcommand{\we}{\overset\sim\to}
\newcommand{\cof}{\rightarrowtail}
\newcommand{\fib}{\twoheadrightarrow}
\newcommand{\tcof}{\overset\sim\rightarrowtail}
\newcommand{\tfib}{\overset\sim\twoheadrightarrow}


\begin{document}
\title{Homological Algebra}
\author{Ikhan Choi}
\maketitle
\tableofcontents







\part{Derived categories}

\chapter{Derived functors}



\chapter{Differential graded categories}

\section{Chain complexes}



\begin{prb}[Differential graded objects]
Let $\cA$ be an abelian category.
A \emph{complex} over $\cA$ is a differential graded object


A differential graded abelian group is traditionally called a \emph{chain complex}.

The following categories are all abelian? projectives and injectives?
We have six categories
$\mathrm{Ch}_{\ge0}(\cA)$ and $\mathrm{Ch}_+(\cA)$, $\mathrm{Ch}^{\ge0}(\cA)$ and $\mathrm{Ch}^+(\cA)$, $\mathrm{Ch}(\cA)$ and $\mathrm{Ch}_b(\cA)$.

\begin{parts}
\item $\mathrm{Ch}_{\ge0}(\cA)$ is abelian.
\end{parts}
\end{prb}

\begin{pf}
Let $A,B\in\mathrm{Ch}_{\ge0}(\cA)$.
The cokernel is given as follows.
\end{pf}




\begin{prb}[Differential graded algbera]
Let $\cA$ be a symmetric monoidal abelian category.
A \emph{differential graded algebra} over $\cA$ is a monoid object in the category of complexes over $\cA$.
For a commutative unital ring $R$, then a differential graded algebra over $\Mod_R$ can be characterized by 

Let $A\in\cA$ be a differential graded algebra.
Then, we can consider complexes over $A$-modules.
\end{prb}



derived category of differential graded category.



\begin{prb}
Let $\cA$ and $\cB$ be abelian categories and suppose $\cA$ has enough injectives, that is, every object $A\in\cA$ admits a monomorphism $A\to I$ for an injective object $I$.
Let $\cF:\cA\to\cB$ be a left-exact functor.

\end{prb}



\section{Triangulated categories}


\begin{prb}[Triangulated categories]
A \emph{triangulated category} is an additive functor $\cD$ together with a translation functor $\cD\to\cD:X\mapsto X[1]$, which is an equivalence of categories, and a collection of distinguished triangles 

\end{prb}




\chapter{}








\part{Homotopical algebra}

\chapter{Model categories}

\section{Model structures}


Dashed arrows mean that they do not commute with solid arrows in general, and bent arrows mean that they are constructed by the property of projective modules.


\begin{prb}[Model structures]
Following the definition of Hovey, a \emph{model structure} on a category $\cM$ is a triple $(\cW,\cC,\cF)$ of subcategories of $\cM$ whose morphisms are called \emph{weak equivalences}, \emph{cofibrations}, and \emph{fibrations} respectively, and a weak equivalence that is also a cofibration or a fibration is called a \emph{} such that
\begin{enumerate}[(i)]
\item (two-out-of-three)
$\cW$,
\item (retract)
\item (lift)
for a commutative square consisting of morphisms $f,g,i,p$ in $\cM$ such that $pf=gi$, if $i$ is a cofibration and $p$ are in $\cC$ and $\cF\cap\cW$, or in $\cC\cap\cW$ and $\cF$, respectively, then there exists $h\in\Mor(\cM)$ such that the following diagram commutes:
\[\begin{tikzcd}
A \dar[swap]{i} \rar{f} & Y \dar{p} \\
X \rar[swap]{g} \ar[dashed]{ur}{\exists h} & B
\end{tikzcd}\qquad\text{ in }\cM.\]

\item (factorization)

\[\begin{tikzcd}[sep=small]
A \ar{rr}{f} \ar[>->,dashed,swap]{dr}{\exists i} && B\\
& C \ar[->>,dashed,swap]{ur}{\exists p} &
\end{tikzcd}\qquad\text{ in }\cM\]
\end{enumerate}
A \emph{model category} is a bicomplete category equipped with a model structure.

\begin{parts}
\item opposite category and overcategory
\item lifting property and retract closedness
\item base change
\end{parts}
\end{prb}
\begin{pf}
To show $\cC$ is closed under retracts (in functorial weak factorization system generally), suppose the horizontal composites $ACA$ are the identities, $CD\in\cC$, and $XY\in\cF$ in the comutative diagram
\[\begin{tikzcd}[sep=small]
A\rar\dar&C\rar\dar[>->]&A\rar\dar&X\dar[->>]\\
B\rar&D\rar&B\rar&Y.
\end{tikzcd}\]
Then, we have the lift $DX$, and $BDX$ defines a lift $BX$, so $AB\in\cC$.

\end{pf}


\begin{prb}
Let $\cM$ be a model category.
An object $X$ is called \emph{cofibrant} if the unique morphism $0\to X$ from the initial object is a cofibration.

The \emph{cofibrant replacement} is the functor $Q:\cM\to\cM$ defined by the functorial factorization (if exists) such that $0\cof QX\tfib X$ for all $X\in\cM$.

An object $Y$ is called \emph{fibrant} if

For a pointed or unpointed space $X$, we always have a fibrant replacement $X\rightarrowtail CX\overset\sim\twoheadrightarrow*$
\end{prb}




\begin{prb}[Ken Brown lemma]
Let $\cC$ be a model category.
Let $\cD$ be a category with weak equivalences.
If $F:\cC\to\cD$ be a functor that takes a acyclic cofibration between cofibrant objects to a weak equivalence, then it takes a weak equivalence between cofibrant objects to a weak equivalence.
\end{prb}
\begin{pf}
Suppose $A$ and $B$ are cofibrant objects, and $AB$ is a weak equivalence.
Factorize $(A\sqcup B)B$ into a cofibration $(A\sqcup B)B'$ followed by a acyclic fibration $B'B$ as we have
\[\begin{tikzcd}[sep=small]
A \dar\ar{rrd}{\sim} &&\\
A\sqcup B \rar[>->] & B' \rar[->>]{\sim} & B.\\
B \uar\ar[equal]{rru} &&
\end{tikzcd}\]
Since cofibrations are closed under pushouts, $A(A\sqcup B)$ and $B(A\sqcup B)$ are cofibrations, so are $AB'$ and $BB'$.
On the other hand, $AB'$ and $BB'$ are weak equivalences, and hence are acyclic cofibrations.
Then, $F(AB')$ and $F(BB')$ are weak equivalences by the assumption, so $F(AB)$ is a weak equivalence.
\end{pf}



\begin{prb}[Homotopy category of a model category]
Let $\cM$ be a model category.
The \emph{cylinder object} for $X\in\cM$ is an object $X'$ such that there is a factorization
\[\begin{tikzcd}[sep=small]
X\sqcup X\rar[>->]&X'\rar[->>]{\sim}&X.
\end{tikzcd}\]
The \emph{path object} for $X\in\cM$ is an object $Y'$ such that there is a factorization
\[\begin{tikzcd}[sep=small]
Y\rar[>->]{\sim}&Y'\rar[->>]&Y\times Y.
\end{tikzcd}\]
The functorial cylinder object for $X$ and the path object for $Y$ are denoted by $X\times I$ and $Y^I$.


left homotopy and right homotopy, cofibrant-fibrant objects.
\end{prb}

\begin{prb}[Derived categories and derived functors]
For a functor $F:\cM\to\cN$ between model categories, a \emph{left derived functor} $LF:\mathrm{h}\cM\to\cN$ is defined as the right Kan extension of $F$ with respect to $\cM\to\mathrm{h}\cN$.
It may not exist in general, but there is an equivalent condition which can be easily investigated.
\end{prb}


A sequence $F\to Y\to B$ is called a \emph{fiber sequence} if it defines a pullback square
\[\begin{tikzcd}
F \rar\dar\pullback & Y \dar \\
* \rar & B
\end{tikzcd}\]


\section{Quillen functors}



\section{Examples of model categories}

\begin{prb}[Projective model structure on chain complexes]
Let $\cA$ be an abelian category with enough projective objects and $\mathrm{Ch}_{\ge0}(\cA)$ be the category of non-negatively graded chain complexes of $\cA$.
A \emph{projective model structure} on $\mathrm{Ch}_{\ge0}(\cA)$ is defined such that a chain map is 
\begin{enumerate}[(i)]
\item a weak equivalence if it is a quasi-isomorphism,
\item a cofibration if it is degree-wise monic with the projective cokernel,
\item a fibration if it is degree-wise epic at the strictly positive degrees.
\end{enumerate}
Then, they define a model structure on $\mathrm{Ch}_{\ge0}(\cA)$, and its homotopy category becomes the classical derived category.
\end{prb}
\begin{pf}
Assuming $\mathrm{Ch}_{\ge0}(\cA)$ is a bicomplete abelian category, we will check that the projective model structure is indeed a model structure.
The two-out-of-three law is clear.
We first prove $(\cC,\cF\cap\cW)$ is a weak factorization system.
Recall that 


(lift at degree zero)

Let
\[\begin{tikzcd}
A \dar[>->,swap]{i} \rar{f} & Y \dar[->>]{p}\dar[swap]{\sim} \\
X \rar[swap]{g} & B
\end{tikzcd}\qquad\text{ in }\mathrm{Ch}_{\ge0}(\cA)\]
be a commutative diagram such that $i:A\to X$ is a cofibration and $p:Y\to B$ is a acyclic fibration.

Consider the diagram at the degree zero
\[\begin{tikzcd}
A_0\rar{f_0}\dar[>->,swap]{i_0}&Y_0\dar[->>]{p_0}\\
X_0\rar[swap]{g_0}\dar[->>]&B_0\\
P_0\uar[dashed,bend left]{\exists s_0}\ar[dashed,bend left=16,shift right=1,pos=.7]{uur}{\exists t_0}&
\end{tikzcd}\qquad\text{ in }\cA,\]
where $i_0:A_0\to X_0$ is monic with projective cokernel $P_0$ by definition of the cofibration $i$, and $p_0:Y_0\to B_0$ is epic by the four lemma on the diagram
\[\begin{tikzcd}
Y_1\rar{\partial_1}\dar[->>]{p_1}&Y_0\rar\dar{p_0}&H_0(Y)\dar[>->>]{(p_0)_*} \rar&0\\
B_1\rar{\partial_1}&B_0\rar&H_0(B)\rar&0
\end{tikzcd}\qquad\text{ in }\cA,\]
where every row is exact since the zeroth homology group is the cokernel of the first boundary map, $p_1:Y_1\to B_1$ is epic since $p$ is a fibration, and $(p_0)_*:H_0(Y)\to H_0(B)$ is an isomorphism since $p$ is a weak equivalence.
By the projectivity of $P_0$, we can lift the identity $1:P_0\to P_0$ to get $s_0:P_0\to X_0$ and lift $g_0s_0:P_0\to B_0$ to get $t_0:P_0\to Y_0$.
Since $X_0$ is the coproduct of $A_0$ and $P_0$, a morphism $h_0:X_0\to Y_0$ is constructed by the universal property.
To check $h_0$ commutes with solid arrows, it suffices to the commutativity with $A_0$ and $B_0$.
Then, $h_0i_0=f_0$ is clear by definition of $h_0$, and $p_0h_0=...=g_0$.

As for degree zero, we can find $h_{n+1}':X_{n+1}\to Y_{n+1}$ such that $h_{n+1}'i_{n+1}=f_{n+1}$ and $p_{n+1}h_{n+1}'=g_{n+1}$.
For $n\ge0$, we also need to check $\partial_{n+1}h_{n+1}'=h_n\partial_{n+1}:X_{n+1}\to Y_n$, which can fail in general.
Define the error $\e:=\partial_{n+1}h_{n+1}'-h_n\partial_{n+1}$.



(factorization)

\end{pf}

\begin{prb}[Hurewicz model structure on chain complexes]
\end{prb}

\begin{prb}[]
Serre and Hurewicz model structures
\end{prb}





\section{}

monoidal, simplicial, pointed, stable model categories

If we let $*$ be the terminal object of $\cC$, then the coslice category or the overcategory $\cC_*:=*\downarrow\cC$, whose objects are morphisms $*\to B$ in $\cC$ for $B\in\cC$, is a pointed model category.
The addition of disjoint basepoint $\cC\to\cC_*$ is left adjoint to the forgetful functor.








\chapter{Simplicial categories}

\section{Simplicial sets}

Simplicial methods convert a differential graded category to a simplicial category via the Dold-Kan correspondence, so that a model structure on the differential graded category becomes simplicial.
In a simplicial model category we can expand simplicial resolutions explicitly.

\begin{prb}[Simplicial sets]
The \emph{simplex category} is the category $\Delta$ defined such that the objects are linearly ordered finite non-empty sets $[n]:=\{0,\cdots,n\}$ and the morphisms are monotone functions.
\[\begin{tikzcd}
\left[0\right] \rar[shift left=2]{d_0}\rar[shift right=2,swap]{d_1} & \left[1\right] \lar["s_0" description]
\end{tikzcd}\]

Consider
\[X_{n+1}\xrightarrow{d_j}X_n\xrightarrow{d_i}X_{n-1},\qquad X_{n+2}\xleftarrow{s_j}X_{n+1}\xleftarrow{s_i}X_n.\]
For $i\in[n]$ and $j\in[n+1]$,
\[d_id_j=\begin{cases}d_{j-1}d_i&\text{ if }j-1\ge i,\\d_jd_{i+1}&\text{ if }j<i+1,\end{cases}\qquad
s_js_i=\begin{cases}s_is_{j-1}&\text{ if }i\le j-1,\\s_{i+1}s_j&\text{ if }i+1>j,\end{cases}\]
and
\[d_js_i=\begin{cases}s_{i-1}d_j&\text{ if }i-1\ge j,\\s_id_{j-1}&\text{ if }i<j-1,\\\id&\text{ otherwise},\end{cases}\qquad
s_id_j=\begin{cases}d_js_{i+1}&\text{ if }j<i+1,\\d_{j+1}s_i&\text{ otherwise}.\end{cases}\qquad\]

A \emph{standard simplex} is a simplicial set given by a representable functor in $\Delta$, and denoted by $\Delta^n=\Delta[n]:=\Hom_\Delta(-,[n]):\Delta^\op\to\mathrm{Set}$ for each $n\ge0$.
For a simplicial set $X$, the Yoneda lemma implies $X_n:=X([n])=\Hom_{\mathrm{Set}_\Delta}(\Delta^n,X)$.



The set of homotopy classes of simplicial functions $[X,Y]$?

\end{prb}


\begin{prb}[Dold-Kan correspondence]
Let $\cA$ be an abelian category.
The Dold-Kan correspondence states that $\cA_\Delta\to\mathrm{Ch}_{\ge0}(\cA)$ is a categorical equivalence.
We also have categorical equivalence between the cosimplicial objects $\cA^\Delta$ and cochain complexes $\mathrm{Ch}^{\ge0}(\cA)$

For a simplicial object $A\in\cA_\Delta$, the \emph{normalized chain complex} or the \emph{Moore complex} of $A$ is the chain complex $NA\in\mathrm{Ch}_{\ge0}(\cA)$ defined such that $NA_n:=\bigcap_{i=1}^n\ker d_i\subset A_n$ and $\partial:=d_0$.
It is known that there is a split exact sequence
\[0\to NA\to CA\to DA\to0\]
in $\mathrm{Ch}_{\ge0}(\cA)$.


The functor $\Gamma:\mathrm{Ch}_{\ge0}(\cA)\to\cA_\Delta$.

The \emph{Eilenberg-MacLane functor} is the composition $H:\mathrm{Ch}_{\ge0}(\Z)\xrightarrow{\Gamma}\mathrm{Ab}_\Delta\to\mathrm{Set}_\Delta$ with the forgetful functor.
Note that the cohomology group is computed by the adjoint relations
\[H(X,A)=[N\Z[X],A]=[\Z[X],\Gamma A]=[X,HA],\qquad X\in\mathrm{Set}_\Delta,\ A\in\mathrm{Ch}_{\ge0}(\Z).\]
If $A\in\mathrm{Ab}$, then $(HA)_n=K(A,n)$ is the Eilenberg-MacLane space for each $n\ge0$ so that $H(X,A)$.

Then, the Moore complex functor $N$ is left adjoint to $K$, i.e.~$[NA,C]=[A,KC]$ for $A\in\cA_\Delta$ and $C\in\mathrm{Ch}_{\ge0}(\cA)$.
\end{prb}
\begin{pf}
Let $\Z\Delta$ be the pre-additive category generated by the simplex category $\Delta$.
\end{pf}


Note that the homology group is
\[\begin{tikzcd}
H_n(-,M):&
\mathrm{Top} \rar{\mathrm{Sing}} &
\mathrm{Set}_\Delta \rar{R[\cdot]\otimes_RM} &
(\mathrm{Mod}_R)_\Delta \rar{C\text{ or }N} &
\mathrm{Ch}_{\ge0}(R) \rar{H_n} &
\mathrm{Mod}_R
\end{tikzcd}\]



\begin{prb}[Simplicial complexes]
A \emph{simplicial complex} is a set $K$ of non-empty finite subsets of a set $V$ which is closed under subsets.
If $V$ is linearly ordered, then we say $K$ is ordered.
To every ordered simplicial complex one can associate a simplicial set as follows.
Let $K_n$ be the set of all ordered tuples $(v_0,\cdots,v_n)$ such that $v_0\le\cdots\le v_n$ and $\{v_0,\cdots,v_n\}\in K$.
Then, for each morphism $\alpha:[n]\to[m]$ in $\Delta$, we can define $\alpha^*:K_m\to K_n$.

\end{prb}

\begin{prb}[Simplicial complexes]
A \emph{simplicial complex} is a collection $X\subset\cP(V)$ of non-empty finite subsets of a set $V$ closed under non-empty subsets.
We denote by $X_n$ for each $n\ge0$ the set of all functions $[n]\to V$ whose image is an element of $X$.
More categorically, it can be essentially defined as a presheaf $X:\mathrm{Fin}_{>0}\to\mathrm{Set}:[n]\mapsto X_n$ on the category of non-zero finite sets which is concrete in the sense that the canonical map $X_n\to\prod_{[n]\leftarrow[0]}X_0$ is injective for each $n\ge0$.

If we fix linear orders on each object of $\mathrm{Fin}_{>0}$, then we obtain a functor $\mathrm{SimpCpx}\to\mathrm{Set}_\Delta$
We claim that this functor is characterized as the left Kan extension of along the opposite forgetful functor relative the concrete presheaves as
\[\begin{tikzcd}[sep=small]
\mathrm{Fin}_{>0}^\op \ar{rr}\ar{dr}\ar[shift right=4,phantom]{rr}{\Rightarrow} && \Delta^\op \ar[dashed]{dl} \\
& \mathrm{Set} &
\end{tikzcd}\]

For a simplicial complex $X$ and a simplicial set $Y$, it suffices to show
\[\Hom_{\mathrm{Cpx}_\Delta}(X,Y)=\Hom_{\mathrm{Set}_\Delta}(X,Y).\]
Suppose $\eta\in\Hom_{\mathrm{Set}_\Delta}(X,Y)$ so that
\[\begin{tikzcd}[sep=small]
X_m \dar[swap]{\eta_m}\rar & X_n \dar{\eta_n} \\
Y_m \rar & Y_n
\end{tikzcd}\]
commutes for all monotone $[m]\leftarrow[n]$.
If $[m]\leftarrow[n]$ is arbitrary, then since
\[\begin{tikzcd}[sep=small]
X_m \dar[swap]{\eta_m}\rar & \displaystyle\prod_{[n]\leftarrow[0]}X_0 \dar & X_n \dar{\eta_n}\lar \\
Y_m \rar & \displaystyle\prod_{[n]\leftarrow[0]}Y_0 & Y_n \lar
\end{tikzcd}\]
still commutes and $Y_n\to\prod_{[n]\leftarrow[0]}Y_0$ is injective by the concreteness, which implies
\[\begin{tikzcd}[sep=small]
X_m \dar[swap]{\eta_m}\rar & X_n \dar{\eta_n} \\
Y_m \rar & Y_n
\end{tikzcd}\]
commutes by cancelation of the injective function.

\end{prb}


\section{Simplicial model categories}
\begin{prb}[Model structures on simplicial sets]
Kan and Joyal model structures.

A \emph{Kan model structure} on $\mathrm{Set}_\Delta$ is defined such that a morphism $f:X\to Y$ of simplicial sets is
\begin{enumerate}[(i)]
\item a weak equivalence if the geometric realization $|X|\to|Y|$ is a homotopy equivalence,
\item a cofibration if it is a monomorphism, i.e.~$f_n$ is injective for all $n\ge0$,
\item a fibration if it is a \emph{Kan fibration}, i.e.~it has the right lifting property with respect to the horn pair $\Lambda_i^n\to\Delta^n$ for all $n\ge0$ and $0\le i\le n$.
\end{enumerate}


In a Joyal model structure,

Via the Dold-Kan correspondence $\cA_\Delta\cong\mathrm{Ch}_{\ge0}(\cA)$, the Kan model structure corresponds to the projective model structures
\end{prb}


\chapter{}







\part{Infinity categories}
\chapter{Models of infinity categories}
\section{}

All the algebraic or topological structures with finitely many underlying sets will be assumed to be small.
For example, groupoids may not be small.
Note that $\mathrm{Set}_\Delta$ is the category of \emph{small} simplicial sets.

For an ordinary category as a nerve, two morphisms are homotopic only if they are identical...

\begin{prb}[Kan complexes]
A simplicial set $Y$ is called a \emph{Kan complex} if $Y\to*$ satisfies the right lifting property with respect to all horns $\Lambda_i^n\to\Delta^n$.
In other words, for every simplicial map $\Lambda_i^n\to Y$ there is a simplicial map $\Delta^n\to Y$ such that
\[\begin{tikzcd}[sep=small]
\Lambda_i^n \dar\rar & Y \\
\Delta^n \ar[dashed]{ur}
\end{tikzcd}\]
commutes in the category of simplicial sets for all $n\ge0$ and $0\le i\le n$.

A geometric model for infinity groupoids.
In a Kan complex, including Sing of a topological space, every morphism is invertible up to homotopy.
\end{prb}


\begin{prb}[Nerves of categories]
A small simplicial set $Y$ is a nerve of a small category if and only if $Y\to*$ satisfies the right unique lifting property with respect to all inner horns $\Lambda_i^n\to\Delta^n$.
In other words, for every simplicial map $\Lambda_i^n\to Y$ there is a unique simplicial map $\Delta^n\to Y$ such that
\[\begin{tikzcd}
\Lambda_i^n \dar\rar & Y \\
\Delta^n \ar{ur}
\end{tikzcd}\]
commutes in the category of simplicial sets for all $n\ge0$ and $0<i<n$.
(Is the smallness of $Y$ important in here?)
\end{prb}


\begin{prb}[Weak Kan complexes and quasi-categories]
Intuitively, an infinity category is an enriched category over the category of small infinity groupoids, which cannot serve a definition because it includes a circular reasoning.
A simplicial set $Y$ is called a \emph{weak Kan complex} or a \emph{quasi-category} if


infinity category, infinity groupoid, infinity bicategory.
\end{prb}

Let $\mathrm{Top}$ be the category of compactly generated weakly Hausdorff spaces.
The singular complex functor $\mathrm{Sing}:\mathrm{Top}\to\mathrm{Set}_\Delta$ admits a left adjoint $|\cdot|:\mathrm{Set}_\Delta\to\mathrm{Top}$ called the geometric realization such that the counit map $|\mathrm{Sing}(X)|\to X$ is a weak homotopy equivalence.

The \emph{infinity category of spaces}, denoted by $\mathrm{Spc}$, is defined as the homotopy-coherent nerve of the category $\mathrm{Grpd}_\infty$ of small infinity groupoids.

\begin{prb}
homotopy category, weak homotopy equivalence, 
\end{prb}


\section{Fibrations}



\section{Stable infinity categories}

examples of stable infinity category: the infinity category of spectra, the dervied category of an abelian category

\begin{prb}
A \emph{stable infinity category} is an infinity category such that
\begin{enumerate}[(i)]
\item there is a zero object,
\item every morphism admits a fiber and cofiber,
\item a triangle is a fiber sequence if and only if it is a cofiber sequence.
\end{enumerate}
It is known that its homotopy category is triangulated.
How can we construct the suspension on the homotopy category?
\end{prb}


\begin{prb}[Differential graded categories]
\end{prb}









\chapter{Infinity topoi}




\end{document}













제가 
하지만 만약 이런 좋은 환경에서 공부하지 못했더라면 이렇게 빨리 회의감을 느끼는 날이 오지는 못했을 것 같다는 생각이 듧니다.
저는 다시 묻습니다. 무엇을 공부의 원동력으로 삼아야 할까요?
과거의 제가 남을 이기기 위해 공부하기 시작했다면, 이제는 


뛰어난 사람들과 뒤처지는 사람들을 나와 비교하지 않게 되고 싶다.
많은 사람들을 사랑할 수 있게 되고 싶다.
싫은 소리를 하는 방법에 대해 알게 되고 싶다.
인생의 유의미하다고 생각되는 일을 같이 해나갈 소중한 사람들을 만나고 싶다.
내 스스로가 어떤 위치에 있는지 자꾸 자기검열을 하게 된다.
