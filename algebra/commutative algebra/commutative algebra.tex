\documentclass{../../large}
\usepackage{../../ikhanchoi}


\newcommand{\hgt}{\operatorname{ht}}
\renewcommand{\fd}{\operatorname{fd}}
\renewcommand{\pd}{\operatorname{pd}}
\newcommand{\edim}{\operatorname{edim}}
\newcommand{\gldim}{\operatorname{gldim}}
\newcommand{\depth}{\operatorname{depth}}

\begin{document}
\title{Commutative Algebra}
\author{Ikhan Choi}
\maketitle
\tableofcontents

\part{Affine schemes}
\chapter{Nullstellensatz}

\section{Radicals}

\section{Affine varieties}
\begin{prb}[Weak nullstellensatz]
\end{prb}
\begin{prb}[Noether normalization theorem]
\end{prb}


$\Spec(\C+(x^2-1)\C[x])=\{0,\C\}$.




\chapter{Primary decomposition}

\section{Primary ideals}
\begin{prb}[Primary ideals]
Let $A$ be a ring.
An ideal $\fq$ of $A$ is called \emph{primary} if $A/\fq$ is non-zero and every zero-divisor of $A/\fq$ is nilpotent.
Let $\fq$ be a primary ideal of $A$.
\begin{parts}
\item The radical $r(\fq)$ is the smallest prime ideal of $A$ containing $\fq$.
\end{parts}
\end{prb}

\section{Uniqueness theorems}
Noether-Lasker

\section{Gr\"obner basis}

\begin{prb}[Buchberger algorithm]
\end{prb}

\section*{Exercises}

primary vs prime powers
primary vs prime radical





\chapter{Localization}
\section{}

For $f\in A$, $A_f:=A[f^{-1}]$.

For $\fp\in\Spec A$, $A_\fp:=(A\setminus\fp)^{-1}A$ is a local ring.

local ring
extension of ideals


Since $S^{-1}A$ is a flat $A$-module for a multiplicative set $S\subset A$, the localization functor $S^{-1}:=-\otimes_AS^{-1}A:\Mod_A\to\Mod_{S^{-1}A}$ is always exact.

\[\Spec(A_\fp)\leftrightarrow\{\fq\in\Spec A:\fq\subset\fp\}.\]

\section{Valuation}

DVR
dedekind domains





\part{Dimension theory}

\chapter{}


\begin{prb}
Let $A$ be a ring.
A strictly increasing finite sequence $(\fp_i)_{i=0}^n$ of prime ideals of $A$ is called a \emph{prime chain} of length $n$ in $A$.
The \emph{height} of a prime ideal $\fp$ of $A$ is the supremum of the length of prime chains containing $\fp$:
\[\hgt(\fp):=\sup\{n:\fp_0\subsetneq\cdots\subsetneq\fp_n=\fp,\ \fp_i\in\Spec A\}.\]
The \emph{Krull dimension} or simply the \emph{dimension} of $A$ is the supremum of the heights of prime ideals of $A$:
\[\dim A:=\sup\{\hgt(\fp):\fp\in\Spec A\}.\]
\end{prb}

\begin{prb}[Krull Hauptidealsatz]
\end{prb}

\begin{prb}[Hilbert polynomials]
\end{prb}



\begin{prb}[Minimal number of generators of modules]
The \emph{embedding dimension} of $A$ is defined $\edim A:=\dim_{A/\fm}\fm/\fm^2$.

For a noetherian local ring $A$, $\edim A=\dim A$ is the minimal number of generators of $\fm$.
\end{prb}



\chapter{Homological dimensions}




\begin{prb}[First change of rings]
Let $A$ be a ring, and let $A':=A/xA$ be the quotient for a regular element $x\in A$.
If $M'$ is a non-zero $A'$-module with $\pd_{A'}(M')<\infty$, then $\pd_A(M')=\pd_{A'}(M')+1$.
\end{prb}
\begin{pf}
We introduce the inductive hypothesis on $p':=\pd_{A'}(M')$.
Consider a short exact sequence of $A'$-modules
\[0\to K'\to F'\to M'\to0,\]
where $F'$ is a free $A'$-module.


On the other hand, we get $\pd_{A'}(K')=p'-1$ because for any $A'$-module $N'$ we have $\Ext_{A'}^i(K',N')=0$ for $i\ge p'$ by the long exact sequence of $A'$-modules
\[0\cong\Ext_{A'}^i(F',N')\to\Ext_{A'}^i(K',N')\to\Ext_{A'}^{i+1}(M',N')\cong0,\qquad i\ge p'.\]
Therefore, the inductive hypothesis deduces
\[\pd_A(M')=\pd_A(K')+1=\pd_{A'}(K')+1=(p'-1)+1=p'.\]
\end{pf}

$\gldim A=\pd_A(A/\fm)$.




\begin{prb}
Let $A$ be a noetherian local ring.

Let $d:=\dim A$, $g:=\gldim A=\pd_A(k)$, and $e:=\edim A$.
\begin{parts}
\item If $A$ is regular, then $A$ has finite global dimension.
\item If $A$ has finite global dimension, then $A$ is regular.
\end{parts}
\end{prb}
\begin{pf}
(a)

(b)
It suffices to show $e=d$.


We want to find $x\in A$ such that
\begin{enumerate}[(i)]
\item $x$ is regular in $A$ with $\gldim A'<\infty$,
\item $x\in\fm\setminus\fm^2$.
\end{enumerate}
where we denote by $A':=A/xA$ the quotient local ring with the maximal ideal $\fm'$ and the residue field $k'=k$.
Since the first condition implies $d=d'+1$ by the first change of rings and the second condition implies that

$d'=\pd_{A'}(k)<\infty$ implies that we can apply the first change of rings.



We have , $d'=e'$ by the inductive hypothesis, and $e'+1=e$ by the fact $\ker(\fm/\fm^2\to\fm'/\fm'^2$

\end{pf}





\end{document}