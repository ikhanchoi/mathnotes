\documentclass{../../large}
\usepackage{../../ikhanchoi}

\newcommand{\Frob}{\mathrm{Frob}}
\newcommand{\ab}{\mathrm{ab}}
\newcommand{\Nm}{\mathrm{Nm}}

\begin{document}
\title{Algebraic Number Theory}
\author{Ikhan Choi}
\maketitle
\tableofcontents

\part{Algebraic numbers}
\chapter{Primes}
an order defines a ring class group, a ring class group defines an abelian extension.
the conductor of this abelian extension divides the conductor of the order.

\section{Local fields}
\begin{prb}[Absolute value]
Let $K$ be a field.
An \emph{absolute value} or a \emph{multiplicative valuation} on $K$ is a function $|\cdot|:K\to[0,\infty)$ such that
\begin{enumerate}[(i)]
\item $x=0$ if $|x|=0$,
\item $|xy|=|x||y|$,
\item $|x+y|\le|x|+|y|$.
\end{enumerate}
Non-archimedean
\end{prb}

\begin{prb}[Local fields]
A \emph{local field} is a locally compact field with a non-trivial absolute value.
The Ostrowski theorem states that a local field is one of the followings:
\begin{enumerate}[(i)]
\item a finite extension of $\Q_p$ for a rational prime $p$,
\item a finite extension of $\F_p((T))$ for a rational prime $p$,
\item $\R$ or $\C$.
\end{enumerate}
Let $K$ be a non-archimedian local field.
Then, the ring of integers $\cO_K$ is a discrete valuation ring, and a generator $\pi$ of the principal maximal ideal $\fm_K$ is called the \emph{prime element} or the \emph{uniformizer}.

Local reciprocity law: there is a unique homomoprhism
\[\phi_K:K^\times\to\Gal(K^\ab/K)=\varprojlim_L\Gal(L/K)\]
such that
\begin{enumerate}
\item for each finite unramified extension $L$ over $K$, which is automatically cyclic, $\phi_{L/K}(\pi)$ is the Frobenius element in $\Gal(L/K)$,
\item for each finite abelian extension $L$ over $K$, it induces an isomorphism $\phi_{L/K}:K^\times/\Nm_{L/K}(L^\times)\to\Gal(L/K)$.
\end{enumerate}
\end{prb}

\begin{prb}[Places]
\end{prb}

\begin{prb}[Units in non-archimedean local fields]
Let $K$ be a non-archimedean local field.
$\cO_K$

\end{prb}


\chapter{Ad\`eles and id\`eles}


\chapter{Galois modules}
\section{Profinite groups}

\section{}

\begin{prb}[Galois modules]
\begin{parts}
\item $L$, $L^\times$, $\cO_L$, $\cO_L^\times$ are all $\Gal(L/K)$-modules.
\item The group of torsion points
\end{parts}
\end{prb}

\begin{prb}[Normal basis theorem]
\end{prb}


\section{Galois cohomology}
\begin{prb}[Set of invariants]
\end{prb}

\begin{prb}[First cohomology groups]
\end{prb}

\begin{prb}[Hilbert 90]
\begin{parts}
\item $H^1(\Gal(L/K),L^\times)\cong0$.
\item $H^1(\Gal(\bar K/K),\bar K)\cong0$.
\item $H^1(\Gal(\bar K/K),\bar K^\times)\cong0$.
\item $H^1(\Gal(\bar K/K),\mu_m)\cong\bar K/\bar K^\times$.
\end{parts}
\end{prb}
\begin{pf}
\end{pf}





\part{Class field theory}
\chapter{Local class field theory}
\section{Lubin-Tate theory}
\section{Kronecker-Weber theorem}
\begin{prb}[Local Kronecker-Weber theorem]
Let $K/\Q_p$ be a finite abelian extension.

\end{prb}





Let $K/\Q$ be a finite abelian extension.
A \emph{conductor} $\ff(L/K)$ of $K/\Q$ is the smallest non-negative integer $n$ such that the higher unit group
\[U^{(n)}=1+\fm_K^n\]
is contained in $N_{L/K}(L^\times)$.

Let $m$ be a conductor of a finite abelian extension $K/\Q$.
Then, we have a surjective group homomorphism
\[\Gal(\Q(\zeta_m)/\Q)\to\Gal(K/\Q)\]
by the Kronecker-Weber theorem.
For a prime $p\in\Z$ that does not divide $m$ so that $p$ is not ramified, then the decomposition group $G_p\le\Gal(K/\Q)$ is a cyclic group generated by the Frobenius element $x\to x^p$, denoted by $\Frob_p$ or $\left(\frac{K/\Q}p\right)$.
Artin map $I_\Q^m\to\Gal(K/\Q)$ of $K/\Q$ maps each prime $p\nmid m$ to the Frobenius element $\Frob_p$.
Artin map factors through $\Gal(\Q(\zeta_m)/\Q)\to\Gal(K/\Q)$!


\chapter{Global class field theory}



\chapter{}




\part{Arithmetic geometry}

\part{Langlands program}
\chapter{Modular forms}




\chapter{$L$-functions}
Riemann $\zeta(s)$

Dedekind $\zeta_K(s)$

Hasse-Weil $\zeta_X(s)$

\section{Dirichlet $L$-functions}

\begin{prb}[Hecke character]
Dirichlet character can be understood as a group homomorphism $\chi:\hat\Z^\times\to\C$ of finite order, which means that there is $n$ such that $\chi$ factors through $(\Z/n\Z)^\times$.

In order to construct an $L$-function from a character, we need to extend a character as a function of ideals.
We interpret $(\Z/n\Z)^\times$ as the ray class group modulo $\fm$.

To extend the order of a character to possibly infinite cases, Hecke character is defined a character of an idele class group $C_K:=\A_K^\times/K^\times$.
\end{prb}


Dirichlet (Hecke) $L$-functions for ray-class characters $\chi:C_K\to\C$:
\[L(\chi,s)=\sum_{\fa}\frac{\chi(\fa)}{N(\fa)^s}=\prod_{\fp\text{ prime}}\frac1{1-\chi(\fp)N(\fp)^{-s}}\]

Artin $L$-functions for a Galois representation $\rho:\Gal(L/K)\to GL_n(\C)$:
\[L(\rho,s)=\prod_{\fp\text{ prime}}\frac1{\det(1-\rho(\Frob_\fp)N(\fp)^{-s})}\]

Elliptic curves $L(E,s)$

Modular forms $L(f,s)$


\chapter{Automorphic representations}

\end{document}