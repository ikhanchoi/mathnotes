\documentclass{../../large}
\usepackage{../../ikhanchoi}

\newcommand{\Frob}{\mathrm{Frob}}
\newcommand{\ab}{\mathrm{ab}}
\newcommand{\Nm}{\mathrm{Nm}}

\begin{document}
\title{Algebraic Number Theory}
\author{Ikhan Choi}
\maketitle
\tableofcontents

\part{Algebraic numbers}
\chapter{Primes}
an order defines a ring class group, a ring class group defines an abelian extension.
the conductor of this abelian extension divides the conductor of the order.

\section{Local fields}
\begin{prb}[Absolute value]
Let $K$ be a field.
An \emph{absolute value} or a \emph{multiplicative valuation} on $K$ is a function $|\cdot|:K\to[0,\infty)$ such that
\begin{enumerate}[(i)]
\item $x=0$ if $|x|=0$,
\item $|xy|=|x||y|$,
\item $|x+y|\le|x|+|y|$.
\end{enumerate}
Non-archimedean
\end{prb}

\begin{prb}[Local fields]
A \emph{local field} is a locally compact field with a non-trivial absolute value.
The Ostrowski theorem states that a local field is one of the followings:
\begin{enumerate}[(i)]
\item a finite extension of $\Q_p$ for a rational prime $p$,
\item a finite extension of $\F_p((T))$ for a rational prime $p$,
\item $\R$ or $\C$.
\end{enumerate}
Let $K$ be a non-archimedian local field.
Then, the ring of integers $\cO_K$ is a discrete valuation ring, and a generator $\pi$ of the principal maximal ideal $\fm_K$ is called the \emph{prime element} or the \emph{uniformizer}.

Local reciprocity law: there is a unique homomoprhism
\[\phi_K:K^\times\to\Gal(K^\ab/K)=\varprojlim_L\Gal(L/K)\]
such that
\begin{enumerate}
\item for each finite unramified extension $L$ over $K$, which is automatically cyclic, $\phi_{L/K}(\pi)$ is the Frobenius element in $\Gal(L/K)$,
\item for each finite abelian extension $L$ over $K$, it induces an isomorphism $\phi_{L/K}:K^\times/\Nm_{L/K}(L^\times)\to\Gal(L/K)$.
\end{enumerate}
\end{prb}

\begin{prb}[Places]
\end{prb}

\begin{prb}[Units in non-archimedean local fields]
Let $K$ be a non-archimedean local field.
$\cO_K$

\end{prb}


\chapter{Ad\`eles and id\`eles}


\chapter{Galois modules}
\section{Profinite groups}

\section{}

\begin{prb}[Galois modules]
\begin{parts}
\item $L$, $L^\times$, $\cO_L$, $\cO_L^\times$ are all $\Gal(L/K)$-modules.
\item The group of torsion points
\end{parts}
\end{prb}

\begin{prb}[Normal basis theorem]
\end{prb}


\section{Galois cohomology}
\begin{prb}[Set of invariants]
\end{prb}

\begin{prb}[First cohomology groups]
\end{prb}

\begin{prb}[Hilbert 90]
\begin{parts}
\item $H^1(\Gal(L/K),L^\times)\cong0$.
\item $H^1(\Gal(\bar K/K),\bar K)\cong0$.
\item $H^1(\Gal(\bar K/K),\bar K^\times)\cong0$.
\item $H^1(\Gal(\bar K/K),\mu_m)\cong\bar K/\bar K^\times$.
\end{parts}
\end{prb}
\begin{pf}
\end{pf}





\part{Class field theory}
\chapter{Local class field theory}
\section{Lubin-Tate theory}
\section{Kronecker-Weber theorem}
\begin{prb}[Local Kronecker-Weber theorem]
Let $K/\Q_p$ be a finite abelian extension.

\end{prb}





Let $K/\Q$ be a finite abelian extension.
A \emph{conductor} $\ff(L/K)$ of $K/\Q$ is the smallest non-negative integer $n$ such that the higher unit group
\[U^{(n)}=1+\fm_K^n\]
is contained in $N_{L/K}(L^\times)$.

Let $m$ be a conductor of a finite abelian extension $K/\Q$.
Then, we have a surjective group homomorphism
\[\Gal(\Q(\zeta_m)/\Q)\to\Gal(K/\Q)\]
by the Kronecker-Weber theorem.
For a prime $p\in\Z$ that does not divide $m$ so that $p$ is not ramified, then the decomposition group $G_p\le\Gal(K/\Q)$ is a cyclic group generated by the Frobenius element $x\to x^p$, denoted by $\Frob_p$ or $\left(\frac{K/\Q}p\right)$.
Artin map $I_\Q^m\to\Gal(K/\Q)$ of $K/\Q$ maps each prime $p\nmid m$ to the Frobenius element $\Frob_p$.
Artin map factors through $\Gal(\Q(\zeta_m)/\Q)\to\Gal(K/\Q)$!


\chapter{Global class field theory}



\chapter{}




\part{Arithmetic geometry}

\part{Langlands program}
\chapter{Modular forms}

modular forms are sections of some line bundles over a moduli stack $\cM$ of complex elliptic curves.
By modular forms, we can investigate the algebraic nature of $\cM$.


Let $N\ge1$ and $k\ge2$.
Let $\chi:(\Z/N\Z)^\times\to\C^\times$ be a Dirichlet character.

Let $\Gamma$ be a congruence subgroup which acts on $\H$.
The vector space of all cusp forms and modular forms weight $k$ with respect to $\Gamma$ is denoted by $S_k(\Gamma)\subset M_k(\Gamma)$.


Since $\Gamma_1(N)$ acts trivially on $S_k(\Gamma_1(N))$, we have an action of $\Gamma_0(N)/\Gamma_1(N)\cong(\Z/N\Z)^\times$ on $S_k(\Gamma_1(N))$, and we define $S_k(N,\chi)$ by the decomposition
\[S_k(\Gamma_1(N))=\bigoplus_{\chi:(\Z/N\Z)^\times\to\C^\times}S_k(N,\chi).\]
We also define $S_k(N):=S_k(N,1)=S_k(\Gamma_0(N))$.

The Hecke operators are defined as a commuting family of endomorphisms $(T_n)_{n=1}^\infty$ on $S_k(\Gamma_1(N))$.
Let $f=\sum_{n\ge1}a_nq^n\in S_k(\Gamma_1(N))$ be a cusp form.
We say $f$ is a normalized eigenform if $a_1=1$ and it is an eigenvector of Hecke operators, and in this case we have $T_nf=a_nf$.
It is known that the field $\Q(f):=\Q(a_n:n\ge1)$ is a finite extension of $\Q$.
We have an $L$-function given by
\[L(f,s):=\sum_{n\ge1}a_nn^{-s}.\]


$G_{\Q_p}$ is a subgroup of $G_\Q$, called the decomposition group, well-defined up to conjugacy.

Let $f\in S_k(N,\chi)$ be a normalized eigenform, and let $\lambda\mid\ell$ be a place.
Then, there is a two-dimensional representation $V_{f,\lambda}$ over an $\ell$-adic field $\Q(f)_\lambda$ of $G_\Q$ such that
\[\Tr_{V_{f,\lambda}}(\Frob_p)=a_p\]
for every prime $p$ such that $p\nmid N\ell$ and $V_{f,\lambda}$ is unramified at $p$, where $\Frob_p\in G_{\Q_p}$ is the Frobenius automorphism.



\chapter{$L$-functions}
Riemann $\zeta(s)$

Dedekind $\zeta_K(s)$

Hasse-Weil $\zeta_X(s)$

\section{Dirichlet $L$-functions}

By the Kronecker-Weber theorem, a continuous one-dimensional complex representation $G_\Q\to\C^\times$ of the absolute Galois group factors through the Galois group $\Gal(\Q(\zeta_n)/\Q)\cong(\Z/n\Z)^\times$ of some cyclotomic extension to be a Dirichlet character.

We also want to study $\ell$-adic Galois representations.


\begin{prb}[Hecke character]
Let $\chi:(\Z/N\Z)^\times\to\C^\times$ be a Dirichlet character.
In order to construct an $L$-function from a character, we need to extend a character as a function of ideals.
We interpret $(\Z/n\Z)^\times$ as the ray class group modulo $\fm$.

To extend the order of a character to possibly infinite cases, Hecke character is defined a character of an idele class group $C_K:=\A_K^\times/K^\times$.
\end{prb}


Dirichlet (Hecke) $L$-functions for ray-class characters $\chi:C_K\to\C$:
\[L(\chi,s)=\sum_{\fa}\frac{\chi(\fa)}{N(\fa)^s}=\prod_{\fp\text{ prime}}\frac1{1-\chi(\fp)N(\fp)^{-s}}\]

Artin $L$-functions for a Galois representation $\rho:\Gal(L/K)\to GL_n(\C)$:
\[L(\rho,s)=\prod_{\fp\text{ prime}}\frac1{\det(1-\rho(\Frob_\fp)N(\fp)^{-s})}\]

Elliptic curves $L(E,s)$

Modular forms $L(f,s)$


\chapter{Automorphic representations}

\end{document}