\documentclass{../../large}
\usepackage{../../ikhanchoi}


\begin{document}
\title{Classical Geometry}
\author{Ikhan Choi}
\maketitle
\tableofcontents

\part{Classical geometry}
\chapter{Euclidean geometry}

\section{Plane geometry}
\section{Solid geometry}
\section{Axiomatization}

\chapter{Non-Euclidean geometry}
\section{Absolute geometry}
axioms 1 to 4
\section{Spherical and elliptic geometry}
axioms 2 and 4
\section{Hyperbolic geometry}
axiomes 1 to 4

Models of hyperbolic geometry (metric description)
Elementary figures
Isometries
Length, volume, angle

\chapter{Non-metric geometry}
\section{Ordered and incidence geometry}
axioms 1 and 2
\section{Affine and projective geometry}
axioms 1,2,5
\section{Conformal and inversive geometry}






\part{Smooth surfaces}
\chapter{Manifolds}
\section{Local coordinates}

\section{Space curves}

\section{Space surfaces}

%         선형독립    벡터들이 한점에서 주어졌을 때 ->
%         선형독립    벡터장이 근방에서 주어졌을 때 -> 일반적으론 2차원에서만
%         선형독립 가환벡터장이 근방에서 주어졌을 때 -> n차원 다돼
%         선형독립    직교벡터들이 한점에서 주어졌을 때
%         선형독립    직교벡터장이 근방에서 주어졌을 때
%         선형독립 직교가환벡터장이 근방에서 주어졌을 때
%             -> 곡률의 선, 점근곡선, 측지좌표

% preimage theorem

Reparametrizations

\begin{thm}
Let $S$ be a regular surface.
Let $v,w$ be linearly independent tangent vectors in $T_pS$ for a point $p\in S$.
Then, $S$ admits a parametrization $\alpha$ such that $\alpha_x|_p=v$ and $\alpha_y|_p=w$.
\end{thm}
\begin{thm}
Let $X,Y$ be linearly independent tangent vector fields on a regular surface $S$.
Then, $S$ admits a parametrization $\alpha$ such that $\alpha_x|_p$ and $\alpha_y|_p$ are parallel to $X|_p,Y|_p$ respectively for each $p\in S$.
\end{thm}
\begin{thm}
Let $X,Y$ be linearly independent tangent vector fields on a regular surface $S$.
If $\pd_XY=\pd_YX$, then $S$ admits a parametrization $\alpha$ such that $\alpha_x|_p=X|_p$ and $\alpha_y|_p=Y|_p$ for each $p\in S$.
\end{thm}

Let $S$ be a regular surface embedded in $\R^3$.
The inner product on $T_pS$ induced from the standard inner product of $\R^3$ can be represented not only as a matrix
\[\mat{1&0&0\\0&1&0\\0&0&1}\]
in the basis $\{(1,0,0),(0,1,0),(0,0,1)\}\subset\R^3$, but also as a matrix
\[\mat{\<\alpha_x,\alpha_x\>&\<\alpha_x,\alpha_y\>\\\<\alpha_y,\alpha_x\>&\<\alpha_y,\alpha_y\>}\]
in the basis $\{\alpha_x|_p,\alpha_y|_p\}\subset T_pS$.

\begin{defn}
\emph{Metric coefficients}
\begin{alignat*}{2}
\<\alpha_x,\alpha_x\>&=:g_{11}&\qquad
\<\alpha_x,\alpha_y\>&=:g_{12}\\
\<\alpha_y,\alpha_x\>&=:g_{21}&
\<\alpha_y,\alpha_y\>&=:g_{22}
\end{alignat*}
\end{defn}

\begin{thm}[Normal coordinates]
...?
\end{thm}




\subsection*{Differentiation of tangent vectors}

\begin{defn}
Let $\alpha:U\to\R^3$ be a regular surface.
The \emph{Gauss map} or \emph{normal unit vector} $\nu:U\to\R^3$ is a vector field on $\alpha$ defined by:
\[\nu(x,y):=\frac{\alpha_x\times \alpha_y}{\|\alpha_x\times \alpha_y\|}(x,y).\]
The set of vector fields $\{\alpha_x|_p,\alpha_y|_p,\nu|_p\}$ forms a basis of $T_p\R^3$ at each point $p$ on $\alpha$.
The Gauss map is uniquely determined up to sign as $\alpha$ changes.
\end{defn}

\begin{defn}[Gauss formula, $\Gamma_{ij}^k$, $L_{ij}$]
Let $\alpha:U\to\R^3$ be a regular surface.
Define indexed families of smooth functions $\{\Gamma_{ij}^k\}_{i,j,k=1}^2$ and $\{L_{ij}\}_{i,j=1}^2$ by the Gauss formula
\begin{alignat*}{2}
\alpha_{xx}&=:\Gamma_{11}^1\alpha_x+\Gamma_{11}^2\alpha_y+L_{11}\nu,&\qquad
\alpha_{xy}&=:\Gamma_{12}^1\alpha_x+\Gamma_{12}^2\alpha_y+L_{12}\nu,\\
\alpha_{yx}&=:\Gamma_{21}^1\alpha_x+\Gamma_{21}^2\alpha_y+L_{21}\nu,&
\alpha_{yy}&=:\Gamma_{22}^1\alpha_x+\Gamma_{22}^2\alpha_y+L_{22}\nu.
\end{alignat*}
The \emph{Christoffel symbols} refer to eight functions $\{\Gamma_{ij}^k\}_{i,j,k=1}^2$.
The Christoffel symbols and $L_{ij}$ \emph{do depend} on $\alpha$.
\end{defn}
We can easily check the symmetry $\Gamma_{ij}^k=\Gamma_{ji}^k$ and $L_{ij}=L_{ji}$.
Also,
\begin{align*}
\pd_XY
&=X^i\pd_i(Y^j\alpha_j)\\
&=X^i(\pd_iY^k)\alpha_k+X^iY^j\pd_i\alpha_j\\
&=\left(X^i\pd_iY^k+X^iY^j\Gamma_{ij}^k\right)\alpha_k+X^iY^jL_{ij}\nu.
\end{align*}

% Examples


\subsection*{Differentiation of normal vector}

The partial derivative $\pd_X\nu$ is a tangent vector field since
\[\<\pd_X\nu,\nu\>=\frac12\pd_X\<\nu,\nu\>=0.\]
Therefore, we can define the following useful operator.
\begin{defn}
Let $S$ be a regular surface embedded in $\R^3$.
The \emph{shape operator} is $\cS:\fX(S)\to\fX(S)$ defined as
\[\cS(X):=-\pd_X\nu.\]
\end{defn}
\begin{prop}
The shape operator is self-adjoint, i.e. symmetric.
\end{prop}
\begin{pf}
Recall that $\pd_XY-\pd_YX$ is a tangent vector field.
Then,
\[\<X,\cS(Y)\>=\<X,-\pd_Y\nu\>=\<\pd_YX,\nu\>=\<\pd_XY,\nu\>=\<\cS(X),Y\>.\qedhere\]
\end{pf}

% The reason of minus sign in the shape operator.

\begin{thm}
Let $\alpha:U\to\R^3$ be a regular surface and $\cS$ be the shape operator.
Then $\cS$ has the coordinate representation
\[\cS=\mat{g_{11}&g_{12}\\g_{21}&g_{22}}^{-1}\mat{L_{11}&L_{12}\\L_{21}&L_{22}}\]
with respect to the frame $\{\alpha_x,\alpha_y\}$ for tangent spaces.
In other words, if we let $X=X^i\alpha_i$ and $\cS(X)=\cS(X)^j\alpha_j$, then
\[\mat{\cS(X)^1\\\cS(Y)^2}=\mat{g_{11}&g_{12}\\g_{21}&g_{22}}^{-1}\mat{L_{11}&L_{12}\\L_{21}&L_{22}}\mat{X^1\\X^2}.\]
\end{thm}
\begin{pf}
Let $\cS(X)^j=\cS_i^jX_i$.
Then,
\[g_{ik}X^i\cS_j^kY^j=\<X,\cS(Y)\>=\<\pd_XY,\nu\>=X^iY^jL_{ij}\]
implies $g_{ik}\,\cS_j^k=L_{ij}$.
\end{pf}

% principal curvature
% mean curvature, gaussian curvature



% curvature tensor?





















\chapter{Fundamental forms}

\section{Riemannian metrics}

\section{Gaussian curvatures}
Theorema egregium
surfaces of constant gaussian curvature

% 제1기본형식, 크리스토펠: 
% 모양 연산자, 제2기본형식: 바인가르텐 이퀘이션

% 가우스곡률
\begin{defn}
Let $\alpha:U\to\R^3$ be a regular surface.
\begin{gather*}
E:=\<\alpha_x,\alpha_x\>=g_{11},\qquad F:=\<\alpha_x,\alpha_y\>=g_{12},\qquad G:=\<\alpha_y,\alpha_y\>=g_{22},\\
L:=\<\alpha_{xx},\nu\>=L_{11},\qquad M:=\<\alpha_{xy},\nu\>=L_{12},\qquad N:=\<\alpha_{yy},\nu\>=L_{22}.
\end{gather*}
\end{defn}


\begin{cor}
We have $GM-FN=EM-FL$, and the \emph{Weingarten equations}:
\begin{align*}
\nu_x&=\frac{FM-GL}{EG-F^2}\alpha_x+\frac{FL-EM}{EG-F^2}\alpha_y,\\
\nu_y&=\frac{FN-GM}{EG-F^2}\alpha_x+\frac{FM-EN}{EG-F^2}\alpha_y.
\end{align*}
\end{cor}



\begin{thm}
\[\Gamma_{ij}^l=\frac12g^{kl}(g_{ik,j}-g_{ij,k}+g_{kj,i}).\]
\end{thm}

\[\frac12(\log g)_x=\Gamma_{11}^1.\]

\[\nu_x\times\nu_y=K\sqrt{\det g}\ \nu.\]
\[\alpha_x\times\alpha_y=\sqrt{\det g}\ \nu\]
\[\<\nu_x\times\nu_y,\alpha_x\times\alpha_y\>=\det\mat{\<\nu_x,\alpha_x\>&\<\nu_x,\alpha_y\>\\\<\nu_y,\alpha_x\>&\<\nu_y,\alpha_y\>}=\det\mat{-L&-M\\-M&-N}=K\det g\]











\begin{prb}[Gaussian curvature formula]
\begin{parts}
\item
In general,
\[K=\frac{LN-M^2}{EG-F^2}.\]
\item
For orthogonal coordinates such that $F\equiv0$,
\[K=-\frac1{2\sqrt{\det g}}\left((\frac1{\sqrt{\det g}}E_y)_y+(\frac1{\sqrt{\det g}}G_x)_x\right).\]
\item
For $f(x,y,z)=0$,
\[K=-\frac1{|\nabla f|^4}\mat[v]{0&\nabla f\\\nabla f^T&\Hess(f)},\]
where $\nabla f$ denotes the gradient $\nabla f=(f_x,f_y,f_z)$.
\item(Beltrami-Enneper) If $\tau$ is the torsion of an asymptotic curve, then
\[K=-\tau^2.\]
\item(Brioschi) $E,F,G$ describes $K$.
\end{parts}
\end{prb}

\begin{pf}
(a) Clear.

(b)
We have $GM=EM$ and
\[\nu_x=-\frac LE\alpha_x-\frac MG\alpha_y,\qquad\nu_y=-\frac ME\alpha_x-\frac NG\alpha_y.\]
\[\nu_x\times\nu_y=\frac{LN-M^2}{EG}\alpha_x\times\alpha_y\]
After curvature tensors...

\end{pf}



\begin{prb}[Computation of Gaussian curvatures]
\begin{parts}
\item
(Monge's patch)
For $(x,y,f(x,y))$,
\[K=\frac{f_{xx}f_{yy}-f_{xy}^2}{(1+f_x^2+f_y^2)^2}.\]

\item
(Surface of revolution).
Let $\gamma(t)=(r(t),z(t))$ be a plane curve with $r(t)>0$.
If $t\mapsto(r(t),z(t))$ is a unit-speed curve, then
\[K=-\frac{r''}r.\]

\item
(Models of hyperbolic planes)
\end{parts}
\end{prb}
\begin{pf}
(b)
Let
\[\alpha(\theta,t)=(r(t)\cos\theta,r(t)\sin\theta,z(t))\]
be a parametrization of a surface of revolution.
Then,
\begin{align*}
\alpha_\theta&=(-r(t)\sin\theta,r(t)\cos\theta,0)\\
\alpha_t&=(r'(t)\cos\theta,r'(t)\sin\theta,z'(t))\\
\nu&=\frac1{\sqrt{r'(t)^2+z'(t)^2}}(z'(t)\cos\theta,z'(t)\sin\theta,-r'(t)),
\end{align*}
and
\begin{align*}
\alpha_{\theta\theta}&=(-r(t)\cos\theta,-r(t)\sin\theta,0)\\
\alpha_{\theta t}&=(-r'(t)\sin\theta,-r'(t)\cos\theta,0)\\
\alpha_{tt}&=(r''(t)\cos\theta,r''(t)\sin\theta,z''(t)).
\end{align*}
Thus we have
\[E=r(t)^2,\quad F=0,\quad G=r'(t)^2+z'(t)^2,\]
and
\[L=-\frac{r(t)z'(t)}{\sqrt{r'(t)^2+z'(t)^2}},\quad M=0,\quad N=\frac{r''(t)z'(t)-r'(t)z''(t)}{\sqrt{r'(t)^2+z'(t)^2}}.\]
Therefore,
\[K=\frac{LN-M^2}{EG-F^2}=\frac{z'(r'z''-r''z')}{r(r'^2+z'^2)^2}.\]
In particular, if $t\mapsto(r(t),z(t))$ is a unit-speed curve, then
\[K=-\frac{r''}r.\]
\end{pf}


% asymptotic curve -> hyperbolic
% line of curvature -> non-umbilic

% minimal surface
% 	회전곡면
% asymptotic curve
% 	Beltrami-Enneper
% ruled surface
% developable surface


\begin{prb}[Local isomorphism]
Surfaces of the same constant Gaussian curvature are locally isomorphic.
\end{prb}
\begin{pf}
Let
\[\mat{\|\alpha_r\|^2&\<\alpha_r,\alpha_t\>\\\<\alpha_t,\alpha_r\>&\|\alpha_t\|^2}=\mat{1&0\\0&h(r,t)^2}\]
be the first fundamental form for a geodesic coordinate chart along a geodesic curve so that $\alpha_{tt}$ and $\alpha_{rr}$ are normal to the surface.
Then,
\[K=-\frac{h_{rr}}h\]
is constant.
Also, since
\[\frac12(h^2)_r+\<\alpha_r,\alpha_{tt}\>=\<\alpha_{rt},\alpha_t\>+\<\alpha_r,\alpha_{tt}\>=\<\alpha_r,\alpha_t\>_t=0\]
implies $h_r=0$ at $r=0$, the function $f:r\mapsto h(r,t)$ satisfies the following initial value problem
\[f_{rr}=-Kf,\quad f(0)=1,\quad f'(0)=0.\]
Therefore, $h$ is uniquely determined by $K$.
\end{pf}



\chapter{Compact smooth surfaces}



% Global theory of curves
%  Isoperimetric inequality
%  Four vertex theorem
%  Ovals

% Global theory of surfaces
%  Minimal surfaces
%  classification of compact surfaces?
%  Hilbert theorem

% Total curvatures
%  Fary-Milnor
%  Fenchel
%  Gauss-Bonnet, euler characteristic













\part{Riemann surfaces}

\chapter{Riemann-Roch theorem}

\chapter{Algebraic curves}

multiplicities, Bezout theorem
divisors, line bundles
Embedding theorem
euler characteristic
(tangent line bundle degree 2-2g, canonical line bundle 2g-2)
$L(D):=H^0(X,\cO(D))$

Jacobian variety (moduli spaces....)
Chow theorem


\chapter{Uniformization}

\iffalse
\section{Symmetry groups}
isometry, conformal, rigid motion, etc.
\section{Discrete subgroups}
continuous group actions?
(convex, locally finite) fundamental domains
tesselation, polygon theorem
Fuchsian and Kleinian
\fi




\part{Topological surfaces}


\chapter{Fundamental groups}

\section{Homotopy}
\begin{prb}
A \emph{homotopy of paths} is a continuous map $h:I\times I\to X$ such that $h(0,)=x_0$ and 
\begin{parts}
\item linear homotopy
\item reparametrization
\end{parts}
\end{prb}
\begin{prb}
The fundamental group is a group
composition
\end{prb}

\begin{prb}[Van Kampen theorem]
\end{prb}

\section{Covering spaces}

path lifting property
universal covering

\chapter{Homology groups}
\section{Singular homology}
\section{Simplicial homology}
\section{Cellular homology}



\chapter{Classification of surfaces}
\section{Combinatorial surfaces}

triangulation
orientability
euler characteristic
genus
connected sum



\end{document}