\documentclass{../../large}
\usepackage{../../ikhanchoi}


\begin{document}
\title{Differential Topology}
\author{Ikhan Choi}
\maketitle
\tableofcontents


\part{De Rham theory}
\chapter{De Rham theory}
\section{De Rham theorem}

\chapter{Hodge theory}
elliptic operators


\part{Cobordism}
\chapter{Morse theory}
\section{Morse functions}

\begin{defn}
Let $M$ be a manifold.
A \emph{Morse function} is a smooth function $f:M\to\R$ such that all critical points are nondegenerate.
\end{defn}

\begin{prop}
Let $M$ be an embedded submanifold of $\R^n$.
For almost every point $p\in\R^n$, the function $f:M\to\R:x\mapsto\|x-p\|^2$ is Morse.
\end{prop}
\begin{pf}
Suppose that $p\in\R^n$ makes $f$ be not Morse so that it possesses a degenerate critical point.
Note that the notation $x$ can denote not only a point variable on $M$ but also the embedding map $M\emb\R^n$.
Let $N\subset M\times\R^n$ be the normal bundle of the tangent bundle $TM$ and define a map $\f:N\to\R^n$ such that $\f(x,y)=x+y$.
We claim that the point $(x,p-x)$ is contained in $N$ and $\f$ is critical at this point if $f$ is degenerate at $x$.

The differential of $f$ is
\[df_x(v)=2(x-p)\cdot dx(v)=2(x-p)\cdot v,\]
so $x$ is critical point if and only if $x-p$ is proportional to $T_xM$.

Let $\{x^i\}_{i=1}^m$ be orthonormal coordinates for $M$ and let $\{e_j\}_{j=1}^{n-m}$ be an orthonormal frame field of $N$.
Define coordinate functions $\{x^i,y^j\}$ on the manifold $N$ by
\[x^i(x,y):=x^i(x),\quad\text{and}\quad y^j(x,y):=y\cdot e_j(x).\]
Then,
\[\left\{\,\pd{x}{x^1},\cdots,\pd{x}{x^m},\pd{y}{y^1},\cdots,\pd{y}{y^{n-m}}\,\right\}\]
always form an orthonormal basis on $\R^n$ and

Since
\[\pd{\f}{x^i}=\pd{x}{x^i}+\pd{y}{x^i}\quad\text{and}\quad\pd{\f}{y^j}=\pd{y}{y^j},\]
we have
\begin{alignat*}{2}
\pd{\f}{x^i}\cdot\pd{x}{x^k}&=\delta_{ik}-y\cdot\pd{x}{x^i}{x^k},\qquad&
\pd{\f}{x^i}\cdot\pd{y}{y^l}&=-y\cdot\pd{y}{x^i}{y^l},\\
\pd{\f}{y^j}\cdot\pd{x}{x^k}&=0,&
\pd{\f}{y^j}\cdot\pd{y}{y^l}&=\delta_{jl}.
\end{alignat*}
To represent $d\f(\pd_{x^1},\cdots,\pd_{y^{n-m}})$ with matrix, we can write
\[\begin{pmatrix}\displaystyle\pd{\f}{x^i}\\\displaystyle\pd{\f}{y^j}\end{pmatrix}
\begin{pmatrix}\displaystyle\pd{x}{x^k}&\displaystyle\pd{y}{y^l}\end{pmatrix}
=\begin{pmatrix}\displaystyle\id-y\cdot\pd{x}{x^i}{x^k}&\displaystyle-y\cdot\pd{y}{x^i}{y^l}\\0&\id\end{pmatrix}.\]
Then,
\[\pd{f}{x^i}{x^j}=2\left(\id+(x-p)\cdot\pd{x}{x^i}{x^j}\right)\]
deduces that $d\f$ is not surjective at $(x,p-x)$.
Therefore, by the Sard theorem, set of such $p$ has measure zero.
\end{pf}

\begin{prop}
Let $M$ be a manifold.
The set of Morse functions is dense in $C^\infty(M)$.
\end{prop}
\begin{pf}
Let $f$ be a smooth function on $M$.
Embed $M$ in $\R^{d-1}$ such that $x\mapsto(x_2,\cdots,x_d)$.
Then, $x\mapsto(f(x),x_2,\cdots,x_d)$ gives an embedding into $R^d$.
Define a sequence $\{\e_n\}_n\subset\R^n$ such that $\e_n\to0$ and the sequence of functions
\[f_n(x):=\frac{\|x+ne_1+\e_n\|^2-n^2}{2n}\]
is Morse, where $\{e_i\}$ denotes the standard basis of $\R^d$.
This can be done by the previous proposition.
Then,
\begin{align*}
f_n(x)&=\frac{(f(x)+n+\e_n\cdot e_1)^2+\cdots+(x_n+\e_n\cdot e_d)^2-n^2}{2n}\\
&=f(x)+\frac{\|x+\e_n\|}{2n}+\e_n\cdot e_1
\end{align*}
proves that $\|f_n-f\|_{C^k(K)}\to0$ on every compact $K\subset M$.
\end{pf}

\begin{thm}[Morse lemma]
Let $p$ be a nondegenerate critical point of a Morse function $f$ on a manifold $M$.
Then, there exists a local chart $(U,\f)$ of $p$ such that
\[f\circ\f^{-1}(x_1,\cdots,x_m)=f(p)-\sum_{i=1}^kx_i^2+\sum_{i=k+1}^nx_i^2\]
for some $k$.
This chart is called \emph{Morse chart}.
\end{thm}
\begin{pf}
%%%
\end{pf}
\begin{cor}
The critical points of a Morse function are isolated.
In particular, on a compact manifold are finitely many critical points of a Morse function.
\end{cor}

\section{Pseudo-gradients}
\begin{defn}
Let $f$ be a Morse function on a manifold $M$.
A \emph{pseudo-gradient} adapted to $f$ is a vector field $X$ such that
\begin{parts}
\item $df(X)<0$ at all noncritical points,
\item there is a Morse chart at critical points in which $X=\grad f$, where the metric is induced from the chart.
\end{parts}
\end{defn}
\begin{prop}
A pseudo-gradient always exists for any Morse functions.
\end{prop}
\begin{pf}
Cover the manifold with charts such that every critical point is contained in a unique chart, which is Morse.
For each chart $(U,\f)$, we can define a vector field on $U$ by
\[X:=-d\f^{-1}(\grad(f\circ\f^{-1})),\]
using the standard metric on $\f(U)$.
Then, we have
\[df(X)=-\<\grad(f\circ\f^{-1}),\grad(f\circ\f^{-1})\>\le0,\]
where the equality holds only at critical points.
With a partition of unity, the vector fields are combined and easily checked to be pseudogradient.
\end{pf}

\begin{defn}
Let $p$ be a critical point of a Morse function $f$ on a manifold $M$.
Denote $\f^s:M\to M$ by the flow of a pseudo-gradient.
A \emph{stable manifold} is defined as
\[W^s(p):=\{\,x\in M:\lim_{s\to\infty}\f^s(x)=p\,\},\]
and an \emph{unstable manifold} is defined as
\[W^u(p):=\{\,x\in M:\lim_{s\to-\infty}\f^s(x)=p\,\}.\]
\end{defn}
\begin{prop}
The stable manifolds and unstable manifolds are manifolds.
Further, they are diffeomorphic open disks.
Moreover, the index of $p$ is equal to
\[\dim W^u(p)=\codim W^s(p)\].
\end{prop}


\chapter{}
\chapter{}



\part{Topological quantum field theory}
\chapter{Chern-Simons theory}



\begin{prb}[Lie algebra-valued forms]
Let $\pi:P\to M$ be a smooth principal $G$-bundle for a compact Lie group $G$.
The three vector bundles $P\times\fg$, $TP$, $T^*P$ over $P$ and their section spaces
\[\Gamma(P\times\fg)=\Omega^0(P,\fg),\qquad\Gamma(TP)=\fX(P),\qquad\Gamma(T^*P)=\Omega^1(P)\]
admit smooth right $G$-actions $\ad^{-1}$, $dR$, $(dR^*)^{-1}$ respectively.
The actions are not equivariant in the sense that they trivially act on the base space $P$.
The right $G$-action on $\Omega^1(P,\fg)$ is given in the identification $\Omega^1(P,\fg):=\Omega^0(P,\fg)\otimes_P\Omega^1(P)$ with the tensor product bundle as $\ad^{-1}\otimes(dR^*)^{-1}$.

\[((dR^*_h)^{-1}(dR^*_g)^{-1}\omega)(X)
=\omega(dR_{g^{-1}}dR_{h^{-1}}X)
=\omega(d(R_{(gh)^{-1}})X)
=((dR^*_{gh})^{-1}\omega)(X)\]


notation: if $F$ is a vector space, then
\[\Omega^k(M,F):=\Gamma(M\times F)\otimes_M\Omega^k(M),\]
and if $E$ is a vector bundle over $M$, then
\[\Omega^k(M,E):=\Gamma(E)\otimes_M\Omega^k(M).\]

 trace and determinant.

Consider the universal enveloping algebra $U(\fg)$.
Then, $\Omega(P,U(\fg))$ is an algebra bundle over $P$ because it is the tensor product of two algebra bundles over $P$.
Concretely, if we denote the multiplication of $\omega_1\in\Omega^k(P,U(\fg))$ and $\omega_2\in\Omega^l(P,U(\fg))$ by $\omega_1\wedge\omega_2\in\Omega^{k+l}(P,U(\fg))$, then it is computed in the geometric wedge product convention as
\[(\omega_1\wedge\omega_2)(X_1,\cdots,X_{k+l}):=\frac1{k!l!}\sum_{\sigma\in S_{k+l}}\sgn(\sigma)\omega_1(X_{\sigma(1)},\cdots,X_{\sigma(k)})\omega_2(X_{\sigma(k+1)},\cdots,X_{\sigma(k+l)})\]
and we also define $[\omega_1,\omega_2]\in\Omega^{k+l}(P,U(\fg))$ such that
\[[\omega_1,\omega_2](X_1,\cdots,X_{k+l}):=\frac1{k!l!}\sum_{\sigma\in S_{k+l}}\sgn(\sigma)[\omega_1(X_{\sigma(1)},\cdots,X_{\sigma(k)}),\omega_2(X_{\sigma(k+1)},\cdots,X_{\sigma(k+l)})]\]
and
\begin{align*}
d\omega(X_0,X_1,\cdots,X_k):&=\frac1{k+1}\sum_{0\le i\le k}(-1)^iX_i(\omega(X_0,\cdots,\hat X_i,\cdots,X_k))\\
&+\frac1{k+1}\sum_{0\le i<j\le k}(-1)^{i+j}\omega([X_i,X_j],X_0,\cdots,\hat X_i,\cdots,\hat X_j,\cdots,X_k).
\end{align*}

Let $\omega\in\Omega^1(P,\fg)$.
We can embed $\Omega(P,\fg)\subset\Omega(P,U(\fg))$ to do computations.
Then,
\[(\omega\wedge\omega)(X,Y)=\omega(X)\omega(Y)-\omega(Y)\omega(X)=[\omega(X),\omega(Y)]\]
and
\[[\omega,\omega](X,Y)=[\omega(X),\omega(Y)]-[\omega(Y),\omega(X)]=2[\omega(X),\omega(Y)]\]
imply that $\omega\wedge\omega=\frac12[\omega,\omega]\in\Omega^2(P,\fg)$.
We also have
\[d\omega(X,Y)=\frac12(X(\omega(Y))-Y(\omega(X)))+\frac12\omega([X,Y]).\]

The coefficient conventions are not so important that coefficients are eventually cancelled when we write an equation of forms.
\end{prb}


\begin{prb}[Ehresmann connections]
Let $\pi:E\to M$ be a smooth fiber bundle.
An \emph{Ehresmann connection} on $E$ is a vector subbundle $HE\to E$ of the tangent bundle $TE$ such that $VE\oplus HE=TE$, where $VE\to E$ is defined by the kernel of $TE\to TM$.
It is the choice of the splitting section of the exact sequence
\[0\to VE\to TE\to\pi^*TM\to0.\]
This horizontal subbundle $HE$ gives rise to a surjective linear bundle map $TE\to VE$ to the vertical subbundle, and also a cartesian square
\[\begin{tikzcd}
HE \rar{d\pi}\dar\pullback & TM\dar\\
E \rar{\pi} & M,
\end{tikzcd}\]
so that we have a Lie algebra bundle map $\fX(M)\to\fX(E)$ with horizontal image, called the \emph{horizontal lift}.

parallel transport
\end{prb}

\begin{prb}[Connection forms]
Let $\pi:P\to M$ be a smooth principal $G$-bundle.
Keep in mind that the vertical bundle $VP$ is canonically trivialized by the right $G$-equivariant bundle map $P\times\fg\to VP\subset TP$ constructed by the image of the injective linear bundle map
\[P\times\fg\to T(P\times G)\to TP,\qquad\text{ over }P,\]
where the first map is $P\times\fg\to T(P\times G):(u,A)\mapsto((u,e),(0,A))$ and the second map is the differential $T(P\times G)\to TP$ of the principal action $P\times G\to P$.
This trivialization induces a right $G$-equivariant $P$-linear embedding $\Omega^0(P,\fg)\to\fX(P)$, and for a constant section $A\in\Omega^0(P,\fg)$ we can assign the \emph{fundamental vector field} $A^\#\in\fX(P)$.

A \emph{connection form} is simply a right $G$-equivariant left inverse of this fundamental vector field map.
More precisely, it is defined as a Lie algebra-valued 1-form $\omega\in\Omega^1(P,\fg)$ or a $P$-linear map $\omega:\fX(P)\to\Omega^0(P,\fg)$ which is
\begin{enumerate}[(i)]
\item right $G$-invariant in the sense that $\omega(dR_gX)=\ad_g^{-1}(\omega(X))\in\Omega^0(P,\fg)$ for all $X\in\fX(P)$,
\item vertical in the sense that $\omega(A^\#)=A$ for constant sections $A\in\fg$.
\end{enumerate}
A connection form decompose the vector bundle $TP$ into the direct sum of $VP$ and the kernel, which gives rise to the corresponding horizontal subbundle.
A connection form $\omega$ projects $X\in\fX(P)$ to the vertical one.
\begin{parts}
\item We can also define a connection as an right $G$-invariant Ehresmann connection $HP\to P$.

\end{parts}
\end{prb}
\begin{pf}
(a)
Since a right $G$-equivariant Ehresmann connection gives rise to a linear map $TP\to VP$, so by composition with $VP\cong P\times \fg$, we get the corresponding connection form $TP\to P\times\fg$.
\end{pf}


\begin{prb}
Note that $G$ is itself a principal $G$-bundle over a point.
There is a natural connection form $\omega_G\in\Omega^1(G,\fg)$ called the \emph{Maurer-Cartan form}, defined such that $\omega_G:\fX(G)\to\Omega^0(G,\fg):X\mapsto(g\mapsto dL_g^{-1}(X|_g))$.
The right $G$-invariance of the Maurer-Cartan form $\omega_G$ is due to
\[\omega_G(dR_h(v))=dL_{gh}^{-1}dR_h(v)=dL_h^{-1}dR_hdL_g^{-1}(v)=\ad_h^{-1}\omega_G(v),\qquad v\in T_gG,\ h\in G.\]
We can check $\omega_G$ is a connection form.
\end{prb}

\begin{prb}[Curvature form]
Let $P$ be a principal $G$-bundle over $M$.
The \emph{curvature form} of the connection form $\omega\in\Omega^1(P,\fg)$ can be defined either the covariant derivative of $\omega$ or the Cartan structural equation $\Omega=d\omega+\frac12[\omega,\omega]\in\Omega^2(P,\fg)$.

The curvature form is horizontal in the sense that for every vertical vector field $X\in\fX(P)$ we have $\iota_X\Omega=0\in\Omega^1(P,\fg)$.
\end{prb}



\begin{prb}[Exterior covariant derivatives]
Let $\pi:P\to M$ be a smooth principal $G$-bundle.
Let $F$ be a faithful representation of $G$.
We have a right $G$-equivariant trivial vector bundle $P\times F$ over $P$, where the action is given such that $(p,f)g=(pg,g^{-1}f)$, and the associated vector bundle $E:=P\times_GF$ over $M$.

We say a $F$-valued form $\omega\in\Omega^k(P,F)$ is \emph{horizontal} if $\iota_X\omega=0$ for vertical $X\in\fX(P)$.
We have
\[\Omega^k(M,E)\cong\Omega_h^k(P,F)^G,\]
where $\Omega_h^k(P,F)^G$ denotes the set of right $G$-invariant horizontal $F$-valued $k$-forms on $P$, since we have a cartesian square
\[\begin{tikzcd}
P\times F \pullback\rar\dar & E \dar \\ P \rar & M.
\end{tikzcd}\]

The exterior derivative $d:\Omega^k(P,F)^G\to\Omega^{k+1}(P,F)^G$ does not preserve horizontality in general.
For a connection form $\omega\in\Omega^1(P,\fg)^G$, we can define the \emph{exterior covariant derivative} $\nabla:\Omega^0(M,E)\to\Omega^1(M,E)$ is defined such that we have a commutative diagram
\[\begin{tikzcd}
\Omega^k(M,E) \rar{\nabla}\dar[equal] & \Omega^{k+1}(M,E)\dar[equal]\\
\Omega_h^k(P,F)^G \rar{d_\omega} & \Omega_h^{k+1}(P,F)^G,
\end{tikzcd}\]
where $d_\omega$ is defined by
\[d_\omega\psi(X_0,\cdots,X_k):=d\psi(HX_0^*,\cdots,HX_k^*),\]
where $HX_i=X_i-\omega(X_i)\in\fX(P)$ are horizontal components of $X_i\in\fX(P)$.
The most important case is $k=0$.
\[\nabla_Xs=\]
\end{prb}



\begin{prb}[Local expression of principal connections]
Let $\pi:P\to M$ be a smooth principal $G$-bundle, where $G$ is a compact Lie group.
Then, we can fix $\{\f_\alpha\}$ be a local trivialization of $\pi$ such that $\f_\alpha:\pi^{-1}(U_\alpha)\to U_\alpha\times G$ is right $G$-equivariant.
Consider a bundle map
\[(\Omega^1(U_\alpha,\fg),\ad^{-1}\otimes\id)\to(\Omega^1(\pi^{-1}(U_\alpha),\fg),\ad^{-1}\otimes(dR^*)^{-1}):A\mapsto\f_\alpha^*(\pr_2^*(\omega_G)+\pr_1^*(A))\]
along the map $\pi^{-1}(U_\alpha)\to U_\alpha$, where $A\in\Omega^1(U_\alpha,\fg)$ and the vertical term $\omega_G$ denotes the Maurer-Cartan form.
In the physical contexts of gauge theory such as the standard model, the 1-form $A\in\Omega^1(U_\alpha,\fg)$ is mainly used to describe principal connections.
\begin{parts}
\item The above local representation is a right $G$-equivariant affine bundle map.
\item The above local representation is injective and the image is exactly the set of connection forms on the trivial principal $G$-bundle $\pi^{-1}(U_\alpha)$ over $U_\alpha$.
\item What is the meaning of the expression $\nabla=d+A$?
\item Local expression of $\Omega=d\omega+\frac12[\omega,\omega]$?
\item What is the meaning of the expression $dA\wedge A+\frac23A\wedge A\wedge A$ and how can we write it in terms of principal connections.
\end{parts}
\end{prb}
\begin{pf}
(a)
We interpret $A\in\Omega^1(U_\alpha,\fg)$ as $A:\fX(U_\alpha)\to\Omega^0(U_\alpha,\fg)$ so that for every $X\in\fX(U_\alpha)$,
\[\]

(c)

Let $L\to M$ be a trivial line bundle.

Let $s\in\Omega^0(M,L)$, so that $ds:\Omega^1(M,L)$ or
\[ds:\fX(M)\to\Omega^0(M,L):X\mapsto X^\mu\partial_\mu s.\]

Let $A\in\Omega^1(M,\fu(1))$ or
\[A:\fX(M)\to\Omega^0(M,\fu(1)):X\mapsto X^\mu A_\mu.\]

We have
\[\nabla_Xs=(\nabla s)(X)=((d+A)s)(X)=X^\mu\partial_\mu s+X^\mu A_\mu s\]
\end{pf}


\[S:=\frac k{4\pi}\int_M\tr\left(dA\wedge A+\frac23A\wedge A\wedge A\right)\]

In this case, the field equation is $F=0$.

\section{Chern-Weil theory}


\[(\Sym^n\fg^*)^G\cong(\Sym^n\ft^*)^W\cong H^{2n}(BG,\R).\]

Given a principal $G$-bundle $P\to X$, this isomorphism induces a ring homomorphism
\[\mathrm{CW}:(\Sym^*\fg^*)^G\to H^{\mathrm{ev}}(X,\R),\]
called the \emph{Chern-Weil homomorphism}.
In fact, when $X$ is a smooth manifold, then there is a direct construction of the Chern-Weil homomorphism using connections.
Choose any connection $\omega$ on $P$, and let $\Omega$ be the curvature.



\[(\Sym^n\fg^*)^G\otimes\Omega^{2n}(P,\fg^{\otimes n})\to\Omega^{2n}(P)\]

\section{Chern-Simons invariants}


\section{Differential cohomology}

\chapter{}
\chapter{}




\part{Index theory}
\chapter{}
\chapter{}
\chapter{}


\part{Symplectic geometry}
\chapter{}
\chapter{}
\chapter{}

\end{document}