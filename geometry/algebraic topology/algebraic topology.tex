\documentclass{../../large}
\usepackage{../../ikhanchoi}


\begin{document}
\title{Algebraic Topology}
\author{Ikhan Choi}
\maketitle
\tableofcontents


\part{}

\chapter{Convenient categories}
\section{Compactly generated weakly Hausdorff spaces}
bicomplete cartesian closed monoidal category.
Here, closed means that the right tensoring admits an adjoint called internal hom functor.
braided and symmetric...?


A pointed space is a pair of a space and a 0-cell.
The smash product is the categorical product in the category of pointed space.


Let $\mathrm{Top}$ be the bicomplete cartesian closed monoidal category of CGWH spaces.
\section{CW complexes}



\section{Simplicial complexes}






\chapter{Cohomology operations}
\section{Eilenberg-Steenrod axioms}
cohomology
cup product
universal coefficient theorem
Poincar\'e duality






\section*{Exercises}

characteristic class of projective spaces





\chapter{Spectral sequences}
\section{Serre spectral sequence}
	(Lyndon-Hochschild-Serre)
\section{Adams spectral sequence}






\chapter{Fiber bundles}


\section{Principal bundles}

\begin{prb}[Locally trivial fiber bundles]
Let $p:E\to B$ be a map and $F$ a space.
An \emph{atlas} or \emph{local trivialization} for $p$ with respect to $F$ is a family $\{\f_\alpha\}$ of homeomorphisms $\f_\alpha:p^{-1}(U_\alpha)\to U_\alpha\times F$ satisfying $p=\pr_{U_\alpha}\f_\alpha$ as maps $p^{-1}(U_\alpha)\to U_\alpha$, indexed by an open cover $\{U_\alpha\}$ of $B$: 
\[\begin{tikzcd}[column sep=tiny]
p^{-1}(U_\alpha) \ar[swap]{dr}{p}\ar{rr}{\f_\alpha} & \, & U_\alpha\times F \ar{dl} \\
& U_\alpha \uar[phantom]{\circlearrowleft} &.
\end{tikzcd}\]
A \emph{locally trivial fiber bundle}, or just a \emph{fiber bundle}, is a map $p:E\to B$ together with an equivalence class of atlases $\{\f_\alpha\}$ for $p$ with respect to $F$.
The spaces $F,E,B$ are called the \emph{fiber space}, \emph{total space}, and \emph{base space} of the fiber bundle $p$.
A \emph{bundle map} between fiber bundles $p_i:E_i\to B_i$ with fiber space $F_i$ for $i\in\{1,2\}$ is a map of pairs $(u,f):(E_1,B_1)\to(E_2,B_2)$ such that
\[\begin{tikzcd}
E_1 \ar[swap]{d}{p_1}\ar{r}{u}\ar[phantom]{dr}{\circlearrowleft} & E_2 \ar{d}{p_2} \\
B_1 \ar[swap]{r}{f} & B_2.
\end{tikzcd}\]
The category $\mathrm{Bun}_F(B)$ is defined such that objects are fiber bundles over a base space $B$ with fiber space $F$, and morphisms are commutative squares.
\begin{parts}
\item $p$ is surjective and open.
\end{parts}
\end{prb}

\begin{prb}[Structure groups]
Let $p:E\to B$ be a fiber bundle whose fiber space $F$ is an effective left $G$-space for a topological group $G$.
An atlas $\{\f_\alpha\}$ for $p$ is called a \emph{$G$-atlas} if there is a family $\{g_{\alpha\beta}\}$, which is unique if it exists by effectiveness of $F$, of maps $g_{\alpha\beta}:U_\alpha\cap U_\beta\to G$ such that
\[\f_\beta\f_\alpha^{-1}(b,v)=(b,g_{\alpha\beta}(b)v),
\qquad b\in U_\alpha\cap U_\beta,\ v\in F.\]
A \emph{$G$-bundle} is a fiber bundle $p$ whose fiber space is an effective left $G$-space together with an equivalence class of $G$-atlases $\{\f_\alpha\}$ for $p$.

A bundle map $(u,f):(E_1,B_1)\to(E_2,B_2)$ between $G$-bundles $p_i:E_i\to B_i$ with a common fiber space $F$ for $i\in\{1,2\}$ is called a \emph{$G$-bundle map} if there is a family $\{h_{\alpha\beta}\}$, which is unique if it exists also by effectiveness of $F$, of maps $h_{\alpha\beta}:U_{1\alpha}\cap f^{-1}(U_{2\beta})\to G$ such that
\[\f_{2\beta}u\f_{1\alpha}^{-1}(b,v)=(f(b),h_{\alpha\beta}(b)v),\qquad b\in U_{1\alpha}\cap f^{-1}(U_{2\beta}),\ v\in F,\]
where $\{\f_{i\alpha}\}$ is a $G$-atlas for $p_i$.
Note that the definition of $G$-bundle maps does not depend on the choice of $G$-atlases of $p_i$.
\begin{parts}
\item A $G$-bundle map $(u,f)$ is an isomorphism if and only if $u$ is a homeomorphism. In particular, the category of $G$-bundles over $B$ with fiber space $F$ is a groupoid.
\item A bundle map $(u,\id_B):(E_1,B)\to(E_2,B)$ is a $G$-bundle map if and only if there is a family $\{h_\alpha\}$ of maps $h_\alpha:U_\alpha\to G$ such that
\[\f_{2\alpha}u\f_{1\alpha}^{-1}(b,v)=(b,h_\alpha(b)v),
\qquad b\in U_\alpha,\ v\in F,\]
where $\{\f_{i\alpha}\}$ is a $G$-atlas for $p_i$ defined on a common open cover $\{U_\alpha\}$ of $B$.
\end{parts}
\end{prb}
\begin{pf}
\end{pf}



\begin{prb}[Fiber bundle construction theorem]
Let $\{U_\alpha\}$ be an open cover of a space $B$, and $G$ a topological group.
A \emph{\v Cech 1-cocyle} on $\{U_\alpha\}$ with coefficients in $G$ is a family $\{g_{\alpha\beta}\}$ of maps $g_{\alpha\beta}:U_\alpha\cap U_\beta\to G$ satisfying the \emph{cocycle condition}:
\[g_{\alpha\gamma}(b)=g_{\beta\gamma}(b)g_{\alpha\beta}(b),\qquad b\in U_\alpha\cap U_\beta\cap U_\gamma.\]
A \emph{\v Cech 0-cochain} on $\{U_\alpha\}$ with coefficients in $G$ is a family $\{h_\alpha\}$ of maps $h_\alpha:U_\alpha\to G$ of maps without any conditions.
The set of \v Cech 1-cocycles and \v Cech 0-cochains on $\{U_\alpha\}$ with coefficients in $G$ are denoted by $\check Z^1(\{U_\alpha\},G)$ and $\check C^0(\{U_\alpha\},G)$ repspectively.
The \emph{first \v Cech cohomology} $\check H^1(\{U_\alpha\},G)$ of $\{U_\alpha\}$ with coefficients in $G$ is defined to be the orbit space of an action of $\check C^0(\{U_\alpha\},G)$ on $\check Z^1(\{U_\alpha\},G)$ defined as
\[(\{h_\alpha\}\{g_{\alpha\beta}\})_{\alpha\beta}(b):=h_\beta(b)g_{\alpha\beta}(b)h_\alpha(b)^{-1},
\qquad b\in U_\alpha\cap U_\beta.\]
We define the \emph{first \v Cech cohomology} of $B$ with coefficients in $G$ as the direct limit of sets
\[\check H^1(B,G):=\varinjlim_{\{U_\alpha\}}\check H^1(\{U_\alpha\},G).\]
Let $F$ be an effective left $G$-space.
Let $\mathrm{Bun}_F(B)$ be the set of all isomorphism classes of $G$-bundles over $B$ with fiber space $F$.
\begin{parts}
\item $\mathrm{Bun}_F(B)\to\check H^1(B,G)$ is well-defined and sujective.
\item $\mathrm{Bun}_F(B)\to\check H^1(B,G)$ is injective.
\end{parts}
\end{prb}
\begin{pf}
(a)
Suppose $p_i:E_i\to B$ are isomorphic $G$-bundles with fiber space $F$, and $u:E_1\to E_2$ is a $G$-bundle isomorphism.
Considering the refinement, we fix an open cover $\{U_\alpha\}$ of $B$ on which $E_i$ are simultaneously locally trivialized.

Let $\{g_{\alpha\beta}\}\in\check Z^1(\{U_\alpha\},G)$.
Define
\[E:=\Bigl(\bigsqcup_\alpha(U_\alpha\times F)\Bigr)/\sim,\]
where $\sim$ is an equivalence relation generated by
\[(b,v)_\alpha\sim(b,g_{\alpha\beta}(b)v)_\beta,
\qquad b\in U_\alpha\cap U_\beta,\ v\in F.\]
Define $p:E\to B$ and $\f_\alpha^{-1}:U_\alpha\times F\to p^{-1}(U_\alpha)$ such that
\[p([(b,v)_\alpha]):=b,\qquad\f_\alpha^{-1}(b,v):=[(b,v)_\alpha],\qquad b\in U_\alpha,\ v\in F.\]
They are clearly continuous and surjective.
We need to show $\f_\alpha^{-1}$ is a hoemomorphism.

We first claim that $\f_\alpha^{-1}$ is injective.
Assume $\f_\alpha^{-1}(b,v)=\f_\alpha^{-1}(b',v')$, which means $(b,v)_\alpha\sim(b',v')_\alpha$.
Then, we have $b=b'$ and there is a finite sequence $(\alpha_i)_{i=0}^n$ such that $\alpha_0=\alpha_n=\alpha$ and
\[v'=g_{\alpha_{n-1}\alpha_n}(b)g_{\alpha_{n-2}\alpha_{n-1}}(b)\cdots g_{\alpha_0\alpha_1}(b)v.\]
Applying the cocycle condition inductively, we obtain $v=v'$, which implies the injectivity of $\f_\alpha^{-1}$.

Next we claim that $\f_\alpha^{-1}$ is open.
The map $\f_\alpha^{-1}$ is given by the composition
\[\f_\alpha^{-1}:U_\alpha\times F\xrightarrow{\iota}\bigsqcup_\alpha U_\alpha\times F\xrightarrow{\pi}E,\]
where $\iota$ and $\pi$ are the canonical inclusion and the canonical projection.
Since $\iota$ is clearly open, it suffices to show $\pi$ is open.
Suppose $V\subset\bigsqcup_\alpha U_\alpha\times F$ is open so that we have $V=\bigsqcup_\alpha V_\alpha\times F$ for open subsets $V_\alpha\subset U_\alpha$.
Observe that for each $\beta$ we have $(b,v)_\beta\in\pi^{-1}\pi(V_\alpha\times F)$ if and only if $(b,v)_\beta\sim(b',v')_\alpha$ for some $(b',v')\in V_\alpha\times F$.
It is equivalent to that $b=b'$ and 
\[v'=g_{\alpha_{n-1}\alpha_n}(b)g_{\alpha_{n-2}\alpha_{n-1}}(b)\cdots g_{\alpha_0\alpha_1}(b)v,\]
where $(\alpha_i)_{i=0}^n$ is a finite sequence such that $\alpha_0=\beta$ and $\alpha_n=\alpha$.
Applying the cocycle condition inductively, we can see that it is just $g_{\alpha\beta}(b)v'=v$.
Thus, the set $\pi^{-1}\pi(V_\alpha\times F)\cap(U_\beta\times F)$ is exactly the same as the inverse image of the open set $V_\alpha\times F$ under the map
\[(U_\alpha\cap U_\beta)\times F\to U_\alpha\times F:(b,v)\mapsto(b,g_{\alpha\beta}(b)v).\]
It concludes that
\[\pi^{-1}\pi(V)=\bigcup_{\alpha,\beta}\pi^{-1}\pi(V_\alpha\times F)\cap(U_\beta\times F)\]
is open, so $\pi(V)$ is open.
Therefore, $\pi$ is open.

Finally, it obviously follows that the transition maps of the $G$-atlas $\{\f_\alpha\}$ constructed as above coincides with the cocycle $\{g_{\alpha\beta}\}$ up to 0-cochain by the cocycle condition, hence the surjectivity.

(b)
\end{pf}


\begin{prb}[Principal bundles]
Let $G$ be a topological group.
A \emph{principal $G$-bundle} is a fiber bundle $\pi:P\to B$ together with a continuous right action of $G$ on $P$ such that $\pi^{-1}(b)$ is a right principal homogeneous $G$-space for each $b\in B$.
The right action $\rho$ is called the \emph{principal right action} or \emph{(global) gauge transformation}.
Having right $G$-equivariant maps as morphisms, let $\mathrm{Prin}_G(B)$ be the isomorphism classes of principal $G$-bundles over $B$.

Let $F$ be an effective left $G$-space.
For a principal $G$-bundle $\pi:P\to B$, define $p:E\to B$ by
\[E:=P\times_GF=P\times F/\sim,\qquad p([u,v]):=\pi(u),\]
where the equivalence relation $\sim$ is generated by
\[(ug,v)\sim(u,gv),\qquad u\in P,\ g\in G,\ v\in F.\]
The $G$-bundle $p$ with fiber space $F$ is called the \emph{associated bundle} to the principal bundle $\pi$.
The associated bundles gives rise to a natural function $\mathrm{Prin}_G(B)\to\mathrm{Bun}_F(B)$.
\begin{parts}
\item $\mathrm{Prin}_G(B)\to\mathrm{Bun}_F(B)$ is well-defined.
\item $\mathrm{Prin}_G(B)\to\mathrm{Bun}_F(B)$ is injective.
\item $\mathrm{Prin}_G(B)\to\mathrm{Bun}_F(B)$ is surjective.
\item A principal bundle is trivial if it has a global section.
\end{parts}
\end{prb}
\begin{pf}
(a)
Let $u:P\to P'$ be a right $G$-equivariant map.
By fixing $x_0\in X$ and using the fact that the left action is free and transitive, define $g_i:U_i\to G$ such that
\[(b,g_i(b)x_0):=\f_i\circ u\circ\f_i^{-1}(b,x_0).\]
The function $g_i$ is continuous since it factors as
\[b\mapsto(b,x_0)\xmapsto{\f_i\circ u\circ\f_i^{-1}}(b,g_i(b)x_0)\mapsto g_i(b)x_0\mapsto g_i(b).\]
The continuity of the last map is due to the assumption that the map $(g,x)\mapsto(gx,x)$ is a homeomorphism.

Then, for every $(b,x)\in U_i\times X$ there is a unique $s\in G$ such that
\[\f_i\circ\rho_s\circ\f_i^{-1}(b,x_0)=(b,x),\]
so we have
\begin{align*}
\f_i\circ u\circ\f_i^{-1}(b,x)
&=(\f_i\circ u\circ\f_i^{-1})\circ(\f_i\circ\rho_s\circ\f_i^{-1})(b,x_0)\\
&=\f_i\circ u\circ\rho_s\circ\f_i^{-1}(b,x_0)\\
&=\f_i\circ\rho_s\circ u\circ\f_i^{-1}(b,x_0)\\
&=(\f_i\circ\rho_s\circ\f_i^{-1})\circ(\f_i\circ u\circ\f_i^{-1})(b,x_0)\\
&=(\f_i\circ\rho_s\circ\f_i^{-1})g_i(b)(b,x_0)\\
&=g_i(b)(\f_i\circ\rho_s\circ\f_i^{-1})(b,x_0)\\
&=g_i(b)(b,x)\\
&=(b,g_i(b)x).
\end{align*}
Hence, $u$ is a $G$-bundle map.

(b)
Let $u:P\to P'$ be a $G$-bundle map over $B$ so that there is a set $\{h_i:U_i\to G\}_i$ of maps such that
\[\f_i\circ u\circ\f_i^{-1}(b,x)=(b,h_i(b)x),
\qquad b\in U_i,\ x\in X.\]
If we write $\rho_s:P\to P:e\mapsto \rho(e,s)$ for $s\in G$, then the induced right action $\f_i\circ\rho_s\circ\f_i^{-1}$ commutes with the left action $\f_i\circ u\circ\f_i^{-1}$ on $U_i\times X$.
Now for every $e\in P_1$, we have
\begin{align*}
\rho_s\circ u(e)
&=\f_i^{-1}\circ(\f_i\circ\rho_s\circ\f_i^{-1})\circ(\f_i\circ u\circ\f_i^{-1})\circ\f_i(e)\\
&=\f_i^{-1}\circ(\f_i\circ u\circ\f_i^{-1})\circ(\f_i\circ\rho_s\circ\f_i^{-1})\circ\f_i(e)\\
&=u\circ\rho_s(e),
\end{align*}
therefore $u$ is right $G$-equivariant.


(c)
Fix $x_0\in X$ and consider the homeomorphism $G\to X:g\to gx_0$.
Define a right action
\[X\times G\to X:(gx_0,s)\mapsto gx_0s:=gsx_0.\]
It defines a right principal homogeneous $X$ that commutes with the left action on $X$.

Define $\rho:P\times G\to P$ such that
\[\f_i\circ\rho_s\circ\f_i^{-1}(b,x)=(b,xs).\]
It is well defined, fiber preserving, continuous.
also for any $b$ and any chart $\f_j$ containing $b$, the action on $\{b\}\times X$ defines a principal homogeneous as we have seen.
Therefore, $\rho$ is a gauge tranformation.

(d)
($\Rightarrow$)
Clear.

($\Leftarrow$)
Let $s:B\to E$ be a global section and define
\[\tilde u:B\times X\to E:(b,gx_0)\mapsto s(b)g\]
for any fixed $x_0\in X$.
Then, the continuous map $(\tilde f,\id_B)$ preserves fibers and defines a right $G$-equivariant isomorphism.
\end{pf}

\begin{prb}[Quotient principal bundles]

\end{prb}


\begin{prb}[Reduction of structure groups]
Let $H$ be a closed subgroup of $G$.
Then, there is a function $\check H^1(B,H)\to\check H^1(B,G)$, which is neither in general injective nor surjective.
If a $G$-bundle $p$ is contained in the image of $\check H^1(B,H)$ through the correspondence $\mathrm{Bun}_F(B)\twoheadrightarrow\check H^1(B,G)$, then we may give a $H$-bundle structure on $p$.

A \emph{reduction} of $G$ to $H$ is a $H$-structure on a principal $G$-bundle.
\end{prb}






\section{Vector bundles}
subbundles, quotient bundles, bundle maps,
constant rank, then ker, im, coker bundles are locally trivial so that they are vector bundles.
pullback: vector bundle structure

vector fields(trivial subbundles), parallelizable
bundle operations: sum, tensor, dual, hom, exterior

reduction and metrics


\[\mathrm{Bun}_{\GL(n,\F)}(B)\to\mathrm{Vect}_n^\F(B)\]
is a faithful essentially surjective functor.
If we drop the non-invertible morphisms in $\mathrm{Vect}_n(B)$, then the functor becomes an equivalence.

\[\Spin(n)\to\SO(n)\to\GL(n,\R)\]

\begin{prb}[Vector bundles]
Let $p:E\to B$ and $p:E'\to B$ be vector bundles.
\begin{parts}
\item A vector bundle map $u$ over $B$ is a vector bundle isomorphism if and only if it is a fiberwise linear isomorphism.
\end{parts}
\end{prb}



Let $1\le n\le\infty$.
If $f,g:B\to G_k(\F^n)$ such that $f^*(\gamma_{k^n})\cong g^*(\gamma_{k^n})$, then $jf\simeq jg$, where $j:G_k(\F^n)\to G_k(\F^{2n})$ is the natural inclusion.


\begin{prb}
Riemannian and Hermitian metrics
spin structures
\end{prb}

\section{Classifying spaces}

In this section(?), our goal is to construct $BG$ and prove the natural isomorphism $\mathrm{Prin}_G\to[-,BG]$.

pullback bundles: universal property, functoriality, restriction,
section prolongation

\begin{prb}[Pullback bundles]
Let $p:E\to B$ be a $G$-bundle with fiber space $F$.
\begin{parts}
\item If $A$ is paracompact and $f_0,f_1:A\to B$ are homotopic, then $G$-bundles $f_0^*p$ and $f_1^*p$ are isomorphic.
\end{parts}
\end{prb}
\begin{pf}

If $p:E\to B\times[0,\frac12]$ and $p':E'\to B\times[\frac12,1]$ are trivial, then 


If $f,g:B'\to B$ are homotopic, then $f^*\xi\cong g^*\xi$.
\end{pf}


\begin{prb}[Universal principal bundles]
Representability of
\[\mathrm{Prin}_G:\mathrm{Top}^{\mathrm{op}}\to\mathrm{Set}.\]
Milnor construction.

\begin{prb}
\item If $EG\to BG$ is universal and $BG$ is paracompact, then $BG$ is unique up to homotopy.
\item If $EG$ is contractible and $BG$ is paracompact, then $EG\to BG$ is universal.
\end{prb}
\end{prb}




\section{Characteristic classes}

\begin{prb}
Let $G$ be a topological group.
Let $A$ be an abelian group and $n$ a positive integer.
A \emph{characteristic class} is a cohomology class on the classifying space, i.e.~an element of $H^n(BG,A)$.
A characteristic class $c\in H^n(BG,A)$ gives rise to a natural transformation $c:\mathrm{Prin}_G\to H^n(-,A)$ via the Brown representation and the Yoneda lemma.
Explicitly, for each $X\in\mathrm{Top}$, we have $c:\mathrm{Prin}_G(X)\to H^n(X,A):E\mapsto f^*c$, where $E\cong f^*(EG)$.
Using the Thom isomorphism and Gysin sequences, we can compute the cohomology of classifying spaces, which makes the classification of principal bundles

Suppose we have $BG=K(A,n)$.
Take $c\in H^n(BG,A)$ which corresponds to the identity of $\End(A)$ in the isomorphism
\[H^n(BG,A)=H^n(K(A,n),A)\cong\Hom(H_n(K(A,n),\Z),A)=\Hom(\pi_n(K(A,n)),A)=\Hom(A,A)=\End(A),\]
which follows from the Hurewicz theorem.
Then, we can show that $c$ defines a natural isomorphism $c:\mathrm{Prin}_G\to H^n(-,A)$.
In this case, we can say $c$ is the \emph{characteristic class} for $G$-bundles.
\end{prb}

\begin{enumerate}
\item Real line bundles are classified by the first Stiefel-Whitney class $w_1\in H^1(BG,A)\cong Aw_1$, where
\[G:=\GL(1,\R)=\Z/2\Z,\qquad A:=\Z/2\Z,\qquad BG=\RP^\infty=K(A,1).\]
\item Complex line bundles are classified by the first Chern class $c_1\in H^2(BG,A)\cong Ac_1$, where
\[G:=\GL(1,\C)=\T,\qquad A:=\Z,\qquad BG=\CP^\infty=K(A,2).\]
\item Real vector bundles.
\[G:=\GL(r,\R),\qquad A:=\Z/2\Z,\qquad BG=\mathrm{Gr}_r(\R^\infty).\]
By Thom and Gysin, we can compute the cohomology ring
\[H^*(BG,A)\cong A[w_1,\cdots,w_r].\]
\end{enumerate}

\begin{prb}[Thom isomorphism]
Let $E$ be a real vector bundle of rank $r$ over a paracompact arc-connected space $B$.
Let $A$ be a principal ideal domain.
Then, there is unique $t(E)\in H^r(E,E_0,A)$ whose image under $H^r(E,E_0,A)\to H^r(E_b,(E_b)_0,A)\cong A$ is the unit for each $b\in B$.
With this $t$, we have an isomorphism
\[\cdot\smile t(E):H^i(E)\to H^{i+r}(E,E_0),\qquad i\ge0.\]
The Euler class can be defined from Thom class.
\end{prb}

\section*{Exercises}

\begin{prb}
Let $G$ be a topological group, and $X$ be a free right $G$-space.
\begin{parts}
\item If the action is proper and the projection $X\to X/G$ admits local sections, then $X\to X/G$ is a principal $G$-bundle.
\end{parts}
\end{prb}

\begin{prb}[Clutching functions]
	
\end{prb}

\begin{prb}
Suppose $F\to E\to B$ is a principal 
\begin{parts}
\item If $X$ is contractible, then $X\to$
\end{parts}
\end{prb}

\begin{prb}[Group quotients]
Sufficient conditions for principal bundles.
Let $G$ be a Lie group and $M$ be a free right smooth $G$-manifold.
\begin{parts}
\item If $G$ is compact, then $M\to M/G$ is a principal $G$-bundle. (Gleason)
\item The irrational slope provides a counterexample if $G$ is not compact.
\item Suppose $M$ is a Lie group.
If $G$ is a closed subgroup of $M$, then $M\to M/G$ is a principal $G$-bundle. (Samelson) In particular, if $N$ is a transitive left smooth $M$-manifold such that $G$ is the isotropy group, then $M\to N$ is a principal $G$-bundle.
\end{parts}
\end{prb}
\begin{pf}
(a) We need to check the local triviality from smoothness and the properness from compactness.
\end{pf}

\begin{prb}[Homogeneous spaces]
They are all principal bundles.
\begin{alignat*}{2}
\rO(n-k)\to\rO(n)\to V_k(\R^n),&&\qquad\qquad\qquad\rU(n-k)\to\rU(n)\to V_k(\C^n),&\\
\rO(n-k)\times\rO(k)\to\rO(n)\to G_k(\R^n),&&\rU(n-k)\times\rU(k)\to\rU(n)\to G_k(\C^n),&\\
T(n)\cap\rO(n)\to\rO(n)\to F(\R^n),&&T(n)\cap\rU(n)\to\rU(n)\to F(\C^n),&\\
&&T(n)\to\GL(n,\C)\to F(\C^n),&
\end{alignat*}
where $T(n)$ is the group of invertible upper triangular matrices.
\[\SO(n)\to\SO^+(1,n)\to\H^n,\qquad\PSO(2)\to\PSL(2,\R)\to\H^2,\qquad ??\to\PSL(2,\C)\to\H^3,\]
where $\PSL(2,\R)\cong\SO(1,2)^+$ is the modular group and $\PSL(2,\C)\cong\SO(1,3)^+$ is the restricted Lorentz group, also called the M\"obius group.
\end{prb}

\begin{prb}[Hopf fibration]
A principal $S^1$-bundle $S^1\to S^3\to S^2$, where we see $S^1$ as the circle group.
The Hopf fibrations are used in describing universal principal bundles off orthogonal or unitary groups.
We have principal bundles:
\begin{parts}
\item The quaternionic construction gives $S^3\to S^7\to S^4$ and the octonianic construction gives $S^7\to S^{15}\to S^8$. Adams' theorem.
\item $\rO(k)\to V_k(\R^n)\to G_k(\R^n)$. In particular, $\Z/2\Z\to S^n\to\RP^n$ for $k=1$.
\item $\rU(k)\to V_k(\C^n)\to G_k(\C^n)$. In particular, $S^1\to S^{2n+1}\to\CP^n$ for $k=1$.
\end{parts}
\end{prb}


Hopf fibration(real, complex, quaternionic, but not octonianic)

In the category of smooth manifolds, if $f$ diffeomorphic, then $\tilde f$ diffeomorphic.



\begin{prb}
\begin{parts}
\item $\R\to S^1$ for $\Z$.
\item $S^\infty\to\CP^\infty$ for $\rU(1)$.
\item $?\to(S^1)^{\vee n}$ for $F_n$.
\item $S^\infty\to\RP^\infty$ for $\Z/2\Z$.
\item $V_n(\R^\infty)\to\mathrm{Gr}_n(\R^\infty)$ for $\rO(n)$.
\end{parts}
\end{prb}





\chapter{}

\section{Obstruction theory}
\section{Hurewicz theorem}
$H_\bullet(\Omega S_n)$ and $H_\bullet(U(n))$
Spin, Spin$_\C$ structure




\chapter{Simplicial methods}














\part{Stable homotopy theory}


\chapter{}
\section{Homotopy groups of spheres}

Freudenthal suspension theorem.
Spanier-Whitehead category, does not contain reduced cohomology theory?
Boardman's stable homotopy category, Lima's notion of spectra, Kan's semi-simplicial category, Whitehead's notion of spectra, and finally Adams's construction of stable homotopy category.
Bousfield localization is a kind of a category of fractions and Adamas spectral sequence.
It leads to the chromatic homotopy.

A commutative monoidal point-set model for the stable homotopy category?
Coordinate-free spectra by May, but commutative and associative only up to homotopy.
$S$-modules in EKMM97, symmetric spectra in HSS00, which are shown Quillen equivalent in Sch01.
They give closed symmetric monoidal model categories of spectra and model categories of ring spectra.
HPS97 axiomatize the stable homotopy theories.



\begin{prb}[Freudenthal suspension theorem]
For each $r$, we have the suspension homomorphism
\[E_n:\pi_{n+r}(S^n)\to\pi_{n+r+1}(S^{n+1}),\qquad n\ge0,\]
which are isomorphisms if $n>r+1$.
The \emph{stable homotopy groups of spheres} is $\lim_n\pi_{n+r}(S^n)$.

For example, it is known that $\pi_{n+1}(S^n)\cong\pi_4(S^3)\cong\Z/2\Z$ for $n>2$.
The computation $\pi_4(S^3)$ is a nice exercise which is done by Serre in his thesis.
\end{prb}


Note that $\pi_{n+r}(S^n)=[S^{n+r},S^n]$.
Suspension $\Sigma:=S^1\wedge-$ is a functor, so it defines a function $S:[X,Y]\to[\Sigma X,\Sigma Y]$.


Spanier-Whitehead and Adama category of spectra.
Triangulatedness of the homotopy category of a stable model category.
a symmetric monoidal smash product and an internal function object.




\part{Generalized cohomology theories}


\chapter{}
\begin{prb}

Note that $\mathrm{hSpc}=\mathrm{hKan}=\mathrm{hCW}$ is triangulated.


\end{prb}
\begin{prb}[Generalized cohomology theories]
A \emph{generalized cohomology theory}, or simply a \emph{cohomology theory}, is defines as a homotopy invariant product-preserving contravariant functor $E^\bullet:\mathrm{Spc}_*^\mathrm{op}\to\mathrm{grAb}$, such that
\begin{enumerate}[(i)]
\item a cofiber sequence is mapped to an exact sequence,\hfill(half-exact)
\item a natural isomorphism $\sigma:E^{\bullet+1}\circ\Sigma\to E^\bullet$ exists, called the \emph{suspension isomorphism}.\hfill(long exact)
\end{enumerate}
A cohomology theory is called \emph{multiplicative} if....

$E^1(\Sigma X)=E^0(X)$.

$K_0(\Sigma A)=K_1(A)$.
\end{prb}





Two motivations for spectra:
\begin{itemize}
\item representation of cohomology theories
\item suspension stabilization
\end{itemize}


\begin{prb}
Let $X$ and $Y$ be pointed CW complexes.
\begin{parts}
\item Suppose $Y$ is $(n-1)$-connected with non-degenerate base point for some $n$. Then, $[X,Y]\to[\Sigma X,\Sigma Y]$ is surjective if $\dim X\le 2n-1$, and bijective if $\dim X\le2n-2$.
\end{parts}
\end{prb}



\begin{prb}
A \emph{spectrum} is a sequence $E:=(E_n)_n$ of pointed spaces together with structure maps, either $\sigma_n:\Sigma E_n\to E_{n+1}$ or $\sigma'_n:E_n\to\Omega E_{n+1}$.
We have
\[[X,E_n]\xrightarrow{\sigma'_n}[X,\Omega E_{n+1}]=[\Sigma X,E_{n+1}].\]
\end{prb}

\begin{prb}[Properties of spectra]
A spectrum $E=(E_n)_n$ is called an \emph{$\Omega$-spectrum} if $\sigma'_n:E_n\to\Omega E_{n+1}$ is a weak homotopy equivalence.
A \emph{ring spectrum} is a spectrum together with a 
\begin{parts}
\item $E$ is an $\Omega$-spectrum if and only if $[-,E_n]$ defines a generalized reduced cohomology theory on based CW complexes.
\end{parts}
\end{prb}




Sphere spectra, Suspension spectra
Eilenberg-MacLane spectra(ordinary cohomology theories), K-theory spectra(K-theories), Thom spectra(cobordism theories)



Let $E^*$ be a (generalized) cohomology theory.
Then, the computation of $\Nat([-,BO(n)],E^*)\cong E^*(BO(n))$ determines all characteristic classes of real vector bundles.




equivariant topology
chromatic homotopy theory
spectral sequences
orthogonal spectra
abstract homotopy theory
Kervaire invariant problem






\chapter{K-theory}


\begin{prb}[K-theory of locally compact Hausdorff spaces]
compactly supported?
one-point compactification?
representability?

What relations do we have?
\end{prb}

\chapter{Cobordisms}




\end{document}