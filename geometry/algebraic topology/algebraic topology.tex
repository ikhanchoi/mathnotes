\documentclass{../../large}
\usepackage{../../ikhanchoi}


\begin{document}
\title{Algebraic Topology}
\author{Ikhan Choi}
\maketitle
\tableofcontents


\part{Homology}


\chapter{Axiomatic homology}

\section{Singular homology}

\section{Eilenberg-Steenrod axioms}
Mayer-Vietoris sequence




\chapter{Homology groups}

\section{Cellular homology}
CW complex,
equivalence,

\section{Simplicial homology}
geometric realization,
equivalence,
smith normal form,
simplicial approximation,




\chapter{Cohomology}

cup product
universal coefficient theorem


\section{Poincar\'e duality}




\part{Homotopy}

\chapter{Homotopy groups}




\chapter{Fibration}
\section{Homotopy lifting property}


Locally trivial bundles

pullback bundles: universal property, functoriality, restriction,
section prolongation

\section{Obstruction theory}
\section{Hurewicz theorem}
$H_\bullet(\Omega S_n)$ and $H_\bullet(U(n))$
Spin, Spin$_\C$ structure



\chapter{Spectral sequences}
\section{Serre spectral sequence}
	(Lyndon-Hochschild-Serre)
\section{Adams spectral sequence}











\part{Fiber bundles}
% Andrew Kobin

\chapter{Fiber bundles}


\section{Principal bundles}

\begin{prb}[Structure groups]
Let $G$ be a topological group and $F$ be a left $G$-space, and $p:E\to B$ be a fiber bundle with fiber $F$.
We say an atlas $\{\f_i:p^{-1}(U_i)\to U_i\times F\}_i$ is a \emph{$G$-atlas} if there is a set $\{g_{ij}:U_i\cap U_j\to G\}_{i,j}$ of maps such that the transition maps are given by
\[\f_j\circ\f_i^{-1}(b,f)=(b,g_{ij}(b)f),
\qquad b\in U_i\cap U_j,\ f\in F.\]

A \emph{$G$-bundle} with fiber $F$ is a fiber bundle $p:E\to B$ that admits a $G$-atlas.
In this case the group $G$ is called the \emph{structure group} of the fiber bundle.
A \emph{$G$-bundle map} is a bundle map $(\tilde u,u):(E,B)\to(E',B')$ between $G$-bundles together with a set $\{h_{ij'}:U_i\cap u^{-1}(U'_{j'})\to G\}_{i,j'}$ such that
\[\f'_{j'}\circ\tilde u\circ \f_i^{-1}(b,f)=(u(b),h_{ij'}(b)f),
\qquad b\in U_i\cap u^{-1}(U'_{j'}),\ f\in F.\]
If $B=B'$, a $G$-bundle map over $B$ is a $G$-bundle map $(\tilde u,u)$ such that $u=\id_B$.
We denote by $\mathbf{Bun}_F(B)$ the category of $G$-bundles over $B$ with fiber $F$.
\begin{parts}
\item If $F$ is a locally compact and locally connected Hausdorff space, then every fiber bundle with fiber $F$ is a $\Homeo(F)$-bundle, where $\Homeo(F)$ is the group of autohomeomorphism group with compact-open topology.
\item A $G$-bundle map $(\tilde u,u)$ is an isomorphism if and only if $u$ is a homeomorphism.
\item A bundle map $(\tilde u,\id_B):(E,B)\to(E',B)$ is a $G$-bundle map if and only if there is a set $\{h_i:U_i\to G\}_i$ such that
\[\f_i'\circ\tilde u\circ\f_i^{-1}(b,f)=(b,h_i(b)f),
\qquad b\in U_i,\ f\in F,\]
where $\{U_i\}$ is an open cover over which both $E$ and $E'$ are trivialized.
\end{parts}
\end{prb}
\begin{pf}
(a)

(b)
($\Rightarrow$)
Clear.

($\Leftarrow$)
The total map $\tilde u$ is continuous bijection because $u$ is a bijection, so it suffices to show $\tilde u^{-1}$ is continuous.
Fix $U_i\subset B$ and $U'_{j'}\subset B'$.
By substitution of $b':=u(b)$, $f':=h_{ij'}(b)f$, we can write
\[\f_i\circ\tilde u^{-1}\circ\f_{j'}'^{-1}(b',f')=(u^{-1}(b'),h_{ij'}(u^{-1}(b'))^{-1}f').\]
Since the local trivializations, the inverse operation of $G$, and the inverse $u^{-1}$ are all continuous, $\tilde u^{-1}$ is also continuous.
\end{pf}



\begin{prb}[Fiber bundle construction theorem]
Let $\cU=\{U_i\}_i$ be an open cover of a topological space $B$, and $G$ be a topological group.
A \emph{\v Cech 1-cocyle} on $\cU$ with coefficients in $G$ is a set $\{g_{ij}:U_i\cap U_j\to G\}_{i,j}$ of maps such that the following \emph{cocycle condition} holds:
\[g_{ik}(b)=g_{jk}(b)g_{ij}(b),\qquad b\in U_i\cap U_j\cap U_k.\]
The set of \v Cech 1-cocycles on $\cU$ with coeffients in $G$ is denoted by $\check Z^1(\cU,G)$.

Let $g\in\check Z^1(\cU,G)$ be a \v Cech 1-cocycle on $\cU$.
We will construct a $G$-bundle with fiber $F$ for any left $G$-space $F$, which is trivialized over $\cU$ in which the transition maps are given by $\{g_{ij}\}$.
Define
\[E:=\left(\coprod_i(U_i\times F)\right)/\sim,\]
where $\sim$ is an equivalence relation generated by
\[(b,f,i)\sim(b,g_{ij}(b)f,j),
\qquad b\in U_i\cap U_j,\ f\in F.\]
Also define $p:E\to B:[b,f,i]\mapsto b$ and $\f_i^{-1}:U_i\times F\to p^{-1}(U_i):(b,f)\mapsto[b,f,i]$, which are clearly continuous and surjective even without the cocycle condition.
\begin{parts}
\item $\f_i^{-1}$ is injective.
\item $\f_i^{-1}$ is open.
\item The transition maps of the local trivialization $\{\f_i\}$ coincides with the cocycle $\{g_{ij}\}$.
\end{parts}
\end{prb}
\begin{pf}
(a)
Suppose $\f_i^{-1}(b,f)=\f_i^{-1}(b',f')$.
Since $(b,y,i)\sim(b',y',i)$, we have $b=b'$ and there is a sequence
\[f'=g_{i_{n-1}i_n}(b)g_{i_{n-2}i_{n-1}}(b)\cdots g_{i_0i_1}(b)f,\]
where $i_0=i_n=i$.
By applying the cocycle condition inductively, we obtain $f=f'$, which implies the injectivity of $\f_i^{-1}$.

(b)
The map $\f_i^{-1}$ factors through $\coprod_i(U_i\times F)$ such that
\[\f_i^{-1}:U_i\times F\to\coprod_i(U_i\times F)\xrightarrow{\pi}p^{-1}(U_i).\]
Since the canonical inclusion to disjoint union is open, it suffices to show the quotient map $\pi:\coprod_i(U_i\times F)\to E$ is open.
Let $V\subset\coprod_i(U_i\times F)$ be open.
Observe that
\[\pi^{-1}\pi(V\cap(U_i\times F))\cap(U_j\times F)\]
is open for each pair of $i$ and $j$ because it is exactly same as the inverse image of the open set $V\cap(U_i\times F)$ under the map
\[(U_i\cap U_j)\times F\subset U_j\times F\to U_i\times F:(b,f)\mapsto(b,g_{ij}(b)f).\]
Here we used the cocycle condition of $\{g_{ij}\}$.
Therefore,
\[\pi^{-1}\pi(V)=\bigcup_{i,j}\pi^{-1}\pi(V\cap(U_i\times F))\cap(U_j\times F)\]
is open, hence the open $\pi$.

(c)
Clear by the cocycle condition.
\end{pf}


\begin{prb}[Cohomologous transitions]
Let $\cU=\{U_i\}_i$ be an open cover of a topological space $B$, and $G$ be a topological group. 
A \emph{\v Cech 0-cochain} on $\cU$ with coefficients in $G$ is a set $\{h_i:U_i\to G\}_i$ of maps.
The group of \v Cech 0-cochains on $\cU$ with coefficients in $G$ is denoted by $\check C^0(\cU,G)$.

The \emph{first \v Cech cohomology group} of $\cU$ with coefficients $G$ is the orbit space of an action on $\check Z^1(\cU,G)$ by $\check C^0(\cU,G)$ defined as follows:
\[(hg)_{ij}(b):=h_j(b)g_{ij}(b)h_i(b)^{-1},
\qquad b\in U_i\cap U_j,\]
which is denoted by $\check H^1(\cU,G)$.
We define the \emph{first \v Cech cohomology group} of $B$ with coefficients in $G$ as the direct limit
\[\check H^1(B,G):=\lim_{\substack{\longrightarrow\\\cU}}\check H^1(\cU,G).\]

Let $F$ be a left $G$-space, and let $\mathrm{Bun}_F(B)$ be the set of isomorphism classes of $G$-bundles over $B$ with fiber $F$.
\begin{parts}
\item $\mathrm{Bun}_F(B)\to\check H^1(B,G)$ is well-defined.
\item $\mathrm{Bun}_F(B)\to\check H^1(B,G)$ is surjective.
\item $\mathrm{Bun}_F(B)\to\check H^1(B,G)$ is injective if $F$ is faithful.
\end{parts}
\end{prb}
\begin{pf}
(a)
Suppose $p:E_1\to B$ and $p':E'\to B$ be isomorphic $G$-bundles with fiber $F$.
Let $u:E\to E'$ be a $G$-bundle isomorphism.
By considering the refinement, we can find an open cover $\cU=\{U_i\}_i$ of $B$ on which $E$ and $E'$ are simultaneously locally trivialized.
\[\{g_{ij}:U_i\cap U_j\to G\}.\]

(b)

(c)
\end{pf}


\begin{prb}[Principal bundles]
Let $G$ be a topological group, and $X$ be a left \emph{principal homogeneous $G$-space}, i.e. a free and transitive left $G$-space such that the shear map $G\times X\to X\times X:(g,x)\mapsto(gx,x)$ is a homeomorphism.

A \emph{principal $G$-bundle} is a $G$-bundle $p:P\to B$ with fiber $X$, often together with a fiber-preserving continuous right action $\rho:P\times G\to P$ such that each chart $\f_i:p^{-1}(U_i)\to U_i\times X$ induces a principal homogeneous right action on $\{b\}\times X\subset U_i\times X$ which commutes with the left action.
The right action $\rho$ is called the \emph{principal right action} or \emph{(global) gauge transformation}.
Note that for each $b\in B$ the \emph{fiber} $\{b\}\times X$ has commuting left and right actions, but the \emph{fiber} $p^{-1}(b)$ can admit only the principal right action.

The category of principal $G$-bundles over $B$ is denoted by $\mathbf{Prin}_G(B)$, and the morphisms are usually defined as right $G$-equivariant maps with respect to the pricipal right actions.
Then, we may consider the forgetful functor $\mathbf{Prin}_G(B)\to\mathbf{Bun}_X(B)$.
\begin{parts}
\item $\mathbf{Prin}_G(B)\to\mathbf{Bun}_X(B)$ is fully faithful, i.e. a bundle map $u:P\to P'$ over $B$ is a $G$-bundle map if and only if it is a right $G$-equivariant map. 
\item $\mathbf{Prin}_G(B)\to\mathbf{Bun}_X(B)$ is surjective, i.e. every $G$-bundle with fiber $X$ has a principal right action.
\item A principal bundle is trivial if it has a global section.
\end{parts}
\end{prb}
\begin{pf}
(a)
($\Rightarrow$)
Let $u:P\to P'$ be a $G$-bundle map over $B$ so that there is a set $\{h_i:U_i\to G\}_i$ of maps such that
\[\f_i\circ u\circ\f_i^{-1}(b,x)=(b,h_i(b)x),
\qquad b\in U_i,\ x\in X.\]
If we write $\rho_s:P\to P:e\mapsto \rho(e,s)$ for $s\in G$, then the induced right action $\f_i\circ\rho_s\circ\f_i^{-1}$ commutes with the left action $\f_i\circ u\circ\f_i^{-1}$ on $U_i\times X$.
Now for every $e\in P_1$, we have
\begin{align*}
\rho_s\circ u(e)
&=\f_i^{-1}\circ(\f_i\circ\rho_s\circ\f_i^{-1})\circ(\f_i\circ u\circ\f_i^{-1})\circ\f_i(e)\\
&=\f_i^{-1}\circ(\f_i\circ u\circ\f_i^{-1})\circ(\f_i\circ\rho_s\circ\f_i^{-1})\circ\f_i(e)\\
&=u\circ\rho_s(e),
\end{align*}
therefore $u$ is right $G$-equivariant.

($\Leftarrow$) let $u:P\to P'$ be a right $G$-equivariant map.
By fixing $x_0\in X$ and using the fact that the left action is free and transitive, define $g_i:U_i\to G$ such that
\[(b,g_i(b)x_0):=\f_i\circ u\circ\f_i^{-1}(b,x_0).\]
The function $g_i$ is continuous since it factors as
\[b\mapsto(b,x_0)\xmapsto{\f_i\circ u\circ\f_i^{-1}}(b,g_i(b)x_0)\mapsto g_i(b)x_0\mapsto g_i(b).\]
The continuity of the last map is due to the assumption that the map $(g,x)\mapsto(gx,x)$ is a homeomorphism.

Then, for every $(b,x)\in U_i\times X$ there is a unique $s\in G$ such that
\[\f_i\circ\rho_s\circ\f_i^{-1}(b,x_0)=(b,x),\]
so we have
\begin{align*}
\f_i\circ u\circ\f_i^{-1}(b,x)
&=(\f_i\circ u\circ\f_i^{-1})\circ(\f_i\circ\rho_s\circ\f_i^{-1})(b,x_0)\\
&=\f_i\circ u\circ\rho_s\circ\f_i^{-1}(b,x_0)\\
&=\f_i\circ\rho_s\circ u\circ\f_i^{-1}(b,x_0)\\
&=(\f_i\circ\rho_s\circ\f_i^{-1})\circ(\f_i\circ u\circ\f_i^{-1})(b,x_0)\\
&=(\f_i\circ\rho_s\circ\f_i^{-1})g_i(b)(b,x_0)\\
&=g_i(b)(\f_i\circ\rho_s\circ\f_i^{-1})(b,x_0)\\
&=g_i(b)(b,x)\\
&=(b,g_i(b)x).
\end{align*}
Hence, $u$ is a $G$-bundle map.

(b)
Fix $x_0\in X$ and consider the homeomorphism $G\to X:g\to gx_0$.
Define a right action
\[X\times G\to X:(gx_0,s)\mapsto gx_0s:=gsx_0.\]
It defines a right principal homogeneous $X$ that commutes with the left action on $X$.

Define $\rho:P\times G\to P$ such that
\[\f_i\circ\rho_s\circ\f_i^{-1}(b,x)=(b,xs).\]
It is well defined, fiber preserving, continuous.
also for any $b$ and any chart $\f_j$ containing $b$, the action on $\{b\}\times X$ defines a principal homogeneous as we have seen.
Therefore, $\rho$ is a gauge tranformation.

(c)
($\Rightarrow$)
Clear.

($\Leftarrow$)
Let $s:B\to E$ be a global section and define
\[\tilde u:B\times X\to E:(b,gx_0)\mapsto s(b)g\]
for any fixed $x_0\in X$.
Then, the continuous map $(\tilde f,\id_B)$ preserves fibers and defines a right $G$-equivariant isomorphism.
\end{pf}

\begin{prb}[Quotient principal bundles]

\end{prb}


\section{Classifying spaces}

Let $\mathrm{Prin}_G(B)$ be the set of isomorphism classes of principal $G$-bundles.
Then, we have a contravariant functor
\[\mathrm{Prin}_G:\mathbf{hTop}_{\mathrm{para}}\to\mathbf{Set}\]
such that there is a natural isomorphism between contravariant functors
\[[-,BG]\to\mathrm{Prin}_G.\]


\begin{prb}[Homotopy properteis]
Let $p:E\to B$ be a vector bundle
\begin{parts}
\item If $p:E\to B\times[0,\frac12]$ and $p':E'\to B\times[\frac12,1]$ are trivial, then 
\item If $f,g:B'\to B$ are homotopic, then $f^*\xi\cong g^*\xi$.
\end{parts}
\end{prb}

\begin{prb}[Finite type]

\end{prb}

\section{Reduction of structure groups}

\section{Vector bundles}
subbundles, quotient bundles, bundle maps,
constant rank, then ker, im, coker bundles are locally trivial so that they are vector bundles.
pullback: vector bundle structure

vector fields(trivial subbundles), parallelizable
bundle operations: sum, tensor, dual, hom, exterior

reduction and metrics

\begin{prb}[Vector bundles]
Let $p:E\to B$ and $p:E'\to B$ be vector bundles.
\begin{parts}
\item A vector bundle map $u$ over $B$ is a vector bundle isomorphism if and only if it is a fiberwise linear isomorphism.
\end{parts}
\end{prb}



Let $1\le n\le\infty$.
If $f,g:B\to G_k(\F^n)$ such that $f^*(\gamma_{k^n})\cong g^*(\gamma_{k^n})$, then $jf\simeq jg$, where $j:G_k(\F^n)\to G_k(\F^{2n})$ is the natural inclusion.


\begin{prb}
Riemannian and Hermitian metrics
\end{prb}

\section*{Exercises}

group quotient gives a principal G-bundle.

Hopf fibration(real, complex, quaternionic, but not octonianic)

In the category of smooth manifolds, if $f$ diffeomorphic, then $\tilde f$ diffeomorphic.


\begin{prb}[Associated bundles]

\[\mathrm{Prin}_G(B)\xrightarrow{\sim}\mathrm{Bun}_X(B)\xrightarrow{\sim}\check H^1(B,G)\hookrightarrow\mathrm{Bun}_F(B)\]
can be given in a more simple way.

\end{prb}




\chapter{Characteristic classes}


\chapter{K-theory}

bott periodicity
Hopf invariant





\part{Stable homotopy theory}
equivariant topology
chromatic homotopy theory
spectral sequences
orthogonal spectra
abstract homotopy theory
Kervaire invariant problem

\end{document}