\documentclass{../../large}
\usepackage{../../ikhanchoi}

\newcommand{\Cl}{\operatorname{Cl}}
\newcommand{\Pic}{\operatorname{Pic}}
\newcommand{\Proj}{\operatorname{Proj}}

\begin{document}
\title{Algebraic Geometry}
\author{Ikhan Choi}
\maketitle
\tableofcontents

\part{}


\chapter{Schemes}


\chapter{Morphisms}

\section{}
smooth, finite type, proper, regular, dominant, unramified, flat, complete intersection
closed immersion


direct image, inverse image




\begin{prb}[Integral schemes]
A scheme $X$ is said to be \emph{integral} if it is non-empty and every non-empty affine open is isomorphic to the spectrum of an integral domain.
A \emph{generic point} of a topological space is a point whose closure is the whole space.
\begin{parts}
\item A scheme is integral if and only if it is reduced and irreducible.
\item An integral scheme has a unique generic point.
\end{parts}
\end{prb}


\begin{prb}[Projective schemes]
We say a variety is \emph{projective} if it is isomorphic to a closed subvariety of $\P^n$ for some $n$.

For a fixed a base ring $A$, let $S$ be a $\Z_{\ge0}$-graded ring such that $S_0=A$, and define the \emph{irrelavent ideal} $S_+:=\bigoplus_{i\ge1}S_i$ of $S$.
The \emph{Proj construction} of $S$ is a scheme $\Proj S$ constructed as follows.
The set $\Proj S$ consists of all homogeneous prime ideals of $S$ not containing $S_+$, the topology is determined by setting $V(\fa):=\{\fp\in\Proj S:\fa\subset\fp\}$ as closed sets where $\fa$ runs through the homogeneous ideals of $S$, and the structure sheaf defined such that $\cO_{\Proj S}(D(f)):=S_{((f))}$ for homogeneous $f\in S_+$, where $S_{(\fp)}:=(S_\fp)_0$ denotes the zeroth graded piece of localized $\Z$-graded rings $S_\fp$, and the set $D(f):=\Proj S\setminus V(f)$ is called a \emph{standard open} of $\Proj S$, which can be shown to be affine.

There is a canonical $\Z$-graded $\cO_{\Proj S}$-modules, of which the graded pieces $\cO(i)$ are line bundles called the \emph{Serre twisting sheaves}.
\end{prb}

Some analysis of line bundles construct projective embeddings


\chapter{Quasi-coherent sheaves}



\chapter{Curves}


In general, a variety over $k$ is meant by an integral(=reduced+irreducible) scheme which is separated and of finite type.
We wan to classify 
\begin{itemize}
\item Hartshorne: integral scheme of dimension 1 which is proper and regular.
\item Vakil: integral scheme of dimension 1 which is projective and regular.
\end{itemize}
I think they are equivalent to smooth complete curves.

\section{Preliminaries}
Invariants
\begin{itemize}
\item genus: $p_a(X)=p_g(X)=h^1(\cO_X)$
\item Weil vs Cartier divisor groups: $\Cl(X)\cong\Pic(X)$
\end{itemize}

Computation tools
\begin{itemize}
\item $|D|\leftrightarrow PH^0(X,\cL(D))$ so that $|D|$ is identified as a projective space
\item $\Omega_X\cong\omega_X$
\item Riemann-Roch theorem: $l(D)-l(K-D)=\deg D+1-g$
\item Hurwitz theorem: $2g(X)-2=\deg f\cdot(2g(Y)-2)+\deg R$
\end{itemize}


birational iff isomorphic
A morphism $f:X\to Y$ induces a field extension $\cK(X)/\cK(Y)$.

\section{Lower genus}
elliptic: invariants, moduli space, structures
hyperelliptic: 
non-hyperelliptic: canonical embedding


\section{Classification by genus and moduli spaces}
Deligne-Mumford: $\cM_g$ for $g\ge2$ is an irreducible quasi-projective variety of dimension $3g-3$.

\section{Classification by degree in $\P^3$}

A divisor $D$ is called \emph{very ample} if $\cL(D)\cong\cO(1)$ in some closed immersion into a projective space.
A divisor $D$ is called \emph{ample} if $\cF\otimes\cL^n$ is generated by global sections for sufficiently large $n$, for each coherent sheaf $\cF$.
A \emph{linear system} is a projective subspace of some complete linear system $|D|\cong\P^{l(D)-1}$, the set of all effective divisors linearly equivalent to $D$, which is identified to a projective space.
The \emph{base locus} of a linear system $\fd$ is the set $\bigcap_{D\in\delta}\supp D$.
It is known that $|D|$ is base point free if and only if $\cL(D)$ is generated by global sections, and a linear system is base point free if and only if some embedding....?

Any choice of a finite system of non-simultaneously vanishing global sections of a globally generated line bundle defines a morphism to a projective space.
If the line bundle is very ample, then the morphism is an embedding.

chow variety or hilbert scheme


\chapter{Surfaces}




\chapter{\'Etale cohomology}

\section{}

\begin{prb}
Let $\f:Y\to X$ be a morphism of schemes.
It is called \emph{\'etale} if it is flat and unramified.
\end{prb}

\end{document}