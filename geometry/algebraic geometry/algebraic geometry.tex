\documentclass{../../large}
\usepackage{../../ikhanchoi}

\newcommand{\Cl}{\operatorname{Cl}}
\newcommand{\Pic}{\operatorname{Pic}}
\newcommand{\Proj}{\operatorname{Proj}}
\newcommand{\Sch}{\mathrm{Sch}}

\begin{document}
\title{Algebraic Geometry}
\author{Ikhan Choi}
\maketitle
\tableofcontents

\part{}


\chapter{Schemes}



\begin{prb}[Affine schemes]
Let $A$ be a ring.
Every ring will be commutative and unital if not mentioned.
The spectrum $\Spec A$ of $A$ is defined as the partially ordered set of all prime ideals of $A$.
It is topologized by the Zariski topology in which a subset of $\Spec A$ is closed if and only if it is given by the zero set $\Spec A/\fa=\{\fp\in\Spec A:\fa\subset\fp\}$ of some ideal $\fa\subset A$.
It also admits a canonical structure sheaf $\cO_{\Spec A}:\mathrm{Open}(\Spec A)^\op\to\mathrm{CRing}$ of rings characterized by
\[\cO_{\Spec A}(D(f)):=A_f=A[f^{-1}],\qquad D(f):=(\Spec A/(f))^c=\{\fp\in\Spec A:f\notin\fp\},\qquad f\in A.\]
In conclusion, a ring $A$ defines a locally ringed space $\Spec A$.
\begin{parts}
\item There is a one-to-one correspondence between Zariski closed sets and radical ideals, and a Zariski closed subset is an upper set.
\item An ideal $\fa$ of $A$ is proper if and only if the zero set $\Spec A/\fa$ is non-empty.
\end{parts}

\end{prb}

\begin{prb}[Schemes]
A \emph{scheme} is a locally affine locally ringed space.
Affine open subsets form a basis.
The existence of an affine open cover is enough.


\end{prb}


A \emph{generic point} of a topological space is a point whose closure is the whole space.
A \emph{closed point} of a topological space is a point which is closed.
specialization and generalization.
Closed points of an affine scheme are exactly maximal ideals.


\[\Spec\Z=\{(p):p\in\Z\text{ prime}\}\cup\{(0)\}.\]
\[\Spec\R[x]=\A^1_\R=\{(x-a):a\in\R\}\cup\{(f):f\in\R[x]\text{ irreducible quadratic}\}\cup\{(0)\}.\]
\[\Spec\Q[x]=\A^1_\Q\]
\[\Spec\F_p[x]=\A^1_{\F_p}=\{(f):f\in\F_p[x]\text{ irreducible}\}\cup\{(0)\}.\]
\[\Spec\C[x,y]=\A^2_\C=\{(x-a,y-b):(a,b)\in\C^2\}\cup\{(f):f\in\C[x,y]\text{ irreducible}\}\cup\{(0)\}.\]


Nulstellensatz states that the set of closed points of the affine scheme $\A^n$ over an algebraically closed field $k$ is exactly $k^n$.
It connects the theory of classical algebraic geometry to scheme theory.
Zariski lemma, somtimes called the Nullstellensatz, states that for a field $k$ the residue field of a maximal ideal of $k[x_1,\cdots,x_n]$ is a finite extension of $k$.
In other words, for a field extension $K/k$, $K$ is finitely generated as $k$-modules if $K$ is finitely generated as $k$-algebras.


\begin{prb}[Functor of points]
The \emph{functor of points} of a scheme $X$ is a functor $\mathrm{Aff}^\op\to\mathrm{Set}:\Spec A\mapsto[\Spec A,X]$ or $\mathrm{Sch}^\op\to\mathrm{Set}:T\mapsto[T,X]$.
A \emph{rational point} of $X$ over a ring $A$ is a morphism $\Spec A\to X$ of schemes.

Conversely, a functor $\mathrm{Aff}^\op\to\mathrm{Set}$ is representable by scheme if and only if it is a sheaf on the site $\mathrm{Aff}$ and it has an open cover by affine schemes.
\end{prb}


\begin{prb}[Quotients and localizations]
\end{prb}

For an ideal $\fa\subset A$, the spectrum of the quotient $\Spec A/\fa$ gives a closed subset of $\Spec A$.
For an element $f\in A$, the localization is $A_f=\{1,f,f^2,\cdots\}^{-1}A$, and the spectrum $\Spec A_f$ gives a distinguished open subset of $\Spec A$ with complement $\Spec A/(f)$, which generate a topological base when $f$ runs through $A$.
For a prime ideal $\fp\subset A$, the localization $A_\fp=(A\setminus\fp)^{-1}A$ is a local ring, and the spectrum $\Spec A_\fp$ gives the set of prime ideals $\Spec A$ contained in $\fp$.


\[\Spec\C[x]_x=\Spec\C[x]\setminus\Spec\C[x]/(x)=\{(x-a):a\in\C\setminus\{0\}\}\cup\{(0)\}\]

\[\Spec\C[x]_{(x)}=\{(x)\}\cup\{(0)\}.\]
\[\Spec\C[x,y]_{(x)}=\{(x,y-b):b\in\C\}\cup\{(x)\}\cup\{(0)\}\]
\[\Spec\Z[x]\text{ over }\Spec\Z\]



\begin{prb}[Integral schemes]
Let $X$ be a scheme.
We say $X$ is \emph{reduced} if every stalk is reduced, that is, it has no non-zero nilpotents, i.e.~``a function is zero if it is zero at every point''.
We say $X$ is \emph{irreducible} if every two open subsets intersects.
It is an algebro-geometric analogue of connectedness.
We say $X$ is \emph{integral} if it is non-empty and every non-empty affine open subset is isomorphic to the spectrum of an integral domain.
\begin{parts}
\item A scheme is integral if and only if it is reduced and irreducible.
\item An integral scheme has a unique generic point $\eta$.
\item The stalk $\cO_{X,\eta}$ at the generic point is naturally identified with the field $K(A)$ of fractions, where $\Spec A$ is any non-empty affine open subset of an integral scheme $X$. So, we can define ``rational functions'' on integral schemes.
\end{parts}
\end{prb}

\begin{prb}[Separated schemes]
\emph{quasi-separated} if the intersection of any two quasi-compact open subsets is quasi-compact.
\end{prb}


\begin{prb}[Schemes of finite type]
Let $X$ be a scheme.
We say $X$ is \emph{quasi-compact} if it the Zariski topology is compact, \emph{locally noetherian} if it is covered by the spectrum of noetherian rings, and \emph{locally of finite type} (over a ring $A$) if it is covered by the spectrum of finitely generated algebras (over $A$).
A \emph{notherian} scheme is a quasi-compact locally notherian scheme, and a scheme of \emph{finite type} is a quasi-compact scheme of locally finite type.
\begin{parts}
\item A notherian scheme is automatically quasi-separated.
\item A noetherian scheme is integral if and only if it is non-empty connected and every stalk is an integral domain.
\item A scheme of finite type over a noetherian ring is noetherian.
\end{parts}
\end{prb}


\begin{prb}[Normal and factorial schemes]
\end{prb}


\section{Constructions for schemes}



\begin{prb}[Projective schemes]
We say a variety is \emph{projective} if it is isomorphic to a closed subvariety of $\P^n$ for some $n$.

For a fixed a base ring $A$, let $S$ be a $\Z_{\ge0}$-graded ring such that $S_0=A$, and define the \emph{irrelavent ideal} $S_+:=\bigoplus_{i\ge1}S_i$ of $S$.
The \emph{Proj construction} of $S$ is a scheme $\Proj S$ constructed as follows.
The set $\Proj S$ consists of all homogeneous prime ideals of $S$ not containing $S_+$, the topology is determined by setting $V(\fa):=\{\fp\in\Proj S:\fa\subset\fp\}$ as closed sets where $\fa$ runs through the homogeneous ideals of $S$, and the structure sheaf defined such that $\cO_{\Proj S}(D(f)):=S_{((f))}$ for homogeneous $f\in S_+$, where $S_{(\fp)}:=(S_\fp)_0$ denotes the zeroth graded piece of localized $\Z$-graded rings $S_\fp$, and the set $D(f):=\Proj S\setminus V(f)$ is called a \emph{standard open} of $\Proj S$, which can be shown to be affine.

There is a canonical $\Z$-graded $\cO_{\Proj S}$-modules, of which the graded pieces $\cO(i)$ are line bundles called the \emph{Serre twisting sheaves}.
\end{prb}

A quasi-projective scheme $X$ over $A$ is of finite type of $A$.
If $A$ is furthermore noetherian, then $X$ is noetherian.





\chapter{Morphisms}

\section{}
smooth, finite type, proper, regular, dominant, unramified, flat, complete intersection
closed immersion


direct image, inverse image





\chapter{Quasi-coherent sheaves}


\part{Birational geometry}

\chapter{Curves}


In general, over an algebraically closed field, a \emph{variety} refers to an integral separated scheme of finite type.
If the underlying field is not algebraically closed, the definition slightly differs depending on the references.
We define a \emph{curve} as a 1-dimensional variety, and we want to classify smooth complete curves over an algebraically closed field $k$.
I think the followings are equivalent to smooth complete curves:
\begin{itemize}
\item Hartshorne: integral scheme of dimension 1 which is proper and regular.
\item Vakil: integral scheme of dimension 1 which is projective and regular.
\end{itemize}


Representations for morphisms when varieties are embedded in a projective space.

\section{Preliminaries}
Invariants
\begin{itemize}
\item genus: $p_a(X)=p_g(X)=h^1(\cO_X)$
\item Weil vs Cartier divisor groups: $\Cl(X)\cong\Pic(X)$
\end{itemize}
The moduli stack $\cM_g$ of each genus.

Computation tools
\begin{itemize}
\item $|D|\leftrightarrow PH^0(X,\cL(D))$ so that $|D|$ is identified as a projective space
\item $\Omega_X\cong\omega_X$
\item Riemann-Roch theorem: $l(D)-l(K-D)=\deg D+1-g$
\item Hurwitz theorem: $2g(X)-2=\deg f\cdot(2g(Y)-2)+\deg R$
\end{itemize}


birational iff isomorphic
A morphism $f:X\to Y$ induces a field extension $\cK(X)/\cK(Y)$.

\section{Lower genus}
elliptic: invariants, moduli space, structures
hyperelliptic: 
non-hyperelliptic: canonical embedding


\section{Classification by genus and moduli spaces}
Deligne-Mumford: $\cM_g$ for $g\ge2$ is an irreducible quasi-projective variety of dimension $3g-3$.

\section{Classification by degree in $\P^3$}

A divisor $D$ is called \emph{very ample} if $\cL(D)\cong\cO(1)$ in some closed immersion into a projective space.
A divisor $D$ is called \emph{ample} if $\cF\otimes\cL^n$ is generated by global sections for sufficiently large $n$, for each coherent sheaf $\cF$.
A \emph{linear system} is a projective subspace of some complete linear system $|D|\cong\P^{l(D)-1}$, the set of all effective divisors linearly equivalent to $D$, which is identified to a projective space.
The \emph{base locus} of a linear system $\fd$ is the set $\bigcap_{D\in\delta}\supp D$.
It is known that $|D|$ is base point free if and only if $\cL(D)$ is generated by global sections, and a linear system is base point free if and only if some embedding....?

Any choice of a finite system of non-simultaneously vanishing global sections of a globally generated line bundle defines a morphism to a projective space.
If the line bundle is very ample, then the morphism is an embedding.

chow variety or hilbert scheme


\chapter{Surfaces}

Kodaira-Enriques

Fano three-folds


Moduli stack...?


\end{document}