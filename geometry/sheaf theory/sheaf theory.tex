\documentclass{../../large}
\usepackage{../../ikhanchoi}


\begin{document}
\title{Sheaves}
\author{Ikhan Choi}
\maketitle
\tableofcontents




\part{Sheaves}

\chapter{Sheaves}



A subset $F\subset E$ is defines a \'etale subbundle if and only if $F$ is open in $E$.
A covering space is nothing but a locally constant local homeomorphism.



\begin{prb}[Sheaves]
Let $X$ be a topological space.
The category $\mathrm{Open}(X)$ is the category defined such that objects are open sets of $X$ and morphisms are inclusions.
A \emph{presheaf} (of sets) on $X$ is a contravariant functor $\cF:\mathrm{Open}(X)^\op\to\mathrm{Set}$.
The category $\mathrm{PSh}(X)$ is defined such that objects are presheaves of sets on $X$ and morphisms are natural transformations.
For a presheaf $\cF$ on $X$, the set $\cF_x:=\colim_{U\ni x}\cF(U)$ and its element is called a \emph{stalk} and a \emph{germ} at $x$ respectively.

A \emph{sheaf} (of sets) on $X$ is a presheaf on $X$ such that for every cover $\{U_i\to U\}$ of an open set $U$ in $X$ we have a bijection
\[\cF(U)\xrightarrow{\sim}\mathrm{eq}\left(\prod_i\cF(U_i)\rightrightarrows\prod_{i,j}\cF(U_i\cap U_j)\right)=\left\{(s_i)\in\prod_i\cF(U_i):s_i|_{U_i\cap U_j}=s_j|_{U_i\cap U_j}\right\}.\]
The injectivity and the surjectivity are sometimes referred as the \emph{locality axiom} and the \emph{gluing axiom}.
The category $\mathrm{Sh}(X)$ of sheaves on $X$ is defined to be the full subcategory of $\mathrm{PSh}(X)$.
\begin{parts}
\item There is a functor $\mathrm{PSh}(X)\to\mathrm{LH}(X):\cF\mapsto(\coprod_{x\in X}\cF_x\to X)$. This functor is called the \emph{\'etale space construction}.
\item There is an equivalence $\mathrm{LH}(X)\to\mathrm{Sh}(X):(Y\to X)\mapsto(U\mapsto\Gamma(U,Y))$. The composition of the \'etale space construction is equal to the \emph{sheafification}.
\end{parts}
\end{prb}
\begin{pf}
(a)

There exists a unique natural topology on $E(\cF)$ such that $p:E\to X$ is a local homeomorphism:
We endow a topology on $E$ generated by a base $\{s(U):s\in\cF(U),\ U\text{ open}\}$.

There exists a unique natural function $\cF(U)\to\Gamma(U,E(\cF))$ such that .. for open subsets $U\subset X$:
For $x\in U$ and $s\in\cF(U)$, we define $s_x\in p^{-1}(x)$ by the image of $\cF(U)\to\cF_x$ at $s$, and it defines a morphism of presheaves $\cF\to\Gamma(E(\cF))$.

(b)
Let $p:Y\to X$ is a local homeomorphism.
The functoriality of $\Gamma(-,Y)$ is clear.
Let $\{U_i\to U\}$ be a cover of an open set $U$ in $X$.
If $s,t\in\cF(U)=\Gamma(U,Y)$ satisfies $s|_{U_i}=t|_{U_i}$ for all $i$, then $s=t$.
If $\{s_i\}$ satisfies $s_i|_{U_i\cap U_j}=s_j|_{U_i\cap U_j}$ for all $i,j$, then $x\mapsto s_i(x)$ is well defined independent of $i$ with $U_i\ni x$, and is continuous.


(?)
Sheafification of other target categories(I do not know the exact condition for it) can also be described as follows:
since the condition for sheaves is conditioned by limits, the fully faithful inclusion $\mathrm{Sh}(X)\to\mathrm{PSh}(X)$ preserves limits, so it admits a left adjoint $\cF\mapsto\cF^\sharp$.
One of the explicit construction is as follows:
\[\cF^\natural(U):=\colim_{\{U_i\to U\}\in J(U)}\mathrm{eq}\left(\prod_i\cF(U_i)\rightrightarrows\prod_{i,j}\cF(U_i\cap U_j)\right)\]
and $\cF^\sharp:=\cF^{\natural\natural}$.

\end{pf}

\begin{prb}[Morphism of sheaves]
epic and monic.
The description for $\cF(U)$ and the description for $\cF_x$.
\end{prb}


\begin{prb}[Operations on sheaves]
inverse image functor(restriction, pullback)
direct image functor(pushforward)
Whitney sum, constant sheaf, subsheaf

\'etale space descriptions
\begin{parts}
\item
\end{parts}
\end{prb}


\begin{prb}[Sheaves of rings]
Let $X$ be a topological space.

A \emph{ringed space} is a pair $(X,\cO_X)$ of a topological space $X$ and a sheaf $\cO_X$ of rings on $X$.

\end{prb}

\begin{prb}[Sheaves of module]
Let $X$ be a topological space.

A sheaf $\cF_X$ of $\cO_X$-modules is called \emph{locally finite} or \emph{finite type} if there is an open cover $\{U_i\}$ of $X$ together with surjective ring homomorphisms $\cO_X(U_i)\twoheadrightarrow\cF_X(U_i)$ for all $i$.
The section space $\Gamma(U,\cO_X)$ will be also denoted by $\cO_X(U)$.
\begin{parts}
\item The category of sheaves of modules over $\cO_X$ is abelian.
\end{parts}
\end{prb}



\begin{prb}[Coherent sheaves]
Let $(X,\cO_X)$ be a ringed space.
Consider a quasi-coherent sheaf with an exact sequence of $\cO_X$-modules
\[\cO_U^q\to\cO_U^p\to\cF_U\to0.\]
The \emph{generating system} is the basis of $\cO_X^p$, and the \emph{relation sheaf} is the kernel of $\cO_X^p\to\cF_X$.
We say $\cF_X$ is \emph{coherent} if $\cF_X$ is of finite type and the relation sheaf is also of finite type.
\begin{parts}
\item If $\cO_X$ is itself a coherent module, then every locally finitely presented $\cO_X$-module is coherent.
\end{parts}
\end{prb}



\begin{prb}[Yoga of coherent sheaves]\,
\begin{parts}
\item extension principle?
\item If a ring $\cO$ has a split epi $\cO\to\cO'$ to a coherent $\cO'$, then $\cO$ is coherent.
\end{parts}
\end{prb}
\begin{pf}
Consider the following diagram in which every row is exact and $K$, $K_1$, $K_2$ are kernels:
\[\begin{tikzcd}[sep=small]
K \rar & \cO^p \rar\dar[equals] & \cO \rar\dar[two heads] & 0\\
K_1 \rar & \cO^p \rar\dar[two heads] & \cO' \rar\dar[equals]\uar[hookrightarrow,bend right] & 0\\
K_2 \rar & \cO'^p \rar & \cO' \rar & 0.
\end{tikzcd}\]
Then, $K_2$ is finitely generated by the coherence of $\cO'$, $K_1$ is finitely generated by the Schanuel lemma, and $K$ is finitely generated by the snake lemma.
\end{pf}




\section{}

Note that affine schemes, complex model spaces, Euclidean open subsets are all locally ringed spaces.

A \emph{scheme} is a ringed space modeled on affine schemes.
A \emph{complex (analytic) space} is a Hausdorff ringed space modeled on complex model spaces.
A \emph{manifold} is a second countable Hausdorff ringed space modeled on Euclidean open subsets $(U,C(U))$.

They are all locally ringed.
For the latter two, the residue field at every stalk is isomorphic to $\C$ or $\R$.
For schemes over a field $k$, the relation between residue fields and $k$ is related to the Nullstellensatz at closed points.


\section{}



\end{document}