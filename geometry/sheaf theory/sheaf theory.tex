\documentclass{../../large}
\usepackage{../../ikhanchoi}


\newcommand{\PSh}{\mathrm{PSh}}
\newcommand{\Sh}{\mathrm{Sh}}
\newcommand{\Open}{\mathrm{Open}}

\begin{document}
\title{Sheaves}
\author{Ikhan Choi}
\maketitle
\tableofcontents




\part{Sheaves}

\chapter{Sheaves}




\begin{prb}[Sheaves]
Let $X$ be a topological space, which has a natural category $\Open(X)$ whose objects are open sets of $X$ and morphisms are inclusions.
Let $\cC$ be a category, set to be the category of sets if no mentioned, but we will see also the case when $\cC$ is the category of abelian groups or commutative unital rings.
A \emph{presheaf} of $\cC$ on $X$ is just a contravariant functor $\cF:\Open(X)^\op\to\cC$.
The category $\PSh(X)$ is defined such that objects are presheaves of sets on $X$ and morphisms are natural transformations.

For a point $x\in X$ and a presheaf $\cF$ of a filtered cocomplete category $\cC$ on $X$, the colimit $\cF_x:=\colim_{U\ni x}\cF(U)$ and its element is called a \emph{stalk} and a \emph{germ} at $x$ respectively.

Note that the set of morphisms $\{U_i\to U\}$ in $\Open(X)$ is an open cover of $U$ if and only if $U$ is given by the coequalizer in the category $\mathrm{Set}$
\[\coprod_{i,j}U_i\cap U_j\rightrightarrows\coprod_iU_i\to U.\]
A \emph{sheaf} of $\cC$ on $X$ is a presheaf of $\cC$ on $X$ such that for every open cover $\{U_i\to U\}$ of any open subset $U$ of $X$ $\cF(U)$ is given by the equalizer in $\cC$
\[\cF(U)\to\prod_i\cF(U_i)\rightrightarrows\prod_{i,j}\cF(U_i\cap U_j).\]
It is equivalent to the bijectivity of
\[\cF(U)\to\left\{(s_i)\in\prod_i\cF(U_i):s_i|_{U_i\cap U_j}=s_j|_{U_i\cap U_j}\right\},\]
and the injectivity and the surjectivity are referred as the \emph{locality axiom} and the \emph{gluing axiom}.
The category $\Sh(X)$ of sheaves on $X$ is defined to be the full subcategory of $\PSh(X)$.

Let $\mathrm{LH}(X)$ be the category of local homeomorphisms over $X$.
We construct a functor
\[\PSh(X)\to\mathrm{LH}(X)\to\Sh(X)\]
called the \emph{sheafification}, which is left adjoint to the inclusion.
The first functor to $\mathrm{LH}(X)$ is called the \emph{\'etale space construction}.
\begin{parts}
\item There is a functor $\PSh(X)\to\mathrm{LH}(X):\cF\mapsto(\coprod_{x\in X}\cF_x\to X)$.
\item There is an equivalence $\mathrm{LH}(X)\to\Sh(X):(Y\to X)\mapsto(U\mapsto\Gamma(U,Y))$.
\end{parts}
\end{prb}
\begin{pf}
(a)

There exists a unique natural topology on $E(\cF)$ such that $p:E\to X$ is a local homeomorphism:
We endow a topology on $E$ generated by a base $\{s(U):s\in\cF(U),\ U\text{ open}\}$.

There exists a unique natural function $\cF(U)\to\Gamma(U,E(\cF))$ such that .. for open subsets $U\subset X$:
For $x\in U$ and $s\in\cF(U)$, we define $s_x\in p^{-1}(x)$ by the image of $\cF(U)\to\cF_x$ at $s$, and it defines a morphism of presheaves $\cF\to\Gamma(E(\cF))$.

(b)
Let $p:Y\to X$ is a local homeomorphism.
The functoriality of $\Gamma(-,Y)$ is clear.
Let $\{U_i\to U\}$ be a cover of an open set $U$ in $X$.
If $s,t\in\cF(U)=\Gamma(U,Y)$ satisfies $s|_{U_i}=t|_{U_i}$ for all $i$, then $s=t$.
If $\{s_i\}$ satisfies $s_i|_{U_i\cap U_j}=s_j|_{U_i\cap U_j}$ for all $i,j$, then $x\mapsto s_i(x)$ is well defined independent of $i$ with $U_i\ni x$, and is continuous.


(?)
Sheafification of other target categories(I do not know the exact condition for it) can also be described as follows:
since the condition for sheaves is conditioned by limits, the fully faithful inclusion $\Sh(X)\to\PSh(X)$ preserves limits, so it admits a left adjoint $\cF\mapsto\cF^\sharp$.
One of the explicit construction is as follows:
\[\cF^\natural(U):=\colim_{\{U_i\to U\}\in J(U)}\mathrm{eq}\left(\prod_i\cF(U_i)\rightrightarrows\prod_{i,j}\cF(U_i\cap U_j)\right)\]
and $\cF^\sharp:=\cF^{\natural\natural}$.



(?)

A subset $F\subset E$ is defines a \'etale subbundle if and only if $F$ is open in $E$.
A covering space is nothing but a locally constant local homeomorphism.


\end{pf}

\begin{prb}[Morphism of sheaves]
epic and monic.
The description for $\cF(U)$ and the description for $\cF_x$.

Let $f:X\to Y$ be a continuous map between topological spaces.
\end{prb}

\begin{prb}[Image functors]
We have two naturally associated functors between the categories of sheaves given as follows.
The \emph{direct image functor} $f_*:\Sh(X)\to\Sh(Y)$ is defined by
\[\begin{tikzcd}[column sep=tiny]
\Open(X)^\op \ar{dr}
\ar[Rightarrow, shorten >=10, shorten <=10, shift right=4, swap]{rr}{f_*}
&& \Open(Y)^\op \ar[swap]{ll}{f^{-1}}\ar{dl}\\
&\mathrm{Set}
\end{tikzcd}\]
and the \emph{inverse image functor} or the \emph{pullback functor} $\Sh(X)\leftarrow\Sh(Y):f^*$ is defined by the left adjoint functor to $f_*$.

\begin{parts}
\item $f_*$ preserves small limits, and $f^*$ preserves small colimits.
\item $f^*$ is left exact in that it preserves every finite limits.
\end{parts}
\end{prb}


\begin{prb}[Operations on sheaves]
Whitney sum, constant sheaf, subsheaf

\'etale space descriptions
\begin{parts}
\item
\end{parts}
\end{prb}


\begin{prb}[Sheaves of rings]
Every ring is assumed to be commutative and unital.
A \emph{ringed space} is a topological space $X$ together with a sheaf $\cO_X$ of rings on $X$.
A \emph{morphism} between ringed spaces is a continuous map $f:Y\to X$ together with a natural transform $f^\sharp:\cO_X\to f_*\cO_Y$.

A \emph{locally ringed space} is a ringed space $X$ on which every stalk is a local ring.
A \emph{morphism} between locally ringed spaces $X$ and $Y$ is a morphism $f$ of ringed spaces such that the ring homomorphism $f^\sharp_x:\cO_{Y,f(x)}\to f_*\cO_{X,x}$ between local rings preserves the maximal ideal at each $x\in X$.

\end{prb}

\begin{prb}[Sheaves of module]
Let $X$ be a ringed space.

An $\cO_X$-module $\cF_X$ is called \emph{locally finite} or \emph{finite type} if there is an open cover $\{U_i\}$ of $X$ together with surjective ring homomorphisms $\cO_X(U_i)^p\twoheadrightarrow\cF_X(U_i)$ for all $i$.
The section space $\Gamma(U,\cO_X)$ will be also denoted by $\cO_X(U)$.
\begin{parts}
\item The category of sheaves of modules over $\cO_X$ is abelian.
\end{parts}
\end{prb}



\begin{prb}[Coherent sheaves]
Let $(X,\cO_X)$ be a ringed space.
Consider a quasi-coherent sheaf with an exact sequence of $\cO_X$-modules
\[\cO_U^q\to\cO_U^p\to\cF_U\to0.\]
The \emph{generating system} is the basis of $\cO_X^p$, and the \emph{relation sheaf} is the kernel of $\cO_X^p\to\cF_X$.
We say $\cF_X$ is \emph{coherent} if $\cF_X$ is of finite type and the relation sheaf is also of finite type.
\begin{parts}
\item If $\cO_X$ is itself a coherent module, then every locally finitely presented $\cO_X$-module is coherent.
\end{parts}
\end{prb}



\begin{prb}[Yoga of coherent sheaves]\,
\begin{parts}
\item extension principle?
\item If a ring $\cO$ has a split epi $\cO\to\cO'$ to a coherent $\cO'$, then $\cO$ is coherent.
\end{parts}
\end{prb}
\begin{pf}
Consider the following diagram in which every row is exact and $K$, $K_1$, $K_2$ are kernels:
\[\begin{tikzcd}[sep=small]
K \rar & \cO^p \rar\dar[equals] & \cO \rar\dar[two heads] & 0\\
K_1 \rar & \cO^p \rar\dar[two heads] & \cO' \rar\dar[equals]\uar[hookrightarrow,bend right] & 0\\
K_2 \rar & \cO'^p \rar & \cO' \rar & 0.
\end{tikzcd}\]
Then, $K_2$ is finitely generated by the coherence of $\cO'$, $K_1$ is finitely generated by the Schanuel lemma, and $K$ is finitely generated by the snake lemma.
\end{pf}




\section{}

Note that affine schemes, complex model spaces, Euclidean open subsets are all locally ringed spaces.


A \emph{scheme} is a locally ringed space such that affine open subsets form a basis.

A \emph{complex (analytic) space} is a Hausdorff locally ringed space such that open complex model spaces form a basis...?

A \emph{manifold} is a Hausdorff locally ringed space such that Euclidean open subsets $(U,C(U,\R))$ form a basis.



In the case of manifolds, a morphism of locally rigned spaces is determined by the underlying continuous map. (maybe)

They are all locally ringed.
For the latter two, the residue field at every stalk is isomorphic to $\C$ or $\R$.
For schemes over a field $k$, the relation between residue fields and $k$ is related to the Nullstellensatz at closed points.


\section{}


\chapter{Stacks}



\chapter{Sites}

\begin{prb}
Let $\cT$ be a category.
A \emph{sieve} over $U\in\cT$ is a subset $\{U_i\to U\}$ of $\cT/U$ that is stable under composition in $\cT$.
If $\cT$ is locally small, then it is equivalently a subfunctor of the representable functor $\Hom(-,U):\cT^\op\to\mathrm{Set}$.
(a subfunctor means that $S:\cT^\op\to\mathrm{Set}$ has $S(U_i)\subset\Hom(U_i,U)$ and each $U_j\to U_i$ in $\cT$ gives $S(U_i)\to S(U_j)$, if we define $\{U_i\to U\}:=\bigcup_{U_i\in\cT}S(U_i)$, then $S$ is closed under composition in $\cT$.)

A \emph{Grothendieck topology} or just a \emph{topology} on $\cT$ is a datum that indicates which sieve is a \emph{covering sieve}, such that
\begin{enumerate}[(i)]
\item $\{U\to U\}$ is a covering sieve,
\item if $\{U_i\to U\}$ is a covering sieve and if $V\to U$ admits $V_i:=U_i\times_UV$ for all $U_i\to U$ in the sieve, then $\{V_i\to V\}$ is a covering sieve,
\item if $\{U_i\to U\}$ is a covering sieve and $\{U_{ij}\to U_i\}$ is a covering sieve for each $i$, then $\{U_{ij}\to U\}$ is a covering sieve.
\end{enumerate}

A category equipped with a topology is called a \emph{site}.

equivalence between topologies?
\end{prb}

\begin{prb}
Let $\cS$ be a site.
A morphism $V\to U$ in $\cS^\wedge$ with $U\in\cS$ is called a \emph{local epimorphism} if $\{V_i\to V\to U\}$ is a covering sieve.

\end{prb}

\chapter{Topoi}


\end{document}