\documentclass{../../large}
\usepackage{../../ikhanchoi}


\begin{document}
\title{Classical Physics}
\author{Ikhan Choi}
\maketitle
\tableofcontents

\part{Classical mechanics}

\chapter{Analytical mechanics}

\section{Lagrangian mechanics}
Newtonian mechanics
\begin{prb}[Laws of motion]
Galilean structure, Galilean group
\end{prb}
\begin{prb}[Conservation laws]
\end{prb}

Calculus of variations
\begin{prb}[Euler-Lagrange equation]
\end{prb}
\begin{prb}[Closed system]
$\pd{\cL}{t}=0$
\end{prb}
\begin{prb}[Definition of generalized momentum]
$\pd{\cL}{q}=0$
\end{prb}
\begin{prb}[Equivalence to Newtonian mechanics]
\end{prb}

Rigid bodies
\begin{prb}[Inertia tensor]
\end{prb}
\begin{prb}[Eulerian angle]
\end{prb}
\begin{prb}[Lagrangian top]
\end{prb}

Oscillation
\begin{prb}[Harmonic oscillator]
\end{prb}
\begin{prb}[Damped oscillation]
\end{prb}
\begin{prb}[Pendulum]
\end{prb}
\begin{prb}[Lissajous curve]
\end{prb}
\begin{prb}[Coupled oscillation]
\end{prb}

Central forces
\begin{prb}[Polar coordinates]
\end{prb}
\begin{prb}[Effective potential]
\end{prb}
\begin{prb}[Kepler's problem]
\end{prb}
\begin{prb}[Rutherford scattering]
\end{prb}

System of particles
\begin{prb}[Closed systems]
\end{prb}
\begin{prb}[Collisions]
\end{prb}
\begin{prb}[Two-body problem]
\end{prb}
\begin{prb}[Three-body problem]
\end{prb}

Euler-Lagrange equations
\begin{prb}[Brachiostochrone]
\end{prb}
\begin{prb}[Geodesic on the sphere]
\end{prb}
\begin{prb}[Dido's isoperimetric problem]
\end{prb}
\begin{prb}[Pendulum with moving support]
A rhenomic system
\end{prb}
\begin{prb}[Sliding beads on a rim]
\end{prb}
\begin{prb}[Double pulley system]
\end{prb}


\section{Hamiltonian mechanics}




\chapter{Continuum mechanics}

\section{Conservation laws}
\section{Fluid mechanics}
\section{Solid mechanics}
plasticity, elasticity?


\chapter{Statistical mechanics}

\section{Thermodynamics}
Laws of thermodynamics
Equation of states
Maxwell's relations

Thermal processes


\section{Kinetic theory}
ergodic hypothesis
Boltzmann statistics
Boltzmann equation, chapman enskog
BBGKY hierarchy
stochastic processes
linear response

\section{Ensembles}
ensembles
microcanonical, canonical, grand canonical
classical gas
Boltzmann distribution

x Two statistics
x Fermi sea
x Bose-Einstein condensation





\part{Classical field theory}

\chapter{Relativity}
\section{Special relativity}
\section{General relativity}
\section{Einstein field equation}
\section{Black holes}



\chapter{Electromagnetism}
\section{Maxwell equations}
We use the mostly minus convention and the Einstien summation convention.
Let $M:=\R^{1,3}$ be the Minkowski space.
Consider a line bundle $L$ over $M$ and take an open subset $U$ on which the bundle is trivialized.

A section of $L$ describes...?
Why is the external current $J$ in $\Omega^3(U,\fg)$?

A connection of $L$ describes a photon field.

The Maxwell equation is the equation of motion of electromagnetic potential $A$ and electromagnetic field $F$, and the inhomogeneous version is written as
\[d*F=\mu_0 J\qquad:\qquad\partial_\nu\,F^{\mu\nu}=\mu_0J^\mu,\]
where $F\in\Omega^2(U,\fg)$ and $J\in\Omega^3(U,\fg)$ such that
\[F:=dA+A\wedge A\qquad:\qquad F_{\mu\nu}=\partial_\mu\,A_\nu-\partial_\nu\,A_\mu,\]
where $A\in\Omega^1(U,\fg)$.
Note that we always have $A\wedge A=0$ because $\fg$ is abelian.

\begin{align*}
F&=dA=d(A_\mu\,dx^\mu)\\
&=dA_\mu\wedge dx^\mu+A_\mu d^2x^\mu\\
&=\partial_\nu\,A_\mu\,dx^\nu\wedge dx^\mu\\
&=\partial_\nu\,A_\mu\,(dx^\nu\otimes dx^\mu-dx^\mu\otimes dx^\nu)\\
&=(\partial_\mu\,A_\nu-\partial_\nu\,A_\mu)\,dx^\mu\otimes dx^\nu.
\end{align*}
or
\begin{align*}
F(X,Y)
&=\partial_\nu\,A_\mu\,(dx^\nu\wedge dx^\mu)(X,Y)\\
&=\partial_\nu\,A_\mu(X^\nu\,Y^\mu-Y^\nu\,X^\mu)\\
&=\partial_\nu\,A_\mu\,X^\nu\,Y^\mu-\partial_\nu\,A_\mu\,Y^\nu\,X^\mu\\
&=\partial_\mu\,A_\nu\,X^\mu\,Y^\nu-\partial_\nu\,A_\mu\,X^\mu\,Y^\nu\\
&=(\partial_\mu\,A_\nu-\partial_\nu\,A_\mu)X^\mu\,Y^\nu.
\end{align*}


The Maxwell equation is given by the Yang-Mills action
\[S[A]=\int_M\tr\left(-\frac1{\mu_0}F\wedge*F-A\wedge J\right)?\qquad:\qquad S=\int d^4x\left[-\frac1{\mu_0}F_{\mu\nu}F^{\mu\nu}-A_\mu J^\mu\right],\]
where $J$ is given as an external current.


The anti-symmetry of $F$ implies that $\partial_\mu\partial_\nu F^{\mu\nu}=0$, so we have the charge conservation $\partial_\mu J^\mu=0$.

quantum fluctuation of continuous field is interpreted as a particle

\begin{itemize}
\item Poincar\'e symmetry
\item Gauge symmetry
\item locality
\end{itemize}


\[A_\mu\mapsto A_\mu'=A_\mu+\partial_\mu\alpha\]

For a local gauge transform $g\in\Omega^0(U,G)$, by taking the logarithm suitably, we can identify $g$ to $\alpha\in\Omega^0(U,\fg)$ which satisfies $\exp\alpha=g$, and we have $d\alpha\in\Omega^1(U,\fg)$.

Since $A\in\Omega^1(U,\fg)$ is in fact a connection form $\omega\in\Omega^1(P|_U,\fg)$ such that ... , the gauge action of $g\in\Omega^0(U,G)$ is given by

\section{Optics}



\chapter{Standard model}


Lagrangian field theory

connection as a section?


\begin{prb}[Maxwell equations by action]
\end{prb}

\[\cL:=-\frac1{4\mu_0}F_{\mu\nu}F^{\mu\nu}-A_\mu J^\mu.\]

Since
\begin{align*}
\frac{\partial\cL}{\partial A_\kappa}
=-\frac{\partial(A_\mu J_\mu)}{\partial A_\kappa}
=-\frac{\partial A_\mu}{\partial A_\kappa}J_\mu
=-\delta_\mu^\kappa J_\mu
=-J^\kappa.
\end{align*}
and
\begin{align*}
\frac{\partial\cL}{\partial(\partial_\kappa A_\lambda)}
=-\frac1{4\mu_0}\frac{\partial(F_{\mu\nu}F^{\mu\nu})}{\partial(\partial_\kappa A_\lambda)}
=-\frac1{2\mu_0}\frac{\partial(\partial_\mu A_\nu\partial^\mu A^\nu-\partial_\mu A_\nu\partial^\nu A^\mu)}{\partial(\partial_\kappa A_\lambda)}
=-\frac1{\mu_0}(\partial^\kappa A^\lambda-\partial^\lambda A^\kappa)
=-\frac1{\mu_0}F^{\kappa\lambda}
\end{align*}
because
\begin{align*}
F_{\mu\nu}F^{\mu\nu}
&=(\partial_\mu A_\nu-\partial_\nu A_\mu)(\partial^\mu A^\nu-\partial^\nu A^\mu)\\
&=\partial_\mu A_\nu\partial^\mu A^\nu-\partial_\mu A_\nu\partial^\nu A^\mu-\partial_\nu A_\mu\partial^\mu A^\nu+\partial_\nu A_\mu\partial^\nu A^\mu\\
&=2(\partial_\mu A_\nu\partial^\mu A^\nu-\partial_\mu A_\nu\partial^\nu A^\mu)
\end{align*}
and
\begin{align*}
\frac{\partial(\partial_\mu A_\nu\partial^\mu A^\nu)}{\partial(\partial_\kappa A_\lambda)}
&=\frac{\partial(\partial_\mu A_\nu)}{\partial(\partial_\kappa A_\lambda)}\partial^\mu A^\nu+\eta_\mu^\rho\eta_\nu^\sigma\partial_\mu A_\nu\frac{\partial(\partial_\rho A_\sigma)}{\partial(\partial_\kappa A_\lambda)}\\
&=\delta_\mu^\kappa\delta_\nu^\lambda\partial^\mu A^\nu+\eta_\mu^\rho\eta_\nu^\sigma\partial_\mu A_\nu\delta_\rho^\kappa\delta^\lambda_\sigma
=2\partial^\kappa A^\lambda
\end{align*}
similarly with
\begin{align*}
\frac{\partial(\partial_\mu A_\nu\partial^\nu A^\mu)}{\partial(\partial_\kappa A_\lambda)}
=2\partial^\lambda A^\kappa,
\end{align*}
the Euler-Lagrange equation is given by
\begin{align*}
0=\frac{\partial\cL}{\partial A_\kappa}-\partial_\kappa\frac{\partial\cL}{\partial(\partial_\kappa A_\lambda)}
=-J^\kappa+\frac1{\mu_0}\partial_\kappa F^{\kappa\lambda}.
\end{align*}



\begin{prb}[Noether theorem for classical fields]
\end{prb}

\[\cL:=-\partial_\mu\phi^*\partial^\mu\phi-m^2\phi^*\phi\]

Under the translation symmetry $\phi(x)\to\phi'(x)=\phi(x-\e)$ with infinitesimal transform parameter vector $\e$, since we have $\delta\phi=-\e^\mu\partial_\mu\phi$ from
\[\phi'=\phi-\e^\mu\partial_\mu\phi+O(\e^2)\]
and $\delta\cL=$ from
\[\cL'=\cL,\]
we can write

\begin{align*}
\delta\cL
&=\frac{\partial\cL}{\partial\phi}\delta\phi+\frac{\partial\cL}{\partial\phi^*}\delta\phi^*+\frac{\partial\cL}{\partial(\partial\phi)}\delta\partial\phi+\frac{\partial\cL}{\partial(\partial\phi^*)}\delta\partial\phi^*\\
&=
\end{align*}



\end{document}