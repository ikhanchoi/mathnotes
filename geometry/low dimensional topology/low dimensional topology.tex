\documentclass{../../large}
\usepackage{../../ikhanchoi}


\begin{document}
\title{Low Dimensional Topology}
\author{Ikhan Choi}
\maketitle
\tableofcontents

\part{Topology of 3-manifolds}
\chapter{}
\chapter{}
\chapter{}


\part{Geometry of 3-manifolds}
\chapter{Hyperbolization}
\section{Geometric structures}


\begin{prb}[Geometric manifolds]
Let $M$ be a smooth manifold, on which we are concerned with metric geometries.
For example, affine, projective, or conformal geometries are not considered to be candidates of our geometries.
Precisely, we suggest to define a \emph{geometric manifold} as a smooth manifold $M$ together with a metric that is
\begin{enumerate}[(i)]
\item geodesically connected,
\item geodesically complete,
\item realized as a Riemannian metric,
\item locally homogeneous: for every $p,q\in M$ an isometry $U_p\to U_q$ between open neighborhoods exists.
\end{enumerate}
Each condition has been obtained by modifying the first four postulates of Euclid's Elements.
\begin{parts}
\item A simply connected manifold $X$ is a geometric manifold if and only if $X$ is a homogeneous Riemannian manifold.
\item A connected manifold $M$ is a geometric manifold if and only if the universal covering $\tilde M$ is a homogeneous Riemannian manifold.
\end{parts}
\end{prb}
\begin{pf}
(a)
($\Rightarrow$)


\end{pf}


\begin{prb}[Space forms]
Let $X$ be a simply connected homogeneous Riemannian manifold.
A \emph{space form} of $X$ is an orbit space $X/\Gamma$ of a subgroup $\Gamma\le\Isom(X)$ such that $X\to X/\Gamma$ is a covering.
If $M$ is a space form of $X$, we say $M$ is \emph{modelled on} $X$.
\begin{parts}
\item $X/\Gamma$ is a space form of $X$ if and only if $\Gamma$ acts properly discontinuously and freely on $X$. (if $X$ is locally compact?)
\item $X/\Gamma$ is a space form of $X$ if and only if $\Gamma$ is a discrete torsion-free subgroup of $\Isom(X)$. (if every periodic isometry of $X$ has a fixed point)
\item There is a following one-to-one correspondence:
\[\begin{array}{ccc}
\dfrac{\{\text{space forms of $X$}\}}{\text{isometry}}
&\xrightarrow{\sim}&
\dfrac{\{\text{discrete torsion-free subgroups of $\Isom(X)$}\}}{\text{conjugacy}},\\
X/\Gamma&\mapsto&\Gamma.
\end{array}\]
\end{parts}
\end{prb}
\begin{pf}

\end{pf}


\begin{prb}[Model geometry]
A \emph{model geometry} is a simply connected homogeneous Riemannian manifold $X$ such that there exists a space form of finite volume.
\begin{parts}
\item There are three model geometries of dimension two, up to isometry.
\item There are eight model geometries of dimension three, up to isometry.
\end{parts}
\end{prb}
\begin{pf}

\end{pf}



\begin{prb}[$(G,X)$-manifolds]
Let $X$ be a topological space.
A \emph{pseudogroup} $\Gamma$ on $X$ is a wide subgroupoid of $\Homeo(X)$ such that $U\mapsto\{g\in\Gamma:\dom g=U\}$ is a sheaf on the topology of $X$.
Let $\Gamma$ be a pseudogroup on $X$, and $M$ be a topological space.
A \emph{$\Gamma$-atlas} on $M$ is an atlas whose charts have $X$ as the codomain and transition maps belong to $\Gamma$.
A \emph{$\Gamma$-structure} on $M$ is defined as an equivalence class of $\Gamma$-atlases on $M$.
\begin{parts}
\item $(G,X):=\{g|_U:g\in G,\ U\in\cT\}$ is a pseudogroup on $X$ for $G\le\Homeo(X)$.
\item...
Note that $G$ does not act on $(G,X)$-manifold $M$.
\end{parts}
\end{prb}

\begin{prb}[Developing and holonomy]
Complete $(G,X)$-manifolds
\end{prb}

\begin{prb}[$(\Isom(X),X)$-manifolds]
Let $X$ be a simply connected homogeneous Riemannian manifold.
We will show that a complete $(G,X)$-structure on a connected manifold corresponds to a geometric structure modelled on $X$.
In this regard, we will call a $(G,X)$-structure that is not complete as \emph{incomplete geometric structure}.
\begin{parts}
\item $M$ is a connected complete $(\Isom(X),X)$-manifold if and only if $M$ is a space form of $X$.
\end{parts}
\end{prb}



\section{Poincar\'e polyhedron theorem}

\begin{prb}[Fundamental polyhedron]
\end{prb}

\begin{prb}[Side pairing]
We want to develop a method for obtaining geometric 3-manifolds by gluing tetrahedra.
Let $X$ be a model geometry and $X/\Gamma$ be a space form of $X$ with finite volume.
\end{prb}

\begin{prb}[Elliptic cycle condition]
Tesselation of $\mathbb{S}^2$ about verteices and edges, but edges are redundant.
\end{prb}

\begin{prb}[Parabolic cycle condition]
Tesselation of $\E^2$ about ideal vertices.

\end{prb}



\section{Hyperbolic Dehn surgery}



\begin{prb}[Cusp and horoball]
\end{prb}

\begin{prb}[Thick-thin decomposition]
Margulis constant.
\end{prb}

\begin{prb}[Thurston's hyperbolic Dehn surgery]
\end{prb}


\section{Mostow rigidity}
Kleinian groups
Several topological invariants: volume, trace fields, etc.


\section{Orbifolds}


\section*{Exercises}

\begin{prb}[Action of compact stabilizer]
Let $M$ be a connected smooth manifold.
Recovering metric from action of compact stabilizers...?
We want to establish the following one-to-one correspondence.
\[\begin{array}{ccc}
\left\{\begin{tabular}{c}Homogeneous\\Riemannian metrics\end{tabular}\right\}
&\xrightarrow{\sim}&
\left\{\begin{tabular}{c}Maximal\\homogeneous smooth group actions\\of compact stabilizers\end{tabular}\right\}.
\end{array}\]
\begin{parts}
\item If $g$ is a homogeneous Riemannian metric on $M$, the group action on $M$ by $\Isom(M,g)$ is maximal among smooth group actions with compact stabilizers.
\item If a smooth group action on $M$ by $G$ is maximal among smooth group actions with compact stabilizers, then there is a homogeneous Riemannian metric on $M$ such that $G\cong\Isom(M,g)$.
\end{parts}
\end{prb}
\begin{pf}
\end{pf}

\begin{prb}[Hyperbolization of punctured surfaces]
Euler characteristic $\chi$.
\end{prb}
\begin{prb}[Figure-eight knot complement]
Find an ideal triangulation.
Find the angle condition for the two ideal tetrahedra.
Find the generators of $\Gamma$.
\end{prb}
\begin{prb}[Geometrically finiteness]
\end{prb}
\begin{prb}[Siegel theorem]
\end{prb}


\chapter{Teichm\"uller theory}
\section{}

\chapter{Geometric group theory}
\section{}


\part{Topology of 4-manifolds}
\chapter{Surgery theory}
Why 4-manifolds are difficult?
\chapter{Intersection forms}
\chapter{Kirby calculus}


\part{Geometry of 4-manifolds}
\chapter{}
\chapter{}
\chapter{}

\end{document}