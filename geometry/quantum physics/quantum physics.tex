\documentclass{../../large}
\usepackage{../../ikhanchoi}


\begin{document}
\title{Quantum Physics}
\author{Ikhan Choi}
\maketitle
\tableofcontents


\part{Quantum fields}

\chapter{Formalism}


\section{Historical backgrounds?}
Wave-particle duality
\begin{prb}[Black body radiation]
(1901)
\end{prb}
\begin{prb}[Photon interaction]
\item Photoelectric effect(1905)
\item Compton scattering(1923)
\end{prb}
\begin{prb}[Atom model]
\begin{parts}
\item Rutherford scattering(1911)
\item Bohr model
\item Franck-Hertz experiment(1914)
\item De Brogile waves(1924)
\end{parts}
\end{prb}
\begin{prb}[Electron diffraction]
\begin{parts}
\item Davisson-Germer(1927)
\item George Pagit Thompson(1928)
\end{parts}
\end{prb}

Nuclear physics
neutrino


\section{Hilbert space formalism}

Interpretations of quantum mechanics
\begin{prb}[Wave function]
Hilbert space, Dirac notation
\end{prb}
\begin{prb}[Pictures]
\end{prb}
\begin{prb}[Copenhagen interpretation]
POVM and measurement
observables and self-adjoint operators
\end{prb}
\begin{prb}[Hidden variable theory]
EPR paradox, Bell's inequality, CHSH inequality
\end{prb}

Canonical quantization
\begin{prb}[Canonical commutation relation]
\end{prb}
\begin{prb}[Weyl quantization]
\end{prb}
\begin{prb}[Stone-von Neumann theorem]
\end{prb}

Spin
\begin{prb}[Projective representations]
\end{prb}

$\Spin(3)\cong\SU(2)$
spin representation
Clebsch-Gordon, singlet and triplet



\section{Schr\"odinger equation}

Time-independent potentials

\begin{prb}[Schr\"odinger operators]
\end{prb}
\begin{prb}[Infinite well]
\end{prb}
\begin{prb}[Harmonic oscillator]
\end{prb}
\begin{prb}[Free particle]
\end{prb}
\begin{prb}[Hydrogen atom]
\end{prb}

Approximation methods
WKB approximation

Relativistic Schr\"odinger equation
fine structure
Klein Gordon equation

Scattering theory



\section{Operator formalism}
\section{Path integral formalism}
functional integral

correlation function
renormalization
Feynman diagram



\chapter{Relativistic fields}
\section{Canonical quantization}
CCR, CAR, Heisenberg group, spin-statistics theorem
Fock representation(universality)
field equations
particles and irreducible representations
wave function
\section{Classical field theory}
\section{Wigner classification}
\section{Dirac equation}
$\Spin(1,3)\cong\SL(2,\C)$
Dirac and Weyl representations
helicity and chirality
positive energy representation







\chapter{Conformal fields}
more general than Lorentz symmetry, locally SO(2,4) for Minkowski

Local operators are not generally fields.
Conformal group is not compact so that the representations may not be given as the direct sum of finite dimensional irreducible representations, which implies that there are no particles in CFT.





\section{}

\begin{prb}
Let $S$ be the Polyakov action
\[S=\frac1{2\pi\alpha'}\int d^2z\,\partial X^\mu\bar\partial X_\mu.\]
The path integral of the total derivate is zero,
\[0=\int[dX]\frac\delta{\delta X_\mu(\sigma)}[e^{-S}\cF[X]].\]
We will use this frequently.
Note that
\[\<\cF[X]\>=\int[dX]e^{-S}\cF[X].\]
\begin{parts}
\item By letting $\cF[X]=1$, we obtain the equation of motion from the Polyakov action is $\partial\bar\partial X^\mu(\sigma)=0$.
So $X^\mu$ span a Riemann surface in the $D$-dimensional space-time.
\item By letting $\cF[X]=X^\nu$, we obtain
\[\frac1{\pi\alpha'}[\partial\bar\partial X^\mu(\sigma)]X^\nu(\sigma')=-\eta^{\mu\nu}\delta^2(\sigma-\sigma').\]
We define the \emph{conformal normal ordering}, which differs from the \emph{creation-annihilation normal ordering}, such that
\[:X^\mu(\sigma):=X^\mu(\sigma),\qquad:X^\mu(\sigma_1)X^\nu(\sigma_2):=X^\mu(\sigma_1)X^\nu(\sigma_2)+\frac{\alpha'}2\eta^{\mu\nu}\ln|z_1-z_2|^2.\]
\end{parts}	
\end{prb}


\begin{prb}[Operator product expansion]
\[A_i(\sigma')A_j(\sigma)=\sum_k c^k_{ij}(\sigma'-\sigma)A_k\cdot\]
\end{prb}

\begin{prb}
We introduce an unknown $j^a$ such that the arbitrary local coordinate transform at $\sigma_0$
\[\phi'_\alpha(\sigma)=\phi_\alpha(\sigma)+\rho(a)\delta\phi_\alpha(\sigma),\qquad \delta\sim\e\]
leads to the first variation term of the functional measure
\begin{align*}
[d\phi']e^{-S[\phi']}
&=[d\phi]e^{-S[\phi]}\left[1+\frac{i\e}{2\pi}\int d^2\sigma\sqrt gj^a(\sigma)\partial\rho(\sigma)+O(\e^2)\right]\\
&=[d\phi]e^{-S[\phi]}\left[1-\frac{i\e}{2\pi}\int d^2\sigma\sqrt g\nabla_aj^a(\sigma)\rho(\sigma)+O(\e^2)\right]
\end{align*}
and
leads to the first variation term of the local operator
\[A'(\sigma_0)=A(\sigma_0)+\rho(a)\delta A(\sigma_0).\]

By the integration by parts, in order for the above transformation for arbitrary $\rho$ to be a symmetry, we have $\nabla_aj^a=0$, where $a$ is the two-dimensional index.
Considering $\rho=1_R$ for some local region $R$, containing $\sigma_0$, and
\[[d\phi']e^{-S[\phi']}A'(\sigma_0)\cdots=[d\phi]e^{-S[\phi]}\left[1-\frac{i\e}{2\pi}\int d^2\sigma\sqrt g\nabla_aj^a(\sigma)\rho(\sigma)+O(\e^2)\right](A(\sigma_0)+\rho(a)\delta A(\sigma_0))\cdots,\]
where the insertion is outside the region $R$ so that the integration by parts is doable, we have the Ward identity by cancelling out $O(\e^2)$ or 
\[\delta A=\frac{i\e}{2\pi}\int d^2\sigma\sqrt g\nabla_aj^a(\sigma)\rho(\sigma)A.\]
Using the divergence and the residue theorem, we have another form of the Ward identity
\[\]

We apply this on space-time translation, the world-sheet translation, and the local conformal transformation(a kind of world-sheet rotation).

\end{prb}

If we consider the world-sheet translation, the Noether current is described by $T_{ab}$.
$T_{ab}$ is trace-less and diagonal in the complex coordinates, we can describe the tensor as a holomorphic function $T(z)$.

Conformal invariance puts strong constraints on the form of OPE involving $T(z)$.


\part{Gauge theory}

\chapter{Yang-Mills theory}


Why spin 1?: vector-like particles can be interpreted as the field in (1/2,1/2) representation of Lorentz group.

\section{Interacting fields}
pair production(1941)

lagrangian of standard model
mass, charge, superselection sectors
mass can be defined as the coefficient of potential term in the Lagrangian.

\section{Higgs mechanism}

\section{Quantum electrodynamics}





\chapter{Supersymmetry}


\chapter{Geometric quantization}
GB, BRST, BV formalisms


\part{}

\chapter{}
Phase transition,
Ginzburg Landau theory,
Thermal states,
Phonon,
Quantum Hall effect,

\chapter{}

\chapter{}




\part{}

\chapter{}



\section*{Open bosonic string}
String field: quantum theory for world-sheet, classical theory for space-time.

Free theory construction

We first consider the boundary CFT, whose Fock space is constructed by bosonic operators $a_n^\mu$ and fermionic ghost operators $b_n,c_n$

Using the BRST operator
BPZ inner product is a symmetric bilinear form

The nilpotency $Q_B^2=0$, which is equivalent to $D=26$, implies the gauge invariance of the action $S:=-\frac12\<\Psi,Q_B\Psi\>_{BPZ}$.



\section*{CFT description of Witten's star product}

Recall that a string field is a state in a boundary CFT.








\end{document}