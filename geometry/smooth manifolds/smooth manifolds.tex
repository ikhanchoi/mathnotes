\documentclass{../../large}
\usepackage{../../ikhanchoi}

\renewcommand{\a}{\alpha}
\newcommand{\II}{\mathrm{II}}
\newcommand{\ssum}{{\textstyle\sum\,}}

\begin{document}
\title{Smooth Manifolds}
\author{Ikhan Choi}
\maketitle
\tableofcontents



% In this note, we only consider smooth maps.
% Is each structure induced on its covering?


\part{Smooth manifolds}


\chapter{Smooth structures}

\section{Local coordinates}


\begin{prb}[Atlases]
Let $M$ be a topological space and fix an integer $m\ge0$.
A \emph{chart} or \emph{local coordinate system} is a topological embedding $\f:U\to\R^m$ from an open subset $U$ of $M$ onto an open subset $\f(U)$ of $\R^m$.
An \emph{atlas} on $M$ is a family $\{\f_\alpha\}$ of charts $\f_\alpha:U_\alpha\to\R^m$ indexed by an open cover $\{U_\alpha\}$ of $M$.
In geography, an atlas means a book of maps of Earth.
Note that the equivalence class of topological atlases is unique if it exists, in the sense that the union of any two atlases is always again an atlas.
A topological space is called \emph{locally Euclidean} if it admits an atlas, which is unique up to equivalence.
A \emph{topological manifold} is defined as a locally Euclidean Hausdorff space.
The paracompactness is frequently assumed in literature, but we will not.

We say an atlas $\{\f_\alpha\}$ is \emph{smooth} if the \emph{transition map}
\[\tau_{\alpha\beta}:=\f_\alpha\circ\f_\beta^{-1}:\f_\beta(U_\alpha\cap U_\beta)\to\f_\alpha(U_\alpha\cap U_\beta)\]
is smooth for each $\alpha$ and $\beta$.
We say two smooth atlases are compatible or equivalent if the union is also a smooth atlas, and a \emph{smooth structure} on $M$ is then defined as an equivalence class of smooth atlases.
A \emph{smooth manifold} is a topological space together with a smooth structure.
The term \emph{manifold} may refer to any of either a topological or smooth manifold, which depends on contexts of references.
The integer $m$ is called the \emph{dimension}.
\begin{parts}
\item Each smooth atlas has a unique smooth structure containing it, and each smooth structure has a unique maximal element.
\end{parts}
\end{prb}

\begin{prb}[Paracompact manifolds]

\end{prb}
\begin{pf}
Let $M$ be a path connected paracompact locally compact Hausdorff topological space.
Since $M$ is locally compact Hausdorff, we have an open cover $\{U_\alpha\}$ of $M$ that consists of relatively compact open sets by taking a refinement.
Since $M$ is paracompact, we may further assume $\{U_\alpha\}$ is locally finite by taking a refinement.
Fix $U_0\in\{U_\alpha\}$, and define $n(U_\alpha)$

\end{pf}

\begin{prb}[Smooth maps and diffeomorphisms]
Let
scalar functions, scalar fields
\begin{parts}
\item The smoothness is independent of the choice of smooth atlases in a smooth structure.
\end{parts}
\end{prb}






\section{Tangent spaces}


\begin{prb}[Tangent spaces of embedded manifolds]
Let $M$ be an $m$-dimensional embedded manifold in $\R^n$.
For a point $p\in M$, take a parameterization $\a$ for $M$ at $p$, and let $x:=\a^{-1}(p)$ be the coordinates of $p$.
The \emph{tangent space} $T_pM$ of $M$ at $p$ is defined as the image of $d\a|_x:\R^m\to\R^n$.
\begin{parts}
\item $T_pM$ is a $m$-dimensional vector subspace of $\R^n$ with a basis $\{\pd_i\alpha(x)\}_{i=1}^m$.
\item If $v\in T_pM$, then we have a smooth curve $\gamma:I\to M$ such that $\gamma(0)=p$ and $\gamma'(0)=v$.
\item If we have a smooth curve $\gamma:I\to M$ such that $\gamma(0)=p$ and $\gamma'(0)=v$, then $v\in T_pM$.
\item The definition of $T_pM$ is independent on the parameterization $\a$.
\end{parts}
\end{prb}


\begin{prb}[Tangent spaces as equivalence classes of curves]
\end{prb}
\begin{prb}[Tangent spaces as derivations]
\end{prb}
the space of derivations on the ring of smooth functions,
the dual space of algebraically defined cotangent spaces.



\begin{prb}

\end{prb}








\section{Differentials}

\begin{prb}
Let $f:M\to N$ be a smooth map.
$df:TM\to TN$ is a bundle map...
\end{prb}






\section{}


\section*{Exercises}
\setcounter{prb}{0}
\begin{prb}[Polar coordinates]
Let $M=\R^2\setminus\{0\}$.
Define a chart $(U,\f)$ by
\[U:=\{\,(x,y)\in M:x\ne0\text{ or }y>0\,\}\]
and $\f=(r,\theta):U\to\R^2$ such that
\[r(x,y):=\sqrt{x^2+y^2},\quad\theta(x,y):=\tan^{-1}\frac yx,\]
where $\tan^{-1}(t):=\int_0^t(1+s^2)^{-1}\,ds$.
\begin{parts}
\item The chart $(U,\f)$ is compatible with the standard smooth structure inherited from $\R^2$.
\item We have
\[r\,\pd{r}=x\,\pd{x}+y\,\pd{y}\text{ and }\pd{\theta}=-y\,\pd{x}+x\,\pd{y}.\]
\end{parts}
\end{prb}

\begin{prb}[Spheres]
Let $\a:\R^2\to\R^3$ be a regular surface given by
\[\a(x,y)=\left(\frac{2x}{1+x^2+y^2},\,\frac{2y}{1+x^2+y^2},\,1-\frac2{1+x^2+y^2}\right).\]
This map gives a parametrization for the sphere $S^2$ without the north pole $(0,0,1)$, and is called the \emph{stereographic projection}.

Spherical coordinates
\begin{parts}
\item All charts above are compatible.
\item There exists at least two charts in an atlas on $S^n$.
\item For the height function $z:S^2\to\R$ given by $z(x,y,z):=z$, we have $\pd_xz(x,y)=4x/(1+x^2+y^2)^2$.
\end{parts}
\end{prb}

\begin{prb}[Projective spaces]
$S^n\to\R P^n$
\end{prb}
\begin{prb}[Stiefel and Grassmann varieties]
$G_1^{n+1}\cong\R P^n$
\end{prb}
\begin{prb}[Parallelizable spheres]
\end{prb}
\begin{prb}[Tagent space of matrix groups]
Jacobi formula
\end{prb}
\begin{prb}[Recovery of compact smooth manifolds]
Let $M$ be a compact smooth manifold.
$C^\infty$ functor is a fully faithful contravariant functor.
\begin{parts}
\item Every unital ring homomorphism $C^\infty(M)\to\R$ is obtained by an evaluation at a point of $M$.
\end{parts}
\end{prb}
\begin{pf}
Suppose $\phi:C^\infty(M)\to\R$ is not an evaluation.
Let $h$ be a positive exhaustion function.
Take a compact set $K:=h^{-1}([0,\phi(h)])$.
For every $p\in K$, we can find $f_p\in C^\infty(M)$ such that $\phi(f_p)\ne f_p(p)$ by the assumption.
Summing $(f_p-\phi(f_p))^2$ finitely on $K$ and applying the extreme value theorem, we obtain a function $f\in C^\infty(M)$ such that $f\ge0$, $f|_K>1$, and $\phi(f)=0$.
Then, the function $h+\phi(h)f-\phi(h)$ is in kernel of $\phi$ although it is strictly positive and thereby a unit.
It is a contradiction.

Alternative proof.
If change the base of $\phi:C^\infty(M)\to\R$ from real to complex, then it becomes a $*$-homomorphism.
Since $C^\infty(M)^+$ is closed under the square root, $\phi$ is positive.
Then, $\phi(f)\le\|f\|$, so we can extend it to a linear map $C(M)\to\R$.
We can check it is a $*$-homomorphism.
\end{pf}









\chapter{Tensor fields}

\section{Vector fields}

\begin{prb}[Vector fields]
Let $\a:U\subset\R^m\to\R^n$ be a parametrization with $M=\im\a$.
A \emph{vector field} is a map $X:M\to\R^n$ such that $X\circ\a:U\to\R^n$ is smooth.
A \emph{tangent vector field} is a vector field $X:M\to\R^n$ such that $X|_p\in T_pM$.
The set of tangent vector fields is often denoted by $\fX(M)$.

The section spaces for special vector bundles which do not use the notation $\Gamma^\infty$: $C^\infty(M)$, $\fX(M)$, $\Omega(M)$.

As a section $X\in\fX(M)$.
As a $C^\infty(M)$-module map $X:\Omega^1(M)\to C^\infty(M)$.
As a differential operator $X:C^\infty(M)\to C^\infty(M)$.
\end{prb}


\begin{prb}
Let $\a:U\subset\R^m\to\R^n$ be a parametrization $M=\im\a$.
\begin{parts}
\item The coordinate representation of a function $f:M\to\R$ is
\[f\circ\a:U\to\R.\]
\item The (external) coordinate representation of a vector field $X:M\to\R^n$ is
\[X\circ\a:U\to\R^n.\]
\item The coordinate representation of a tangent vector field $X:M\to\R^n$ is
\[(X^1\circ\a,\,\cdots,\,X^m\circ\a):U\to\R^m\]
where $X=\sum_iX^i\a_i$.
\end{parts}
\end{prb}

\begin{prb}
Let $\a$ be an $m$-dimensional parametrization with $M=\im\a$.
The value of $\pd_i\a=\a_i:M\to\R^3$ is always a tanget vector at each point $p=\a(x)$, and $\a_i$ becomes a vector field.

Let $s$ be either a smooth function or vector field on $\a$.
Then, we can compute the directional derivative as
\[\pd_is:=\pd_i(s\circ\a)=\pd_t(s\circ\gamma)\]
by taking $\gamma(t)=\a(x+te_i)$, where $e_i$ is the $i$-th standard basis vector for $\R^m$.
\end{prb}

\begin{prb}
Let $M$ be the image of a parametrization $\a:U\subset\R^m\to\R^n$.
Let $v=\sum_iv^i\a_i|_p\in T_pM$ be a tangent vector at $p=\a(x)$.
For a function $f:M\to\R$, its partial derivative is defined by
\[\pd_vf(p):=\sum_{i=1}^mv^i\pd_i(f\circ\a)(x)\in\R.\]
For a vector field $X:M\to\R^n$, its partial derivative is defined by
\[\pd_vX|_p:=\sum_{i=1}^mv^i\pd_i(X\circ\a)(x)\in\R^n.\]
This definition is not dependent on parametrization $\a$.
\end{prb}

\begin{prb}
Let $M$ be the image of a parametrization.
Let $X$ be a tangent vector field on $M$.
\begin{parts}
\item If $f$ is a function, then so is $\pd_Xf$.
\item If $Y$ is a vector field, then so is $\pd_XY$.
\item If $Y$ is a tangent vector field, then so is $\pd_XY-\pd_YX$.
\end{parts}
\end{prb}
\begin{pf}
(a) and (b) are clear.
For (c), if we let $X=\sum_iX^i\a_i$ and $Y=\sum_jY^j\a_j$ for a parametrization $\a:U\subset\R^m\to\R^n$, then
\begin{align*}
\pd_XY-\pd_YX
&=\pd_X(\ssum_jY^j\a_j)-\pd_Y(\ssum_iX^i\a_i)\\
&=\ssum_j[(\pd_XY^j)\a_j+Y^j\pd_X\a_j]-\ssum_i[(\pd_YX^i)\a_i+X^i\pd_Y\a_i]\\
&=\ssum_j[(\pd_XY^j)\a_j+Y^j\ssum_iX^i\pd_i\a_j]-\ssum_i[(\pd_YX^i)\a_i+X^i\ssum_jY^j\pd_i\a_j]\\
&=\ssum_j(\pd_XY^j)\a_j-\ssum_i(\pd_YX^i)\a_i\\
&=\ssum_i(\pd_XY^i-\pd_YX^i)\a_i.\qedhere
\end{align*}
\end{pf}

\begin{prb}
Let $M$ be the image of a parametrization $\a$.
For derivatives of functions on $M$ by tangent vectors, we will use
\[\pd_{\a_i}f=\pd_if,\quad\pd_{\a_t}f=\pd_tf=f',\quad\pd_{\a_x}f=\pd_xf=f_x.\]
For derivatives of vector fields on $M$ by tangent vectors, we will use
\[\pd_{\a_i}X=\pd_iX,\quad\pd_{\a_t}X=\pd_tX=X',\quad\pd_{\a_x}X=\pd_xX=X_x.\]
We will \emph{not} use $f_i$ or $X_i$ for $\pd_if$ and $\pd_iX$ because it is confusig with coordinate representations, and \emph{not} use the nabula symbol $\nabla_v$ in this sense because it will be devoted to another kind of derivatives introduced in Section 4.
\end{prb}

\section{Tensor fields of higher order}
tensor bundle
tensor fields,

\section{Differential forms}
forms, exterior structures, pullback, interior product
\begin{prb}[Exterior derivatives]
Let $M$ be a smooth manifold.
An \emph{exterior derivative} is a super-derivation $d:\Omega^*(M)\to\Omega^*(M)$ of degree one such that $\Omega^0(M)\to\Omega^1(M)$ is the usual differential of functions and $d^2=0$.
\begin{parts}
\item $d$ uniquely exists.
\end{parts}
\end{prb}

\begin{prb}[Interior products]
Let $M$ be a smooth manifold.
For $X\in\fX(M)$, an \emph{interior product} with $X$ is a super-derivation $\iota_X:\Omega^*(M)\to\Omega^*(M)$ of degree minus one such that $\Omega^1(M)\to\Omega^0(M)$ is the usual pairing.
\end{prb}

\section{Lie derivatives}

\begin{prb}[Flow on manifolds]
From $\f:(-\e,\e)\times M\to M$, we can define a vector field by
\[M\to T((-\e,\e)\times M)\to TM:p\mapsto((0,p),(1,0))\mapsto\frac d{dt}\f_t(p)|_{t=0}.\]

\end{prb}



\section*{Exercises}
\begin{prb}[Orientation]
\end{prb}











\chapter{Submanifolds}


\section{Constant rank theorem}

\begin{prb}[Constant rank theorem]
Let $M$ and $N$ be smooth manifolds of dimensions $m$ and $n$.
Let $f:M\to N$ be a smooth map such that $f(p)=q$ for some points $p\in M$ and $q\in N$.
Fix an integer $0\le k\le n$.
For a pair of charts $\f:U\to\R^m$ at $p$ and $\psi:V\to\R^n$ at $q$ such that $f(U)\subset V$, the coordinate representation $\tilde f:=\psi f\f^{-1}:\f(U)\to\psi(V)$ of $f$ is written as
\[\tilde f(x,y)=(a(x,y),b(x,y))\subset\R^k\times\R^{n-k},\qquad(x,y)\in\f(U)\subset\R^k\times\R^{m-k}.\]
Then, the differential $df\in\Gamma^\infty(\Hom(TM,TN))$ on $U$ is represented by a field of the Jacobian matrices


\[\begin{array}{rccc}
D\tilde f:&\f(U)&\to&\Hom_\R(\R^k\times\R^{m-k},\R^k\times\R^{n-k})\\[8pt]
:&(x,y)&\mapsto&\mat{\dfrac{\partial a}{\partial x}(x,y)&\dfrac{\partial a}{\partial y}(x,y)\\[12pt]\dfrac{\partial b}{\partial x}(x,y)&\dfrac{\partial b}{\partial y}(x,y)}
\end{array}\]
Suppose the differential of $f$ has a locally constant rank $k$ at $p$.
\begin{parts}
\item There exist charts $\f:U\to\R^m$ at $p$ and $\psi:V\to\R^n$ at $q$ such that $f(U)\subset V$ and $\partial a/\partial x$ is a $k\times k$ invertible matrix on $\f(U)$.
\item There exist charts $\f:U\to\R^m$ at $p$ and $\psi:V\to\R^n$ at $q$ such that $f(U)\subset V$ and
\[D\tilde f=\mat{\id_k&0\\\ast&0}\qquad\text{ on }\f(U).\]
\item There exist charts $\f:U\to\R^m$ at $p$ and $\psi:V\to\R^n$ at $q$ such that $f(U)\subset V$ and
\[D\tilde f=\mat{\id_k&0\\0&0}\qquad\text{ on }\f(U).\]
\item There exist charts $\f:U\to\R^m$ at $p$ and $\psi:V\to\R^n$ at $q$ such that $f(U)\subset V$ and $\tilde f(x,y)=(x,0)$.
\end{parts}
\end{prb}
\begin{pf}
(a)
Let $(U,\f)$ and $(V,\psi)$ be local charts at $p$ and $q$ such that $f(U)\subset V$.
The Jacobian matrix $D\tilde f|_{(x,y)}$ is of rank $k$ for every $(x,y)\in\f(U)$.
For each $(x,y)\in\f(U)$, the matrix $D\tilde f|_{(x,y)}$ has an invertible $k\times k$ minor submatrix.
Let $A:\R^m\to\R^m$ and $B:\R^n\to\R^n$ be permutation matrices that reorder the coordinates in such a way that the invertible $k\times k$ minor submatrix becomes the leading principal minor submatrix.

Define reparametrizations $\f':=A\circ\f:U\to A(\f(U))$ and $\psi':=B\circ\psi:V\to B(\psi(V))$.
Then, they are clearly local charts and
\[D(\psi'\circ f\circ\f'^{-1})=D(B\circ\psi\circ f\circ\f^{-1}\circ A^{-1})=B\circ D\tilde f\circ A^{-1}\]
has an invertible leading principal minor submatrix of dimension $k\times k$ at every $(x,y)\in\f(U)$.

(b)
Let $(U,\f)$ and $(V,\psi)$ be local charts at $p$ and $q$ satisfying the conditions given in the part (a).
Consider a map $F:\f(U)\to\R^m$ defined by
\[F(x,y):=(a(x,y),y).\]
Then, since
\[DF=\mat{\pd{a}{x}&\pd{a}{y}\\0&\id_{m-k}}\]
is smooth and invertible everywhere on $\f(U)$, there exists an open neighborhood $\f(U')\subset\f(U)$ of $\f(p)$ such that the restriction $F:\f(U')\to F(\f(U'))$ is a diffeomorphism by the inverse function theorem.

Define a reparamterization $\f':=F\circ\f:U'\to F(\f(U'))$.
Then, it is clearly a local chart and
\begin{align*}
D(\psi\circ f\circ\f'^{-1})
&=D(\psi\circ f\circ\f^{-1}\circ F^{-1})
=D\tilde f\circ(DF)^{-1}\\
&=\mat{\pd{a}{x}&\pd{a}{y}\\[4pt]\pd{b}{x}&\pd{b}{y}}\mat{\left(\pd{a}{x}\right)^{-1}&-\left(\pd{a}{x}\right)^{-1}\pd{a}{y}\\[4pt]0&\id_{m-k}}
=\mat{\id_k&0\\\ast&\ast}=\mat{\id_k&0\\\ast&0}.
\end{align*}
The last equality holds because the transpose of this matrix also has rank $k$, and the conditions are satisfied with the local charts $(U',\f')$ and $(V,\psi)$.

(c)
Let $(U,\f)$ and $(V,\psi)$ be local charts at $p$ and $q$ satisfying the conditions given in the part (b).
Consider a map $G:\psi(V)\to\R^n$ defined by
\[G(z,w):=(z,-b(z)+w).\]
Then, since
\[DG=\mat{\id_k&0\\-\pd{b}{x}&\id_{n-k}}\]
is smooth and invertible everywhere on $\psi(V)$, there exists an open neighborhood $\psi(V')\subset\psi(V)$ of $\psi(q)$ such that the restriction $G:\psi(V')\to G(\psi(V'))$ is a diffeomorphism by the inverse function theorem.

Define a reparamterization $\psi':=G\circ\psi:V'\to G(\psi(V'))$.
Then, it is clearly a local chart and
\begin{align*}
D(\psi'\circ f\circ\f^{-1})
&=D(G\circ\psi\circ f\circ\f^{-1})
=DG\circ D\tilde f\\
&=\mat{\id_k&0\\-\pd{b}{x}&\id_{n-k}}\mat{\id_k&0\\\pd{b}{x}&0}
=\mat{\id_k&0\\0&0}.
\end{align*}
Hence, the conditions are satisfied with the local charts $(U,\f)$ and $(V',\psi')$.

(d)
Let $(U,\f)$ and $(V,\psi)$ be local charts at $p$ and $q$ satisfying the conditions given in the part (c).
Then, by translating constants for these local coordinate systems, we obtain $\tilde f(x,y)=(x,0)$.
\end{pf}






\begin{prb}[Preimage theorem]
Let $M$ and $N$ are smooth manifolds of dimensions $m$ and $n$.
Let $f:M\to N$ be a smooth map.
A \emph{critical point} is a point $p\in M$ such that $df|_p$ is not surjective, and a \emph{critical value} is a point $q\in N$ such that $f(p)=q$ for some critical point $p$.
If $q\in N$ is not a critical value, then it is called a \emph{regular value}.

Suppose $q\in N$ is a regular value of $f$, and $p\in M$ be any points satisfying $f(p)=q$.
We will show that $f^{-1}(q)$ is an embedded submanifold of $M$.
Since the set of full rank matrices is open, the rank of $df$ is locally contant at $p$.
By the constant rank theorem, we have local charts $(U,\f)$ and $(V,\psi)$ at $p$ and $q$ such that
\[\f(p)=(0,0)\in\R^n\times\R^{m-n},\quad\psi(q)=0\in\R^n,\quad\text{and}\quad\tilde f(x,y)=x.\]
\begin{parts}
\item $(U\cap f^{-1}(q),\f|_{U\cap f^{-1}(q)})$ is an $(m-n)$-dimensional chart at $p$ on $f^{-1}(q)$.
\item The charts of the form $(U\cap f^{-1}(q),\f|_{U\cap f^{-1}(q)})$ defines a smooth atlas.
\item The inclusion is an embedding.
\end{parts}
\end{prb}
\begin{pf}
(a)
Note that every open subset of $U\subset f^{-1}(q)$ is of the form $W\cap f^{-1}(q)$ for an open set $W\subset U$.
Since $\f(W)$ is open in $\R^m$ for any open $W\subset U$,
\begin{align*}
\f(W\cap f^{-1}(q))
&=\f(W)\cap\f(f^{-1}(q))\\
&=\f(W)\cap\tilde f^{-1}(\psi(q))\\
&=\f(W)\cap\tilde f^{-1}(0)\\
&=\f(W)\cap(\{0\}\times\R^{m-n})
\end{align*}
is open in $\{0\}\times\R^{m-n}$.
It means that the restriction of $\f$ on $U\cap f^{-1}(q)$ is an injective open map, so it is a topological embedding into the Euclidean space $\{0\}\times\R^{m-n}$.

\end{pf}


\section{Embeddings}


\begin{prb}[Immersion is a local embedding]
Let $f:M\to N$ be an immersion at $p\in M$.
Then, there is a local chart $(V,\psi)$ at $f(p)$ such that
\begin{parts}
\item $W=f(M)\cap V$ is an embedded submanifold of $V$,
\item there is a retract $V\to W$.
\end{parts}
\end{prb}
\begin{pf}
Since the set of full rank matrices is open, the rank of $df$ is locally contant at $p$.
By the constant rank theorem, we have
\[\f(p)=0\in\R^m,\quad\psi(f(p))=(0,0)\in\R^m\times\R^{n-m},\quad\text{and}\quad\tilde f(x)=(x,0).\]
Let $W:=f(M)\cap V$.
Then, the injectivity of $\f$ shows that
\[\psi(W)=\psi(f(U))=\psi\circ f\circ\f^{-1}(\f(U))=\{(x,0)\in\R^m\times\R^{n-m}:x\in\f(U)\}\]
is an open subset of $\R^m$, so $(W,\psi|_W)$ is a chart at $f(p)$.

Transition maps are smooth?

The inclusion is a smooth embedding?
\end{pf}

\begin{prb}[Extension of smooth functions]
from an embedded manifold.
\end{prb}


Let $f:M\to N$ be an injective immersion.
There exists unique smooth structure on $f(M)$ such that $f$ and $i$ are smooth.

Let $f:M\to N$ be an embedding.
There exists unique smooth structure on $f(M)$ such that $i$ are smooth.






\section{Foliations}
\begin{prb}[Foliation]
\end{prb}


























\part{Riemannian manifolds}

% metric tensor
% connections
% geodesics, completeness

% parallel transport
% covariant derivative
% curvature
% sectional curvature, Ricci, Riem
% submanifolds, covering
% homogeneous
% Jacobi field
% variational formula
% comparison theory

% bochner technique
% symmetric spaces and holonomy

\chapter{Metrics and connections}
\section{Riemannian metric}
We say a quantity is \emph{intrinsic} in two different contexts: one is the embedding independency, and the other is the coordinates independency.

Riemannian measure

\begin{itemize}
\item Intrinsic: $g_{ij}$, $\Gamma_{ij}^k$, $K$, ${R^l}_{ijk}$;
\item Not intrinsic: $\nu$, $L_{ij}$, $\kappa_i$, $H$.
\end{itemize}

\begin{ex}
Let $\a:(-\log2,\log2)\times(0,2\pi)\to\R^3$ and $\beta:(-\frac34,\frac34)\times(0,2\pi)\to\R^3$ be regular surfaces given by
\[\a(x,\theta)=(\cosh x\cos\theta,\,\cosh x\sin\theta,\,x),\qquad
\beta(r,z)=(r\cos z,\,r\sin z,\,z).\]
Their Riemannian metrics are
\[\mat{\cosh^2x&0\\0&\cosh^2x}_{(\a_x,\a_\theta)},\qquad\mat{1&0\\0&1+r^2}_{(\beta_r,\beta_z)}.\]

Define a map $f:\im\a\to\im\beta$ by
\[f:\a(x,\theta)\mapsto\beta(\sinh x,\theta)=(r(x,\theta),z(x,\theta)).\]
The Jacobi matrix of $f$ is computed
\[df|_{\a(x,\theta)}=\mat{\cosh x&0\\0&1}_{(\a_x,\a_\theta)\to(\beta_r,\beta_z)}.\]
Since $f$ is a diffeomorphism and
\[\mat{\cosh^2x&0\\0&\cosh^2x}=\mat{\cosh x&0\\0&1}^T\mat{1&0\\0&1+r^2}\mat{\cosh x&0\\0&1},\]
the map $f$ is an isometry.
\end{ex}



\section{Connections}

\begin{prb}[Affine connection]
Let $M$ be a smooth manifold.
An \emph{affine connection} on $M$ is a bilinear map
\[\nabla:\fX(M)\times\fX(M)\to\fX(M):(X,Y)\mapsto\nabla_XY\]
such that
\begin{enumerate}[(i)]
\item $C^\infty(M)$-linear in the first argument,
\item the \emph{Leibniz rule} holds:
\[\nabla_X(fY)=(Xf)Y+f\nabla_XY,\qquad f\in C^\infty(M).\]
\end{enumerate}
\end{prb}

\begin{prb}[Levi-Civita connection]
Let $M$ be a Riemannian manifold.
A \emph{metric connection} is an affine connection $\nabla$ such that $\nabla g=0$.
A \emph{Levi-Civita connection} is a metric connection $\nabla$ such that $\nabla T=0$.
\begin{parts}
\item $\nabla$ is a metric connection if and only if $Z\<X,Y\>=\<\nabla_ZX,Y\>+\<X,\nabla_ZY\>$.
\item $\nabla$ is a Levi-Civita connection if and only if $\nabla_XY-\nabla_YX=[X,Y]$.
\item There exists a unique Levi-Civita connection on $M$.
\end{parts}
\end{prb}
\begin{pf}
(Uniqueness)
Suppose $\nabla$ is a Levi-Citiva connection on $M$.
\begin{align*}
2\<\nabla_XY,Z\>&=\pd_X\<Y,Z\>+\pd_Y\<X,Z\>-\pd_Z\<X,Y\>\\
&\qquad-\<[X,Z],Y\>-\<[Y,Z],X\>+\<[X,Y],Z\>.
\end{align*}

(Existence)
\end{pf}

\begin{prb}
Let $S$ be a regular surface embedded in $\R^3$.
If we define Christoffel symbols as the Gauss formula, then
\[\fX(S)\times\fX(S)\to\fX(S):(X^i\a_i,Y^j\a_j)\mapsto\left(X^i\pd_iY^k+X^iY^j\Gamma_{ij}^k\right)\a_k\]
defines a Levi-Civita connection.
\end{prb}


\begin{prb}[Connection form]

\end{prb}


% 좌표변환에 대하여 어떻게 변하는지 - 텐서에 대하여

% 크리스토펠은 좌표변환이 잘 안됨: 텐서가 아니라서
% 공변미분: 근데 걍 3차원공간 편미분에다가 이 좌표변환 추가 텀 붙은 크리스토펠 텀을 추가하면 그 미분 결과가 텐서(접벡터)가 됨


\begin{prb}[Covariant derivative as orthogonal projection]
We are going to think about ``intrinsic'' derivatives for tangent vectors.
For coordinate independence, directional derivatives of a tangent vector field should be at least a tangent vector field, which is false for the obvious partial derivatives in the embedded surface setting; for example, $\rT$ is a tangent vector, but $\rN=\kappa\rT'$ is not tangent.

Recall that the Gauss formula reads
\[\pd_i\a_j=\Gamma_{ij}^k\a_k+L_{ij}\nu\]
so that we have
\begin{align*}
\pd_XY
&=X^i\pd_i(Y^j\a_j)\\
&=X^i(\pd_iY^k)\a_k+X^iY^j\pd_i\a_j\\
&=\left(X^i\pd_iY^k+X^iY^j\Gamma_{ij}^k\right)\a_k+X^iY^jL_{ij}\nu.
\end{align*}
If we write $\nabla_XY=\left(X^i\pd_iY^k+X^iY^j\Gamma_{ij}^k\right)\a_k$, then it embodies the orthogonal projection of $\pd_XY$ onto its tangent space, and we have
\[\pd_XY=\nabla_XY+\II(X,Y)\nu.\]

Let $\a:U\to\R^n$ be an $m$-dimensional parametrization with $\im\a=M$.
Let $X=X^i\a_i$ and $Y=Y^j\a_j$ be tangent vector fields on $M$.
The \emph{covariant derivative} of $Y$ along $X$ is defined as the orthogonal projection of the partial derivative $\pd_XY$ onto the tangent space:
\[\nabla_XY:=\left(X^i\pd_iY^k+X^iY^j\Gamma_{ij}^k\right)\a_k.\]
\begin{parts}
\item
Covariant derivatives are intrinsic.
In other words, the above definition does not depend on the choice of parametrizations.
\end{parts}
\end{prb}
\begin{pf}
Recall that the Christoffel symbols transform as follows:
\[X^iY^j\Gamma_{ij}^k=X^aY^b\left(\Gamma_{ab}^c+\pd{x^i}{x^a}\pd{x^j}{x^b}\pd[2]{x^c}{x^i}{x^j}\right)\pd{x^k}{x^c}.\]
Thus, we have
\begin{align*}
&\left(X^i\pd_iY^k+X^iY^j\Gamma_{ij}^k\right)\a_k\\
&\quad=X^a\pd{x^a}\left(Y^c\pd{x^k}{x^c}\right)\a_k+X^aY^b\left(\pd{x^i}{x^a}\pd{x^j}{x^b}\pd[2]{x^c}{x^i}{x^j}+\Gamma_{ab}^c\right)\pd{x^k}{x^c}\a_k\\
&\quad=X^a\pd{Y^c}{x^a}\a_c+X^aY^b\left(\pd[2]{x^k}{x^a}{x^b}\pd{x^c}{x^k}+\pd{x^i}{x^a}\pd{x^j}{x^b}\pd[2]{x^c}{x^i}{x^j}\right)\a_c+X^aX^b\Gamma_{ab}^c\a_c\\
&\quad=\left(X^a\pd_aY^c+X^aY^b\Gamma_{ab}^c\right)\a_c
\end{align*}
since
\[\pd[2]{x^j}{x^a}{x^b}\pd{x^c}{x^j}+\pd{x^i}{x^a}\pd{x^j}{x^b}\pd[2]{x^c}{x^i}{x^j}=\pd{x^a}\left(\pd{x^j}{x^b}\pd{x^c}{x^j}\right)=\pd_a\delta_b^c=0.\qedhere\]
\end{pf}




\section{Geodesics}
Geodesic equation
Hopf-Rinow theorem
Exponential map, Gauss lemma
Jacobi fields
Cartan-Hadamrd



% 측지선에 대한 기저(n,T,S)
% 측지꼬임/곡률



\chapter{Curvature}




\chapter{}
























\part{Lie groups}
\chapter{Lie correspondence}
\section{Exponential map}
\begin{prb}[Exponential map]
\end{prb}
\begin{prb}[Surjectivity of exponential map]
\end{prb}

\begin{prb}[Lie functor]
\end{prb}

\begin{prb}[Covering spaces of Lie groups]
\end{prb}

\section{Second theorem}
\begin{prb}[Derivative of the exponential map]
Let $G$ be a Lie group.
\begin{parts}
\item
\[\dd{s}\exp(sX)=\exp(sX)X\]
for $s\in\R$ and $X\in\fg$.
\item
\[\pd{s}\]
\end{parts}
\end{prb}

\begin{prb}[Baker-Campbell-Hausdorff formula]
Let $G$ be a Lie group.
Let $X,Y\in\fg$ such that $\exp(X)\exp(Y)$
Define
\[Z(t):=\log(\exp(X)\exp(tY))\]
\end{prb}

\begin{prb}[]
\begin{parts}
\item The Lie functor
\[\mathrm{Lie}:\mathrm{LieGrp}_{simple}\to\mathrm{LieAlg}_\R\]
is fully faithful.
\end{parts}
\end{prb}


\section{Third theorem}
\begin{prb}[Ado's theorem]
\end{prb}

\begin{prb}[Lie's third theorem]
Also called the Cartan-Lie theorem.
\begin{parts}
\item The Lie functor
\[\mathrm{Lie}:\mathrm{LieGrp}_{simple}\to\mathrm{LieAlg}_\R\]
is essentially surjective.
\end{parts}
\end{prb}

\section{Fundamental groups of Lie groups}






\chapter{Compact Lie groups}
\section{Special orthogonal groups}
\section{Special unitary groups}
\section{Symplectic groups}

\section*{Exercises}
\begin{prb}[Lorentz group]
$\SL(2,\C)\to\SO^+(1,3)$
\begin{parts}
\item $O(1,3)$ has four components and $SO^+(1,3)$ is the identity component. Orthochronous $O^+(1,3)$, proper $SO(1,3)$.
\end{parts}
\end{prb}





\chapter{Representations of Lie groups}
\section{Peter-Weyl theorem}
\section{Spin representations}
Clifford algebra




\part{Complex manifolds}

\chapter{Complex structures}

\chapter{K\"ahler manifolds}

\chapter{}


\end{document}