\documentclass{../../large}
\usepackage{../../ikhanchoi}


\begin{document}
\title{Quantum Field Theory}
\author{Ikhan Choi}
\maketitle
\tableofcontents


\part{Quantum fields}

\chapter{Free fields}
\section{Canonical quantization}
CCR, CAR, Heisenberg group, spin-statistics theorem
Fock representation(universality)
field equations
particles and irreducible representations
wave function
\section{Path integral}
functional integral
\section{Correlation function}
renormalization
Feynman diagram



\chapter{Relativistic fields}
\section{Wigner classification}
\section{Dirac equation}
$\Spin(1,3)\cong\SL(2,\C)$
Dirac and Weyl representations
helicity and chirality
positive energy representation







\chapter{Conformal fields}
more general than Lorentz symmetry, locally SO(2,4) for Minkowski

Local operators are not generally fields.
Conformal group is not compact so that the representations may not be given as the direct sum of finite dimensional irreducible representations, which implies that there are no particles in CFT.




\part{Gauge theory}

\chapter{Yang-Mills theory}


Why spin 1?: vector-like particles can be interpreted as the field in (1/2,1/2) representation of Lorentz group.

\section{Interacting fields}
pair production(1941)

lagrangian of standard model
mass, charge, superselection sectors
mass can be defined as the coefficient of potential term in the Lagrangian.

\section{Higgs mechanism}

\section{Quantum electrodynamics}



\chapter{Supersymmetry}


\chapter{Geometric quantization}
GB, BRST, BV formalisms


\part{}

\chapter{}
Phase transition,
Ginzburg Landau theory,
Thermal states,
Phonon,
Quantum Hall effect,

\chapter{}

\chapter{}




\part{}


\end{document}