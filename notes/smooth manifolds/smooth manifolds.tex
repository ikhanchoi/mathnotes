\documentclass{../note}
\usepackage{../../ikany}

\def\a{\alpha}
\def\I{\mathrm{I}}
\def\II{\mathrm{II}}
\def\ssum{{\textstyle\sum\,}}

\begin{document}
\title{Smooth Manifolds}
\author{Ikhan Choi}
\maketitle
\tableofcontents



% In this note, we only consider smooth maps.


\part{Smooth manifolds}


\chapter{Smooth structures}

\section{Local coordinate systems}


\begin{prb}[Local coordinates]
Let $M$ be a topological space and $p\in M$ a point.
Consider a fixed positive integer $m$.
An $m$-dimensional (local) \emph{coordinate system}, or (local) \emph{chart}, at $p$ is a pair $(U,\f)$ consisting of an open neighborhood $U$ of $p$ and a topological embedding $\f:U\to\R^m$.
The embedding $\f$ is called a \emph{coordinate map}, and each component of $\f$ with respect to a basis of $\R^m$ is called a \emph{coordinate function}.

An $m$-dimensional \emph{atlas} on $M$ is an indexed family $\cA=\{(U_\alpha,\f_\alpha)\}_\alpha$ of $m$-dimensional local charts such that every point is contained in some $U_\alpha$, that is, $\{U_\alpha\}_\alpha$ is a cover of $M$.
In geography, an atlas means a book of maps of Earth.
A term \emph{locally Euclidean space} is sometimes used to refer a topological space $M$ together with an $m$-dimensional atlas.
\begin{parts}
\item 
Let $U=\{(x,y)\in\R^2:x\ne0\text{ or }y>0\}$.
For two functions $r,\theta:U\to\R$ defined by
\[r(x,y):=\sqrt{x^2+y^2},\quad\theta(x,y):=2\tan^{-1}\frac y{x+\sqrt{x^2+y^2}},\]
the map
\[U\to\R^2:(x,y)\mapsto(r(x,y),\theta(x,y))\]
is a coordinate map, where $\tan^{-1}(t):=\int_0^t(1+s^2)^{-1}\,ds$.
\end{parts}
\end{prb}

\begin{prb}[Smooth atlases]
Let $M$ be a topological space and $m$ a positive integer.
A \emph{smooth atlas} on $M$ is an atlas $\cA$ on $M$ such that every \emph{transition map}
\[\tau_{\alpha\beta}:=\f_\beta\circ\f_\alpha^{-1}:\f_\alpha(U_\alpha\cap U_\beta)\to\f_\beta(U_\alpha\cap U_\beta)\]
is smooth for all $(U_\alpha,\f_\alpha),(U_\beta,\f_\beta)\in\cA$.
Let $\cA$ be a smooth atlas on $M$.
Then, we can define the \emph{smoothness} of a function $f:M\to\R$ with respect to $\cA$ as follows: we say $f$ is smooth if its \emph{coordinate representation}
\[f\circ\f^{-1}:\f_\alpha(U)\to\R\]
is smooth for all $(U,\f)\in\cA$.

Two smooth atlas $\cA_1$ and $\cA_2$ are called \emph{equivalent} if $\cA_1\cup\cA_2$ is also a smooth atlas.
A \emph{smooth structure} on $M$ is a maximal smooth atlas $\cA$; there is no smooth atlas $\cA'$ that contains $\cA$ properly.
\begin{parts}
\item For a given smooth atlas, every transition map is a diffeomorphism.
\item If two atlases $\cA_1$ and $\cA_2$ are equivalent, then a function $f:M\to\R$ is smooth with respect to $\cA_1$ if and only if it is smooth with respect to $\cA_2$.
\item There is a one-to-one correspondence between smooth structures and equivalence classes of smooth atlases. Therefore, we can describe a smooth structure by giving a particular smooth atlas.
\end{parts}
\end{prb}

\begin{prb}[Manifolds]
A \emph{topological manifold} is defined as a second-countable and Hausdorff space together with a maximal atlas, and a \emph{smooth manifold} is defined as a second-countable and Hausdorff space together with a smooth structure.
The term \emph{manifold} may refer to any of either a topological or a smooth manifold, which depends on contexts of each reference.
\begin{parts}
\item The long line admits a smooth structure, and it is Hausdorff but not second countable.
\item The line with two origins admits a smooth structure, and it is second countable but not Hausdorff.
\end{parts}
\end{prb}

\begin{prb}[Partition of unity]
\end{prb}

\begin{prb}[Smooth maps and diffeomorphisms]
scalar functions, scalar fields
\end{prb}

\begin{prb}[Embedded manifolds]
a \emph{embedded manifold} or a \emph{regular manifold}.
\emph{parametrization}

If $\a:U\to\R^n$ is a topological embedding, then we can endow with a unique smooth structure on $\im\a$ such that $\a$ is smooth.(?)
\begin{parts}
\item The image of a regular parameterization is an embedded manifold.
\item Every open subset of a embedded manifold is a embedded manifold.
\item Monge patch.
\item The sphere $S^2=\{\,(x,y,z)\in\R^3:x^2+y^2+z^2=1\,\}$ is a regular surface.
\item The set $\{\,(x,y)\in\R^2:y^2=x^3+x^2\,\}$ is not a regular curve.
\item The set $\{\,(x,y)\in\R^2:y=|x|\,\}$ is not a regular curve.
\end{parts}
\end{prb}






\section{Tangent spaces}


\begin{prb}[Tangent spaces of embedded manifolds]
Let $M$ be an $m$-dimensional embedded manifold in $\R^n$.
For a point $p\in M$, take a parameterization $\a$ for $M$ at $p$, and let $x:=\a^{-1}(p)$ be the coordinates of $p$.
The \emph{tangent space} $T_pM$ of $M$ at $p$ is defined as the image of $d\a|_x:\R^m\to\R^n$.
\begin{parts}
\item $T_pM$ is a $m$-dimensional vector subspace of $\R^n$ with a basis $\{\pd_i\alpha(x)\}_{i=1}^m$.
\item If $v\in T_pM$, then we have a smooth curve $\gamma:I\to M$ such that $\gamma(0)=p$ and $\gamma'(0)=v$.
\item If we have a smooth curve $\gamma:I\to M$ such that $\gamma(0)=p$ and $\gamma'(0)=v$, then $v\in T_pM$.
\item The definition of $T_pM$ is independent on the parameterization $\a$.
\end{parts}
\end{prb}


\begin{prb}[Tangent spaces as equivalence classes of curves]
\end{prb}
\begin{prb}[Tangent spaces as derivations]
\end{prb}
the space of derivations on the ring of smooth functions,
the dual space of algebraically defined cotangent spaces.












\section{Differentials}




\section*{Exercises}


\begin{prb}[Smooth structure on spheres]
Let $\a:\R^2\to\R^3$ be a regular surface given by
\[\a(x,y)=\left(\frac{2x}{1+x^2+y^2},\,\frac{2y}{1+x^2+y^2},\,1-\frac2{1+x^2+y^2}\right).\]
This map gives a parametrization for the sphere $S^2$ without the north pole $(0,0,1)$, and is called the \emph{stereographic projection}.
Let $f:S^2\setminus\{(0,0,1)\}\to\R$ be the height function of $\a$ defined by
\[f(p):=z\]
for $p=(x,y,z)\in S^2\setminus\{(0,0,1)\}$.
Its coordinate representation is
\[f\circ\a(x,y)=1-\frac2{1+x^2+y^2}.\]
Then, the directional derivative is
\[\pd_xf=\pd{(f\circ\a)}{x}=\pd{x}\left(1-\frac2{1+x^2+y^2}\right)=\frac{4x}{(1+x^2+y^2)^2}.\]
Note that $\pd_xf\ne\pd_{(1,0,0)}z=0$.
\begin{parts}
\item The minimal cardinality of a smooth atlas on $S^n$ is two.
\end{parts}
\end{prb}

\begin{prb}[Smooth structure on projective spaces]
\end{prb}
\begin{prb}[Stiefel and Grassmann varieties]
\end{prb}
\begin{prb}[Parallelization of spheres]
\end{prb}
\begin{prb}[Tagent space of matrix groups]
Jacobi formula
\end{prb}
\begin{prb}[Recovery of compact smooth manifolds]
Let $M$ be a compact smooth manifold.
$C^\infty$ functor is a fully faithful contravariant functor.
\begin{parts}
\item Every ring homomorphism $C^\infty(M)\to\R$ is obtained by an evaluation at a point of $M$.
\end{parts}
\end{prb}
\begin{pf}
Suppose $\phi:C^\infty(M)\to\R$ is not an evaluation.
Let $h$ be a positive exhaustion function.
Take a compact set $K:=h^{-1}([0,\phi(h)])$.
For every $p\in K$, we can find $f_p\in C^\infty(M)$ such that $\phi(f_p)\ne f_p(p)$ by the assumption.
Summing $(f_p-\phi(f_p))^2$ finitely on $K$ and applying the extreme value theorem, we obtain a function $f\in C^\infty(M)$ such that $f\ge0$, $f|_K>1$, and $\phi(f)=0$.
Then, the function $h+\phi(h)f-\phi(h)$ is in kernel of $\phi$ although it is strictly positive and thereby a unit.
It is a contradiction.
\end{pf}









\chapter{Tensor fields}

\section{Vector fields}

\begin{prb}[Vector fields]
Let $\a:U\subset\R^m\to\R^n$ be a parametrization with $M=\im\a$.
A \emph{vector field} is a map $X:M\to\R^n$ such that $X\circ\a:U\to\R^n$ is smooth.
A \emph{tangent vector field} is a vector field $X:M\to\R^n$ such that $X|_p\in T_pM$.
The set of tangent vector fields is often denoted by $\fX(M)$.
\end{prb}


\begin{prb}
Let $\a:U\subset\R^m\to\R^n$ be a parametrization $M=\im\a$.
\begin{parts}
\item The coordinate representation of a function $f:M\to\R$ is
\[f\circ\a:U\to\R.\]
\item The (external) coordinate representation of a vector field $X:M\to\R^n$ is
\[X\circ\a:U\to\R^n.\]
\item The coordinate representation of a tangent vector field $X:M\to\R^n$ is
\[(X^1\circ\a,\,\cdots,\,X^m\circ\a):U\to\R^m\]
where $X=\sum_iX^i\a_i$.
\end{parts}
\end{prb}

\begin{prb}
Let $\a$ be an $m$-dimensional parametrization with $M=\im\a$.
The value of $\pd_i\a=\a_i:M\to\R^3$ is always a tanget vector at each point $p=\a(x)$, and $\a_i$ becomes a vector field.

Let $s$ be either a smooth function or vector field on $\a$.
Then, we can compute the directional derivative as
\[\pd_is:=\pd_i(s\circ\a)=\pd_t(s\circ\gamma)\]
by taking $\gamma(t)=\a(x+te_i)$, where $e_i$ is the $i$-th standard basis vector for $\R^m$.
\end{prb}

\begin{prb}
Let $M$ be the image of a parametrization $\a:U\subset\R^m\to\R^n$.
Let $v=\sum_iv^i\a_i|_p\in T_pM$ be a tangent vector at $p=\a(x)$.
For a function $f:M\to\R$, its partial derivative is defined by
\[\pd_vf(p):=\sum_{i=1}^mv^i\pd_i(f\circ\a)(x)\in\R.\]
For a vector field $X:M\to\R^n$, its partial derivative is defined by
\[\pd_vX|_p:=\sum_{i=1}^mv^i\pd_i(X\circ\a)(x)\in\R^n.\]
This definition is not dependent on parametrization $\a$.
\end{prb}

\begin{prb}
Let $M$ be the image of a parametrization.
Let $X$ be a tangent vector field on $M$.
\begin{parts}
\item If $f$ is a function, then so is $\pd_Xf$.
\item If $Y$ is a vector field, then so is $\pd_XY$.
\item If $Y$ is a tangent vector field, then so is $\pd_XY-\pd_YX$.
\end{parts}
\end{prb}
\begin{pf}
(a) and (b) are clear.
For (c), if we let $X=\sum_iX^i\a_i$ and $Y=\sum_jY^j\a_j$ for a parametrization $\a:U\subset\R^m\to\R^n$, then
\begin{align*}
\pd_XY-\pd_YX
&=\pd_X(\ssum_jY^j\a_j)-\pd_Y(\ssum_iX^i\a_i)\\
&=\ssum_j[(\pd_XY^j)\a_j+Y^j\pd_X\a_j]-\ssum_i[(\pd_YX^i)\a_i+X^i\pd_Y\a_i]\\
&=\ssum_j[(\pd_XY^j)\a_j+Y^j\ssum_iX^i\pd_i\a_j]-\ssum_i[(\pd_YX^i)\a_i+X^i\ssum_jY^j\pd_i\a_j]\\
&=\ssum_j(\pd_XY^j)\a_j-\ssum_i(\pd_YX^i)\a_i\\
&=\ssum_i(\pd_XY^i-\pd_YX^i)\a_i.\qedhere
\end{align*}
\end{pf}

\begin{prb}
Let $M$ be the image of a parametrization $\a$.
For derivatives of functions on $M$ by tangent vectors, we will use
\[\pd_{\a_i}f=\pd_if,\quad\pd_{\a_t}f=\pd_tf=f',\quad\pd_{\a_x}f=\pd_xf=f_x.\]
For derivatives of vector fields on $M$ by tangent vectors, we will use
\[\pd_{\a_i}X=\pd_iX,\quad\pd_{\a_t}X=\pd_tX=X',\quad\pd_{\a_x}X=\pd_xX=X_x.\]
We will \emph{not} use $f_i$ or $X_i$ for $\pd_if$ and $\pd_iX$ because it is confusig with coordinate representations, and \emph{not} use the nabula symbol $\nabla_v$ in this sense because it will be devoted to another kind of derivatives introduced in Section 4.
\end{prb}

\section{Tensor fields of higher order}
tensor bundle
tensor fields,

\section{Differential forms}
forms, exterior structures, pullback, interior product


\section{Lie derivatives}
\begin{prb}[Integral curves]
\end{prb}

\section*{Exercises}
\begin{prb}[Orientation]
\end{prb}











\chapter{Submanifolds}


\section{Constant rank theorem}

\begin{prb}[Constant rank theorem]
Let $M$ and $N$ be smooth manifolds of dimensions $m$ and $n$, and $f:M\to N$ a smooth map.
Let $p\in M$ and $q\in N$ such that $f(p)=q$.
For each pair of local charts $(U,\f)$ at $p$ and $(V,\psi)$ at $q$ such that $f(U)\subset V$, we can introduce functions $a:\f(U)\to\R^k$ and $b:\f(U)\to\R^{n-k}$ such that the coordinate representation $\tilde f:\f(U)\to\psi(V)$ of $f$ is written as
\[\tilde f(x,y):=\psi\circ f\circ\f^{-1}(x,y)=(a(x,y),b(x,y))\]
for $x\in\R^k$ and $y\in\R^{m-k}$ with $(x,y)\in\f(U)$.
Then, the differential $df$ on $U$ is represented by its Jacobian matrix
\[
D\tilde f|_{(x,y)}=\mat{\pd{a}{x}&\pd{a}{y}\\[4pt]\pd{b}{x}&\pd{b}{y}}.\]
Suppose the differential of $f$ has a locally constant rank $k$ at $p$.
\begin{parts}
\item There exists local charts $(U,\f)$ at $p$ and $(V,\psi)$ at $q$ such that $f(U)\subset V$ and $\partial a/\partial x$ is a $k\times k$ invertible matrix everywhere.
\item There exists local charts $(U,\f)$ at $p$ and $(V,\psi)$ at $q$ such that $f(U)\subset V$ and
\[D\tilde f|_{(x,y)}=\mat{\id_k&0\\\ast&0}.\]
\item There exists local charts $(U,\f)$ at $p$ and $(V,\psi)$ at $q$ such that $f(U)\subset V$ and
\[D\tilde f|_{(x,y)}=\mat{\id_k&0\\0&0}.\]
\item There exists local charts $(U,\f)$ at $p$ and $(V,\psi)$ at $q$ such that $f(U)\subset V$ and $\tilde f(x,y)=(x,0)$.
\end{parts}
\end{prb}
\begin{pf}
(a)
Let $(U,\f)$ and $(V,\psi)$ be local charts at $p$ and $q$ such that $f(U)\subset V$ and the Jacobian matrix $D\tilde f|_{(x,y)}$ is of rank $k$ for every $(x,y)\in\f(U)$.
For each $(x,y)\in\f(U)$, the matrix $D\tilde f|_{(x,y)}$ has an invertible $k\times k$ minor submatrix.
Let $A:\R^m\to\R^m$ and $B:\R^n\to\R^n$ be permutation matrices that reorder the coordinates in such a way that the invertible $k\times k$ minor submatrix becomes the leading principal minor submatrix.

Define reparametrizations $\f':=A\circ\f:U\to A(\f(U))$ and $\psi':=B\circ\psi:V\to B(\psi(V))$.
Then, they are clearly local charts and
\[D(\psi'\circ f\circ\f'^{-1})=D(B\circ\psi\circ f\circ\f^{-1}\circ A^{-1})=B\circ D\tilde f\circ A^{-1}\]
has an invertible leading principal minor submatrix of dimension $k\times k$ at every $(x,y)\in\f(U)$.

(b)
Let $(U,\f)$ and $(V,\psi)$ be local charts at $p$ and $q$ satisfying the conditions given in the part (a).
Consider a map $F:\f(U)\to\R^m$ defined by
\[F(x,y):=(a(x,y),y).\]
Then, since
\[DF|_{(x,y)}=\mat{\pd{a}{x}&\pd{a}{y}\\0&\id_{m-k}}\]
is smooth and invertible everywhere on $\f(U)$, there exists an open neighborhood $\f(U')\subset\f(U)$ of $\f(p)$ such that the restriction $F:\f(U')\to F(\f(U'))$ is a diffeomorphism by the inverse function theorem.

Define a reparamterization $\f':=F\circ\f:U'\to F(\f(U'))$.
Then, it is clearly a local chart and
\begin{align*}
D(\psi\circ f\circ\f'^{-1})
&=D(\psi\circ f\circ\f^{-1}\circ F^{-1})
=D\tilde f\circ(DF)^{-1}\\
&=\mat{\pd{a}{x}&\pd{a}{y}\\[4pt]\pd{b}{x}&\pd{b}{y}}\mat{\left(\pd{a}{x}\right)^{-1}&-\left(\pd{a}{x}\right)^{-1}\pd{a}{y}\\[4pt]0&\id_{m-k}}
=\mat{\id_k&0\\\ast&\ast}=\mat{\id_k&0\\\ast&0}.
\end{align*}
The last equality holds because the transpose of this matrix has rank $k$, and the conditions are satisfied with the local charts $(U',\f')$ and $(V,\psi)$.

(c)
Let $(U,\f)$ and $(V,\psi)$ be local charts at $p$ and $q$ satisfying the conditions given in the part (b).
Then, we have $\tilde f(x,y)=(x,b(x))$ for all $(x,y)\in\f(U)$.
Consider a map $G:\psi(V)\to\R^n$ defined by
\[G(x,z):=(x,z-b(x)).\]
Then, since
\[DG|_{(x,z)}=\mat{\id_k&0\\-\pd{b}{x}&\id_{n-k}}\]
is smooth and invertible everywhere on $\psi(V)$, there exists an open neighborhood $\psi(V')\subset\psi(V)$ of $\psi(q)$ such that the restriction $G:\psi(V')\to G(\psi(V'))$ is a diffeomorphism by the inverse function theorem.

Define a reparamterization $\psi':=G\circ\psi:V'\to G(\psi(V'))$.
Then, it is clearly a local chart and
\begin{align*}
D(\psi'\circ f\circ\f^{-1})
&=D(G\circ\psi\circ f\circ\f^{-1})
=DG\circ D\tilde f\\
&=\mat{\id_k&0\\-\pd{b}{x}&\id_{n-k}}\mat{\id_k&0\\\pd{b}{x}&0}
=\mat{\id_k&0\\0&0}.
\end{align*}
Hence, the conditions are satisfied with the local charts $(U,\f)$ and $(V',\psi')$.

(d)
Let $(U,\f)$ and $(V,\psi)$ be local charts at $p$ and $q$ satisfying the conditions given in the part (c).
Then, by translating constants for these local coordinate systems, we obtain $\tilde f(x,y)=(x,0)$.
\end{pf}






\begin{prb}[Preimage theorem]
Let $M$ and $N$ are smooth manifolds of dimensions $m$ and $n$.
Let $f:M\to N$ be a smooth map.
A \emph{critical point} is a point $p\in M$ such that $df|_p$ is not surjective, and a \emph{critical value} is a point $q\in N$ such that $f(p)=q$ for some critical point $p$.
If $q\in N$ is not a critical value, then it is called a \emph{regular value}.

Suppose $q\in N$ is a regular value of $f$, and $p\in M$ be any points satisfying $f(p)=q$.
We will show that $f^{-1}(q)$ is an embedded submanifold of $M$.
Since the set of full rank matrices is open, the rank of $df$ is locally contant at $p$.
By the constant rank theorem, we have local charts $(U,\f)$ and $(V,\psi)$ at $p$ and $q$ such that
\[\f(p)=(0,0)\in\R^n\times\R^{m-n},\quad\psi(q)=0\in\R^n,\quad\text{and}\quad\tilde f(x,y)=x.\]
\begin{parts}
\item $(U\cap f^{-1}(q),\f|_{U\cap f^{-1}(q)})$ is an $(m-n)$-dimensional chart at $p$ on $f^{-1}(q)$.
\item The charts of the form $(U\cap f^{-1}(q),\f|_{U\cap f^{-1}(q)})$ defines a smooth atlas.
\item The inclusion is an embedding.
\end{parts}
\end{prb}
\begin{pf}
(a)
Note that every open subset of $U\subset f^{-1}(q)$ is of the form $W\cap f^{-1}(q)$ for an open set $W\subset U$.
Since $\f(W)$ is open in $\R^m$ for any open $W\subset U$,
\begin{align*}
\f(W\cap f^{-1}(q))
&=\f(W)\cap\f(f^{-1}(q))\\
&=\f(W)\cap\tilde f^{-1}(\psi(q))\\
&=\f(W)\cap\tilde f^{-1}(0)\\
&=\f(W)\cap(\{0\}\times\R^{m-n})
\end{align*}
is open in $\{0\}\times\R^{m-n}$.
It means that the restriction of $\f$ on $U\cap f^{-1}(q)$ is an injective open map, so it is a topological embedding into the Euclidean space $\{0\}\times\R^{m-n}$.

\end{pf}


\section{Embeddings}

\begin{prb}[Immersion is a local embedding]
Let $f:M\to N$ be an immersion at $p\in M$.
Then, there is a local chart $(V,\psi)$ at $f(p)$ such that
\begin{parts}
\item $W=f(M)\cap V$ is an embedded submanifold of $V$,
\item there is a retract $V\to W$.
\end{parts}
\end{prb}
\begin{pf}
Since the set of full rank matrices is open, the rank of $df$ is locally contant at $p$.
By the constant rank theorem, we have
\[\f(p)=0\in\R^m,\quad\psi(f(p))=(0,0)\in\R^m\times\R^{n-m},\quad\text{and}\quad\tilde f(x)=(x,0).\]
Let $W:=f(M)\cap V$.
Then, the injectivity of $\f$ shows that
\[\psi(W)=\psi(f(U))=\psi\circ f\circ\f^{-1}(\f(U))=\{(x,0)\in\R^m\times\R^{n-m}:x\in\f(U)\}\]
is an open subset of $\R^m$, so $(W,\psi|_W)$ is a chart at $f(p)$.

Transition maps are smooth?

The inclusion is a smooth embedding?
\end{pf}

\begin{prb}[Extension of smooth functions]
from an embedded manifold.
\end{prb}


Let $f:M\to N$ be an injective immersion.
There exists unique smooth structure on $f(M)$ such that $f$ and $i$ are smooth.

Let $f:M\to N$ be an embedding.
There exists unique smooth structure on $f(M)$ such that $i$ are smooth.



\section{Distributions}
\begin{prb}[Foliation]
\end{prb}


























\part{Riemannian manifolds}

% metric tensor
% connections
% geodesics, completeness
% parallel transport
% covariant derivative
% curvature
% sectional curvature, Ricci, Riem
% submanifolds, covering
% homogeneous
% Jacobi field
% variational formula
% Comparison theory

\chapter{Intrinsic geometry}

We say a quantity on a surface is \emph{intrinsic} if it is independent of how the surface is embedded in space.

Notations: Einstein summation convention, set of vector fields.

To $n$-dimensional.

\section{Covariance and contravariance}
% 좌표변환에 대하여 어떻게 변하는지 - 텐서에 대하여

% 크리스토펠은 좌표변환이 잘 안됨: 텐서가 아니라서
% 공변미분: 근데 걍 3차원공간 편미분에다가 이 좌표변환 추가 텀 붙은 크리스토펠 텀을 추가하면 그 미분 결과가 텐서(접벡터)가 됨

\section{Theorema Egregium}
% metric 정의
% isometry
% 가우스의 놀라운 정리: 크리스토펠 -> 곡률텐서 -> 가우스곡률?

\begin{itemize}
\item Intrinsic: $g_{ij}$, $\Gamma_{ij}^k$, $K$, ${R^l}_{ijk}$;
\item Not intrinsic: $\nu$, $L_{ij}$, $\kappa_i$, $H$.
\end{itemize}

Isometry
\begin{ex}
Let $\a:(-\log2,\log2)\times(0,2\pi)\to\R^3$ and $\beta:(-\frac34,\frac34)\times(0,2\pi)\to\R^3$ be regular surfaces given by
\[\a(x,\theta)=(\cosh x\cos\theta,\,\cosh x\sin\theta,\,x),\qquad
\beta(r,z)=(r\cos z,\,r\sin z,\,z).\]
Their Riemannian metrics are
\[\mat{\cosh^2x&0\\0&\cosh^2x}_{(\a_x,\a_\theta)},\qquad\mat{1&0\\0&1+r^2}_{(\beta_r,\beta_z)}.\]

Define a map $f:\im\a\to\im\beta$ by
\[f:\a(x,\theta)\mapsto\beta(\sinh x,\theta)=(r(x,\theta),z(x,\theta)).\]
The Jacobi matrix of $f$ is computed
\[df|_{\a(x,\theta)}=\mat{\cosh x&0\\0&1}_{(\a_x,\a_\theta)\to(\beta_r,\beta_z)}.\]
Since $f$ is a diffeomorphism and
\[\mat{\cosh^2x&0\\0&\cosh^2x}=\mat{\cosh x&0\\0&1}^T\mat{1&0\\0&1+r^2}\mat{\cosh x&0\\0&1},\]
the map $f$ is an isometry.
\end{ex}




\chapter{Covariant derivatives}

\section{Orthogonal projection}
We are going to think about ``intrinsic'' derivatives for tangent vectors.
For coordinate independence, directional derivatives of a tangent vector field should be at least a tangent vector field, which is false for the obvious partial derivatives in the embedded surface setting; for example, $\rT$ is a tangent vector, but $\rN=\kappa\rT'$ is not tangent.

Recall that the Gauss formula reads
\[\pd_i\a_j=\Gamma_{ij}^k\a_k+L_{ij}\nu\]
so that we have
\begin{align*}
\pd_XY
&=X^i\pd_i(Y^j\a_j)\\
&=X^i(\pd_iY^k)\a_k+X^iY^j\pd_i\a_j\\
&=\left(X^i\pd_iY^k+X^iY^j\Gamma_{ij}^k\right)\a_k+X^iY^jL_{ij}\nu.
\end{align*}
If we write $\nabla_XY=\left(X^i\pd_iY^k+X^iY^j\Gamma_{ij}^k\right)\a_k$, then it embodies the orthogonal projection of $\pd_XY$ onto its tangent space, and we have
\[\pd_XY=\nabla_XY+\II(X,Y)\nu.\]

\begin{defn}
Let $\a:U\to\R^n$ be an $m$-dimensional parametrization with $\im\a=M$.
Let $X=X^i\a_i$ and $Y=Y^j\a_j$ be tangent vector fields on $M$.
The \emph{covariant derivative} of $Y$ along $X$ is defined as the orthogonal projection of the partial derivative $\pd_XY$ onto the tangent space:
\[\nabla_XY:=\left(X^i\pd_iY^k+X^iY^j\Gamma_{ij}^k\right)\a_k.\]
\end{defn}

\begin{prop}
Covariant derivatives are intrinsic.
In other words, the above definition does not depend on the choice of parametrizations.
\end{prop}
\begin{pf}
Recall that the Christoffel symbols transform as follows:
\[X^iY^j\Gamma_{ij}^k=X^aY^b\left(\Gamma_{ab}^c+\pd{x^i}{x^a}\pd{x^j}{x^b}\pd[2]{x^c}{x^i}{x^j}\right)\pd{x^k}{x^c}.\]
Thus, we have
\begin{align*}
&\left(X^i\pd_iY^k+X^iY^j\Gamma_{ij}^k\right)\a_k\\
&\quad=X^a\pd{x^a}\left(Y^c\pd{x^k}{x^c}\right)\a_k+X^aY^b\left(\pd{x^i}{x^a}\pd{x^j}{x^b}\pd[2]{x^c}{x^i}{x^j}+\Gamma_{ab}^c\right)\pd{x^k}{x^c}\a_k\\
&\quad=X^a\pd{Y^c}{x^a}\a_c+X^aY^b\left(\pd[2]{x^k}{x^a}{x^b}\pd{x^c}{x^k}+\pd{x^i}{x^a}\pd{x^j}{x^b}\pd[2]{x^c}{x^i}{x^j}\right)\a_c+X^aX^b\Gamma_{ab}^c\a_c\\
&\quad=\left(X^a\pd_aY^c+X^aY^b\Gamma_{ab}^c\right)\a_c
\end{align*}
since
\[\pd[2]{x^j}{x^a}{x^b}\pd{x^c}{x^j}+\pd{x^i}{x^a}\pd{x^j}{x^b}\pd[2]{x^c}{x^i}{x^j}=\pd{x^a}\left(\pd{x^j}{x^b}\pd{x^c}{x^j}\right)=\pd_a\delta_b^c=0.\qedhere\]
\end{pf}



\section{Connection}

\begin{prb}[Affine connection]
Let $M$ be a smooth manifold
An \emph{affine connection} on $M$ is a map
\[\nabla:\fX(M)\times\fX(M)\to\fX(M):(X,Y)\mapsto\nabla_XY\]
such that
\begin{enumerate}[(i)]
\item $C^\infty(M)$-linear in the first argument $X$,
\item the \emph{Leibniz rule}
\[\nabla_X(fY)=XfY+f\nabla_XY\]
for $f\in C^\infty(M)$ in the second argument $Y$ is satisfied.
\end{enumerate}
\end{prb}

\begin{prb}[Levi-Civita connection]
Let $M$ be a Riemannian manifold.
A \emph{metric connection} is an affine connection $\nabla$ such that $\nabla g=0$.
A \emph{Levi-Civita connection} is a metric connection $\nabla$ such that $\nabla T=0$.
\begin{parts}
\item $\nabla$ is a metric connection if and only if $Z\<X,Y\>=\<\nabla_ZX,Y\>+\<X,\nabla_ZY\>$.
\item $\nabla$ is a Levi-Civita connection if and only if $\nabla_XY-\nabla_YX=[X,Y]$.
\item There exists a unique Levi-Civita connection on $M$.
\end{parts}
\end{prb}
\begin{pf}
(Uniqueness)
Suppose $\nabla$ is a Levi-Citiva connection on $M$.
\begin{align*}
2\<\nabla_XY,Z\>&=\pd_X\<Y,Z\>+\pd_Y\<X,Z\>-\pd_Z\<X,Y\>\\
&\qquad-\<[X,Z],Y\>-\<[Y,Z],X\>+\<[X,Y],Z\>.
\end{align*}

(Existence)
\end{pf}

\begin{prb}
Let $S$ be a regular surface embedded in $\R^3$.
If we define Christoffel symbols as the Gauss formula, then
\[\fX(S)\times\fX(S)\to\fX(S):(X^i\a_i,Y^j\a_j)\mapsto\left(X^i\pd_iY^k+X^iY^j\Gamma_{ij}^k\right)\a_k\]
defines a Levi-Civita connection.
\end{prb}


\begin{prb}[Connection form]

\end{prb}


\section{Curvature tensor}



\chapter{Parallel transport}
% linearity, holonomy




















\part{Local theory of curves and surfaces}

\chapter{Local theory of curves}

\section{Parametrization}

By definition, a regular curve has at least one parametrization.
However, a given parametrization may not have useful properties, so we often take a new parametrization.
The existence of a parametrization with certain properties is one of the main problems in differential geometry.
Practically, the existence proof is usually done by constructing a \emph{diffeomorphism} between open sets in $\R^m$; a bijective smooth map whose inverse is also smooth.

We introduce the arc-length reparametrization.
It is the most general choice for the local study of curves.
\begin{defn}
A parametrization $\a$ of a regular curve is called a \emph{unit speed curve} or an \emph{arc-length parametrization} when it satisfies $\|\a'\|=1$.
\end{defn}
\begin{thm}
Every regular curve may be assumed to have unit speed.
Precisely, for every regular curve, there is a parametrization $\a$ such that $\|\a'\|=1$.
\end{thm}
\begin{pf}
By the definition of regular curves, we can take a parametrization $\beta:I_t\to\R^d$ for a given regular curve.
We will construct an arc-length parametrization from $\beta$.

Define $\tau:I_t\to I_s$ such that
\[\tau(t):=\int_0^t\|\beta'(s)\|\,ds.\]
Since $\tau$ is smooth and $\tau'>0$ everywhere so that $\tau$ is strictly increasing, the inverse $\tau^{-1}:I_s\to I_t$ is smooth by the inverse function theorem; $\tau$ is a diffeomorphism.
Define $\a:I_s\to\R^d$ by $\a:=\beta\circ\tau^{-1}$.
Then, by the chain rule,
\[\a'=\dd{\a}{s}=\dd{\beta}{t}\dd{\tau^{-1}}{s}=\beta'\left(\dd{\tau}{t}\right)^{-1}=\frac{\beta'}{\|\beta'\|}.\qedhere\]
\end{pf}




\section{Frenet-Serret frame}

The Frenet-Serret frame is a standard frame for a curve, and it is in particular effective when we assume the arc-length parametrization.
It is defined for nondegenerate regular curves, i.e. nowhere straight curves.
It provides with a useful orthonormal basis of $T_p\R^3\supset T_p\gamma(I)$ for points $p$ on a regular curve $\gamma:I\to\R^3$.
\begin{prb}
A regular curve $\gamma:I\to\R^3$ is called \emph{non-degenerate} if the normalized tangent vector $\gamma'/\|\gamma'\|$ is never locally constant everywhere.
In other words, $\gamma$ is nowhere straight.
\end{prb}

\begin{defn}[Frenet-Serret frame]
Let $\a$ be a nondegenerate curve.
The \emph{tangent unit vector}, \emph{normal unit vector}, \emph{binormal unit vector} are $T_p\R^3$-valued vector fields on $\a$ defined by:
\[\rT(t):=\frac{\a'(t)}{\|\a'(t)\|},\qquad\rN(t):=\frac{\rT'(t)}{\|\rT'(t)\|},\qquad\rB(t):=\rT(t)\times\rN(t).\]
The set of vector fields $\{\rT,\rN,\rB\}$, which is called \emph{Frenet-Serret frame}, forms an orthonormal basis of $T_p\R^3$ at each point $p$ on $\a$.
The Frenet-Serret frame is uniquely determined up to sign as $\a$ changes.
\end{defn}

We study the derivatives of the Frenet-Serret frame and their coordinate representations.
In the coordinate representations on the Frenet-Serret frame, important geometric measurements such as curvatrue and torsion come out as coefficients.

\begin{defn}
Let $\a$ be a nondegenerate curve.
The \emph{curvature} and \emph{torsion} are scalar fields on $\a$ defined by:
\[\kappa(t):=\frac{\<\rT'(t),\rN(t)\>}{\|\a'\|},\quad\tau(t):=-\frac{\<\rB'(t),\rN(t)\>}{\|\a'\|}.\]
Note that $\kappa>0$ cannot vanish by definition of nondegenerate curve.
This definition is independent on $\a$.
\end{defn}

\begin{prb}\emph{Frenet-Serret formula.}
Let $\gamma$ be a non-degenerate regular curve.
Then,
\[\mat{\rT'\\\rN'\\\rB'}=\|\gamma'\|\mat{0&\kappa&0\\-\kappa&0&\tau\\0&-\tau&0}\mat{\rT\\\rN\\\rB}.\]
\begin{parts}
\item $\rT'=\|\gamma'\|\kappa\rN$.
\item $\rB'=-\|\gamma'\|\tau\rN$.
\item $\rN'=-\|\gamma'\|\kappa\rT+\|\gamma'\|\tau\rB$.
\end{parts}
\end{prb}
\begin{pf}
Note that $\{\rT,\rN,\rB\}$ is an orthonormal basis.

(a)
Two vectors $\rT'$ and $\rN$ are parallel by definition of $\rN$.
By the definition of $\kappa$, we get $\rT'=\|\gamma'\|\kappa\rN$.

(b)
Since $\<\rT,\rB\>=0$ and $\<\rB,\rB\>=1$ are constant, we have
\[\<\rB',\rT\>=\<\rB,\rT\>'-\<\rB,\rT'\>=0,\qquad\<\rB',\rB\>=\tfrac12\<\rB,\rB\>'=0.\]
By the definition of $\tau$, we get $\rB'=-\|\a'\|\tau\rN$.

(c)
Since
\begin{align*}
\<\rN',\rT\>&=-\<\rN,\rT'\>=-\|\a'\|\kappa,\\
\<\rN',\rN\>&=\tfrac12\<\rN,\rN\>'=0,\\
\<\rN',\rB\>&=-\<\rN,\rB'\>=\|\a'\|\tau,
\end{align*}
we have
\[\rN'=\|\a'\|(-\kappa\rT+\tau\rB).\qedhere\]
\end{pf}
\begin{rmk}
Let $\rX(t)$ be the curve of orthogonal matrices $(\rT(t),\rN(t),\rB(t))^T$.
Then, the Frenet-Serret formula reads
\[\rX'(t)=A(t)\rX(t)\]
for a matrix curve $A(t)$ that is completely determined by $\kappa(t)$ and $\tau(t)$, if we let us only consider arc-length parametrized curves.
This is a typical form of an ODE system, so we can apply the Picard-Lindel\"of theorem to get the following proposition: if we know $\kappa(t)$ and $\tau(t)$ for all time $t$, and if $\rT(0)$ and $\rN(0)$ are given so that an initial condition
\[\rX(0)=(\rT(0),\,\rN(0),\,\rT(0)\times\rN(0))\]
is established, then the solution $\rX(t)$ exists and uniquely determined in a short time range.
Furthermore, if $\a(0)$ is given in addition, the integration
\[\a(t)=\a(0)+\int_0^t\rT(s)\,ds\]
provides a complete formula for unit speed parametrization $\a$.
\end{rmk}
\begin{rmk}
Skew-symmetry in the Frenet-Serret formula is not by chance.
Let $\rX(t)=(\rT(t),\rN(t),\rB(t))^T$ and write $\rX'(t)=A(t)\rX(t)$ as we did in the above remark.
Since $\rX(t+h)=R_t(h)\rX(t)$ for a family of special orthogonal matrices $\{R_t(h)\}_h$ with $R_t(0)=I$, we can describe $A(t)$ as 
\[A(t)=\left.\dd{R_t}{h}\right\rvert_{h=0}.\]
By differentiating the relation $R_t^T(h)R_t(h)=I$ with respect to $h$, we get to know that $A(t)$ is skew-symmetric for all $t$.
In other words, the tangent space $T_I\SO(3)$ forms a skew symmetric matrix.
\end{rmk}













\section{Computational problems}

The following proposition gives the most effective and shortest way to compute the Frenet-Serret apparatus in general case.
If we try to reparametrize the given curve into a unit speed curve or find $\kappa$ by differentiating $\rT$, then we must encounter the normalizing term of the form $\sqrt{(-)^2+(-)^2+(-)^2}^{-1}$, and it must be painful when time is limited.
The Frenet-Serret frame is useful in proofs of interesting propositions, but not a good choice for practical computation.
Instead, a computation from derivatives of parametrization is highly recommended.
\begin{prop}
Let $\a$ be a nondegenerate curve.
Then,
\[\kappa=\frac{\|\a'\times\a''\|}{\|\a'\|^3},\qquad\tau=\frac{\a'\times\a''\cdot\a'''}{\|\a'\times\a''\|}\]
and
\[\rT=\frac{\a'}{\|\a'\|},\qquad\rB=\frac{\a'\times\a''}{\|\a'\times\a''\|},\qquad\rN=\rB\times\rT.\]
\end{prop}
\begin{pf}
If we let $s=\|\a'\|$, then
\begin{align*}
\a'&=s\rT,\\
\a''&=s'\rT+s^2\kappa\rN,\\
\a'''&=(s''-s^3\kappa^2)\rT+(3ss'\kappa+s^2\kappa')\rN+(s^3\kappa\tau)\rB.
\end{align*}
Now the formulas are easily derived.
\end{pf}

% Examples


\section{General problems}

We are interested in regular curves, not a particular parametrization.
By the Theorem 2.1, we may always assume that a parametrization $\a$ has unit speed.
Let $\a$ be a nondegenerate unit speed space curve, and let $\{\rT,\rN,\rB\}$ be the Frenet-Serret frame for $\a$.

Consider a diagram as follows:
\begin{cd}
\<\a,\rT\>=\ ?\ar{r}\ar{d} & \<\a,\rN\>=\ ? \ar{l}\ar{d}\ar{r} & \<\a,\rB\>=\ ? \ar{l}\ar{d} \\
\<\a',\rT\>=1 & \<\a',\rN\>=0 &\<\a',\rB\>=0.
\end{cd}
Here the arrows indicate which term we are able to get by differentiation.
For example, if we know a condition
\[\<\a(t),\rT(t)\>=f(t),\]
then we can obtain
\[\<\a(t),\rN(t)\>=\frac{f'(t)-1}{\kappa(t)}\]
by direct differentiation since we have known $\<\a',\rT\>$ but not $\<\a,\rN\>$.
Further, we get
\[\<\a(t),\rB(t)\>=\frac{\left(\frac{f'(t)-1}{\kappa(t)}\right)'+\kappa(t)f(t)}{\tau(t)}\]
since we have known $\<\a,\rT\>$ and $\<\a',\rN\>$ but not $\<\a,\rB\>$.
Thus, $\<\a,\rT\>=f$ implies
\[\a(t)=f(t)\cdot\rT+\frac{f'(t)-1}{\kappa(t)}\cdot\rN+\frac{\left(\frac{f'(t)-1}{\kappa(t)}\right)'+\kappa(t)f(t)}{\tau(t)}\cdot\rB,\]
when given $\tau(t)\ne0$.

We suggest a strategy for space curve problems:
\begin{itemize}
\item Build and differentiate equations of the following form:
\[\<\ \text{(interesting vector)},\ \text{(Frenet-Serret basis)}\ \>\ =\ \text{(some function)}.\]
\item Aim for finding the coefficients of the position vector in the Frenet-Serret frame, and obtain relations of $\kappa$ and $\tau$ by comparing with assumptions.
\item Heuristically find a constant vector and show what you want directly.
\end{itemize}
Here we give example solutions of several selected problems.
Always $\a$ denotes a reparametrized unit speed nondegenerate curve in $\R^3$.


If
\[f=\<\a-p,\rT\>,\quad g=\<\a-p,\rN\>,\quad h=\<\a-p,\rB\>,\]
then
\[f'=1+\kappa g,\quad g'=-\kappa f+\tau h,\quad h'=-\tau g.\]

\begin{prb}
A curve whose normal lines always pass through a fixed point lies in a circle.
\end{prb}
\begin{sol}
\Step{1}[Formulate conditions]
By the assumption, there is a constant point $p\in\R^3$ such that the vectors $\a-p$ and $\rN$ are parallel so that we have
\[\<\a-p,\rT\>=0,\qquad\<\a-p,\rB\>=0.\]
Our goal is to show that $\|\a-p\|$ is constant and there is a constant vector $v$ such that $\<\a-p,v\>=0$.

\Step{2}[Collect information]
Differentiate $\<\a-p,\rT\>=0$ to get
\[\<\a-p,\rN\>=-\frac1\kappa.\]
Differentiate $\<\a-p,\rB\>=0$ to get
\[\tau=0.\]

\Step{3}[Complete proof]
We can deduce that $\|\a-p\|$ is constant from
\[(\|\a-p\|^2)'=\<\a-p,\a-p\>'=2\<\a-p,\rT\>=0.\]
Also, if we heuristically define a vector $v:=\rB$, then $v$ is constant since
\[v'=-\tau\rN=0,\]
and clearly $\<\a-p,v\>=0$
\end{sol}

\begin{prb}
A spherical curve of constant curvature lies in a circle.
\end{prb}
\begin{sol}
\Step{1}[Formulate conditions]
The condition that $\a$ lies on a sphere can be given as follows: for a constant point $p\in\R^3$,
\[\|\a-p\|=\const.\]
Also we have
\[\kappa=\const.\]

\Step{2}[Collect information]
Differentiate $\|\a-p\|^2=\const$ to get
\[\<\a-p,\rT\>=0.\]
Differentiate $\<\a-p,\rT\>=0$ to get
\[\<\a-p,\rN\>=-\frac1\kappa.\]
Differentiate $\<\a-p,\rN\>=-1/\kappa=\const$ to get
\[\tau\<\a-p,\rB\>=0.\]

There are two ways to show that $\tau=0$.

\emph{Method 1}:
Assume that there is $t$ such that $\tau(t)\ne0$.
By the continuity of $\tau$, we can deduce that $\tau$ is locally nonvanishing.
In other words, we have $\<\a-p,\rB\>=0$ on an open interval containing $t$.
Differentiate $\<\a-p,\rB\>=0$ at $t$ to get $\<\a-p,\rN\>=0$ near $t$, which is a contradiction.
Therefore, $\tau=0$ everywhere.

\emph{Method 2}:
Since $\<\a-p,\rB\>$ is continuous and
\[\<\a-p,\rB\>=\pm\sqrt{\|\a-p\|^2-\<\a-p,\rT\>^2-\<\a-p,\rN\>^2}=\pm\const,\]
we get $\<\a-p,\rB\>=\const$.
Differentiate to get $\tau\<\a-p,\rN\>=0$.
Finally we can deduce $\tau=0$ since $\<\a-p,\rN\>\ne0$.

\Step{3}[Complete proof]
The zero torsion implies that the curve lies on a plane.
A planar curve in a sphere is a circle.
\end{sol}

\begin{prb}
A curve such that $\tau/\kappa=(\kappa'/\tau\kappa^2)'$ lies on a sphere.
\end{prb}
\begin{sol}
\Step{1}[Find the center heuristically]
If we assume that $\a$ is on a sphere so that we have $\|\a-p\|=r$ for constants $p\in\R^3$ and $r>0$, then by the routine differentiations give
\[\<\a-p,\rT\>=0,\qquad\<\a-p,\rN\>=-\frac1\kappa,\qquad\<\a-p,\rB\>=-\left(\frac1\kappa\right)'\frac1\tau,\]
that is,
\[\a-p=-\frac1\kappa\rN-\left(\frac1\kappa\right)'\frac1\tau\rB.\]

\Step{2}[Complete proof]
Let us get started the proof.
Define
\[p:=\a+\frac1\kappa\rN+\left(\frac1\kappa\right)'\frac1\tau\rB.\]
We can show that it is constant by differentiation.
Also we can show that
\[\<\a-p,\a-p\>\]
is constant by differentiation.
So we are done.
\end{sol}

\begin{prb}
A curve with more than one Bertrand mates is a circular helix.
\end{prb}
\begin{sol}
\Step{1}[Formulate conditions]
Let $\beta$ be a Bertrand mate of $\a$ so that we have
\[\beta=\a+\lambda\rN,\qquad\rN_\beta=\pm\rN,\]
where $\lambda$ is a function not vanishing somewhere and $\{\rT_\beta,\rN_\beta,\rB_\beta\}$ denotes the Frenet-Serret frame of $\beta$.
We can reformulate the conditions as follows:

Note that $\beta$ is not unit speed.

\Step{2}[Collect information]
Differentiate $\<\beta-\a,\rN\>=\lambda$ to get
\[\lambda=\const\ne0.\]
Differentiate $\<\beta-\a,\rT\>=0$ and $\<\beta-\a,\rB\>=0$ to get
\[\<\rT_\beta,\rT\>=\frac{1-\lambda\kappa}{\|\beta'\|},\qquad\<\rT_\beta,\rB\>=\frac{\lambda\tau}{\|\beta'\|}.\]
Differentiate $\<\rT_\beta,\rT\>$ and $\<\rT_\beta,\rB\>$ to get
\[\frac{1-\lambda\kappa}{\|\beta'\|}=\const,\qquad\frac{\lambda\tau}{\|\beta'\|}=\const.\]
Thus, there exists a constant $\mu$ such that
\[1-\lambda\kappa=\mu\lambda\tau\]
if $\a$ is not planar so that $\tau\ne0$.

We have shown that the torsion is either always zero or never zero at every point: $\lambda\tau/\|\beta'\|=\const$.
The problem can be solved by dividing the cases, but in this solution we give only for the case that $\a$ is not planar; the other hand is not difficult.

\Step{3}[Complete proof]
If
\[\beta=\a+\lambda\rN,\qquad\tilde\beta=\a+\tilde\lambda\rN\]
are different Bertrand mates of $\a$ with $\lambda\ne\tilde\lambda$, then $(\kappa,\tau)$ solves a two-dimensional linear system
\begin{align*}
\kappa+\mu\tau&=\lambda^{-1},\\
\kappa+\tilde\mu\tau&=\tilde\lambda^{-1}.
\end{align*}
It is nonsingular since $\mu=\tilde\mu$ implies $\lambda=\tilde\lambda$, which means we can represent $\kappa$ and $\tau$ in terms of constants $\lambda,\tilde\lambda,\mu,$ and $\tilde\mu$.
Therefore, $\kappa$ and $\tau$ are constant.
\end{sol}

Here is a well-prepared problem set for exercises.

\begin{prb}[Plane curves]
Let $\a$ be a nondegenerate curve in $\R^3$.
TFAE:
\begin{parts}
\item the curve $\a$ lies on a plane,
\item $\tau=0$,
\item the osculating plane constains a fixed point.
\end{parts}
\end{prb}

\begin{prb}[Helices]
Let $\a$ be a nondegenerate curve in $\R^3$.
TFAE:
\begin{parts}
\item the curve $\a$ is a helix,
\item $\tau/\kappa=\const$,
\item normal lines are parallel to a plane.
\end{parts}
\end{prb}

\begin{prb}[Sphere curves]
Let $\a$ be a nondegenerate curve in $\R^3$.
TFAE:
\begin{parts}
\item the curve $\a$ lies on a sphere,
\item $(1/\kappa)^2+((1/\kappa)'/\tau)^2=\const$,
\item $\tau/\kappa=(\kappa'/\tau\kappa^2)'$,
\item normal planes contain a fixed point.
\end{parts}
\end{prb}

\begin{prb}[Bertrand mates]
Let $\a$ be a nondegenerate curve in $\R^3$.
TFAE:
\begin{parts}
\item the curve $\a$ has a Bertrand mate,
\item there are two constants $\lambda\ne0,\mu$ such that $1/\lambda=\kappa+\mu\tau$.
\end{parts}
\end{prb}












\chapter{Local theory of surfaces}

\section{Reparametrization}

%         선형독립    벡터들이 한점에서 주어졌을 때 ->
%         선형독립    벡터장이 근방에서 주어졌을 때 -> 일반적으론 2차원에서만
%         선형독립 가환벡터장이 근방에서 주어졌을 때 -> n차원 다돼
%         선형독립    직교벡터들이 한점에서 주어졌을 때
%         선형독립    직교벡터장이 근방에서 주어졌을 때
%         선형독립 직교가환벡터장이 근방에서 주어졌을 때
%             -> 곡률의 선, 점근곡선, 측지좌표

% preimage theorem

\begin{thm}
Let $S$ be a regular surface.
Let $v,w$ be linearly independent tangent vectors in $T_pS$ for a point $p\in S$.
Then, $S$ admits a parametrization $\a$ such that $\a_x|_p=v$ and $\a_y|_p=w$.
\end{thm}
\begin{thm}
Let $X,Y$ be linearly independent tangent vector fields on a regular surface $S$.
Then, $S$ admits a parametrization $\a$ such that $\a_x|_p$ and $\a_y|_p$ are parallel to $X|_p,Y|_p$ respectively for each $p\in S$.
\end{thm}
\begin{thm}
Let $X,Y$ be linearly independent tangent vector fields on a regular surface $S$.
If $\pd_XY=\pd_YX$, then $S$ admits a parametrization $\a$ such that $\a_x|_p=X|_p$ and $\a_y|_p=Y|_p$ for each $p\in S$.
\end{thm}

Let $S$ be a regular surface embedded in $\R^3$.
The inner product on $T_pS$ induced from the standard inner product of $\R^3$ can be represented not only as a matrix
\[\mat{1&0&0\\0&1&0\\0&0&1}\]
in the basis $\{(1,0,0),(0,1,0),(0,0,1)\}\subset\R^3$, but also as a matrix
\[\mat{\<\a_x,\a_x\>&\<\a_x,\a_y\>\\\<\a_y,\a_x\>&\<\a_y,\a_y\>}\]
in the basis $\{\a_x|_p,\a_y|_p\}\subset T_pS$.

\begin{defn}
\emph{Metric coefficients}
\begin{alignat*}{2}
\<\a_x,\a_x\>&=:g_{11}&\qquad
\<\a_x,\a_y\>&=:g_{12}\\
\<\a_y,\a_x\>&=:g_{21}&
\<\a_y,\a_y\>&=:g_{22}
\end{alignat*}
\end{defn}

\begin{thm}[Normal coordinates]
...?
\end{thm}




\section{Differentiation of tangent vectors}

\begin{defn}
Let $\a:U\to\R^3$ be a regular surface.
The \emph{Gauss map} or \emph{normal unit vector} $\nu:U\to\R^3$ is a vector field on $\a$ defined by:
\[\nu(x,y):=\frac{\a_x\times \a_y}{\|\a_x\times \a_y\|}(x,y).\]
The set of vector fields $\{\a_x|_p,\a_y|_p,\nu|_p\}$ forms a basis of $T_p\R^3$ at each point $p$ on $\a$.
The Gauss map is uniquely determined up to sign as $\a$ changes.
\end{defn}

\begin{defn}[Gauss formula, $\Gamma_{ij}^k$, $L_{ij}$]
Let $\a:U\to\R^3$ be a regular surface.
Define indexed families of smooth functions $\{\Gamma_{ij}^k\}_{i,j,k=1}^2$ and $\{L_{ij}\}_{i,j=1}^2$ by the Gauss formula
\begin{alignat*}{2}
\a_{xx}&=:\Gamma_{11}^1\a_x+\Gamma_{11}^2\a_y+L_{11}\nu,&\qquad
\a_{xy}&=:\Gamma_{12}^1\a_x+\Gamma_{12}^2\a_y+L_{12}\nu,\\
\a_{yx}&=:\Gamma_{21}^1\a_x+\Gamma_{21}^2\a_y+L_{21}\nu,&
\a_{yy}&=:\Gamma_{22}^1\a_x+\Gamma_{22}^2\a_y+L_{22}\nu.
\end{alignat*}
The \emph{Christoffel symbols} refer to eight functions $\{\Gamma_{ij}^k\}_{i,j,k=1}^2$.
The Christoffel symbols and $L_{ij}$ \emph{do depend} on $\a$.
\end{defn}
We can easily check the symmetry $\Gamma_{ij}^k=\Gamma_{ji}^k$ and $L_{ij}=L_{ji}$.
Also,
\begin{align*}
\pd_XY
&=X^i\pd_i(Y^j\a_j)\\
&=X^i(\pd_iY^k)\a_k+X^iY^j\pd_i\a_j\\
&=\left(X^i\pd_iY^k+X^iY^j\Gamma_{ij}^k\right)\a_k+X^iY^jL_{ij}\nu.
\end{align*}

% Examples


\section{Differentiation of normal vector}

The partial derivative $\pd_X\nu$ is a tangent vector field since
\[\<\pd_X\nu,\nu\>=\frac12\pd_X\<\nu,\nu\>=0.\]
Therefore, we can define the following useful operator.
\begin{defn}
Let $S$ be a regular surface embedded in $\R^3$.
The \emph{shape operator} is $\cS:\fX(S)\to\fX(S)$ defined as
\[\cS(X):=-\pd_X\nu.\]
\end{defn}
\begin{prop}
The shape operator is self-adjoint, i.e. symmetric.
\end{prop}
\begin{pf}
Recall that $\pd_XY-\pd_YX$ is a tangent vector field.
Then,
\[\<X,\cS(Y)\>=\<X,-\pd_Y\nu\>=\<\pd_YX,\nu\>=\<\pd_XY,\nu\>=\<\cS(X),Y\>.\qedhere\]
\end{pf}

% The reason of minus sign in the shape operator.

\begin{thm}
Let $\a:U\to\R^3$ be a regular surface and $\cS$ be the shape operator.
Then $\cS$ has the coordinate representation
\[\cS=\mat{g_{11}&g_{12}\\g_{21}&g_{22}}^{-1}\mat{L_{11}&L_{12}\\L_{21}&L_{22}}\]
with respect to the frame $\{\a_x,\a_y\}$ for tangent spaces.
In other words, if we let $X=X^i\a_i$ and $\cS(X)=\cS(X)^j\a_j$, then
\[\mat{\cS(X)^1\\\cS(Y)^2}=\mat{g_{11}&g_{12}\\g_{21}&g_{22}}^{-1}\mat{L_{11}&L_{12}\\L_{21}&L_{22}}\mat{X^1\\X^2}.\]
\end{thm}
\begin{pf}
Let $\cS(X)^j=\cS_i^jX_i$.
Then,
\[g_{ik}X^i\cS_j^kY^j=\<X,\cS(Y)\>=\<\pd_XY,\nu\>=X^iY^jL_{ij}\]
implies $g_{ik}\,\cS_j^k=L_{ij}$.
\end{pf}

% principal curvature
% mean curvature, gaussian curvature



% curvature tensor?


\section{Computational problems}
% 제1기본형식, 크리스토펠: 
% 모양 연산자, 제2기본형식: 바인가르텐 이퀘이션

% 가우스곡률
\begin{defn}
Let $\a:U\to\R^3$ be a regular surface.
\begin{gather*}
E:=\<\a_x,\a_x\>=g_{11},\qquad F:=\<\a_x,\a_y\>=g_{12},\qquad G:=\<\a_y,\a_y\>=g_{22},\\
L:=\<\a_{xx},\nu\>=L_{11},\qquad M:=\<\a_{xy},\nu\>=L_{12},\qquad N:=\<\a_{yy},\nu\>=L_{22}.
\end{gather*}
\end{defn}


\begin{cor}
We have $GM-FN=EM-FL$, and the \emph{Weingarten equations}:
\begin{align*}
\nu_x&=\frac{FM-GL}{EG-F^2}\a_x+\frac{FL-EM}{EG-F^2}\a_y,\\
\nu_y&=\frac{FN-GM}{EG-F^2}\a_x+\frac{FM-EN}{EG-F^2}\a_y.
\end{align*}
\end{cor}



\begin{thm}
\[\Gamma_{ij}^l=\frac12g^{kl}(g_{ik,j}-g_{ij,k}+g_{kj,i}).\]
\end{thm}

\[\frac12(\log g)_x=\Gamma_{11}^1.\]

\[\nu_x\times\nu_y=K\sqrt{\det g}\ \nu.\]
\[\a_x\times\a_y=\sqrt{\det g}\ \nu\]
\[\<\nu_x\times\nu_y,\a_x\times\a_y\>=\det\mat{\<\nu_x,\a_x\>&\<\nu_x,\a_y\>\\\<\nu_y,\a_x\>&\<\nu_y,\a_y\>}=\det\mat{-L&-M\\-M&-N}=K\det g\]











\begin{thm}[Gaussian curvature formula]
$ $\\[-12pt]
\begin{parts}
\item
In general,
\[K=\frac{LN-M^2}{EG-F^2}.\]
\item
For orthogonal coordinates such that $F\equiv0$,
\[K=-\frac1{2\sqrt{\det g}}\left((\frac1{\sqrt{\det g}}E_y)_y+(\frac1{\sqrt{\det g}}G_x)_x\right).\]
\item
For $f(x,y,z)=0$,
\[K=-\frac1{|\nabla f|^4}\mat[v]{0&\nabla f\\\nabla f^T&\Hess(f)},\]
where $\nabla f$ denotes the gradient $\nabla f=(f_x,f_y,f_z)$.
\item(Beltrami-Enneper) If $\tau$ is the torsion of an asymptotic curve, then
\[K=-\tau^2.\]
\item(Brioschi) $E,F,G$ describes $K$.
\end{parts}
\end{thm}

\begin{pf}$ $\\[-12pt]
\begin{parts}
\item Clear.
\item
We have $GM=EM$ and
\[\nu_x=-\frac LE\a_x-\frac MG\a_y,\qquad\nu_y=-\frac ME\a_x-\frac NG\a_y.\]
\[\nu_x\times\nu_y=\frac{LN-M^2}{EG}\a_x\times\a_y\]
After curvature tensors...
\end{parts}
\end{pf}



\begin{ex}
\begin{parts}
\item
(Monge's patch)
For $(x,y,f(x,y))$,
\[K=\frac{f_{xx}f_{yy}-f_{xy}^2}{(1+f_x^2+f_y^2)^2}.\]
\item
(Surface of revolution).
Let $\gamma(t)=(r(t),z(t))$ be a plane curve with $r(t)>0$.
Let
\[\a(\theta,t)=(r(t)\cos\theta,r(t)\sin\theta,z(t))\]
be a parametrization of a surface of revolution.

Then,
\begin{align*}
\a_\theta&=(-r(t)\sin\theta,r(t)\cos\theta,0)\\
\a_t&=(r'(t)\cos\theta,r'(t)\sin\theta,z'(t))\\
\nu&=\frac1{\sqrt{r'(t)^2+z'(t)^2}}(z'(t)\cos\theta,z'(t)\sin\theta,-r'(t)),
\end{align*}
and
\begin{align*}
\a_{\theta\theta}&=(-r(t)\cos\theta,-r(t)\sin\theta,0)\\
\a_{\theta t}&=(-r'(t)\sin\theta,-r'(t)\cos\theta,0)\\
\a_{tt}&=(r''(t)\cos\theta,r''(t)\sin\theta,z''(t)).
\end{align*}
Thus we have
\[E=r(t)^2,\quad F=0,\quad G=r'(t)^2+z'(t)^2,\]
and
\[L=-\frac{r(t)z'(t)}{\sqrt{r'(t)^2+z'(t)^2}},\quad M=0,\quad N=\frac{r''(t)z'(t)-r'(t)z''(t)}{\sqrt{r'(t)^2+z'(t)^2}}.\]
Therefore,
\[K=\frac{LN-M^2}{EG-F^2}=\frac{z'(r'z''-r''z')}{r(r'^2+z'^2)^2}.\]
In particular, if $t\mapsto(r(t),z(t))$ is a unit-speed curve, then
\[K=-\frac{r''}r.\]

\item
(Models of hyperbolic planes)
\end{parts}
\end{ex}


% asymptotic curve -> hyperbolic
% line of curvature -> non-umbilic

% minimal surface
% 	회전곡면
% asymptotic curve
% 	Beltrami-Enneper
% ruled surface
% developable surface

% 밀만 파커 다 풀어보기




\section{General problems}

% isometric....?

\begin{thm}
Surfaces of the same constant Gaussian curvature are locally isomorphic.
\end{thm}
\begin{pf}
Let
\[\mat{\|\a_r\|^2&\<\a_r,\a_t\>\\\<\a_t,\a_r\>&\|\a_t\|^2}=\mat{1&0\\0&h(r,t)^2}\]
be the first fundamental form for a geodesic coordinate chart along a geodesic curve so that $\a_{tt}$ and $\a_{rr}$ are normal to the surface.
Then,
\[K=-\frac{h_{rr}}h\]
is constant.
Also, since
\[\frac12(h^2)_r+\<\a_r,\a_{tt}\>=\<\a_{rt},\a_t\>+\<\a_r,\a_{tt}\>=\<\a_r,\a_t\>_t=0\]
implies $h_r=0$ at $r=0$, the function $f:r\mapsto h(r,t)$ satisfies the following initial value problem
\[f_{rr}=-Kf,\quad f(0)=1,\quad f'(0)=0.\]
Therefore, $h$ is uniquely determined by $K$.
\end{pf}



\chapter{Geodesics}
% 측지선에 대한 기저(n,T,S)
% 측지꼬임/곡률
% 측지선 방정식
% 측지완비: 호프 리노프
% 지수사상: 가우스 보조정리
% 야코비장
% 카르탕 아다마르






























\part{Global theory of curves and surfaces}

\chapter{Global theory of curves}
\section{Isoperimetric inequality}
\section{Four vertex theorem}
\section{Ovals}


\chapter{Global theory of surfaces}
% Global theory
%  곡선: 등주부등식, (네 꼭짓점, 펜첼/페리-밀너), 볼록성/오발
%  곡면: 최소곡면, (컴팩트곡면분류, 가우스-보네), 임베딩문제,
\section{Minimal surfaces}
\section{Classification of compact surfaces}
\section{The Hilbert theorem}

\chapter{Total curvatures}
\section{The Fary-Minor theorem}
Fenchel's theorem
\section{The Gauss-Bonnet theorem}

\end{document}