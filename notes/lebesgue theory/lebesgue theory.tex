\documentclass{../note}
\usepackage{../../ikany}


\begin{document}
\title{Lebesgue Theory}
\author{Ikhan Choi}
\maketitle
\tableofcontents

\part{Measure theory}



\chapter{Measures and $\sigma$-algebras}

\section{Definition of measures}



\chapter{Carath\'eodory extension}

\begin{prb}[Outer measures]
Let $\Omega$ be a set.
An \emph{outer measure} on $\Omega$ is a function $\mu^*:\cP(\Omega)\to[0,\infty]$ with $\mu^*(\varnothing)=0$ such that
\begin{enumerate}[(i)]
\item if $E_1\subset E_2$, then $\mu^*(E_1)\le\mu^*(E_2)$,
\hfill(monotonicity)
\item $\mu^*(\bigcup_{i=1}^\infty E_i)\le\sum_{i=1}^\infty\mu^*(E_i)$,
\hfill(countable subadditivity)
\end{enumerate}
for any $\{E_i\}_{i=1}^\infty\subset\cP(\Omega)$.
\begin{parts}
\item A function $\mu^*:\cP(\Omega)\to[0,\infty]$ with $\mu^*(\varnothing)=0$ is an outer measure if and only if $\mu^*(E)\le\sum_{i=1}^\infty\mu^*(E_i)$ whenever $E\subset\bigcup_{i=1}^\infty E_i$.
\item
Let $\cA\subset\cP(\Omega)$ with $\varnothing\in\cA$.
If a function $\rho:\cA\to[0,\infty]$ satisfies $\rho(\varnothing)=0$, then we can associate an outer measure $\mu^*:\cP(\Omega)\to[0,\infty]$ by defining as
\[\mu^*(E):=\inf\left\{\,\sum_{i=1}^\infty\rho(A_i):E\subset\bigcup_{i=1}^\infty A_i,\ A_i\in\cA\,\right\},\]
where we use the convention $\inf\varnothing=\infty$.
\end{parts}
\end{prb}


\begin{prb}[Carath\'eodory measure]
Let $\mu^*$ be an outer measure on a set $\Omega$.
A subset $A\subset \Omega$ is called \emph{Carath\'eodory measurable} relative to $\mu^*$ if
\[\mu^*(E)=\mu^*(E\cap A)+\mu^*(E\setminus A)\]
for every subset $E\subset\Omega$.
Let $\cM$ be the collection of all Carath\'eodory measurable subsets relative to $\mu^*$.
\begin{parts}
\item $\cM$ is an algebra and $\mu^*$ is finitely additive on $\cM$.
\item $\cM$ is a $\sigma$-algebra and $\mu^*$ is countably additive on $\cM$.
\item The measure $\mu:=\mu^*|_\cM:\cM\to[0,\infty]$ is complete.
We call $\mu$ the \emph{Carath\'eodory measure} constructed from $\rho$.
\end{parts}
\end{prb}


\begin{prb}[Carath\'eodory extension theorem]
Let $\cA\subset\cP(\Omega)$ with $\varnothing\in\cA$.
Let $\rho:\cA\to[0,\infty]$ with $\rho(\varnothing)=0$.
Consider two conditions
\begin{enumerate}[(i)]
\item $A\subset\bigcup_{i=1}^\infty A_i$ implies $\rho(A)\le\sum_{i=1}^\infty\rho(A_i)$,
\item for any $\e>0$ and $B,A$ there are $A_1,A_2$ such that $B\cap A\subset A_1$, $B\setminus A\subset A_2$ and $\rho(B)+\e>\rho(A_1)+\rho(A_2)$.
\end{enumerate}
Let $\mu^*:\cP(\Omega)\to[0,\infty]$ be the associated outer measure of $\rho$, and $\mu:\cM\to[0,\infty]$ the measure defined by the restriction of $\mu^*$ on Carath\'eodory measurable subsets.
\begin{parts}
\item $\mu^*|_\cA=\rho$ if (i) is satisfied.
\item $\cA\subset\cM$ if (ii) is satisfied.
\end{parts}
\end{prb}
\begin{pf}
(a)
Clearly $\mu^*(A)\le\rho(A)$ for $A\in\cA$.

We may assume $\mu^*(A)<\infty$.
For arbitrary $\e>0$ there is $\{A_i\}_{i=1}^\infty$ such that $A\subset\bigcup_{i=1}^\infty A_i$ and
\[\mu^*(A)+\e>\sum_{i=1}^\infty\rho(A_i)\ge\rho(A).\]

(b)
Let $E\in\cP(\Omega)$ and $A\in\cA$.
Then,
$E\subset \bigcup_{i=1}^\infty A_i$ and $A_i\cap A\subset A_{i,1}$ and $A_i\setminus A\subset A_{i,2}$ such that
\begin{align*}
\mu^*(E)+\e>\sum_{i=1}^\infty(\rho(A_i)+\frac\e{2^{i+1}})
&>\sum_{i=1}^\infty\rho(A_{i,1})+\sum_{i=1}^\infty\rho(A_{i,2})\\
&\ge\mu^*(E\cap A)+\mu^*(E\setminus A).
\end{align*}
\end{pf}



\begin{prb}[Carath\'eodory extension from semi-ring]
Let $\cA\subset\cP(\Omega)$ be a semi-ring of sets on a set $X$.
A function $\rho:\cA\to[0,\infty]$ with $\rho(\varnothing)=0$ is called a \emph{pre-measure} if
\begin{enumerate}[(i)]
\item $\rho(\bigsqcup_{i=1}^\infty A_i)\le\sum_{i=1}^\infty\rho(A_i)$,
\hfill(disjoint countable subadditivity)
\item $\rho(\bigsqcup_{i=1}^nA_i)=\sum_{i=1}^n\rho(A_i)$,
\hfill(finite additivity)
\end{enumerate}
for any $\{A_i\}_{i=1}^\infty\subset\cA$ with $\bigsqcup_{i=1}^\infty A_i\in\cA$ and $n\in\N$.

Let $\mu^*:\cP(\Omega)\to[0,\infty]$ be the associated outer measure of $\rho$, and $\mu:\cM\to[0,\infty]$ the measure defined by the restriction of $\mu^*$ on Carath\'eodory measurable subsets.
\begin{parts}
\item A pre-measure is a priori countably additive.
\end{parts}
\end{prb}



\begin{prb}[Uniqueness of Carath\'eodory extensions]
The Carath\'eodory extension theorem provides with a uniqueness theorem for measures.
\end{prb}



Monotone class lemma: alternative direct proof method without using Carath\'eodory extension.


\chapter{Measures on the real line}

distribution functions
helly's selection
non-measurable set


\section*{Exercises}
\begin{prb}*
A Lebesgue measurable set in $\R$ with positive measure contains an arbitrarily long subsequence of an arithmetic progression.
\end{prb}















\part{Lebesgue integral}


\chapter{Measurable functions}

\section{Extended real numbers}


\section{Simple functions}
Pointwise limit of simple functions is measurable.
\begin{pf}
Let $f(x)=\lim_{n\to\infty}s_n(x)$.

\end{pf}

Every measurable extended real-valued function is a pointwise limit of simple functions.





\begin{prb}[Egorov's theorem]
Let $(\Omega,\mu)$ be a finite measure space.
Let $(f_n:\Omega\to\R)_n$ be a sequence of a.e. convergent measurable functions.
For $\e>0$, there exists a measurable $E_\e\subset\Omega$ such that $\mu(\Omega\setminus E_\e)<\e$ and $f_n$ uniformly convergent on $E_\e$.
\end{prb}
\begin{pf}
Assume $f_n\to0$.
The set of convergence is
\[\bigcap_{k>0}\ \bigcup_{n_0>0}\ \bigcap_{n\ge n_0}\{\,x:|f_n(x)|<\tfrac1k\,\},\]
which is a full set.
We want to get rid of the dependence on the point $x$ of $n_0$ in the union $\bigcup_{n_0>0}$.
Since
\[\bigcap_{n\ge n_0}\{\,x:|f_n(x)|<\frac1k\,\}\]
is increasing as $n_0\to\infty$ to a full set for each $k>0$, we can find $n_0(k,\e)$ such that
\[\mu(\bigcap_{n\ge n_0}\{\,x:|f_n(x)|<\tfrac1k\,\})>\mu(\Omega)-\frac\e{2^k}.\]
Then,
\[\mu(\bigcap_{k>0}\ \bigcap_{n\ge n_0}\{\,x:|f_n(x)|<\tfrac1k\,\})>\mu(\Omega)-\e.\]
If we define
\[E_\e:=\bigcap_{k>0}\ \bigcap_{n\ge n_0}\{\,x:|f_n(x)|<\tfrac1k\,\},\]
then for any $k>0$ and $x\in E_\e$, and with the $n_0(k,\e)$ we have chosen,
we have
\[n\ge n_0\quad\Rightarrow\quad |f_n(x)|<\frac1k.\]
\end{pf}



Since $\{f_n(x)\}_n$ diverges if and only if
\[\exists k>0,\quad\forall n_0>0,\quad\exists n>n_0:\quad|f_n(x)-f(x)|>\tfrac1k,\]
we have
\begin{align*}
\{x:\{f_n(x)\}_n\text{ diverges}\}
&=\bigcup_{k>0}\bigcap_{n_0>0}\bigcup_{n>n_0}\{x:|f_n-f|>\tfrac1k\}\\
&=\bigcup_{k>0}\limsup_n\{x:|f_n-f|>\tfrac1k\}.
\end{align*}
Since for every $k$ we have
\begin{align*}
\limsup_n\{x:|f_n-f|>\tfrac1k\}
&\subset\limsup_{n>k}\{x:|f_n-f|>\tfrac1n\}\\
&=\limsup_n\{x:|f_n-f|>\tfrac1n\},
\end{align*}
we have
\[\{x:\{f_n(x)\}_n\text{ diverges}\}\subset\limsup_n\{x:|f_n-f|>\tfrac1n\}.\]











\chapter{Convergence theorems}
\section{Definition of Lebesgue integral}
\section{Convergence theorems}

Stein: Egorov $\to$ BCT $\to$ Fatou $\to$ MCT $\to$ L1 is a measure\\
Stein: BCT + L1 is a measure $\to$ DCT\\
Folland: MCT $\to$ Fatou $\to$ DCT $\to$ BCT





\section{Radon-Nikodym theorem}


\section{Modes of convergence}

\begin{prb}[Convergence in measure]
Let $(X,\mu)$ be a measure space.
Let $f_n$ be a sequence of measurable functions.
If $f_n$ converges to $f$ in measure, then $f_n$ has a subsequence that converges to $f$ $\mu$-a.e.
\end{prb}
\begin{pf}
We can extract a subsequence $f_{n_k}$ such that
\[\mu(\{x:|f_{n_k}-f|>\tfrac1k\})>\tfrac1{2^k}.\]
Since
\[\sum_{k=1}^\infty\mu(\{x:|f_{n_k}-f|>\tfrac1k\})<\infty,\]
by the Borel-Canteli lemma, we get
\[\mu(\limsup_k\{x:|f_{n_k}-f|>\tfrac1k\})=0.\]
Therefore, $f_{n_k}$ converges $\mu$-a.e.
\end{pf}




\chapter{Product measures}
\section{Fubini-Tonelli theorem}
\section{Lebesgue measure on Euclidean spaces}







\part{Linear operators}



\chapter{Lebesgue spaces}
\section{$L^p$ spaces}
\section{$L^2$ spaces}
\section{Dual spaces}
riesz representations








\chapter{Bounded linear operators}
\section{Continuity}
Schur test

\section{Density arguments}
extension of operators

\section{Interpolation}
weak Lp, marcinkiewicz




\chapter{Convergence of linear operators}
\section{Translation and multiplication operators}

\section{Convolution type operators}
approximation of identity

\section{Computation of integral transforms}











\part{Fundamental theorem of calculus}

\chapter{Weak derivatives}

The space of weakly differentiable functions with respect to all variables $=W_\loc^{1,1}$.

\begin{prb}[Product rule for weakly differentiable functions]
We want to show that if $u$, $v$, and $uv$ are weakly differentiable with respect to $x_i$, then $\pd_{x_i}(uv)=\pd_{x_i}uv+u\pd_{x_i}v$.
\begin{parts}
\item If $u$ is weakly differentiable with respect to $x_i$ and $v\in C^1$, then $\pd_{x_i}(uv)=\pd_{x_i}uv+u\pd_{x_i}v$.
\end{parts}
\end{prb}


\begin{prb}[Interchange of differentiation and integration]
Let $f:\Omega\to\R$
such that $f(x,y)$ and $\pd_{x_i}f(x,y)$ are both locally integrable in $x$ and integrable $y$.
Then,
\[\pd_{x_i}\int f(x,y)\,dy=\int\pd_{x_i}f(x,y)\,dy\]
where $\pd_{x_i}$ denotes the weak partial derivative.
\end{prb}





\chapter{Absolutely continuity}

\begin{parts}
\item $f$ is $\Lip_\loc$ iff $f'$ is $L_\loc^\infty$
\item $f$ is $\textrm{AC}_\loc$ iff $f'$ is $L_\loc^1$
\end{parts}
\begin{parts}
\item $f$ is $\Lip$ iff $f'$ is $L^\infty$
\item $f$ is $\textrm{AC}$ iff $f'$ is $L^1$
\item $f$ is $\textrm{BV}$ iff $f'$ is a finite regular Borel measure
\end{parts}



\chapter{Lebesgue differentiation theorem}

\end{document}




\pd{f}{x}{y}가 원점에서 연속이 아니면 못바꾼다?
라고 생각하지 말고 계산된 편미분계수가 편도함수를 잘 표현하지 못할 수 있다고 이해하자.
즉, 편미분계수는 C2임을 보일 때 말고는 C2가 아닌 경우에 쓸모가 없다
