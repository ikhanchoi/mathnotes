\documentclass{../note}
\usepackage{../../ikany}


\begin{document}
\title{Lebesgue Theory}
\author{Ikhan Choi}
\maketitle
\tableofcontents

\part{Measure theory}


\chapter{Measures and $\sigma$-algebras}

\section{Definition of measures}

\section{The Carath\'eodory extension theorem}
\begin{prb}[Outer measures]
Let $X$ be a set.
An \emph{outer measure} on $X$ is a function $\mu^*:\cP(X)\to[0,\infty]$ with $\mu^*(\varnothing)=0$ such that
\begin{parts}[(i)]
\item if $E\subset E'$, then $\mu^*(E)\le\mu^*(E')$,\hfill(monotonicity)
\item $\mu^*(\bigcup_{i=1}^\infty E_i)\le\sum_{i=1}^\infty\mu^*(E_i)$.\hfill(countable subadditivity)
\end{parts}

\begin{parts}
\item A function $\mu^*:\cP(X)\to[0,\infty]$ with $\mu^*(\varnothing)=0$ is an outer measure if and only if $E\subset\bigcup_{i=1}^\infty E_i$ implies $\mu^*(E)\le\sum_{i=1}^\infty\mu^*(E_i)$.
\item
Let $\cA\subset\cP(X)$ such that $\varnothing\in\cA$.
If a function $\rho:\cA\to[0,\infty]$ satisfies $\rho(\varnothing)=0$, then we can associate an outer measure $\mu^*:\cP(X)\to[0,\infty]$ by defining as
\[\mu^*(E):=\inf\left\{\,\sum_{i=1}^\infty\rho(A_i):E\subset\bigcup_{i=1}^\infty A_i,\ A_i\in\cA\,\right\},\]
where we use the convention $\inf\varnothing=\infty$.
\end{parts}
\end{prb}


\begin{prb}[Carath\'eodory measurability]
Let $\mu^*$ be an outer measure on a set $X$.
A subset $A\subset X$ is called \emph{Carath\'eodory measurable} relative to $\mu^*$ if
\[\mu^*(E)=\mu^*(E\cap A)+\mu^*(E\cap A^c)\]e
for every subset $E\subset X$.
Let $\cM$ be the collection of all Carath\'eodory measurable subsets relative to $\mu^*$.
\begin{parts}
\item $\cM$ is an algebra and $\mu^*$ is finitely additive on $\cM$.
\item $\cM$ is a $\sigma$-algebra and $\mu^*$ is countably additive on $\cM$, that is, the restriction $\mu:=\mu^*|_\cM:\cM\to[0,\infty]$ is a measure.
\item The measure $\mu$ is complete.
\end{parts}
\end{prb}

\begin{prb}[The Carath\'eodory extension theorem]
Let $\cA\subset\cP(X)$ be a semi-ring of sets on a set $X$ and $\rho:\cA\to[0,\infty]$ a function with $\rho(\varnothing)=0$.
If the function $\rho$ satisfies
\begin{parts}[(i)]
\item $\rho(A)=\sum_{i=1}^n\rho(A_i)$ for $A\in\cA$ a disjoint union of $\{A_i\}_{i=1}^n\subset\cA$,\hfill(finite additivity)
\item $\rho(A)\le\sum_{i=1}^\infty\rho(A_i)$ for $A\in\cA$ a disjoint union of $\{A_i\}_{i=1}^\infty\subset\cA$,\\\null\hfill((disjoint) countable subadditivity)
\end{parts}
then it is called a \emph{premeasure}.

Let $\mu^*:\cP(X)\to[0,\infty]$ be the associated outer measure of $\rho$, and $\mu:\cM\to[0,\infty]$ the measure defined from $\mu^*$ on Carath\'eodory measurable subsets.
We call $\mu$ the \emph{Carath\'eodory measure} constructed from $\rho$.
\begin{parts}
\item If $\rho$ is finitely additive, then $\cA\subset\cM$.
\item If $\rho$ is countably subadditive, then $\mu^*(A)=\rho(A)$ for every $A\in\cA$.
\item If $\rho$ is a premeasure, then $\mu$ is an extension of $\rho$ and called \emph{Carath\'eodory extension} of $\rho$.
\item In particular, a premeasure is a priori countably additive in the sense that $\rho(A)=\sum_{i=1}^\infty\rho(A_i)$ for $A\in\cA$ a disjoint countable union of $\{A_i\}_{i=1}^\infty\subset\cA$.
\end{parts}
\end{prb}






\chapter{Measures on Euclidean spaces}

\chapter{Measurable functions}

\chapter{}






\part{Integration}

\chapter{Lebesgue integration}
\section{Definition of Lebesgue integration}
\section{Convergence theorems}

\section{Modes of convergence}

Since $\{f_n(x)\}_n$ diverges if and only if
\[\exists k>0,\quad\forall n_0>0,\quad\exists n>n_0:\quad|f_n(x)-f(x)|>\tfrac1k,\]
we have
\begin{align*}
\{x:\{f_n(x)\}_n\text{ diverges}\}
&=\bigcup_{k>0}\bigcap_{n_0>0}\bigcup_{n>n_0}\{x:|f_n-f|>\tfrac1k\}\\
&=\bigcup_{k>0}\limsup_n\{x:|f_n-f|>\tfrac1k\}.
\end{align*}
Since for every $k$ we have
\begin{align*}
\limsup_n\{x:|f_n-f|>\tfrac1k\}
&\subset\limsup_{n>k}\{x:|f_n-f|>\tfrac1n\}\\
&=\limsup_n\{x:|f_n-f|>\tfrac1n\},
\end{align*}
we have
\[\{x:\{f_n(x)\}_n\text{ diverges}\}\subset\limsup_n\{x:|f_n-f|>\tfrac1n\}.\]




\begin{thm}
Let $(X,\mu)$ be a measure space.
Let $f_n$ be a sequence of measurable functions.
If $f_n$ converges to $f$ in measure, then $f_n$ has a subsequence that converges to $f$ $\mu$-a.e.
\end{thm}
\begin{pf}
We can extract a subsequence $f_{n_k}$ such that
\[\mu(\{x:|f_{n_k}-f|>\tfrac1k\})>\tfrac1{2^k}.\]
Since
\[\sum_{k=1}^\infty\mu(\{x:|f_{n_k}-f|>\tfrac1k\})<\infty,\]
by the Borel-Canteli lemma, we get
\[\mu(\limsup_k\{x:|f_{n_k}-f|>\tfrac1k\})=0.\]
Therefore, $f_{n_k}$ converges $\mu$-a.e.
\end{pf}



\chapter{Product measures}
\section{The Fubini theorem}
\section{The Lebesgue measure on Euclidean spaces}

\chapter{Lebesgue spaces}
\section{$L^p$ spaces}
\section{$L^2$ spaces}
\section{The Riesz representation theorem}

\chapter{Integral operators}
\section{Bounded linear operators}
\section{Regular integral operators}
\section{Convolution type operators}
\section{Weak $L^p$ spaces}
\section{Interpolation theorems}

\part{Fundamental theorem of calculus}

\chapter{Absolute continuous functions}

\chapter{Functions of bounded variation}

\chapter{}

\chapter{The Lebesgue differentiation theorem}

\end{document}