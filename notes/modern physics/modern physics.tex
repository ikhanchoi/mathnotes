\documentclass{../note}
\usepackage{../../ikany}


\begin{document}
\title{Modern Physics}
\author{Ikhan Choi}
\maketitle
\tableofcontents

\part{Quantum mechanics}
\chapter{Quantization}
\section{Interpretations of quantum mechanics}
Pictures
Hilbert space, wave functions, Dirac notation
Copenhagen interpretation and measurement
observables and self-adjoint operators
EPR paradox, Bell's inequality, CHSH inequality
\section{Canonical commutation relation}
canonical quantization
Weyl quantization
fourier transform
Stone-von Neumann theorem

\chapter{Schr\"odinger equation}
\section{Time-independent potentials}
Infinite well
Harmonic oscillator
Free particle
Hydrogen atom
\section{Approximation methods}
WKB approximation
\section{Atoms}
\section{Scattering theory}

\chapter{Spin}
\section{}
\section{Dirac equation}
Pair production(1941)
\section{Wigner classification}






\part{Statistical physics}

\chapter{Statistical mechanics}
\section{Ensembles}
ensembles
microcanonical, canonical, grand canonical
classical gas
Boltzmann distribution
\section{Quantum statistics}
Two statistics
Fermi sea
Bose-Einstein condensation


\chapter{Condensed matter physics}
\section{Solid state physics}
phonon
\section{Quantum Hall effect}


\chapter{Renormalization group}
\section{Phase transition}
Magnetic models
Ginzburg Landau theory







\part{Quantum field theory}
\chapter{Perturbative field theory}
\section{Path integral formulation}
\section{Field equations}
\section{Interacting fields}
Feynman diagram

\chapter{Non-perturbative field theory}
\section{Algebraic quantum field theory}

\chapter{Nonabelian gauge theory}





\part{}
\chapter{Supersymmetry}



\end{document}