\documentclass{../note}
\usepackage{../../ikany}


\begin{document}
\title{Probability Theory}
\author{Ikhan Choi}
\maketitle
\tableofcontents


\part{Random variables}
\chapter{Measure theory for probability}

\section{Uniqueness of measures}
\begin{prb}[Dynkin's $\pi$-$\lambda$ theorem]
Let $\cP$ be a $\pi$-system and $\cL$ a $\lambda$-system respectively.
Denote by $\ell(\cP)$ the smallest $\lambda$-system containing $\cP$.
\begin{parts}
\item If $A\in\ell(\cP)$, then $\cG_A:=\{B:A\cap B\in\ell(\cP)\}$ is a $\lambda$-system.
\item $\ell(\cP)$ is a $\pi$-system.
\item If a $\lambda$-system is a $\pi$-system, then it is a $\sigma$-algebra.
\item If $\cP\subset\cL$, then $\sigma(\cP)\subset\cL$.
\end{parts}
\end{prb}

\begin{prb}
\end{prb}

\section{Kolmogorov extension theorem}










\chapter{Probability distributions}

sample space, events
random variable, distributions, expectation


sample space of an "experiment"

equally likely outcomes
	coin toss
	dice roll
	ball drawing
	number permutation
	life time of a light bulb


discrete vs continuous
joint, conditional, expectation

\chapter{Independence}





















\part{Limit theorems}
\chapter{Laws of large numbers}
\chapter{Central limit theorems}
\chapter{}

\part{Stochastic processes}
\chapter{Martingales}
\chapter{Markov chains}
\chapter{Wiener process}

\part{Stochastic calculus}

\end{document}