\documentclass{../crs}
\usepackage{../../ikany}

\title{Analysis 5 : Functional Analysis}

\begin{document}
\maketitle
\tableofcontents





\chapter{Topological vector spaces}

\section{Elementary properties}
definition - how to use the continuity of vector space operations effectively
homeomorphism by translation and dialation: local base at 0
uniformity pseudometrics, basic classification
translation invariant metric
completely regular (up to 3.5)
boundedness and continuity


\section{Classification}
\begin{rd}[row sep={50pt,between origins}, column sep={50pt,between origins}]
& inner product \ar{d} &&&& Hilbert space \ar{d} & \\
& \parbox{10em}{\centering finite\\seminorms\\(= normable)} \ar{d} &&&& Banach space \ar{d} & \\
& \parbox{10em}{\centering countable\\seminorms} \ar{ld}\ar{rd} &&&& Fr\'echet space \ar{ld}\ar{rd} & \\
  \parbox{10em}{\centering countable\\pseudometrics\\(= metrizable)} \ar{rd} &
& \parbox{10em}{\centering uncountable\\seminorms\\(= locally convex)} \ar{ld} &&
  F-space \ar{rd} && (no name) \ar{ld} \\
& \parbox{10em}{\centering uncountable\\pseudometrics} &&&& (no name) &
\end{rd}

\begin{prop}
Let $\rho$ be a pseudometric.
Then,
\[B(0,1)\subset\frac{B(0,1)+B(0,1)}2\subset\frac12B(0,2).\]
If $\rho$ is a seminorm, then the equalities hold.
\end{prop}
I say this as $\frac12B(0,2)$ is ``fatter'' than $B(0,1)$.

















\chapter{Locally convex spaces}
\section{Seminorms}
minkowski functional
locally boundedness
polar

\section{The Hahn-Banach theorem}

\section{Weak topology}

\section{}
\begin{prop}
Let $X$ be a topological vector space.
Then, $X$ has an open convex set if and only if $X^*\ne0$.
\end{prop}
\begin{prop}
For $0<p<1$, $L^p$ is not locally convex set.
If $\mu$ has infinite values, then $(L^p)^*=0$.
\end{prop}
\begin{pf}
Assume $L^p$ is locally convex.
We have $R,r>0$ and a convex set $C$ such that $B_r\subset C\subset B_R$.
We can construct a convex combination of functions in $B_r$ which is not in $B_R$.
It is a contradiction.
\end{pf}

\chapter{Banach spaces}



\section{Barreled spaces}

\subsection{The Baire category theorem}

\subsection{Uniform boundedness principle}
\begin{thm}[Uniform boundedness principle]
Let $X$ be a barreled space and $Y$ be a topological vector space.
Let $\cF\subset B(X,Y)$.
If $\cF$ is pointwisely bounded, then $\cF$ is equicontinuous.
\end{thm}
\begin{cor}
Let $\cF\in\subset B(H)$ for a Hilbert space $H$.
If $\cF$ is poinwisely bounded, then $\cF$ is norm-bounded.
\end{cor}



\subsection{Open mapping theorem}
\begin{thm}[Open mapping theorem]
Let $X$ be a topological vector space and $Y$ be a metrizable barreled space.
Let $T\colon X\to Y$ be linear.
If $T$ is surjective and continuous, then $T$ is open.
\end{thm}
\begin{pf}
If we let $U$ be an open neighborhood in $X$, then we want to show $TU$ is a neighborhood.
Because $T$ is surjective so that $\cl{TU}$ is absorbent, $\cl{TU}$ is a neighborhood.
Note that an open set intersects $\cl{TU}$ also intersects $TU$.

If there exist two sequences of balanced open neighborhoods $U_n\subset X$ and $V_n\subset Y$ with
\begin{cond}
\item $U_1+\cdots+U_n\subset U$,
\item $V_n\subset\cl{TU_n}$,
\item $\bigcap_{n\in\N}V_n=\{0\}$,
\end{cond}
then we can show $V_1\subset TU$.
Here is the proof:
Suppose $y\in V_1$.
Then,
\begin{cd}[row sep={30pt,between origins}, column sep={140pt,between origins}]
y\cap V_1\ne\mt\ar{r}&y\cap\cl{TU_1}\ne\mt\ar{r}&(y+V_2)\cap TU_1\ne\mt\ar{lld}\\
(y+TU_1)\cap V_2\ne\mt\ar{r}&(y+TU_1)\cap\cl{TU_2}\ne\mt\ar{r}&((y+TU_1)+V_3)\cap TU_2\ne\mt\ar{lld}\\
(y+TU_1+TU_2)\cap V_3\ne\varnothing\ar{r}&\quad\cdots.&
\end{cd}
From the first columns, and by the conditions (1) and (3), we obtain
\[(y+TU)\cap\bigcap_{n\in\N}V_n\ne\mt.\]
Therefore, the set $y+TU$ contains 0, hence $y\in TU$.

Let us show the existence of such sequences.
At first, take $U_n=2^{-n}U$ for (1).
Then we can take $\{V_n\}_n$ with (2) as we mentioned above.
Simultaneously we can have it satisfy (3) because $Y$ is metrizable.
\end{pf}
\begin{cor}
Let $X$ be metrizable and $Y$ be barreled.
Then, the open mapping theorem holds.
\end{cor}
\begin{pf}
The quotient of metrizable space is also metrizable, so $Y$ is a metrizble barreled space.
\end{pf}
\begin{cor}[The Banach Isomorphy]
A continuous linear bijection onto a metrizable barreled space is a homeomorphism, i.e. topological isomorphism.
\end{cor}
\begin{cor}[The first isomorphism theorem]
Let $T:X\to Y$ be a bounded linear operator between Banach spaces.
Then, the induced map $X/\ker T\to\im T$ is a topological isomorphism.
\end{cor}















\chapter{Hilbert spaces}
DO NOT contain topics that can be generalized within Banach algebras or any other operator algebras(e.g. polar decomposition, Gelfand theory, functional calculus, spectral resolution)

\begin{thm}
Let $X$ be complete and $Y$ be complete metrizable.
The range of a continuous operator $T:X\to Y$ is closed if and only if the induced linear isomorphism
\[\frac X{\ker T}\to\im T\]
has a continuous inverse so that it becomes a topological isomorphism.
\end{thm}
\begin{pf}
One direction is easy.

For the other direction, suppose $\im T$ is closed in $Y$.
Note that the metrizability condition of $Y$ is set in order to apply the open mapping theorem.
\end{pf}
\begin{cor}
Let $T:X\to Y$ be a bounded operator between Banach spaces.
Then, $T$ is bounded below if and only if $\im T$ is closed and $T$ is injective.
\end{cor}



\section{Spectral theory}


\subsection{Closed operators}
\begin{defn}
An operator $A$ is said to be \emph{closable} if
\[x_n\text{ and }Ax_n\text{ are Cauchy}\impl\lim_{n\to\infty}Ax_n=A\lim_{n\to\infty}x_n.\]
Note that the opposite direction is always true.
\end{defn}
\subsubsection{Properties of closed operators}
For closed operators, we introduce a new norm. 
\begin{thm}
Let $A,B$ be closed operators between Banach spaces.
Then, $A+B$ is closed iff
\[\|Ax\|+\|Bx\|\les\|(A+B)x\|+\|x\|\]
for $x\in D(A)\cap D(B)$, i.e. $A$ and $B$ are $A+B$-bounded.
It is paraphrased by
\[\|x\|_A+\|x\|_B\sim\|x\|_{A+B}.\]
\end{thm}
\begin{pf}
($\Leftarrow$) Suppose $(x_n,(A+B)x_n)$ is Cauchy.
Then, the inequality gives that $Ax_n$ and $Bx_n$ are Cauchy.
Since $A$ and $B$ are closed, we have $\lim Ax_n=A\lim x_n$ and $\lim Bx_n=B\lim x_n$.
So $\lim(A+B)x_n=\lim Ax_n+\lim Bx_n=A\lim x_n+B\lim x_n=(A+B)\lim x_n$.

($\Rightarrow$)
\end{pf}
\begin{thm}
Let $A$ be a closed, and $B$ be a closable operator between Banach spaces with $D(A)\subset D(B)$.
Then, $A+B$ is closed if
\[\|Bx\|\le\alpha\|Ax\|+c\|x\|\]
for some $\alpha<1$.
\end{thm}
\begin{pf}
\[\|Ax\|\le\|(A+B)x\|+\|Bx\|\le\|(A+B)x\|+\alpha\|Ax\|+c\|x\|\]
implies
\[\|Ax\|\les\|(A+B)x\|+\|x\|.\]

\end{pf}

\begin{prop}[Closed graph theorem]
For $T\in D_{cl}(X,Y)$,
\[T\text{ is unbounded}\iff T\text{ is not everywhere defined}.\]
\end{prop}

Closed operators,
\begin{cond}
\item provide with the optimal extended domain for adjoint operators,
\item have maximal essential domains,
\item are closed under invertibility,
\item do not distinguish everywhere defined denslely defined, since everywhere definedness is equivalent to boundedness.
\end{cond}


\subsubsection{Decomposition of spectrum for closed operators}

When a Banach algebra is realized as a concrete operator space, then the spectral theory on it changes drastically.

Note that since decomposition of spectrum is orginated for application to quantum mechanics, this traditional definition is usually for closed operators.
Even though the following definitions can be applied for non-closable operators, but it does not make sense in any senses.
So, every operator in this subsection is assumed to be \emph{closed}.

Let $X=Y$ in order to see $L(X,Y)$ as a ring.
Let $B(X)\subset D(X)\subset L(X)$ be the spaces of \emph{everywhere defined operators, densely defined operators, and just linear operators} respectively.
Note that $D(X)$ is not a vector space.
For $T\in L(X)$,
\[\lambda\begin{cases}\text{is in }\rho(T)\\\text{is in }\sigma_c(T)\\\text{is in }\sigma_r(T)\\\text{is in }\sigma_p(T)\end{cases}\qquad\textit{iff}\qquad R_\lambda(T)\begin{cases}\in B(X)\\\in D(X)\setminus B(X)\\\in L(X)\setminus D(X)\\\text{cannot be defined.}\end{cases}.\]

Discrete spectrum is defined to consist of scalars having finite dimensional eigenspace and is isolated from any other elements in spectrum.
\clearpage
\subsection{Densly defined operators}
\subsubsection{Adjoint}
Adjoint is defined for densely defined operators:
For Banach spaces, we have
\[\adj:D(X,Y)\to L_{cl}(Y^*,X^*)\]
that is not injetcive. (I don't know it's surjective)

For reflexive $Y$, we have
\[\adj:D_{cl}(X,Y)\to D_{cl}(Y^*,X^*)\]
that is inj? surj?

For reflexive $X$, we have
\[adj:D_{closable}(X)\Rightarrow D_{cl}(X^*).\]
For $f:X\to Y$, ``I'' define the predicate $f:A\Rightarrow B$ by
\[f(A)=B\quad\text{and}\quad A=f^{-1}(B).\]

\begin{thm}
The adjoint $B_{cl}(H)\to{\sim}B_{cl}(H)$ can be extended to $D_{cl}(H)\to{\sim}D_{cl}(H)$.
\end{thm}
\begin{thm}
For $T\in D_{cl}(H)$, $H=\ker T\dsum\cl{\im T^*}$.
\end{thm}
The space $D_{cl}$ is optimized when we think adjoints for reflexive spacse.

unitarily equivalence can defined for $T_1\in L(H_1)$ and $T_2\in L(H_2)$.


\subsection{Self-adjoint operators}
\begin{defn}
Let $T\in L(H)$ be satisfy $T\subset T^*$, i.e. $\inn{Tx,y}=\inn{x,Ty}$ for all $x,y\in D(T)$.
Then, we have definitions by the following diagram:
\begin{rd}
&& Bounded self-adjoint \ar{d}\\
Hermitian \rds{r}{densely defined}\lds{rru}{everywhere defined} & Symmetric \ar{l}\rds{r}{$D(T)=D(T^*)$} & Self-adjoint \ar{l}
\end{rd}
\end{defn}
\begin{prop}
Hermitian iff the numerical range is in $\R$.
\end{prop}
\begin{prop}
A symmetric operator is closable.
\end{prop}
\begin{pf}
Since $T$ is dense and $T\subset T^*$, $T^*$ is dense.
Therefore, $T$ is closable.
\end{pf}


\section{Compact operators}

\section{Nuclear operators}





















\chapter{Operator algebra}
We are concerned with algebras, which get action by a scalar field.
In this chapter, the scalar field is always assumed to be $\C$ unless any mention.

% 1
\section{Banach algebras and $C^*$-algebras}

% 1-1
\subsection{Banach algebras}

% 1-1-1
\subsubsection{Normed algebras}
In this book, we adopt the following definition:
\begin{defn}
A \emph{algebra} is a vector space $\cA$ over $\F=\R\text{ or }\C$ with a ring structure comparable with scalars.
\end{defn}
In this book, we always assume for the term \emph{algebra} that:
\begin{cond}
\item every algebra is associative,
\item every algebra is unital,
\item base field is either $\R$ or $\C$.
\end{cond}
Nonunital cases are dealt with specially, so we basically assume every algebra we concern contains unity.




\begin{defn}
A \emph{normed algebra} is an algebra $\cA$ with a norm $\|\cdot\|$ such that $\|xy\|\le\|x\|\|y\|$ for all $x,y\in\cA$ and $\|e\|=1$.
\end{defn}


For $a$ in a normed algebra, let us define:
\[\nu(x):=\limsup_{n\to\infty}\|x^n\|^{\frac1n}.\]
For this function, we have basic properties:
\begin{prop}
Let $\cA$ be a normed algebra.
We have
\begin{cond}
\item $\nu(x)=\inf_n\|x^n\|^{\frac1n}=\lim_{n\to\infty}\|x^n\|^{\frac1n}$,
\item $\nu(\lambda x)=|\lambda|\nu(x)$,
\item $\nu(xy)=\nu(yx)$,
\item $\nu(x^n)=\nu(x)^n$.
\end{cond}
\end{prop}
\begin{pf}
(1)
Let $\lambda=\inf_n\|x^n\|^{\frac1n}$.
Since $\lambda\le\|x^n\|^{\frac1n}$ for all $n$, we have $\lambda\le\nu(x)$.

Conversely, take $\e>0$.
There is an integer $m$ such that $\|x^m\|^{\frac1m}<\lambda+\e$.
For each $n$, divide $n$ by $m$ to write $n=qm+r$.
Limiting the inequality
\[\|x^n\|^{\frac1n}\le\|x^m\|^{\frac qn}\|x\|^{\frac rn}<(\lambda+\e)^{\frac{qm}n}\|x\|^{\frac rn},\]
we get $\nu(x)\le\lambda+\e$, hence $\nu(x)\le\lambda$.

The limit is obvious from
\[\nu(x)=\inf_n\|x^n\|^{\frac1n}\le\liminf_{n\to\infty}\|x^n\|^{\frac1n}\le\limsup_{n\to\infty}\|x^n\|^{\frac1n}=\nu(x).\]

(3)
Let $\e>0$.
For $n$ such that $\|x\|^{\frac1n}<(1+\e)$ and $\|x^{-1}\|^{\frac1n}<(1+\e)$, we have
\[\|(xy)^n\|^{\frac1n}\le\|x\|^{\frac1n}\|(yx)^n\|^{\frac1n}\|x^{-1}\|^{\frac1n}<(1+\e)^2\|(yx)^n\|^{\frac1n}.\]
Limiting $n\to\infty$,
\[\nu(xy)\le(1+\e)^2\nu(yx),\]
which implies $\nu(xy)\le\nu(yx)$.
The other direction is same.
\end{pf}
\begin{thm}
Let $\cA$ be a normed algebra.
Then, $\nu(x)=\|x\|$ for all $x\in\cA$ if and only if $\|x^2\|=\|x\|^2$ for all $x\in\cA$.
\end{thm}
\begin{pf}
($\Rightarrow$)
See (4) of the previous proposition.

($\Leftarrow$)
By (1) of the previous proposition, and since subsequences converge to the original limit point,
\[\nu(x)=\lim_{n\to\infty}\|x^n\|^{\frac1n}=\lim_{n\to\infty}\|x^{2^n}\|^{\frac1{2^n}}=\|x\|.\]
\end{pf}
\begin{thm}
Let $\cA$ be a normed algebra and let $x,y\in\cA$.
If $xy=yx$, then $\nu(xy)\le \nu(x)\nu(y)$ and $\nu(x+y)\le\nu(x)+\nu(y)$.
\end{thm}

We introduce Banach algerbas.
\begin{defn}[Banach algebra]
A \emph{Banach algebra} is a normed algebra that is complete with its norm.
\end{defn}
A main difference between Banach algebras and plain normed algebras is that we can construct an absolutely convergent series to show the existence of a specific element.
For examples, we have the following propositions:
\begin{prop}[Power series]
Let $\cA$ be a Banach algebra.
\begin{cond}
\item If $x\in\cA$ satisfies $\|x\|<1$, then $e-x$ is invertible.
\item The set of invertible elements $\cA^\times$ is open.
\item The inverse $\cA^\times\to\cA^\times$ is a homeomorphism.
\end{cond}
\end{prop}


% 1-1-2
\subsubsection{Complex Banach algebras}

One can use abolute convergent series in Banach algebras.
In particular, when the base field of the Banach algebra is $\C$, the theory of power series plays a powerful role.
This property comes out when we introduce holomorphic functional calculus.

\begin{defn}
Let $\cA$ be an algebra.
The \emph{spectrum} of $x\in\cA$ is defined as the set:
\[\sigma_\cA(x):=\{\,\lambda\in\C:\lambda e-x\text{ is not invertible.}\,\}.\]
Its complement is called \emph{resolvent set} and denoted as $\rho_\cA(x)\subset\C$.
The \emph{resolvent function} $R_x:\rho_\cA(x)\subset\C\to\cA^\times$ is defined by $R_x(\lambda)=(\lambda e-x)^{-1}$
\end{defn}
\begin{thm}
Let $\cA$ be a Banach algebra.
For every $a\in\cA$, the spectrum $\sigma(a)$ is compact.
\end{thm}
Furthermore, if the scalar field is the complex field $\C$, then the spectrum is always nonempty.
\begin{thm}
Let $\cA$ be a complex Banach algebra.
For every $a\in\cA$, the spectrum $\sigma(a)$ is nonempty.
\end{thm}

This is the Gelfand-Mazur theorem.
\begin{thm}[The Gelfand-Mazur theorem]
Every complex Banach division algebra is isomorphic to $\C$.
\end{thm}
\begin{pf}
Suppose $\cA$ is a unital Banach algebra in which every nonzero element is invertible.
For $a\in\cA$, the spectrum has an element $\lambda\in\sigma(a)$.
The non-invertibility of $a-\lambda e$ implies $a-\lambda e=0$, that is, $\cA\subset\C e\cong\C$.
Hence $\cA\cong\C$.
\end{pf}
\begin{thm}[The Gelfand-Mazur theorem]
Every real Banach division algebra is isomorphic to either $\R$, $\C$, or $\H$.
\end{thm}





\subsection{$C^*$-algebras}

\begin{thm}
Every nonunital $C^*$-algebra is a $C^*$-subalgebra of a unital $C^*$-algebra.
In particular, it is a maximal ideal of a codimension 1.
\end{thm}
\begin{pf}
Let $\cA$ be a nonunital $C^*$-algebra.
Since $\cA$ is a Banach space, the space of bounded operators $B(\cA)$ is a Banach algebra.
Define a normed $*$-algebra $\tilde\cA$ as the subalgebra:
\[\tilde\cA:=\{\,L_{x+\lambda e}\in B(\cA):x\in\cA,\,\lambda\in\C\,\}.\]
Since $\tilde\cA\cong \cA\oplus\C$, let us write $L_{x+\lambda e}$ as $(x,\lambda)$.

\Step{1}[$\cA$ is a normed $*$-subalgebra of $\tilde\cA$]
The $C^*$-algebra $\cA$ is recognized to be a $*$-subalgebra of $\tilde\cA$ with an injection $\cA\to\tilde\cA:x\mapsto L_x$.
Note that the norm in $\tilde\cA$ is induced from $B(\cA)$ as
\[\|(x,\lambda)\|=\sup_{y\in\cA}\frac{\|xy+\lambda y\|}{\|y\|}.\]
For $\lambda=0$, putting $y=x^*/\|x\|$, we get
\[\|(x,0)\|=\sup_{y\in\cA}\frac{\|xy\|}{\|y\|}=\|x\|\]
by the $C^*$-identity, hence the norm of $\cA$ agrees the norm of $\tilde\cA$.

\Step{2}[$\tilde\cA$ is Banach]
Suppose $(x_n,\lambda_n)$ is Cauchy in $\tilde\cA$.
Since $\cA$ is complete so that it is closed in $\tilde\cA$, we can induce a norm on the quotient $\tilde\cA/\cA$ so that the canonical projection is (Cauchy) continuous.
A finite dimensional vector space over $\C$ is alwyas Banach, so the Cauchy sequence $\lambda_n$ converges to a complex number $\lambda$.
By the inequality
\[\|x\|\le\|(x,\lambda)\|+|\lambda|,\]
the sequence $x_n$ is also Cauchy in $\cA$, so $x_n$ converges to $x\in\cA$.
Also, the inequality $\|(x,\lambda)\|\le\|x\|+|\lambda|$ implies that $(x_n,\lambda_n)$ converges to $(x,\lambda)$

\Step{3}[$\tilde\cA$ is $C^*$]
\end{pf}




\subsection{Gelfand theory}
This subsection is about theory of commutative Banach algebras.
Since a Banach algebra or a $C^*$-algebra generated by one element is always commutative, commutative theory plays a powerful role when we are interested in a specific element.
Applications are found in functional calculi.

The Gelfand-Mazur theorem says that every Banach field is $\C$, and this implies:
\begin{thm}
Let $\cA$ be a commutative unital Banach algebra.
There is one to one correspondence between maximal ideals and characters.
\end{thm}
\begin{pf}
A character $\cA\to\C$ defines a maximal ideal by its kernel.
The main interest is in the converse.

For a maximal ideal $\fm\subset\cA$, we have a Banach field $\cA/\fm$, which is isomorphic to $\C$ by the Gelfand-Mazur theorem.
The projection gives a character, which has $\fm$ as its kernel.
\end{pf}

\begin{thm}
Let $\cA$ be a commutative unital Banach algebra.
TFAE:
\begin{cond}
\item $\lambda=\phi(a)$ for some $\phi\in\sigma(\cA)$,
\item $\lambda\in\sigma(a)$.
\end{cond}
\end{thm}
\begin{pf}
(1)$\Rightarrow$(2).
If $a-\lambda$ has an inverse $b$, then we should have
\[1=\phi(1)=\phi(a-\lambda)\phi(b)=(\phi(a)-\lambda)\phi(b)\]
for all $\phi\in\sigma(\cA)$.
It implies $\phi(a)\ne\lambda$.

(2)$\Rightarrow$(1).
Suppose $a-\lambda$ is not invertible.
In the language of commutative ring theory, $a-\lambda$ is a non-unit, and it is contained in a maximal ideal by Zorn's lemma.
As we have seen in the above theorem, a maximal ideal is identified with a character that has itself as the kernel.
Take this character and get the desired result.
\end{pf}

\begin{cor}
Let $\cA$ be a commutative unital Banach algebra.
TFAE:
\begin{cond}
\item there is $\phi\in\sigma(\cA)$ such that $\phi(a)=0$,
\item $a$ is not invertible in $\cA$.
\end{cond}
\end{cor}
\begin{cor}
Let $\cA$ be a unital Banach algebra.
All elements in the open ball $B(e,1)$ in $\cA$ are invertiable.
\end{cor}
\begin{ex}
For $\cA=C_b(X)$, given the locally compact Hausdorff $X$, the ball is $B(e,1)=\{f\in C(X):0<|f(x)|<2\quad\text{for all }x\}$.
Every function in this set is invertible.
\end{ex}


\section{Functional calculus}

Holomorphic functional calculus can be done on Banach algebras, while continuous functional calculus should be on $C^*$-algebras.

\subsection{Holomorphic functional calculus}
Let $a$ be a nonzero element in a unital Banach algebra $\cA$.
We can define a commutative unital Banach algebra
\[\cl{\{p(a):p\in\C[x]\}}.\]
We say that it is generated by $a$.

\begin{thm}[Holomorphic functional calculus]
Let $\cA$ be a unital Banach algebra generated by a nonzero element $a$.
Let $f$ be a holomorphic function on $\sigma(a)$.
Then, there is an element $f(a)\in\cA$ such that
\[\phi(f(a))=f(\phi(a))\]
for all characters $\phi$ on $\cA$.
\end{thm}
\begin{pf}
It is realized by
\[f(a):=\frac1{2\pi}\int_Cf(\lambda)(\lambda-a)^{-1}\,d\lambda.\]
\end{pf}

\begin{ex}[Failure of continuous functional calculus]

\end{ex}


\subsection{Continuous functional calculus}
Let $a$ be a nonzero element in a unital $C^*$-algebra $\cA$.
If $a$ is normal, then a commutative unital $C^*$-algebra, which is more precisely given by
\[C^*(a):=\cl{\{p(a,a^*):p\in\C[x,y]\}},\]
is defined.
We say that it is generated by $a$.

\begin{thm}[Continuous functional calculus]
Let $\cA$ be a unital $C^*$-algebra generated by a nonzero element $a$.
Let $f$ be a continuous function on $\sigma(a)$.
Then, there is an element $f(a)\in\cA$ such that
\[\phi(f(a))=f(\phi(a))\]
for all characters $\phi$ on $\cA$.
\end{thm}

Functional calculus is interested in the possibility of representation of elements of $C^*(a)$ as a ``function of'' $a$.
For example, we want to make sure that we can define square root or exponential function on $C^*(a)$.




By the Gelfand-Naimark theorem, we have an algebra isomorphism
\[C^*(a)\cong C(\sigma(a)),\]
which is also called Gelfand representation.
It implies the above theorem can be conversed: every element in $C^*(a)$ is represented by a continuous function on $\sigma(a)$.




\subsection{Adjoint and spectra}
\begin{thm}
Let $\cA$ is a $C^*$-algebra.
TFAE:
\begin{cond}
\item $aa^*=1$ (unitary)
\item $\sigma(a)\subset\T$
\end{cond}
\end{thm}
\begin{pf}
(1)$\Rightarrow$(2).
From the spectral radius formula $\|a\|=r(a)$, $\|a\|=1$ implies $|\lambda|\ge1$ for all $\lambda\in\sigma(a)$.
Since $a^{-1}$ is also unitary, we have $|\lambda|=1$.
Here, the spectral mapping theorem is used.

(2)$\Rightarrow$(1).
\end{pf}

\begin{thm}
Let $\cA$ is a $C^*$-algebra.
TFAE:
\begin{cond}
\item $a=a^*$ (self-adjoint)
\item $\sigma(a)\subset\R$
\end{cond}
\end{thm}
\begin{pf}
WLOG, suppose $\cA$ is generated by $a$.

(1)$\Rightarrow$(2).
By holomorphic functional calculus We can define $e^{ia}\in\cA$.
It is unitary by the spectral mapping theorem.
We get the desired result.

(2)$\Rightarrow$(1).
Let $\phi$ be a pure state.
Since $i(a-a^*)$ is self-adjoint, $\phi(i(a-a^*))\subset\R$, so $\phi(a-a^*)$ only contains purely imaginary numbers.
By the condition, $\sigma(a)=\cl{\sigma(a^*)}\subset\R$ gives $\phi(a)=\phi(a^*)$.
By the Stone-Weierstrass theorem, we get $a=a^*$.
\end{pf}

For positiveness, we use the Stone-Weierstrass theorem to construct a square root.


\section{The Gelfand-Naimark theorems}
\subsubsection{Simple histories and statements}
This note was organized in order to read the book``characterizations of $C^*$-algebras'' by Robert S. Doran and Victor A. Belfi.

The first abstract treatment of normed linear space was given in Banach's 1920 thesis.
$C^*$-algebras made their first appearance in 1943 in the now famous paper of Gelfad and Naimark.
The present term ``Banach algebra'' was used for the first time in 1945 by W. Ambrose.

One of early results in Banach algebra theory was generalize the classical theorem of Frobenius to the Gelfand-Mazur theorem.
\begin{thm}[Frobenius]
Every finite dimensional division algebra over $\C$ is isomorphic to $\C$.
\end{thm}
\begin{thm}[Gelfand-Mazur, 1938]
Every normed division algebra over $\C$ is isomorphic to $\C$.
\end{thm}
\begin{thm}[Mazur, 1938]
Every normed division algebra over $\R$ is isomorphic to $\R$, $\C$, or $\H$.
\end{thm}

Many important Banach algebras carry a natural involution.

In 1943, Gelfand and Naimark proved that an involutive unital Banach algebra satisfying the following three conditions
\begin{cond}
\item $\|x^*x\|=\|x^*\|\|x\|$,\hfill($C^*$-identity)
\item $\|x^*\|=\|x\|$,
\item $e+x^*x$ is invertible
\end{cond}
is same with(isometrically isomorphic to) $C^*$-algebra.
They immediately asked in a footnote if the second and third conditions are able to be deleted, which indeed turned out to be true later; the proof is very hard.
For simplicity, we often introduce the identity $\|x^*x\|=\|x\|^2$ to define $C^*$-algebras. (Since it deduces the other two conditions quite clearly.)
However, we will choose the $C^*$-condition $\|x^*x\|=\|x^*\|\|x\|$ for the definition of $C^*$-algebras, and prove the other conditions in the next section.
\begin{defn}
A \emph{$C^*$-algebra} is a involutive Banach algerba satisfying the $C^*$-identity.
\end{defn}


The followings are the statements of the Gelfand-Naimark theorem.
\begin{thm}[Gelfand-Naimark I]
A commutative $C^*$-algebra is isometrically $^*$-isomorphic to $C_0(X)$ for a locally compact Hausdorff space $X$.
\end{thm}
\begin{thm}[Gelfand-Naimark II]
A $C^*$-algebra is isometrically $^*$-isomorphic to a closed $^*$-subalgebra of $B(H)$ for a Hilbert space $H$.
\end{thm}

\begin{thm}
If $x,y$ commutes, then $\sigma(xy)\subset\sigma(x)\sigma(y)$.
\end{thm}
\subsection{Commutative Banach algebras}

The Gelfand representation $A\to C_0(\hat A)$ can be defined for commutative Banach algebras.

\begin{defn}
A \emph{symmetric} Banach algebra is an involutive Banach algebra for which the Gelfand representation preserves the involution.
We will not consider non-symmetric involutive Banach algebras in this section.
\end{defn}
Notice the following implication:
\begin{cd}
& \text{symmetirc Banach algebra} \ar[dr] &\\
\text{$C^*$-algebra} \ar[ur]\ar[dr]&& \text{Banach algebra}.\\
& \text{semisimple Banach algebra} \ar[ur] &
\end{cd}

Let $A$ be a commutative Banach algebra.
\begin{thm}
If $A$ is semisimple, then the Gelfand representation is a monomorphism; it is injective.
\end{thm}
\begin{pf}
It is because the kernel is given by the Jacobson radical.
\end{pf}
\begin{thm}
If $A$ is symmetric, then the Gelfand representation is an epimorphism; it has a dense range.
\end{thm}
\begin{pf}
The image is closed under all operations except involution, separates points, and vanishes nowhere.
If $A$ is symmetric, then the image is closed under involution.
Thus, by the Stone-Weierstrass theorem, we get the result.
\end{pf}

$C^*$-algebras are semisimple and symmetric (even if it is noncommutative).
\begin{thm}
A $C^*$-algebra is semisimple.
\end{thm}
\begin{thm}
A $C^*$-algebra is symmetric.
\end{thm}
\begin{pf}[1]
It is by Arens.
\end{pf}
\begin{pf}[2]
It is by Fukamiya.
\end{pf}
Furethermore,
\begin{thm}
If $A$ is a commutative $C^*$-algebra, then the Gelfand representation is isometric.
\end{thm}
Since an isometry is injective and has a closed range, therefore, it should be isometric $^*$-isomorphism.






\end{document}














