\documentclass{../note}
\usepackage{../../ikany}

\def\a{\alpha}
\def\I{\mathrm{I}}
\def\II{\mathrm{II}}
\def\ssum{{\textstyle\sum\,}}


\begin{document}
\title{Smooth Surfaces}
\author{Ikhan Choi}
\maketitle
\tableofcontents


\section*{Acknowledgement}
This note is written for teaching during the undergraduate tutoring program in 2019 fall semester.
Main resources I refered are books by Manfredo P. do Carmo \cite{}, and Richard S. Millman and George D. Parker \cite{}.






% In this note, we only consider smooth maps.





\chapter{Smooth manifolds}
\section{Parametrizations and Coordinates}

For each text on classical differential geometry, the definitions frequently vary.
In this note, we define as follows.
\begin{defn}
An \emph{$m$-dimensional parametrization} is a smooth map $\a:U\to\R^n$ such that
\begin{parts}
\item $U\subset\R^m$ is open,
\item the Fr\'echet derivative $d\a:U\times\R^m\to\R^n$ is injective everywhere,\hfill(immersion)
\item $\a$ is one-to-one,\hfill(immersed submanifold)
\item $\a$ is a homeomorphism onto its image.\hfill(embedded submanifold)
\end{parts}
The Euclidean space $\R^n$ is called the \emph{ambient space}.
\end{defn}

The first condition is necessary to avoid differentiating at points that are not in the interior of domain.
Of course, it is possible to generalize the definition of differentiation on boundary points, but we will not introduce the notion because it goes out of the scope of this note.

The second condition is the most important one.
This condition is paraphrased as follows: the set of column vectors of $d\a|_x:\R^m\to\R^n$, which are exactly the partial derivatives $\{\pd_i\a(x)\}_{i=1}^m\subset\R^n$ of $\a$ with respect to each direction, is linearly independent at every point $x\in U$.
Differential geoemtry do not consider parametrizations that fail this.
This condition is vital for considering an appropriate and well-defined linear approximation of curves or surfaces.
If it is not staisfied, every definition including tangent spaces in differential geometry can suffer.

Above the second condition, we call the image of a parametrization $\a$ a (immersed) submanifold if the third condition is satisfied.
If $\a$ is not one-to-one, then two distinct ordered tuples of real numbers -- we call the tuples coordinates -- may represent the same point.
Namely, this condition allows to use the inverse map $\a^{-1}:\im\a\to U$.
By this, we can recognize a parametrization as the inverse of a coordinate map.
To describe a geometric object that cannot be covered by a single injective parametrization, such as a circle or a sphere, we can admit several parametrizations: see (3) in Example 1.1.

The forth condition is stronger than the third, so we may assume just (1), (2), and (4) in Definition 1.1.
It is sometimes called a proper patch in some references.
This condition is introduced to exclude some exceptional examples such as lemniscates: see (4) Example 1.1.

\begin{defn}
A subset $M\subset\R^n$ is called a \emph{regular curve} (resp. \emph{regular surface}) if there exists a one-dimensional (resp. two-dimensional) parametrization whose image is exactly $M$.
\end{defn}

All curves and surfaces in this note are assumed to be regular: all of the four conditions are satisfied.
One can notice that this definition is exactly same as an embedded submanifold of $\R^n$ that can be covered by a single parametrization.

We also just often say that $\a$ is a regular curve (resp. regular surface) for \emph{a particular parametrization $\a$}.
However, note that a curve or surface admits infinitely many parametrizations.
We can solve many geometry or physics problems very easily by choosing an appropriate parametrization.
Related to the choice of parametrizations, the following issues are always importantly considered when developing a theory of differential geometry:
\begin{itemize}
\item Well-definedness of a structure with respect to the dependency on parametrizations(coordinates).
\item Existence of a parametrization(coordinates) that has nice properties we want.
\end{itemize}

\begin{defn}
Let $M$ be the image of a parametrization $\a:U\subset\R^m\to\R^n$.
The inverse $\f:M\to U$ of a parametrization is called a \emph{coordinate map}.
\end{defn}

Coordinates and parametrizations have perfectly equivalent information except that they are inverses of each other (only if parametrization satisfies the injectivity).
A parametrization tends to be taken in order for explicit computations in the ambient space like $\R^3$ while coordinates are more usefully taken in verifications of abstract propositions.
We use the term \emph{reparametrization} to refer to nothing but a choice of another parametrization for the same curve or surface.
As said, the choice of coordinate(parametrization) is important in differential geometry.

\begin{ex}
\begin{parts}
\item
Let $\a:\R\to\R^3$ be a map given by
\[\a(t)=(\cos t,\sin t, t).\]
Since $d\a|_t(1)=\a'(t)=(-\sin t,\cos t,1)$ is always nonzero so that $d\a$ is injective everywhere, $\a$ is a parametrization of the regular curve
\[\{\,(x,y,z)\in\R^3:x=\cos z,\ y=\sin z\,\}.\]
Notice that it is enough to check $\a'(t)\ne0$ for a curve parametrization $\a$ to show the injectivity of $d\a$.
This curve is an example of circular helices.

\item
Let $\a:\R\to\R^3$ be a map given by
\[\a(t)=(t^3,t^6,t^9).\]
Since $d\a|_t(1)=\a'(t)=(3t^2,6t^5,9t^8)$ is zero when $t=0$, \emph{it would be better to avoid calling $\a$ a parametrization}.
Instead, the restrictions $\a_+:(0,\infty)\to\R^3$ and $\a_-:(-\infty,0)\to\R^3$ satisfy the axioms of parametrization at the beginning.

However, by reparametrization, we can show the image of $\a$ is a regular curve, that is, we can find a parametrization that shares the image with $\a$, even though \emph{we sometimes say that $\a$ is not a regular curve} according to the fact $\a'$ can vanish.
Consider $\beta:\R\to\R^3$ defined by
\[\beta(t)=(t,t^2,t^3).\]
This map has the same image $\im\a=\im\beta$, but $\beta'(t)=(1,2t,3t^2)\ne0$ for all $t\in\R$.

\item
Let $S^1$ be the unit circle in $\R^2$, precisely
\[S^1:=\{\,(x,y)\in\R^2:x^2+y^2=1\,\}.\]
It cannot be covered by a single parametrization, so we can consider two different parametrizations $\a:(0,2\pi)\to\R^2$ and $\beta:(\pi,3\pi)\to\R^2$ for $S^1$:
\[\alpha(t)=(\cos t,\sin t),\qquad\beta(t)=(\cos t,\sin t).\]
Then, we have $S^1=\im\a\cup\im\beta$.
If we want to investigate the geometry of $S^1$ near the point $(1,0)$, we can choose $\beta$ rather than $\a$ because $(1,0)\notin\im\a$.

\item
Let $\a,\beta:(0,2\pi)\to\R^2$ be maps given by
\[\a(t)=(\sin t,\sin2t),\qquad\beta(t)=(\sin t,-\sin2t).\]
They are one-to-one smooth maps such that $d\a|_t(1)\ne0\ne d\beta|_t(1)$, and one can check that they have common images; it is shaped like the character `$\infty$'.
A problem occurs when we think tangent vectors at $(0,0)$:
\[\a'(0)=(1,2),\qquad\beta'(0)=(1,-2)\]
imply that the notion of tangent vectors at a point do depend on the choice of parametrizations.

Generally, since one philosophy of parametrizations is to view them as \emph{identifications} between curved spaces and flat Euclidean spaces, we want for them to have the images of open sets be open with respect to subspace topology.
Thus, we assume parametrizations are homeomorphisms onto its images.

\item
Let $f:\R^2\to\R$ be any smooth function and $\a:\R^2\to\R^3$ be a map given by
\[\a(x,y)=(x,y,f(x,y)).\]
Then, $\a$ is a two-dimensional parametrization because
\begin{align*}
d\a|_{(x,y)}(1,0)&=\pd{\a}{x}(x,y)=\Bigl(1,0,\pd{f}{x}(x,y)\Bigr),\\
d\a|_{(x,y)}(0,1)&=\pd{\a}{y}(x,y)=\Bigl(0,1,\pd{f}{y}(x,y)\Bigr)
\end{align*}
are linearly independent for every $(x,y)\in\R^2$.
A parametrization of this form is called a \emph{Monge patch}.
Notice that it is enough to check that the two partial derivatives $\pd_x\a$ and $\pd_y\a$ are linearly independent for a surface parametrization $\a$.

Let $S=\im\a$ be the regular surface determined by $\a$, and let $p$ be a point on the surface $S$ so that we have $p=(x,y,f(x,y))$.
Associated with $\a$, a coordinate map $\f:S\to\R^2$ is defined as the inverse of $\a$:
\[\f(p):=\a^{-1}(p)=(x,y).\]
This map $\f$ consists of two real-valued functions on $S$,
\[x:S\to\R:p\mapsto x,\qquad y:S\to\R:p\mapsto y.\]
In this regard, we often write the coordinates $\f$ as $(x,y)$.

\item
Let
\[S=\{\,(x,y)\in\R^2:x>0\text{ or }y\ne0\,\}.\]
The set $S$ is a regular surface.
Consider two different coordinates
\begin{gather*}
(x,y):S\to S:(x,y)\mapsto(x,y),\\
(r,\theta):S\to\R_{>0}\times(-\pi,\pi):(x,y)\mapsto\left(\sqrt{x^2+y^2},\ 2\tan^{-1}\frac y{\sqrt{x^2+y^2}+x}\right),
\end{gather*}
where $\tan^{-1}(t):=\int_0^t\frac{ds}{1+s^2}$.
They are the inverses of parametrizations $\a:S\to\R^2$ and $\beta:(0,\infty)\times(-\pi,\pi)\to\R^2$ defined by
\[\a(x,y)=(x,y),\qquad\beta(r,\theta)=(r\cos\theta,r\sin\theta).\]
The coordinate maps $(x,y)$ and $(r,\theta)$ are called \emph{Cartesian coordinates} and \emph{polar coordinates} respectively.
\end{parts}
\end{ex}











\section{Differentiation}
Differentiation in differential geometry can be understood in many different viewpoints.
We, here, review the two kinds of main usages of differentiation: differentiation of parametrizations, and differentiation by directional vectors.
Do not forget that all differentiations in this note will be done thanks to the structure of the ambient space $\R^n$.

\subsection{Differentiation of parametrizations}
We introduce the notion of tangent spaces, geometrically the spaces of vectors that starts from each base point, by differentiation of parametrization.
In this note we define tangent spaces by the image of the Fr\'echet derivative $d\a$ of a parametrization $\a$.

\begin{defn}
Let $M$ be the image of a parametrization $\a:U\subset\R^m\to\R^n$.
Let $p\in M$ be a point and $x=\a^{-1}(p)\in U$ be its coordinates.
The \emph{tangent space} of $M$ at $p$, denoted by $T_pM$, is the image of the Fr\'echet derivative $d\a|_x:\R^m\to\R^n$.
\end{defn}

Since $d\a_x$ is an injective linear transformation at every $x$, the tangent space $T_pM$ is an $m$-dimensional linear subspace of $\R^n$.
For a parametrization $\a$, the tangent space $T_pM$ has a basis $\{\a_i|_p\}_{i=1}^m$, which is frequently chosen as the standard.
Now, after a definition, we must show its consistency.

\begin{prop}
Let $M$ be the image of a parametrization $\a:U\subset\R^m\to\R^n$.
Let $p\in M$ and $v\in\R^n$.
Then, the two are equivalent:
\begin{parts}
\item $v\in T_pM$;
\item there is a regular curve $\gamma:I\to M$ such that $\gamma(0)=p$ and $\gamma'(0)=v.$
\end{parts}
In particular, $T_pM$ is independent on the parametrization $\a$.
\end{prop}
\begin{pf}
%%%
\end{pf}

\begin{rmk}
We can easily check that $T_p\R^n=\R^n$ for any $p\in\R^n$.
The notation $T_p\R^n$ will be used to emphasize that a vector in $\R^n$ is geometrically recognized to cast from the point $p$.
Since $T_p\R^n=\R^n=T_q\R^n$ for every pair of points $p,q\in\R^n$, summation and inner product of a vector in $T_p\R^n$ and a vector in $T_q\R^n$ make sense.
This identification of tangent spaces are allowed \emph{only for the case of linear spaces} such as $\R^n$.
(In fact, the identification $T_p\R^n=\R^n$ is \emph{natural} in categorical language.)
\end{rmk}
\begin{rmk}
One way to view tangent spaces is to see them as domains and codomains of Fr\'echet derivatives.
For open sets $U\subset\R^m$ and $V\subset\R^n$, the Fr\'echet derivative of a smooth map $F:U\to V$ at $x\in U$ is a linear transformation $dF|_x:T_xU\to T_{F(x)}V$.
Since $T_xU=\R^m$ and $T_{F(x)}V=\R^n$, the original definition on Euclidean spaces agrees with it.
In this reason, the Fr\'echet derivative $dF$ is also called a \emph{tangent map}, \emph{pushforward}, or \emph{differential} in differential geoemtry.
\end{rmk}


\begin{notn*}
Let $\a$ be a parametrization for a regular curve or surface $M$.
For derivatives of $\a$, we will use the following notations:
\[\pd_t\a=\a',\quad\pd_x\a=\a_x,\quad\pd_i\a=\a_i.\]
The set $\{\a_i\}_i$ will be used to denote a basis of tangent space $T_pM$.
\end{notn*}

% Examples




\subsection{Differentiation by tangent vectors}

% explanation

\begin{defn}
Let $\a:U\subset\R^m\to\R^n$ be a parametrization with $M=\im\a$.
\begin{parts}
\item A \emph{scalar field}, \emph{smooth function}, or just a \emph{function} is a function $f:M\to\R$ such that $f\circ\a:U\to\R$ is smooth.
\item A \emph{vector field} is a map $X:M\to\R^n$ such that $X\circ\a:U\to\R^n$ is smooth.
\item A \emph{tangent vector field} is a vector field $X:M\to\R^n$ such that $X|_p\in T_pM$.
\end{parts}
The set of tangent vector fields is often denoted by $\fX(M)$.
\end{defn}
\begin{rmk}
In general, the word \emph{vector fields} are basically assumed to be tangent.
However, we will distinguish them in this note.
\end{rmk}

The following proposition proves that the smoothness of functions and vector fields does not depend on parametrizations.

\begin{prop}
Let $\a:U\subset\R^m\to\R^n$ and $\beta:V\subset\R^m\to\R^n$ be parametrizations with same image $M=\im\a=\im\beta$.
Then, the map $\beta^{-1}\circ\a:U\to V$ is smooth.
\end{prop}
\begin{pf}
%%%
\end{pf}
\begin{rmk}
The map $\beta^{-1}\circ\a$ is called the \emph{transition map}.
\end{rmk}

\begin{defn}
Let $\a:U\subset\R^m\to\R^n$ be a parametrization $M=\im\a$.
\begin{parts}
\item The coordinate representation of a function $f:M\to\R$ is
\[f\circ\a:U\to\R.\]
\item The (external) coordinate representation of a vector field $X:M\to\R^n$ is
\[X\circ\a:U\to\R^n.\]
\item The coordinate representation of a tangent vector field $X:M\to\R^n$ is
\[(X^1\circ\a,\,\cdots,\,X^m\circ\a):U\to\R^m\]
where $X=\sum_iX^i\a_i$.
\end{parts}
\end{defn}



\begin{defn}
Let $M$ be the image of a parametrization $\a:U\subset\R^m\to\R^n$.
Let $v=\sum_iv^i\a_i|_p\in T_pM$ be a tangent vector at $p=\a(x)$.
For a function $f:M\to\R$, its partial derivative is defined by
\[\pd_vf(p):=\sum_{i=1}^mv^i\pd_i(f\circ\a)(x)\in\R.\]
For a vector field $X:M\to\R^n$, its partial derivative is defined by
\[\pd_vX|_p:=\sum_{i=1}^mv^i\pd_i(X\circ\a)(x)\in\R^n.\]
This definition is not dependent on parametrization $\a$.
\end{defn}

\begin{prop}
Let $M$ be the image of a parametrization.
Let $X$ be a tangent vector field on $M$.
\begin{parts}
\item If $f$ is a function, then so is $\pd_Xf$.
\item If $Y$ is a vector field, then so is $\pd_XY$.
\item If $Y$ is a tangent vector field, then so is $\pd_XY-\pd_YX$.
\end{parts}
\end{prop}
\begin{pf}
(1) and (2) are clear.
For (3), if we let $X=\sum_iX^i\a_i$ and $Y=\sum_jY^j\a_j$ for a parametrization $\a:U\subset\R^m\to\R^n$, then
\begin{align*}
\pd_XY-\pd_YX
&=\pd_X(\ssum_jY^j\a_j)-\pd_Y(\ssum_iX^i\a_i)\\
&=\ssum_j[(\pd_XY^j)\a_j+Y^j\pd_X\a_j]-\ssum_i[(\pd_YX^i)\a_i+X^i\pd_Y\a_i]\\
&=\ssum_j[(\pd_XY^j)\a_j+Y^j\ssum_iX^i\pd_i\a_j]-\ssum_i[(\pd_YX^i)\a_i+X^i\ssum_jY^j\pd_i\a_j]\\
&=\ssum_j(\pd_XY^j)\a_j-\ssum_i(\pd_YX^i)\a_i\\
&=\ssum_i(\pd_XY^i-\pd_YX^i)\a_i.\qedhere
\end{align*}
\end{pf}

\begin{notn*}
Let $M$ be the image of a parametrization $\a$.
For derivatives of functions on $M$ by tangent vectors, we will use
\[\pd_{\a_i}f=\pd_if,\quad\pd_{\a_t}f=\pd_tf=f',\quad\pd_{\a_x}f=\pd_xf=f_x.\]
For derivatives of vector fields on $M$ by tangent vectors, we will use
\[\pd_{\a_i}X=\pd_iX,\quad\pd_{\a_t}X=\pd_tX=X',\quad\pd_{\a_x}X=\pd_xX=X_x.\]
We will \emph{not} use $f_i$ or $X_i$ for $\pd_if$ and $\pd_iX$ because it is confusig with coordinate representations, and \emph{not} use the nabula symbol $\nabla_v$ in this sense because it will be devoted to another kind of derivatives introduced in Section 4.
\end{notn*}

\begin{ex} % example 많이 추가하기
\begin{parts}
\item
Let $\a$ be an $m$-dimensional parametrization with $M=\im\a$.
The value of $\pd_i\a=\a_i:M\to\R^3$ is always a tanget vector at each point $p=\a(x)$, and $\a_i$ becomes a vector field.

Let $s$ be either a smooth function or vector field on $\a$.
Then, we can compute the directional derivative as
\[\pd_is:=\pd_i(s\circ\a)=\pd_t(s\circ\gamma)\]
by taking $\gamma(t)=\a(x+te_i)$, where $e_i$ is the $i$-th standard basis vector for $\R^m$.

\item
Let $\a:\R^2\to\R^3$ be a regular surface given by
\[\a(x,y)=\left(\frac{2x}{1+x^2+y^2},\,\frac{2y}{1+x^2+y^2},\,1-\frac2{1+x^2+y^2}\right).\]
This map gives a parametrization for the sphere $S^2$ without the north pole $(0,0,1)$, and is called the \emph{stereographic projection}.
Let $f:S^2\setminus\{(0,0,1)\}\to\R$ be the height function of $\a$ defined by
\[f(p):=z\]
for $p=(x,y,z)\in S^2\setminus\{(0,0,1)\}$.
Its coordinate representation is
\[f\circ\a(x,y)=1-\frac2{1+x^2+y^2}.\]
Then, the directional derivative is
\[\pd_xf=\pd{(f\circ\a)}{x}=\pd{x}\left(1-\frac2{1+x^2+y^2}\right)=\frac{4x}{(1+x^2+y^2)^2}.\]
Note that $\pd_xf\ne\pd_{(1,0,0)}z=0$.
\end{parts}
\end{ex}


\section{Linear algebra on tangent spaces}


% 내적 외적 삼중곱
% 벡터의 좌표표현 안쪽 기저와 바깥 기저

% 좌표는 접공간의 기저를 준다
% frame은 R^3 풀백번들에 정의하자



























\chapter{Local Theory of Curves and Surfaces}

\section{Curves}

\subsection{Parametrization}

By definition, a regular curve has at least one parametrization.
However, a given parametrization may not have useful properties, so we often take a new parametrization.
The existence of a parametrization with certain properties is one of the main problems in differential geometry.
Practically, the existence proof is usually done by constructing a \emph{diffeomorphism} between open sets in $\R^m$; a bijective smooth map whose inverse is also smooth.

We introduce the arc-length reparametrization.
It is the most general choice for the local study of curves.
\begin{defn}
A parametrization $\a$ of a regular curve is called a \emph{unit speed curve} or an \emph{arc-length parametrization} when it satisfies $\|\a'\|=1$.
\end{defn}
\begin{thm}
Every regular curve may be assumed to have unit speed.
Precisely, for every regular curve, there is a parametrization $\a$ such that $\|\a'\|=1$.
\end{thm}
\begin{pf}
By the definition of regular curves, we can take a parametrization $\beta:I_t\to\R^d$ for a given regular curve.
We will construct an arc-length parametrization from $\beta$.

Define $\tau:I_t\to I_s$ such that
\[\tau(t):=\int_0^t\|\beta'(s)\|\,ds.\]
Since $\tau$ is smooth and $\tau'>0$ everywhere so that $\tau$ is strictly increasing, the inverse $\tau^{-1}:I_s\to I_t$ is smooth by the inverse function theorem; $\tau$ is a diffeomorphism.
Define $\a:I_s\to\R^d$ by $\a:=\beta\circ\tau^{-1}$.
Then, by the chain rule,
\[\a'=\dd{\a}{s}=\dd{\beta}{t}\dd{\tau^{-1}}{s}=\beta'\left(\dd{\tau}{t}\right)^{-1}=\frac{\beta'}{\|\beta'\|}.\qedhere\]
\end{pf}




\subsection{Differentiation of Frenet-Serret frame}

The Frenet-Serret frame is a standard frame for a curve, and it is in particular effective when we assume the arc-length parametrization.
It is defined for nondegenerate regular curves, i.e. nowhere straight curves.
It provides with a useful orthonormal basis of $T_p\R^3\supset T_pC$ for points $p$ on a regular curve $C$.
\begin{defn}
We call a curve parametrized as $\a:I\to\R^3$ is \emph{nondegenerate} if the normalized tangent vector $\a'/\|\a'\|$ is never locally constant everywhere.
In other words, $\a$ is nowhere straight.
\end{defn}

\begin{defn}[Frenet-Serret frame]
Let $\a$ be a nondegenerate curve.
The \emph{tangent unit vector}, \emph{normal unit vector}, \emph{binormal unit vector} are $T_p\R^3$-valued vector fields on $\a$ defined by:
\[\rT(t):=\frac{\a'(t)}{\|\a'(t)\|},\qquad\rN(t):=\frac{\rT'(t)}{\|\rT'(t)\|},\qquad\rB(t):=\rT(t)\times\rN(t).\]
The set of vector fields $\{\rT,\rN,\rB\}$, which is called \emph{Frenet-Serret frame}, forms an orthonormal basis of $T_p\R^3$ at each point $p$ on $\a$.
The Frenet-Serret frame is uniquely determined up to sign as $\a$ changes.
\end{defn}

We study the derivatives of the Frenet-Serret frame and their coordinate representations.
In the coordinate representations on the Frenet-Serret frame, important geometric measurements such as curvatrue and torsion come out as coefficients.

\begin{defn}
Let $\a$ be a nondegenerate curve.
The \emph{curvature} and \emph{torsion} are scalar fields on $\a$ defined by:
\[\kappa(t):=\frac{\<\rT'(t),\rN(t)\>}{\|\a'\|},\quad\tau(t):=-\frac{\<\rB'(t),\rN(t)\>}{\|\a'\|}.\]
Note that $\kappa>0$ cannot vanish by definition of nondegenerate curve.
This definition is independent on $\a$.
\end{defn}

\begin{thm}[Frenet-Serret formula]
Let $\a$ be a nondegenerate curve.
Then,
\[\mat{\rT'\\\rN'\\\rB'}=\|\a'\|\mat{0&\kappa&0\\-\kappa&0&\tau\\0&-\tau&0}\mat{\rT\\\rN\\\rB}.\]
\end{thm}
\begin{pf}
Note that $\{\rT,\rN,\rB\}$ is an orthonormal basis.
We first show the first and third rows, and the second row later.

\Step{1}[Show that $\rT',\rB',\rN$ are parallel]
Two vectors $\rT'$ and $\rN$ are parallel by definition of $\rN$.
Since $\<\rT,\rB\>=0$ and $\<\rB,\rB\>=1$ are constant, we have
\[\<\rB',\rT\>=\<\rB,\rT\>'-\<\rB,\rT'\>=0,\qquad\<\rB',\rB\>=\tfrac12\<\rB,\rB\>'=0,\]
which show $\rB'$ and $\rN$ are parallel.
By the definition of $\kappa$ and $\tau$, we get
\[\rT'=\|\a'\|\kappa\rN,\qquad\rB'=-\|\a'\|\tau\rN.\]

\Step{2}[Describe $\rN'$]
Since
\begin{align*}
\<\rN',\rT\>&=-\<\rN,\rT'\>=-\|\a'\|\kappa,\\
\<\rN',\rN\>&=\tfrac12\<\rN,\rN\>'=0,\\
\<\rN',\rB\>&=-\<\rN,\rB'\>=\|\a'\|\tau,
\end{align*}
we have
\[\rN'=\|\a'\|(-\kappa\rT+\tau\rB).\qedhere\]
\end{pf}
\begin{rmk}
Let $\rX(t)$ be the curve of orthogonal matrices $(\rT(t),\rN(t),\rB(t))^T$.
Then, the Frenet-Serret formula reads
\[\rX'(t)=A(t)\rX(t)\]
for a matrix curve $A(t)$ that is completely determined by $\kappa(t)$ and $\tau(t)$, if we let us only consider arc-length parametrized curves.
This is a typical form of an ODE system, so we can apply the Picard-Lindel\"of theorem to get the following proposition: if we know $\kappa(t)$ and $\tau(t)$ for all time $t$, and if $\rT(0)$ and $\rN(0)$ are given so that an initial condition
\[\rX(0)=(\rT(0),\,\rN(0),\,\rT(0)\times\rN(0))\]
is established, then the solution $\rX(t)$ exists and uniquely determined in a short time range.
Furthermore, if $\a(0)$ is given in addition, the integration
\[\a(t)=\a(0)+\int_0^t\rT(s)\,ds\]
provides a complete formula for unit speed parametrization $\a$.
\end{rmk}
\begin{rmk}
Skew-symmetry in the Frenet-Serret formula is not by chance.
Let $\rX(t)=(\rT(t),\rN(t),\rB(t))^T$ and write $\rX'(t)=A(t)\rX(t)$ as we did in the above remark.
Since $\rX(t+h)=R_t(h)\rX(t)$ for a family of special orthogonal matrices $\{R_t(h)\}_h$ with $R_t(0)=I$, we can describe $A(t)$ as 
\[A(t)=\left.\dd{R_t}{h}\right\rvert_{h=0}.\]
By differentiating the relation $R_t^T(h)R_t(h)=I$ with respect to $h$, we get to know that $A(t)$ is skew-symmetric for all $t$.
In other words, the tangent space $T_I\SO(3)$ forms a skew symmetric matrix.
\end{rmk}













\subsection{Computational problems}

The following proposition gives the most effective and shortest way to compute the Frenet-Serret apparatus in general case.
If we try to reparametrize the given curve into a unit speed curve or find $\kappa$ by differentiating $\rT$, then we must encounter the normalizing term of the form $\sqrt{(-)^2+(-)^2+(-)^2}^{-1}$, and it must be painful when time is limited.
The Frenet-Serret frame is useful in proofs of interesting propositions, but not a good choice for practical computation.
Instead, a computation from derivatives of parametrization is highly recommended.
\begin{prop}
Let $\a$ be a nondegenerate curve.
Then,
\[\kappa=\frac{\|\a'\times\a''\|}{\|\a'\|^3},\qquad\tau=\frac{\a'\times\a''\cdot\a'''}{\|\a'\times\a''\|}\]
and
\[\rT=\frac{\a'}{\|\a'\|},\qquad\rB=\frac{\a'\times\a''}{\|\a'\times\a''\|},\qquad\rN=\rB\times\rT.\]
\end{prop}
\begin{pf}
If we let $s=\|\a'\|$, then
\begin{align*}
\a'&=s\rT,\\
\a''&=s'\rT+s^2\kappa\rN,\\
\a'''&=(s''-s^3\kappa^2)\rT+(3ss'\kappa+s^2\kappa')\rN+(s^3\kappa\tau)\rB.
\end{align*}
Now the formulas are easily derived.
\end{pf}

% Examples


\subsection{General problems}

We are interested in regular curves, not a particular parametrization.
By the Theorem 2.1, we may always assume that a parametrization $\a$ has unit speed.
Let $\a$ be a nondegenerate unit speed space curve, and let $\{\rT,\rN,\rB\}$ be the Frenet-Serret frame for $\a$.

Consider a diagram as follows:
\begin{cd}
\<\a,\rT\>=\ ?\ar{r}\ar{d} & \<\a,\rN\>=\ ? \ar{l}\ar{d}\ar{r} & \<\a,\rB\>=\ ? \ar{l}\ar{d} \\
\<\a',\rT\>=1 & \<\a',\rN\>=0 &\<\a',\rB\>=0.
\end{cd}
Here the arrows indicate which term we are able to get by differentiation.
For example, if we know a condition
\[\<\a(t),\rT(t)\>=f(t),\]
then we can obtain
\[\<\a(t),\rN(t)\>=\frac{f'(t)-1}{\kappa(t)}\]
by direct differentiation since we have known $\<\a',\rT\>$ but not $\<\a,\rN\>$.
Further, we get
\[\<\a(t),\rB(t)\>=\frac{\left(\frac{f'(t)-1}{\kappa(t)}\right)'+\kappa(t)f(t)}{\tau(t)}\]
since we have known $\<\a,\rT\>$ and $\<\a',\rN\>$ but not $\<\a,\rB\>$.
Thus, $\<\a,\rT\>=f$ implies
\[\a(t)=f(t)\cdot\rT+\frac{f'(t)-1}{\kappa(t)}\cdot\rN+\frac{\left(\frac{f'(t)-1}{\kappa(t)}\right)'+\kappa(t)f(t)}{\tau(t)}\cdot\rB,\]
when given $\tau(t)\ne0$.

We suggest a strategy for space curve problems:
\begin{itemize}
\item Build and differentiate equations of the following form:
\[\<\ \text{(interesting vector)},\ \text{(Frenet-Serret basis)}\ \>\ =\ \text{(some function)}.\]
\item Aim for finding the coefficients of the position vector in the Frenet-Serret frame, and obtain relations of $\kappa$ and $\tau$ by comparing with assumptions.
\item Heuristically find a constant vector and show what you want directly.
\end{itemize}
Here we give example solutions of several selected problems.
Always $\a$ denotes a reparametrized unit speed nondegenerate curve in $\R^3$.



\begin{prb}
A curve whose normal lines always pass through a fixed point lies in a circle.
\end{prb}
\begin{sol}
\Step{1}[Formulate conditions]
By the assumption, there is a constant point $p\in\R^3$ such that the vectors $\a-p$ and $\rN$ are parallel so that we have
\[\<\a-p,\rT\>=0,\qquad\<\a-p,\rB\>=0.\]
Our goal is to show that $\|\a-p\|$ is constant and there is a constant vector $v$ such that $\<\a-p,v\>=0$.

\Step{2}[Collect information]
Differentiate $\<\a-p,\rT\>=0$ to get
\[\<\a-p,\rN\>=-\frac1\kappa.\]
Differentiate $\<\a-p,\rB\>=0$ to get
\[\tau=0.\]

\Step{3}[Complete proof]
We can deduce that $\|\a-p\|$ is constant from
\[(\|\a-p\|^2)'=\<\a-p,\a-p\>'=2\<\a-p,\rT\>=0.\]
Also, if we heuristically define a vector $v:=\rB$, then $v$ is constant since
\[v'=-\tau\rN=0,\]
and clearly $\<\a-p,v\>=0$
\end{sol}

\begin{prb}
A spherical curve of constant curvature lies in a circle.
\end{prb}
\begin{sol}
\Step{1}[Formulate conditions]
The condition that $\a$ lies on a sphere can be given as follows: for a constant point $p\in\R^3$,
\[\|\a-p\|=\const.\]
Also we have
\[\kappa=\const.\]

\Step{2}[Collect information]
Differentiate $\|\a-p\|^2=\const$ to get
\[\<\a-p,\rT\>=0.\]
Differentiate $\<\a-p,\rT\>=0$ to get
\[\<\a-p,\rN\>=-\frac1\kappa.\]
Differentiate $\<\a-p,\rN\>=-1/\kappa=\const$ to get
\[\tau\<\a-p,\rB\>=0.\]

There are two ways to show that $\tau=0$.

\emph{Method 1}:
Assume that there is $t$ such that $\tau(t)\ne0$.
By the continuity of $\tau$, we can deduce that $\tau$ is locally nonvanishing.
In other words, we have $\<\a-p,\rB\>=0$ on an open interval containing $t$.
Differentiate $\<\a-p,\rB\>=0$ at $t$ to get $\<\a-p,\rN\>=0$ near $t$, which is a contradiction.
Therefore, $\tau=0$ everywhere.

\emph{Method 2}:
Since $\<\a-p,\rB\>$ is continuous and
\[\<\a-p,\rB\>=\pm\sqrt{\|\a-p\|^2-\<\a-p,\rT\>^2-\<\a-p,\rN\>^2}=\pm\const,\]
we get $\<\a-p,\rB\>=\const$.
Differentiate to get $\tau\<\a-p,\rN\>=0$.
Finally we can deduce $\tau=0$ since $\<\a-p,\rN\>\ne0$.

\Step{3}[Complete proof]
The zero torsion implies that the curve lies on a plane.
A planar curve in a sphere is a circle.
\end{sol}

\begin{prb}
A curve such that $\tau/\kappa=(\kappa'/\tau\kappa^2)'$ lies on a sphere.
\end{prb}
\begin{sol}
\Step{1}[Find the center heuristically]
If we assume that $\a$ is on a sphere so that we have $\|\a-p\|=r$ for constants $p\in\R^3$ and $r>0$, then by the routine differentiations give
\[\<\a-p,\rT\>=0,\qquad\<\a-p,\rN\>=-\frac1\kappa,\qquad\<\a-p,\rB\>=-\left(\frac1\kappa\right)'\frac1\tau,\]
that is,
\[\a-p=-\frac1\kappa\rN-\left(\frac1\kappa\right)'\frac1\tau\rB.\]

\Step{2}[Complete proof]
Let us get started the proof.
Define
\[p:=\a+\frac1\kappa\rN+\left(\frac1\kappa\right)'\frac1\tau\rB.\]
We can show that it is constant by differentiation.
Also we can show that
\[\<\a-p,\a-p\>\]
is constant by differentiation.
So we are done.
\end{sol}

\begin{prb}
A curve with more than one Bertrand mates is a circular helix.
\end{prb}
\begin{sol}
\Step{1}[Formulate conditions]
Let $\beta$ be a Bertrand mate of $\a$ so that we have
\[\beta=\a+\lambda\rN,\qquad\rN_\beta=\pm\rN,\]
where $\lambda$ is a function not vanishing somewhere and $\{\rT_\beta,\rN_\beta,\rB_\beta\}$ denotes the Frenet-Serret frame of $\beta$.
We can reformulate the conditions as follows:
\iffalse
\begin{cd}[cells={text width=60pt, align=center}]
\<\beta-\a,\rT\>=0 \ar{r}\ar{d}& \<\beta-\a,\rN\>=\lambda \ar{r}\ar{l}\ar{d}& \<\beta-\a,\rB\>=0 \ar{l}\ar{d} \\
\<\rT_\beta,\rT\>=? \ar{r}\ar{d}& \<\rT_\beta,\rN\>=0 \ar{r}\ar{l}\ar{d}& \<\rT_\beta,\rB\>=? \ar{l}\ar{d} \\
\<\rN_\beta,\rT\>=0 \ar{r}& \<\rN_\beta,\rN\>=\pm1 \ar{r}\ar{l}& \<\rN_\beta,\rB\>=0 \ar{l}.
\end{cd}
\fi
Note that $\beta$ is not unit speed.

\Step{2}[Collect information]
Differentiate $\<\beta-\a,\rN\>=\lambda$ to get
\[\lambda=\const\ne0.\]
Differentiate $\<\beta-\a,\rT\>=0$ and $\<\beta-\a,\rB\>=0$ to get
\[\<\rT_\beta,\rT\>=\frac{1-\lambda\kappa}{\|\beta'\|},\qquad\<\rT_\beta,\rB\>=\frac{\lambda\tau}{\|\beta'\|}.\]
Differentiate $\<\rT_\beta,\rT\>$ and $\<\rT_\beta,\rB\>$ to get
\[\frac{1-\lambda\kappa}{\|\beta'\|}=\const,\qquad\frac{\lambda\tau}{\|\beta'\|}=\const.\]
Thus, there exists a constant $\mu$ such that
\[1-\lambda\kappa=\mu\lambda\tau\]
if $\a$ is not planar so that $\tau\ne0$.

We have shown that the torsion is either always zero or never zero at every point: $\lambda\tau/\|\beta'\|=\const$.
The problem can be solved by dividing the cases, but in this solution we give only for the case that $\a$ is not planar; the other hand is not difficult.

\Step{3}[Complete proof]
If
\[\beta=\a+\lambda\rN,\qquad\tilde\beta=\a+\tilde\lambda\rN\]
are different Bertrand mates of $\a$ with $\lambda\ne\tilde\lambda$, then $(\kappa,\tau)$ solves a two-dimensional linear system
\begin{align*}
\kappa+\mu\tau&=\lambda^{-1},\\
\kappa+\tilde\mu\tau&=\tilde\lambda^{-1}.
\end{align*}
It is nonsingular since $\mu=\tilde\mu$ implies $\lambda=\tilde\lambda$, which means we can represent $\kappa$ and $\tau$ in terms of constants $\lambda,\tilde\lambda,\mu,$ and $\tilde\mu$.
Therefore, $\kappa$ and $\tau$ are constant.
\end{sol}

Here is a well-prepared problem set for exercises.

\begin{prb}[Plane curves]
Let $\a$ be a nondegenerate curve in $\R^3$.
TFAE:
\begin{parts}
\item the curve $\a$ lies on a plane,
\item $\tau=0$,
\item the osculating plane constains a fixed point.
\end{parts}
\end{prb}

\begin{prb}[Helices]
Let $\a$ be a nondegenerate curve in $\R^3$.
TFAE:
\begin{parts}
\item the curve $\a$ is a helix,
\item $\tau/\kappa=\const$,
\item normal lines are parallel to a plane.
\end{parts}
\end{prb}

\begin{prb}[Sphere curves]
Let $\a$ be a nondegenerate curve in $\R^3$.
TFAE:
\begin{parts}
\item the curve $\a$ lies on a sphere,
\item $(1/\kappa)^2+((1/\kappa)'/\tau)^2=\const$,
\item $\tau/\kappa=(\kappa'/\tau\kappa^2)'$,
\item normal planes contain a fixed point.
\end{parts}
\end{prb}

\begin{prb}[Bertrand mates]
Let $\a$ be a nondegenerate curve in $\R^3$.
TFAE:
\begin{parts}
\item the curve $\a$ has a Bertrand mate,
\item there are two constants $\lambda\ne0,\mu$ such that $1/\lambda=\kappa+\mu\tau$.
\end{parts}
\end{prb}












\section{Surfaces}

\subsection{Parametrization}

%         선형독립    벡터들이 한점에서 주어졌을 때 ->
%         선형독립    벡터장이 근방에서 주어졌을 때 -> 일반적으론 2차원에서만
%         선형독립 가환벡터장이 근방에서 주어졌을 때 -> n차원 다돼
%         선형독립    직교벡터들이 한점에서 주어졌을 때
%         선형독립    직교벡터장이 근방에서 주어졌을 때
%         선형독립 직교가환벡터장이 근방에서 주어졌을 때
%             -> 곡률의 선, 점근곡선, 측지좌표

% preimage theorem

\begin{thm}
Let $S$ be a regular surface.
Let $v,w$ be linearly independent tangent vectors in $T_pS$ for a point $p\in S$.
Then, $S$ admits a parametrization $\a$ such that $\a_x|_p=v$ and $\a_y|_p=w$.
\end{thm}
\begin{thm}
Let $X,Y$ be linearly independent tangent vector fields on a regular surface $S$.
Then, $S$ admits a parametrization $\a$ such that $\a_x|_p$ and $\a_y|_p$ are parallel to $X|_p,Y|_p$ respectively for each $p\in S$.
\end{thm}
\begin{thm}
Let $X,Y$ be linearly independent tangent vector fields on a regular surface $S$.
If $\pd_XY=\pd_YX$, then $S$ admits a parametrization $\a$ such that $\a_x|_p=X|_p$ and $\a_y|_p=Y|_p$ for each $p\in S$.
\end{thm}

Let $S$ be a regular surface embedded in $\R^3$.
The inner product on $T_pS$ induced from the standard inner product of $\R^3$ can be represented not only as a matrix
\[\mat{1&0&0\\0&1&0\\0&0&1}\]
in the basis $\{(1,0,0),(0,1,0),(0,0,1)\}\subset\R^3$, but also as a matrix
\[\mat{\<\a_x,\a_x\>&\<\a_x,\a_y\>\\\<\a_y,\a_x\>&\<\a_y,\a_y\>}\]
in the basis $\{\a_x|_p,\a_y|_p\}\subset T_pS$.

\begin{defn}
\emph{Metric coefficients}
\begin{alignat*}{2}
\<\a_x,\a_x\>&=:g_{11}&\qquad
\<\a_x,\a_y\>&=:g_{12}\\
\<\a_y,\a_x\>&=:g_{21}&
\<\a_y,\a_y\>&=:g_{22}
\end{alignat*}
\end{defn}

\begin{thm}[Normal coordinates]
...?
\end{thm}


\subsection{Differentiation of tangent vectors}



\begin{defn}
Let $\a:U\to\R^3$ be a regular surface.
The \emph{Gauss map} or \emph{normal unit vector} $\nu:U\to\R^3$ is a vector field on $\a$ defined by:
\[\nu(x,y):=\frac{\a_x\times \a_y}{\|\a_x\times \a_y\|}(x,y).\]
The set of vector fields $\{\a_x|_p,\a_y|_p,\nu|_p\}$ forms a basis of $T_p\R^3$ at each point $p$ on $\a$.
The Gauss map is uniquely determined up to sign as $\a$ changes.
\end{defn}

\begin{defn}[Gauss formula, $\Gamma_{ij}^k$, $L_{ij}$]
Let $\a:U\to\R^3$ be a regular surface.
Define indexed families of smooth functions $\{\Gamma_{ij}^k\}_{i,j,k=1}^2$ and $\{L_{ij}\}_{i,j=1}^2$ by the Gauss formula
\begin{alignat*}{2}
\a_{xx}&=:\Gamma_{11}^1\a_x+\Gamma_{11}^2\a_y+L_{11}\nu,&\qquad
\a_{xy}&=:\Gamma_{12}^1\a_x+\Gamma_{12}^2\a_y+L_{12}\nu,\\
\a_{yx}&=:\Gamma_{21}^1\a_x+\Gamma_{21}^2\a_y+L_{21}\nu,&
\a_{yy}&=:\Gamma_{22}^1\a_x+\Gamma_{22}^2\a_y+L_{22}\nu.
\end{alignat*}
The \emph{Christoffel symbols} refer to eight functions $\{\Gamma_{ij}^k\}_{i,j,k=1}^2$.
The Christoffel symbols and $L_{ij}$ \emph{do depend} on $\a$.
\end{defn}
We can easily check the symmetry $\Gamma_{ij}^k=\Gamma_{ji}^k$ and $L_{ij}=L_{ji}$.
Also,
\begin{align*}
\pd_XY
&=X^i\pd_i(Y^j\a_j)\\
&=X^i(\pd_iY^k)\a_k+X^iY^j\pd_i\a_j\\
&=\left(X^i\pd_iY^k+X^iY^j\Gamma_{ij}^k\right)\a_k+X^iY^jL_{ij}\nu.
\end{align*}

% Examples


\subsection{Differentiation of normal vector}

The partial derivative $\pd_X\nu$ is a tangent vector field since
\[\<\pd_X\nu,\nu\>=\frac12\pd_X\<\nu,\nu\>=0.\]
Therefore, we can define the following useful operator.
\begin{defn}
Let $S$ be a regular surface embedded in $\R^3$.
The \emph{shape operator} is $\cS:\fX(S)\to\fX(S)$ defined as
\[\cS(X):=-\pd_X\nu.\]
\end{defn}
\begin{prop}
The shape operator is self-adjoint, i.e. symmetric.
\end{prop}
\begin{pf}
Recall that $\pd_XY-\pd_YX$ is a tangent vector field.
Then,
\[\<X,\cS(Y)\>=\<X,-\pd_Y\nu\>=\<\pd_YX,\nu\>=\<\pd_XY,\nu\>=\<\cS(X),Y\>.\qedhere\]
\end{pf}

% The reason of minus sign in the shape operator.

\begin{thm}
Let $\a:U\to\R^3$ be a regular surface and $\cS$ be the shape operator.
Then $\cS$ has the coordinate representation
\[\cS=\mat{g_{11}&g_{12}\\g_{21}&g_{22}}^{-1}\mat{L_{11}&L_{12}\\L_{21}&L_{22}}\]
with respect to the frame $\{\a_x,\a_y\}$ for tangent spaces.
In other words, if we let $X=X^i\a_i$ and $\cS(X)=\cS(X)^j\a_j$, then
\[\mat{\cS(X)^1\\\cS(Y)^2}=\mat{g_{11}&g_{12}\\g_{21}&g_{22}}^{-1}\mat{L_{11}&L_{12}\\L_{21}&L_{22}}\mat{X^1\\X^2}.\]
\end{thm}
\begin{pf}
Let $\cS(X)^j=\cS_i^jX_i$.
Then,
\[g_{ik}X^i\cS_j^kY^j=\<X,\cS(Y)\>=\<\pd_XY,\nu\>=X^iY^jL_{ij}\]
implies $g_{ik}\,\cS_j^k=L_{ij}$.
\end{pf}

% principal curvature
% mean curvature, gaussian curvature



% curvature tensor?


\subsection{Computational problems}
% 제1기본형식, 크리스토펠: 
% 모양 연산자, 제2기본형식: 바인가르텐 이퀘이션

% 가우스곡률
\begin{defn}
Let $\a:U\to\R^3$ be a regular surface.
\begin{gather*}
E:=\<\a_x,\a_x\>=g_{11},\qquad F:=\<\a_x,\a_y\>=g_{12},\qquad G:=\<\a_y,\a_y\>=g_{22},\\
L:=\<\a_{xx},\nu\>=L_{11},\qquad M:=\<\a_{xy},\nu\>=L_{12},\qquad N:=\<\a_{yy},\nu\>=L_{22}.
\end{gather*}
\end{defn}


\begin{cor}
We have $GM-FN=EM-FL$, and the \emph{Weingarten equations}:
\begin{align*}
\nu_x&=\frac{FM-GL}{EG-F^2}\a_x+\frac{FL-EM}{EG-F^2}\a_y,\\
\nu_y&=\frac{FN-GM}{EG-F^2}\a_x+\frac{FM-EN}{EG-F^2}\a_y.
\end{align*}
\end{cor}



\begin{thm}
\[\Gamma_{ij}^l=\frac12g^{kl}(g_{ik,j}-g_{ij,k}+g_{kj,i}).\]
\end{thm}

\[\frac12(\log g)_x=\Gamma_{11}^1.\]

\[\nu_x\times\nu_y=K\sqrt{\det g}\ \nu.\]
\[\a_x\times\a_y=\sqrt{\det g}\ \nu\]
\[\<\nu_x\times\nu_y,\a_x\times\a_y\>=\det\mat{\<\nu_x,\a_x\>&\<\nu_x,\a_y\>\\\<\nu_y,\a_x\>&\<\nu_y,\a_y\>}=\det\mat{-L&-M\\-M&-N}=K\det g\]











\begin{thm}[Gaussian curvature formula]
$ $\\[-12pt]
\begin{parts}
\item
In general,
\[K=\frac{LN-M^2}{EG-F^2}.\]
\item
For orthogonal coordinates such that $F\equiv0$,
\[K=-\frac1{2\sqrt{\det g}}\left((\frac1{\sqrt{\det g}}E_y)_y+(\frac1{\sqrt{\det g}}G_x)_x\right).\]
\item
For $f(x,y,z)=0$,
\[K=-\frac1{|\nabla f|^4}\mat[v]{0&\nabla f\\\nabla f^T&\Hess(f)},\]
where $\nabla f$ denotes the gradient $\nabla f=(f_x,f_y,f_z)$.
\item(Beltrami-Enneper) If $\tau$ is the torsion of an asymptotic curve, then
\[K=-\tau^2.\]
\item(Brioschi) $E,F,G$ describes $K$.
\end{parts}
\end{thm}

\begin{pf}$ $\\[-12pt]
\begin{parts}
\item Clear.
\item
We have $GM=EM$ and
\[\nu_x=-\frac LE\a_x-\frac MG\a_y,\qquad\nu_y=-\frac ME\a_x-\frac NG\a_y.\]
\[\nu_x\times\nu_y=\frac{LN-M^2}{EG}\a_x\times\a_y\]
After curvature tensors...
\end{parts}
\end{pf}



\begin{ex}
\begin{parts}
\item
(Monge's patch)
For $(x,y,f(x,y))$,
\[K=\frac{f_{xx}f_{yy}-f_{xy}^2}{(1+f_x^2+f_y^2)^2}.\]
\item
(Surface of revolution).
Let $\gamma(t)=(r(t),z(t))$ be a plane curve with $r(t)>0$.
Let
\[\a(\theta,t)=(r(t)\cos\theta,r(t)\sin\theta,z(t))\]
be a parametrization of a surface of revolution.

Then,
\begin{align*}
\a_\theta&=(-r(t)\sin\theta,r(t)\cos\theta,0)\\
\a_t&=(r'(t)\cos\theta,r'(t)\sin\theta,z'(t))\\
\nu&=\frac1{\sqrt{r'(t)^2+z'(t)^2}}(z'(t)\cos\theta,z'(t)\sin\theta,-r'(t)),
\end{align*}
and
\begin{align*}
\a_{\theta\theta}&=(-r(t)\cos\theta,-r(t)\sin\theta,0)\\
\a_{\theta t}&=(-r'(t)\sin\theta,-r'(t)\cos\theta,0)\\
\a_{tt}&=(r''(t)\cos\theta,r''(t)\sin\theta,z''(t)).
\end{align*}
Thus we have
\[E=r(t)^2,\quad F=0,\quad G=r'(t)^2+z'(t)^2,\]
and
\[L=-\frac{r(t)z'(t)}{\sqrt{r'(t)^2+z'(t)^2}},\quad M=0,\quad N=\frac{r''(t)z'(t)-r'(t)z''(t)}{\sqrt{r'(t)^2+z'(t)^2}}.\]
Therefore,
\[K=\frac{LN-M^2}{EG-F^2}=\frac{z'(r'z''-r''z')}{r(r'^2+z'^2)^2}.\]
In particular, if $t\mapsto(r(t),z(t))$ is a unit-speed curve, then
\[K=-\frac{r''}r.\]

\item
(Models of hyperbolic planes)
\end{parts}
\end{ex}


% asymptotic curve -> hyperbolic
% line of curvature -> non-umbilic

% minimal surface
% 	회전곡면
% asymptotic curve
% 	Beltrami-Enneper
% ruled surface
% developable surface

% 밀만 파커 다 풀어보기




\subsection{General problems}

% isometric....?

\begin{thm}
Surfaces of the same constant Gaussian curvature are locally isomorphic.
\end{thm}
\begin{pf}
Let
\[\mat{\|\a_r\|^2&\<\a_r,\a_t\>\\\<\a_t,\a_r\>&\|\a_t\|^2}=\mat{1&0\\0&h(r,t)^2}\]
be the first fundamental form for a geodesic coordinate chart along a geodesic curve so that $\a_{tt}$ and $\a_{rr}$ are normal to the surface.
Then,
\[K=-\frac{h_{rr}}h\]
is constant.
Also, since
\[\frac12(h^2)_r+\<\a_r,\a_{tt}\>=\<\a_{rt},\a_t\>+\<\a_r,\a_{tt}\>=\<\a_r,\a_t\>_t=0\]
implies $h_r=0$ at $r=0$, the function $f:r\mapsto h(r,t)$ satisfies the following initial value problem
\[f_{rr}=-Kf,\quad f(0)=1,\quad f'(0)=0.\]
Therefore, $h$ is uniquely determined by $K$.
\end{pf}































\chapter{Riemannian Manifolds}
Notations: Einstein summation convention, set of vector fields.

To $n$-dimensional.

\section{Coordinates intrinsicness}
% 좌표변환에 대하여 어떻게 변하는지 - 텐서에 대하여

% 크리스토펠은 좌표변환이 잘 안됨: 텐서가 아니라서
% 공변미분: 근데 걍 3차원공간 편미분에다가 이 좌표변환 추가 텀 붙은 크리스토펠 텀을 추가하면 그 미분 결과가 텐서(접벡터)가 됨

\section{Metric intrinsicness}
% metric 정의
% isometry
% 가우스의 놀라운 정리: 크리스토펠 -> 곡률텐서 -> 가우스곡률?

\begin{itemize}
\item Intrinsic: $g_{ij}$, $\Gamma_{ij}^k$, $K$, ${R^l}_{ijk}$;
\item Not intrinsic: $\nu$, $L_{ij}$, $\kappa_i$, $H$.
\end{itemize}

Isometry
\begin{ex}
Let $\a:(-\log2,\log2)\times(0,2\pi)\to\R^3$ and $\beta:(-\frac34,\frac34)\times(0,2\pi)\to\R^3$ be regular surfaces given by
\[\a(x,\theta)=(\cosh x\cos\theta,\,\cosh x\sin\theta,\,x),\qquad
\beta(r,z)=(r\cos z,\,r\sin z,\,z).\]
Their Riemannian metrics are
\[\mat{\cosh^2x&0\\0&\cosh^2x}_{(\a_x,\a_\theta)},\qquad\mat{1&0\\0&1+r^2}_{(\beta_r,\beta_z)}.\]

Define a map $f:\im\a\to\im\beta$ by
\[f:\a(x,\theta)\mapsto\beta(\sinh x,\theta)=(r(x,\theta),z(x,\theta)).\]
The Jacobi matrix of $f$ is computed
\[df|_{\a(x,\theta)}=\mat{\cosh x&0\\0&1}_{(\a_x,\a_\theta)\to(\beta_r,\beta_z)}.\]
Since $f$ is a diffeomorphism and
\[\mat{\cosh^2x&0\\0&\cosh^2x}=\mat{\cosh x&0\\0&1}^T\mat{1&0\\0&1+r^2}\mat{\cosh x&0\\0&1},\]
the map $f$ is an isometry.
\end{ex}




\section{Covariant derivatives}

\subsection{Orthogonal projection}
We are going to think about ``intrinsic'' derivatives for tangent vectors.
For coordinate independence, directional derivatives of a tangent vector field should be at least a tangent vector field, which is false for the obvious partial derivatives in the embedded surface setting; for example, $\rT$ is a tangent vector, but $\rN=\kappa\rT'$ is not tangent.

Recall that the Gauss formula reads
\[\pd_i\a_j=\Gamma_{ij}^k\a_k+L_{ij}\nu\]
so that we have
\begin{align*}
\pd_XY
&=X^i\pd_i(Y^j\a_j)\\
&=X^i(\pd_iY^k)\a_k+X^iY^j\pd_i\a_j\\
&=\left(X^i\pd_iY^k+X^iY^j\Gamma_{ij}^k\right)\a_k+X^iY^jL_{ij}\nu.
\end{align*}
If we write $\nabla_XY=\left(X^i\pd_iY^k+X^iY^j\Gamma_{ij}^k\right)\a_k$, then it embodies the orthogonal projection of $\pd_XY$ onto its tangent space, and we have
\[\pd_XY=\nabla_XY+\II(X,Y)\nu.\]

\begin{defn}
Let $\a:U\to\R^n$ be an $m$-dimensional parametrization with $\im\a=M$.
Let $X=X^i\a_i$ and $Y=Y^j\a_j$ be tangent vector fields on $M$.
The \emph{covariant derivative} of $Y$ along $X$ is defined as the orthogonal projection of the partial derivative $\pd_XY$ onto the tangent space:
\[\nabla_XY:=\left(X^i\pd_iY^k+X^iY^j\Gamma_{ij}^k\right)\a_k.\]
\end{defn}

\begin{prop}
Covariant derivatives are intrinsic.
In other words, the above definition does not depend on the choice of parametrizations.
\end{prop}
\begin{pf}
Recall that the Christoffel symbols transform as follows:
\[X^iY^j\Gamma_{ij}^k=X^aY^b\left(\Gamma_{ab}^c+\pd{x^i}{x^a}\pd{x^j}{x^b}\pd[2]{x^c}{x^i}{x^j}\right)\pd{x^k}{x^c}.\]
Thus, we have
\begin{align*}
&\left(X^i\pd_iY^k+X^iY^j\Gamma_{ij}^k\right)\a_k\\
&\quad=X^a\pd{x^a}\left(Y^c\pd{x^k}{x^c}\right)\a_k+X^aY^b\left(\pd{x^i}{x^a}\pd{x^j}{x^b}\pd[2]{x^c}{x^i}{x^j}+\Gamma_{ab}^c\right)\pd{x^k}{x^c}\a_k\\
&\quad=X^a\pd{Y^c}{x^a}\a_c+X^aY^b\left(\pd[2]{x^k}{x^a}{x^b}\pd{x^c}{x^k}+\pd{x^i}{x^a}\pd{x^j}{x^b}\pd[2]{x^c}{x^i}{x^j}\right)\a_c+X^aX^b\Gamma_{ab}^c\a_c\\
&\quad=\left(X^a\pd_aY^c+X^aY^b\Gamma_{ab}^c\right)\a_c
\end{align*}
since
\[\pd[2]{x^j}{x^a}{x^b}\pd{x^c}{x^j}+\pd{x^i}{x^a}\pd{x^j}{x^b}\pd[2]{x^c}{x^i}{x^j}=\pd{x^a}\left(\pd{x^j}{x^b}\pd{x^c}{x^j}\right)=\pd_a\delta_b^c=0.\qedhere\]
\end{pf}



\subsection{Connections}

We will give a coordinate-free axiomatic definition of covariant derivatives and show that they coincide.
By doing this, we obtain an alternative proof for the statement that covariant derivatives are intrinsic.

\begin{defn}[Affine connection]
Let $M$ be the image of a parametrization.
An \emph{affine connection} is a map $\nabla:\fX(M)\times\fX(M)\to\fX(M)$ such that
\begin{parts}
\item $\nabla_{(-)}Y:\fX(M)\to\fX(M):X\mapsto\nabla_XY$ is $C^\infty(M)$-linear;
\item $\nabla_X(-):\fX(M)\to\fX(M):Y\mapsto\nabla_XY$ is $\R$-linear;
\item the Leibniz rule
\[\nabla_X(fY)=(\pd_Xf)Y+f\nabla_XY\]
is satisfied.
\end{parts}
\end{defn}

\begin{defn}[Metric connection]
Let $M$ be the image of a parametrization and $\<\,,\>$ be a Riemannian metric on $M$.
A \emph{metric connection} is an affine connection $\nabla:\fX(M)\times\fX(M)\to\fX(M)$ such that:
\[\pd_Z\<X,Y\>=\<\nabla_ZX,Y\>+\<X,\nabla_ZY\>.\]
\end{defn}

\begin{defn}[Levi-Civita connection]
Let $M$ be the image of a parametrization and $\<\,,\>$ be a Riemannian metric on $M$.
A \emph{Levi-Civita connection} is a metric connection $\nabla:\fX(M)\times\fX(M)\to\fX(M)$ such that:
\[\nabla_XY-\nabla_YX=\pd_XY-\pd_YX.\]
\end{defn}

\begin{thm}
Let $\a:U\to\R^n$ be an $m$-dimensional parametrization with $M=\im\a$.
Then, there is a unique Levi-Civita connection on $M$.
\end{thm}
\begin{pf}
(Uniqueness)
Suppose $\nabla$ is a Levi-Citiva connection on $M$.
\begin{align*}
2\<\nabla_XY,Z\>&=\pd_X\<Y,Z\>+\pd_Y\<X,Z\>-\pd_Z\<X,Y\>\\
&\qquad-\<[X,Z],Y\>-\<[Y,Z],X\>+\<[X,Y],Z\>.
\end{align*}

(Existence)
\end{pf}

Our claim is that this definition is equivalent to the above coordinate dependent definition, the Levi-Civita connection, of the covariant derivative.

\begin{prop}
Let $S$ be a regular surface embedded in $\R^3$.
If we define Christoffel symbols as the Gauss formula, then
\[\fX(S)\times\fX(S)\to\fX(S):(X^i\a_i,Y^j\a_j)\mapsto\left(X^i\pd_iY^k+X^iY^j\Gamma_{ij}^k\right)\a_k\]
defines a Levi-Civita connection.
\end{prop}




\section{Parallel transport}
% linearity, holonomy



\section{Geodesics}
% 측지선에 대한 기저(n,T,S)
% 측지꼬임/곡률
% 측지선 방정식
% 측지완비: 호프 리노프
% 지수사상: 가우스 보조정리
% 야코비장
% 카르탕 아다마르

\section{Curvature}



\chapter{Global Theory of Curves and Surfaces}

% Global theory
%  곡선: 등주부등식, (네 꼭짓점, 펜첼/페리-밀너), 볼록성/오발
%  곡면: 최소곡면, (컴팩트곡면분류, 가우스-보네), 임베딩문제,

\end{document}