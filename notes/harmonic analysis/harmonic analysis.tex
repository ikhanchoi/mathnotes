\documentclass{../note}
\usepackage{../../ikany}


\begin{document}
\title{Harmonic Analysis}
\author{Ikhan Choi}
\maketitle
\tableofcontents

\part{Fourier analysis}

\chapter{Fourier series}

\chapter{Fourier transform}
\section{Fourier transform of $L^1$ functions}
inversion
Riemann-Lebesgue

\section{Fourier transform of $L^p$ functions}
plancherel and for $L^2$,

\section{Tempered distributions}


\chapter{}

\part{Singular integral operators}
\chapter{Cald\'eron-Zygmund theory}

\section{Hilbert transform}

\section{Calder\'on-Zygmund operator of convolution type}
\begin{prb}[Calder\'on-Zygmund decomposition of sets]
Let $E_nf$ be the conditional expectation with repect to the $\sigma$-algebra generated by dyadic cubes with side length $2^{-n}$.
Let $Mf=\sup_nE_n|f|$ be the maximal function, and let $\Omega:=\{x:Mf(x)>\lambda\}$ for fixed $\lambda>0$.
For $x\in\Omega$ let $Q_x$ be the maximal dyadic cube such that $x\in Q_x$ and
\[\frac1{|Q_x|}\int_{Q_x}|f|>\lambda.\]
\begin{parts}
\item
$\{Q_x:x\in\Omega\}$ is a countable partition of $\Omega$.
\item
We have an weak type estimate $|\Omega|\le\frac1\lambda\|f\|_{L^1}$.
\item
$\|f\|_{L^\infty(\R^d\setminus\Omega)}\le\lambda$.
\item
For $x\in\Omega$
\[\frac1{|Q_x|}\int_{Q_x}|f|\le2^d\lambda.\]
\end{parts}
\end{prb}

\begin{prb}[Calder\'on-Zygmund decomposition of functions]
Let
\[g(x):=\begin{cases}|f(x)|&,x\notin\Omega\\\frac1{|Q_x|}\int_{Q_x}|f|&,x\in\Omega\end{cases}\]
and $b_i:=(|f|-g)\chi_{Q_i}$ so that $|f|=g+b$ where $b=\sum_ib_i$.
\begin{parts}
\item $\|g\|_{L^1}=\|f\|_{L^1}$ and $\|g\|_{L^\infty}\lesssim_d\lambda$.
\item $\|b\|_{L^1}\le2\|f\|_{L^1}$ and $\int b_i=0$.
\end{parts}
\end{prb}
\begin{pf}

\end{pf}


\begin{prb}[Calder\'on-Zygmund operator of convolution type]
Let $T:\cD(\R^d)\to\cD'(\R^d)$ be a \emph{singular integral operator of convolution type} in the sense that there is $K\in L_\loc^1(\R^d\setminus\{0\})\cap\cD'(\R^d)$ such that
\[Tf(x)=\int K(x-y)f(y)\,dy\]
for all $f\in\cD(\R^d)$, whenever $x\notin\supp f$.
If $T$ is $L^2$-bounded
\[\|Tf\|_{L^2}\lesssim\|f\|_{L^2}\]
and satisfies the \emph{H\"ormander condition}
\[\int_{|x|>2|y|}|K(x-y)-K(x)|\,dx\lesssim1,\]
then it is called a \emph{Calder\'on-Zygmund} operator.

Let $f=g+b=g+\sum_ib_i$ be the Calder\'on-Zygmund decomposition, and let $\Omega^*:=\bigcup_iQ_i^*$ where $Q_i^*$ is the cube with the same center as $Q_i$ and whose sides are $2\sqrt d$ times longer. 
\begin{parts}
\item
The $L^2$-boundedness implies
\[|\{x:|Tg(x)|>\tfrac\lambda2\}|\lesssim_d\frac1\lambda\|f\|_{L^1}.\]
\item
The H\"ormander condition implies
\[|\{x:|Tb(x)|>\tfrac\lambda2\}\setminus\Omega^*|\lesssim_d\frac1\lambda\|f\|_{L^1}.\]
\item
\end{parts}
\end{prb}
\begin{pf}
(a)
Using the Chebyshev inequality and the H\"older inequality,
\[|\{x:|Tg(x)|>\frac\lambda2\}|
\le\frac4{\lambda^2}\|Tg\|_{L^2(\Omega)}^2
\le\frac{4C}{\lambda^2}\|g\|_{L^2(\Omega)}^2
\le\frac{4C}{\lambda^2}\|g\|_{L^1(\Omega)}\|g\|_{L^\infty(\Omega)}.
\]

(b)
Write
\[|\{x:|Tb(x)|>\tfrac\lambda2\}\setminus\Omega^*|
\le\frac2\lambda\int_{\R^d\setminus\Omega^*}|Tb(x)|\,dx
\le\frac2\lambda\sum_i\int_{\R^d\setminus Q_i^*}|Tb_i(x)|\,dx.\]
Since $x\in\R^d\setminus Q_i^*$ does not belong to $\supp b_i\subset Q_i$ and $\int b_i=0$, we have
\[Tb_i(x)=\int_{Q_i}K(x-y)b_i(y)\,dy=\int_{Q_i}[K(x-y)-K(x)]b_i(y)\,dy,\]
and
\[\int_{\R^d\setminus Q_i^*}|Tb_i(x)|\,dx
=\int_{Q_i}|b_i(y)|\int_{\R^d\setminus Q_i^*}|K(x-y)-K(x)|\,dx\,dy
\lesssim\|b_i\|_{L^1}.\]
(We need to show it is valid even though $b_i$ is not smooth)

(c)

\end{pf}

\section{$L^2$-boundedness of truncated integrals}

\section{Calder\'on-Zygmund operator of non-convolution type}
standard kernels



\section*{Exercises}
\begin{prb}[Gradient size condition]
Let $|\nabla K(x)|\lesssim\frac1{|x|^{d+1}}$ for $x\ne0$.
Then, convolution with $K$ is a Calder\'on-Zygmund operator.
\end{prb}




\chapter{Littlewood-Paley theory}
\chapter{Multiplier theorems}

\part{Pseudo-differential operators}


\part{Oscillatory integral operators}

\end{document}