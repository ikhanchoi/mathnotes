\documentclass{../note}
\usepackage{../../ikany}


\begin{document}
\title{Classical Physics}
\author{Ikhan Choi}
\maketitle
\tableofcontents

\part{Mechanics}
\chapter{Newtonian mechanics}
\section{Laws of motion}
\begin{prb}[Galilean structure]
\end{prb}
\begin{prb}[Galilean group]
\end{prb}
\begin{prb}[Conservation laws]
\end{prb}
\section{Oscillation}
\begin{prb}[Harmonic oscillator]
\end{prb}
\begin{prb}[Damped oscillation]
\end{prb}
\begin{prb}[Pendulum]
\end{prb}
\begin{prb}[Lissajous curve]
\end{prb}
\begin{prb}[Coupled oscillation]
\end{prb}
\section{Central forces}
\begin{prb}[Polar coordinates]
\end{prb}
\begin{prb}[Effective potential]
\end{prb}
\begin{prb}[Kepler's problem]
\end{prb}
\begin{prb}[Rutherford scattering]
\end{prb}
\section{System of particles}
\begin{prb}[Closed systems]
\end{prb}
\begin{prb}[Collisions]
\end{prb}
\begin{prb}[Two-body problem]
\end{prb}
\begin{prb}[Three-body problem]
\end{prb}

\section*{Exercises}
method of similarity (scaling)



\chapter{Lagrangian mechanics}
\section{Calculus of variations}
\begin{prb}[Euler-Lagrange equation]
\end{prb}
\begin{prb}[Closed system]
$\pd{\cL}{t}=0$
\end{prb}
\begin{prb}[Definition of generalized momentum]
$\pd{\cL}{q}=0$
\end{prb}
\begin{prb}[Equivalence to Newtonian mechanics]
\end{prb}
\section{Rigid bodies}
\begin{prb}[Inertia tensor]
\end{prb}
\begin{prb}[Eulerian angle]
\end{prb}
\begin{prb}[Lagrangian top]
\end{prb}
\section*{Exercises}
\begin{prb}[Brachiostochrone]
\end{prb}
\begin{prb}[Geodesic on the sphere]
\end{prb}
\begin{prb}[Dido's isoperimetric problem]
\end{prb}
\begin{prb}[Pendulum with moving support]
A rhenomic system
\end{prb}
\begin{prb}[Sliding beads on a rim]
\end{prb}
\begin{prb}[Double pulley system]
\end{prb}


\chapter{Hamiltonian mechanics}

\section*{Exercises}






\part{}


\chapter{Fluid mechanics}

\chapter{Waves}

\chapter{Thermodynamics}
\section{Equilibrium}
Equation of states
Maxwell's relations
Thermal processes
\section{Ensembles}
ensembles
microcanonical, canonical, grand canonical
classical gas

\chapter{Kinetic theory}
ergodic hypothesis
Boltzmann statistics
Boltzmann equation
BBGKY hierarchy
stochastic processes
linear response

\chapter{}
\chapter{}



\part{Classical field theory}
\chapter{Relativity}
\section{Special relativity}
\section{General relativity}
\section{Einstein field equation}
\section{Black holes}

\chapter{Electromagnetism}

\chapter{Lagrangian field theory}

\end{document}