\documentclass{../exp}
\usepackage{../../ikany}

\title{Finite Group Theory}

\begin{document}
\maketitle
\tableofcontents

\iffalse
presentation
	quotient of a free group
	homomorphism 잡기
		free group을 정의역으로 먼저 잡고 well-defined 보여서 quotient로 내리기
	element들을 두 제너레이터로 표현하기
isomorphism 보이기
	isomorphism theorem
	order argument
subgroup criterion: 항등원, 역원, 닫힘 순으로 증명
centralizer, center, normalizer
	각 그룹의 계산 -> 기본적으로 노가다, 마지막에 라그랑지 확인
	센트럴라이저의 대칭성
	N(H)/C(H) 정리, G/Z(G)=Inn(G)
stabilizer

cyclic group


Heisenberg group


Three actions
- left multiplication on coset space
- conjugation on a set of some subgroups
- conjugation on a normal subgroup



아벨리안임을 보이기
	G/Z(G) is cyclic
	|G|=p^2
	n-abelian
		G^n commutes G^{n-1}
		surjective cube endomorphism
		n-abelian implies n(n-1)-central
		order prime to n(n-1)



----------
#
아벨군: 순환군, n-abelian 등 여러 조건, 기본정리
대칭군: 생성원, 트랜지티브, 교대군과 부호
콕세터: 이면군, 프레젠테이션
선형군: 

#
군 확장
컴포지션 시리즈의 이해
반직접: / 계산->아벨군의 자기동형군
중심적: 군코호몰로지 / 계산->보편계수정리

#
액션, 실로우: existence, congruence condition
	카운팅:
	인덱스: 푸앵카레 정리, least prime ind
	소인수분해 형태별 분석
	케이스 나누기에 매우 좋은 조건을 제공
		노말 실로우 -> 바로 반직접곱
p: 1 - 
p2: 2
p3: 5
p4: 14, 15
pq: p<q
	p/q-1: 1
	p|q-1: 2
p2q: p<q
pq2: p<q
p3q?
pqr

#
p군: 비자명센터, 개수 겁나많음
닐포턴트: 피팅, 프라티니
솔버블?
센트럴 시리즈?
단순군: 단순군 아니기 테크닉, 단순군 보이기 테크닉, 교대군과 리타입 선형군


\fi
\section{Special groups}

\subsection{Abelian groups}
\subsection{Symmetric groups}
\subsection{Coxeter groups}
\subsection{Linear groups}


\section{Sylow theory}
\begin{defn}[Sylow $p$-subgroup]
Let $G$ be a finite group of order $n=p^am$ for a prime $p\nmid m$.
A \emph{Sylow $p$-subgroup} is a subgroup of order $p^a$.
We are going to denote the set of Sylow $p$-subgroups by $\Syl_p(G)$ and the number of Sylow $p$-subgroups by $n_p(G)$.
\end{defn}

\begin{thm}[The Sylow theorem]
Let $G$ be a finite group of order $n=p^am$ for a prime $p\nmid m$.
Then,
\[p\mid n_p-1,\qquad n_p\mid m\]
for some $k\in\N$.
\end{thm}
\begin{pf}
\Step{1}[Sylow $p$-subgroups exist]
We apply mathematical induction.
The base step is trivial.
Suppose every finite group of order less than $n$ possesses a Sylow $p$-subgroup.

By applying the orbit-stabilizer theorem for the action $G\acts G$ by conjugation, build the class equation
\[|G|=|Z(G)|+\sum_i|G:C_G(g_i)|.\]
There are two cases: $p\mid|Z(G)|$ or $p\nmid|Z(G)|$.

\Case{1}[$p\mid|Z(G)|$]
The group $G$ has a normal subgroup of order $p$ by applying Cauchy's theorem for abelian groups on the center.
Then, the inverse image of a Sylow $p$-subgroup of the quotient group is also a Sylow $p$-subgroup of $G$.

\Case{2}[$p\nmid|Z(G)|$]
Since $p\mid n$, we have $p\nmid|G:C_G(g)|$ for some $g\in G$.
Then, a Sylow $p$-subgroup of the centralizer is also a Sylow $p$-subgroup of $G$.

Therefore, we are done for Step 1.

\bigskip
\Step{2}[Sylow $p$-subgroup that is normal is unique]
Note that $p$ does not divide the order of the quotient group.
Every $p$-subgroup should be contained in the Sylow $p$-subgroup, the kernel of the quotient map.
The Sylow $p$-subgroup is clearly unique.

\bigskip
\Step{3}[Sylow $p$-subgroups get action by conjugation]
Let $P$ be a Sylow $p$-subgroup of $G$.
We construct class equations via the orbit-stabilizer theorm for various actions to extract information on $n_p$.
Note that stabilizers in any setwise conjugation action is exactly normalizers.
\begin{cond}
\item The action $P\acts\Syl_p(G)$ gives
\[n_p=1+\sum_i|P:N_P(P_i)|\]
since $P=N_P(P_i)$ implies $P\trianglelefteq N_G(P_i)$ and $P=P_i$.
\item Suppose the action $G\acts\Syl_p(G)$ is not transitive.
Take another Sylow $p$-subgroup $P'$ is not conjugate with $P$ in $G$.
The two actions $P\acts\Orb_G(P)$ and $P'\acts\Orb_G(P)$ gives
\[|\Orb_G(P)|=1+\sum_i|P:N_P(P_i)|=\sum_i|P':N_{P'}(P_i)|.\]
It deduces $|\Orb_G(P)|\equiv0,1\pmod{p}$ simultaneously, which leas a contradiction.
\item The action $G\acts\Syl_p(G)$ gives
\[n_p=|G:N_G(P_i)|\]
for all $P_i\in\Syl_p(G)$ because the action is transitive.
\end{cond}
Then, (1) proves $p\mid n_p-1$, and (3) proves $n_p\mid m$.
\end{pf}

\begin{cor}
Let $G$ be a finite group.
Then,
\begin{cond}
\item every pair of two Sylow $p$-subgroup is conjugate.
\item every $p$-subgroup is contained in a Sylow $p$-subgroup.
\item a Sylow $p$-subgroup is normal if and only if $n_p=1$.
\end{cond}
\end{cor}

\begin{thm}
Alternative proof for existence of $p$-groups.
\end{thm}
\begin{pf}
Let $|G|=p^{a+b}m$.
Let $\cP_{p^a}$ be the set of all $p^a$-sets in $G$.
Give $G\acts\cP_{p^a}$ by left multiplication.
Since $v_p(|\cP_{p^a}|)=v_p({p^a(p^bm)\choose p^a})=b$, there is an orbit $\cO$ such that $v_p(|\cO|)\le b$.
We have transitive action $G\acts\cO$ and the stabilizer $H$ satisfies $p^a\mid|G|/|\cO|=|H|$.
Since $H\acts\cO$ trivially, $H\acts A$ for $A\in\cO\subset\cP_{p^a}$.
It is only possible when $H\subset A$, hence $|H|=p^a$.
\end{pf}


What we want to find is subgroup lattices.
A subgroup lattice particularly contains data about orders and conjugacy classes of subgroups.

In order to show the existence of subgroups of paricular orders:
\begin{cond}
\item $p$-group theory, (including Cuachy and Sylow)
\item extension theory, (what can subgroups of subgroups do?)
\item normalizers,
\item kernel of permutation representation
\end{cond}

In order to find the size of conjugacy classes:
\begin{cond}
\item measure the order of normalizers, (find some groups normalize a subgroup)
\item count elements,
\end{cond}



\section{Extensions}

\begin{prop}
Let $N$ and $H$ be groups.
Then, the following objects have one-to-one correspondences among each other.
\begin{cond}
\item isomorphic types of groups $G$ such that a sequence \begin{es}0\>N\>G\>H\>0\end{es} is exact and right split,
\item isomorphic types of groups $G$ such that $N\normal G\ge H$ with $G=NH$ and $N\cap H=1$,
\item group actions $H\acts N$ preserving the group structure of $N$.
\end{cond}
\end{prop}
\begin{defn}
The group $G$ in the previous proposition is called the \emph{semidirect product} of $N$ and $H$.
\end{defn}


\begin{es}
0\>F\>E\>G\>0.
\end{es}
Four data $G,F,\f:G\to\Aut(F),c:G\x G\to F$ completely determine the extension $E$.

Suppose we have an extension $F\to E\to G$.
There is a \emph{set-theoretic section} $s:G\to E$.
The number of $s$ is $|G||F|$.

Definition of \emph{action} $\f$:
For two sections $s$ and $s'$, $s(g)$ and $s'(g)$ acts on $F$ equivalently.
Thus, we can define a \emph{group homomorphism} $\f:G\to\Aut(F)$ independently on sections.

Definition of \emph{2-cocycle} $c$:
It is a \emph{set-theoretic function} $c:G\x G\to F$ defined by $c(g,g')=s(g)s(g')s(gg')^{-1}$ for a section $s$.
Actually, $c$ depends on the section $s$, and $c$ measures how much $s$ fails to be a group homomorphism.
It requires the cocycle condition for the associativity of group operation, i.e.
\[c(g,h)c(gh,k)=\f_g(c(h,k))c(g,hk)\]
should be satisfied.
Conversely, a map $G\x G\to F$ satisfying the condition the cocycle condition gives a associative group operation on $G$.

If $F$ is abelian, then the set of cocycles forms an abelian group, and is denoted by $Z^2(G,F)$.
The boundaries are also defined in abelian $F$ case.


\begin{cond}
\item $\f$, $c$ is trivial $\iff$ direct product,
\item $c$ is trivial $\iff$ $s$ is a homomorphism $\iff$ semidirect product,
\item $\f$ is trivial $\iff$ central extension.
\end{cond}

Group cohomology is defined for a group $G$ and $G$-module $A$ (three data: $G,A,\f$.
What is important is that the cohomology depends on the action of $G$ on $A$.

If $\f$ is trivial so that $A$ is just an abelian group, then the universal coefficient theorem can be applied.
\end{document}
