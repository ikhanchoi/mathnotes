\documentclass{../note}
\usepackage{../../ikany}



\begin{document}
\title{Representation Theory}
\author{Ikhan Choi}
\maketitle
\tableofcontents

\part{Finite group representations}
\chapter{Character theory}

\section{Irreducible representations}
\begin{prb}[Definition of group representations]
\end{prb}

\begin{prb}[Intertwining maps]
\end{prb}

\begin{prb}[Subrepresentations]
We say \emph{invariant} or \emph{stable}
\end{prb}

\begin{prb}[Irreducible representations]
indecomposable and irreducible
\end{prb}

\begin{prb}[Maschke's theorem]
Let $G$ be a finite group and $k$ a field of characteristic coprime to $|G|$.
Let $(\rho,V)$ be a finite-dimensional representation of $G$ over $k$.
Let $W$ be an invariant subspace of $V$.
\begin{parts}
\item There is an invariant subspace $W^\perp$ of $V$ that is a complement of $W$.
\item Every finite-dimensional representation of $G$ over $k$ is isomorphic to the direct sum of irreducible representations of $G$ over $k$.
\item If $k=\R$ or $\C$, then there is a inner product on $V$ such that $W^\perp$ is orthogonal to $W$. With this innerproduct, $\rho(g)$ is orthogonal (resp. unitary) for all $g\in G$.
\end{parts}
\end{prb}

\begin{prb}[Schur's lemma]
Let $G$ be a finite group and $k$ a field.
Let $(\rho_1,V_1)$ and $(\rho_2,V_2)$ be irreducible representations of $G$ over $k$.
Let $\psi\in\hom_G(V_1,V_2)$ be an intertwining map.
\begin{parts}
\item If $V_1$ and $V_2$ are not isomorphic, then $\psi=0$.
\item If $V_1$ and $V_2$ are isomorphic, then $\psi$ is a homothety.
\end{parts}
\end{prb}




\section{Group algebra}


\begin{prb}[Modules and representations]
ring <-> group
module <-> representation
finitely generated <-> finite dimensional
\end{prb}

\begin{prb}[Wedderburn's theorem]
central idempotents
dimension computation
\end{prb}

\begin{prb}[Group algebra]
regular representation
$k[G]$-module and $G$-representation correspondence
\begin{parts}
\item $\C[G]$ is the direct sum of all irreducible representations.
\item $|G|=\sum_{[V]\in\hat G}(\dim V)^2$.
\end{parts}
\end{prb}

\begin{prb}
The number of irreducible representations and the number of conjugacy classes
double counting on $Z(\C[G])$.
\end{prb}



\section{Characters}




\begin{prb}[Space of class functions]
Ring and inner product structure on the space of class functions.
\begin{parts}
\item $\dim\hom_G(V_1,V_2)=\<\chi_{V_1},\chi_{V_2}\>$.
\item Irreducible characters form an orthonormal basis of the space of class functions.
\end{parts}
\end{prb}

\begin{prb}[Characters classify representations]
Let $G$ be a finite group and let $\mathbf{Rep}(G)$ be the category of finite-dimensional representations of $G$ over $\C$.
\[\Tr:\mathbf{Rep}(G)\to\{\text{finite sum of irreducible characters}\}\]
surjectivity: trivial
injectivity:
	Suppose two characters are equal.
	Maschke -> all characters are sum of irreducible characters
	Schur -> orthogonality, so the coefficients are all equal
	irreducible-factor-wisely construct an isomoprhism.
\end{prb}



\begin{prb}[Character table]
computation of matrix elements by character table
abelian group, 1dim rep lifting
\begin{center}
$\begin{array}{c|ccc}
S^3&e&(12)&(123)\\\hline
1&1&1&1\\
\e&1&-1&1\\
\rho&2&0&-1
\end{array}$
\end{center}
\end{prb}






the dual inner product: conjugacy check
relation to normal subgroups
center of rep




algebraic integer
dim of irrep divides group order
burnside pq theorem





\chapter{Classification of representations}
\section{Symmetric groups}
young tableux

\section{Linear groups over finite fields}
GL2 and SL2 over finite fields

\section{Induced representations}
induction and restriction of reps (from and to subgroup)
frobenius reciprocity, mackey theory





tensoring, complex, real
symmetric, exterior


\chapter{Brauer theory}












\part{Lie algebras}
\chapter{Semisimple Lie algebras}
Solvability and nilpotency
Engel's theorem
Killing forms

Casimir element
Weyl's theorem

Cartan subalgebra uniqueness? (conjugacy theorem)




\chapter{Root systems}
root space decomposition
integrality
Weyl group

Coxeter graph
Dynkin diagram
Real forms

Isomorphism theorem

Existence theorem
Universal enveloping algebra
PBW theorem
Verma module




\chapter{Representations of Lie algebras}
\section{Representations of $\fsl(2,\C)$}
\begin{prb}[Pauli matrices]
Pauli matrices are
\[\sigma_1=\mat{0&1\\1&0},\quad\sigma_2=\mat{0&-i\\i&0},\quad\sigma_3=\mat{1&0\\0&-1}.\]
\begin{parts}
\item $\{\sigma_1,\sigma_2,\sigma_3\}$ is a basis of complex Lie algebra $\fsl(2,\C)$, and $\{i\sigma_1,i\sigma_2,i\sigma_3\}$ is a basis of real Lie algebra $\fso(3)$.
\item For a unit vector $n=(n_1,n_2,n_3)\in\R^3$, $n_1\sigma_1+n_2\sigma_2+n_3\sigma_3$ has eigenvalues $\pm1$.
\end{parts}
\end{prb}
\section{Highest weight theorem}

\section{Multiplicity formulas}

\section*{Exercises}
\begin{prb}[Triplets and quadraplets]
Let $(\pi_2,V_2)$ be the irreducible representation of $\fsl(2,\C)$ of degree two.
Consider $V_2\otimes V_2$.
Cartan element $S_z$.
$V_2^{\otimes3}$.
\end{prb}
\begin{prb}[Casimir element]
Casimir element decomposes a representation into irreducible representations.
\end{prb}








\part{Lie groups}
\chapter{Lie correspondence}
\section{Baker-Campbell-Hausdorff formula}
Lie's three theorems
\section{Fundamental groups of Lie groups}

\chapter{Compact Lie groups}
\section{Special orthogonal groups}
\section{Special unitary groups}
\section{Symplectic groups}

\section*{Exercises}
\begin{prb}[Lorentz group]
$\SL(2,\C)\to\SO^+(1,3)$
\begin{parts}
\item $O(1,3)$ has four components and $SO^+(1,3)$ is the identity component. Orthochronous $O^+(1,3)$, proper $SO(1,3)$.
\end{parts}
\end{prb}

\chapter{Representations of Lie groups}
\section{Peter-Weyl theorem}
\section{Spin representations}
Clifford algebra






\part{Hopf algebras}
\chapter{}
\chapter{Quantum groups}



\end{document}