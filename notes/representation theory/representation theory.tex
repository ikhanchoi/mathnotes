\documentclass{../note}
\usepackage{../../ikany}



\begin{document}
\title{Representation Theory}
\author{Ikhan Choi}
\maketitle
\tableofcontents

\part{Finite group representations}
\chapter{Character theory}

\begin{prb}[Definition of group representations]
\end{prb}

\begin{prb}[Interwining maps]
\end{prb}

\begin{prb}[Irreducible representations]
indecomposable and irreducible
\end{prb}

\begin{prb}[Maschke's theorem]
\end{prb}






\begin{prb}[Space of interwining maps and inner product]
$\Hom_G(V,W)$
dimension is equal to the inner product of characters
\end{prb}



direct sum of rep -> sum of char

injectivity proof
	Suppose two characters p and r are equal.
	Maschke: all characters are sum of irreducible characters
	Schur: orthogonality, so the coefficients are all equal
	irreducible-factor-wisely construct an isomoprhism.

% 일단 C로 쓰고 난 다음 일반화해서 쓰자
irreducible characters form an ONB of the space of class functions
	proof: irred number counting
	group algebra double counting?
surjectivity desciption
	nonnegative integral linear combination of irreducible characters

character table: %옆으로는 켤레류(도메인), 아래로는 irr char(함수)
computation of matrix elements by character table
abelian group, 1dim rep lifting


\begin{prb}[Modules and representations]
ring <-> group
module <-> representation
finitely generated <-> finite dimensional
\end{prb}

\begin{prb}[Group algebra]
or group ring,
regular representation
$k[G]$-module and $G$-representation correspondence
\end{prb}

\begin{prb}[Wedderburn's theorem]
central idempotents
dimension computation
\end{prb}
any irrep is a summand of CG, and the dimension arg implies CG is dsum of all irrep.




tensoring, complex, real
symmetric, exterior




the dual inner product: conjugacy check
relation to normal subgroups
center of rep




algebraic integer
dim of irrep divides group order
burnside pq theorem


\chapter{Computation of irreducible representations}
\section{Symmetric groups}
young tableux

\section{Linear groups over finite fields}
GL2 and SL2 over finite fields

\section{Induced representations}
induction and restriction of reps (from and to subgroup)
frobenius reciprocity, mackey theory



\chapter{Brauer theory}




\part{Lie groups}
\chapter{Lie correspondence}
Lie's three theorems
Baker-Campbell-Hausdorff formula
\chapter{Classical groups}
SO, SU
\chapter{Representations of compact groups}
unitary representation
fundamental group obstruction
infinite dimension: Peter Weyl
projective representations




\part{Lie algebras}
\chapter{Semisimplicity}
killing forms, cartan subalgebra
\chapter{Root systems}
dynkin digram
real forms
\chapter{Representations of Lie algebras}
universal enveloping algebra, pbw theorem, verma module
highest weight theorem



\part{Quantum groups}
\chapter{Hopf algebras}
\chapter{}


\end{document}