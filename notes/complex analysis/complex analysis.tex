\documentclass{../note}
\usepackage{../../ikany}


\begin{document}
\title{Complex Analysis}
\author{Ikhan Choi}
\maketitle
\tableofcontents


\part{Holomorphic functions}


\chapter{Cauchy theory}
\section{Complex differentiability}

\begin{prb}
\end{prb}

\begin{prb}[Cauchy-Riemann equation]
\begin{parts}
\item For $f\in C^1(\Omega,\R^2)$, $f$ is holomorphic if and only if it satisfies the Cauchy-Riemann equation. (Is the $C^1$ condition necessary?)
\end{parts}
\end{prb}


\section{Contour integral}
\begin{prb}[Definition of contour integral]
Let $\Omega\subset\C$ be a domain.
Let $f\in C(\Omega)$ and let $\gamma:[a,b]\to\Omega$ be a $C^1$ curve.
Then, the \emph{contour integral} of $f$ along the curve $\gamma$ is defined by
\[\int_\gamma f(z)\,dz:=\int_a^bf(\gamma(t))\gamma'(t)\,dt.\]
In the language of differential geometry, it is a special case of integration with the pullback form $\gamma^*(f(z)\,dz)$.
We can extend the definition of contour integral to \emph{piecewise $C^1$ curves}, which will be meant by \emph{contours}.
\end{prb}
ML lemma
\[\int_\gamma f'(z)\,dz=f(\gamma(b))-f(\gamma(a)).\]
\[\int_{|z|=1}z^n\,dz=\begin{cases}2\pi i&\text{ if }n=-1,\\0&\text{ otherwise }.\end{cases}\]

\begin{prb}[Cauchy theorem]
We will describe the idea with the notion of homotopy.
Stokes theorem(or Green's formula) and the proof by homotopy triangulation.
\end{prb}
\begin{prb}[Cauchy integral formula]
Remind the proof of the mean value property for harmonic functions.
The proof essentially have a shrinking process and boundedness of the difference quotient.
Higher order version: can be prove before the analyticity?
\end{prb}



\begin{prb}[Estimates of polynomials]
Let $p\in\C[z]$ be a polynomial of degree $n$ such that
\[p(z)=\sum_{k=0}^na_kz^k,\quad a_n\ne0.\]
\begin{parts}
\item $|p(z)|\lesssim|z|^n$.
\item There is $R>0$ such that $|p(z)|\gtrsim|z|^n$ for $|z|\ge R$.
\end{parts}
\end{prb}
\begin{pf}
(b)
We want to justify that the leading term $a_nz^n$ is dominant in the series $\sum_{k=0}^na_kz^k$ when $|z|$ is sufficiently large.
Let $\e>0$.
Since $p(z)-a_nz^n$ is of degree at most $n-1$, we can take $R>0$ such that for $|z|\ge R$ we can control the relative error as
\[\left|\frac{p(z)-a_nz^n}{a_nz^n}\right|<\e,\]
which implies
\[|p(z)|\ge(1-\e)|a_n||z^n|.\]
\end{pf}


\begin{prb}[Cauchy estimates]
\begin{parts}
\item If an entire function $f$ satisfies $|f(x)|\lesssim1+|x|^n$, then $f$ is a polynomial of degree at most $n$. In particular, the \emph{Liouville theorem} follows; a bounded entire function is constant.
\end{parts}
\end{prb}

\begin{prb}[Morera theorem]
antiderivative
\end{prb}

\begin{prb}[Goursat theorem]
The $C^1$ condition in the definition of holomorphic functions is necessary to apply the Stokes theorem when we prove the Cauchy theorem.
However, the $C^1$ condition can be dropped and the pointwise complex differentiability is sufficient to check a function is holomorphic.
\end{prb}


\section{Power series}
\begin{prb}[Analyticity of holomorphic functions]
One direction is direct, the other direction requires the Cauchy integral formula.
\end{prb}

\begin{prb}[Laurent series]
\end{prb}


local geometry, branch points?

\begin{prb}[Open mapping theorem]
\end{prb}
Maximum principle
Schwarz lemma

\begin{prb}[Identity theorem]
\end{prb}
inverse function



\section*{Exercises}
\begin{prb}[Wirtinger derivatives]
\end{prb}
\begin{prb}[Branch of logarithm]
on simply connected domain.
and nth root.
\end{prb}
\begin{prb}[Log$r$ on $\C\setminus\{0\}$]
\end{prb}


\section*{Problems}

\begin{enumerate}
\item If a holomorphic function has positive real parts on the open unit disk then $|f'(0)|<2\Re f(0)$.
\item If at least one coefficient in the power series of a holomorphic function at each point is 0 then the function is a polynomial.
\item If a holomorphic function on a domain containing the closed unit disk is injective on the unit circle, then so is on the disk.
\item For a holomorphic function $f$ and every $z_0$ in the domain, there are $z_1\ne z_2$ such that $\frac{f(z_1)-f(z_2)}{z_1-z_2}=f'(z_0)$.
\item Let $f:\Omega\to\C$ be a holomorphic function on a domain. Then, $\bar{f(z)}=f(\bar z)$ if and only if $f(z)\in\R$ for $z\in\Omega\cap\R$.
\item For two linearly independent entire functions, one cannot dominate the other.
\item The uniform limit of injective holomorphic function is either constant or injective.
\item If the set of points in a domain $U\subset\C$ at which a sequence of bounded holomorphic functions converges has a limit point, then it compactly converges.
\item Find all entire functions $f$ satisfying $f(z)^2=f(z^2)$.
\item An entire function maps every unbounded sequence to an unbounded sequence is a polynomial.
\item Let $f$ be a holomorphic function on the open unit disk such that $f(0)=1$ and $f'(0)>2$. Then, there is $z$ such that $|z|<1$ and $f(z)$ is pure imaginary.
\end{enumerate}



\chapter{Singularities}

\section{Classification of singularities}
\begin{prb}[Isolated singularities]
\end{prb}
\begin{prb}[Riemann removable singularity theorem]
\end{prb}
\begin{prb}[Laurent expansion at an isolated singularity]
\end{prb}
\begin{prb}[Casorati-Weierstrass theorem]
\end{prb}
\begin{prb}[Picard's theorems]
\end{prb}
Riemann sphere and meromorphic functions?

\section{Residue theorem}
\begin{prb}

\end{prb}


\begin{prb}[Unit circle substitution]
\[\int_0^{2\pi}\frac{dx}{1+a\cos x}=\frac{2\pi}{\sqrt{1-a^2}},\quad-1<a<1\]
\end{prb}



\section{Contour integrals}


\begin{prb}[Semicircular contour]
We want to justify the following definite integral:
\[\int_0^\infty\frac{\cos x}{x^2+1}\,dx=\frac\pi{2e}.\]
This can be viewed as a special value of the characteristic function of the \emph{Cauchy distribution} in probability theory.
Define $f:\C\setminus\{\pm i\}\to\C$ be such that
\[f(z)=\frac{e^{iz}}{z^2+1},\]
and for $R>0$ let $C$ be a \emph{semicircular contour} defined by
\[\left\{
\begin{alignedat}{2}
C_1&:x\mapsto x,&\quad&x\in[-R,R],\\
C_2&:\theta\mapsto Re^{i\theta},&&\theta\in[0,\pi].
\end{alignedat}
\right.\]
\begin{parts}
\item $\lim_{R\to\infty}\int_Cf(z)\,dz=\frac\pi e$.
\item $\lim_{R\to\infty}\int_{C_1}f(z)\,dz=2\int_0^\infty\frac{\cos x}{x^2+1}\,dx$.
\item $\lim_{R\to\infty}\int_{C_2}f(z)\,dz=0$. More generally, the following holds: Let $h$ be a holomorphic function on a domain containing the arcs $C_2$ for every large $R>0$.
If $h$ vanishes at infinity, then
\[\lim_{R\to\infty}\int_{C_2}e^{iz}h(z)\,dz=0.\]
This is called the \emph{Jordan lemma}.
\end{parts}
\end{prb}
\begin{pf}
(a)
Note that for sufficiently large $R$, the function $f$ has only one pole at $z=i$ in the interior of $C$, which is simple; define
\[g(z)=f(z)(z-i)=\frac{e^{iz}}{z+i}.\]
Then, by the residue theorem, we obtain
\[\int_Cf(z)\,dz=\int_C\frac{g(z)}{z-i}\,dz=2\pi i\cdot g(i)=\frac\pi e\]
for large $R$.

(b)
For the path $C_1$, we have
\[\lim_{R\to\infty}\int_{C_1}f(z)\,dz=\lim_{R\to\infty}\int_{-R}^Rf(x)\,dx=2\int_0^\infty f(x)\,dx\]
by the definition of improper integrals.
Since $f$ is even on $\R$, we get the conclusion.

(c)
Let $M_R=\max_{z\in C_2}|h(z)|$.
Since $\sin\theta\ge\frac2\pi\theta$ for $0\le\theta\le\frac\pi2$, we have
\begin{align*}
|\int_{C_2}e^{iz}h(z)\,dz|
&=|\int_0^\pi e^{iRe^{i\theta}}h(Re^{i\theta})\,iRe^{i\theta}\,d\theta|\\
&\le M_RR\int_0^\pi e^{-R\sin\theta}\,d\theta\\
&=2M_RR\int_0^{\frac\pi2}e^{-R\sin\theta}\,d\theta\\
&\le2M_RR\int_0^{\frac\pi2}e^{-R\frac2\pi\theta}\,d\theta\\\
&=M_R\pi(1-e^{-R}).
\end{align*}
So we are done because $\lim_{R\to\infty}M_R=0$.
\end{pf}

\begin{prb}[Indented contour]
We want to justify the \emph{Dirichlet integral}:
\[\int_0^\infty\frac{\sin x}x\,dx=\frac\pi2.\]
Define $f:\C\setminus\{0\}\to\C$ such that
\[f(z)=\frac{e^{iz}}z,\]
and for $r,R>0$ let $C$ be a \emph{indented contour} defined by
\[\left\{
\begin{alignedat}{2}
C_1&:x\mapsto x,&\quad&x\in[r,R],\\
C_2&:\theta\mapsto Re^{i\theta},&&\theta\in[0,\pi],\\
C_3&:x\mapsto x,&&x\in[-R,-r],\\
C_4&:\theta\mapsto re^{\pi-\theta},&&\theta\in[0,\pi].
\end{alignedat}
\right.\]
The indented contour is effective when $f$ has a simple pole at zero.
\begin{parts}
\item $\lim_{R\to\infty,r\to0}[\int_{C_1}f(z)\,dz+\int_{C_3}f(z)\,dz]=2i\int_0^\infty\frac{\sin x}x\,dx$.
\item $\lim_{R\to\infty}\int_{C_2}f(z)\,dz=0$.
\item $\lim_{r\to0}\int_{C_4}f(z)\,dz=\pi i$.
\end{parts}
\end{prb}
\begin{pf}
(b)
It follows from the Jordan lemma.

(c)
If $f$ has a simple pole at zero, we can conduct a partial fraction decomposition
\[f(z)=\frac cz+h(z)\]
such that $c$ is a constant and $h$ is holomorphic at zero.
Then, it is easy to see
\[\int_{C_4}f(z)\,dz=\int_{C_4}\frac{dz}z+\int_{C_4}\frac{e^{iz}-1}z\,dz\to\pi i+0\]
as $r\to\infty$.
\end{pf}

\begin{prb}[Sector contour]
We want to justify the \emph{Fresnel integral}:
\[\int_0^\infty\cos x^2\,dx=\sqrt{\frac\pi8}\]
Define $f:\C\setminus\{0\}\to\C$ such that
\[f(z)=e^{iz^2},\]
and for $R>0$ let $C$ be a \emph{circular sector contour} defined by
\[\left\{
\begin{alignedat}{2}
C_1&:x\mapsto x,&\quad&x\in[0,R],\\
C_2&:\theta\mapsto Re^{i\theta},&&\theta\in[0,\tfrac\pi4],\\
C_3&:x\mapsto(R-x)e^{\frac\pi4i},&&x\in[0,R].
\end{alignedat}
\right.\]
\begin{parts}
\item
\end{parts}
\end{prb}
\begin{pf}
(b)

\end{pf}

\begin{prb}[Keyhole contour]
the \emph{keyhole contour} or the \emph{Hankel contour}

\[\int_0^\infty\frac{x^{a-1}}{1+x}=\frac\pi{\sin\pi a}\quad(0<a<1),\quad\int_1^\infty\frac{dx}{x\sqrt{x^2-1}}\]
$\log z$ trick
\[\int_0^\infty\frac{dx}{1+x^3}\]
\end{prb}

\begin{prb}[Rectangular contour]
Fourier integral?
\[\int_0^\infty\frac{\sin x}{e^x-1}\,dx,\quad\int_0^\infty\frac{\cos x}{\cosh x}\,dx\]
\end{prb}







\section{Zeros and poles}
\begin{prb}[Argument principle]
\begin{parts}
\item $f$ has either a zero or a pole at $a$ if and only if $f'(z)/f(z)$ has a simple pole at $a$. 
\item We have a partial fraction decomposition
\[\frac{f'(z)}{f(z)}=\frac{\ord_a(f)}{z-a}+\frac{g'(z)}{g(z)},\]
where $g(z):=f(z)/(z-a)^{\ord_a(f)}$ is holomorphic at $a$.
\item
\[\int_C\frac{f'(z)}{f(z)}\,dz=2\pi i(\text{number of zeros}-\text{number of poles}).\]
\item
\[\int_C\frac{f'(z)}{f(z)}h(z)\,dz=2\pi i\sum_a\ord_a(f)h(a).\]
\item Winding number
\end{parts}
\end{prb}


\begin{prb}[Rouch\'e theorem]
Let $f$ be a meromorphic function on $\Omega$.
\begin{parts}
\item
If $h:[0,1]\times\Omega\to\C$ is continuous, then 
\[\int_C\frac{f'(z)}{f(z)}\,dz=\int_C\frac{g'(z)}{g(z)}\,dz.\]
In particular, if $|g(z)|<|f(z)|$ on $z\in C$, then
\[\int_C\frac{f'(z)}{f(z)}\,dz=\int_C\frac{f'(z)+g'(z)}{f(z)+g(z)}\,dz.\]
\end{parts}
\end{prb}





\section*{Exercises}
\begin{prb}[Fundamental theorem of algebra]
proof by the Liouville theorem, and proof by the Rouch\'e theorem.
\end{prb}
\begin{prb}[Computation of Fourier transforms]
Cauchy distribution,
sector and Gaussian integral,
rectangular integral
\end{prb}
\begin{prb}[Laplace transforms]
\end{prb}
\begin{prb}[Gamma function]
Hankel representation
\end{prb}
\begin{prb}[Abel-Plana formula]
\end{prb}

Sokhotski-Plemelj theorem,
Kramers-Konig relations,
Titchmarsh theorem for Hilbert transform,
Phragm\'en-Lindel\"of principle,
Carlson's theorem

\section*{Problems}
\begin{enumerate}
\item We have $\int_0^{2\pi}\frac{d\theta}{1+\cos^2\theta}=\sqrt2\pi$.
\item We have
\[\]
\end{enumerate}




\chapter{Polynomial approximation}
\section{Mittag-Leffler theorem}
\begin{prb}[Compact convergence of holomorphic functions]
\begin{parts}
\item injectivity preservation: Hurwitz theorem
\end{parts}
\end{prb}
\begin{prb}[Principal part]
For a meromorphic function $f$, we say a polynomial $p$ without constant term is a \emph{principal part} of $f$ at $z_0$ if we have a partial fraction decomposition
\[f(z)=p\left(\frac1{z-z_0}\right)+h(z),\]
where $h(z)$ is holomorphic at $z_0$.
It is unique.
pre-assigned principal parts
\end{prb}

\section{Weierstrass factorization theorem}
Infinite product

\section{Runge's approximation}
Mergelyan









\part{Geometric function theory}

\chapter{Conformal mappings}
\section{Riemann sphere and open unit disk}
\begin{prb}[Conformality of holomorphic maps]
\end{prb}
\begin{prb}[M\"obius transform]
generators,
fixed points
\end{prb}
\begin{prb}[Blaschke factors]
\end{prb}

\section{Riemann mapping theorem}


\begin{prb}[Normal family]
locally bounded, then compact (Montel)
\end{prb}
\begin{prb}[Riemann mapping theorem]
Let $\Omega\subset\C$ be a simply connected domain such that $\Omega\ne\C$.
\[\cF=\{f:\Omega\to\D\mid f\text{ is injective and holomorphic, and }f(z_0)=0\}\]
\begin{parts}
\item There exists an injective holomorphic function $f:\Omega\to\D$.
\item If $0\in\Omega_1\subsetneq\D$, then there is a conformal mapping $h:\Omega_1\to\Omega_2$ such that $h(0)=0$ and $|h'(0)|>1$, where $0\in\Omega_2\subset\D$.
\item The supremum of $|f'(0)|$ is attained in $\cF$....
\item There exists a conformal mapping $f:\Omega\to\D$.
\end{parts}
\end{prb}


\section*{Exercises}
\begin{prb}[Special solution of Laplace' equation]
\end{prb}
\begin{prb}[Normal family for meromorphic functions]
\end{prb}

\section*{Problems}
\begin{enumerate}
\item Find a conformal mapping that maps the open unit disk onto $A:=\{\,z\in\C:\max\{|z|,|z-1|\}<1\,\}$.
\end{enumerate}

\chapter{Univalent functions}
\section{Bierbach conjecture}
\section{Harmonic functions}
harmonic conjugates
conformal change






\chapter{}

Maximum principle; Schwarz's lemma, Lindelöf principle,
Nevanlinna theory?

\section{Riemann-Hilbert problem}
Hilbert transform
almost everywhere convergence, Hardy-Littlewood maximal function

\section{Quasi-conformal mappings}
Beltrami equations and Teichm\"uler theory?











\part{Riemann surfaces}

\chapter{Analytic continuation}
\section{Branch cuts}
We can represent $f$ with any coordinate system(usually polar coordinates).

Define $f:\{re^{i\theta}:r>0,-\pi<\theta<\pi\}\to\C$ such that
\[f(re^{i\theta}):=\log r+i\theta.\]
Then, $e^{f(z)}=z$.
Define $f:\{x+iy:y\ne0\text{ or }-1<x<1\}\to\C$ such that
\[f(z):=\frac1{\sqrt{r_+r_-}}e^{i\frac{\theta_++\theta_-}2},\]
where $z-1=r_+e^{i\theta_+}$ and $z+1=r_-e^{i\theta_-}$.
Then, $f(z)$ is a branch of $1/\sqrt{z^2-1}$.
\section{Monodromy}
\section{Covering surfaces}
\section{Algebraic functions}
\section{Elliptic curves}

\chapter{Differential forms}

\chapter{Uniformization theorem}




\part{Several complex variables}


\end{document}