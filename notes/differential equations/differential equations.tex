\documentclass{../note}
\usepackage{../../ikany}


\begin{document}
\title{Differential Equations}
\author{Ikhan Choi}
\maketitle
\tableofcontents

\part{Linear ordinary differential equations}
\chapter{Constant coefficient equations}
\section{Characteristic equations}
\section{Complex roots}
\section{Repeated roots}

\chapter{Variable coefficient equations}
\section{Series solution}
\section{Fuch's theorem}
\section{Orthogonal polynomials}
\section{Sturm-Liouville theory}
\section{The Frobenius method}
Fuch's theorem

\chapter{Inhomogeneous equations}
\section{Method of undetermined coefficients}
\section{Variation of parameters}
\section{Damped oscillation}
\section{The Laplace transform}
discontinuous data gluing








\part{Nonlinear ordinary differential equations}

\chapter{Nonlinear ordinary differential equations}
\section{The Picard-Lindel\"of theorem}
\section{Integrating factors}

\chapter{Dynamical systems}
\section{Equillibria}
Bifurcations\\
Stability theory\\
Hamiltonian systems

\section{Planar dynamical systems}
Examples from ecology, electrical engineerings\\
Poincar\'e-Bendixon


\chapter{Chaos}
Attractors







\part{Linear partial differential equations}


\chapter{Laplace's equation}
\section{Harmonic functions}
\begin{prb}[Mean value property]
\end{prb}
\begin{prb}[Maximum principle]
\end{prb}


\begin{prb}[Newtonian potential]
\end{prb}
\begin{prb}[Dirichlet problem for half space]
\end{prb}
\begin{prb}[Dirichlet problem for open ball]
\end{prb}

\section{Poisson equation}
% How can we introduce the Dirac delta function in this note?
\begin{prb}[Weak derivative]

\end{prb}
\begin{prb}[Dirac delta function]
Let $\Omega$ be an open subset of $\R^d$.
The \emph{Dirac delta function} is a linear functional $\delta:C_c^\infty(\Omega)\to\R$ defined by $\delta(\f):=\f(0)$.
We conventionally use the function-like notation $\delta(x)$ to denote $\f(0)$ by
\[\int\delta(x)\f(x)\,dx.\]


\end{prb}

\begin{prb}[Fundamental solution of the Laplace equation]
Let $d\ge2$.
The \emph{Fundamental solution of the Laplace equation} is a function $\Phi:\R^d\setminus\{0\}\to\R$ that solves the boundary value problem
\[\left\{\begin{alignedat}{2}
-\Delta\Phi(x)&=\delta(x) &\quad&\text{ in }\R^d,\\
\Phi(x)&\to0 &&\text{ as }|x|\to\infty.
\end{alignedat}\right.\]
\begin{parts}
\item The funcdamental solution is given by
\[\Phi(x):=\begin{cases}-\frac1{2\pi}\log|x|&\text{ if }d=2\\\frac1{(d-2)\omega_d}\frac1{|x|^{d-2}}&\text{ if }d\ge3\end{cases}.\]
In particular, $\Phi$ and $\nabla\Phi$ are locally integrable on $\R^d$ but $\nabla^2\Phi$ is not.
\item For $u\in C_0^2(\R^d)$,
\[u(x)=-\int\Phi(x-y)\Delta u(y)\,dy.\]
\end{parts}
\end{prb}
\begin{pf}
Note that $\nabla\Phi(y)\cdot\nabla u(x-y)$ is integrable in $y$.
Then,
\begin{align*}
-\int\Phi(y)\Delta u(x-y)\,dy
&=-\int\nabla\Phi(y)\cdot\nabla u(x-y)\,dy\\
&=-\lim_{\e\to\infty}\int_{|y|\ge\e}\nabla\Phi(y)\cdot\nabla u(x-y)\,dy\\
&=-\lim_{\e\to\infty}\int_{|y|=\e}\nabla\Phi(y)u(x-y)\cdot\nu\,dS.
\end{align*}
Since
\[\nabla\Phi(x)=-\frac1{\omega_d}\frac x{|x|^d},\quad\nu=\frac x{|x|},\]
we get
\[-\int\Phi(y)\Delta u(x-y)\,dy=\lim_{\e\to\infty}\frac1{\omega_d\e^{d-1}}\int_{|y|=\e}u(x-y)\,dS_y=u(x).\]

\end{pf}

\begin{prb}[Green's function of the Poisson equation]
Let $\Omega$ be a bounded open subset of $\R^d$ for $d\ge2$.
\emph{Green's function of the Poisson equation} is a function $G:\Omega^2\setminus\{(x,x)\in\Omega\}\to\R$ that solves the boundary value problem
\[\left\{\begin{alignedat}{2}
-\Delta_yG(x,y)&=\delta(x-y)\quad & \text{ in }&y\in\Omega\setminus\{x\},\\
G(x,y)&=0 & \text{ on }&y\in\partial\Omega.
\end{alignedat}\right.\]
for each $x\in\Omega$.

Define $\phi:\Omega^2\to\R$ to be a function that solves the boundary value problem
\[\left\{\begin{alignedat}{2}
-\Delta_y\phi(x,y)&=0 & \text{ in }&y\in\Omega,\\
\phi(x,y)&=\Phi(x-y)\quad & \text{ on }&y\in\partial\Omega.
\end{alignedat}\right.\]
for each $x\in\Omega$.
Assume for the domain $\Omega$ that there exists a unique $\phi$.
\begin{parts}
\item Green's function is given by
\[G(x,y)=\Phi(x-y)-\phi(x,y),\]
where $\Phi$ is the fundamental solution of the Laplace equation.
Physically, $y\mapsto-\phi(x,y)$ has a meaning of the electric potential generated by the induced surface charge of a grounded conductor provided a point charge is at $x$.
\item The \emph{Green representation formula} holds: for $u\in C^2(\Omega)\cap C(\bar\Omega)$,
\[u(x)=-\int_\Omega G(x,y)\Delta u(y)\,dy-\int_{\partial\Omega}u(y)\nabla_yG(x,y)\cdot\nu\,dS_y.\]
\end{parts}
\end{prb}

\begin{prb}[Existence and uniqueness of Poisson equation]
representation formulas describe the solution assuming 

\end{prb}

\section{Helmholtz equation}






\chapter{Heat equation}
\section{Heat kernel}
\section{Duhamel's principle}
\section{Separation of variables}






\chapter{Wave equation}
\section{First order partial differential equations}
\section{Initial value problems}
d'Alambert\\
Kirchhoff\\
odd reflection

\section{Boundary value problems}







\part{Nonlinear partial differential equations}

\chapter{Fluid dynamics}
\section{Burger's equation}
\section{Euler's equation}
\section{Navier-Stokes equation}

\chapter{Integrable field equations}
\section{Korteweg-de Vries equation}
\section{Boussinesq equation}
\section{Kadomtsev-Petviashvili equation}

sine-Gordon equation
nonlinear Sch\"rodinger equatoin

\chapter{Nonlinear waves and diffusion}
\section{Nonlinear wave equation}
\section{Nonlinear diffusion equation}




\end{document}