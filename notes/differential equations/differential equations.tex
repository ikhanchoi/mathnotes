\documentclass{../note}
\usepackage{../../ikany}


\begin{document}
\title{Differential Equations}
\author{Ikhan Choi}
\maketitle
\tableofcontents

\part{Linear ordinary differential equations}
\chapter{Constant coefficient equations}
\section{Characteristic equations}
\section{Complex roots}
\section{Repeated roots}

\chapter{Variable coefficient equations}
\section{Series solution}
\section{Fuch's theorem}
\section{Orthogonal polynomials}
\section{Sturm-Liouville theory}
\section{The Frobenius method}
Fuch's theorem

\chapter{Inhomogeneous equations}
\section{Method of undetermined coefficients}
\section{Variation of parameters}
\section{Damped oscillation}
\section{The Laplace transform}
discontinuous data gluing








\part{Nonlinear ordinary differential equations}

\chapter{Nonlinear ordinary differential equations}
\section{The Picard-Lindel\"of theorem}
\section{Integrating factors}

\chapter{Dynamical systems}
\section{Equillibria}
Bifurcations\\
Stability theory\\
Hamiltonian systems

\section{Planar dynamical systems}
Examples from ecology, electrical engineerings\\
Poincar\'e-Bendixon


\chapter{Chaos}
Attractors







\part{Linear partial differential equations}


\chapter{Laplace's equation}
\section{Harmonic functions}
\begin{prb}[Mean value property]
\end{prb}
\begin{prb}[Maximum principle]
\end{prb}


\begin{prb}[Newtonian potential]
\end{prb}
\begin{prb}[Dirichlet problem for half space]
\end{prb}
\begin{prb}[Dirichlet problem for open ball]
\end{prb}



\begin{prb}[Fundamental solution of the Laplace equation]
Consider a boundary problem
\[\left\{\begin{alignedat}{2}
-\Delta u(x)&=f(x) &\quad&\text{ in }\R_x^d,\\
u(x)&=0 &&\text{ on }|x|=\infty.
\end{alignedat}\right.\]
A function
\[\Phi(x):=\begin{cases}-\frac1{2\pi}\log|x|&\text{ if }d=2\\\frac1{(d-2)\omega_d}\frac1{|x|^{d-2}}&\text{ if }d\ge3\end{cases}\]
defined on $\R_x^d$ for $d\ge2$ is called \emph{fundamental solution of Laplace's equation}.
\begin{parts}
\item $\Phi$ and $\nabla\Phi$ are locally integrable on $\R_x^d$ but $\Delta\Phi$ is not.
\item $\Delta\Phi$ is a tempered distribution on $\R_x^d$.
\item $-\Delta\Phi(x)=\delta(x)$ in $\R_x^d$.
\item $u$ solves the boundary problem if and only if it satisfies a representation formula $u=\Phi*f$, if $\Phi*f$ is a well-defined distribution on $\R_x^d$.
\end{parts}
\end{prb}
\begin{pf}
(c)
Let $\f\in\cD(\R_x^d)$.
Then, $\nabla\Phi(x)\cdot\nabla\f(x)\in L^1(\R_x^d)$ gives
\begin{align*}
-\int\Phi(x)\Delta\f(x)\,dx
&=-\lim_{\e\to\infty}\int_{|x|\ge\e}\nabla\Phi(x)\cdot\nabla\f(x)\,dx\\
&=-\lim_{\e\to\infty}\int_{|x|=\e}\nabla\Phi(x)\f(x)\cdot\nu\,dS
+\lim_{\e\to\infty}\int_{|x|\ge\e}\Delta\Phi(x)\f(x)\,dx.
\end{align*}
Since
\[\nabla\Phi(x)=-\frac1{\omega_d}\frac x{|x|^d},\quad\nu=\frac x{|x|},\]
and $\Delta\Phi(x)=0$ for $x\ne0$, we get
\[-\int\Phi(x)\Delta\f(x)\,dx=\lim_{\e\to\infty}\frac1{\omega_d\e^{d-1}}\int_{|x|=\e}\f(x)\,dS=\f(x).\]

(d)
Note that $\Phi=\tilde\Phi$.
If $u$ is a solution of the boundary problem, then
\[\<\Phi*f,\f\>=\<f,\Phi*\f\>=\<u,-\Delta(\Phi*\f)\>=\<u,\Phi*(-\Delta\f)\>=\<u,\f\>.\]
Conversely, if we let $u=\Phi*f$, then
\[\<u,-\Delta\f\>=\<\Phi*f,-\Delta\f\>=\<f,\tilde\Phi*(-\Delta\f)\>=\<f,\Phi*(-\Delta\f)\>=\<f,\f\>\]
and
\end{pf}

\begin{prb}[Green's function]
Let $U$ be a bounded open subset of $\R_x^d$ with $C^1$ boundary.
Consider a boundary value problem
\[\left\{\begin{alignedat}{2}
-\Delta u(x)&=f(x) & \text{ in }&U,\\
u(x)&=g(x) & \text{ on }&\partial U.
\end{alignedat}\right.\]
A \emph{corrector} is a function $\phi(x,y)$ on $U\times U$ defined as the solution of the boundary value problem
\[\left\{\begin{alignedat}{2}
-\Delta_y\phi(x,y)&=0 & \text{ in }&y\in U,\\
\phi(x,y)&=\Phi(x-y) & \text{ on }&y\in\partial U,
\end{alignedat}\right.\]
for each $x\in U$.
We assume a well-known fact that the solution $\phi$ uniquely exists and $\phi\in H^1(U)$, proved later.
Then, \emph{Green's function} for $U$ is a function on $U\times U$ defined by
\[G(x,y):=\Phi(x-y)-\phi(x,y).\]
\begin{parts}
\item If $g(x)=0$ on $\partial U$, then for $x\in U$,
\[u(x)=-\int_UG(x,y)\Delta u(y)\,dy.\]
\item If $f(x)=0$ in $U$, then for $x\in U$,
\[u(x)=\int_{\partial U}u(y)\nabla_yG(x,y)\cdot\nu\,dS(y).\]
\item $u$ solves the boundary problem if and only if it satisfies a representation formula
\[u(x)=\int_UG(x,y)f(y)\,dy+\int_{\partial U}g(y)\nabla_yG(x,y)\nu\cdot dS(y),\]
if the right-hand side is well defined distribution on $\R_x^d$.
\end{parts}
\end{prb}
\begin{pf}
\end{pf}


\section{Green's representation formula}







\chapter{Heat equation}
\section{Heat kernel}

\section{Duhamel's principle}







\chapter{Wave equation}
\section{First order partial differential equations}
\section{Initial value problems}
d'Alambert\\
Kirchhoff\\
odd reflection

\section{Boundary value problems}







\part{Nonlinear partial differential equations}

\chapter{Fluid dynamics}
Burger's equation\\
Euler's equation\\
Navier-Stokes equation

\chapter{Integrable field equations}
Korteweg-de Vries equation\\
Boussinesq equation\\
Kadomtsev-Petviashvili equation

sine-Gordon equation
nonlinear Sch\"rodinger equatoin

\chapter{Nonlinear waves and diffusion}
Nonlinear wave equation\\
Nonlinear diffusion equation




\end{document}