\documentclass{../note}
\usepackage{../../ikany}

\newcommand{\wk}{\text{\sc wk}}
\newcommand{\wot}{{\text{\sc wot}}}
\newcommand{\sot}{{\text{\sc sot}}}

\begin{document}
\title{Functional Analysis}
\author{Ikhan Choi}
\maketitle
\tableofcontents

\part{Topological vector spaces}


\chapter{Locally convex spaces}
\section{General vector topologies}
canonical uniformity.
canonical bornology.
metrizability(Birkhoff-Kakutani).
boundedness and continuity

\section{Seminorms and convex sets}
boundedness by seminorms, normability

\section{Continuous linear functionals}
\begin{prb}
Let $\{x_i^*\}_{i=1}^n\subset X^*$.
If $x^*\in X^*$ vanishes on $\bigcap_{i=1}^n\ker x_i^*$, then $x^*$ is a linear combination of $\{x_i^*\}$.
\end{prb}

\begin{prb}[Dual space]
\end{prb}

\begin{prb}[Adjoint operator]
\end{prb}


\section{Hahn-Banach theorem}

\begin{prb}[Hahn-Banach theorem]
\end{prb}








\chapter{Barreled spaces}

\section{Uniform boundedness principle}
\begin{prb}[Barreled spaces]
A \emph{barrel} is an absorbing, balanced, convex, and closed subset of $X$.
A \emph{barreled space} is a topological space in which every barrel is a neighborhood of zero.
\end{prb}

% If a closed convex cone contains a dense subset of absorbing at a point, then it is entire?

\begin{prb}[Uniform boundedness principle]
Let $\cT$ be a set of continuous linear operators from $X$ to $Y$.
Suppose $\bigcup_{T\in\cT}Tx$ is bounded for each $x\in D$, where $D\subset X$.
\begin{parts}
\item If $D$ is dense in $X$, then $\bigcap_{T\in\cT}T^{-1}\bar U$ is absorbing.
\item If $X$ is barreled, then $\cT$ is equicontinuous.
\end{parts}
\end{prb}



\section{Baire category theorem}

\begin{prb}[Baire spaces]
A topological space is called a \emph{Baire space} if the intersection of countable open dense subsets is dense.
\end{prb}


\begin{prb}[Absorbing set]
Let $X$ be a topological vector space that is Baire.
\begin{parts}
\item A closed and absorbing set has non-empty interior.
\item A closed, convex, and absorbing set is a neighborhood of zero.
\end{parts}
\end{prb}


\begin{prb}[The Baire category theorem]
\begin{parts}
\end{parts}
\end{prb}



\section{Open mapping theorem}

\begin{prb}[Open mapping theorem]
Let $X$ be a F-space and $Y$ a barreled space.
Suppose $T:X\to Y$ is continuous and surjective.
\begin{parts}
\item $\bar{TB}$ is a neighborhood of zero.
\item $TB$ is a neighborhood of zero.
\end{parts}
\end{prb}

\begin{pf}
(a)
Let $B=B_1$ be an open ball in $X$.
There is an open neighborhood $U$ of zero such that $U-U\subset B$.
The set $\bar{TU}$ is clearly closed, and the surjectivity of $T$ implies $\bar{TU}$ is absorbing.
Since $Y$ is barreled, $\bar{TU}$ has a non-empty interior in $Y$.
Thus, $\bar{TB}$ is a neighborhood of zero.

(b)
We claim $\bar{TB_{1/2}}\subset TB$.
Take $y_1\in\bar{TB_{1/2}}$.
To construct $x\in B$ such that $Tx=y_1$, we use the metrizability and completeness of $X$.
Since $\bar{TB_{1/2^{n+1}}}$ are neighborhoods of zero, we can inductively construct sequences $x_n\in B_{1/2^n}$ and $y_n\in\bar{TB_{1/2^n}}$ such that $Tx_n\in y_n+\bar{TB_{1/2^{n+1}}}$ and $y_{n+1}:=Tx_n-y_n$.
Let $x:=\sum_{n=1}^\infty x_n\in B$.
Then,
\[Tx=\lim_{n\to\infty}\sum_{i=1}^nTx_i=\lim_{n\to\infty}\sum_{i=1}^ny_{i+1}-y_i=y_1.\qedhere\]
\end{pf}


\section*{Exercises}

\begin{prb}
Let $(T_n)$ be a sequence in $B(X,Y)$.
If $T_n$ coverges strongly then $\|T_n\|$ is bounded by the uniform boundedness principle.
\end{prb}

\begin{prb}
There is a closed absorbing set in $\ell^2(\Z_{\ge0})$ that is not a neighborhood of zero;
\[\bar B(0,1)\setminus\bigcup_{i=2}^\infty B(i^{-1}e_i,i^{-2})\]
is a counterexample.
\end{prb}



\chapter{Fr\'echet, Banach, and Hilbert spaces}

\section{Fr\'echet spaces}
dual is not Fr\'echet.

\section{Banach spaces}
dual is Banach.
Basis problem, Mazur' duck.

\section{Hilbert spaces}
Projections. Reducing subspaces.
Hilbert space classification by cardinal.
Riesz representation theorem.
\begin{prb}
\begin{parts}
\item A Banach space $X$ is isometrically isomorphic to a Hilbert space if there is a bounded linear projection on every closed subspace of $X$.
\end{parts}
\end{prb}

\begin{prb}[Riesz representation theorem]
Let $H$ be a Hilbert space over a field $\F$, which is either $\R$ of $\C$.


We use the bilinear form $\<-,-\>:X\times X^*\to\F$ of canonical duality.
\emph{Dirac} notation $\<-|-\>$ for the inner product of a complex Hilbert spaces such that $\<x,y\>=\<y|x\>$.
The Riesz representation theorem states that a continuous linear functional on a Hilbert space is represented by the inner product with a vector.
\begin{parts}
\item For each $x^*\in H^*$, there is a unique $x\in H$ such that $\<y,x^*\>=\<y,x\>$ for every $y\in H$.
\item $H\to H^*:x\mapsto\<-,x\>$ is a natural linear and anti-linear isomorphism if $\F=\R$ and $\C$, respectively.
\end{parts}
\end{prb}



\section{Bounded linear operators}
\begin{prb}[Bounded belowness in Banach spaces]
Let $T\in B(X,Y)$ for Banach spaces $X$ and $Y$.
The following statements are equivalent:
\begin{parts}
\item $T$ is bounded below.
\item $T$ is injective and has closed range.
\item $T$ is a topological isomorphism onto its image.
\end{parts}
\end{prb}

\begin{prb}[Bounded belowness in Hilbert spaces]
Let $T\in B(H,K)$ for Hilbert spaces $H$ and $K$.
The following statements are equivalent:
\begin{parts}
\item $T$ is bounded below.
\item $T$ is left invertible.
\item $T^*$ is right invertible.
\item $T^*T$ is invertible.
\end{parts}
\end{prb}

\begin{prb}[Injectivity and surjectivity of adjoint]
Let $T\in B(X,Y)$ for Banach spaces $X$ and $Y$.
\begin{parts}
\item $T^*$ is injective if and only if $T$ has dense range.
\item $T^*$ is surjective if and only if $T$ is bounded below.
\end{parts}
\end{prb}

\begin{prb}[Normal operators]
For $T\in B(H)$, we have an obvious fact $(\im T)^\perp=\ker T^*$.
Suppose $T$ is normal.
\begin{parts}
\item $\ker T=\ker T^*$.
\item $T$ is bounded below if and only if $T$ is invertible.
\item If $T$ is surjective, then $T$ is invertible.
\end{parts}
\end{prb}

\begin{prb}[Invariant and Reducing subsapces]
Let $K$ be a closed subspace of $H$.
\begin{parts}
\item $K$ is reducing for $T$ if and only if $K$ is invariant for $T$ and $T^*$.
\item $K$ is reducing for $T$ if and only if $TP=PT$, where $P$ is the orthogonal projection on $K$.
\end{parts}
\end{prb}
% self adjoint operators
% invariant but not reducing for unitary operators
% eigenspaces
% matrix representation




\section*{Exercises}
\begin{prb}
There is no metric $d$ on $C([0,1])$ such that $d(f_n,f)\to0$ if and only if $f_n\to f$ pointwise as $n\to\infty$ for every sequence $f_n$.
Note that this problem is slightly different to the non-metrizability of the topology of pointwise convergence.
\end{prb}

\begin{prb}
Let $T$ be an invertible linear operator on a normed space.
Then, $T^{-2}+\|T\|^{-2}$ is injective if it is surjective.
\end{prb}

\begin{prb}
We show that there is no projection from $\ell^\infty$ onto $c_0$.
\end{prb}

\begin{prb}[Schur's property of $\ell^1$]
\end{prb}

\begin{prb}
Let $\f:L^\infty([0,1])\to\ell^\infty(\N)$ be an isometric isomorphism.
Suppose $\f$ is realised as a sequence of bounded linear functionals on $L^\infty$.
\begin{parts}
\item
Show that $\f^*(\ell^1)\subset L^1$ where $\ell^1$ and $L^1$ are considered as closed linear subspaces of $(\ell^\infty)^*$ and $(L^\infty)^*$ respectively.
\item Show that $\f^*$ is indeed an isometric isomorphism, and deduce $\f$ cannot be realised as bounded linear functionals on $L^\infty$.
\end{parts}
\end{prb}










\part{Weak topologies}
\chapter{Dual space of Banach spaces}
\section{Weak and weak$^*$ topologies}
boundedness, incompleteness


\begin{prb}[Weak convergence by dense set]
Let $X$ be a Banach space, $D$ a subset of $X^*$, and $\bar D$ the norm closure of $D$.
For example, if $X$ has a predual $X_*\subset X^*$ and $D$ is dense in $X_*$, then $\sigma(X,\bar D)$ is the weak$^*$ topology.
\begin{parts}
\item There is a squence $x_n\in X$ converges to zero in $\sigma(X,D)$ but not in $\sigma(X,\bar D)$.
\item A sequence $x_n\in X$ converges to zero in $\sigma(X,\bar D)$ if in $\sigma(X,D)$, if $\|x_n\|\le1$.
\end{parts}
\end{prb}
\begin{pf}
(b)
Let $x^*\in\bar D$ and choose $y^*\in D$ such that $\|x^*-y^*\|<\e$
Then,
\[|\<x_n,x^*\>|\le\|x_n\|\|x^*-y^*\|+|\<x_n,y^*\>|.\]
\end{pf}





\section{Weak compactness}
\begin{prb}[Banach-Alaoglu theorem]
\end{prb}
\begin{prb}[Eberlein-\v Smulian theorem]
\end{prb}
\begin{prb}[James' theorem]
\end{prb}

\section{Weak density}
Bishop-Phelps theorem
\begin{prb}[Goldstine's theorem]
Let $X$ be a Banach space and $J:X\to X^{**}$ the canonical embedding.
Let $\{x_i^*\}_{i=1}^m\subset X^*$ and $x^{**}\in X^{**}$.
\begin{parts}
\item There is $x\in X$ such that $\<x_i^*,J(x)\>=\<x_i^*,x^{**}\>$ for all $i$.
\item If $\|x^{**}\|\le1$, then there is $x\in X$ such that $\|x\|\le1+\e$ and $\<x_i^*,J(x)\>=\<x_i^*,x^{**}\>$ for all $i$, for any $\e>0$ .
\item $J(\bar B_X)$ is weak$^*$-dense in $\bar B_{X^{**}}$
\end{parts}
\end{prb}
\begin{pf}
(b)
Let $z\in X$ such that $\<x_i^*,J(x)\>=\<x_i^*,x^{**}\>$ for all $i$.
Let $Y$ be the set of all $y\in X$ such that $\<x_i^*,J(y)\>=0$ for all $i$.
Then, $z+Y$ is the closed affine subsapce of $X$ containing all $y\in X$ such that $\<x_i^*,J(y)\>=\<x_i^*,x^{**}\>$ for all $i$.
If we assume $z+Y$ does not contain any $x\in X$ such that $\|x\|\le1+\e$, then $\dist(z,Y)=\dist(0,z+Y)>1+\e$.
By the Hahn-Banach theorem, there is $y^*\in X^*$ such that $\|y^*\|=1$, $y^*|_Y=0$, and $\<z,y^*\>>1+\e$.
Then, $y^*$ is a linear combination of $\{x_i^*\}_{i=1}^m$, so
\[1+\e<\<z,y^*\>=\<y^*,J(z)\>=\<y^*,x^{**}\>\le\|x^{**}\|\|y^*\|\le1.\]

(c)
Fix $x^{**}\in X^{**}$ such that $\|x^{**}\|\le1$ and let
\[U=\bigcap_{i=1}^m\,\{\,y^{**}\in X^{**}:|\<x_i^*,y^{**}-x^{**}\>|<1\,\}\]
be an open weak$^*$-neighborhood of $x^{**}$.
Choose $\e>0$ such that
\[\e\max_{1\le i\le m}\|x_i^*\|<1.\]
By the part (b), there is $x\in X$ such that $\|x\|\le1+\e$ and $\<x_i^*,x^{**}\>=\<x_i^*,J(x)\>$.
If we let $y:=(1+\e)^{-1}x$, then $\|y\|\le 1$ so that
\[|\<x_i^*,J(y)-x^{**}\>|=|\<x_i^*,J(y)-J(x)\>|=|\<x_i^*,\e J(y)\>|\le\e\|x_i^*\|\|y\|<1\]
for all $i$ implies $J(y)\in U$, hence we get $J(\bar B_X)\cap U\ne\varnothing$.
\end{pf}


\section{Krein-Milman theorem}
Choquet theory



\section*{Exercises}
\begin{prb}[James' space]
not reflexive but isometrically isomorphic to bidual
\end{prb}


\begin{prb}[Predual correspondence]
Let $X$ be a Banach space.
Let
\[\{\,(Y,\f)\mid\f:X\to Y^*\text{ is an isometric isorphism}\,\}\]
and
\[\{\,Z\le X^*\mid\bar{B_X}\text{ is compact Hausdorff in }(X,\sigma(X,Z))\,\}.\]

\[(Y,\f)\mapsto\im\f^*|_{J(Y)}\]

\begin{parts}
\item The map is well-defined.
\item The map is surjective. (by Goldstein)
\item The map is injective up to isomorphism for $Y$.
\end{parts}
\end{prb}

\begin{prb}
Let $X$ be a closed subspace of a Banach space $Y$ and \[i:X\to Y\] the inclusion.
Suppose $X$ and $Y$ have preduals $X_*$ and $Y_*$ respectively.
Let \[j:=i^*|_{Y_*}:Y_*\to Z\subset X^*,\]
where $Z:=i^*(Y_*)^-$.
Then we can show
\[j^*:Z^*\subset X^{**}\to Y\]
coincides with $i$ on $X\cap Z^*$.
From the existence of $X_*$ we have $X^{**}\to X$, which is restricted to define a map $k:Z^*\to X$.
\begin{cd}
&X\ar{r}{i}&Y\\
X^{**}\ar{ur}\ar{r}&Z^*\ar{u}{k}\ar{ur}{j}&
\end{cd}
We can show $k$ is an isomorphism so that we have
\[X_*\cong Y_*/Y_*\cap\ker(i^*).\]
\end{prb}

\begin{prb}[Mazur's lemma]

\end{prb}

\begin{prb}[Dunford-Pettis property]

\end{prb}












\chapter{Polar topologies}
\section{Dual pair}
\section{Strong topologies}
Mackey-Arens


\chapter{Operator topologies}
\begin{prb}[Compact left multiplications and SOT]
Let $T_n$ be a sequence of bounded linear operators on a Hilbert space that converges in SOT.
For compact $K$, $T_n K$ converges in norm, but $KT_n$ generally does not unless $T$ is self-adjoint.
\end{prb}

\begin{prb}
Let $f$ be a linear functional on $B(H)$ for a Hilbert space $H$.
Then, TFAE:
\begin{parts}
\item $f$ is $\wot$-continuous,
\item $f$ is $\sot$-continuous,
\item $f(T)=\sum_{i=1}^n\<Tx_i,y_i\>$ for some $x_i,y_i$.
\end{parts}
\end{prb}
\begin{pf}
(2)$\impl$(3) is the only nontrivial implication.
By the definition of $\sot$, there exists $v\in \cH^n$ such that
\[|f(T)|\le\|T^{\oplus n}v\|.\]
The functional $f:\cA\to\C$ factors through $\cH^n$ such that
\[\cA\to{v}\cH^n\to\C.\]
\end{pf}












\part{Spectral theory}


\chapter{Compact operators}

$K(X,Y)$ is closed in $B(X,Y)$.
$K(X)$ is an ideal of $B(X)$.
adjoint is $K(X,Y)\to K(Y^*,X^*)$.
integral operators are compact.
riesz operator, quasi-nilpotent operator.

\section{Finite-rank operators}
\section{Fredholm operators}

\begin{prb}
A bounded linear operator $T:X\to Y$ between Banach spaces is called a \emph{Fredholm} operator if its kernel is finite dimensional and its range is finite codimensional.
\begin{parts}
\item A Fredholm operator $T$ has closed range.
\end{parts}
\end{prb}
\begin{pf}
(a)
Let $C$ be a finite dimensional subsapce of $Y$ such that $\im T\oplus C=Y$.
Let $\tilde T:X/\ker T\to Y$ be the induced operator of $T$.
Define $S:(X/\ker T)\oplus C\to Y$ such that $S(x+\ker T,c):=\tilde T(x+\ker T)+c$.
Then, $S$ is an topological isomorhpism between Banach spaces by the open mapping theorem, so $S(X/\ker T\oplus\{0\})=\im\tilde T=\im T$ is closed.
\end{pf}

\begin{prb}[Atkinson's theorem]
An operator $T\in B(X,Y)$ is Fredholm if and only if there is $S\in B(Y,X)$ such that $TS-I$ and $ST-I$ is finite rank.
\end{prb}

\begin{prb}[Fredholm index]
locally constant, in particular, continuous.
composition makes the addition of indices.
\end{prb}

\section{Nuclear operators}
tensor products


\section*{Exercises}
\begin{prb}
If $T:L^2([0,1])\to L^2([0,1])$ is a compact operator, then for any $\e>0$ there is a constant $C_\e>0$ such that
\[\|Tf\|_{L^2}\lesssim\e\|f\|_{L^2}+C_\e\|f\|_{L^1}.\]
\end{prb}
\begin{pf}
Suppose there is $\e>0$ such that we have sequence $f_n\in L^2$ satisfying $\|f_n\|_2=1$ and
\[\|Tf_n\|_2>\e+n\|f_n\|_1.\]
By the compactness of $T$, there is a subsequence $Tf_{n_k}$ converges to $g\ne0$ in $L^2$.
Then, $\|f_{n_k}\|_1\to0$ implies $f_{n_k}\to0$ weakly in $L^2$, hence also for $Tf_{n_k}$.
It means $g=0$, which contradicts to the assumption.
\end{pf}











\chapter{Normal operators}
\section{Spectral theorem for compact normal operators}
There is an orthonormal basis $E\subset H$ such that
\[T=\sum_{e\in E}\lambda_e|e\>\<e|.\]

\section{Spectral theorem for bounded normal operators}
\[T=\int_{\sigma(T)}\lambda\,dE(\lambda).\]



\chapter{Unbounded operators}



% point spectrum, approximate point spectrum
Kato-Rellich theorem











\part{Operator algebras}
\chapter{Banach algebras}

\section{Spectral theory of unital Banach algebras}
% every algebra is associative and over \C

\begin{prb}[Unital Banach algebras]
\begin{parts}
\item If $\|a\|<1$, then $1-a$ is invertible. So $\cA^\times$ is open.
\item $\cA^\times\to\cA:a\mapsto a^{-1}$ is differentiable.
\item $\C\setminus\sigma(a)\to\cA:\lambda\mapsto(\lambda-a)^{-1}$ is differentiable.
\end{parts}
\end{prb}

\begin{prb}[Vector-valued complex function theory]
Let $\Omega$ be an open subset of $\C$ and $X$ a Banach space.
For a vector-valued function $f:\Omega\to X$, we say $f$ is \emph{differentiable} if the limit
\[\lim_{\lambda\to\lambda_0}\mu^{-1}(f(\lambda)-f(\lambda_0))\]
exists in $X$, and \emph{weakly differentiable} if the limit
\[\lim_{\lambda\to\lambda_0}\mu^{-1}\<f(\lambda)-f(\lambda_0),x^*\>\]
exists in $\C$ for each $x^*\in X^*$.
Then, the followings are all equivalent.
\begin{parts}
\item $f$ is differentiable.
\item $f$ is weakly differentiable.
\item For each $\lambda_0\in\Omega$, there is a sequence $(x_k)_{k=0}^\infty$ such that the power series
\[\sum_{k=0}^\infty(\lambda-\lambda_0)^kx_k\]
converges to $f(\lambda)$ absolutely and uniformly on any closed ball $\bar{B(\lambda_0,r)}\subset\Omega$.
\end{parts}
\end{prb}


\begin{prb}[Gelfand-Mazur]
$\sigma(a)$ is non-empty.
In particular, if $\cA^\times=\cA\setminus\{0\}$, then $\cA\cong\C$.
\end{prb}

\begin{prb}[Beurling]
\[r(a)=\inf_{n\ge1}\|a^n\|^{1/n}=\lim_{n\to\infty}\|a^n\|^{1/n}\le\|a\|.\]
\end{prb}
\begin{pf}
Let $\lambda\in\C$ such that $|\lambda|<r(a)^{-1}$.
Then we have $\lambda^{-1}\notin\sigma(a)$ so that $1-\lambda a=\lambda(\lambda^{-1}-a)$ is invertible.


Then, $1-\lambda a=\sum_{i=0}^\infty(\lambda a)^i$.


If $|\lambda|<\|a\|^{-1}\le r(a)^{-1}$, then the inverse of $1-\lambda a$ is given by the power series.
If $|\lambda|<r(a)^{-1}$, then we can only deduce the invertibility of $1-\lambda a$.
Complex function theory let us to write the inverse even if we have only $|\lambda|<r(a)^{-1}$.
Also, the radius of convergence is exactly $r(a)^{-1}$.
\end{pf}

\begin{prb}[Spectrum in closed subalgebras]
For fixed element, smaller the ambient algebra, less ``holes'' in the spectrum.
Let $\cB\subset\cA$ be a closed subalgebra containing $1_\cA$.
Note that $\cB$ may be unital even for $1_\cA\notin\cB$.
\begin{parts}
\item $\cB^\times$ is clopen in $\cA^\times\cap\cB$.
\end{parts}
\end{prb}





\section{Ideals}
\begin{prb}[Ideals]
\begin{parts}
\item If $I$ is a left ideal, then $\cA/I$ is a left $\cA$-module.
\end{parts}
\end{prb}

\begin{prb}[Modular left ideals]
A left ideal $I$ is called \emph{modular} if there is $e\in\cA$ such that $a-ae\in I$ for all $a\in\cA$.
The element $e$ is called a \emph{right modular unit} for $I$.
\begin{parts}
\item $I$ is modular if and only if $\cA/I$ is unital(?).
\item A proper modular left ideal is contained in a maximal left ideal.
\item $I$ is a maximal modular left ideal if and only if $I$ is a modular maximal left ideal.
\item There is a non-modular maximal ideal in the disk algebra.
\end{parts}
\end{prb}

\begin{prb}[Closed ideals]
\begin{parts}
\item closure of proper left ideal is proper left.
\item maximal modular left ideal is closed.
\end{parts}
\end{prb}


\begin{prb}[Unitization]
Let $\cA$ be an algebra.
Recall that we always assume algebras are associative.
Consider an embedding $\cA\to B(\cA):a\mapsto L_a$, where $L_a(b)=ab$.
Define
\[\tilde\cA:=\{\,L_a+\lambda\id_{B(\cA)}:a\in\cA,\lambda\in\C\,\}.\]
Note that this construction is available even for unital $\cA$.
\begin{parts}
\item If $\cA$ is normed, then $\tilde\cA$ is a normed algebra such that there is an isometric embedding $\cA\to\tilde\cA$.
\item If $\cA$ is Banach, then $\tilde\cA$ is a Banach algebra.
\item $\cA\oplus\C$ is topologically isomorphic to $\tilde\cA$ as normed spaces.
\end{parts}
\end{prb}
\begin{pf}
(a)
The space of bounded operators $B(\cA)$ is a normd algebra.
Then, $\tilde\cA$ is a normed $*$-algebra with induced norm
\[\|L_a+\lambda\id_{B(\cA)}\|=\sup_{b\in\cA}\frac{\|ab+\lambda b\|}{\|b\|}\]
Then, $\cA$ is a normed $*$-subalgebra of $\tilde\cA$ because the norm and involution of $\cA$ agree with $\tilde\cA$.

(b)
Suppose $(x_n,\lambda_n)$ is Cauchy in $\tilde\cA$.
Since $\cA$ is complete so that it is closed in $\tilde\cA$, we can induce a norm on the quotient $\tilde\cA/\cA$ so that the canonical projection is (uniformly) continuous so that $\lambda_n$ is Cauchy.
Also, the inequality $\|x\|\le\|(x,\lambda)\|+|\lambda|$ shows that $x_n$ is Cauchy in $\cA$.

Since a finite dimensional normed space is always Banach and $\cA$ is Banach, $\lambda_n$ and $x_n$ converge.
Finally, the inequality $\|(x,\lambda)\|\le\|x\|+|\lambda|$ implies that $(x_n,\lambda_n)$ converges.

(c)
Check the topology on $\cA\oplus\C$ in detail...
\end{pf}



unitization, homomorphisms, category(direct sum, product, etc.)

$B(\C^n)$ is simple, but $B(X)$ is not simple.

% approximate identity, norm of left multiplication



\section{Gelfand theory of commutative Banach algebras}

also important spectrum for non-unital banach algebras
Banach algebra of single generator
semisimplicity and symmetricity

\begin{prb}[Character space]
Let $\cA$ be a commutative Banach algbera.
A \emph{character} of $\cA$ is a non-zero homomorphism $\f:\cA\to\C$.
Denote by $\sigma(\cA)$ the set of all characters of $\cA$.
We will show that all characters are bounded.
Then, endow with the weak$^*$ topology on $\sigma(\cA)$ from the inclusion $\sigma(\cA)\subset\cA^*$.
We call this space as the \emph{character space} or the \emph{spectrum} of $\cA$.
Let $\f\in\sigma(\cA)$.
\begin{parts}
\item $\|\f\|=1$.
\item If $\cA$ is unital, then $\sigma(\cA)$ is compact and Hausdorff.
\item Even if $\cA$ is non-unital, $\sigma(\cA)$ is locally compact and Hausdorff.
\end{parts}
\end{prb}


\begin{prb}[Gelfan-Naimark representation]
Let $\cA$ be a commutative Banach algebra.
\[\Gamma:\cA\to C_0(\sigma(\cA)).\]
\begin{parts}
\item $\Gamma(\cA)$ separates points.
\item $\Gamma$ has closed range if
\item $\Gamma$ is injective if
\item $\Gamma$ is isometric if $r(a)=\|a\|$ for all $a\in\cA$.
\end{parts}
\end{prb}

\section{Holomorphic functional calculus}

Dunford-Reisz functional calculus



\section*{Exercises}
\begin{prb}
Let $\cA$ be a unital algebra.
\begin{parts}
\item $\sigma(ab)\setminus\{0\}=\sigma(ba)\setminus\{0\}$.
\item If $\sigma(a)$ is non-empty, then $\sigma(p(a))=p(\sigma(a))$.
\end{parts}
\end{prb}
\begin{pf}
(a)
Intuitively, the inverse of $1-ab$ is $c=1+ab+abab+\cdots$.
Then, $1+bca=1+ba+baba+\cdots$ is the inverse of $1-ba$.
\end{pf}

$C_b(\Omega)$ $\ell^\infty(S)$ $L^\infty(\Omega)$ $B_b(\Omega)$ $A(\D)$
$B(X)$

\begin{prb}
In $C(\R)$, the modular ideals correspond to compact sets.
\end{prb}

\begin{prb}[Disk algebra]
\begin{parts}
\item Every continuous homomorphism is an evaluation.
\end{parts}
\end{prb}

\begin{prb}[Polynomial convexity]
(conway)
\end{prb}

\begin{prb}[Inclusion relation on spectra]
\begin{parts}
\item $\sigma(a+b)\subset\sigma(a)+\sigma(b)$ and $\sigma(ab)\subset\sigma(a)\sigma(b)$ for unital cases.
\item $\sigma(a^{-1})=\sigma(a)^{-1}$ for unital cases.
\item $r(a)^n=r(a^n)$.
\end{parts}
\end{prb}

spectral radius is upper semi-continuous











\chapter{C$^*$-algebras}

\section{C$^*$ identity}
% normal elements, real/imaginary part
% *-homomorphism


\begin{prb}[C$^*$ identity]
% history
A normed $*$-algebra $\cA$ is called a \emph{C$^*$-algebra} if
\begin{parts}
\item $\cA$ is Banach,
\item $\cA$ satisfies the C$^*$-identity: $\|x^*x\|=\|x\|^2$.
\end{parts}
\end{prb}


\begin{prb}[Unitization of C$^*$-algebras]
\[(L_a+\lambda\id_{B(\cA)})^*=L_{a^*}+\bar\lambda\id_{B(\cA)}.\]
\end{prb}
\begin{pf}
The C$^*$-identity easily follows from the following inequality:
\begin{align*}
\|(x,\lambda)\|^2&=\sup_{\|y\|=1}\|xy+\lambda y\|^2\\
&=\sup_{\|y\|=1}\|(xy+\lambda y)^*(xy+\lambda y)\|\\
&=\sup_{\|y\|=1}\|y^*((x^*x+\lambda x^*+\bar\lambda x)y+|\lambda|^2y)\|\\
&\le\sup_{\|y\|=1}\|(x^*x+\lambda x^*+\bar\lambda x)y+|\lambda|^2y\|\\
&=\|(x,\lambda)^*(x,\lambda)\|.\qedhere
\end{align*}
\end{pf}


\begin{prb}[Spectra of normal elements]
Let $\cA$ be a C$^*$-algebra.
\begin{parts}
\item If $a\in\cA$ is unitary, then $\sigma(a)\subset\T$.
\item If $a\in\cA$ is self-adjoint, then $\sigma(a)\subset\R$.
\end{parts}
\end{prb}
\begin{pf}
(a) (b)
By the holomorphic functional calculus,
\[e^{itx}=\sum_{n=1}^\infty\frac{(itx)^n}{n!}.\]
Since the involution is continuous,
\[(e^{itx})^*=\sum_{n=1}^\infty\frac{(-itx)^n}{n!}=e^{-itx},\]
so we have $\|e^{itx}\|^2=\|e^{itx}e^{-itx}\|=1$.
Then, the inequality
\[1=\|e^{itx}\|\ge|h(e^{itx})|=|e^{ith(x)}|=e^{-t\Im h(x)}\]
proves $h(x)\in\R$.

\end{pf}



\section{Continuous functional calculus}

\begin{prb}[Gelfand-Naimark representation for C$^*$-algebras]
For a commutative unital C$^*$-algebra $\cA$, consider the Gelfand transform $\Gamma:\cA\to C(\sigma(\cA))$.
\begin{parts}
\item $\Gamma$ is a $*$-homomorphism.
\item $\Gamma$ is an isometry.
\item $\Gamma$ is a $*$-isomorphism.
\end{parts}
\end{prb}
\begin{pf}
(a)

(b)
Note that we have
\[\|\hat x\|=\sup_{h\in\sigma(\cA)}|\hat x(h)|=\sup_{h\in\sigma(\cA)}|h(x)|=r(x).\]
For self adjoint $x\in\cA$, since we have $\|x\|^2=\|x^*x\|=\|x^2\|$, the spectral radius coincides with the norm by the Gelfand formula for spectral radius in Banach algebras:
\[r(x)=\lim_{n\to\infty}\|x^{2^n}\|^{1/2^n}=\|x\|.\]
Hence
\[\|x\|^2=\|x^*x\|=\|\hat{x^*x}\|=\|\hat{x}^*\hat{x}\|=\|\hat{x}\|\]
for arbitrary $x\in\cA$.


$\Gamma(\cA)$ is a unital $*$-subalgebra of $C(\sigma(\cA))$, and it separates points by definition.
By the Stone-Weierstrass theorem, $\Gamma(\cA)$ is dense in $C(\sigma(\cA))$.
The step 2 shows that $\Gamma(\cA)$ is complete and hence closed so that $\Gamma(\cA)=C(\sigma(\cA)$.
\end{pf}



\begin{prb}[Finitely generated C$^*$-algebras]
joint spectrum.
\end{prb}


\begin{prb}[Continuous functional calculus]
1. $\id\mapsto a$, 2. $(f+g)(a)=f(a)+g(a)$, $(fg)(a)$, 3. $(f\circ g)(a)=f(g(a))$.
\end{prb}


We have shown unitary element has spectrum in the circle, and self-adjoint element has spectrum in real line. The converses of these two statements also hold if we assume $a$ is normal.




\section{Positive linear functionals}


\begin{prb}

\begin{parts}
\item If $a,b\ge0$, then $a+b\ge0$.
\item If $a^*a\le0$, then $a^*a=0$.
\item $a^*a\ge0$ for all $a\in\cA$.
\end{parts}
\end{prb}


\begin{prb}[Operator monotone functions]
\begin{parts}
\item inverse
\item conjugation
\end{parts}
\end{prb}



\begin{prb}[Operator monotonicity of square and commitativity]
Let $\cA$ be a C$^*$-algebra in which the square function is operator monotone, that is, $0\le a\le b$ implies $a^2\le b^2$ for any positive elements $a$ and $b$ in $\cA$.
We are going to show that $\cA$ is necessarily commutative.
Let $a$ and $b$ denote arbitrary positive elements of $\cA$.
\begin{parts}
\item
Show that $ab+ba\ge0$.
\item
Let $ab=c+id$ where $c$ and $d$ are self adjoints.
Show that $d^2\le c^2$.
\item
Suppose $\lambda>0$ satisfies $\lambda d^2\le c^2$.
Show that $c^2d^2+d^2c^2-2\lambda d^4\ge0$.
\item
Show that $\lambda(cd+dc)^2\le(c^2-d^2)^2$.
\item
Show that $\sqrt{\lambda^2+2\lambda-1}\cdot d^2\le c^2$ and deduce $d=0$.
\item
Extend the result for general exponent: $\cA$ is commitative if $f(x)=x^\beta$ is operator monotone for $\beta>1$.
\end{parts}
\end{prb}


\begin{prb}[Injective *-homomorphism is an isometry]ß
% https://math.stackexchange.com/questions/434706/sufficient-condition-for-a-homomorphism-between-c-algebras-being-isometric/435105#435105
\end{prb}


\section{Representation theory}


\section{Gelfand-Naimark-Siegel representation}


\begin{prb}[States on unitization]
Let $\cA$ and $\tilde\cA\cong\cA\oplus\C$ be a C$^*$-algebra and its unitization respectively.
Let $\tilde\rho=\rho\oplus\lambda$ be a bounded linear functional on $\tilde\cA$, where $\rho\in\cA^*$ and $\lambda\in\C^*=\C$.
\begin{parts}
\item $\tilde\rho$ is positive if and only if $\lambda\ge0$ and $0\le\rho\le\lambda$.
\item $\tilde\rho$ is a state if and only if $\lambda=1$ and $\rho$ is positive with $\|\rho\|\le1$.
\item $\tilde\rho$ is a pure state if and only if $\lambda=1$ and $\rho$ is either a pure state or zero.
\end{parts}
\end{prb}


\section*{Exercises}

\begin{prb}
A C$^*$-algebra is commutative if and only if a function $f(x)=\frac x{1+x}$ is operator subadditive.
%L\"owner-Heinz inequality
\end{prb}





\chapter{Von Neumann algebras}
\section{The double commutant theorem}

\begin{thm}[Double commutant theorem]
Let $\cA$ be a non-degenerate C$^*$-subalgebra of $B(H)$.
\begin{parts}
\item $\cA'$ and $\cA''$ are weakly closed.
\item For $a\in\cA''$ and $\xi\in H$, there is a sequence $a_n\in\cA$ such that $a_n(\xi)\to a(\xi)$.
\item For $a\in\cA''$ and $\xi_1,\cdots,\xi_m\in H$, there is a sequence $a_n\in\cA$ such that $a_n(\xi_i)\to a(\xi_i)$ for all $i$.
\item $\cA$ is von Neumann algebra if and only if $\cA=\cA''$.
\end{parts}
\end{thm}
\begin{pf}
(b)
Let $K:=\bar{\cA\xi}$ be the cyclic subspace of $\xi$ in $H$ and $p$ its orthogonal projection.
We claim $a\xi\in K$.
For every $b\in\cA$, we have $bK\subset K$ because the multiplication by $b$ is continuous on $H$, and $b^*K\subset K$ because $\cA$ is self-adjoint.
It means that $K$ reduces all $b\in\cA$, and then $bp=pb$ implies $ap=pa$, so $K$ also reduces $a$.
Therefore, $aK\subset K$ proves $a\xi=\lim_\alpha e_\alpha a\xi\in K$, where $e_\alpha$ is an approximate identity of $\cA$.

(e)
Since $\bar\cA^\wot$ is closed convex, $\bar\cA^\sot=\bar\cA^\wot$.
Also, $\cA''$ is weakly closed, $\bar\cA^\wot\subset\cA''$.
\end{pf}


\section{The Kaplansky density theorem}

\section{Borel functional calculus}
resolution of identity

normal operator theories: multiplicity, invariant subspaces

\section{Traces}
Every trace of factor is faithful


\begin{prb}
Normal states is a state in which the monotone convergence theorem holds.
Precisely, a state $\rho$ is \emph{normal} if a monotone net $a_\alpha$ strongly converges to $a$ then $\rho(a_\alpha)\to\rho(\alpha)$.
\end{prb}





\end{document}