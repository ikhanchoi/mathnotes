\documentclass{../note}
\usepackage{../../ikany}


\begin{document}
\title{Algebraic Structures}
\author{Ikhan Choi}
\maketitle
\tableofcontents

\part{Groups}
\chapter{Subgroups}
subgroups
homomorphisms, image, kernel, inverse images
normality, quotient, coset counting
direct sum, direct product

\chapter{Group actions}
\section{Orbits and stabilizers}
Invariants on orbit space.
The size and number of orbits.

\begin{prb}[Transitive actions]
stabilizer of an action is well defined

\end{prb}

\begin{prb}[Free actions]
no fixed point,
trivial stabilizer for any point,
every orbit has 1-1 correspondence to group
\end{prb}

\section{Action by conjugation}
\section{Action by left multiplication}


\section{Automorphism groups}
\begin{prb}[Outer automorphism group]
duality for center
\end{prb}



\chapter{Symmetry groups}

elements by order
elements by conjugacy class
subgroups by conjugacy class


\section{Cyclic groups}
\section{Symmetric groups}
\section{Matrix groups}
dihedral groups


\section*{Exercises}

\begin{prb}
Let $G$ be a finite group.
If $G/Z(G)$ is cylic, then $G$ is abelian.
\end{prb}

\begin{prb}
Let $G$ be a finite group.
If the cube map $x\mapsto x^3$ is a surjective endomorhpism, then $G$ is abelian.
\end{prb}

\begin{prb}
Show that a finite symmetric group has two generators.
\end{prb}

\begin{prb}
Show that a group of order $2p$ for a prime $p$ has exactly two isomorphic types.
\end{prb}

\begin{prb}
Show that a group $G$ is abelian if $|G|=p^2$ for a prime $p$.
\end{prb}

\begin{prb}
Let $G$ be a finite group of order $n$ and $p$ the smallest prime divisor of $n$.
Show that a subgroup of $G$ of index $p$ is normal in $G$.
\end{prb}

\begin{prb}[Primitive roots]
We find all $n$ such that $(\Z/n\Z)^\times$ is cyclic.
\end{prb}

\begin{prb}[$p$-groups]
\begin{parts}
\item A nontrivial normalizer of a $p$-group meets its center out of identity.
\item A proper subgroup of a finite $p$-group is a proper subgroup of its normalizer. In particular, every finite $p$-group is nilpotent.
\end{parts}
\end{prb}

\begin{prb}
Show that a finite group $G$ satisfying $\sum_{g\in G}\ord(g)\le2n$ is abelian.
\end{prb}

\begin{prb}
Show that the order of a group with trivial automorphism group is at most two.
\end{prb}

\begin{prb}
Find all homomorphic images of $A_4$ up to isomorphism.
\end{prb}

\section*{Problems}
\begin{enumerate}
\item 
\end{enumerate}







\part{Rings}
\chapter{Ideals}


\chapter{Integral domains}

\section*{Exercises}
\begin{prb}
Show that a finite integral domain is a field.
\end{prb}

\begin{prb}
Show that every ring of order $p^2$ for a prime $p$ is commutative.
\end{prb}

\begin{prb}
Show that a semiring with multiplicative identity and cancellative addtion has commutative addition.
\end{prb}

\begin{prb}
Show that the complement of a saturated monoid in a commutative ring is a union of prime ideals.
\end{prb}


\chapter{Polynomial rings}
\section{Irreducible polynomials}
relation to maximal ideals
Irreducibles over several fields













\part{Modules}

\chapter{Exact sequences}
free modules
inj, proj

\chapter{Hom set and tensor products}
hom and duality
tensor product
algebras?

\chapter{Modules over a principal ideal domain}
invariant factors and elementary divisors

















\part{Vector spaces}


\chapter{}
\section{Dual space}

\begin{prb}[Double dual space]
\end{prb}

\section{Bilinear and sesquilinear forms}

\begin{prb}[Polarization identity]
\begin{parts}
\item Let $F$ be a field of characteristic not $2$. If $\<-,-\>$ is a symmetric bilinear form, then
\[\<x,y\>=\frac12(\|x+y\|^2-\|x\|^2-\|y\|^2).\]
\item Let $F=\C$. If $\<-,-\>$ is a sesquilinear form, then
\[\<x,y\>=\frac14\sum_{k=0}^3i^k\|x+i^ky\|^2.\]
\item isometry check
\end{parts}
\end{prb}

\begin{prb}[Cauchy-Schwarz inequality]
\begin{parts}
\item Let $F=\R$. If $\<-,-\>$ is a positive semi-definite symmetric bilinear form, then
\item Let $F=\C$. If $\<-,-\>$ is a positive semi-definite Hermitian form, then
\end{parts}
\end{prb}

\begin{prb}[Dual space identification]
Let $\<-,-\>$ be a non-degenerate bilinear form
\end{prb}

\section{Adjoint}
\begin{prb}[Adjoint linear transforms]
\end{prb}







\chapter{Normal forms}
\section{Rational canonical form}
\begin{prb}[Finitely generated $\F\lbrack x\rbrack$-modules]
\end{prb}
\begin{prb}[Cyclic subspaces]
\end{prb}
\section{Jordan normal form}


\section{Conjugacy classes in matrix groups}

\begin{prb}[Conjugacy classes of $\GL_2(\F_p)$]
The conjugacy classes are classified by the Jordan normal forms.
There are four cases: for some $a$ and $b$ in $\F_p$,
\begin{parts}
\item $\mat{a&0\\0&b}$: $\binom{p-1}2=\frac{(q-1)(q-2)}2$ classes of size $\frac{|G|}{(q-1)^2}=q(q+1)$.
\item $\mat{a&0\\0&a}$: $q-1$ classes of size $1$.
\item $\mat{a&1\\0&a}$: $q-1$ classes of size $\frac{|G|}{q(q-1)}=q^2-1$.
\item otherwise, the eigenvalues are in $\F_{p^2}\setminus\F_p$.
In this case, the number of conjugacy classes is same as the number of monic irreducible qudratic polynomials over $\F_p$; $\frac{|\F_{p^2}|-|\F_p|}2=\frac{p(p-1)}2$ classes.
Their size is $\frac{p(p-1)}2$.
\end{parts}
\end{prb}

\section{Spectral theorems}



\section*{Exercises}




\chapter{Tensor algebras}
Exterior algebras
Symmetric algebras





\end{document}