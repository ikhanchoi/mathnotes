\documentclass{../note}
\usepackage{../../ikany}


\begin{document}
\title{Algebraic Structures}
\author{Ikhan Choi}
\maketitle
\tableofcontents

\part{Groups}
\chapter{Subgroups}
subgroups
homomorphisms, image, kernel, inverse images
normality, quotient, coset counting
direct sum, direct product

\chapter{Group actions}
\section{Orbits and stabilizers}
Invariants on orbit space.
The size and number of orbits.

\begin{prb}[Transitive actions]
stabilizer of an action is well defined

\end{prb}

\begin{prb}[Free actions]
no fixed point,
trivial stabilizer for any point,
every orbit has 1-1 correspondence to group
\end{prb}

\section{Action by conjugation}
\section{Action by left multiplication}


\chapter{Symmetry groups}

elements by order
elements by conjugacy class
subgroups by conjugacy class


\section{Cyclic groups}
\section{Symmetric groups}
\section{Matrix groups}
dihedral groups



\section*{Exercises}

\begin{prb}
Let $G$ be a finite group.
If $G/Z(G)$ is cylic, then $G$ is abelian.
\end{prb}

\begin{prb}
Let $G$ be a finite group.
If $x\mapsto x^3$ is a surjective endomorhpism, then $G$ is abelian.
\end{prb}











\part{Rings}
\chapter{Ideals}
\chapter{Integral domains}


\chapter{Polynomial rings}
\section{Irreducible polynomials}
relation to maximal ideals
Irreducibles over several fields













\part{Modules}

\chapter{Exact sequences}
free modules
inj, proj

\chapter{Hom functor and tensor products}
hom and duality
tensor product
algebras?

\chapter{Modules over a principal ideal domain}
invariant factors and elementary divisors

















\part{Vector spaces}


\chapter{Multilinear forms}
Duality
Adjoints
Inner product




\chapter{Normal forms}
\section{Finitely generated $\F[x]$-modules}
cyclic subspaces

\section{Similarity}
GL, SL, PSL? % Galois cohomology.......


\section{Spectral theorems}



\section*{Exercises}


\chapter{Tensor algebras}
Exterior algebras
Symmetric algebras





\end{document}