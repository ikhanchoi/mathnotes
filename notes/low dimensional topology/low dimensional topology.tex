\documentclass{../note}
\usepackage{../../ikany}


\begin{document}
\title{Low Dimensional Topology}
\author{Ikhan Choi}
\maketitle
\tableofcontents

\part{Topology of 3-manifolds}
\chapter{}
\chapter{}
\chapter{}


\part{Geometry of 3-manifolds}
\chapter{Hyperbolization}
\section{Geometric structures}

I need to check more carefully the followings...
All statements in here may not be true, anyway.

\begin{prb}[Geometric structure]
Let $M$ be a connected smooth manifold.
We are concerned with geometric structures on $M$.
We restrict our interests on geometries that having length, area, volume, and angle measurements.
For example, affine, projective, or conformal geometries are not considered to be candidates of geometric structures.
Precisely, we suggest to define a \emph{geometric structure} as a metric $d$ on $M$ such that
\begin{enumerate}[(i)]
\item $(M,d)$ is geodesically connected,
\item $(M,d)$ is geodesically complete,
\item $(M,d)$ is a Riemannian manifold,
\item $(M,d)$ is locally homogeneous.
\end{enumerate}
In other words, a geometric structure on $M$ is a Riemannian metric satisfying (i), (ii), and (iv).
Each condition has been obtained by modifying the first four postulates of Euclid's Elements.
\begin{parts}
\item $M$ is a geometric manifold if and only if $M$ is isometric to $X/\Gamma$, where $X$ is a simply connected geometric manifold and $\Gamma$ is a torsion-free discrete subgroup of $\Isom(X)$. In this case, we say $M$ is a \emph{space form} of $X$.
\item If $M$ is simply connected, then a geometric structure is the same thing as a homogeneous Riemannian metric.
\end{parts}
\end{prb}
\begin{pf}
(a) % 지금 실력으로 손 못 대겠다..
($\Rightarrow$)

($\Leftarrow$)

(b)
($\Rightarrow$)
Ambrose-Singer theorem.

($\Leftarrow$)
Let $g$ be a homogeneous Riemannian metric on $M$.
We will prove $g$ satisfies (ii) and (i).
\end{pf}


\begin{prb}[Homogeneous Riemannian metrics]
Let $M$ be a connected smooth manifold.
We want to establish the following correspondence.
\[\left\{\begin{tabular}{c}Homogeneous\\Riemannian metrics\end{tabular}\right\}\leftrightarrow\left\{\begin{tabular}{c}Homogeneous\\maximal smooth group actions\\with compact stabilizers\end{tabular}\right\}.\]
\begin{parts}
\item If $g$ is a homogeneous Riemannian metric on $M$, the group action on $M$ by $\Isom(M,g)$ is maximal among smooth group actions with compact stabilizers.
\item If a smooth group action on $M$ by $G$ is maximal among smooth group actions with compact stabilizers, then there is a homogeneous Riemannian metric on $M$ such that $G\cong\Isom(M,g)$.
\end{parts}
\end{prb}
\begin{pf}
\end{pf}

\begin{prb}[Pseudogroup structure]
Let $(X,\cT)$ be a topological space.
A \emph{pseudogroup} on $X$ is a wide subgroupoid $\Gamma$ of $\Homeo(X)$ such that $\cT\to\mathbf{Set}:U\mapsto\{\,g\in\Gamma:\dom g=U\,\}$ is a separated presheaf; it satisfies the locality, but not the gluing axiom.
Let $\Gamma$ be a pseudogroup on $X$, and $M$ be a topological space.
A \emph{$\Gamma$-atlas} on $M$ is an atlas whose charts have $X$ as the codomain and transition maps belong to $\Gamma$.
A \emph{$\Gamma$-structure} on $M$ is defined as an equivalence class of $\Gamma$-atlases on $M$.
\begin{parts}
\item For $G\le\Homeo(X)$, $\{\,g|_U:g\in G,\ U\in\cT\,\}$ is a pseudogroup on $X$. We write this pseudogroup as $(G,X)$.
\item...
Note that $G$ does not act on $(G,X)$-manifold $M$.
\end{parts}
\end{prb}

\begin{prb}[Complete $(G,X)$-structure]
Let $(G,X)$ be a model geometry, and $M$ be a connected smooth manifold.

Developing map and holonomy.

We will show the equivalence of the following statements.
\begin{enumerate}[(i)]
\item $M$ admits a geometric structure such that the universal covering is $X$.
\item $M$ admits a complete $(G,X)$-structure.
\end{enumerate}
Therefore, for a model geometry $(G,X)$, a complete $(G,X)$-structure on $M$ be called a \emph{geometric structure} on $M$.
\begin{parts}
\item (i) implies (ii).
\item (ii) implies (i).
\end{parts}
\end{prb}

analyticity of isometries of homogeneous Riemannian manifolds?

examples.

\begin{prb}[Thurston's eight geometries]
We define a \emph{model geometry} or a \emph{Thurston geometry} as a simply connected geometric manifold $X$ such that there exists at least one space form of finite volume.
\end{prb}


\section{Mostow rigidity}
Kleinian groups
Several topological invariants: volume, trace fields, etc.


\section{Hyperbolization Dehn surgery}

\begin{prb}[Ideal triangulation of knot complement]
Cusped hyperbolic 3-manifolds
\end{prb}

\begin{prb}[Cusp and horoball]
\end{prb}

\begin{prb}[Thick-thin decomposition]
Margulis constant.
\end{prb}

\begin{prb}[Thurston's Hyperbolic Dehn surgery]
\end{prb}


\section{Orbifolds}




\chapter{Teichm\"uller theory}
\section{}

\chapter{Geometric group theory}
\section{}


\part{Topology of 4-manifolds}
\chapter{Surgery theory}
\chapter{Intersection forms}
\chapter{Kirby calculus}


\part{Geometry of 4-manifolds}
\chapter{}
\chapter{}
\chapter{}

\end{document}