\documentclass{../note}
\usepackage{../../ikany}


\begin{document}
\title{Partial Differential Equations}
\author{Ikhan Choi}
\maketitle
\tableofcontents


\part{Distributions and Sobolev Spaces}
\chapter{Distributions}


\section{Extension of linear operators}
Let $T:\cD\to\cD'$ be a contiuous linear operator.
We can always define the adjoint $T^*:\cD\subset\cD''\to\cD'$.
The most reasonable extension of $T$ is $T:(T^*(\cD))'\to\cD'$.
For $f\in(T^*(\cD))'$, we can define $\<T(f),\f\>:=\<f,T^*\f\>$ for $\f\in\cD$.

Suppose $T:(\cD,\cT)\to(T(\cD),\cS)$ is proved to be continuous.
If $(\cD,\cT)\to(T^*(\cD))'$ and $(T(\cD),\cS)\to\cD'$ are embeddings, then the extension of $T$ to the completion of $(\cD,\cT)$ agrees with $T:(T^*(\cD))'\to\cD'$.

\section{Convolutions}
For example, if $\Phi$ is locally integrable, then since $(T_\Phi)^*=T_{\tilde\Phi}$ and $\Phi*\f\in\cE=C^\infty$ for $\f\in\cD$, the convolution operator $T_\Phi:\cE'\to\cD'$ can be defined on the space of compactly supported distributions.

\textbf{Problem:}
If $g*f$ is well-defined, is $f*g$ also well-defined?
In other words, if $f\in(T_{\tilde g}(\cD))'$ so that $g*f\in\cD'$, then $g\in(T_{\tilde f}(\cD))'$? Are they same?
\[\<g,\tilde f*\f\>=\]


\begin{prb}
\begin{parts}
\item If a test function $\f$ satisfies $\<1,\f\>=0$, then there is $v\in\R^d$ and a test function $\psi$ such that $\f=v\cdot\nabla\psi$.
\item If a distribution has zero derivative, then it is a constant.
\end{parts}
\end{prb}





\chapter{Sobolev spaces}

\section{Definition and examples}
\begin{prb}[Sobolev space is a Banach space]
\end{prb}
\begin{prb}[Difference quotient]
\end{prb}
\begin{prb}[Interior approximation]
\end{prb}
\begin{prb}[Myers-Serrin theorem]
\end{prb}

\section{Extensions and restrictions}
\begin{prb}[Lipschitz boundary]
\end{prb}
\begin{prb}[Extension theorem]
\end{prb}
\begin{prb}[Trace theorem]
\end{prb}
\begin{prb}[Vanishing at boundary]
zero trace, whole domain
\end{prb}

\section{Sobolev embeddings}
\begin{prb}[Gagliardo-Nirenberg-Sobolev inequality]
\end{prb}
\begin{prb}[H\"older spaces]
\end{prb}
\begin{prb}[Morrey inequality]
\end{prb}
\begin{prb}[Poincar\'e inequality] BMO
\end{prb}
\begin{prb}[Rellich-Kondrachov theorem]
Let $\Omega$ be bounded open subset of $\R^d$ with Lipschitz boundary.
Let $1\le p<d$ and $1\le q<p^*$ where $p^*:=\frac{dp}{d-p}$ denotes the Sobolev conjugate.
Let $(u_n)_n$ be a bounded sequence in $W^{1,p}(\Omega)$.
We may assume it is also bounded in $W^{1,1}(\R^d)$ by the embedding $W^{1,p}(\Omega)\subset W^{1,1}(\Omega)$ and the extension theorem.
Let $\eta_\e$ be a standard mollifier.
\begin{parts}
\item There is a subsequence of $(\eta_\e*u_n)_n$ that is Cauchy in $L^q(\Omega)$ for each $\e>0$.
\item $\sup_n\|\eta_\e*u_n-u_n\|_{L^1(\Omega)}\to0$ as $\e\to0$.
\item $\sup_n\|\eta_\e*u_n-u_n\|_{L^q(\Omega)}\to0$ as $\e\to0$.
\item There is a subsequence of $(u_n)_n$ that is Cauchy in $L^q(\Omega)$.
\item $W^{k,p}(\Omega)\to W^{l,q}(\Omega)$ is a compact embedding if
\[\frac ld-\frac1q<\frac kd-\frac1p.\]
\end{parts}
\end{prb}
\begin{pf}
(a)
The sequence $(\eta_\e*u_n)_n$ is pointwise bounded from
\[\|\eta_\e*u_n\|_{C_0(\R^d)}\le\|\eta_\e\|_{C_0(\R^d)}\|u_n\|_{L^1(\R^d)}\lesssim_\e1,\]
and equicontinuous from
\[\|\nabla\eta_\e*u_n\|_{C_0(\R^d)}\le\|\nabla\eta_\e\|_{C_0(\R^d)}\|u_n\|_{L^1(\R^d)}\lesssim_\e1.\]
By the Arzela-Ascoli theorem, since $\bar\Omega$ is compact, there is a subsequence $(\eta_\e*u_{n_k})_k$ that is Cauchy in $C(\bar\Omega)$, and hence in $L^q(\Omega)$.

(b)
Write
\begin{align*}
\eta_\e*u_n(x)-u_n(x)
&=\frac1{\e^d}\int\eta\left(\frac{x-y}\e\right)(u_n(y)-u_n(x))\,dy\\
&=\int\eta(y)(u_n(x-\e y)-u_n(x))\,dy\\
&=\int\eta(y)\int_0^1\dd{t}(u_n(x-t\e y))\,dt\,dy\\
&=\int\eta(y)\int_0^1(-\e y)\cdot\nabla u_n(x-t\e y)\,dt\,dy.
\end{align*}
Then, since $|y|\ge1$ if $\eta(y)>0$,
\[\|\eta_\e*u_n-u_n\|_{L^1(\R^d)}\le\e\int\eta(y)\int_0^1\int|\nabla u_n(x-t\e y)|\,dx\,dt\,dy=\e\|\nabla u_n\|_{L^1(\R^d)}.\]

(c)
The interpolation
\[\|\eta_\e*u_n-u_n\|_{L^q(\Omega)}\le\|\eta_\e*u_n-u_n\|_{L^1(\Omega)}^\theta\|\eta_\e*u_n-u_n\|_{L^{p^*}(\Omega)}^{1-\theta}\]
for $q=\frac\theta1+\frac{1-\theta}p$ with $0<\theta\le1$ and the Gagliardo-Nireberg-Sobolev inequality
\[\|\eta_\e*u_n-u_n\|_{L^{p^*}(\Omega)}\lesssim\|\eta_\e*u_n-u_n\|_{W^{1,p}(\Omega)}\lesssim1\]
give the $L^q$ version of the part (b),
\[\sup_n\|\eta_\e*u_n-u_n\|_{L^q(\Omega)}\to0\]
as $\e\to0$.

(d)
By the part (c), for any $\delta>0$, there is $\e>0$ such that
\[\sup_n\|\eta_\e*u_n-u_n\|_{L^q(\Omega)}<\frac\delta2,\]
so for a subsequence $(\eta_\e*u_{n_k})_k$ that is Cauchy in $L^q(\Omega)$, we have
\[\|u_{n_k}-u_{n_{k'}}\|_{L^q(\Omega)}\le\|\eta_\e*u_{n_k}-\eta_\e*u_{n_{k'}}\|_{L^q(\Omega)}+\delta,\]
and by the diagonal argument reducing $\delta$ to zero, we can construct the desired subsequence.

(e)
\end{pf}




\chapter{More on Sobolev spaces}
\section{Fractional Sobolev spaces}
\section{Fourier transform methods}
\section{Almost everywhere differentiability}
Lipschitz, Rademacher
\section{Vector-valued functions}








\part{Elliptic equations}

\chapter{Existence}
\section{Lax-Milgram theorem}
\section{Fredholm alternative}
\section{Perron's method}

\chapter{Regularity}
\section{$L^p$ theory}
\begin{prb}[Interior regularity in $H^2$]
Let $\Omega$ be bounded open subset of $\R^d$ and $L:\cD'(\Omega)\to\cD'(\Omega)$ a uniformly elliptic operator given by
\[Lu:=-\pd_j(a^{ij}\pd_iu)+b^i\pd_iu+cu\]
for $a^{ij}\in C^1(\Omega)$, $b^i\in L^\infty(\Omega)$, and $c\in L^\infty(\Omega)$.

Fix an open subset $U\Subset\Omega$ and $\zeta\in C_c^\infty(\Omega)$ a cutoff function such that $\zeta=1$ in $U$.
Let $\f:=-\pd_k^{-h}(\zeta^2\pd_k^hu)$ for $k=1,\cdots,d$ and sufficiently small $h>0$.
\begin{parts}
\item We have
\[\|\nabla u\|_{L^2(U)}\lesssim\|Lu\|_{L^2(\Omega)}+\|u\|_{L^2(\Omega)}\]
for all $u$ such that $Lu,u\in L^2(\Omega)$
\item We have
\[\int\zeta^2|\pd_k^h\nabla u|^2\lesssim\int a^{ij}\pd_iu\pd_j\f+\|\nabla u\|_{L^2(\Omega)}\]
for all $u\in H^1(\Omega)$.
\item We have
\[\int\zeta^2|\pd_k^h\nabla u|^2\lesssim\|Lu\|_{L^2(\Omega)}+\|u\|_{H^1(\Omega)}\]
for all $u$ such that $Lu\in L^2(\Omega)$ and $u\in H^1(\Omega)$.
\item We have
\[\|u\|_{H^2(U)}\lesssim\|Lu\|_{L^2(\Omega)}+\|u\|_{L^2(\Omega)}\]
for all $u$ such that  $Lu,u\in L^2(\Omega)$.
\end{parts}
\end{prb}
\begin{pf}
(a)
Since $\zeta^2u\in H_0^1(\Omega)$,
\begin{align*}
\int\zeta^2|\nabla u|^2
&\lesssim\int a^{ij}\zeta^2\pd_iu\pd_ju\\
&=\int a^{ij}\;\pd_iu\;\pd_j(\zeta^2u)-\int a^{ij}\;\pd_iu\;\pd_j(\zeta^2)u\\
&=\int(Lu-b^i\pd_iu-cu)\;\zeta^2u-\int a^{ij}\;\pd_iu\;2\zeta\pd_j\zeta\;u\\
&\lesssim\int(|Lu\;u|+|u\;\zeta\nabla u|+|u|^2+|u\;\zeta\nabla u|)\\
&\lesssim\int(|Lu|^2+|u|^2)+\frac1\e\int|u|^2+\e\int\zeta^2|\nabla u|^2.
\end{align*}
Taking small $\e>0$, we are done.

(b)
Write
\begin{align*}
\int a^{ij}\pd_iu\pd_j\f
&=-\int a^{ij}\pd_iu\pd_j\pd_k^{-h}(\zeta^2\pd_k^hu)\\
&=\int\pd_k^h(a^{ij}\pd_iu)\;\pd_j(\zeta^2\pd_k^hu)\\
&=\int\pd_k^ha^{ij}\;\pd_iu\;\pd_j(\zeta^2)\;\pd_k^hu
+\int\pd_k^ha^{ij}\;\pd_iu\;\zeta^2\;\pd_j\pd_k^hu\\
&+\int a^{ij}\;\pd_k^h\pd_iu\;\pd_j(\zeta^2)\;\pd_k^hu
+\int a^{ij}\;\pd_k^h\pd_iu\;\zeta^2\;\pd_j\pd_k^hu.
\end{align*}
The last term out of the four terms controls the difference quotient $|\pd_k^h\nabla u|$ as
\[\int a^{ij}\;\pd_k^h\pd_iu\;\zeta^2\;\pd_j\pd_k^hu\gtrsim\int\zeta^2|\pd_k^h\nabla u|^2,\]
and the absolute values of other three terms are estimated up to constant by
\begin{align*}
&\int\zeta|\nabla u||\pd_k^h u|+\int\zeta^2|\nabla u||\pd_k^h\nabla u|+\int\zeta|\pd_k^h\nabla u||\pd_k^hu|\\
&\qquad\lesssim
\left(1+\frac1\e\right)\int\zeta^2|\nabla u|^2+\left(1+\frac1\e\right)\int|\pd_h^ku|^2+\e\int\zeta^2|\pd_k^h\nabla u|^2\\
&\qquad\lesssim
\left(1+\frac1\e\right)\int|\nabla u|^2+\e\int\zeta^2|\pd_k^h\nabla u|^2.
\end{align*}
Therefore,
\[\int\zeta^2|\pd_k^h\nabla u|^2\lesssim\int a^{ij}\pd_iu\pd_j\f+\left(1+\frac1\e\right)\int|\nabla u|^2+\e\int\zeta^2|\pd_k^h\nabla u|^2,\]
and taking small $\e>0$, we are done.

(c)
Note that
\[\int a^{ij}\pd_iu\pd_j\f=\int(Lu-b^i\pd_iu-cu)\f\]
since $\f\in H_0^1(\Omega)$.
Because
\[\int(Lu-b^i\pd_iu-cu)\f
\lesssim\frac1\e\int(|Lu|^2+|\nabla u|^2+|u|^2)+\e\int|\f|^2\]
and
\begin{align*}
\int|\f|^2&=\int|\pd_k^{-h}(\zeta^2\pd_k^hu)|^2\\
&\lesssim\int|\nabla(\zeta^2\pd_k^hu)|^2\\
&\lesssim\int|\pd_k^hu|^2+\int\zeta^2|\pd_k^h\nabla u|^2\\
&\lesssim\int|\nabla u|^2+\int\zeta^2|\pd_k^h\nabla u|^2,
\end{align*}
we obtain
\[\int(Lu-b^i\pd_iu-cu)\f
\lesssim\frac1\e\int(|Lu|^2+|u|^2)+\left(\e+\frac1\e\right)\int|\nabla u|^2+\e\int\zeta^2|\pd_k^h\nabla u|^2.\]
Taking small $\e>0$, we are done.
\end{pf}



\section{Schauder theory}
\section{Weyl's lemma}

\chapter{}
\section{Maximum principle}
\section{Eigenvalue problems}











\part{Evolution equations}

\chapter{Parabolic equations}
\section{Galerkin approximation}

\chapter{Hyperbolic equations}

\chapter{Semigroup theory}






\end{document}