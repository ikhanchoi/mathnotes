\documentclass{../note}
\usepackage{../../ikany}


\begin{document}
\title{Foundations of Calculus}
\author{Ikhan Choi}
\maketitle
\tableofcontents

\part{Convergence}
\chapter{Sequences}
% 문제의식: 입실론 델타의 직접적 이용하지 않는 수렴성 증명

\section{Control of the error}
preserving inequalities
limsup and liminf
\section{Approximate sequences}
\section{Bounded sequences}
monotone convergence
Bolzano-Weierstrass
\section{Recursive sequences}
?
\section*{Exercises}
\begin{prb}
Every real sequence $(a_n)_{n=1}^\infty$ has a monotonic subsequence $(a_{n_k})_{k=1}^\infty$ such that $\lim_{k\to\infty}a_{n_k}=\limsup_{n\to\infty}a_n$.
\end{prb}



\chapter{Series}
% 문제의식: 시리즈의 수렴 및 발산 여부 판정
% 점근적 성질(단조성, 진동, 레이트)을 배운다 (판정법)
% 영역 분할 메커니즘을 배운다
% 멱급수는 해석함수에서 따로 다룬다

\section{Absolute convergence}
\begin{prb}[Unconditional convergence]
\end{prb}


\section{Convergence tests}

comparison
limit comparison
cauchy condensation
integral....

ratio
root



\begin{prb}[Abel transform]
\[A_k(B_k-B_{k-1})+(A_k-A_{k-1})B_{k-1}=A_kB_k-A_{k-1}B_{k-1}\]
\[\sum_{m<k\le n}A_kb_k=A_nB_n-A_mB_m-\sum_{m<k\le n}a_kB_{k-1}.\]
\end{prb}

abel test
\begin{prb}[Dirichlet test]
\end{prb}




\begin{prb}[Mertens' theorem]
If $\sum_{k=0}^\infty a_k$ converges to $A$ absolutely and $\sum_{k=0}^\infty b_k$ converges to $B$, then their Cauchy product $\sum_{k=0}^\infty c_k$ with $c_k:=\sum_{l=0}^ka_lb_{k-l}$ converges to $AB$.
\end{prb}
\begin{pf}
Let
\[A_n:=\sum_{k=0}^na_k,\ B_n:=\sum_{k=0}^nb_k, \text{ and } C_n:=\sum_{k=0}^nc_k.\]
Consider the regions
\[T_n:=\{(k,l)\in\Z_{\ge0}^2:k+l\le n\},\qquad R_m:\{(k,l)\in\Z_{\ge0}^2:k\le m\}.\]
Write
\begin{align*}
AB-C_n&=\sum_{k\le m}\sum_{l>n-k}a_kb_l+\sum_{k>m}\sum_{l\ge0}a_kb_l-\sum_{m<k\le n}\sum_{l\le n-k}a_kb_l\\
&=\sum_{k\le m}a_k(B-B_{n-k})+\sum_{k>m}a_kB-\sum_{m<k\le n}a_kB_{n-k}.
\end{align*}

The first term
\[|\sum_{k\le m}a_k(B-B_{n-k})|\le(\max_k|a_k|)(\sum_{l\ge n-m}|B-B_l|)\]
converges to zero as $n\to\infty$ for fixed $m$, the second term
\[|\sum_{k>m}a_kB|\le|A-A_m||B|\]
converges to zero as $m\to\infty$ for any $n$, and finally the third term
\[|\sum_{m<k\le n}a_kB_{n-k}|\le(\sum_{k>m}|a_k|)(\max_l|B_l|)\]
converges to zero as $m\to\infty$ for any $n$.

Fix $m$ such that the second and third terms are bounded by arbitrary $\frac\e2>0$ so that
\[|C_n-AB|\le|\sum_{k\le m}a_k(B-B_{n-k})|+\frac\e2+\frac\e2.\]
Then, by taking $n\to\infty$, we obtain
\[\limsup_{n\to\infty}|C_n-AB|\le\e.\] 
Since $\e$ is arbitrary, we have
\[\lim_{n\to\infty}C_n=AB.\qedhere\]
\end{pf}



\section*{Exercises}

\begin{prb}
If $a_n\to0$, then $\frac1n\sum_{k=1}^na_k\to0$.
\end{prb}

\begin{prb}
If $a_n\ge0$ and $\sum a_n$ diverges, then $\sum\frac{a_n}{1+a_n}$ also diverges.
\end{prb}

\begin{prb}
If $a_n\downarrow0$ and $S_n\le1+na_n$, then $S_n\le1$.
\end{prb}






\chapter{Metrics and norms}
% 문제의식: 노름과 메트릭 추상화, 동치는 다루지 않음
\section{Metric spaces}
\begin{prb}[Definition of metric spaces]
Let $X$ be a set.
A \emph{metric} is a function $d:X\times X\to\R_{\ge0}$ such that
\begin{parts}[(i)]
\item $d(x,y)=0$ if and only if $x=y$, \hfill(nondegeneracy)
\item $d(x,y)=d(y,x)$ for all $x,y\in X$, \hfill(symmetry)
\item $d(x,z)\le d(x,y)+d(y,z)$ for all $x,y,z\in X$. \hfill(triangle inequality)
\end{parts}
A pair $(X,d)$ of a set $X$ and a metric on $X$ is called a \emph{metric space}.
We often write it simply $X$.
\begin{parts}
\item
A normed space $X$ is a metric space with a metric defined by $d(x,y):=\|x-y\|$.
\item
A subset of a metric space is a metric space with a metric given by restriction.
\end{parts}
\end{prb}

\begin{prb}[System of open balls]
A metric is often misunderstood as something that measures a distance between two points and belongs to the study of geoemtry.
The main function of a metric is to make a system of small balls, sets of points whose distance from specified center points is less than fixed numbers.
The balls centered at each point provide a concrete images of ``system of neighborhoods at a point'' in a more intuitive sense.
In this viewpoint, a metric can be considered as a structure that lets someone accept the notion of neighborhoods more friendly.

Note that taking either $\e$ or $\delta$ in analysis really means taking a ball of the very radius.
Investigation of the distribution of open balls centered at a point is now an important problem.

Let $X$ be a metric space.
A set of the form 
\[\{y\in X:d(x,y)<\e\}\]
for $x\in X$ and $\e>0$ is called an \emph{open ball centered at $x$ with radius $\e$} and denoted by $B(x,\e)$ or $B_\e(x)$.
\end{prb}

\begin{prb}[Convergence and continuity in metric spaces]
Let $\{x_n\}_n$ be a sequence of points on a metric space $(X,d)$.
We say that a point $x$ is a \emph{limit} of the sequence or the sequence \emph{converges to $x$} if for arbitrarily small ball $B(x,\e)$, we can find $n_0$ such that $x_n\in B(x,\e)$ for all $n>n_0$.
If it is satisfied, then we write
\[\lim_{n\to\infty}x_n=x,\]
or simply $x_n\to x$ as $n\to\infty$.
We say a sequence is \emph{convergent} if it converges to a point.
If it does not converge to any points, then we say the sequence \emph{diverges}.

A function $f:X\to Y$ between metric spaces is called \emph{continuous at $x\in X$} if for any ball $B(f(x),\e)\subset Y$, there is a ball $B(x,\delta)\subset X$ such that $f(B(x,\delta))\subset B(f(x),\e)$.
The function $f$ is called \emph{continuous} if it is continuous at every point on $X$.
\begin{parts}
\item A sequence $x_n$ in a metric space $X$ converges to $x\in X$ if and only if $d(x_n,x)$ converges to zero.
\item 
Let $f:X\to Y$ be a function between two metric spaces.
If there is a constant $C$ such that $d(x,y)\le Cd(f(x),f(y))$ for all $x$ and $y$ in $X$, then $f$ is continuous.
In this case, $f$ is particularly called \emph{Lipschitz continuous} with the \emph{Lipschitz constant} $C$.
\end{parts}
\end{prb}


\section{Normed spaces}
banach space

\section{Open sets and closed sets}
convergence, limit point
\section{Compact sets}
\section{Connected sets}



\section*{Exercises}








\part{Real functions}

\chapter{Continuous functions}
% 문제의식: 연속함수의 성질을 엄밀하게 기술할 수 있는 보조정리 증명, 연속함수 공간 개요
\section{Intermediate and extreme value theorems}
\section{Uniform continuity}
\section{Uniform convergence}

\section*{Exercises}

\begin{prb}
The set of local minima of a convex real function is connected.
\end{prb}

\begin{prb}
Let $f:\R\to\R$ be continuous.
The equation $f(x)=c$ cannot have exactly two solutions for every constant $c\in\R$.
\end{prb}

\begin{prb}
A continuous function that takes on no value more than twice takes on some value exactly once.
\end{prb}

\begin{prb}
Let $f$ be a function that has the intermediate value property.
If the preimage of every singleton is closed, then $f$ is continuous.
\end{prb}

\begin{prb}*
If a sequence of real functions $f_n\colon[0,1]\to[0,1]$ satisfies $|f(x)-f(y)|\le|x-y|$ whenever $|x-y|\ge\frac1n$, then the sequence has a uniformly convergent subsequence.
\end{prb}

\chapter{Differentiable functions}
% 로피탈..
\section{Monotonicty and convexity}
\section{Mean value theorem}
Darboux
\section{Taylor's theorem}
\section{Differentiable class}
completeness

\section*{Exercises}

\begin{prb}
If $\lim_{x\to\infty}f(x)=a$ and $\lim_{x\to\infty}f'(x)=b$, then $a=0$.
\end{prb}

\begin{prb}
Let $f$ be a real $C^2$ function with $f(0)=0$ and $f''(0)\ne0$.
Defined a function $\xi$ such that $f(x)=xf'(\xi(x))$ with $|\xi|\le|x|$, we have $\xi'(0)=1/2$.
\end{prb}

\begin{prb}
Let $f$ be a $C^2$ function such that $f(0)=f(1)=0$.
We have $\|f\|\le\frac18\|f''\|$.
\end{prb}

\begin{prb}
A smooth function such that for each $x$ there is $n$ having the $n$th derivative vanish is a polynomial.
\end{prb}

\begin{prb}
If a real $C^1$ function $f$ satisfies $f(x)\ne0$ for $x$ such that $f'(x)=0$, then in a bounded set there are only finite points at which $f$ vanishes.
\end{prb}

\begin{prb}
Let a real function $f$ be differentiable.
For $a<a'<b<b'$ there exist $a<c<b$ and $a'<c'<b'$ such that $f(b)-f(a)=f'(c)(b-a)$ and $f(b')-f(a')=f'(c')(b'-a')$.
\end{prb}

\begin{prb}
Let $f$ be a differentiable function on the unit closed interval.
If $f(0)=0$ there is $c$ such that $cf'(c)=f(c)$. (Flett)
\end{prb}

\begin{prb}
Let $f$ be a differentiable function on the unit closed interval.
If $f(0)=0$ there is $c$ such that $cf(c)=(1-c)f'(c)$.
\end{prb}


\chapter{Analytic functions}
\section{Convergence of power series}
uniform convergence and absolute convergence, abel theorem?
differentiation
convergence of radius
sum, product, composition, reciprocal?
closed under uniform convergence
\section{Complex analytic functions}
complex domain
(real analytic iff its domain contains real line)
convergence of radius, revisited
identity theorem
\section{Special functions}
hypergeometric, bessel, gamma, zeta

\section*{Exercises}



\part{Integration}

\chapter{Riemann integration}

\section{Riemann integral}
tagged partition
\section{Henstock-Kurzweil intergral}
bounded compact support <-> lebesgue
\section{Improper integral}
\section{Fundamental theorem of calculus for continuous functions}

\section*{Exercises}
\begin{prb}
Find the value of $\lim_{n\to\infty}\frac1n\left(\sum_{k=1}^n\frac1nf\left(\frac kn\right)-\int_0^1f(x)\,dx\right)$.
\end{prb}

\begin{prb}
If $xf'(x)$ is bounded and $x^{-1}\int_0^xf\to L$ then $f(x)\to L$ as $x\to\infty$.
\end{prb}


\chapter{Integrable functions}
\section{}
% 적분이 함수의 크기를 재는 수단임을 배운다
% 각 함수들의 적분가능성을 배운다
% 급수와 마찬가지로 적분이 수렴하거나 발산하게 되는 메커니즘을 배운다(영역, 함수가 얼마나 특이한지)


\chapter{}





\part{Multivariable Calculus}
\chapter{Fre\'chet derivatives}
% 편미분이 그냥 계산도구라는 것을 배운다
% 편미분의 교환도 사실은 항상 다 되는 걸... 배울 수 있나
% 접공간 개념을 배운다
\section{}


\chapter{Inverse function theorem}
% 고정점정리로 함수의 역을 찾아내는 기본 전략을 배운다
% 선형근사의 파워를 배운다
\section{Banach fixed point theorem}
\section{Variations of the inverse function theorem}




\chapter{Differential forms}
% 텐서곱과 테일러 근사를 배운다
% 웨지곱과 다중적분을 배운다



\end{document}