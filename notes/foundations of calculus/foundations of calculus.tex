\documentclass{../note}
\usepackage{../../ikany}


\begin{document}
\title{Foundations of Calculus}
\author{Ikhan Choi}
\maketitle
\tableofcontents

\part{Sequences}
\chapter{Convergence}
% 입실론 델타의 직접적 이용보다 절댓갑을 0으로 보내는 것의 편리함을 배운다
% 중간근사 방법으로 각 항을 입실론으로 바운드하거나 0으로 수렴시키는 메커니즘을 익힌다




% 단조수렴정리
% 수렴하는 수열은 유계


\section*{Exercises}
\begin{prb}
Every real sequence $(a_n)_{n=1}^\infty$ has a monotonic subsequence $(a_{n_k})_{k=1}^\infty$ such that $\lim_{k\to\infty}a_{n_k}=\limsup_{n\to\infty}a_n$.
\end{prb}



\chapter{Series}
% 영역을 분할하여 극한을 구하는 메커니즘을 배운다
% 각 수열의 점근적 성질(단조성, 진동, 레이트)이 급수의 수렴발산에 미치는 영향을 배운다

\section{Convergence tests}

\begin{prb}[Abel transform]
\[A_k(B_k-B_{k-1})+(A_k-A_{k-1})B_{k-1}=A_kB_k-A_{k-1}B_{k-1}\]
\[\sum_{m<k\le n}A_kb_k=A_nB_n-A_mB_m-\sum_{m<k\le n}a_kB_{k-1}.\]
\end{prb}

\begin{prb}[Dirichlet test]
\end{prb}

\begin{prb}[Mertens' theorem]
If $\sum_{k=0}^\infty a_k$ converges to $A$ absolutely and $\sum_{k=0}^\infty b_k$ converges to $B$, then their Cauchy product $\sum_{k=0}^\infty c_k$ with $c_k:=\sum_{l=0}^ka_lb_{k-l}$ converges to $AB$.
\end{prb}
\begin{pf}
Let
\[A_n:=\sum_{k=0}^na_k,\ B_n:=\sum_{k=0}^nb_k, \text{ and } C_n:=\sum_{k=0}^nc_k.\]
Consider the regions
\[T_n:=\{(k,l)\in\Z_{\ge0}^2:k+l\le n\},\qquad R_m:\{(k,l)\in\Z_{\ge0}^2:k\le m\}.\]
Write
\begin{align*}
AB-C_n&=\sum_{k\le m}\sum_{l>n-k}a_kb_l+\sum_{k>m}\sum_{l\ge0}a_kb_l-\sum_{m<k\le n}\sum_{l\le n-k}a_kb_l\\
&=\sum_{k\le m}a_k(B-B_{n-k})+\sum_{k>m}a_kB-\sum_{m<k\le n}a_kB_{n-k}.
\end{align*}

The first term
\[|\sum_{k\le m}a_k(B-B_{n-k})|\le(\max_k|a_k|)(\sum_{l\ge n-m}|B-B_l|)\]
converges to zero as $n\to\infty$ for fixed $m$, the second term
\[|\sum_{k>m}a_kB|\le|A-A_m||B|\]
converges to zero as $m\to\infty$ for any $n$, and finally the third term
\[|\sum_{m<k\le n}a_kB_{n-k}|\le(\sum_{k>m}|a_k|)(\max_l|B_l|)\]
converges to zero as $m\to\infty$ for any $n$.

Fix $m$ such that the second and third terms are bounded by arbitrary $\frac\e2>0$ so that
\[|C_n-AB|\le|\sum_{k\le m}a_k(B-B_{n-k})|+\frac\e2+\frac\e2.\]
Then, by taking $n\to\infty$, we obtain
\[\limsup_{n\to\infty}|C_n-AB|\le\e.\] 
Since $\e$ is arbitrary, we have
\[\lim_{n\to\infty}C_n=AB.\qedhere\]
\end{pf}



\section*{Exercises}

\begin{prb}
If $a_n\to0$, then $\frac1n\sum_{k=1}^na_k\to0$.
\end{prb}

\begin{prb}
If $a_n\ge0$ and $\sum a_n$ diverges, then $\sum\frac{a_n}{1+a_n}$ also diverges.
\end{prb}

\begin{prb}
If $a_n\downarrow0$ and $S_n\le1+na_n$, then $S_n\le1$.
\end{prb}






\chapter{Open sets and closed sets}
% 컴팩트성에서 오픈 커버를 잡는 일반적인 방법에 대해 배운다



\section*{Exercises}



\part{Real functions}

\chapter{Continuous functions}
% 열린 집합, 컴팩트 집합, 연결 집합과의 관계를 배운다
% 균등 연속과 그냥 연속의 차이점을 배운다
% 균등수렴을 새로운 노름으로서 배운다

\section*{Exercises}

\begin{prb}
The set of local minima of a convex real function is connected.
\end{prb}

\begin{prb}
Let $f:\R\to\R$ be continuous.
The equation $f(x)=c$ cannot have exactly two solutions for every constant $c\in\R$.
\end{prb}

\begin{prb}
A continuous function that takes on no value more than twice takes on some value exactly once.
\end{prb}

\begin{prb}
Let $f$ be a function that has the intermediate value property.
If the preimage of every singleton is closed, then $f$ is continuous.
\end{prb}

\begin{prb}*
If a sequence of real functions $f_n\colon[0,1]\to[0,1]$ satisfies $|f(x)-f(y)|\le|x-y|$ whenever $|x-y|\ge\frac1n$, then the sequence has a uniformly convergent subsequence.
\end{prb}

\chapter{Differentiable functions}
% 여러 미분가능공간들의 개념을 배운다
% 테일러 정리를 통한 근사에 대한 큰 그림을 배운다
% 미분으로 함수의 개략적 성질 파악하는 건 대체로 다들 알겠지?
% 로피탈..

\section*{Exercises}

\begin{prb}
If $\lim_{x\to\infty}f(x)=a$ and $\lim_{x\to\infty}f'(x)=b$, then $a=0$.
\end{prb}

\begin{prb}
Let $f$ be a real $C^2$ function with $f(0)=0$ and $f''(0)\ne0$.
Defined a function $\xi$ such that $f(x)=xf'(\xi(x))$ with $|\xi|\le|x|$, we have $\xi'(0)=1/2$.
\end{prb}

\begin{prb}
Let $f$ be a $C^2$ function such that $f(0)=f(1)=0$.
We have $\|f\|\le\frac18\|f''\|$.
\end{prb}

\begin{prb}
A smooth function such that for each $x$ there is $n$ having the $n$th derivative vanish is a polynomial.
\end{prb}

\begin{prb}
If a real $C^1$ function $f$ satisfies $f(x)\ne0$ for $x$ such that $f'(x)=0$, then in a bounded set there are only finite points at which $f$ vanishes.
\end{prb}

\begin{prb}
Let a real function $f$ be differentiable.
For $a<a'<b<b'$ there exist $a<c<b$ and $a'<c'<b'$ such that $f(b)-f(a)=f'(c)(b-a)$ and $f(b')-f(a')=f'(c')(b'-a')$.
\end{prb}

\begin{prb}
Let $f$ be a differentiable function on the unit closed interval.
If $f(0)=0$ there is $c$ such that $cf'(c)=f(c)$. (Flett)
\end{prb}

\begin{prb}
Let $f$ be a differentiable function on the unit closed interval.
If $f(0)=0$ there is $c$ such that $cf(c)=(1-c)f'(c)$.
\end{prb}


\chapter{Analytic functions}
% 열린 집합과 세트되는 개념이라는 걸 배운다
% 복소수랑 같이 배워서 수렴반경까지 배운다


\section*{Exercises}



\part{Integration}

\chapter{Riemann integration}
% 급수의 극한으로서의 적분을 배운다
% 조각연속 개념이 대체 왜 나왔는지 배운다
% 미적분학의 기본정리를 좀 강조해서 잘 배운다

\section*{Exercises}
\begin{prb}
Find the value of $\lim_{n\to\infty}\frac1n\left(\sum_{k=1}^n\frac1nf\left(\frac kn\right)-\int_0^1f(x)\,dx\right)$.
\end{prb}

\begin{prb}
If $xf'(x)$ is bounded and $x^{-1}\int_0^xf\to L$ then $f(x)\to L$ as $x\to\infty$.
\end{prb}

\chapter{Henstock-Kurzweil intergation}
% 이상적분의 수학적 정의나 표기가 그냥 특수한 약속 정도인 것을 배운다


\chapter{}
% 적분이 함수의 크기를 재는 수단임을 배운다
% 각 함수들의 적분가능성을 배운다
% 급수와 마찬가지로 적분이 수렴하거나 발산하게 되는 메커니즘을 배운다(영역, 함수가 얼마나 특이한지)


\part{Multivariable Calculus}
\chapter{Fre\'chet derivatives}
% 편미분이 그냥 계산도구라는 것을 배운다
% 편미분의 교환도 사실은 항상 다 되는 걸 배운다
% 접공간 개념을 배운다


\section{Inverse function theorem}
% 고정점정리로 함수의 역을 찾아내는 기본 전략을 배운다
% 선형근사의 파워를 배운다


\chapter{Differential forms}
% 텐서곱과 테일러 근사를 배운다
% 웨지곱과 다중적분을 배운다

\chapter{Stokes' theorem}


\end{document}