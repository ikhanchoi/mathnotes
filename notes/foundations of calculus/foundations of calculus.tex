\documentclass{../note}
\usepackage{../../ikany}


\begin{document}
\title{Foundations of Calculus}
\author{Ikhan Choi}
\maketitle
\tableofcontents

\part{Convergence}
\chapter{Sequences}

\section{Limit of sequences}

preserving inequalities
limsup and liminf


\section{Extended real numbers}
\begin{prb}[Operations in the extended real numbers]
We can extend addition (except $\infty+(-\infty)$), subtraction, multiplication (except $\infty\times0$), division (except dividing by zero).
\end{prb}

\begin{prb}[Limits in the extended real numbers]
\end{prb}

\section{Control of the error}

sufficiently large
asymptotic expressions

Approximate sequences and change of limits

\begin{prb}[Change of limits]
\[|a_n-a|\le|a_n-b_{mn}|+|b_{mn}-b_m|+|b_m-a|\]
\[\lim_m\sup_n|a_n-b_{mn}|=0\]
\[\lim_n|b_{mn}-b_m|=0\]
\end{prb}

\[a_n=b_{mn}+c_{mn}\le b_{mn}+\e\]



\section{Bounded sequences}
monotone convergence
Bolzano-Weierstrass


\section*{Exercises}
\begin{prb}
\end{prb}
\begin{prb}[Newton method]
\end{prb}
\section*{Problems}
\begin{enumerate}
\item Show that every real sequence $(a_n)_{n=1}^\infty$ has a subsequence $(a_{n_k})_{k=1}^\infty$ such that $\lim_{k\to\infty}a_{n_k}=\limsup_{n\to\infty}a_n$.
\end{enumerate}





\chapter{Series}

\section{Absolute convergence}
\begin{prb}[Unconditional convergence]
\end{prb}


\section{Convergence tests}

comparison
limit comparison
cauchy condensation
integral....

ratio
root


\begin{prb}[Abel transform]
\[A_k(B_k-B_{k-1})+(A_k-A_{k-1})B_{k-1}=A_kB_k-A_{k-1}B_{k-1}\]
\[\sum_{m<k\le n}A_kb_k=A_nB_n-A_mB_m-\sum_{m<k\le n}a_kB_{k-1}.\]
\end{prb}

abel test
\begin{prb}[Dirichlet test]
\end{prb}


\begin{prb}[Mertens' theorem]
If $\sum_{k=0}^\infty a_k$ converges to $A$ absolutely and $\sum_{k=0}^\infty b_k$ converges to $B$, then their Cauchy product $\sum_{k=0}^\infty c_k$ with $c_k:=\sum_{l=0}^ka_lb_{k-l}$ converges to $AB$.
\begin{parts}
\item We have
\[\lim_{m\to\infty}\sup_n\sum_{k=m+1}^n\sum_{l=n-k+1}^na_kb_l=0.\]
\item We have for each $m$ that
\[\lim_{n\to\infty}\sum_{k=1}^m\sum_{l=n-k+1}^na_kb_l=0\]
\end{parts}
\end{prb}
\begin{pf}
Let
\[A_n:=\sum_{k=0}^na_k,\ B_n:=\sum_{k=0}^nb_k,\quad\text{ and }\quad C_n:=\sum_{k=0}^nc_k.\]


As $m\to\infty$.
\[\Bigl|\sum_{k=m+1}^n\sum_{l=n-k+1}^na_kb_l\Bigr|\le\sum_{k=m+1}^n|a_k|\Bigl|\sum_{l=n-k+1}^nb_l\Bigr|=\sum_{k=m+1}^n|a_k||B_n-B_{n-k}|\lesssim\sum_{k=m+1}^\infty|a_k|\to0.\]

For fixed $m$, as $n\to\infty$,
\[\Bigl|\sum_{k=0}^m\sum_{l=n-k+1}^na_kb_l\Bigr|\le\sum_{k=0}^m|a_k|\Bigl|\sum_{l=n-k+1}^nb_l\Bigr|=\sum_{k=0}^m|a_k||B_n-B_{n-k}|\to\sum_{k=0}^m|a_k||B-B|=0.\]

We will prove
\[A_nB_n-C_n=\sum_{k=0}^n\sum_{l=n-k+1}^na_kb_l\to0\]
as $n\to\infty$.
For $\e>0$, take $m$ such that
\[|\sup_n\sum_{k=m+1}^n\sum_{l=n-k+1}^na_kb_l|<\e.\]
Then for every $n$ we have
\[|\sum_{k=0}^n\sum_{l=n-k+1}^na_kb_l|\le\e+|\sum_{k=0}^n\sum_{l=n-k+1}^na_kb_l|.\]
Taking limits $n\to\infty$ and $\e\to0$ in order, we are done.
\end{pf}




\section*{Exercises}
\begin{prb}[Ces\`aro mean]

\end{prb}


\section*{Problems}
\begin{enumerate}
\item If $a_n\to0$, then $\frac1n\sum_{k=1}^na_k\to0$.
\item If $a_n\ge0$ and $\sum a_n$ diverges, then $\sum\frac{a_n}{1+a_n}$ also diverges.
\item If $a_n\downarrow0$ and $S_n\le1+na_n$, then $S_n\le1$.
\end{enumerate}





\chapter{Metrics and norms}
% 문제의식: 노름과 메트릭 추상화, 동치는 다루지 않음
\section{Metric spaces}
\begin{prb}[Definition of metric spaces]
Let $X$ be a set.
A \emph{metric} is a function $d:X\times X\to\R_{\ge0}$ such that
\begin{parts}[(i)]
\item $d(x,y)=0$ if and only if $x=y$, \hfill(nondegeneracy)
\item $d(x,y)=d(y,x)$ for all $x,y\in X$, \hfill(symmetry)
\item $d(x,z)\le d(x,y)+d(y,z)$ for all $x,y,z\in X$. \hfill(triangle inequality)
\end{parts}
A pair $(X,d)$ of a set $X$ and a metric on $X$ is called a \emph{metric space}.
We often write it simply $X$.
\begin{parts}
\item
A normed space $X$ is a metric space with a metric defined by $d(x,y):=\|x-y\|$.
\item
A subset of a metric space is a metric space with a metric given by restriction.
\end{parts}
\end{prb}

\begin{prb}[System of open balls]
A metric is often misunderstood as something that measures a distance between two points and belongs to the study of geoemtry.
The main function of a metric is to make a system of small balls, sets of points whose distance from specified center points is less than fixed numbers.
The balls centered at each point provide a concrete images of ``system of neighborhoods at a point'' in a more intuitive sense.
In this viewpoint, a metric can be considered as a structure that lets someone accept the notion of neighborhoods more friendly.

Note that taking either $\e$ or $\delta$ in analysis really means taking a ball of the very radius.
Investigation of the distribution of open balls centered at a point is now an important problem.

Let $X$ be a metric space.
A set of the form 
\[\{y\in X:d(x,y)<\e\}\]
for $x\in X$ and $\e>0$ is called an \emph{open ball centered at $x$ with radius $\e$} and denoted by $B(x,\e)$ or $B_\e(x)$.
\end{prb}

\begin{prb}[Convergence and continuity in metric spaces]
Let $\{x_n\}_n$ be a sequence of points on a metric space $(X,d)$.
We say that a point $x$ is a \emph{limit} of the sequence or the sequence \emph{converges to $x$} if for arbitrarily small ball $B(x,\e)$, we can find $n_0$ such that $x_n\in B(x,\e)$ for all $n>n_0$.
If it is satisfied, then we write
\[\lim_{n\to\infty}x_n=x,\]
or simply $x_n\to x$ as $n\to\infty$.
We say a sequence is \emph{convergent} if it converges to a point.
If it does not converge to any points, then we say the sequence \emph{diverges}.

A function $f:X\to Y$ between metric spaces is called \emph{continuous at $x\in X$} if for any ball $B(f(x),\e)\subset Y$, there is a ball $B(x,\delta)\subset X$ such that $f(B(x,\delta))\subset B(f(x),\e)$.
The function $f$ is called \emph{continuous} if it is continuous at every point on $X$.
\begin{parts}
\item A sequence $x_n$ in a metric space $X$ converges to $x\in X$ if and only if $d(x_n,x)$ converges to zero.
\item 
Let $f:X\to Y$ be a function between two metric spaces.
If there is a constant $C$ such that $d(x,y)\le Cd(f(x),f(y))$ for all $x$ and $y$ in $X$, then $f$ is continuous.
In this case, $f$ is particularly called \emph{Lipschitz continuous} with the \emph{Lipschitz constant} $C$.
\end{parts}
\end{prb}


\section{Normed spaces}
banach space

\section{Open sets and closed sets}
convergence, limit point
\section{Compact sets}
\section{Connected sets}



\section*{Exercises}








\part{Real functions}

\chapter{Continuous functions}
% 문제의식: 연속함수의 성질을 엄밀하게 기술할 수 있는 보조정리 증명, 연속함수 공간 개요
\section{Intermediate and extreme value theorems}
\section{Uniform convergence}

\begin{pf}
Divide the error
\[|f(x_n)-f(x)|\le|f(x_n)-f_m(x_n)|+|f_m(x_n)-f_m(x)|+|f_m(x)-f(x)|.\]
Using the uniform convergence, we can take $m$ such that $\|f_m-f\|<\e$, so we have
\[|f(x_n)-f(x)|<\e+|f_m(x_n)-f_m(x)|+\e.\]
Then, taking $\limsup_{n\to\infty}$ on the both-hand sides, we get
\[\limsup_{n\to\infty}|f(x_n)-f(x)|\le\e+0+\e=2\e.\]
Since $\e>0$ has been arbitrarily taken,
\[\lim_{n\to\infty}|f(x_n)-f(x)|=0.\]
\end{pf}

\section{Arzela-Ascoli theorem}
\section{Stone-Weierstrass theorem}

\section*{Exercises}

\section*{Problems}
\begin{enumerate}
\item The set of local minima of a convex real function is connected.
\item Let $f:\R\to\R$ be continuous.
The equation $f(x)=c$ cannot have exactly two solutions for every constant $c\in\R$.
\item A continuous function that takes on no value more than twice takes on some value exactly once.
\item Let $f$ be a function that has the intermediate value property.
If the preimage of every singleton is closed, then $f$ is continuous.
\item* If a sequence of real functions $f_n\colon[0,1]\to[0,1]$ satisfies $|f(x)-f(y)|\le|x-y|$ whenever $|x-y|\ge\frac1n$, then it has a uniformly convergent subsequence.
\end{enumerate}

\chapter{Differentiable functions}
% 로피탈..
\section{Monotonicty and convexity}
\section{Mean value theorem}
Darboux
\section{Taylor theorem}
\section{Differentiable class}
completeness

\section*{Exercises}
\begin{prb}[Variations on the mean value theorem]
Let $f$ be a differentiable function on the unit closed interval.
\begin{parts}
\item If $f(0)=0$ there is $c$ such that $cf'(c)=f(c)$. (Flett)
\item If $f(0)=0$ there is $c$ such that $cf(c)=(1-c)f'(c)$.
\end{parts}
\end{prb}

\begin{prb}[Convergence rates of recursive sequences]
If $a_{n+1}=a_n-f(a_n)$, $f(0)=0$, $f(x)>0$ for $0<x<\e$, $f\in C^2$? then
\[f'(a_n)\sim\lim_{x\to0+}\frac{f'(x)^2}{f''(x)f(x)}\frac1n.\]
\end{prb}

\begin{pf}

\end{pf}

\section*{Problems}
\begin{enumerate}
\item If $\lim_{x\to\infty}f(x)=a$ and $\lim_{x\to\infty}f'(x)=b$, then $a=0$.
\item Let $f$ be a real $C^2$ function with $f(0)=0$ and $f''(0)\ne0$.
Defined a function $\xi$ such that $f(x)=xf'(\xi(x))$ with $|\xi|\le|x|$, we have $\xi'(0)=1/2$.
\item Let $f$ be a $C^2$ function such that $f(0)=f(1)=0$.
We have $\|f\|\le\frac18\|f''\|$.
\item A smooth function such that for each $x$ there is $n$ having the $n$th derivative vanish is a polynomial.
\item If a real $C^1$ function $f$ satisfies $f(x)\ne0$ for $x$ such that $f'(x)=0$, then in a bounded set there are only finite points at which $f$ vanishes.
\item Let a real function $f$ be differentiable.
For $a<a'<b<b'$ there exist $a<c<b$ and $a'<c'<b'$ such that $f(b)-f(a)=f'(c)(b-a)$ and $f(b')-f(a')=f'(c')(b'-a')$.
\item* Let $a_{n+1}=\sin a_n$, $a_n=1$. Show that
\[a_n=\sqrt3n^{-\frac12}-\frac{3\sqrt3}{20}n^{-\frac32}+o(n^{-\frac32}).\]
\end{enumerate}


\chapter{Analytic functions}
\section{Power series}
uniform convergence and absolute convergence, abel theorem?
differentiation
convergence of radius
sum, product, composition, reciprocal?
closed under uniform convergence
\section{Complex analytic functions}
complex domain
(real analytic iff its domain contains real line)
convergence of radius, revisited
identity theorem
\section{Special functions}
hypergeometric, bessel, gamma, zeta

\section*{Exercises}



\part{Integration}

\chapter{Riemann integral}
\section{Riemann integral}
tagged partition
\section{Henstock-Kurzweil intergral}
bounded compact support <-> lebesgue
\section{Improper integral}
\section{Fundamental theorem of calculus for continuous functions}

\section*{Exercises}
\begin{prb}
Find the value of $\lim_{n\to\infty}\frac1n\left(\sum_{k=1}^n\frac1nf\left(\frac kn\right)-\int_0^1f(x)\,dx\right)$.
\end{prb}

\begin{prb}
Find all $a>0$ and $b>0$ such that $\int_0^\infty x^{-b}|\tan x|^a\,dx$ converges.
\end{prb}

\section*{Problems}
\begin{enumerate}
\item* If $xf'(x)$ is bounded and $x^{-1}\int_0^xf\to L$ then $f(x)\to L$ as $x\to\infty$.
\end{enumerate}


\chapter{Integrable functions}
\section{}
% 적분이 함수의 크기를 재는 수단임을 배운다
% 각 함수들의 적분가능성을 배운다
% 급수와 마찬가지로 적분이 수렴하거나 발산하게 되는 메커니즘을 배운다(영역, 함수가 얼마나 특이한지)


\chapter{}





\part{Multivariable Calculus}
\chapter{Fre\'chet derivatives}
% 편미분이 그냥 계산도구라는 것을 배운다
% 편미분의 교환도 사실은 항상 다 되는 걸... 배울 수 있나
% 접공간 개념을 배운다
\section{Tangent spaces}
\begin{prb}[Vector fields]

\end{prb}

\section{Inverse function theorem}






\chapter{Differential forms}
% 다음 두 단원 동안: 좌표 변환, 미분 형식, 적분 정의
\section{Multilinear algebra}

\begin{prb}[Tensor product]
\end{prb}

\begin{prb}[Wedge product]
\end{prb}



\begin{prb}[One-forms]
\end{prb}


\begin{prb}[Multiple integral]
volume forms,
stone weierstrass and fubini
\end{prb}



\section{Vector calculus}

\begin{prb}[Exterior derivative]
\end{prb}

\begin{prb}[Musical isomorphisms]
\end{prb}

\begin{prb}[Inner product of differential forms]
ONB
\end{prb}

\begin{prb}[Hodge star operator]
Identification of 2-forms and vector fields
\end{prb}

\begin{prb}[Gradient, curl, and divergence]
\end{prb}

\begin{prb}[Potentials]
\end{prb}

\begin{prb}[Vector calculus identities]
\end{prb}


\section*{Exercises}

\begin{prb}[Multivariable Taylor's theorem]
Symmetric product
\end{prb}

\begin{prb}[Vector analysis in two dimension]
\end{prb}

\begin{prb}[Geometric algebra]
\end{prb}






\chapter{Stokes theorems}

\section{Local coordinates}

\begin{prb}[Spherical coordinates]
Let $U=\R^3\setminus\{\,(x,y,z):x=0,\ y\ge0\,\}$.
\[(x,y,z)=(r\sin\theta\cos\f,r\sin\theta\sin\f,r\cos\theta)\]
for $(r,\theta,\f)\in(0,\infty)\times(0,\pi)\times(0,2\pi)$.
Orthonormal bases are
\[\left(\pd_r,\ \frac1r\pd_\theta,\ \frac1{r\sin\theta}\pd_\f\right),\]
\[(dr,\ r\,d\theta,\ r\sin\theta\,d\f),\]
\[(r^2\sin\theta\,d\theta\wedge d\f,\ r\sin\theta\,d\f\wedge dr,\ r\,dr\wedge d\theta).\]
\begin{parts}
\item
\item The Laplacian is given by
\[\Delta f=\frac1{r^2}\pd{r}\left(r^2\pd{f}{r}\right)+\frac1{r^2\sin\theta}\pd{\theta}\left(\sin\theta\pd{f}{\theta}\right)+\frac1{r^2\sin^2\theta}\pd[2]{f}{\f}.\]
\end{parts}
\end{prb}
\begin{pf}
Write $df$ in the orthonormal basis
\begin{align*}
df&=\pd{f}{r}\,dr+\pd{f}{\theta}\,d\theta+\pd{f}{\f}\,d\f\\
&=\left(\pd{f}{r}\right)\,dr+\left(\frac1r\pd{f}{\theta}\right)\,r\,d\theta+\left(\frac1{r\sin\theta}\pd{f}{\f}\right)\,r\sin\theta\,d\f.
\end{align*}
After taking the Hodge star operator
\begin{align*}
{}*df&=\left(\pd{f}{r}\right)\,r^2\sin\theta\,d\theta\wedge d\f+\left(\frac1r\pd{f}{\theta}\right)\,r\sin\theta\,d\f\wedge dr+\left(\frac1{r\sin\theta}\pd{f}{\f}\right)\,r\,dr\wedge d\theta\\
&=r^2\sin\theta\pd{f}{r}\,d\theta\wedge d\f+\sin\theta\pd{f}{\theta}\,d\f\wedge dr+\frac1{\sin\theta}\pd{f}{\f}\,dr\wedge\theta,
\end{align*}
the differential is computed as
\begin{align*}
d*df&=d\left(r^2\sin\theta\pd{f}{r}\right)\,d\theta\wedge d\f+d\left(\sin\theta\pd{f}{\theta}\right)\,d\f\wedge dr+d\left(\frac1{\sin\theta}\pd{f}{\f}\right)\,dr\wedge\theta\\
&=\left[\sin\theta\pd{r}\left(r^2\pd{f}{r}\right)+\pd{\theta}\left(\sin\theta\pd{f}{\theta}\right)+\frac1{\sin\theta}\pd[2]{f}{\f}\right]\,dr\wedge d\theta\wedge d\f,
\end{align*}
so that we have
\begin{align*}
\Delta f={}*d*df&=\frac1{r^2\sin\theta}\left[\sin\theta\pd{r}\left(r^2\pd{f}{r}\right)+\pd{\theta}\left(\sin\theta\pd{f}{\theta}\right)+\frac1{\sin\theta}\pd[2]{f}{\f}\right]\\
&=\frac1{r^2}\pd{r}\left(r^2\pd{f}{r}\right)+\frac1{r^2\sin\theta}\pd{\theta}\left(\sin\theta\pd{f}{\theta}\right)+\frac1{r^2\sin^2\theta}\pd[2]{f}{\f}
\end{align*}

\end{pf}





\section{Integration on curves and surfaces}

\begin{prb}[Line integral]
\end{prb}

\begin{prb}[Surface integral]
\end{prb}


\section{Stokes theorems}
% 미분기하보다 피디이스럽게
\begin{prb}[Bump functions]
\end{prb}

\begin{prb}[Partition of unity]
\end{prb}

\begin{prb}
\end{prb}


\end{document}