\documentclass{../note}
\usepackage{../../ikany}


\begin{document}
\title{General Topology}
\author{Ikhan Choi}
\maketitle
\tableofcontents

\part{Topological spaces}


\chapter{Topologies}

\section{Filters and bases}
\section{Open sets and closed sets}
\section{Interior, closure, and boundary}


\section*{Exercises}

\begin{prb}
If $A^\circ\subset B$ and $B$ is closed, then $(A\cup B)^\circ\subset B$.
\end{prb}




\chapter{Fundamental constructions}
\section{Subspace topology}
\section{Quotient topology}
\section{Product topology}



\chapter{Nets and sequences}
\section{Nets and subnets}
\section{Eventuality filters}
\section{Sequential spaces}

\begin{prb}
Let $X$ be a topological space.
A subset $U$ is called \emph{sequentially open} in $X$ if every sequence that converges to a point in $U$ is eventually in $U$.
A \emph{sequential space} is a topological space in which every sequentially open set is open.
\begin{parts}
\item Every open set is sequentially open.
\item
\end{parts}
\end{prb}

\begin{prb}[Sequential continuity]
Let $f:X\to Y$ be a map between topological spaces.
\begin{parts}
\item If $X$ is sequential, then the sequential continuity implies the continuity of $f$.
\end{parts}
\end{prb}
\begin{pf}
Suppose $f$ is not continuous so that there is an open set $V$ in $Y$ such that $f^{-1}(V)$ is not open.
Since $X$ is sequential, $f^{-1}(Y)$ is not sequentially open.
Take a seuqence $x_n$ in $X\setminus f^{-1}(V)$ that converges to $x\in f^{-1}(V)$.
Then, $f(x_n)$ dose not converges since $f(x_n)\notin V$ and $f(x)\in V$, so $f$ is not sequentially continuous.
\end{pf}





\part{Topological properties}
\chapter{Compactness}
\section{Compact spaces}
\section{Sequential compactness}
\section{Local compactness}
\section{Tychonoff theorem}


\chapter{Connectedness}
\section{Connected spaces}
\section{Path connectedness}
\section{Local connectedness}



\chapter{Separation axioms}
\section{Separation axioms}
subspace and product


\section{Normal spaces}
not for product

compact Hausdorff
order topology
metric topology?

\section{Urysohn lemma and Tietze extension}



\section*{Exercises}


\begin{itemize}
\item A second countable locally compact space is $\sigma$-compact.
\item A locally compact Hausdorff space is completely regular.
\end{itemize}


\part{Topological structures}

\chapter{Metric spaces}

\section{Equivalence of metrics}
\begin{prb}[Comparison of metrics]
Let $d_1$ and $d_2$ be metrics on a set $X$.
We say that $d_1$ is \emph{stronger than} $d_2$ (equivalently, $d_2$ is \emph{weaker than} $d_1$) or $d_1$ \emph{refines} $d_2$ if for any $x\in X$ and $\e>0$, there is $\delta>0$ such that
\[B_1(x,\delta)\subset B_2(x,\e),\]
where $B_1$ and $B_2$ refer to balls within the metrics $d_1$ and $d_2$ respectively.
\begin{parts}
\item
This refinement relation is a preorder.
\item
$d_1$ is stronger than $d_2$ if and only if every sequence that converges to $x\in X$ in $d_1$ converges to $x$ in $d_2$.
\item
$d_1$ is stronger than $d_2$ if and only if the identity map $\id:(X,d_1)\to(X,d_2)$ is continuous.
\end{parts}
\end{prb}
\begin{pf}
(a)
It is enough to show the transitivity.
Suppose there are three metric $d_1$, $d_2$, and $d_3$ on a set $X$ such that $d_1$ is stronger than $d_2$ and $d_2$ is stronger than $d_3$.
For $i=1,2,3$, let $B_i$ be a notation for the balls defined with the metric $d_i$.

Take $x\in X$ and $\e>0$ arbitrarily.
Then, we can find $\e'>0$ such that
\[B_2(x,\e')\subset B_3(x,\e).\]
Also, we can find $\delta>0$ such that
\[B_1(x,\delta)\subset B_2(x,\e').\]
Therefore, we have $B_1(x,\delta)\subset B_3(x,\e)$ which implies that $d_1$ refines $d_3$.

(b)
($\Rightarrow$)
Let $\{x_n\}_n$ be a sequence in $X$ that converges to $x$ in $d_1$.
By the assumption, for an arbitrary ball $B_2(x,\e)=\{y:d_2(x,y)<\e\}$, there is $\delta>0$ such that
\[B_1(x,\delta)\subset B_2(x,\e),\]
where $B_1(x,\delta)=\{y:d_1(x,y)<\delta\}$.
Since $\{x_n\}_n$ converges to $x$ in $d_1$, there is an integer $n_0$ such that
\[n>n_0\impl x_n\in B_1(x,\delta).\]
Combining them, we obtain an integer $n_0$ such that
\[n>n_0\impl x_n\in B_2(x,\e).\]
It means $\{x_n\}$ converges to $x$ in the metric $d_2$.

($\Leftarrow$)
We prove it by contradiction.
Assume that for some point $x\in X$ we can find $\e_0>0$ such that there is no $\delta>0$ satisfying $B_1(x,\delta)\subset B_2(x,\e_0)$.
In other words, at the point $x$, the difference set $B_1(x,\delta)\setminus B_2(x,\e_0)$ is not empty for every $\delta>0$.
Thus, we can choose $x_n$ to be a point such that
\[x_n\in B_1\left(x,\tfrac1n\right)\setminus B_2(x,\e_0)\]
for each positive integer $n$ by putting $\delta=\frac1n$.

We claim $\{x_n\}_n$ converges to $x$ in $d_1$ but not in $d_2$.
For $\e>0$, if we let $n_0=\ceil{\frac1\e}$ so that we have $\frac1{n_0}\le\e$, then
\[n>n_0\impl x_n\in B_1\left(x,\tfrac1n\right)\subset B_1(x,\e).\]
So $\{x_n\}_n$ converges to $x$ in $d_1$.
However in $d_2$, for $\e=\e_0$, we can find such $n_0$ like $d_1$ since
\[x_n\notin B_2(x,\e_0)\]
for every $n$.
Therefore, $\{x_n\}$ does not converges to $x$ in $d_2$.
\end{pf}

\begin{prb}[Equivalence of metrics]
Let $d_1$ and $d_2$ be metrics on a set $X$.
They are said to be \emph{(topologically) equivalent} if they refines each other.
\begin{parts}
\item
$d_1$ and $d_2$ are equivalent if for each $x$ in $X$ there exist two constants $C_1$ and $C_2$ such that $d_2(x,y)\le C_1d_1(x,y)$ and $d_1(x,y)\le C_2d_2(x,y)$ for all $y$ in $X$.
\item
$d_1$ and $d_2$ are equivalent if $d_2=f\circ d_1$ for a monotonically increasing $f:\R_{\ge0}\to\R_{\ge0}$ that is continuous at $0$.
\end{parts}
\end{prb}
\begin{pf}
(a)
Let $d_1$ and $d_2$ be metrics on a set $X$.
Suppose for each point $x$ there exists a constant $C$ which may depend on $x$ such that $d_2(x,y)\le Cd_1(x,y)$ for all $y\in Y$.
We will show $d_1$ is stronger than $d_2$.

(b)
For any ball $B_1(x,\e)$, we have a smaller ball
\[B_2(x,f(\e))\subset B_1(x,\e)\]
since $f(d(x,y))<f(\e)$ implies $d(x,y)<\e$.
Conversely, take an arbitrary ball $B_2(x,\e)$.
Since $f$ is continuous at 0, we can find $\delta>0$ such that
\[d(x,y)<\delta\impl f(d(x,y))<\e,\]
which implies $B_1(x,\delta)\subset B_2(x,\e)$.
\end{pf}






A metric can be viewed as a function that takes a sequence as input and returns whether the sequence converges or diverges.
That is, a metric acts like a criterion which decides convergence of sequences.
Take note on the fact that the sequence of real numbers defined by $x_n=\frac1n$ converges in standard metric but diverges in discrete metric.
Like this example, even for the same sequence on a same set, the convergence depends on the attached metrics.
What we are interested in is comparison of metrics and to find a proper relation structure.
If a sequence converges in a metric $d_2$ but diverges in another metric $d_1$, we would say $d_1$ has stronger rules to decide the convergence.
Refinement relation formalizes the idea.

Unlike metrics, there exist two different topologies that have same sequential convergence data.
For example, a sequence in an uncountable set with cocountable topology converges to a point if and only if it is eventually at the point, which is same with discrete topology.
This means the informations of sequence convergence are not sufficient to uniquely characterize a topology.
Instead, convergence data of generalized sequences also called nets, recover the whole topology.
For topologies having a property called the first countability, it is enough to consider only usual sequences in spite of nets.
What we did in this subsection is not useless because topology induced from metric is a typical example of first countable topologies.
These kinds of problems will be profoundly treated in Chapter 3.

One can ask some results for the equivalence of metrics characterized by a same set of continuous functions.
However, they are generally difficult problems: is it possible to recover the base space from a continuous function space or a path space?


Topologies are occasionally described by not a single but several metrics.
It provides a useful method to construct a metric or topology, which can be applied to a quite wide range of applications.
Specifically, in a conventional way, metrics can be summed or taken maximum to make another metric out of olds.
The following proposition can be easily generalized to an arbitrary finite number of metrics by mathematical induction.

\begin{prop}
Let $d_1$ and $d_2$ be metrics on a set $X$.
For a sequecne $\{x_n\}_n$ in $X$, the following statements are all equivalent:
\begin{parts}
\item it converges to $x$ in both $d_1$ and $d_2$,
\item it converges to $x$ in $d_1+d_2$,
\item it converges to $x$ in $\max\{d_1,d_2\}$.
\end{parts}
In particular, the metrics $d_1+d_2$ and $\max\{d_1,d_2\}$ are equivalent.
\end{prop}
\begin{pf}
We skip to prove $d_1+d_2$ and $\max\{d_1,d_2\}$ are metrics.

(b) or (c)$\Rightarrow$(a)
The inequalities $d_i\le d_1+d_2$ and $d_i\le\max\{d_1,d_2\}$ imply the desired results.

(a)$\Rightarrow$(b)
For $\e>0$, we may find positive integers $n_1$ and $n_2$ such that $n>n_1$ and $n>n_2$ imply $d_1(x_n,x)<\frac\e2$ and $d_2(x_n,x)<\frac\e2$ respectively.
If we define $n_0:=\max\{n_1,n_2\}$, then
\[n>n_0\impl d_1(x_n,x)+d_2(x_n,x)<\e.\]

(a)$\Rightarrow$(c)
Take $n_0$ as we did previously.
Then,
\[n>n_0\impl\max\{d_1(x_n,x),d_2(x_n,x)\}<\frac\e2<\e.\qedhere\]
\end{pf}

\begin{rmk}
In general, for any norm $\|\cdot\|$ on $\R^2$, the function $\|(d_1,d_2)\|$ defines another equivalent metric.
\end{rmk}

There is also a method for combining not only finite family of metrics, but also infinite family of metrics.
Since the sum of infinitely many positive numbers may diverges to infinity, we cannot sum the metrics directly.
The strategy is to ``bound'' the metrics.
We call a metric bounded when the range of metric function is bounded.

\begin{prop}
Every metric possesses an equivalent bounded metric.
\end{prop}
\begin{pf}
Let $d$ be a metric on a set.
Let $f$ be a bounded, monotonically increasing, and subadditive function on $\R_{\ge0}$ that is continuous at 0 and satisfies $f^{-1}(0)=\{0\}$.
The mostly used examples are
\[f(x)=\frac x{1+x}\quad\text{and}\quad f(x)=\min\{x,1\}.\]
Then, $f\circ d$ is a bounded metric equivalent to $d$ by Example 1.4.
\end{pf}

\begin{defn}
Let $d$ be a metric on a set $X$.
A \emph{standard bounded metric} means either metric
\[\min\{d,1\}\quad\text{or}\quad\frac d{d+1},\]
and we will denote it by $\hat d$.
\end{defn}

\begin{prop}
Let $\{d_i\}_{i\in\N}$ be a countable family of metrics on a set $X$.
For a sequence $\{x_n\}_n$ in $X$, the following statements are all equivalent:
\begin{parts}
\item it converges in $d_i$ for every $i$,
\item it converges in a metric
\[d(x,y):=\sum_{i\in\N}\,2^{-i}\hat d_i(x,y),\]
\item it converges in a metric
\[d'(x,y):=\sup_{i\in\N}\,i^{-1}\hat d_i(x,y).\]
\end{parts}
In particular, the metrics $d$ and $d'$ are equivalent.
\end{prop}
\begin{pf}
The functions $d$ and $d'$ in (b) and (c) are well-defined by the monotone convergence theorem and the least upper bound property.
We skip checking for them to satisfy the triangle inequality and be metrics.

(b) or (c)$\Rightarrow$(a)
We have ineuqalities $\hat d_i\le2^id$ and $\hat d_i\le id'$ for each $i$, so convergence in $d$ or $d'$ implies the convergence in each $\hat d_i$.
The equivalence of $\hat d_i$ and $d_i$ implies the desired result.

(a)$\Rightarrow$(b)
Suppose a sequence $\{x_n\}_n$ converges to a point $x$ in $d_i$ for every index $i$.
Take an arbitrary small ball $B(x,\e)=\{y:d(x,y)<\e\}$ with metric $d$.
By the assumption, we can find $n_i$ for each $i$ satisfying
\[n>n_i\impl\hat d_i(x_n,x)<\frac\e2.\]
Define $k:=\lceil1-\log_2\e\rceil$ so that we have $2^{-k}\le\frac\e2$.
With this $k$, define
\[n_0:=\max_{1\le i\le k}n_i.\]
If $n>n_0$, then
\begin{align*}
d(x_n,x)&=\sum_{i=1}^k2^{-i}\hat d_i(x_n,x)+\sum_{i=k+1}^\infty2^{-i}\hat d_i(x_n,x)\\
&<\sum_{i=1}^k2^{-i}\frac\e2+\sum_{i=k+1}^\infty2^{-i}\\
&<\frac\e2+2^{-k}\le\e,
\end{align*}
so $x_n$ converges to $x$ in the metric $d$.

(a)$\Rightarrow$(c)
Suppose a sequence $\{x_n\}_n$ converges to a point $x$ in each $d_i$, and take an arbitrary small ball $B(x,\e)=\{y:d(x,y)<\e\}$ with metric $d$.
By the assumption, we can find $n_i$ for each $i$ satisfying
\[n>n_i\impl\hat d_i(x_n,x)<\e.\]
Define $k:=\lceil\frac1\e\rceil$ so that we have $k^{-1}\le\e$.
With this $k$, define
\[n_0:=\max_{1\le i\le k}n_i.\]
If $n>n_0$, then
\[i^{-1}\hat d_i(x,y)\le\hat d_i(x,y)<\e\quad\text{for}\quad i\le k\]
and
\[i^{-1}\hat d_i(x,y)\le i^{-1}<k^{-1}\le\e\quad\text{for}\quad i>k\]
imply $d(x_n,x)<\e$, which means that $x_n$ converges to $x$ in the metric $d$.
\end{pf}

Combination of uncountably many metrics does not result in a single metric, but a topology which cannot be induced from a metric in general.
It will be discussed in the rest of the note.

\begin{rmk}
A metric
\[d''(x,y)=\sup_{i\in\N}\,d_i(x,y)\]
is not used because the convergence in this metric is a stronger condition than the convergence with respect to each metric $d_i$.
In other words, this metric generates a finer(stronger) topology than the topology generated by subbase of balls.
For example, the topology on $\R^\N$ generated by this metric defined with the projection pseudometrics is exactly what we often call the box topology.
\end{rmk}


We can also form a metric by summation of generalized metrics, called pseudometrics, which is defined by missing the nondegeneracy condition from the original definition of metric.

\begin{defn}
A function $\rho:X\times X\to\R_{\ge0}$ is called a \emph{pseudometric} if
\begin{parts}
\item $\rho(x,x)=0$ for all $x\in X$,
\item $\rho(x,y)=\rho(y,x)$ for all $x,y\in X$, \hfill(symmetry)
\item $\rho(x,z)\le \rho(x,y)+\rho(y,z)$ for all $x,y,z\in X$. \hfill(triangle inequality)
\end{parts}
\end{defn}

For pseudometrics, it is possible to duplicate every definition we studied in metric spaces: convergence of a sequence, continuity between a set endowed with a pseudometric, refinement and equivalence relations, and countable sum of bounded pseudometrics to make a new pseudometric.
Furthermore, every statement for metrics can be generalized to pseudometrics since we have not actually used the condition that $d(x,y)=0$ implies $x=y$.
In fact, we have a flaw that the limit of a convergent sequence may not be unique within a pseudometric.

\begin{ex}
Let $\rho(x,y)=\rho((x_1,x_2),(y_1,y_2))=|x_1-y_1|$ be a pseudometric on $\R^2$.
Consider a sequence $\{(\frac1n,0)\}_n$.
Since $(0,c)$ satisfies 
\[\rho((\tfrac1n,0),(0,c))=\tfrac1n\to0\quad\text{as}\quad n\to\infty\]
for any real number $c$, the sequence converges to $(x_1,x_2)$ if and only if $x_1=0$.
\end{ex}

Although sequences may have several limits in each pseudometric, the sum of a family of pseudometrics can allow the sequences to have at most one limit, only if the sum satisfies the axioms of a metric.

\begin{defn}
A family of pseudometrics $\{\rho_\alpha\}_\alpha$ on a set $X$ is said to \emph{separate points} if the condition 
\[\rho_\alpha(x,y)=0\quad\text{for all $\alpha$}\]
implies $x=y$.
\end{defn}
\begin{prop}
\begin{parts}
\item
A finite family of pseudometrics $\{\rho_i\}_{i=1}^N$ separates points if and only if the pseudometric $\rho:=\sum_{i=1}^N\rho_i$ is a metric.
\item
A countable family of pseudometrics $\{\rho_i\}_{i\in\N}$ separates points if and only if the pseudometric defined by
\[\rho:=\sum_{i\in\N}2^{-i}\tilde\rho_i\quad\text{or}\quad\sup_{i\in\N}i^{-1}\tilde\rho_i,\]
where $\tilde\rho_i$ is either $\min\{\rho_i,1\}$ or $\rho_i/(\rho_i+1)$, is a metric.
\end{parts}
\end{prop}



\section{Metrization theorems}

Urysohn metrization theorem: regular second countable

Paracompactness
Dieudonne's theorem: paracompact Hausdorff is normal.


\section{Polish spaces}





\section*{Exercises}

\begin{prb}[Discrete metrics]
Let $X$ be a set, and define a metric as
\[d(x,y):=\begin{cases}0&,x=y\\1&,x\ne y\end{cases}.\]
This metric is called \emph{discrete}.
\begin{parts}
\item
The discrete metric is a strongest metric on $X$.
\item 
\item
A sequence $x_n$ converges to a point $x$ in a discrete metric space if and only if there is $n_0$ such that $n\ge n_0$ implies $x_n=x$.
\end{parts}
\end{prb}

\begin{prb}
\begin{parts}
\item
Let $d$ be a metric on a set $X$.
Let $f:\R_{\ge0}\to\R_{\ge0}$ be a function such that $f^{-1}(0)=\{0\}$.
If $f$ is monotonically increasing and subadditive, then $f\circ d$ satisfies the triangle inequality, hence is another metric on $X$.
Note that a function $f$ is called \emph{subadditive} if $f(x+y)\le f(x)+f(y)$ for all $x,y$ in the domain.
\item
Let $G=(V,E)$ be a connected graph.
Define $d:V\times V\to\Z_{\ge0}\subset\R_{\ge0}$ as the distance of two vertices; the length of shortest path connecting two vertices.
Then, $(V,d)$ is a metric space.
\item
Let $\cP(X)$ be the power set of a finite set $X$.
Define $d:\cP(X)\times\cP(X)\to\Z_{\ge0}\subset\R_{\ge0}$ as the cardinality of the symmetric difference; $d(A,B):=|(A-B)\cup(B-A)|$.
Then $(\cP(X),d)$ is a metric space.
\item
Let $C$ be the set of all compact subsets of $\R^d$.
Recall that a subset of $\R^d$ is compact if and only if it is closed and bounded.
Then, $d:C\times C\to\R_{\ge0}$ defined by
\[d(A,B):=\max\{\,\sup_{a\in A}\inf_{b\in B}\|a-b\|,\ \sup_{b\in B}\inf_{a\in A}\|a-b\|\,\}\]
is a metric on $C$.
It is a little special case of \emph{Hausdorff metric}.
\end{parts}
\end{prb}


\begin{prb}[Kuratowski embedding]
While every subset of a normed space is a metric space, we have a converse statement that every metric space is in fact realized as a subset of a normed space.
Let $X$ be a metric space, and denote by $C_b(X)$ the space of continuous and bounded real-valued functions on $X$ with uniform norm given by
\[\|f\|=\sup_{x\in X}|f(x)|.\]
Fix a point $p\in X$, which will serve as the origin.
\begin{parts}
\item Show that a map $\phi:X\to C_b(X)$ such that
\[[\phi(x)](t)=d(x,t)-d(p,t)\]
is well-defined.
\item Show that the map $\phi$ is an isometry; $d(x,y)=\|\phi(x)-\phi(y)\|$.
\end{parts}
\end{prb}

\begin{prb}[Equivalence of norms in finite dimension]
Let $V$ be a vector space of dimension $d$ over $\F=\R$ or $\C$.
Fix a basis $\{e_i\}_{i=1}^d$ on $V$ and let $x=\sum_{i=1}^dx_ie_i$ denote an arbitrary element of $V$.
We will prove all norms are equivalent to the standard Euclidean norm defined for this fixed basis:
\[\|x\|_2:=(\sum_{i=1}^d|x_i|^2)^{\frac12}.\]
With this standard norm any theorems studied in elementary analysis including the Bolzano-Weierstrass theorem are allowed to be applied.
Take a norm $\|\cdot\|$ on $V$ which may differ to $\|\cdot\|_2$.
\begin{parts}
\item Find a constant $C_2$ such that $\|x\|\le C_2\|x\|_2$ for all $x\in V$.
\item Show that if no constant $C$ satisfies $\|x\|_2\le C\|x\|$, then there exists a sequence $\{x_n\}_n$ such that $\|x_n\|_2=1$ and $\|x_n\|<\tfrac1n$.
\item Show that there exists a constant $C$ satisfies $\|x\|_2\le C\|x\|$.
\end{parts}
\end{prb}


A $\sigma$-compact locally compact space is paracompact.
A second countable regular space is paracompact.






\chapter{Uniform spaces}

\section{Uniformities and pseudometrics}


\section{Completely regular spaces}
uniformizability
Stone-C\v{e}ch compactification

\section{Cauchy structure}
completion

\section{Relative compactness}
compact uniform space is complete
relatively compact

\section*{Exercises}
\begin{prb}
Even for a first countable uniform space, completeness and sequential completeness are not equivalent.
\end{prb}










\chapter{Topological groups}

\section{Topological properties}
uniform structures
Birkhoff-Kakutani
metrizability

\section{Topological group actions}














\part{Continuous Function Spaces}
\chapter{Compact-open topology}

\section{Topology by convergence}

Topologies on $C_c(X)$ for LCH $X$: weaker to stronger
\begin{parts}
\item Topology of compact convergence: $\bar{C_c(X)}=\bar{C_{\mathrm{loc}}(X)}=C_{\mathrm{loc}}(X)$.
\item Topology of uniform convergence: $\bar{C_c(X)}=C_0(X)$, $C_c(X)^*=M(X)$.
\item Inductive topology: $\bar{C_c(X)}:=\bar{\colim_{U\Subset X}C_c(U)}=C_c(X)$.
\end{parts}
The space $C_{\mathrm{loc}}(X)$ is defined to be $C(X)$ as a set endowed with the topology of compact convergence.





\chapter{Approximation of continuous functions}
\section{Arzela-Ascoli theorem}

\section{Stone-Weierstrass theorem}

\begin{prb}[Bernstein polynomial]
We want to show $\R[x]$ is dense in $C([0,1],\R)$.
Let $f\in C([0,1],\R)$ and define \emph{Berstein polynomials} $B_n(f)\in\R[x]$ for each $n$ such that
\[B_n(f)(x):=\sum_{k=0}^nf\left(\frac kn\right)\binom nkx^k(1-x)^{n-k}.\]
\begin{parts}
\item $B_n(f)$ uniformly converges to $f$ on $[0,1]$.
\item There is a sequence $p_n\in\R[x]$ with $p_n(0)=0$ uniformly convergent to $x\mapsto|x|$ on $[-1,1]$.
\end{parts}
\end{prb}
\begin{pf}
(b)
Let
\[B_n(x):=\sum_{k=0}^n\left|1-\frac{2k}n\right|\binom nk(1-2x)^k(2x-1)^{n-k}.\]
Since $B_n(x)\to|x|$ uniformly on $[-1,1]$ and $B_n(0)\to0$, we have $B_n(x)-B_n(0)\to|x|$ uniformly on $[-1,1]$.
\end{pf}

\begin{prb}[Taylor series of square root]
We want to show the absolute value is approximated by polynomials in $C([-1,1],\R)$ in another way.
Let
\[f_n(x):=\sum_{k=0}^n a_k(x-1)^k\]
be the partial sum of the Taylor series of the square root function $\sqrt x$ at $x=1$.
\begin{parts}
\item By Abel's theorem, $f_n$ uniformly converges to $\sqrt x$ on $[0,1]$
\item There is a sequence $p_n\in\R[x]$ with $p_n(0)=0$ uniformly convergent to $x\mapsto|x|$ on $[-1,1]$.
\end{parts}
\end{prb}


\begin{prb}[Proof of Stone-Weierstrass theorem]
Let $X$ be a compact Hausdorff space and $S\subset C(X,\R)$.
We say that $S$ \emph{separates points} if for every distinct $x$ and $y$ in $X$ there is $f\in S$ such that $f(x)\ne f(y)$, and that $S$ \emph{vanishes nowhere} if for every $x$ in $X$ there is $f\in S$ such that $f(x)\ne0$.

Let $\cA=\bar{S\R[S]}$ be the real Banach subalgebra of $C(X,\R)$ generated by $S$.
\begin{parts}
\item $\cA$ is a lattice.
\item $\cA$ is equal to $C(X,\R)$
\end{parts}
\end{prb}




Locally compact version,
complex version








\begin{prb}
Some examples
\begin{parts}
\item $z\R[z]$ is dense in $C([1,2],\R)$.
\item $\C[z]$ is dense in $C([0,1],\C)$.
\item $z\C[z,\bar z]$ is dense in $C(\T,\C)$.
\end{parts}
\end{prb}












\chapter{Topological measures}
\section{Baire and Borel measures}
locally finite Baire measures can define a linear functional on $C_c$.


\section{Measures on metric spaces}

\begin{prb}[Regularity of Borel measures]
Let $X$ be a topological space.
We confine our discusssion of regularity only on Borel measures.
For a Borel measure $\mu$ on $X$, we say $\mu$ is \emph{outer regular} if
\[\mu(E)=\inf\{\,\mu(U):E\subset U,\ U\text{ open}\,\},\]
and say $\mu$ is \emph{inner regular} if
\[\mu(E)=\sup\{\,\mu(F):F\subset E,\ F\text{ closed}\,\}.\]
Let $\mu$ be a regular Borel measure on $X$.
\begin{parts}
\item If $E$ is $\sigma$-finite, then for any $\e>0$ there are open $U$ and closed $F$ such that $F\subset E\subset U$ and $\mu(U\setminus F)<\e$.
\end{parts}
\end{prb}





\begin{prb}[Finiteness and regularity]
On which spaces are all Borel measures regular or Radon or etc?

\begin{parts}
\item In a topological space in which every closed set is $G_\delta$, every finite measure is regular.
\end{parts}
\end{prb}





\begin{prb}[Approximation by continuous functions]
Let $\mu$ be a regular Borel measure on a topological space $X$.
\begin{parts}
\item If $X$ is normal, then $C_b(X)\cap L^p(\mu)$ is dense in $L^p(\mu)$ for $1\le p<\infty$.
\item If $X$ is locally compact Hausdorff, then $C_c(X)$ is dense in $L^p(\mu)$ for $1\le p<\infty$.
\end{parts}
\end{prb}
\begin{pf}
(a)
Let $\e>0$ and suppose $E\subset X$ is measurable with $\mu(E)<\infty$.
Since $E$ is $\sigma$-finite, we have open $U$ and closed $F$ such that $F\subset E\subset U$ and $\mu(U\setminus F)<\e/2$.
By the Urysohn lemma, there is a continuous function $g:X\to[0,1]$ such that $g|_{U^c}=0$ and $g|_F=1$.
Then,
\[\int|\1_E-g|\,d\mu=\int_{U\setminus F}|\1_E-g|\,d\mu\le2\mu(U\setminus K)<\e.\]

(b)
\end{pf}


\begin{prb}[Luzin's theorem]
The theorem of Luzin is about the continuous extension.
One of the frequent usages is to approximate indicator functions with continuous functions.

Let $\mu$ be a regular Borel measure on a topological space $X$.
Let $f:X\to\R$ be a Borel measurable function.
\begin{parts}
\item If $\mu$ is $\sigma$-finite, then for any $\e>0$ there is closed $F\subset X$ such that $f|_F:F\to\R$ is continuous.
\item If $X$ is normal and $\mu$ is $\sigma$-finite, then for any $\e>0$ there is closed $F\subset X$ and continuous $g:X\to\R$ such that $f|_F=g|_F$ and $\mu(X\setminus F)<\e$.
\item If $X$ is locally compact Hausdorff and $\mu$ is $\sigma$-finite, then for any $\e>0$ there is compact $K\subset X$ and continuous $g:X\to\R$ such that $f|_K=g|_K$ and $\mu(X\setminus K)<\e$.
\item For the both above cases, if $f$ is bounded, then $g$ also can be taken to be bounded.
\end{parts}
\end{prb}
\begin{pf}
(a)
First we assume $f$ is bounded so that $f\in L^1(\mu)$.
Since $\R$ is second countable, we have a base $(V_n)_{n=1}^\infty$ of $\R$.
Since $\mu$ is $\sigma$-finite, for each $n$ we can take open $U_n$ and closed $F_n$ such that $F_n\subset f^{-1}(V_n)\subset U_n$ and $\mu(U_n\setminus F_n)<\e/2^n$.
Define $F:=\left(\bigcup_{n=1}^\infty U_n\setminus F_n\right)^c$ so that $\mu(X\setminus F)<\e$ and $F$ is closed.
Then, the identity $f^{-1}(V_n)\cap F=U_n\cap F$ proves the continuity of $f|_F$.

(b)
Since $\mu$ is regular, with the density of continuous functions, construct a sequence $(f_n)$ of continuous functions $X\to\R$ such that $f_n\to f$ in $L^1$.
By taking a subsequence, we may assume $f_n\to f$ pointwise.

Since $\mu$ is finite, by the Egorov theorem, there is a measurable $A\subset X$ such that $f_n\to f$ uniformly on $A$ and $\mu(X\setminus A)<\e/2$.
Since $\mu$ is inner regular, we have closed $F\subset A$ such that $\mu(A\setminus F)<\e/2$, so that we have $\mu(X\setminus F)<\e$.
Then, $f$ is continuous on $A$, and of course on $F$.
Since $X$ is normal, by the Tietz extension theorem, there is a continuous $g:X\to\R$ such that $f|_F=g|_F$.

(b)

\end{pf}






\section{Measures on locally compact Hausdorff spaces}


compact  closed set not containing infty
open     open not containing infty
closed   closed set containing infty

for a measure that ``vanishes at infty'' = tight
two definitions of inner regularity is equivalent.

IRK -> IRF
IRK + sigma finite -> tight

\begin{prb}[Tightness and inner regularity]
We have a similar but confusing concept called tightness; we say a Borel measure $\mu$ on a topological space $X$ is \emph{tight} if for any $\e>0$ there is a compact $K\subset X$ such that $\mu(X\setminus K)<\e$.

History of Bourbaki's text.
\begin{parts}
\item
\end{parts}
\end{prb}

Thm. The measure contructed by RMK is lf and regular(cpt version).
1. open set is approx by cpt sets (by def of rho, if X is LCH)
2. meas set is approx by opn sets (by def of outer meas)
3. sigma finite set is approx by cpt sets (by thm)

\section{Riesz-Markov-Kakutani representation theorem}

Consider
\begin{cd}
\text{regBorel}_{fin} \rar[hook]\dar[hook]& \text{Borel}_{fin} \rar\dar[hook]& \text{Baire}_{fin} \rar\dar[hook]& C_b^{*+} \dar[->>]&\\
\text{regBorel}_{locfin} \rar[hook]& \text{Borel}_{locfin} \rar& \text{Baire}_{locfin} \rar& C_c^{*+}\rar[hook]& \text{pos lin on }C_c.
\end{cd}
for locally compact Hausdorff $X$.


$\text{Borel}_{locfin}\to \text{pos lin on }C_c$ is surjective for all topological spaces.

$\text{regBorel}_{fin}\to C_b^{*+}$ is injective for normal spaces.

$\text{regBorel}_{locfin}\to C_c^{*+}$ is injective for locally compact Hausdorff spaces.(maybe)



\begin{prb}[Existence of Borel measures]
Let $(X,\cT)$ be a topological space.
For a positive linear functional $I$ on $C_c(X)$, define a set function $\rho:\cT\to[0,\infty]$ such that
\[\rho(U):=\sup\,\{\,I(f):f\in C_c(U,[0,1])\,\}.\]
Define the associated outer measure $\mu^*:\cP(X)\to[0,\infty]$ such that
\[\mu^*(E):=\inf\left\{\,\sum_{i=1}^\infty\rho(U_i):E\subset\bigcup_{i=1}^\infty U_i,\ U_i\text{ open}\,\right\}.\]
Let $\cM$ be the $\sigma$-algebra of Carath\'eodory measurable subsets relative to $\mu^*$, $\cB$ the Borel $\sigma$-algebra, and $\mu:=\mu^*|_\cB$.
\begin{parts}
\item $\mu^*|_\cT=\rho$ if $X$ is Hausdorff.
\item $\cB\subset\cM$.
\item $I(f)=\int f\,d\mu$ for $f\in C_c(X)$ if $X$ is Hausdorff.
\end{parts}
\end{prb}
\begin{pf}
(a)
Since $\rho$ is already monotone, if we only show $\rho$ is countably subadditive, then we can rewrite the outer measure as
\[\mu^*(E)=\inf\,\{\,\rho(U):E\subset U,\ U\text{ open}\,\},\]
so the identity $\mu^*(U)=\rho(U)$ for open $U$ clearly follows.

Take any open $U$ and countable family $\{U_i\}_{i=1}^\infty$ of open sets whose union is $U$.
Take any $f\in C_c(U,[0,1])$ and let $K:=\supp f$.
Since $K$ is compact, there is a finite subcover $\{U_j\}_{j=1}^n$ of $K$, and since $K$ is paracompact Hausdorff, there is a partition of unitiy $\{\chi_j\}_j$ on $K$ subordinate to the open cover $\{U_j\cap K\}_j$.
Note that $\supp\chi_j\subset U_j\cap K$ for each $j$.

The set $\supp(f\chi_j)$ is closed in $K$ so the compactness, and we also have the inclusion $\supp(f\chi_j)\subset\supp\chi_j\subset U_j$.
For every $0<a\le 1$, since $(f\chi_j)^{-1}((a,1])$ is open in the interior of $K$ and $(f\chi_j)^{-1}([a,1])$ is closed in $K$, $f\chi_j$ is continuous on $U_j$.
Now we have checked $f\chi_j\in C_c(U_j,[0,1])$.

Then, because $I$ is linear so that it preserves finite sum, we have
\[I(f)=I\left(\sum_{j=1}^n f\chi_j\right)=\sum_{j=1}^n I(f\chi_j)\le\sum_{j=1}^n\rho(U_j)\le\sum_{i=1}^\infty\rho(U_i).\]
Since $f$ is arbitrary, we get $\rho(U)\le\sum_{i=1}^\infty\rho(U_i)$.

(b)
It suffices to show $\cT\subset\cM$.
Clearly $\mu^*(E)\le\mu^*(E\cap U)+\mu^*(E\setminus U)$ for any measurable $E$ and open $U$.
For the opposite direction, take $\e>0$.
Note that we may assume $\mu^*(E)<\infty$.
There are open $U_i$ such that $E\subset\bigcup_{i=1}^\infty U_i$ and
\[\mu^*(E)+\frac\e3>\sum_{i=1}^\infty\rho(U_i).\]
Take $f_i\in C_c(U_i\cap U,[0,1])$ such that
\[\rho(U_i\cap U)-\frac13\cdot\frac\e{2^i}<I(f_i),\]
and take $g_i\in C_c(U_i\setminus\supp f_i,[0,1])$ such that
\[\rho(U_i\setminus\supp f_i)-\frac13\cdot\frac\e{2^i}<I(g_i).\]
Then, since $f_i+g_i\in C_c(U_i,[0,1])$, we have
\begin{align*}
\rho(U_i)\ge I(f_i+g_i)
&>\rho(U_i\cap U)+\rho(U_i\setminus\supp f_i)-\frac23\cdot\frac\e{2^i}\\
&\ge\rho(U_i\cap U)+\rho(U_i\setminus U)-\frac23\cdot\frac\e{2^i}.
\end{align*}
It implies
\[\mu^*(E)+\e>\sum_{i=1}^\infty\rho(U_i\cap U)+\sum_{i=1}^\infty\rho(U_i\setminus U))\ge\mu^*(E\cap U)+\mu^*(E\setminus U)\]
because $E\cap U\subset\bigcup_{i=1}^\infty U_i\cap U$ and $E\setminus U\subset\bigcup_{i=1}^\infty U_i\setminus U$.


(c)
We first claim that for $g\in C_c(X,[0,1])$ we have
\[\mu(K)\le I(g)\le\mu(K'),\]
where $K$ and $K'$ are compact and open sets such that $g|_K=1$ and $g|_{K'^c}=0$ respectively.
The one inequality $I(g)\le\mu(K')$ directly follows from $I(g)\le\rho(U)$ for every open $U$ containing $K'$.
For the other, let $0<\e<1$ and $U:=g^{-1}((1-\e,1])$, so that $K\subset U\subset\supp g$.
For any $h\in C_c(U,[0,1])$, the inequality $(1-\e)h\le g$ implies $I(h)\le(1-\e)^{-1}I(g)$, so
\[\mu(K)\le\mu(U)=\rho(U)\le(1-\e)^{-1}I(g).\]
By limiting $\e\to0$, we get $\mu(K)\le I(g)$, the claim proved.

Since $C_c(X)$ is the linear span of $C_c(X,[0,1])$, it is enough to show $I(f)=\int f\,d\mu$ for $f\in C_c(X,[0,1])$.
For a fixed positive integer $n$ and for each index $1\le i\le n$, let $K_i:=f^{-1}([i/n,1])$ and define
\[f_i(x):=\begin{cases}0&\text{ if }x\in K_{i-1}^c,\\1/n&\text{ if }x\in K_i,\\f(x)-(i-1)/n&\text{ if }x\in K_{i-1}\setminus K_i,\end{cases}\]
where $K_0:=\supp f$.
Note that $nf_i\in C_c(X,[0,1])$ and $f=\sum_{i=1}^nf_i$.
For $1\le i\le n$ we have $\mu(K_i)<\infty$ because $K_i$ is compact subsets contained in a locally compact Hausdorff space $U:=f^{-1}((0,1])$, but $\mu(K_0)$ is possibly infinite.
By the previous claim and the property of integral, we have
\[\frac{\mu(K_i)}n\le I(f_i),\qquad\frac{\mu(K_i)}n\le\int f_i\,d\mu\]
for $1\le i\le n$ and
\[I(f_i)\le\frac{\mu(K_{i-1})}n,\qquad\int f_i\,d\mu\le\frac{\mu(K_{i-1})}n\]
for $2\le i\le n$.
Then, using the above inequalities and $\mu(K_n)\ge0$, we have
\[I(f)\le I(f_1)+\int f\,d\mu\quad\text{and}\quad\int f\,d\mu\le\int f_1\,d\mu+I(f).\]
Note that $f_1=\min\{f,1/n\}$ is a sequence of functions indexed by $n$.
By the monotone convergence theorem, $\int f_1\,d\mu\to0$ as $n\to\infty$.
We now show $I(f_1)$ converges to zero.

For $g\in C_0(U,[0,1])$, it clearly has compact support, and and it is also continuous because $g^{-1}((a,1])$ is open in $U$ and $g^{-1}([a,1])$ is closed in $K$ for any $0<a\le1$, so that we have $C_0(U)\subset C_c(X)$.
We also have $f_1\in C_0(U)$ since $f_1^{-1}([\e,1])$ is a compact set in $U$ for every $\e>0$.
Therefore, $I$ is a positive linear functional on $C_0(U)$.
Assume that $I$ is not bounded; there is no constant $C$ such that $g\in C_0(U,[0,1])$ implies $I(g)\le C$.
Construct a sequence $(h_k)_{k=1}^\infty$ in $C_0(U,[0,1])$ such that $I(h_k)\ge2^k$, and define $h:=\sum_{k=1}^\infty h_k/2^k$ so that $h\in C_0(U,[0,1])$.
Then, $I(h)\ge \sum_{k=1}^mI(h_k)/2^k\ge m$ for every $m>0$, it contradicts to the assumption, which means that there is a constant $C$ such that $I(g)\le C$ for all $g\in C_0(U,[0,1])$, and it proves $I(f_1)\le C/n\to0$ as $n\to\infty$.
Therefore, $I(f)=\int f\,d\mu$.
\end{pf}

\begin{prb}[Uniqueness of regular Borel measures]
a
\begin{parts}
\item If $X$ is normal, then $\text{regBorel}_{fin}\to C_b^{*+}$ is injective.
\item If $X$ locally compact Hausdorff, then $\text{regBorel}_{locfin}\to C_c^{*+}$ is injective.
\end{parts}
\end{prb}
\begin{pf}
(a)
Let $E$ be any measurable set.
Since $X$ is normal and $\mu+\nu$ is a finite regular Borel measure, by the Luzin theorem, there are closed $F\subset X$ and $g\in C_b(X)$ with $0\le g\le1$ such that $\1_E|_F=g|_F$ and $(\mu+\nu)(X\setminus F)<\e/2$.
Then,
\begin{align*}
|\mu(E)-\nu(E)|&=|\int\1_E\,d\mu-\int\1_E\,d\nu|\\
&\le\int_{X\setminus F}|\1_E-g|\,d\mu+\int_{X\setminus F}|\1_E-g|\,d\nu\\
&<2(\mu+\nu)(X\setminus F)=\e.
\end{align*}

\end{pf}


\begin{prb}[Dual of continuous function spaces]
\end{prb}



\section*{Exercises}

\begin{prb}
Let $X$ be a topological space, and $Y$ a second countable space.
Let $\mu$ be a regular Borel measure on $X$.
Let $f:X\to Y$ be a Borel measurable function.
For any $\e>0$, there is a closed $F\subset X$ such that $\mu(X\setminus F)<\e$ and $f|_F:F\to Y$ is continuous.
\end{prb}
\begin{pf}
Let $(V_n)_{n=1}^\infty$ be a base of $Y$.
For each $n$, take open $U_n$ and closed $F_n$ such that $F_n\subset f^{-1}(V_n)\subset U_n$ and $\mu(U_n\setminus F_n)<\e/2^n$.
Define $F:=\left(\bigcup_{n=1}^\infty U_n\setminus F_n\right)^c$ so that $\mu(X\setminus F)<\e$ and $F$ is closed.
Then, the identity $f^{-1}(V_n)\cap F=U_n\cap F$ proves the continuity of $f|_F$.
\end{pf}








%%%%%%%%%%%

\section*{asdf}


\begin{lem}
Let $\mu$ be a Borel measure on a LCH $X$.
Then, $\mu$ is inner regular on open sets iff
\[\mu(U)=\|\mu\|_{C_c(U)^*}\]
for every open $U$ in $X$.
\end{lem}
\begin{pf}
($\Leftarrow$)
($\ge$)
For $f\in C_c(U)$, we have
\[|\int f\,d\mu|=|\int_Uf\,d\mu|\le\mu(U)\,\|f\|.\]

($\le$)
Since $\mu$ is inner regular on $U$, there is a compact set $K\subset U$ such that $\mu(U)-\mu(K)<\e$ (for the case $\mu(U)=\infty$, we can deal with separately).
We can find a nonnegative function $f\in C_c(U)$ with $f|_K \equiv 1$ and $f\le1$ by the construction of Urysohn.
Then, for all $\e>0$ we have
\[\mu(U)<\mu(K)+\e\le\int f\,d\mu+\e\le\|\mu\|_{C_c^*(U)}+\e.\]

($\Rightarrow$)
Let $f\in C_c(U)$ be a function such that $\|f\|=1$ and
\[\mu(U)-\e<\int f\,d\mu.\]
Let $K=\supp(f)$.
Then
\[\mu(K)\ge\int f>\mu(U)-\e.\]
\end{pf}
% 이런 거 쓸 때 메져가 유한인지 무한인지 케이스 나누고 증명쓰는 게 좋겠다

\begin{prop}
A Radon measure is inner regular on all $\sigma$-finite Borel sets.(Folland's)
\end{prop}
\begin{pf}
First we approximate Borel sets of finite measure, with compact sets.
Let $E$ be a Borel set with $\mu(E)<\infty$ and $U$ be an open set containing $E$.
By outer regularity, there is an open set $V\supset U-E$ such that
\[\mu(V)<\mu(U-E)+\frac\e2.\]
By inner regularity, there is a compact set $K\subset U$ such that
\[\mu(K)>\mu(U)-\frac\e2.\]
Then, we have a compact set $K-V\subset K-(U-E)\subset E$ such that
\begin{align*}
\mu(K-V)&\ge\mu(K)-\mu(V)\\
&>\left(\mu(U)-\frac\e2\right)-\left(\mu(U-E)+\frac\e2\right)\\
&\ge\mu(E)-\e.
\end{align*}
It implies that a Radon measure is inner regular on Borel sets of finite measures.

Suppose $E$ is a $\sigma$-finite Borel set so that $E=\bigcup_{n=1}^\infty E_n$ with $\mu(E_n)<\infty$.
We may assume $E_n$ are pairwise disjoint.
Let $K_n$ be a compact subset of $E_n$ such that
\[\mu(K_n)>\mu(E_n)-\frac\e{2^n},\]
and define $K=\bigcup_{n=1}^\infty K_n\subset E$.
Then,
\[\mu(K)=\sum_{n=1}^\infty\mu(K_n)>\sum_{n=1}^\infty\left(\mu(E_n)-\frac\e{2^n}\right)=\mu(E)-\e.\]
Therefore, a Radon measure is inner regular on all $\sigma$-finite Borel sets.
\end{pf}

\begin{thm}
If every open set in $X$ is $\sigma$-compact(i.e. Borel sets and Baire sets coincide), then every locally finite Borel measure is regular.
\end{thm}
\begin{prop}
In a second countable space, every open set is $\sigma$-compact(i.e. Borel sets and Baire sets coincide).
\end{prop}


Two corollaries are presented as follows:
\begin{rd}[column sep={120pt,between origins}]
\parbox{7em}{\centering locally finite \\ Borel regular} \rar &
\parbox{5em}{\centering Radon} \rar \lar[dashed, bend right, swap]{$X$ is $\sigma$-compact} &
\parbox{7em}{\centering locally finite \\ Borel} \ar[dashed, bend left]{ll}{$X$ is second countable}
\end{rd}




\begin{prb}
Let $X$ be compact.
A positive linear functional $\rho$ on $C(X)$ is bounded with norm $\rho(1)$.
\end{prb}
\begin{pf}
Since $0\le\rho(\|f\|\pm f)=\|f\|\rho(1)\pm\rho(f)$, we have $|\rho(f)|\le\rho(1)\|f\|$.
\end{pf}

\begin{prb}
Let $X$ be a locally compact Hausdorff space.
\begin{parts}
\item The Baire $\sigma$-algebra is generated by compact $G_\delta$ sets.
\item If $X$ is second countable, then every Baire set is Borel.
\end{parts}
\end{prb}
\begin{sol}
(b)
(A second countable locally compact space is $\sigma$-compact.

Since $X$ is $\sigma$-compact and Hausdorff, every closed set is a countable union of compact sets, so the Borel $\sigma$-algebra on $X$ is generated by compact sets.)

Since locally compact Hausdorff space is regular, the Urysohn metrization implies $X$ is metrizable, and every closed sets in metrizable space is $G_\delta$ set.
\end{sol}



%%%%%%%%%%%%%%%
\begin{prb}
Let $X$ be compact.
There is a map from the set of finite Baire measures to the set of positive linear functionals on $C(X)$.
\end{prb}
\begin{sol}
A function in $C(X)$ is Baire measurable and bounded.
Thus the integration is well-defined.
\end{sol}

\begin{prb}
Let $X$ be compact.
There is a map from the set of positive linear functionals on $C(X)$ to the set of finite regular Borel measures.
\end{prb}
\begin{sol}
i. and ii. and iii. of Theorem 7.2.
\end{sol}

\begin{prb}
Let $X$ be compact.
Let $\rho$ be a positive linear functional on $C(X)$.
Let $\nu$ be the regular Borel measure associated to $\rho$.
Then, $\rho(f)=\int f\,d\nu$.
\end{prb}
\begin{sol}
iv. of Theorem 7.2.
\end{sol}

\begin{prb}
Let $X$ be compact.
Let $\nu$ be a finite regular Borel measure.
Let $\nu'$ be the regular Borel measure associated to the positive linear functional $f\mapsto\int f\,d\nu$.
Then, $\nu=\nu'$ on Borel sets.
\end{prb}
\begin{sol}
Theorem 7.8.
\end{sol}

The two results above establish the correspondence between positive linear functionals and regular Borel measures.
The following is an additional topic: Borel extension of Baire measures.
\begin{prb}
Let $X$ be compact.
Let $\mu$ be a finite Baire measure.
Let $\nu$ be the regular Borel measure associated to the positive linear functional $f\mapsto\int f\,d\mu$.
Then, $\mu=\nu$ on Baire sets.
\end{prb}
\begin{sol}
Let $\mu,\nu$ be finite Baire measures.
Enough to show if $\int f\,d\mu=\int f\,d\nu$ then $\mu=\nu$ according to the preceding two results.

Enough to show the regularity of Baire measures.
\end{sol}
























\end{document}
