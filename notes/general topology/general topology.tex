\documentclass{../note}
\usepackage{../../ikany}


\begin{document}
\title{General Topology}
\author{Ikhan Choi}
\maketitle
\tableofcontents

\part{Topological spaces}

\chapter{Metric spaces}
\section{Metrics}

\begin{prb}[Definition of metric spaces]
Let $X$ be a set.
A \emph{metric} is a function $d:X\times X\to\R_{\ge0}$ such that
\begin{parts}[(i)]
\item $d(x,y)=0$ if and only if $x=y$, \hfill(nondegeneracy)
\item $d(x,y)=d(y,x)$ for all $x,y\in X$, \hfill(symmetry)
\item $d(x,z)\le d(x,y)+d(y,z)$ for all $x,y,z\in X$. \hfill(triangle inequality)
\end{parts}
A pair $(X,d)$ of a set $X$ and a metric on $X$ is called a \emph{metric space}.
We often write it simply $X$.
\begin{parts}
\item
Show that a normed space $X$ is a metric space with $d(x,y)=\|x-y\|$.
\item
Show that a subset of a metric space is a metric space with a metric given by restriction.
\end{parts}
\end{prb}

\begin{prb}[Discrete metrics]
Let $X$ be a set, and define a metric as
\[d(x,y):=\begin{cases}0&,x=y\\1&,x\ne y\end{cases}.\]
This metric is called \emph{discrete}.
\begin{parts}
\item
Show that the discrete metric is a strongest metric on $X$.
\item
Investiagte all convergent sequences in a discrete metric space.
\end{parts}
\end{prb}

\begin{prb}[Comparison of metrics]
For metrics $d_1$ and $d_2$ on a set $X$, it is said that $d_1$ is \emph{stronger than} $d_2$ (equivalently, $d_2$ is \emph{weaker than} $d_1$) or $d_1$ \emph{refines} $d_2$ if for any $x\in X$ arbitrary $\e>0$ has $\delta>0$ such that
\[B_1(x,\delta)\subset B_2(x,\e),\]
where $B_1$ and $B_2$ refer to balls within the metrics $d_1$ and $d_2$ respectively.
\begin{parts}
\item
Show that this refinement relation is a preorder.
\item
Show that $d_1$ is stronger than $d_2$ if and only if every sequence that converges to $x\in X$ in $d_1$ converges to $x$ in $d_2$.
\item
Show that $d_1$ is stronger than $d_2$ if and only if the identity map $\id:(X,d_1)\to(X,d_2)$ is continuous.
\end{parts}
\end{prb}

\begin{prb}[Equivalence of metrics]
Two metrics $d_1$ and $d_2$ on a set $X$ are equivalent if one of the followings are satisfied:
\begin{parts}
\item
Show that $d_1$ and $d_2$ are equivalent if for each point $x$ in $X$ there exist two constants $C_1$ and $C_2$ such that
\[d_2(x,y)\le C_1d_1(x,y)\quad\text{and}\quad d_1(x,y)\le C_2d_2(x,y)\]
for all $y$ in $X$.
\item
Show that $d_1$ and $d_2$ are equivlane tif $d_2=f\circ d_1$ for a monotonically increasing function $f:\R_{\ge0}\to\R_{\ge0}$ that is continuous at $0$.
\end{parts}
\end{prb}


\begin{defn}
Let $X$ be a set.
A \emph{metric} is a function $d:X\times X\to\R_{\ge0}$ such that
\begin{parts}
\item $d(x,y)=0$ if and only if $x=y$, \hfill(nondegeneracy)
\item $d(x,y)=d(y,x)$ for all $x,y\in X$, \hfill(symmetry)
\item $d(x,z)\le d(x,y)+d(y,z)$ for all $x,y,z\in X$. \hfill(triangle inequality)
\end{parts}
A pair $(X,d)$ of a set $X$ and a metric on $X$ is called a \emph{metric space}.
We often write it simply $X$.
\end{defn}

\begin{ex}
\leavevmode
\begin{parts}
\item
A normed space $X$ is a metric space.
Precisely, the norm structure naturally defines a real-valued function $d$ on $X\times X$ defined by $d(x,y):=\|x-y\|$ and it satisfies the axioms of metric.
\item
Let $(X,d)$ be a metric space.
Every subset of $X$ has a natural induced metric, just the restriction of original metric $d$.
\item
Let $X$ be a set.
Then, a function $d:X\times X\to\R_{\ge0}$ defined by
\[d(x,y):=\begin{cases}0&,x=y\\1&,x\ne y\end{cases}\]
is a metric on $X$.
This metric is sometimes called \emph{discrete metric} because balls can separate all single points out.
\item
Let $d$ be a metric on a set $X$.
Let $f:\R_{\ge0}\to\R_{\ge0}$ be a function such that $f^{-1}(0)=\{0\}$.
If $f$ is monotonically increasing and subadditive, then $f\circ d$ satisfies the triangle inequality, hence is another metric on $X$.
Note that a function $f$ is called \emph{subadditive} if $f(x+y)\le f(x)+f(y)$ for all $x,y$ in the domain.
\item
Let $G=(V,E)$ be a connected graph.
Define $d:V\times V\to\Z_{\ge0}\subset\R_{\ge0}$ as the distance of two vertices; the length of shortest path connecting two vertices.
Then, $(V,d)$ is a metric space.
\item
Let $\cP(X)$ be the power set of a finite set $X$.
Define $d:\cP(X)\times\cP(X)\to\Z_{\ge0}\subset\R_{\ge0}$ as the cardinality of the symmetric difference; $d(A,B):=|(A-B)\cup(B-A)|$.
Then $(\cP(X),d)$ is a metric space.
\item
Let $C$ be the set of all compact subsets of $\R^d$.
Recall that a subset of $\R^d$ is compact if and only if it is closed and bounded.
Then, $d:C\times C\to\R_{\ge0}$ defined by
\[d(A,B):=\max\{\,\sup_{a\in A}\inf_{b\in B}\|a-b\|,\ \sup_{b\in B}\inf_{a\in A}\|a-b\|\,\}\]
is a metric on $C$.
It is a little special case of \emph{Hausdorff metric}.
\end{parts}
\end{ex}

A metric is often misunderstood as something that measures a distance between two points and belongs to the study of geoemtry.
The main function of a metric is to make a system of small balls, sets of points whose distance from specified center points is less than fixed numbers.
The balls centered at each point provide a concrete images of ``system of neighborhoods at a point'' in a more intuitive sense.
In this viewpoint, a metric can be considered as a structure that lets someone accept the notion of neighborhoods more friendly.

\begin{defn}
Let $X$ be a metric space.
A set of the form 
\[\{y\in X:d(x,y)<\e\}\]
for $x\in X$ and $\e>0$ is called an \emph{open ball centered at $x$ with radius $\e$} and denoted by $B(x,\e)$ or $B_\e(x)$.
\end{defn}

The adjective ``open'' is in order to distinguish from closed balls $\cl{B(x,\e)}=\{y\in X:d(x,y)\le\e\}$.
The terms ``open'' and ``closed'' will be discussed later.
Now let us reformulate the definitions of limits and continuity with open balls.

\begin{defn}
Let $\{x_n\}_n$ be a sequence of points on a metric space $(X,d)$.
We say that a point $x$ is a \emph{limit} of the sequence or the sequence \emph{converges to $x$} if for arbitrarily small ball $B(x,\e)$, we can find $n_0$ such that $x_n\in B(x,\e)$ for all $n>n_0$.
If it is satisfied, then we write
\[\lim_{n\to\infty}x_n=x,\]
or simply $x_n\to x$ as $n\to\infty$.
We say a sequence is \emph{convergent} if it converges to a point.
If it does not converge to any points, then we say the sequence \emph{diverges}.

A function $f:X\to Y$ between metric spaces is called \emph{continuous at $x\in X$} if for any ball $B(f(x),\e)\subset Y$, there is a ball $B(x,\delta)\subset X$ such that $f(B(x,\delta))\subset B(f(x),\e)$.
The function $f$ is called \emph{continuous} if it is continuous at every point on $X$.
\end{defn}

The convergence can be also characterized by limits in $\R$: a sequence $\{x_n\}$ in a metric space $X$ converges to $x\in X$ if and only if
\[\lim_{n\to\infty}d(x_n,x)=0.\]
Note that taking either $\e$ or $\delta$ in analysis really means taking a ball of the very radius.
For continuity of a function, we can describe it intuitively that no matter how small ball is taken in the codomain, we can take much smaller ball in the domain.
Investigation of the distribution of open balls centered at a point is now an important problem.

\begin{ex}
\leavevmode
\begin{parts}
\item
Let $X$ be the discrete metric space.
Every ball centered at a point $x$ with respect to the discrete metric is either a singleton $B(x,\e)=\{x\}$ when $\e\le1$, or the entire space $B(x,\e)=X$ when $\e>1$.
In particular, a sequence $\{x_n\}_n$ converges to $x$ if and only if it is eventually $x$; there is a positive integer $n_0$ such that $x_n=x$ for all $n>n_0$.
\item
Let $X$ and $Y$ be metric spaces.
If $X$ is equipped with the discrete metric, then every function $f:X\to Y$ is continuous on the discrete metric.
\item
Let $f:X\to Y$ be a function between two metric spaces.
If there is a constant $C$ such that $d(x,y)\le Cd(f(x),f(y))$ for all $x$ and $y$ in $X$, then $f$ is continuous.
In this case, $f$ is particularly called \emph{Lipschitz continuous} with the \emph{Lipschitz constant} $C$.
\end{parts}
\end{ex}


\section{Topological equivalence}
A metric can be viewed as a function that takes a sequence as input and returns whether the sequence converges or diverges.
That is, a metric acts like a criterion which decides convergence of sequences.
Take note on the fact that the sequence of real numbers defined by $x_n=\frac1n$ converges in standard metric but diverges in discrete metric.
Like this example, even for the same sequence on a same set, the convergence depends on the attached metrics.
What we are interested in is comparison of metrics and to find a proper relation structure.
If a sequence converges in a metric $d_2$ but diverges in another metric $d_1$, we would say $d_1$ has stronger rules to decide the convergence.
Refinement relation formalizes the idea.

\begin{defn}
Let $d_1$ and $d_2$ are metrics on a set $X$.
We say $d_1$ is \emph{stronger than} $d_2$ (equivalently, $d_2$ is \emph{weaker than} $d_1$) or $d_1$ \emph{refines} $d_2$, if for any $x\in X$ and for arbitrary $\e>0$ we can find $\delta>0$ such that
\[B_1(x,\delta)\subset B_2(x,\e).\]
The notations $B_1$ and $B_2$ refer to balls defined with the metrics $d_1$ and $d_2$ respectively.
\end{defn}

\begin{prop}
The refinement relation is a preorder.
\end{prop}
\begin{pf}
It is enough to show the transitivity.
Suppose there are three metric $d_1$, $d_2$, and $d_3$ on a set $X$ such that $d_1$ is stronger than $d_2$ and $d_2$ is stronger than $d_3$.
For $i=1,2,3$, let $B_i$ be a notation for the balls defined with the metric $d_i$.

Take $x\in X$ and $\e>0$ arbitrarily.
Then, we can find $\e'>0$ such that
\[B_2(x,\e')\subset B_3(x,\e).\]
Also, we can find $\delta>0$ such that
\[B_1(x,\delta)\subset B_2(x,\e').\]
Therefore, we have $B_1(x,\delta)\subset B_3(x,\e)$ which implies that $d_1$ refines $d_3$.
\end{pf}

\begin{prop}
Let $d_1$ and $d_2$ be metrics on a set $X$.
Then, $d_1$ is stronger than $d_2$ if and only if every sequence that converges to $x\in X$ in $d_1$ converges to $x$ in $d_2$.
\end{prop}
\begin{pf}
($\Rightarrow$)
Let $\{x_n\}_n$ be a sequence in $X$ that converges to $x$ in $d_1$.
By the assumption, for an arbitrary ball $B_2(x,\e)=\{y:d_2(x,y)<\e\}$, there is $\delta>0$ such that
\[B_1(x,\delta)\subset B_2(x,\e),\]
where $B_1(x,\delta)=\{y:d_1(x,y)<\delta\}$.
Since $\{x_n\}_n$ converges to $x$ in $d_1$, there is an integer $n_0$ such that
\[n>n_0\impl x_n\in B_1(x,\delta).\]
Combining them, we obtain an integer $n_0$ such that
\[n>n_0\impl x_n\in B_2(x,\e).\]
It means $\{x_n\}$ converges to $x$ in the metric $d_2$.

($\Leftarrow$)
We prove it by contradiction.
Assume that for some point $x\in X$ we can find $\e_0>0$ such that there is no $\delta>0$ satisfying $B_1(x,\delta)\subset B_2(x,\e_0)$.
In other words, at the point $x$, the difference set $B_1(x,\delta)\setminus B_2(x,\e_0)$ is not empty for every $\delta>0$.
Thus, we can choose $x_n$ to be a point such that
\[x_n\in B_1\left(x,\tfrac1n\right)\setminus B_2(x,\e_0)\]
for each positive integer $n$ by putting $\delta=\frac1n$.

We claim $\{x_n\}_n$ converges to $x$ in $d_1$ but not in $d_2$.
For $\e>0$, if we let $n_0=\ceil{\frac1\e}$ so that we have $\frac1{n_0}\le\e$, then
\[n>n_0\impl x_n\in B_1\left(x,\tfrac1n\right)\subset B_1(x,\e).\]
So $\{x_n\}_n$ converges to $x$ in $d_1$.
However in $d_2$, for $\e=\e_0$, we can find such $n_0$ like $d_1$ since
\[x_n\notin B_2(x,\e_0)\]
for every $n$.
Therefore, $\{x_n\}$ does not converges to $x$ in $d_2$.
\end{pf}

\begin{ex}
Let $d_1$ and $d_2$ be metrics on a set $X$.
Suppose for each point $x$ there exists a constant $C$ which may depend on $x$ such that
\[d_2(x,y)\le Cd_1(x,y)\]
for all $Y$.
Then, $d_1$ is stronger than $d_2$.
\end{ex}
\begin{ex}
There is always no stronger metric than the discrete metric.
In other words, discrete metric is the strongest metric.
\end{ex}

Now we define the equivalence relation.

\begin{defn}
Two metrics on a set are called \emph{topologically equivalent} if the sets of open balls centered at each point are mutually nested; in other words, they refines each other.
\end{defn}

When two metrics are topologically equivalent, they are also said to induce exactly the same topology.
The word ``topologically'' is frequently omitted.
We can easily check that two metrics are equivalent if and only if they share the same sequential convergence data: a sequence converges to $x\in X$ in $d_1$ if and only if it converges to $x$ in $d_2$.
The following theorem give sufficient conditions for equivalence.

\begin{thm}
Two metrics $d_1$ and $d_2$ on a set $X$ are equivalent if one of the followings are satisfied:
\begin{parts}
\item for each point $x$ there exist two constants $C_1$ and $C_2$ which may depend on $x$ such that
\[d_2(x,y)\le C_1d_1(x,y)\quad\text{and}\quad d_1(x,y)\le C_2d_2(x,y)\]
for all $y$ in $X$,
\item $d_2=f\circ d_1$ for a monotonically increasing function $f:\R_{\ge0}\to\R_{\ge0}$ that is continuous at $0$.
\end{parts}
\end{thm}
\begin{pf}
(a)
It is a corollary of Example 1.10.

(b)
For any ball $B_1(x,\e)$, we have a smaller ball
\[B_2(x,f(\e))\subset B_1(x,\e)\]
since $f(d(x,y))<f(\e)$ implies $d(x,y)<\e$.
Conversely, take an arbitrary ball $B_2(x,\e)$.
Since $f$ is continuous at 0, we can find $\delta>0$ such that
\[d(x,y)<\delta\impl f(d(x,y))<\e,\]
which implies $B_1(x,\delta)\subset B_2(x,\e)$.
\end{pf}

\iffalse
\begin{rmk}
Unlike metrics, there exist two different topologies that have same sequential convergence data.
For example, a sequence in an uncountable set with cocountable topology converges to a point if and only if it is eventually at the point, which is same with discrete topology.
This means the informations of sequence convergence are not sufficient to uniquely characterize a topology.
Instead, convergence data of generalized sequences also called nets, recover the whole topology.
For topologies having a property called the first countability, it is enough to consider only usual sequences in spite of nets.
What we did in this subsection is not useless because topology induced from metric is a typical example of first countable topologies.
These kinds of problems will be profoundly treated in Chapter 3.
\end{rmk}
\begin{rmk}
One can ask some results for the equivalence of metrics characterized by a same set of continuous functions.
However, they are generally difficult problems: is it possible to recover the base space from a continuous function space or a path space?
\end{rmk}
\fi

\section{Sum of metrics}
Topologies are occasionally described by not a single but several metrics.
It provides a useful method to construct a metric or topology, which can be applied to a quite wide range of applications.
Specifically, in a conventional way, metrics can be summed or taken maximum to make another metric out of olds.
The following proposition can be easily generalized to an arbitrary finite number of metrics by mathematical induction.

\begin{prop}
Let $d_1$ and $d_2$ be metrics on a set $X$.
For a sequecne $\{x_n\}_n$ in $X$, the following statements are all equivalent:
\begin{parts}
\item it converges to $x$ in both $d_1$ and $d_2$,
\item it converges to $x$ in $d_1+d_2$,
\item it converges to $x$ in $\max\{d_1,d_2\}$.
\end{parts}
In particular, the metrics $d_1+d_2$ and $\max\{d_1,d_2\}$ are equivalent.
\end{prop}
\begin{pf}
We skip to prove $d_1+d_2$ and $\max\{d_1,d_2\}$ are metrics.

(b) or (c)$\Rightarrow$(a)
The inequalities $d_i\le d_1+d_2$ and $d_i\le\max\{d_1,d_2\}$ imply the desired results.

(a)$\Rightarrow$(b)
For $\e>0$, we may find positive integers $n_1$ and $n_2$ such that $n>n_1$ and $n>n_2$ imply $d_1(x_n,x)<\frac\e2$ and $d_2(x_n,x)<\frac\e2$ respectively.
If we define $n_0:=\max\{n_1,n_2\}$, then
\[n>n_0\impl d_1(x_n,x)+d_2(x_n,x)<\e.\]

(a)$\Rightarrow$(c)
Take $n_0$ as we did previously.
Then,
\[n>n_0\impl\max\{d_1(x_n,x),d_2(x_n,x)\}<\frac\e2<\e.\qedhere\]
\end{pf}

\begin{rmk}
In general, for any norm $\|\cdot\|$ on $\R^2$, the function $\|(d_1,d_2)\|$ defines another equivalent metric.
\end{rmk}

There is also a method for combining not only finite family of metrics, but also infinite family of metrics.
Since the sum of infinitely many positive numbers may diverges to infinity, we cannot sum the metrics directly.
The strategy is to ``bound'' the metrics.
We call a metric bounded when the range of metric function is bounded.

\begin{prop}
Every metric possesses an equivalent bounded metric.
\end{prop}
\begin{pf}
Let $d$ be a metric on a set.
Let $f$ be a bounded, monotonically increasing, and subadditive function on $\R_{\ge0}$ that is continuous at 0 and satisfies $f^{-1}(0)=\{0\}$.
The mostly used examples are
\[f(x)=\frac x{1+x}\quad\text{and}\quad f(x)=\min\{x,1\}.\]
Then, $f\circ d$ is a bounded metric equivalent to $d$ by Example 1.4.
\end{pf}

\begin{defn}
Let $d$ be a metric on a set $X$.
A \emph{standard bounded metric} means either metric
\[\min\{d,1\}\quad\text{or}\quad\frac d{d+1},\]
and we will denote it by $\hat d$.
\end{defn}

\begin{prop}
Let $\{d_i\}_{i\in\N}$ be a countable family of metrics on a set $X$.
For a sequence $\{x_n\}_n$ in $X$, the following statements are all equivalent:
\begin{parts}
\item it converges in $d_i$ for every $i$,
\item it converges in a metric
\[d(x,y):=\sum_{i\in\N}\,2^{-i}\hat d_i(x,y),\]
\item it converges in a metric
\[d'(x,y):=\sup_{i\in\N}\,i^{-1}\hat d_i(x,y).\]
\end{parts}
In particular, the metrics $d$ and $d'$ are equivalent.
\end{prop}
\begin{pf}
The functions $d$ and $d'$ in (b) and (c) are well-defined by the monotone convergence theorem and the least upper bound property.
We skip checking for them to satisfy the triangle inequality and be metrics.

(b) or (c)$\Rightarrow$(a)
We have ineuqalities $\hat d_i\le2^id$ and $\hat d_i\le id'$ for each $i$, so convergence in $d$ or $d'$ implies the convergence in each $\hat d_i$.
The equivalence of $\hat d_i$ and $d_i$ implies the desired result.

(a)$\Rightarrow$(b)
Suppose a sequence $\{x_n\}_n$ converges to a point $x$ in $d_i$ for every index $i$.
Take an arbitrary small ball $B(x,\e)=\{y:d(x,y)<\e\}$ with metric $d$.
By the assumption, we can find $n_i$ for each $i$ satisfying
\[n>n_i\impl\hat d_i(x_n,x)<\frac\e2.\]
Define $k:=\lceil1-\log_2\e\rceil$ so that we have $2^{-k}\le\frac\e2$.
With this $k$, define
\[n_0:=\max_{1\le i\le k}n_i.\]
If $n>n_0$, then
\begin{align*}
d(x_n,x)&=\sum_{i=1}^k2^{-i}\hat d_i(x_n,x)+\sum_{i=k+1}^\infty2^{-i}\hat d_i(x_n,x)\\
&<\sum_{i=1}^k2^{-i}\frac\e2+\sum_{i=k+1}^\infty2^{-i}\\
&<\frac\e2+2^{-k}\le\e,
\end{align*}
so $x_n$ converges to $x$ in the metric $d$.

(a)$\Rightarrow$(c)
Suppose a sequence $\{x_n\}_n$ converges to a point $x$ in each $d_i$, and take an arbitrary small ball $B(x,\e)=\{y:d(x,y)<\e\}$ with metric $d$.
By the assumption, we can find $n_i$ for each $i$ satisfying
\[n>n_i\impl\hat d_i(x_n,x)<\e.\]
Define $k:=\lceil\frac1\e\rceil$ so that we have $k^{-1}\le\e$.
With this $k$, define
\[n_0:=\max_{1\le i\le k}n_i.\]
If $n>n_0$, then
\[i^{-1}\hat d_i(x,y)\le\hat d_i(x,y)<\e\quad\text{for}\quad i\le k\]
and
\[i^{-1}\hat d_i(x,y)\le i^{-1}<k^{-1}\le\e\quad\text{for}\quad i>k\]
imply $d(x_n,x)<\e$, which means that $x_n$ converges to $x$ in the metric $d$.
\end{pf}

Combination of uncountably many metrics does not result in a single metric, but a topology which cannot be induced from a metric in general.
It will be discussed in the rest of the note.

\begin{rmk}
A metric
\[d''(x,y)=\sup_{i\in\N}\,d_i(x,y)\]
is not used because the convergence in this metric is a stronger condition than the convergence with respect to each metric $d_i$.
In other words, this metric generates a finer(stronger) topology than the topology generated by subbase of balls.
For example, the topology on $\R^\N$ generated by this metric defined with the projection pseudometrics is exactly what we often call the box topology.
\end{rmk}


We can also form a metric by summation of generalized metrics, called pseudometrics, which is defined by missing the nondegeneracy condition from the original definition of metric.

\begin{defn}
A function $\rho:X\times X\to\R_{\ge0}$ is called a \emph{pseudometric} if
\begin{parts}
\item $\rho(x,x)=0$ for all $x\in X$,
\item $\rho(x,y)=\rho(y,x)$ for all $x,y\in X$, \hfill(symmetry)
\item $\rho(x,z)\le \rho(x,y)+\rho(y,z)$ for all $x,y,z\in X$. \hfill(triangle inequality)
\end{parts}
\end{defn}

For pseudometrics, it is possible to duplicate every definition we studied in metric spaces: convergence of a sequence, continuity between a set endowed with a pseudometric, refinement and equivalence relations, and countable sum of bounded pseudometrics to make a new pseudometric.
Furthermore, every statement for metrics can be generalized to pseudometrics since we have not actually used the condition that $d(x,y)=0$ implies $x=y$.
In fact, we have a flaw that the limit of a convergent sequence may not be unique within a pseudometric.

\begin{ex}
Let $\rho(x,y)=\rho((x_1,x_2),(y_1,y_2))=|x_1-y_1|$ be a pseudometric on $\R^2$.
Consider a sequence $\{(\frac1n,0)\}_n$.
Since $(0,c)$ satisfies 
\[\rho((\tfrac1n,0),(0,c))=\tfrac1n\to0\quad\text{as}\quad n\to\infty\]
for any real number $c$, the sequence converges to $(x_1,x_2)$ if and only if $x_1=0$.
\end{ex}

Although sequences may have several limits in each pseudometric, the sum of a family of pseudometrics can allow the sequences to have at most one limit, only if the sum satisfies the axioms of a metric.

\begin{defn}
A family of pseudometrics $\{\rho_\alpha\}_\alpha$ on a set $X$ is said to \emph{separate points} if the condition 
\[\rho_\alpha(x,y)=0\quad\text{for all $\alpha$}\]
implies $x=y$.
\end{defn}
\begin{prop}
\begin{parts}
\item
A finite family of pseudometrics $\{\rho_i\}_{i=1}^N$ separates points if and only if the pseudometric $\rho:=\sum_{i=1}^N\rho_i$ is a metric.
\item
A countable family of pseudometrics $\{\rho_i\}_{i\in\N}$ separates points if and only if the pseudometric defined by
\[\rho:=\sum_{i\in\N}2^{-i}\tilde\rho_i\quad\text{or}\quad\sup_{i\in\N}i^{-1}\tilde\rho_i,\]
where $\tilde\rho_i$ is either $\min\{\rho_i,1\}$ or $\rho_i/(\rho_i+1)$, is a metric.
\end{parts}
\end{prop}




\chapter{Topological spaces}

\section{Filters and topologies}

\section{Open sets and closed sets}

\section{Interior, closure, and boundary}

\section{Fundamental constructions}


\chapter{Fundamental constructions}






\part{Topological Structures}

\chapter{Nets and sequences}

\chapter{Completeness}

\chapter{Uniform spaces}

\section{Definitions of uniformity}






\chapter*{Problems}

\begin{prb}[Kuratowski embedding]
While every subset of a normed space is a metric space, we have a converse statement that every metric space is in fact realized as a subset of a normed space.
Let $X$ be a metric space, and denote by $C_b(X)$ the space of continuous and bounded real-valued functions on $X$ with uniform norm given by
\[\|f\|=\sup_{x\in X}|f(x)|.\]
Fix a point $p\in X$, which will serve as the origin.
\begin{parts}
\item Show that a map $\phi:X\to C_b(X)$ such that
\[[\phi(x)](t)=d(x,t)-d(p,t)\]
is well-defined.
\item Show that the map $\phi$ is an isometry; $d(x,y)=\|\phi(x)-\phi(y)\|$.
\end{parts}
\end{prb}

\begin{prb}[Equivalence of norms in finite dimension]
Let $V$ be a vector space of dimension $d$ over $\F=\R$ or $\C$.
Fix a basis $\{e_i\}_{i=1}^d$ on $V$ and let $x=\sum_{i=1}^dx_ie_i$ denote an arbitrary element of $V$.
We will prove all norms are equivalent to the standard Euclidean norm defined for this fixed basis:
\[\|x\|_2:=(\sum_{i=1}^d|x_i|^2)^{\frac12}.\]
With this standard norm any theorems studied in elementary analysis including the Bolzano-Weierstrass theorem are allowed to be applied.
Take a norm $\|\cdot\|$ on $V$ which may differ to $\|\cdot\|_2$.
\begin{parts}
\item Find a constant $C_2$ such that $\|x\|\le C_2\|x\|_2$ for all $x\in V$.
\item Show that if no constant $C$ satisfies $\|x\|_2\le C\|x\|$, then there exists a sequence $\{x_n\}_n$ such that $\|x_n\|_2=1$ and $\|x_n\|<\tfrac1n$.
\item Show that there exists a constant $C$ satisfies $\|x\|_2\le C\|x\|$.
\end{parts}
\end{prb}



\part{Topological properties}
\chapter{Compactness}
\chapter{Connectedness}
\chapter{Separability Axioms}
\section{Regular spaces}
\section{Normal spaces}










\part{Continuous Function Spaces}
\chapter{Compact-open topology}


Topologies on $C_c(X)$ for LCH $X$: weaker to stronger
\begin{parts}
\item Topology of compact convergence: $\cl{C_c(X)}=\cl{C_{\mathrm{loc}}(X)}=C_{\mathrm{loc}}(X)$.
\item Topology of uniform convergence: $\cl{C_c(X)}=C_0(X)$, $C_c(X)^*=M(X)$.
\item Inductive topology: $\cl{C_c(X)}:=\cl{\colim_{U\Subset X}C_c(U)}=C_c(X)$.
\end{parts}
The space $C_{\mathrm{loc}}(X)$ is defined to be $C(X)$ as a set endowed with the topology of compact convergence.



\chapter{Properties of continuous function spaces}

\section{The Arzela-Ascoli theorem}



















\section{The Stone-Weierstrass theorem}

\subsection{The classical Weierstrass approximation theorem}
\begin{prb}[Bernstein polynomial]
Let $f\in C([0,1],\R)$.
Define
\[B_n(f):=\sum_{k=0}^nf\left(\frac kn\right)\binom nkx^k(1-x)^{n-k}.\]
\begin{parts}
\item $B_n(f)$ uniformly converges to $f$ on $[0,1]$. In particular $\R[x]$ is 
\item For any $\e>0$, there is a polynomial $p\in\R[x]$ such that $p(0)=0$ and $\sup_{x\in[-1,1]}\bigl||x|-p(x)\bigr|<\e$.
\end{parts}
\end{prb}

\begin{prb}[Taylor series of square root]
Let
\[f_n(x):=\sum_{k=0}^n a_k(x-1)^k\]
be the partial sum of the Taylor series of the square root function $\sqrt x$ at $x=1$.
\begin{parts}
\item By Abel's theorem, $f_n$ uniformly converges to $\sqrt x$ on $[0,1]$
\item For any $\e>0$, there is a polynomial $p\in\R[x]$ such that $p(0)=0$ and $\sup_{x\in[-1,1]}\bigl||x|-p(x)\bigr|<\e$.
\end{parts}
\end{prb}


\subsection{Proof of the Stone-Weierstrass theorem}
\begin{prb}
Let $X$ be a compact Hausdorff space and $S\subset C(X,\R)$.
We say that $S$ \emph{separates points} if for every distinct $x$ and $y$ in $X$ there is $f\in S$ such that $f(x)\ne f(y)$, and that $S$ \emph{vanishes nowhere} if for every $x$ in $X$ there is $f\in S$ such that $f(x)\ne0$.

Let $\cA=\cl{S\R[S]}$ be the real Banach subalgebra of $C(X,\R)$ generated by $S$.
\begin{parts}
\item $\cA$ is a lattice.
\item $\cA$ is equal to $C(X,\R)$
\end{parts}
\end{prb}





\subsection{Generalizations of the Stone-Weierstrass theorem}

Locally compact version and complex version








\begin{prb}
Some examples
\begin{parts}
\item $z\R[z]$ is dense in $C([1,2],\R)$.
\item $\C[z]$ is dense in $C([0,1],\C)$.
\item $z\C[z,\bar z]$ is dense in $C(\T,\C)$.
\end{parts}
\end{prb}






\chapter{The Riesz-Markov-Kakutani representation theorem}
\section{Baire measures}

\section{Regular Borel measures}

\begin{thm}
Let $\mu$ be a Borel measure.
Then, the followings are equivalent:
\begin{parts}
\item$\mu$ is inner regular on $\sigma$-bounded sets
\item $\mu$ is outer regular on $\sigma$-bounded sets.
\end{parts}
\end{thm}


A regular measure of an open set $U$ can be realized by the norm of integration functional on $C_c(U)$.
\begin{lem}
Let $\mu$ be a Borel measure on a LCH $X$.
Then, $\mu$ is inner regular on open sets iff
\[\mu(U)=\|\mu\|_{C_c^*(U)}\]
for every open $U$ in $X$.
\end{lem}
\begin{pf}
($\ge$)
For $f\in C_c(U)$, we have
\[|\int f\,d\mu|=|\int_Uf\,d\mu|\le\mu(U)\,\|f\|.\]

($\le$)
Since $\mu$ is inner regular on $U$, there is a compact set $K\subset U$ such that $\mu(U)-\mu(K)<\e$ (for the case $\mu(U)=\infty$, we can deal with separately).
We can find a nonnegative function $f\in C_c(U)$ with $f|_K \equiv 1$ and $f\le1$ by the construction of Urysohn.
Then, for all $\e>0$ we have
\[\mu(U)<\mu(K)+\e\le\int f\,d\mu+\e\le\|\mu\|_{C_c^*(U)}+\e.\qedhere\]

($\Rightarrow$)
Let $f\in C_c(U)$ be a function such that $\|f\|=1$ and
\[\mu(U)-\e<\int f\,d\mu.\]
Let $K=\supp(f)$.
Then
\[\mu(K)\ge\int f>\mu(U)-\e.\]
\end{pf}
% 이런 거 쓸 때 메져가 유한인지 무한인지 케이스 나누고 증명쓰는 게 좋겠다


Positivity of linear functional itself implies a rather strong continuity property.
\begin{thm}
Let $X$ be LCH.
A positive linear functional on $C_c(X)$ is continuous with respect to the inductive topology.
\end{thm}
\begin{pf}
Let $I$ be a positive linear functional on $C_c(X)$.
We want to show every restriction of $I$ on $C_c(U)\emb C_c(X)$ for $U\Subset X$ is continuous for uniform norm.

Choose a nonnegative $\f\in C_c(X)$ such that $\f|_{\cl{U}}\equiv1$ using the Urysohn lemma.
Then,
\[|I(f)|\le I(|f|)=I(\f|f|)\le I(\f\|f\|)=I(\f)\|f\|\lesssim_U\|f\|\]
for $f\in C_c(U)$.
\end{pf}


Locally compact Hausdorff spaces have at least two important applications in abstract analysis related to measure theory: one is locally compact groups and the associated Harr measures in abstract harmonic analysis, the other is the Gelfand-Naimark theorem which states every commutative $C^*$-algebra can be represented as a function space on a locally compact Hausdorff space.
In the set of this section, we assume every base space $X$ is locally compact Hausdorff.

Note that locally finite measures are compact finite but the converser holds only if in locally compact Hausdorff spaces.
We want to consider locally finite Borel measures as the minimally compatible measures with a given topology on $X$.
For locally finite Borel measures, a set is finite-measured if and only if it is contained in a compact set.
\begin{defn}
A \emph{Radon measure} is a Borel measure on $X$ which satisfies the following three conditions:
\begin{parts}
\item locally finite,
\item outer regular on all Borel sets,
\item inner regular on all open sets.
\end{parts}
\end{defn}

Radon measures are rather simply characterized when the base space $X$ is $\sigma$-compact.
The following proposition proves the equivalence between regularity and Radonness of locally finite Borel measure on a $\sigma$-compact space.
\begin{prop}
A Radon measure is inner regular on all $\sigma$-finite Borel sets.(Folland's)
\end{prop}
\begin{pf}
First we approximate Borel sets of finite measure, with compact sets.
Let $E$ be a Borel set with $\mu(E)<\infty$ and $U$ be an open set containing $E$.
By outer regularity, there is an open set $V\supset U-E$ such that
\[\mu(V)<\mu(U-E)+\frac\e2.\]
By inner regularity, there is a compact set $K\subset U$ such that
\[\mu(K)>\mu(U)-\frac\e2.\]
Then, we have a compact set $K-V\subset K-(U-E)\subset E$ such that
\begin{align*}
\mu(K-V)&\ge\mu(K)-\mu(V)\\
&>\left(\mu(U)-\frac\e2\right)-\left(\mu(U-E)+\frac\e2\right)\\
&\ge\mu(E)-\e.
\end{align*}
It implies that a Radon measure is inner regular on Borel sets of finite measures.

Suppose $E$ is a $\sigma$-finite Borel set so that $E=\bigcup_{n=1}^\infty E_n$ with $\mu(E_n)<\infty$.
We may assume $E_n$ are pairwise disjoint.
Let $K_n$ be a compact subset of $E_n$ such that
\[\mu(K_n)>\mu(E_n)-\frac\e{2^n},\]
and define $K=\bigcup_{n=1}^\infty K_n\subset E$.
Then,
\[\mu(K)=\sum_{n=1}^\infty\mu(K_n)>\sum_{n=1}^\infty\left(\mu(E_n)-\frac\e{2^n}\right)=\mu(E)-\e.\]
Therefore, a Radon measure is inner regular on all $\sigma$-finite Borel sets.
\end{pf}
We get a corollary:
\begin{cor}
If $X$ is $\sigma$-compact, then a locally finite Borel measure is Radon if and only if it is regular.
\end{cor}

\begin{thm}
If every open set in $X$ is $\sigma$-compact(i.e. Borel sets and Baire sets coincide), then every locally finite Borel measure is regular.
\end{thm}
\begin{prop}
In a second countable space, every open set is $\sigma$-compact(i.e. Borel sets and Baire sets coincide).
\end{prop}


Two corollaries are presented as follows:
\begin{rd}[column sep={120pt,between origins}]
\parbox{7em}{\centering locally finite \\ Borel regular} \rar &
\parbox{5em}{\centering Radon} \rar \lar[dashed, bend right, swap]{$X$ is $\sigma$-compact} &
\parbox{7em}{\centering locally finite \\ Borel} \ar[dashed, bend left]{ll}{$X$ is second countable}
\end{rd}


Many applications assume $X$ is an open subset of a Euclidean space, so $X$ is usually second countable.
In this case, the followings will be synonym: A measure is
\begin{parts}
\item 
\end{parts}



\[L_{\text{loc}}^1=\text{a.c. measures}\subset\text{LfB measures}\subset\text{Radon measures}\subset\cD'.\]


\begin{thm}
Every finite Radon measure is regular.
\end{thm}







\section{Continuous dual of $C_0$}
In this section, we always assume $X$ is a locally compact Hausdorff space.
Hence we can use the Urysohn lemma in the following way: If a compact subset $K$ and a closed subset $F$ are disjoint, then by applying the Urysohn lemma on a compact neighborhood of $K$, we can find a continuous function $\f:X\to[0,1]$ such that $\f|_K=1$ and $\f|_F=0$.
In particular, there always exists a ``continuous characteristic function'' $\f\in C_c(X)$ with $\f|_K=1$.

There are two Riesz-Markov-Kakutani theorems: the first theorem describes the positive elements in $C_c(X)^*$ as Radon measures when the natural colimit topology is assumed, and the second theorem describes $C_c(X)^*$ as the space of finite Radon measures when uniform topology is assumed.




\begin{thm}[The Riesz-Markov-Kakutani representation theorem]
Let $X$ be LCH.
Let $I$ be a positive linear functional on $C_c(X)$.
Then, there is a unique Radon measure $\mu$ on $X$ such that
\[I(f)=\int f\,d\mu.\]
\end{thm}
\begin{pf}
\Step{0}[Uniqueness]
Let $\mu$ be a Radon measure on $X$.
By the local finiteness, $\int f\,d\mu$ is finite for all $f\in C_c(X)$.
Suppose we know the value of $\int f\,d\mu$ for all $f\in C_c(X)$.
By the inner regularity,
\[\mu(U)=\|\mu\|_{C_c^*(U)}\]
is determined for all open $U$.
By the outer regularity,
\[\mu(E)=\inf\{\,\mu(U):E\subset U,\ U\text{ is open}\,\}\]
is determined for all Borel $E$.
So we are done.

\Step{1}[Carath\'eodory construction]
Define a set function $\mu$ on the topology by
\[\mu(U):=\|I\|_{C_c^*(U)}.\]
If $\mu$ is a premeasure, then
\[\mu(E):=\inf\{\,\mu(U):E\subset U,\ U\text{ is open}\,\}\]
defines a Borel measure by the Carath\'eodory extension.
We only need countable additivity of $\mu$ for open sets to show $\mu$ is a premeasure.

\Step{2}[A lemma]
Let $f|_K\equiv1$.
For $U=\{x:f(x)>1-\e\}$, there is $g\in C_c(U)$ such that $g\le1$ and $I(g)>\mu(U)-\e$.
Then,
\[\mu(K)\subset\mu(U)<I(g)+\e<\tfrac1{1-\e}I(f)+\e\]
implies $\mu(K)\le I(f)$.

Let $f\le1$.
There is an open set $U\supset\supp(f)$ such that $\mu(U)<\mu(\supp(f))+\e$ since $\supp(f)$ is Borel.
For $U\supset\supp(f)$, there is $g\in C_c(U)$ such that $g\le1$ and $g|_{\supp(f)}\equiv1$ by the Urysohn lemma.
Then,
\[I(f)\le I(g)\le\mu(U)<\mu(\supp(f))+\e\]
implies $I(f)\le\mu(\supp(f))$.

\Step{3}[Realization as integration]
Suppose $f\in C_c(X)$ and $0\le f\le1$.
\[\mu(f^{-1}(1))\le I(f)\le\mu(f^{-1}(0)).\]


\Step{4}[Radonness]
Trivial.


\end{pf}

\section{Topological measures}
\begin{prb}
Let $X$ be compact.
A positive linear functional $\rho$ on $C(X)$ is bounded with norm $\rho(1)$.
\end{prb}
\begin{pf}
Since $0\le\rho(\|f\|\pm f)=\|f\|\rho(1)\pm\rho(f)$, we have $|\rho(f)|\le\rho(1)\|f\|$.
\end{pf}

\begin{prb}
Let $X$ be a locally compact Hausdorff space.
\begin{parts}
\item The Baire $\sigma$-algebra is generated by compact $G_\delta$ sets.
\item If $X$ is second countable, then every Baire set is Borel.
\end{parts}
\end{prb}
\begin{sol}
(b)
(A second countable locally compact space is $\sigma$-compact.

Since $X$ is $\sigma$-compact and Hausdorff, every closed set is a countable union of compact sets, so the Borel $\sigma$-algebra on $X$ is generated by compact sets.)

Since locally compact Hausdorff space is regular, the Urysohn metrization implies $X$ is metrizable, and every closed sets in metrizable space is $G_\delta$ set.
\end{sol}

\subsection{The Riesz-Kakutani theorem for positive linear functionals}
\begin{prb}
Let $X$ be compact.
There is a map from the set of finite Baire measures to the set of positive linear functionals on $C(X)$.
\end{prb}
\begin{sol}
A function in $C(X)$ is Baire measurable and bounded.
Thus the integration is well-defined.
\end{sol}

\begin{prb}
Let $X$ be compact.
There is a map from the set of positive linear functionals on $C(X)$ to the set of finite regular Borel measures.
\end{prb}
\begin{sol}
i. and ii. and iii. of Theorem 7.2.
\end{sol}


\begin{prb}
Let $X$ be compact.
Let $\rho$ be a positive linear functional on $C(X)$.
Let $\nu$ be the regular Borel measure associated to $\rho$.
Then, $\rho(f)=\int f\,d\nu$.
\end{prb}
\begin{sol}
iv. of Theorem 7.2.
\end{sol}

\begin{prb}
Let $X$ be compact.
Let $\nu$ be a finite regular Borel measure.
Let $\nu'$ be the regular Borel measure associated to the positive linear functional $f\mapsto\int f\,d\nu$.
Then, $\nu=\nu'$ on Borel sets.
\end{prb}
\begin{sol}
Theorem 7.8.
\end{sol}

The two results above establish the correspondence between positive linear functionals and regular Borel measures.
The following is an additional topic: Borel extension of Baire measures.
\begin{prb}
Let $X$ be compact.
Let $\mu$ be a finite Baire measure.
Let $\nu$ be the regular Borel measure associated to the positive linear functional $f\mapsto\int f\,d\mu$.
Then, $\mu=\nu$ on Baire sets.
\end{prb}
\begin{sol}
Let $\mu,\nu$ be finite Baire measures.
Enough to show if $\int f\,d\mu=\int f\,d\nu$ then $\mu=\nu$ according to the preceding two results.

Enough to show the regularity of Baire measures.
\end{sol}







\begin{itemize}
\item A second countable locally compact space is $\sigma$-compact.
\item A $\sigma$-compact locally compact space is paracompact.
\item A second countable regular space is paracompact.
\item A locally compact Hausdorff space is regular.
\end{itemize}

semiring
$\sigma$-finiteness implies the uniqueness





\end{document}
