\documentclass{../note}
\usepackage{../../ikany}


\begin{document}
\title{Operator Algebra}
\author{Ikhan Choi}
\maketitle
\tableofcontents

\part{C$^*$-algebras}
\chapter{}




\section{Multiplier algebra}

\begin{prb}[Multiplier algebra]
Let $\cA$ be a C$^*$-algebra.
A \emph{double centralizer} of $\cA$ is a pair $(L,R)$ of bounded linear maps on $\cA$ such that $aL(b)=R(a)b$ for all $a,b\in\cA$.
The \emph{multiplier algebra} $M(\cA)$ of $\cA$ is defined to be the set of all double centralizers of $\cA$.
\end{prb}

\begin{prb}[Essential ideals]
\begin{parts}
\item Hilbert C$^*$-module description
\end{parts}
\end{prb}

\begin{prb}[Examples of multiplier algebras]
\begin{parts}
\item $M(K(H))\cong B(H)$.
\item $M(C_0(\Omega))\cong C_b(\Omega)$.
\end{parts}
\end{prb}
\begin{pf}
(a)

(b)
First we claim $C_0(\Omega)$ is an essential ideal of $C_b(\Omega)$.
Since $C_b(\Omega)\cong C(\beta\Omega)$, and since closed ideals of $C(\beta\Omega)$ are corresponded to open subsets of $\beta\Omega$, $C_0(\Omega)\cap J$ is not trivial for every closed ideal $J$ of $C_b(\Omega)$.

Now we have an injective $^*$-homomorphism $C_b(\Omega)\to M(C_0(\Omega))$, for which we want to show the surjectivity.
Let $g\in M(C_0(\Omega))^+$.
\end{pf}

\begin{prb}[Strict topology]
\end{prb}



\section{Hereditary C$^*$-subalgebras}
\begin{prb}[Hereditary C$^*$-subalgebra and state embedding]

\end{prb}










\part{Von Neumann algebras}

\chapter{Factor classifications}

\section{}
\begin{prb}[Semi-finite traces]
Let $M$ be a von Neumann algebra and $\tau$ is a trace.
For a trace $\tau$
\begin{parts}
\item $\tau$ is semi-finite if and only if $x\in M^+$ has a net $x_\alpha\in L^1(M,\tau)^+$ such that $x_\alpha\uparrow x$ strongly.
\item Let $\tau$ be normal and faithful. Then, $\tau$ is semi-finite if and only if
\[\tau(x)=\sup\{\,\tau(y):y\le x,\ y\in L^1(M,\tau)^+\,\}\quad\text{ for }\quad x\in M^+.\]
\end{parts}
\end{prb}

\section{}



\chapter{}
\section{Connes' bicentralizer problem}




\part{Operator K-theory}
\chapter{Brown-Douglas-Fillmore theory}
\begin{prb}[Haagerup property]
\end{prb}

Baum-Connes conjecture
Non-commutative geometry
Elliott theorem














\part{Subfactor theory}








\end{document}