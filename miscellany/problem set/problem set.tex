\documentclass{../../large}
\usepackage{../../ikhanchoi}

\begin{document}
\title{Problem set}
\author{Ikhan Choi}
\maketitle
\tableofcontents

\part{Algebra}
\chapter{Equations}
\section{Polynomials}
\section{Simultaneous equations}
\section{Real solutions}

\begin{prb}[Factorization]
\end{prb}
\begin{prb}[Discriminant]
\end{prb}
\begin{prb}[Image of square function]
\end{prb}
\begin{prb}[Intermediate value theorem]
\end{prb}

\section{Integer solutions}
\begin{prb}[Factorization]
\end{prb}
\begin{prb}[Square roots]
\end{prb}
\begin{prb}[Gaps between perfect squares]
\end{prb}


\chapter{Inequalities}
\section{Symmetry}
\section{Homogeneity}


\chapter{Functions}
\section{Properties of functions}
\section{Functions over $\R$}
\section{Other domains}




\part{Combinatorics}

\chapter{Counting}
\section{Orbits}
\section{Generating functions}

\chapter{Algorithms}
\section{Invariants}
\section{Games}

\chapter{Graphs}
\section{Double counting}
\section{Non-constructive existence}
Pigeonhole principle,
Probabilistic methods,
Extremal theory




\part{Geometry}
\chapter{Plane geometry}
\section{Angle chasing}
Cyclic quadrilaterals

\section{Length ratios}
menelous and ceva

\section{Triangle centers}

\section{Conics}



\chapter{Analytic methods}
\section{Trigonometry}
\section{Complex variables}
\section{Barycentric coordinates}



\chapter{Transformations}
\section{Similarity}
spiral homothety

\section{Inversion}

\section{Projectivity}




\part{College math}
\chapter{Calculus}


% Limits, Differentiation, Integration, Multivariable

\section{Asymptotics}
\section{Infinite series}
Let $a_n$ be a real sequence and $S_n:=a_1+\cdots+a_n$ be the partial sum.
\begin{enumerate}
\item Show that if $a_n\downarrow0$ and $S_n\le1+na_n$, then $S_n\le1$.
\end{enumerate}



\section{Indefinite integrals}
\section{Integral inequalities}






\chapter{Linear algebra}


\section{Determinants}

\section{Spectrum}
canonical forms

\section{Commuting matrices}
two by two matrices

\section{Positive definiteness}










\part{General physics}
\setcounter{chapter}{0}



% 역학 거의 10
% 유체 25
% 파동 70
% 열 90

% 전자기 거의 370
% 광 100
% 현대물리 250

\chapter{Mechanics}


\section{Equation of motion}
\begin{prb}[Projectile motion]
Let $e$ and $g$ be the coefficient of restitution and gravitational acceleration. 
\begin{parts}
\item At the time a particle at the origin is thrown with an initial speed $v_0$, another particle at $(a,b)$ with $b>0$ begins a free fall. Find the minimum value of $v_0$ such that two particles collide in the region $y\ge0$.
\item A particle at the ground is projected with an initial speed $v_0$ at an angle $\theta$, towards a vertical wall as far away as $L$. Find $v_0$ such that it bounces back to its original position after striking the wall and ground only once. 
\item A particle is released at height $h$ from a plane inclined at an angle $\theta$ to the horizontal. Find the length $l$ from the first point of collision to the point at which the particle begins to slide down.
\end{parts}
\end{prb}
\begin{sol}
(b)
$R:=v_0^2\sin^2\theta/2g$
\[L+e(R-L)+eR=2L.\]
\end{sol}

\begin{prb}[Normal force]
\end{prb}
\begin{prb}[Tension]
\end{prb}
\begin{prb}[Pulley]
\end{prb}
\begin{prb}[Friction]
\end{prb}


\section{Rigid body}

\begin{prb}[Equilibrium]
\end{prb}
\begin{prb}[Rolling disks]
\end{prb}
\begin{prb}[Parallel axis]
\end{prb}


\section{Conservation laws}
\begin{prb}[Gravitational energy]
\end{prb}
\begin{prb}[Elastic energy]
\end{prb}
\begin{prb}[Ellastic collision]
\end{prb}
\begin{prb}[Inelastic collision]
\end{prb}
\begin{prb}[Angular momentum]
\end{prb}



\section{Centripetal force}
\begin{prb}[Circular motion]
\end{prb}
\begin{prb}[Oscillation]
\end{prb}
\begin{prb}[Central force]
effective potential
\end{prb}
\begin{prb}[Fictitious force]
\end{prb}


\section{Fluids}



\chapter{Waves}
% 반사투과 경계조건 정상파
% 도플러 중첩 맥놀이
\section{Waves on a string}
\begin{prb}[Boundary conditions]
\end{prb}
\begin{prb}[Superposition]
\end{prb}
\begin{prb}[Standing waves]
\end{prb}

\section{Sound waves}
\begin{prb}[Doppler effect]
\end{prb}
beat,
Helmholtz resonator,
supersonic waves,
shock,






\chapter{Thermodynamics}

\section{First law}
Equation of states?
\begin{prb}[Kinetic theory]
\end{prb}
\begin{prb}[Quasi-static processes]
\end{prb}
\begin{prb}[Composite systems]
\end{prb}

\section{Second law}
\begin{prb}[Entropy]
\end{prb}
\begin{prb}[Free energies]
\end{prb}






\chapter{Electromagnetism}
\section{Electrostatics}
% 축전기

\section{Magnetostatics}
% 하전입자의 운동

\section{Electromagnetic induction}


\section{Circuits}






\chapter{Optics}
\section{Geometric optics}
% 반사굴절 기하광학 렌즈

\section{Interference}

\section{Diffraction}
% 무지개


\chapter{Modern physics}
\section{Special relativity}

\section{Atoms}
% 
\section{Nuclear physics}


\end{document}