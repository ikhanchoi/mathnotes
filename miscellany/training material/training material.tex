\documentclass{../../large}
\usepackage{../../ikhanchoi}

\begin{document}
\tableofcontents

\part{Algebra}
\chapter{Equations}
\section{Polynomials}
\section{Simultaneous equations}
\section{Real solutions}
factorization
discriminant
image of square(polynomial)
intermediate value
\section{Integer solutions}
factorization
root identity
image of square(polynomial)


\chapter{Inequalities}
\section{Symmetry}
\section{Homogeneity}


\chapter{Functions}
\section{Properties of functions}
\section{Functions over $\R$}
\section{Other domains}




\part{Combinatorics}

\chapter{Counting}
\section{Orbits}
\section{Generating functions}

\chapter{Algorithms}
\section{Invariants}
\section{Games}

\chapter{Graphs}
\section{Double counting}
\section{Non-constructive existence}
Pigeonhole principle,
Probabilistic methods,
Extremal theory




\part{Geometry}
\chapter{Plane geometry}
\section{Angle chasing}
Cyclic quadrilaterals

\section{Length ratios}
menelous and ceva

\section{Triangle centers}

\section{Conics}



\chapter{Analytic methods}
\section{Trigonometry}
\section{Complex variables}
\section{Barycentric coordinates}



\chapter{Transformations}
\section{Similarity}
spiral homothety

\section{Inversion}

\section{Projectivity}




\part{Collegiate courses}
\chapter{Calculus}
\section{Asymptotics}
\section{Infinite series}
Let $a_n$ be a real sequence and $S_n:=a_1+\cdots+a_n$ be the partial sum.
\begin{enumerate}
\item Show that if $a_n\downarrow0$ and $S_n\le1+na_n$, then $S_n\le1$.
\end{enumerate}
\section{Indefinite integrals}
\section{Integral inequalities}



\chapter{Linear algebra}
\section{Determinants}

\section{Spectrum}
canonical forms

\section{Commuting matrices}
two by two matrices

\section{Positive definiteness}





\chapter{General physics}
\section{Mechanics}
% Kinematics: 운동방정식 회전운동 평형 / 보존법칙
% Kinetics: 포물선 관성력 원운동 단진동 중심력

Let $e$ and $g$ be the coefficient of restitution and gravitational acceleration. 
\begin{enumerate}
\item At the time a particle at the origin is thrown with an initial speed $v_0$, another particle at $(a,b)$ with $b>0$ begins a free fall. Find the minimum value of $v_0$ such that two particles collide in the region $y\ge0$.
\item A particle at the ground is projected with an initial speed $v_0$ at an angle $\theta$, towards a vertical wall as far away as $L$. Find $v_0$ such that it bounces back to its original position after striking the wall and ground only once. 
\item A particle is released at height $h$ from a plane inclined at an angle $\theta$ to the horizontal. Find the length $l$ from the first point of collision to the point at which the particle begins to slide down.
\end{enumerate}

\section{Waves}
% 경계조건 정상파 도플러 중첩 맥놀이
\section{Thermodynamics}
% 상태(이론빌딩) 열적과정 복합계 엔트로피 자유에너지
\section{Electromagnetism}
% 정전기학(1) 축전기 정자기학(4) 전자기유도(2) 하전입자운동
% 직류회로 교류회로
\section{Optics}
% 반사굴절 기하광학 간섭 회절 편광 무지개 프레넬
\section{Atoms}
% 


\end{document}