\documentclass{../../large}
\usepackage{../../ikhanchoi}


\begin{document}
\title{Geometry I}
\author{Ikhan Choi}
\maketitle
\tableofcontents

\part{Classical geometry}
\chapter{Euclidean geometry}

\section{Plane geometry}
\section{Solid geometry}
\section{Axiomatization}

\chapter{Non-Euclidean geometry}
\section{Absolute geometry}
axioms 1 to 4
\section{Spherical and elliptic geometry}
axioms 2 and 4
\section{Hyperbolic geometry}
axioms 1 to 4

Models of hyperbolic geometry (metric description)
Elementary figures
Isometries
Length, volume, angle

\chapter{Non-metric geometry}
\section{Ordered and incidence geometry}
axioms 1 and 2
\section{Affine and projective geometry}
axioms 1,2,5
\section{Conformal and inversive geometry}





\part{Classical topology}


\chapter{Fundamental groups}

\section{Homotopy}


\begin{prb}[Pointed spaces]
\end{prb}



\begin{prb}[Mapping cylinders]

\[M_f:=(I\times X)\sqcup_fY.\]

We have a natural factorization
\[\begin{tikzcd}[sep=tiny]
X\ar{dr}\ar{rr}{f}&&Y\\
&M_f\ar[swap]{ur}{\sim}&
\end{tikzcd}\]
where $M_f\to Y$ is a trivial cofibration.
\end{prb}

\begin{prb}[Cones]
Let $X$ be a pointed or non-empty unpointed space.
The \emph{cone} or the \emph{fibrant replacement} of $X$ is a pointed space defined by the quotient of the cylinder space
\[CX:=(I\times X)/(\{0\}\times X),\]
together with a base-point given as the image of $\{0\}\times X$.
We have two natural maps $X\to CX\xrightarrow{\sim}*$ where $CX\to*$ is a trivial fibration.
\end{prb}



\begin{prb}[Mapping cones]
Let $f:X\to Y$ be a pointed or unpointed map.
The \emph{mapping cone} or the \emph{homotopy cofiber} of a continuous map $f$ is defined by
\[C_f:=CX\sqcup_fY.\]

\end{prb}

\[X\xrightarrow{f}Y\to C_f\to\Sigma X\]

\begin{prb}
The \emph{fundamental group} $\pi_1(X)$ of a pointed space $X$ is defined as the homotopy class of pointed maps $\gamma:S^1\to X$.
In other notations, $\pi_1(X):=[S^1,X]$.

The fundamental group is a group.
It is a homotopy invariant product-preserving functor.
\end{prb}

We say $p:Y\to B$ has the \emph{homotopy lifting property} if it has the right lifting propert with respect to $\iota_0:A\to A\times I$ for any space $A$.

We say $i:A\to X$ has the \emph{homotopy extension property} if it has the left lifting property with respect to $\pi_0:B^I\to B$ for any space $B$.



\begin{prb}
For path connected spaces, it is customary to forget the basepoint.
In this case, $\pi_1(X)$ should be understood as an isomorphism class of groups, not a group
\end{prb}

\begin{prb}[Van Kampen theorem]
\end{prb}

\section{Covering spaces}

% a covering space is a fiber bundle which is a sheaf.


\begin{prb}
A continuous map $p:E\to B$ is called a \emph{covering} if there is an open cover $\{V_j\}$ of $B$ such that each open subset $V_j$ of $B$ is \emph{evenly covered} by $p$, that is, 
\end{prb}


\begin{prb}[Unique homotopy lifting property]
Let $p:E\to B$ be a covering map.
We prove that it has the \emph{unique homotopy lifting property} in the sense that for any topological space $A$ together with the inclusion $i:A\to I\times A:a\mapsto(0,a)$, and for any continuous maps $f:A\to E$ and $g:I\times A\to B$ satisfying $pf=gi$, there is a unique lift $h:I\times A\to E$ such that we have a commutative diagram
\[\begin{tikzcd}
A\rar{f}\dar[>->,swap]{i}&E\dar[->>]{p}\\
I\times A\rar[swap]{g}\ar{ur}{\exists!h}&B.
\end{tikzcd}\]
When $A=*$, then the unique homotopy lifting property is also called the unique path lifting property.

\begin{parts}
\item The closed unit interval $I$ is the unique compact Hausdorff space such that every open cover has a linear, finite, connected, relatively compact refinement.
\item uniqueness of path lifting.
\item existence of homotopy lifting.
\end{parts}
\end{prb}
\begin{pf}



(a)


We first prove the uniqueness of the lift $h$ when $A$ is a point.



If $h$ and $h'$ are lifts of $g$ along $p$ such that their images of $J$ are contained in single sheets $\tilde V$ and $\tilde V'$, and if $h(t)=h'(t)$ for some $t\in J$, then since $h=p|_{\tilde V}^{-1}g$ and $h=p|_{\tilde V'}^{-1}g$, where $p(\tilde V)=p(\tilde V')=V$, then $\tilde V\cap\tilde V'\ne\varnothing$ implies $\tilde V=\tilde V'$, so $h(t)=h'(t)$ for all $t\in J$...





(b)
For an open cover $\{V_i\}$ of $B$ consisting of open subsets evenly covered by $p$, take an open cover $\{J_i\times U_i\}$ of $I\times A$ that is a refinement $\{f^{-1}(V_i)\}$.


We start from fixing $a\in A$.
Let $b:=pf(a)=gi(a)$.

\[\begin{tikzcd}
U_0\rar{f}\dar[>->,swap]{i}&s_0(V_0)\dar[->>]{p}\\
J_0\times U_0\rar[swap]{g}\ar{ur}{\exists!h_0}&V_0.
\end{tikzcd}\]

Take $V_0$ such that $b\in V_0$.

Take a local section $s_0:V_0\to E$ such that $f(\{a\})\cap s_0(V_0)\ne\varnothing$, i.e.~$s_0(b)=f(a)$.

Take $J_0\times U_0$ such that $i(a)\in J_0\times U_0$, and $g:J_0\times U_0\to V_0$.

We may assume $U_0$ is small such that $f:U_0\to s_0(V_0)$ since $s_0(V_0)$ is open in $E$.



Define $h_0:J_0\times U_0\to E$ by $h_0:=s_0g$ on $J_0\times U_0$.

Then, $h_0i=f$ on $U_0$ since
\[\{s_0pf(a'),f(a')\}\subset p^{-1}(b')\cap s_0(V_0),\qquad a'\in U_0\]
implies $s_0pf=f$ on $U_0$ so that
\[h_0i(a')=s_0gi(a')=s_0pf(a')=f(a'),\qquad a'\in U_0.\]

Also, $ph_0=g$ on $J_0\times U_0$ by
\[ph_0(t,a')=ps_0g(t,a')=g(t,a'),\qquad(t,a')\in J_0\times U_0.\]


\[\begin{tikzcd}
&s_0(V_0)&\\
(J_0\cap J_1)\times U_1\ar{ur}{h_0}&&s_1(V_1)\dar[->>]{p}\\
&J_1\times U_1\rar[swap]{g}\ar{ur}{\exists!h_1}&V_1.
\end{tikzcd}\]

Take $V_1$.

Take $J_1\times U_1$.

Take a local section $s_1:V_1\to E$ such that $h_0((J_0\cap J_1)\times\{a\})\cap s_1(V_1)\ne\varnothing$.

\qquad(\because $ph_0((J_0\cap J_1)\times\{a\})=g((J_0\cap J_1)\times\{a\})\subset g(J_1\times U_1)\subset V_1$.)


Then, $h_0:(J_0\cap J_1)\times\{a\}\to s_1(V_1)$ since $(J_0\cap J_1)\times\{a\}$ is \textbf{connected} and $s_1(V_1)$ is \textbf{clopen} in $p^{-1}(V_1)$.

We may assume $U_1\subset U_0$ is small such that $a\in U_1$ and $h_0:(K_0\cap K_1)\times U_1\to s_1(V_1)$ since $\bar{(K_0\cap K_1)}\times\{a\}$ is \textbf{compact}.
($\bar K_i\subset J_i$ implies $\bar{(K_0\cap K_1)}\subset J_0\cap J_1$.)


Define $h_1:J_1\times U_1\to E$ by $h_1=s_1g$ on $J_1\times U_1$.

\qquad(\because $g(J_1\times U_1)\subset V_1$ and $s_1$ is defined on $V_1$)

Then, for $(t,a')\in(J_0\cap J_1)\cap U_1$, since $h_0(t,a')\in h_0((J_0\cap J_1)\times U_1)\subset s_1(V_1)$ and $ph_0(t,a')=g(t,a')$, and since $h_1(t,a')=s_1g(t,a')\in s_1(V_1)$ and $ph_1(t,a')=ps_1g(t,a')=g(t,a')$, we have $\{h_0(t,a'),h_1(t,a')\}\subset s_1(V_1)\cap p^{-1}(g(t,a'))=\{s_1g(t,a')\}$ so that $h_0(t,a')=h_1(t,a')$.

Furthermore, $ph_1(t,a')=g(t,a')$ for $(t,a')\in J_1\cap U_1$.




\bigskip

So far, we showed for each $a\in A$ there is an open neighborhood $U$ together with a continuous map $h:I\times U\to E$ satisfying $hi(a')=f(a')$ and $ph(t,a')=g(t,a')$ for all $a'\in U$ and $t\in I$.

Uniqueness at $a$ also follows by induction.

Suppose $h:I\times U\to E$ and $h':I\times U'\to E$ satisfy $hi=f=h'i$ on $U\cap U'$ and $ph=g=ph'$ on $I\times(U\cap U')$.
We claim $h=h'$ on $I\times(U\cap U')$.

Fix $a\in U\cap U'$.

Suppose we have a properly constructed open cover $\{J_i\times U_i\}_{i=0}^n$ of $I\times\{a\}$.


\bigskip

Take evenly covered $V_0$ by $p$ such that $g(J_0\times U_0)\subset V_0$.

Choose any local section $s_0:V_0\to p^{-1}(V_0)$ such that $s_0gi(a)=f(a)$. (Note that $h(0,a)=f(a)=h'(0,a)$.)

\qquad(\because $pf(a)=gi(a)$.)


For $t\in J_0$, since $h_0(J_0\times\{a\})\cap s_0(V_0)\ni\{f(a)=s_0g(0,a)\}\ne\varnothing$, we have $h(t,a)\in h_0(J_0\times\{a\})\subset s_0(V_0)$ by the \textbf{connectedness} of $h_0(J_0\times\{a\})$, and we also have $ph(t,a)=g(t,a)$, so $h(t,a)\in s_0(V_0)\cap p^{-1}(g(t,a))=\{s_0g(t,a)\}$.
This is same for $h'$, so $h(t,a)=s_0g(t,a)=h'(t,a)$.

Thus $h=h'$ on $J_0\times\{a\}$.

Inductively.... $h=h'$ on $J_i\times\{a\}$.



\end{pf}



As a corollary, if $\gamma_0$ and $\gamma_1$ are endpoint-preserving homotopic paths in $X$ and have lifts $\tilde\gamma_0$ and $\tilde\gamma_1$ such that $\tilde\gamma_0(0)=\tilde\gamma_1(0)$, then $\tilde\gamma_0$ and $\tilde\gamma_1$ are basepoint-preserving homotopic.

As a corollary, if $p:(E,y_0)\to(B,b_0)$ is a pointed covering map, the induced map $p_*:\pi_1(E,y_0)\to\pi_1(B,b_0)$ is injective.


(Existence)


A locally trivial fiber bundle is a Serre fibration.
A locally trivial fiber bundle over a paracompact Hausdorff space is a Hurewicz fibration.
A covering map is a Hurewicz fibration.

homotopy lifting property = lifting property with respect to trivial cofibrations....
$A\to I\times A$ is enough for trivial cofibrations?


\bigskip
Take an open cover $\{V\}$ of $B$ evenly covered open subsets by $p$.
Take open covers $\{J\}$ of $I$ and $\{U\}$ of $A$ such that a subcover of the open cover $\{J\times U\}$ of $I\times A$ refines $\{g^{-1}(V)\}$.


Fix $a\in A$.
Take an open cover $\{J_t^\circ\times U_t\}$ of $I\times\{a\}\subset I\times A$ such that $J_t$ compact, $U_t$ open, and $g(J_t\times U_t)\subset V_t=V_{g(t,a)}$.
By the property of $I$, we may consider the open cover $\{J_i\times U_i\}_{i=0}^n$ of $I\times\{a\}$ in $I\times A$ such that blabla.








\begin{prb}[Universal covering]
connected, locally path connected, semi-locally simply connected


Let $p:(\tilde B,\tilde b_0)\to(B,b_0)$ be a covering map.
Then, since $\pi_1(B,b_0)$ acts on the fiber $p^{-1}(b_0)$ freely and transitively, so it can be endowed with a group structure by fixing the identity as the basepoint $\tilde b_0$, which is isomorphic to $\pi_1(B,b_0)$.
Thus, to compute the fundamental group, we are sufficient to describe the group structure of $p^{-1}(b_0)$.

\begin{parts}
\item $\pi_1(B,b_0)$ acts on the fiber $p^{-1}(b_0)\subset\tilde B$ freely and transitively.
\end{parts}
\end{prb}
\begin{pf}

\end{pf}


\begin{prb}[Classification of covering spaces]
connected, locally path connected, semi-locally simply connected

We say $p$ is \emph{regular} if $p_*(\pi_1(Y,y_0))$ is normal in $\pi_1(B,b_0)$.

$\pi_1(B,b_0)/p_*(\pi_1(Y,y_0))\to p^{-1}(b_0)$ is always injective, and bijective if $Y$ is path connected.
\end{prb}


Examples: $S^1$, $\RP^n$.


\section*{Exercises}

\begin{prb}[Projective spaces]
For $n\ge1$, $\CP^n$ is simply connected. (fiber sequence...?)
\end{prb}




\chapter{Homology groups}
\section{Singular homology}

\begin{prb}[Eilenberg-Steenrod axioms]
\end{prb}

For a space $X$, we can associate a simplicial set $\mathrm{Sing}_\bullet(X):=\{\sigma:\Delta^\bullet\to X\text{ continuous}\}$ called the \emph{singular simplicial set}.

For a simplicial set $S_\bullet$ and an $R$-module $M$, the functors
\[R[-]\otimes_RM:\mathrm{Set}\to\mathrm{Mod}_R,\qquad\Hom_R(R[-],M):\mathrm{Set}^\op\to\mathrm{Mod}_R\]
give rise to the \emph{simplicial module} or Dold-Kan corresponded to the \emph{differential graded module} or conventionally called the \emph{chain complex} $C_\bullet(S,M)\in(\mathrm{Mod}_R)_\Delta\cong\mathrm{Ch}_{\ge0}(R)$, and the \emph{cochain complex} $C^\bullet(S,M)\in(\mathrm{Mod}_R)^\Delta\cong\mathrm{Ch}^{\ge0}(R)$.

The singular homology is given by the composition
\[\mathrm{Top}\to\mathrm{Set}_\Delta\to\mathrm{Ch}_{\ge0}(R)\to(\mathrm{Mod}_R)_{\ge0}.\]



geometric realization of a simplicial set.

For a CW complex $X$, we have $X\simeq|\mathrm{Sing}_\bullet(X)|$.

Can we find a simplicial set $S_\bullet$ such that $X\approx|S_\bullet|$ but which is easier to compute?

there are two ways to think about such simplicial sets: simplicial complex, (regular) cell complex?



How to compute simplicial maps from continuous maps, in terms of generators?

\section{Simplicial homology}



Simplicial homology is defined for simplicial complex, which is purely combinatorial.
The singular chain complex of a topological space is the most natural simplicial complex on it.
The simplicial homology of this is the singular homology as it is just the definition.

One can associate some other simplicial complexes by \emph{triangulations} to a topological space which are more convenient to compute the homology.
We now have to investigate which conditions make a simplicial complex generate same homology groups with singular homology.

Let $X$ be a simplicial complex.
A \emph{geometric realization} of $X$ is a topological space $|X|$ defined by....
For a topological space, a \emph{triangulation} is a homeomorphism from the geometric realization of a simplicial complex to the topological space.

\section{Cellular homology}




\section{Cohomology}

representability by the $K(A,n)$!



relative cohomology

vector bundles
Thom isomorphism
characteristic classes, blabla




\chapter{Classification of surfaces}
\section{Combinatorial surfaces}



triangulation
euler characteristic
genus


manifolds:
connected sum
orientability
Poincar\'e duality

If $M$ is an orientable closed $n$-dimensional topological manifold with the fundamental class $[M]\in H^n(M,R)$, then $[M]\frown\cdot:H^k(M,R)\to H_{n-k}(M,R)$ is an isomorphism for all $k$.



We classify closed connected surfaces.

- triangulation: it is homeomorphic to the geometric realization of a simplicial complex.

- planar diagram: it has the standard cell complex.








\end{document}