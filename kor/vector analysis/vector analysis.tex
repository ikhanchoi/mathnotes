\documentclass{../../large}
\usepackage[kor]{../../ikhanchoi}




\begin{document}
\title{벡터해석학}
\author{최익한}
\maketitle
\tableofcontents



\part{다변수미적분학}


\chapter{다변수미분}

\section{}
\section{}
\section{}
\section{}

다변수 일차 미분: 편미분, 전미분
다변수 고차 미분: 헤시안, 체인룰, 다변수 테일러
다중적분: 넓이 부피, 푸비니, 적분영역 설정



\chapter{다중적분}

\section{}
\section{}
\section{}
\section{}




\chapter{곡선과 곡면}

\section{좌표계와 미분}
극원구
속도 및 가속도 변환, 전향력
\section{좌표계와 적분}
극원구
자코비안
\section{곡선}
평면곡선(매개, 레벨, 곡률), 길이, 공간곡선(TNB)
\section{곡면}
공간곡면(매개,레벨), 접평면, 법선벡터, 넓이, 곡률







\part{벡터미적분학}

\chapter{벡터장}
\section{기울기}
Lagrange승수, 법선벡터
\section{발산}
\section{회전}
\section{Laplace연산자}
벡터항등식






\chapter{미분형식}
쐐기곱, 외미분, 호지별, 퍼텐셜, 좌표변환

\section{}
\section{}
\section{}
\section{}






\chapter{스토크스 정리}
\section{선적분}
길이, 일
\section{Green정리와 발산정리}
보존력
\section{면적분}
면적, 선속
\section{Kelvin-Stokes정리}

연속방정식, 공변미분 등 응용



\end{document}