\documentclass{../../large}
\usepackage{../../ikhanchoi}

\newcommand{\Cl}{\mathrm{Cl}}

\begin{document}
\title{Algebra I}
\author{Ikhan Choi}
\maketitle
\tableofcontents

\part{Groups}
\chapter{Natural numbers}

\section{Peano arithmetic}




\section{}

\begin{prb}[Von Neumann construction]

\end{prb}

algebraic and order structures


\section{Integers and rational numbers}


\section{Divisibility}












\chapter{Groups}
% 군의 활용에 집중, 군의 분류에는 관심을 끈다

\section{Groups}
\begin{prb}[Groups]
A \emph{group} is a set $G$ equipped with a binary operation $\cdot:G\times G\to G$ and a constant $e\in G$ satisfying
\begin{enumerate}[(i)]
\item for all $g,h,k\in G$ we have $(gh)k=g(hk)$,\hfill(associativity)
\item for all $g\in G$ we have $ge=eg=g$,\hfill(identity)
\item for all $g\in G$ there is $g^{-1}\in G$ such that $gg^{-1}=g^{-1}g=e$.\hfill(inverses)
\end{enumerate}
A group $G$ is called \emph{commutative} or \emph{abelian} if it satisfies
\begin{enumerate}[(i)]
\setcounter{enumi}{3}
\item for all $g,h\in G$ we have $gh=hg$.\hfill(commutativity)
\end{enumerate}
The equipped binary operation on a group is sometimes called the \emph{group structure}, and the constant $e$ is called the \emph{identity}.
We say a group is \emph{additive} if we use the symbol $+$, $0$, $-g$ for the group structure, the identity, and the inverse of an element $g$ of a group, and \emph{multiplicative} if we omit the symbol for the group structure and use the notation $e$ or $1$ for the identity.
For an abelian group, we basically regard it additive.
\begin{parts}
\item For $g_1,\cdots,g_n\in G$, the value of $g_1\cdots g_n$ is well-defined independently of how the expression is bracketed.
\item The identity of $G$ and the inverse of each element $g\in G$ are uniquely determined by the group structure.
\item $(g^{-1})^{-1}$ and $(gh)^{-1}=h^{-1}g^{-1}$ for all $g,h\in G$.
\item The left and right ancellation laws.
\end{parts}
\end{prb}


\begin{prb}[Group homomorphisms]
\end{prb}

\section{Subgroups}

\begin{prb}[Subgroups]
Lagrange theorem, cosets and index

subgroup lattice
\end{prb}

\begin{prb}[Generators]
group presentation
orders of elements
\end{prb}


\begin{prb}[Direct sums]
\end{prb}


\section{Quotient groups}
\begin{prb}[Normal subgroups]
\end{prb}

\begin{prb}[Isomorphism theorems]
\end{prb}

\begin{prb}[Direct products]
\end{prb}




\section{Examples of groups}
\begin{prb}[Cyclic groups]
\end{prb}
\begin{prb}[Dihedral groups]
\end{prb}
\begin{prb}[Dicyclic groups]
Quaternion group
\end{prb}
\begin{prb}[Symmetric and alternating groups]
sign homomorphism
generators, transpositions
cycle type
\end{prb}
\begin{prb}[Linear groups]
general, special
\end{prb}






\chapter{Group actions}

\section{Actions and representations}

Let $G$ be a group and $X$ be a set.
A \emph{left action} of $G$ on $X$ is a function $G\times X\to X:(g,x)\to gx$ such that $g(hx)=(gh)x$ and $ex=x$.
A \emph{left $G$-set} is a set $X$ together with a left action of $G$ on $X$.
We may define right actions and right $G$-sets similarly.

effective, free, transitive actions.
The orbit spaces of a left $G$-set $X$ is a set $G\backslash$ of orbits.
When we do not have to emphasize the $G$-space is left, that is we do not deal with both left and right actions simultaneously, we often write the orbit space just by $X/G$.

Let $H$ be a subgroup of $G$.
A left coset is an element of the orbit space of the right action $G\times H\to G$ of $H$ on $G$ given by the right multiplication.
Here we can define a left multiplication action of $G$ on $G/H$, which is transitive.



\begin{prb}[Automorphism groups]
\end{prb}





\section{Orbits and stabilizers}
Invariants on orbit space.

\begin{prb}[Orbit-stabilizer theorem]
The size of orbits.
The number of orbits.
The class equation.
\end{prb}


\begin{prb}[Transitive actions]
\begin{parts}
\item Stabilizers are all isomorphic.
\end{parts}
\end{prb}

\begin{prb}[Free actions]
no fixed point,
trivial stabilizer for any point,
every orbit has 1-1 correspondence to group
\end{prb}

\section{Action by left multiplication}

\section{Action by conjugation}
\begin{prb}[Centralizers and normalizers]
\end{prb}

\begin{prb}[Conjugacy classes of elements]
\end{prb}

\begin{prb}[Conjugacy classes of subgroups]
\end{prb}

H has index n  : G can act on Sym(G/H) : left mul
K normalizes H : K -> NG(H) -> NG(H)/H  with ker = KnH
K normalizes H : K -> NG(H) -> Aut(H)  with ker = CG(H)







\section*{Exercises}

\section*{Problems}

\begin{enumerate}
\item Show that a group of order $2p$ for a prime $p$ has exactly two isomorphic types.
\item Let $G$ be a finite group of order $n$ and $p$ the smallest prime divisor of $n$. Show that a subgroup of $G$ of index $p$ is normal in $G$.
\item Show that a finite group $G$ satisfying $\sum_{g\in G}\ord(g)\le2n$ is abelian.
\item Find all homomorphic images of $A_4$ up to isomorphism.
\item For a prime $p$, find the number of subgroups of $Z_{p^2}\times Z_{p^3}$ of order $p^2$.
\item Let $G$ be a finite group. If $G/Z(G)$ is cylic, then $G$ is abelian.
\item Let $G$ be a finite group. If the cube map $x\mapsto x^3$ is a surjective endomorhpism, then $G$ is abelian.
\item Show that if $|G|=p^2$ for a prime $p$, then a group $G$ is abelian.
\item Show that the order of a group with only on automorphism is at most two.
\end{enumerate}










\part{Rings}
\chapter{Rings}
\section{Rings}
\begin{prb}[Rings]
A \emph{ring} is an additive abelian group $R$ equipped with a binary operation $\cdot:R\times R\to R$ satisfying
\begin{enumerate}
\item[(i)] for all $r,s,t\in R$ we have $(rs)t=r\cdot(s\cdot t)$,\hfill(associativity)
\end{enumerate}
and the compatibility condition
\begin{enumerate}
\item[(v)] for all $r,s,t\in R$ we have $r(s+t)=rs+rt$ and $(r+s)t=rt+st$.\hfill(distributivity)
\end{enumerate}
A \emph{unital ring} is a ring $R$ equipped with a constant $1\in R\setminus\{0\}$ called the \emph{unity} such that
\begin{enumerate}
\item[(ii)] for all $r\in R$ we have $r1=1r=r$,\hfill(identity)
\end{enumerate}
and a \emph{division ring} is a unital ring $R$ such that
\begin{enumerate}
\item[(iii)] for all $r\in R\setminus\{0\}$ there is $r^{-1}\in R$ such that $rr^{-1}=r^{-1}r=1$,\hfill(inverses)
\end{enumerate}
A ring $R$ is called \emph{commutative} if
\begin{enumerate}
\item[(iv)] for all $r,s\in R$ we have $rs=sr$,\hfill(commutativity)
\end{enumerate}
and a \emph{field} is a commutative division ring.
\end{prb}

\section{Ideals}

\begin{prb}[Ideals]
Let $R$ be a commutative unital ring.
\end{prb}

\begin{prb}[Quotient rings]
\end{prb}
\begin{prb}[Isomorphism theorems]
\end{prb}


\section{Maximal and prime ideals}
fields and integral domains
existence by Zorn's lemma

\section{Operations on ideals}

\section*{Exercises}
size of units, the number of ideals








\chapter{Integral domains}
\section{Unique factorization domains}
\section{Principal ideal domains}

\begin{prb}
In a principal ideal domain $R$,
\begin{parts}
\item every irreducible element is prime, \hfill(Euclid's lemma)
\item every two elements has greatest common divisor, \hfill(existence of gcd)
\item the gcd is given as a $R$-linear combination, \hfill(B\'zout's identity)
\item factorization into primes is unique up to permutation, \hfill(UFD)
\item every prime ideal is maximal. \hfill(Krull dimension 1)
\end{parts}
\end{prb}


\section{Noetherian rings}

\section*{Exercises}
\begin{prb}[Primitive roots]
We find all $n$ such that $(\Z/n\Z)^\times$ is cyclic.
\end{prb}

\section*{Problems}
\begin{enumerate}
\item Show that a finite integral domain is a field.
\item Show that every ring of order $p^2$ for a prime $p$ is commutative.
\item Show that a semiring with multiplicative identity and cancellative addtion has commutative addition.
\item Show that the complement of a saturated monoid in a commutative ring is a union of prime ideals.
\end{enumerate}





\chapter{Polynomial rings}
\section{Irreducible polynomials}
relation to maximal ideals
Irreducibles over several fields
\begin{prb}[Gauss lemma]
\end{prb}
\begin{prb}[Eisenstein criterion]
\end{prb}

\section{Polynomial rings over a field}
\begin{prb}[Euclidean algorithm for polynoimals]
\end{prb}
\begin{prb}[Polynomial rings over UFD]
\end{prb}
\begin{prb}[Hilbert's basis theorem]
\end{prb}

maximal ideals and monic irreducibles




\end{document}