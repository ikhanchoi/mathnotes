\documentclass{../../large}
\usepackage{../../ikhanchoi}


\begin{document}
\title{Geometry II}
\author{Ikhan Choi}
\maketitle
\tableofcontents





\part{Smooth surfaces}

\chapter{Smooth manifolds}
\section{Local coordinates}

\begin{prb}
Let $n$ be a positive integer.
A topological space $M$ is called \emph{locally Euclidean} of dimension $n$ if there is an open cover $\{U_\alpha\}$ of $M$ such that each open set $U_\alpha$ is homeomorphic to an open set of $\R^n$.
A \emph{topological manifold} is defined as a paracompact Hausdorff locally Euclidean space.

A \emph{chart} of dimension $n$ on $M$ is a topological embedding $\f:U\to\R^n$ of an open subset $U$ of $M$ onto an open subset $\f(U)$ of $\R^n$.
An \emph{atlas} of dimension $n$ on a topological space is a family $\{\f_\alpha\}$ of charts $\f_\alpha:U_\alpha\to\R^n$ such that $\{U_\alpha\}$ is an open cover of $M$.
By definition, a topological space is locally Euclidean if and only if it admits an atlas.

Two smooth atlases are called \emph{smoothly equivalent} if their addition is a smooth atlas.
\end{prb}

Given a smooth atlas, we can define the \emph{smoothness} of a function $f:M\to\R$ with respect to the smooth atlas as follows: we say that $f$ is smooth if its \emph{coordinate representation} $f\f_\alpha^{-1}:\f_\alpha(U)\to\R$ is smooth for all charts $\f_\alpha$.

\begin{prb}[Immersions and embeddings]

If $\alpha:U\to\R^n$ is a topological embedding, then we can endow with a unique smooth structure on $\im\alpha$ such that $\alpha$ is smooth.(?)
\begin{parts}
\item The image of a regular parameterization is an embedded manifold.
\item Every open subset of a embedded manifold is a embedded manifold.
\item Monge patch.
\item The sphere $S^2=\{\,(x,y,z)\in\R^3:x^2+y^2+z^2=1\,\}$ is a regular surface.
\item The set $\{\,(x,y)\in\R^2:y^2=x^3+x^2\,\}$ is not a regular curve.
\item The set $\{\,(x,y)\in\R^2:y=|x|\,\}$ is not a regular curve.
\end{parts}
\end{prb}


\section{Space curves}

\section{Space surfaces}

%         선형독립    벡터들이 한점에서 주어졌을 때 ->
%         선형독립    벡터장이 근방에서 주어졌을 때 -> 일반적으론 2차원에서만
%         선형독립 가환벡터장이 근방에서 주어졌을 때 -> n차원 다돼
%         선형독립    직교벡터들이 한점에서 주어졌을 때
%         선형독립    직교벡터장이 근방에서 주어졌을 때
%         선형독립 직교가환벡터장이 근방에서 주어졌을 때
%             -> 곡률의 선, 점근곡선, 측지좌표

% preimage theorem

Reparametrizations

\begin{thm}
Let $S$ be a regular surface.
Let $v,w$ be linearly independent tangent vectors in $T_pS$ for a point $p\in S$.
Then, $S$ admits a parametrization $\alpha$ such that $\alpha_x|_p=v$ and $\alpha_y|_p=w$.
\end{thm}
\begin{thm}
Let $X,Y$ be linearly independent tangent vector fields on a regular surface $S$.
Then, $S$ admits a parametrization $\alpha$ such that $\alpha_x|_p$ and $\alpha_y|_p$ are parallel to $X|_p,Y|_p$ respectively for each $p\in S$.
\end{thm}
\begin{thm}
Let $X,Y$ be linearly independent tangent vector fields on a regular surface $S$.
If $\pd_XY=\pd_YX$, then $S$ admits a parametrization $\alpha$ such that $\alpha_x|_p=X|_p$ and $\alpha_y|_p=Y|_p$ for each $p\in S$.
\end{thm}

Let $S$ be a regular surface embedded in $\R^3$.
The inner product on $T_pS$ induced from the standard inner product of $\R^3$ can be represented not only as a matrix
\[\mat{1&0&0\\0&1&0\\0&0&1}\]
in the basis $\{(1,0,0),(0,1,0),(0,0,1)\}\subset\R^3$, but also as a matrix
\[\mat{\<\alpha_x,\alpha_x\>&\<\alpha_x,\alpha_y\>\\\<\alpha_y,\alpha_x\>&\<\alpha_y,\alpha_y\>}\]
in the basis $\{\alpha_x|_p,\alpha_y|_p\}\subset T_pS$.

\begin{defn}
\emph{Metric coefficients}
\begin{alignat*}{2}
\<\alpha_x,\alpha_x\>&=:g_{11}&\qquad
\<\alpha_x,\alpha_y\>&=:g_{12}\\
\<\alpha_y,\alpha_x\>&=:g_{21}&
\<\alpha_y,\alpha_y\>&=:g_{22}
\end{alignat*}
\end{defn}

\begin{thm}[Normal coordinates]
...?
\end{thm}




\subsection*{Differentiation of tangent vectors}

\begin{defn}
Let $\alpha:U\to\R^3$ be a regular surface.
The \emph{Gauss map} or \emph{normal unit vector} $\nu:U\to\R^3$ is a vector field on $\alpha$ defined by:
\[\nu(x,y):=\frac{\alpha_x\times \alpha_y}{\|\alpha_x\times \alpha_y\|}(x,y).\]
The set of vector fields $\{\alpha_x|_p,\alpha_y|_p,\nu|_p\}$ forms a basis of $T_p\R^3$ at each point $p$ on $\alpha$.
The Gauss map is uniquely determined up to sign as $\alpha$ changes.
\end{defn}

\begin{defn}[Gauss formula, $\Gamma_{ij}^k$, $L_{ij}$]
Let $\alpha:U\to\R^3$ be a regular surface.
Define indexed families of smooth functions $\{\Gamma_{ij}^k\}_{i,j,k=1}^2$ and $\{L_{ij}\}_{i,j=1}^2$ by the Gauss formula
\begin{alignat*}{2}
\alpha_{xx}&=:\Gamma_{11}^1\alpha_x+\Gamma_{11}^2\alpha_y+L_{11}\nu,&\qquad
\alpha_{xy}&=:\Gamma_{12}^1\alpha_x+\Gamma_{12}^2\alpha_y+L_{12}\nu,\\
\alpha_{yx}&=:\Gamma_{21}^1\alpha_x+\Gamma_{21}^2\alpha_y+L_{21}\nu,&
\alpha_{yy}&=:\Gamma_{22}^1\alpha_x+\Gamma_{22}^2\alpha_y+L_{22}\nu.
\end{alignat*}
The \emph{Christoffel symbols} refer to eight functions $\{\Gamma_{ij}^k\}_{i,j,k=1}^2$.
The Christoffel symbols and $L_{ij}$ \emph{do depend} on $\alpha$.
\end{defn}
We can easily check the symmetry $\Gamma_{ij}^k=\Gamma_{ji}^k$ and $L_{ij}=L_{ji}$.
Also,
\begin{align*}
\pd_XY
&=X^i\pd_i(Y^j\alpha_j)\\
&=X^i(\pd_iY^k)\alpha_k+X^iY^j\pd_i\alpha_j\\
&=\left(X^i\pd_iY^k+X^iY^j\Gamma_{ij}^k\right)\alpha_k+X^iY^jL_{ij}\nu.
\end{align*}

% Examples


\subsection*{Differentiation of normal vector}

The partial derivative $\pd_X\nu$ is a tangent vector field since
\[\<\pd_X\nu,\nu\>=\frac12\pd_X\<\nu,\nu\>=0.\]
Therefore, we can define the following useful operator.
\begin{defn}
Let $S$ be a regular surface embedded in $\R^3$.
The \emph{shape operator} is $\cS:\fX(S)\to\fX(S)$ defined as
\[\cS(X):=-\pd_X\nu.\]
\end{defn}
\begin{prop}
The shape operator is self-adjoint, i.e. symmetric.
\end{prop}
\begin{pf}
Recall that $\pd_XY-\pd_YX$ is a tangent vector field.
Then,
\[\<X,\cS(Y)\>=\<X,-\pd_Y\nu\>=\<\pd_YX,\nu\>=\<\pd_XY,\nu\>=\<\cS(X),Y\>.\qedhere\]
\end{pf}

% The reason of minus sign in the shape operator.

\begin{thm}
Let $\alpha:U\to\R^3$ be a regular surface and $\cS$ be the shape operator.
Then $\cS$ has the coordinate representation
\[\cS=\mat{g_{11}&g_{12}\\g_{21}&g_{22}}^{-1}\mat{L_{11}&L_{12}\\L_{21}&L_{22}}\]
with respect to the frame $\{\alpha_x,\alpha_y\}$ for tangent spaces.
In other words, if we let $X=X^i\alpha_i$ and $\cS(X)=\cS(X)^j\alpha_j$, then
\[\mat{\cS(X)^1\\\cS(Y)^2}=\mat{g_{11}&g_{12}\\g_{21}&g_{22}}^{-1}\mat{L_{11}&L_{12}\\L_{21}&L_{22}}\mat{X^1\\X^2}.\]
\end{thm}
\begin{pf}
Let $\cS(X)^j=\cS_i^jX_i$.
Then,
\[g_{ik}X^i\cS_j^kY^j=\<X,\cS(Y)\>=\<\pd_XY,\nu\>=X^iY^jL_{ij}\]
implies $g_{ik}\,\cS_j^k=L_{ij}$.
\end{pf}

% principal curvature
% mean curvature, gaussian curvature



% curvature tensor?





















\chapter{Fundamental forms}

\section{Riemannian metrics}

\section{Gaussian curvatures}
Theorema egregium
surfaces of constant gaussian curvature

% 제1기본형식, 크리스토펠: 
% 모양 연산자, 제2기본형식: 바인가르텐 이퀘이션

% 가우스곡률
\begin{defn}
Let $\alpha:U\to\R^3$ be a regular surface.
\begin{gather*}
E:=\<\alpha_x,\alpha_x\>=g_{11},\qquad F:=\<\alpha_x,\alpha_y\>=g_{12},\qquad G:=\<\alpha_y,\alpha_y\>=g_{22},\\
L:=\<\alpha_{xx},\nu\>=L_{11},\qquad M:=\<\alpha_{xy},\nu\>=L_{12},\qquad N:=\<\alpha_{yy},\nu\>=L_{22}.
\end{gather*}
\end{defn}


\begin{cor}
We have $GM-FN=EM-FL$, and the \emph{Weingarten equations}:
\begin{align*}
\nu_x&=\frac{FM-GL}{EG-F^2}\alpha_x+\frac{FL-EM}{EG-F^2}\alpha_y,\\
\nu_y&=\frac{FN-GM}{EG-F^2}\alpha_x+\frac{FM-EN}{EG-F^2}\alpha_y.
\end{align*}
\end{cor}



\begin{thm}
\[\Gamma_{ij}^l=\frac12g^{kl}(g_{ik,j}-g_{ij,k}+g_{kj,i}).\]
\end{thm}

\[\frac12(\log g)_x=\Gamma_{11}^1.\]

\[\nu_x\times\nu_y=K\sqrt{\det g}\ \nu.\]
\[\alpha_x\times\alpha_y=\sqrt{\det g}\ \nu\]
\[\<\nu_x\times\nu_y,\alpha_x\times\alpha_y\>=\det\mat{\<\nu_x,\alpha_x\>&\<\nu_x,\alpha_y\>\\\<\nu_y,\alpha_x\>&\<\nu_y,\alpha_y\>}=\det\mat{-L&-M\\-M&-N}=K\det g\]











\begin{prb}[Gaussian curvature formula]
\begin{parts}
\item
In general,
\[K=\frac{LN-M^2}{EG-F^2}.\]
\item
For orthogonal coordinates such that $F\equiv0$,
\[K=-\frac1{2\sqrt{\det g}}\left((\frac1{\sqrt{\det g}}E_y)_y+(\frac1{\sqrt{\det g}}G_x)_x\right).\]
\item
For $f(x,y,z)=0$,
\[K=-\frac1{|\nabla f|^4}\mat[v]{0&\nabla f\\\nabla f^T&\Hess(f)},\]
where $\nabla f$ denotes the gradient $\nabla f=(f_x,f_y,f_z)$.
\item(Beltrami-Enneper) If $\tau$ is the torsion of an asymptotic curve, then
\[K=-\tau^2.\]
\item(Brioschi) $E,F,G$ describes $K$.
\end{parts}
\end{prb}

\begin{pf}
(a) Clear.

(b)
We have $GM=EM$ and
\[\nu_x=-\frac LE\alpha_x-\frac MG\alpha_y,\qquad\nu_y=-\frac ME\alpha_x-\frac NG\alpha_y.\]
\[\nu_x\times\nu_y=\frac{LN-M^2}{EG}\alpha_x\times\alpha_y\]
After curvature tensors...

\end{pf}



\begin{prb}[Computation of Gaussian curvatures]
\begin{parts}
\item
(Monge's patch)
For $(x,y,f(x,y))$,
\[K=\frac{f_{xx}f_{yy}-f_{xy}^2}{(1+f_x^2+f_y^2)^2}.\]

\item
(Surface of revolution).
Let $\gamma(t)=(r(t),z(t))$ be a plane curve with $r(t)>0$.
If $t\mapsto(r(t),z(t))$ is a unit-speed curve, then
\[K=-\frac{r''}r.\]

\item
(Models of hyperbolic planes)
\end{parts}
\end{prb}
\begin{pf}
(b)
Let
\[\alpha(\theta,t)=(r(t)\cos\theta,r(t)\sin\theta,z(t))\]
be a parametrization of a surface of revolution.
Then,
\begin{align*}
\alpha_\theta&=(-r(t)\sin\theta,r(t)\cos\theta,0)\\
\alpha_t&=(r'(t)\cos\theta,r'(t)\sin\theta,z'(t))\\
\nu&=\frac1{\sqrt{r'(t)^2+z'(t)^2}}(z'(t)\cos\theta,z'(t)\sin\theta,-r'(t)),
\end{align*}
and
\begin{align*}
\alpha_{\theta\theta}&=(-r(t)\cos\theta,-r(t)\sin\theta,0)\\
\alpha_{\theta t}&=(-r'(t)\sin\theta,-r'(t)\cos\theta,0)\\
\alpha_{tt}&=(r''(t)\cos\theta,r''(t)\sin\theta,z''(t)).
\end{align*}
Thus we have
\[E=r(t)^2,\quad F=0,\quad G=r'(t)^2+z'(t)^2,\]
and
\[L=-\frac{r(t)z'(t)}{\sqrt{r'(t)^2+z'(t)^2}},\quad M=0,\quad N=\frac{r''(t)z'(t)-r'(t)z''(t)}{\sqrt{r'(t)^2+z'(t)^2}}.\]
Therefore,
\[K=\frac{LN-M^2}{EG-F^2}=\frac{z'(r'z''-r''z')}{r(r'^2+z'^2)^2}.\]
In particular, if $t\mapsto(r(t),z(t))$ is a unit-speed curve, then
\[K=-\frac{r''}r.\]
\end{pf}


% asymptotic curve -> hyperbolic
% line of curvature -> non-umbilic

% minimal surface
% 	회전곡면
% asymptotic curve
% 	Beltrami-Enneper
% ruled surface
% developable surface


\begin{prb}[Local isomorphism]
Surfaces of the same constant Gaussian curvature are locally isomorphic.
\end{prb}
\begin{pf}
Let
\[\mat{\|\alpha_r\|^2&\<\alpha_r,\alpha_t\>\\\<\alpha_t,\alpha_r\>&\|\alpha_t\|^2}=\mat{1&0\\0&h(r,t)^2}\]
be the first fundamental form for a geodesic coordinate chart along a geodesic curve so that $\alpha_{tt}$ and $\alpha_{rr}$ are normal to the surface.
Then,
\[K=-\frac{h_{rr}}h\]
is constant.
Also, since
\[\frac12(h^2)_r+\<\alpha_r,\alpha_{tt}\>=\<\alpha_{rt},\alpha_t\>+\<\alpha_r,\alpha_{tt}\>=\<\alpha_r,\alpha_t\>_t=0\]
implies $h_r=0$ at $r=0$, the function $f:r\mapsto h(r,t)$ satisfies the following initial value problem
\[f_{rr}=-Kf,\quad f(0)=1,\quad f'(0)=0.\]
Therefore, $h$ is uniquely determined by $K$.
\end{pf}


% Global theory of curves
%  Isoperimetric inequality
%  Four vertex theorem
%  Ovals

% Global theory of surfaces
%  Minimal surfaces

% Total curvatures
%  Fary-Milnor
%  Fenchel
%  Gauss-Bonnet, euler characteristic




\chapter{Geometric surfaces}

\section{}

\begin{prb}[Geometric manifolds]
We are concerned only with metric geometries.
A \emph{geometric manifold} is a smooth manifold together with a metric that is
\begin{enumerate}[(i)]
\item geodesically connected,
\item geodesically complete,
\item realized by a Riemannian metric,
\item locally homogeneous, i.e.~every pair of points is connected by an isometry between neigborhoods.
\end{enumerate}
Each condition has been obtained by modifying the first four postulates of Euclid's Elements.
On a smooth manifold, we define a \emph{geometric structure} as a locally homogeneous Riemannian metric, which embodies the third and fourth postulates.
The completeness is sometimes assumed.
A geometric manifold is a connected smooth manifold together with a complete geometric structure.
\begin{parts}
\item On a connected smooth surface, a geometric structure is same as a Riemannian metric of constant Gaussian curvature.
\item The isotropy group of a manifold with a geometric structure is a compact Lie group uniquely determined up to isomorphism.

\item On a connected smooth manifold, there is a unique complete geometric structure up to...?
\end{parts}
\end{prb}


\begin{prb}
A \emph{geometry} is a smooth $G$-manifold $X$ such that
\begin{enumerate}
\item $X$ is connected and simply connected,
\item $G$ acts on $X$ effectively and transitively,
\item the isotropy group is compact,
\item there exists a cocompactly freely properly discretely acting subgroup of $G$.
\end{enumerate}
A \emph{maximal geometry} or a \emph{model geometry} is a geometry such that there is no properly large group $G'$ satisfying the above assumptions.
We want to classify model geometries up to intertwining maps, or equivalently as subgroups of $\Diff(X)$ up to conjugacy.


\begin{parts}
\item
\item A surface is geometric if and only if it is universally covered by a simply connected geometric surface, which is one of $\S^2$, $\E^2$, and $\H^2$.
\item A complete geometric structure lifts to a complete geometric structure on the universal covering, and a complete geometric structure on a simply connected manifold is homogeneous.
\end{parts}
\end{prb}
\begin{pf}
Let $M$ be a connected smooth manifold, and let $X$ be the universal covering.
A complete locally homogeneous Riemannian metric on $M$ naturally induces a complete homogeneous Riemannian metric on $X$ such that the covering map is a local isometry.
Let $G$ be the group of isometries on $X$.
Is the geometry $(X,G)$ maximal?
For another complete locally homogeneous Riemannian metric on $M$, we have $(X,G')$.
In which sense $(X,G)$ and $(X,G')$ are equivalent?


Let $(X,G)$ be a model geometry.
If $\dim X=n$, then the isotropy subgroup $G_x$ is compact, we can show that there is a $G_x$-invariant inner-product on $T_xX$ by using the Haar integral.
Thus, $G_x$ is naturally embedded in $\GL(T_xX)$.
If $(X,G')$ is a geometry such that $G\subset G'$, then $G'_x$ is also embedded in $\GL(T_xX)$.
Does the maximality imply $G$ is isomorphic to $\rO(n)$?



\end{pf}



\begin{prb}[Properly discontinuous actions]

A proper action of a discrete group on a locally compact Hausdorff space is said to be \emph{properly discontinuous}.

On locally compact spaces: properly discontinuous action iff orbit counted with multiplicity is locally finite iff orbit is discrete and stabilizer is finite

Fuchsian groups act on H2 properly discontinuously (and faithfully) (and PSL(2,R) has compact stabilizer so that it has finite stabilizer), so the quotient is an orbifold.

Fuchsian group acts on H2 freely iff it is torsion-free because an element of PSL(2,R) has finite order iff it has a fixed point (properly discontinuous action is free iff the stabilizer is trivial at every point iff the quotient is a manifold.)

torsion-free Fuchsian groups conjugate iff quotient space form is isometric?
\end{prb}

(convex, locally finite) fundamental domains
poincare polygon

\begin{prb}
Liebmann theorem
Hilbert problem
\end{prb}





\part{Riemann surfaces}


\chapter{Uniformization}

\begin{prb}[Riemann surfaces]


A \emph{Riemann surface} is a connected two-dimensional complex manifold.
\end{prb}

Recall that a \emph{meromorphic function} on $X$ is a holomorphic function $f:U\to\C$ on an open subset $U\subset X$ such that $X\setminus U$ is discrete closed in $X$ and the Laurent expansion at each point of $X\setminus U$ has the finite principal part.
The set of meromorphpic functions $\cM(X)$ is a field extension over the complex field $\C$ by the removable singularity theorem.



On a surface, complex structures, conformal structures, geomtric structures of curvature in $\{-1,0,1\}$ are all equivalent.
The equivalence between the last two is usually called the uniformization.





\begin{itemize}
\item $g=0$: Riemann sphere (spherical) $\to$ Riemann sphere itself
\item $g=1$: complex plane (Euclidean) $\to$ elliptic curves
\item $g\ge2$: open unit disk (hyperbolic) $\to$ hyperbolic surfaces, classified by Fuchsian groups(with which properties?)
\end{itemize}

\begin{prb}
holomorphic actions of $\PSL(2,\C)$ and $\SO(3)$ on the Riemann sphere $\CP^1$.

properly discontinuous actions of $\PSL(2,\Z)$ on the complex plane $\C$.
\end{prb}

Let $p:Y\to X$ be a non-constant holomorphic map.
A point $y\in Y$ is called a \emph{branch point} or a \emph{ramification point} over $x\in X$ if $p(y)=x$ and $p$ is not injective on any neighborhoods $V$ of $y$.

For a non-constant holomorphic map, it is unbranched if and only if it is locally homeomorphic.

For a proper non-constant holomorphic map, a branch point is isolated in $X$ (closed and discrete), and the multiplicity is finitely well-defined.

\begin{prb}
Let $p:Y\to X$ be a local homeomorphism from a topological space $Y$ to a locally complex space $X$.
Then, $Y$ admits a unique complex structure such that $p$ is holomorphic.
\end{prb}
\begin{pf}
Consider the set of all open embeddings $\psi:V\to\C$ that factor through $p$ and a chart on $X$.
It is a complex atlas since $\psi_\alpha\psi_\beta^{-1}=\f_\alpha p(\f_\beta p)^{-1}=\f_\alpha\f_\beta^{-1}$ is biholomorphic
The domains $V$ form an open cover since $p$ is a local homeomorphism.
Therefore, it defines a complex atlas.
The uniqueness follows from the identity theorem.
\end{pf}


For a holomorphic function germ in $\cO_x$, there is a maximal analytic continuation.
An analytic continuation of a holomorphic function germ in $\cO_x$ is a triple of an unbranched non-constant holomorphic map $p:Y\to X$, a holomorphic function $f:Y\to\C$, and a point $y\in Y$ such that $p_*f_y$ is the germ.


$dz$ and $d\bar z$ forms a basis of a two-dimensional complex vector space $T_x^*X\otimes_\R\C=\Omega^1_x(X,\C)$.

Laurent expansion of a holomorphic 1-form on $U$ at an isolated singularity.
A \emph{meromorphic 1-form} is defined by the finiteness of the negative part of the Laurent expansion.


\chapter{Riemann-Roch theorem}

The ultimate goal is to prove a compact Riemann surface is algebraic.

cpt Riemann surfaces, existence of meromorphic, Chow, projective embedding, GAGA, coherence, sheaf cohomology

Not using sheaf cohomology and doing everything in line bundles?
How can we define line bundle over an algebraic curve?
How can we describe $\cO^\times$, $\cM^\times$ without sheaves?

How can we translate the following fact in terms of line bundles:
sheaf of holomorphic/meromorphic functions and regular/rational functions: the category of coherent sheaves are equivalent!


\section{\v Cech cohomology}

On a compact Riemann surface $X$,
\begin{parts}
\item $\underline\Z$, $\underline\C$ are constant sheaves.
\item $\cO$, $\cO(D)$, $\cK$ are invertible sheaves over $\underline\C$.
\item $\cM$ is a locally free sheaf over $\underline\C$.
\item $\underline\Z$, $\underline\C$, $\cO^\times$, $\cM^\times$ are sheaves over $\underline\Z$.
\item $\underline\C$ is not locally free over $\underline\C$.
\item $\cO=\Omega^0$ and $\cK=\Omega^1$ is the space of holomorphic functions and holomorphic 1-forms.
\end{parts}

\begin{prb}[Leray cover]
\end{prb}

\begin{prb}[Dolbeault theorem]
\end{prb}

\section{Divisors and line bundles}


Let $X$ be a compact Riemann surface.
The Cartier divisor group is just the global section space $\mathrm{Div}(X):=H^0(X,\cM^\times/\cO^\times)$ of the sheaf $\cM^\times/\cO^\times$ of abelian groups.
The Picard group, defined by the group of line bundles, is just the cohomology group $\mathrm{Pic}(X):=H^1(X,\cO^\times)$.
We have an exact sequence of abelian groups
\[H^0(X,\cM^\times)\to\mathrm{Div}(X)\to\mathrm{Pic}(X)\to H^1(X,\cM^\times).\]
We will prove $H^1(X,\cM^\times)=0$ be showing the surjectivity of $\mathrm{Div}(X)\to\mathrm{Pic}(X)$.

If we use the exponential exact sequence $0\to\underline\Z\to\cO\to\cO^\times\to0$, we can define the first Chern class $\mathrm{Pic}(X)\to H^2(X,\Z)$.




On a scheme, a line bundle is usually defined as the locally free $\cO$-module of rank one.
On nice schemes, a sheaf would be invertible if and only if it is a line bundle.
But more precisely, they are different objects.
For a locally free sheaf $\cF$ of finite rank, then the total space can be defined as $\Spec(\Sym\cF^\vee)\to X$.
(Recall that $\A_k^r=\Spec k[x_1,\cdots,x_r]=\Spec\Sym\<x_1,\cdots,x_r\>_k$, where $\<x_1,\cdots,x_r\>_k$ denotes the $k$-linear span)

Let $\pi:L\to X$ be a line bundle on a compact Riemann surface $X$.
It means that $\pi$ is a holomorphic map and there is an family $\{f_\alpha\}$ of biholomorphic maps $\f_\alpha:\pi^{-1}(U_\alpha)\to U_\alpha\times\C$.




\begin{prb}[Divisors]
Let $X$ be a compact Riemann surface.
A \emph{Weil divisor} $D$ on $X$ is an element of the free abelian group generated by points of $X$.
A \emph{Cartier divisor} is a family $\{f_\alpha\}$ of non-zero meromorphic functions $f_\alpha\in\cM^\times(U_\alpha)$ indexed by an open cover $\{U_\alpha\}$ of $X$ such that $f_\alpha/f_\beta$ is extended to a holomorphic function on $U_\alpha\cap U_\beta$.




By compactness of $X$, a non-zero meromorphic function $f\in\cM^\times(X)$ gives rise to a Weil divisor $(f):=\sum_{P\in X}\ord_P(f)P$, called the \emph{principal divisor} of $f$.

Let $D=\sum_i n_iP_i$ be a Weil divisor on $X$.
Each point $P\in X$ has a meromorphic function $f$ on an open neighborhood $U$ of $P$ such that $(f)=D$ on $U$.
It implies that there is a collection $\{f_\alpha\}$ of meromorphic functions $f_\alpha$ defined on $U_\alpha$, where $\{U_\alpha\}$ is an open cover of $X$, such that $f_\alpha/f_\beta$ is a well-defined holomorphic functions on $U_\alpha\cap U_\beta$.
In other words, a Cartier divisor is assigned to each Weil divisor.


A Cartier divisor defines a line bundle.

Here is a direct way to construct a line bundle from Weil divisors.
Note that a field $\cM(X)=\cM^\times(X)\cup\{0\}$ of meromorphic functions contains the field of constant functions $\C$.
For a Weil divisor $D$ on $X$, define
\[L(D):=\{f\in\cM^\times(X):(f)+D\ge0\}\cup\{0\}.\]
It is a complex vector subspace of $\cM(X)$.
We can construct a line bundle $\cO(D)$ such that $L(D)=H^0(X,\cO(D))$.

\end{prb}




\begin{prb}
Given $\{P_i\}_{i=1}^n$ points and $\{f_i\}_{i=1}^n$ principal parts, there is a meromorphic function $f$ with pre-described principal parts if and only if for every holomorphic 1-form $\omega$ we have $\sum_{i=1}^n\Res(f_i\omega,P_i)=0$.
\end{prb}

\begin{prb}
\[l(D)-l(K-D)=\deg(D)+1-g.\]
The genus can be defined by $g=h^0(X,\Omega^1)$.
For algebraic curves, it can be proved as follows:
Assuming the Serre duality, we have $\chi(D)=h^0(D)-h^1(D)=l(D)-l(K-D)$ and $\chi(0)=h^0(0)-h^1(0)=1-g$.
Then, the Riemann-Roch is boiled down to $\chi(D)=\deg(D)+\chi(0)$, which can be shown inductively.

However, we want to prove a compact Riemann surface is projective as an application of the Riemann-Roch theorem, we need to prove the Riemann-Roch theorem without theory of algebraic curves.
\begin{parts}
\item If $\deg D<0$, then $l(D)=0$.
\end{parts}
\end{prb}
\begin{pf}
(a)
Let $f\in L(D)\setminus\{0\}$.
Then, $(f)+D\ge0$ and $\deg(f)=0$ imply $\deg D\ge0$, which is a contradiction.

(b)
Let $D=0$.
Then, it follows from $l(K)=g$ and $l(0)=1$.

Let $D>0$.
We may assume $D=\sum_{i=1}^nn_iP_i$ with $n_i>0$. (why?)
Let
\[V_i:=\Bigl\{\sum_{k=-n_i}^{-1}c_k(z-P_i)^k:c_k\in\C\Bigr\}\]
and $V:=\bigoplus_{i=1}^nV_i$. (how can we define the principal part of $f$ on Riemann surface?)
Then, $\dim V=\deg D$.
Define $L(D)\to V$ by principal part at each point $p_i$.


\end{pf}

\begin{prb}[Embedding theorem]
Let $X$ be a compact Riemann surface.

Let $L$ be a line bundle over $X$, and let $(s_i)_{i=0}^n$ be a family of sections of $L$ such that every point $P$ of $X$ has a section $s_i$ which does not vanish at $P$.

The \emph{complete linear system} of a Weil divisor $D$ on $X$ is the set
\[|D|:=\{(f)+D:f\in\cO(X)\}.\]
Then, by the map $L(D)\setminus\{0\}\to|D|:f\mapsto...$
the set $|D|$ can be identified with the projective space $(L(D)\setminus\{0\})/\C^\times=\CP^{l(D)-1}$.
Let $(f_i)_{i=0}^{l(D)-1}$ be an ordered basis of $L(D)$.

For a linear system $\Delta$ of projective dimension $n-1$, we can take (how?) a basis $(e_i)_{i=0}^{n-1}$ such that the following map is well-defined:
\[X\setminus\mathrm{Bl}(\Delta)\to\CP^{n-1}:p\mapsto(e_0:\cdots:e_{n-1}).\]
\end{prb}






\section{Serre duality}

Mittag-Leffler


0→O(-D)→O→O(D)→0  for effective D

adjunction formula? hyperplane section?

\section{Hodge decomposition}


\begin{prb}[Finiteness of genus]
Let $X$ be a compact Riemann surface.
\[0\to\underline\C\to\cO\xrightarrow{d}\cK\to0.\]

\[\begin{array}{c|ccc}
h^-(X,-) & \C & \cO & \cK \\\hline
0 & 1 & 1 & g \\
1 & 2g & g & 1 \\
2 & 1 & 0 & \,
\end{array}\]

We start from
\begin{itemize}
\item $h^0(X,\C)=1$ by connectedness.
\item $h^0(X,\cO)=1$ by the Liouville theorem.
\end{itemize}
By the Serre duality,
\begin{itemize}
\item $h^0(X,\cK)=h^1(X,\cO)$.
\item $h^1(X,\cK)=h^0(X,\cO)=1$.
\end{itemize}
Then,
\begin{itemize}
\item $h^1(X,\C)=2g$ by the rank-nullity theorem from $h^1(X,\cO)=g$ and $h^0(X,\cK)=g$.
\end{itemize}
Also,
\begin{itemize}
\item $h^2(X,\C)=1$ by the Poincar\'e theorem.
\item $h^2(X,\cO)=0$ by the Dolbeault theorem?
\end{itemize}



The \emph{Euler characteristic} is defined by
\[\chi:=h^0(X,\C)-h^1(X,\C)+h^2(X,\C),\]
and the \emph{genus} can be defined by the formula
\[\chi=:2-2g.\]

Since $h^0(X,\C)=1$, $h^2(X,\C)=1$, by the rank-nullity theorem we have
\[g=h^1(X,\cO).\]


Let $D\subset\C$ be an open set.
Then, $U$ has the canonical volume form $dx\wedge dy$.
Let $L^2(D,\cO)$ be the completion of sqaure-integrable holomorphic functions on $D$ with respect to the $L^2$-norm.
We have $L^2(D,\cO)\subset L^2(D)$.
\begin{parts}
\item $L^2(D,\cO)\subset\Gamma(D,\cO)=\cO(D)$.
\item If $D'\Subset D$, then for every $\e>0$ there is a finite-codimensional linear subspace $A\subset L^2(D,\cO)$ such that
\[\|f\|_{L^2(D')}\le\e\|f\|_{L^2(D)},\qquad f\in A.\]
\end{parts}
\end{prb}
\begin{pf}
(a)
For $r>0$, let $U_r:=\{z\in U:B(z,r)\subset U\}$.
It suffices to show
\[\|f\|_{L^\infty(U_r)}\le\frac1{\sqrt\pi r}\|f\|_{L^2(U)},\qquad f\in\cO(U).\]

(b)

Take $r>0$ and a finite subset $\{a_j\}$ of $D'$ such that $D'\subset\bigcup_jB(a_j,\frac r2)$ and $\bigcup_jB(a_j,r)\subset D$.
Take $n$ such that $2^{-n-1}|\{a_j\}|^{\frac12}<\e$.
Let
\[A:=\{f\in L^2(D,\cO):f^{(k)}(a_j)=0\text{ for all $k\le n$ and $j$}\}.\]
If $f\in A$, then at $a_j$ we have the power series expansion
\[f(z)=\sum_{k=n}^\infty c_k(z-a_j)^k,\qquad z\in B(a_j,r).\]
Thus
\begin{align*}
\|f\|_{L^2(D')}^2&\le\sum_j\|f\|_{L^2(B(a_j,\frac r2))}^2\\
&=\sum_j\sum_{k=n}^\infty\frac{\pi(\frac r2)^{2n+2}}{k+1}|c_k|^2\\
&=2^{-2n-2}\sum_j\sum_{k=n}^\infty\frac{\pi r^{2n+2}}{k+1}|c_k|^2\\
&=2^{-2n-2}\sum_j\|f\|_{L^2(B(a_j,r))}^2\\
&\le2^{-2n-2}|\{a_j\}|\|f\|_{L^2(D)}^2,\qquad f\in A,
\end{align*}
so the desired inequality follows.

(c)
14.6?

(d)
\[L:=\{(\xi,\eta,\zeta)\in Z_{L^2}^1(\cU,\cO)\oplus Z_{L^2}^1(\cV,\cO)\oplus C_{L^2}^0(\cW,\cO):\eta=\xi+\delta\zeta\text{ on }\cW\}.\]

Since the projection$L\to Z_{L^2}^1(\cV,\cO)$ is a surjective bounded linear operator, by the Bartle-Graves theorem, there is a non-necessary linear bounded map $Z_{L^2}^1(\cV,\cO)\to L$ which is a right inverse.


\end{pf}

A method using Hodge decomposition may be more natural?







\chapter{Algebraic curves}

\section{Projective varieties}

multiplicities, Bezout theorem

\section{Chow theorem}
divisors, line bundles
euler characteristic
(tangent line bundle degree 2-2g, canonical line bundle 2g-2)
$L(D):=\Gamma(X,\cO(D))=H^0(X,\cO(D))$

Jacobian variety (moduli spaces....)


\begin{prb}[Chow theorem]
A complex submanifold of a projective space is algebraic.
\end{prb}


\section{Moduli spaces}





\end{document}