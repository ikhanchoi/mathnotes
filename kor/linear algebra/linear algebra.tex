\documentclass{../../large}
\usepackage[kor]{../../ikhanchoi}



\begin{document}
\title{선형대수학}
\author{최익한}
\maketitle
\tableofcontents





\part{벡터공간}

\chapter{선형변환}

\section{부분공간}



벡터공간의 정의

부분공간의 확인

교집합과 직합


\begin{example}
\end{example}
\begin{proof}[풀이]

\end{proof}


\section{기저}
선형독립과 선형종속, 기저의 정의.

차원이 잘 정의되는가.

기저를 잡는다는 것의 의미. 그리고 좌표.


\section{행렬}

선형변환의 정의.

행렬표현.

좌표변환.




\section{열공간과 영공간}
열공간, 영공간

단사, 전사, 동형.

rank-nullity

세 가지 행렬연산의 의미.

Gauss 소거법, LU 분해




\begin{example}
$m$과 $n$을 양의 정수라 하고 $r$을 $m$ 이하의 음이 아닌 정수라 할 때, 계수가 $r$인 선형변환 $T:\R^n\to\R^m$들에 의해 생성되는 벡터공간의 차원을 구하여라.
\end{example}








\chapter{고윳값}

\section{정사각행렬}

역행렬


이차정사각행렬에 대한 예제

\section{대각합과 행렬식}

대각합의 묘사.

행렬식의 묘사.


이차정사각행렬에 대한 예제


\section{특성다항식}

특성다항식, 최소다항식, 대수적 중복도


스펙트럼사상정리


이차정사각행렬에 대한 예제

\[\mat{1&0\\0&-1},\quad\mat{0&-1\\1&0},\quad\mat{1&0\\0&1},\quad\mat{1&1\\0&1}\]


\section{복소벡터공간}

고윳값의 정의는 고유벡터가 존재하는 것이기 때문에 실벡터공간이냐 복소벡터공간이냐에 따라 달라진다.












\chapter{고유공간}



\section{순환분해}
Smith 표준형
Frobenius 표준형




\section{대각화}

Jordan 표준형

기하적 중복도
켤레류
$F[x]$-가군과 닮음


\section{교환자}

동시대각화

$AB=BA$라는 것은 $[A,B]=0$, $A$가 $B$의 고유공간에 작용한다는 것.


\section{}














\part{내적공간}



\chapter{수반행렬}

\section{반쌍선형 형식}

비퇴화성

각도 개념을 정의할 수 있다.

정규직교기저를 생각할 수 있게 된다.

피타고라스 정리를 정의한다고 생각할 수도 있다.

쌍대공간과의 동형



\section{정규행렬}

정규, 대칭, 직교


특이값분해
극분해

\section{사영행렬}

사영, 등장


Gram-Schmidt
QR분해
최소제곱법

\section{스펙트럼 정리}









\chapter{정부호행렬}

\section{쌍선형 형식}
쌍선형/이차형식, 실베스터,

\section{}







\chapter{행렬해석학}

\section{행렬 노름}

\section{행렬 지수함수}

\section{확률행렬과 인접행렬}

Markov사슬, Peron-Frobenius

합성, 스펙트럼 그래프

\section{수치선형대수학}
Toeplitz행렬, 순환행렬, 빠른 Fourier변환


\end{document}