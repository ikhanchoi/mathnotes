\documentclass{../../large}
\usepackage{../../ikhanchoi}


\begin{document}
\title{Analysis I}
\author{Ikhan Choi}
\maketitle



% \chapter*{Preface}
% the main objectives
% the audience
% the structure of the book
% how to use this book
% acknowledgements
% references


\tableofcontents


\part{Limits}
\chapter{Real numbers}

\section{Ordered fields}

\begin{prb}[Fields]
Roughly, a field is a set where the four fundamental arithmetic operations are justified.
Mathematically, a \emph{field} is a set $F$ equipped with two binary operations $+:F\times F\to F$ and $\cdot:F\times F\to F$, and two constants $0\in F$ and $1\in F\setminus\{0\}$, called the \emph{addition}, \emph{multiplication}, \emph{zero}, and \emph{one} respectively, such that the addition and zero satisfy
\begin{enumerate}[(i)]
\item for all $a,b,c\in F$, we have $(a+b)+c=a+(b+c)$,\hfill(associativity)
\item for all $a\in F$, we have $a+0=0+a=a$,\hfill(identity)
\item for all $a\in F$, there is $-a\in F$ such that $a+(-a)=(-a)+a=0$,\hfill(inverses)
\item for all $a,b\in F$, we have $a+b=b+a$,\hfill(commutativity)
\end{enumerate}
and the multiplication and one satisfy
\begin{enumerate}[(i)]
\setcounter{enumi}{4}
\item for all $a,b,c\in F$, we have $(a\cdot b)\cdot c=a\cdot(b\cdot c)$,\hfill(associativity)
\item for all $a\in F$, we have $a\cdot 1=1\cdot a=a$,\hfill(identity)
\item for all $a\in F\setminus\{0\}$, there is $a^{-1}\in F$ such that $a\cdot a^{-1}=a^{-1}\cdot a=1$,\hfill(inverses) 
\item for all $a,b\in F$, we have $a\cdot b=b\cdot a$,\hfill(commutativity)
\end{enumerate}
and finally the following compatibility condition for the two binary operations is satisfied:
\begin{enumerate}[(i)]
\setcounter{enumi}{8}
\item for all $a,b,c\in F$, we have $(a+b)\cdot c=a\cdot c+b\cdot c$.\hfill(distributivity)
\end{enumerate}
Let $F$ be a field and let $a,b\in F$ and $u\in F\setminus\{0\}$.
We use the usual notations for $n\in\Z_{\ge0}$
\[a-b:=a+(-b),\quad ab:=a\cdot b,\quad a/u:=au^{-1},\quad na:=a+\cdots+a,\quad a^n:=a\cdots a,\quad u^{-n}:=(u^{-1})^n.\]
\begin{parts}
\item The identities are uniquely determined by the addition and the multiplication.
\item The inverses are also uniquely determined by the addition and the multiplication.
\item $-(-u)=u$ and $(u^{-1})^{-1}=u$.
\item We have $a0=0$, $a(-b)=-(ab)$, and $(-a)(-b)=ab$.
\end{parts}
\end{prb}
\begin{pf}
(a)
Before beginning the proof, let us make the statement we want to show more clear.
What we want to prove is that the zero and one are redundant in the definition of a field, in the sense that they do not have to be given as specified data of a field.
More precisely, the statement that identities, zero and one, are unique means that there are at most one zero and one one, where a zero is defined as an element $0\in F$ such that $a+0=0+a=a$ for all $a\in F$, and a one is defined as an element $1\in F$ such that $a\cdot 1=1\cdot a=a$ for all $a\in F$.
By showing their uniqueness, we can equivalently define a field as a set $F$ equipped just with two binary operations $+$ and $\cdot$ satisfying (i)\sim(ix), but the four axioms (ii), (iii), (vi), (vii) are replaced into
\begin{enumerate}
\item[(ii')] there is a zero,
\item[(iii')] for all $a\in F$, there is $-a\in F$ such that $a+(-a)=(-a)+a$ is a zero,
\item[(vi')] there is a one,
\item[(vii')] for all $a\in F\setminus\{0\}$, there is $a^{-1}\in F$ such that $a\cdot a^{-1}=a^{-1}\cdot a$ is a one.
\end{enumerate}
The original definition specifies the zero and the one, but after we prove the uniqueness, it is enough to be given an axiom which guarantees their existences, (ii') and (vi').
From this definition without zero and one, in order to specify the zero and one, since we already know that there exist a zero and a one by the axiom (ii') and (iii'), we are enough to prove that there should be at most a single zero and a single one in a field.
We first show the zero is uniquely determined by the addition.

The proof is short.
Suppose $0,0'\in F$ are zeros so that they simultaneously satisfy the second axiom $a+0=0+a=a$ and $a+0'=0'+a=a$ for all $a\in F$.
Then, $0=0+0'=0'$, and it means that the zero is unique.
Similarly, $1,1'\in F\setminus\{0\}$ are ones so that they satisfy $a\cdot 1=1\cdot a=a$ and $a\cdot 1'=1'\cdot a=a$ for all $a\in F$.
Then, $1=1\cdot 1'=1'$, so the one is unique.



(b)
Next, we will prove that there are at most a single additive or multiplicative inverse for each given element $a\in F$ or $a\in F\setminus\{0\}$, respectively.


We have the cancellation law that $a+c=b+c$ implies $a=a+0=a+(c-c)=(a+c)-c=(b+c)-c=b+(c-c)=b+0=b$ for addition, and similarly for multiplication.
Then, $(-c)+c=0=-c'+c$ implies $-c=-c'$ and $c^{-1}c=1=c'^{-1}c$ implies $c^{-1}=c'^{-1}$.


(c)
Since $c-c=0=-(-c)-c$ by the definition of inverses, the cancellation of $-c$ implies $c=-(-c)$.
Similarly for multiplication.

(d)
First, $a0=a(0+0)=a0+a0$ implies $a0=0$ by cancellation.
The others follow from $a(-b)+ab=a(-b+b)=a0=0$ and $(-a)(-b)=-(a(-b))=-(-(ab))=ab$.
\end{pf}

\begin{prb}[Total orders]
A \emph{totally ordered set} is a set equipped with a binary relation $\le$ on $F$ called a \emph{total order} or a \emph{linear order}, such that
\begin{enumerate}[(i)]
\item for all $a\in F$, we have $a\le a$,\hfill(reflexivity)
\item for all $a,b,c\in F$, if $a\le b$ and $b\le c$, then $a\le c$,\hfill(transitivity)
\item for all $a,b\in F$, if $a\le b$ and $b\le a$, then $a=b$,\hfill(anti-symmetry)
\item for all $a,b\in F$, we have $a\le b$ or $b\le a$.\hfill(totality)
\end{enumerate}
For $a,b$ in a totally ordered set, we write $a<b$ if $a\le b$ and $a\ne b$.

Let $S$ be a subset of a totally ordered set $F$.
An element $b\in F$ is called an \emph{upper bound} of $S$ if $a\le b$ for all $a\in S$, and if there is an upper bound of $S$, then we say $S$ is \emph{bounded above}.
The \emph{maximum} of $S$ is an upper bound contained in $S$.
The \emph{least upper bound} or the \emph{supremum} of $S$ is a smallest upper bound $s\in F$ in the sense that $s$ is an upper bound of $S$ and $s\le b$ if $b$ is any upper bound of $S$.
Symmetrically, we can also define \emph{lower bound}, \emph{boundedness below}, and the \emph{minimum}, \emph{greatest lower bound} or \emph{infimum}, of $S$.
If $S$ is bounded above and below, then it is called \emph{bounded}.
\begin{parts}
\item A supremum is unique if it exists.
\end{parts}
\end{prb}
\begin{pf}

\end{pf}

\begin{prb}[Ordered fields]
A \emph{totally ordered field}, or simply an \emph{ordered field} is a field $F$ together with a total order $\le$ on $F$ such that we have the compatibility conditions of the field structure and the order structure in the sense that
\begin{enumerate}[(i)]
\item for $a,b,c\in F$, if $a\le b$, then $a+c\le b+c$,
\item for $a,b\in F$, if $a\ge0$ and $b\ge0$, then $ab\ge0$.
\end{enumerate}
Let $F$ be an ordered field, and let $a,b,c\in F$.
\begin{parts}
\item If $a\ge0$ and $b\ge0$, then $a+b\ge0$.
\item If $a\le b$ and $c\ge0$ then $ac\le bc$, and if $a\le b$ and $c\le0$, then $ac\ge bc$.
\item If $c\ge0$, then $-c\le0$, and if $0<a\le b$, then $0<b^{-1}\le a^{-1}$.
\item We always have $a^2\ge0$.
\end{parts}
\end{prb}
\begin{pf}
(b)

(c)

(d)
If $a\ge0$, then $a^2\ge0$ is clear from the second condition for ordered fields.
If $a<0$, then $a^2=(-a)^2\ge0$ because $-a\ge0$.


\end{pf}


\begin{prb}[Definition of rational numbers]
If $F$ is an ordered field, then there is a unique non-zero field homomorphism $\Q\to F$.
\end{prb}

\begin{pf}


Suppose there is $n\in\Z_{>0}$ such that $n1=0$ in $F$.
From $0<1$, we can prove that $(n-1)1\ge0$ by the mathmatical induction, which leads to a contradiction $0=n1=(n-1)1+1\ge0+1=1$.
Therefore, there is no $n\in\Z_{>0}$ such that $n1=0$.
\end{pf}




\section{Least upper bound property}



\begin{prb}[Definition of real numbers]
We say a totally ordered set is \emph{Dedekind complete} if every non-empty bounded subset admits a supremum and a infimum, and satisfies the \emph{least upper bound property} if every non-empty bounded above subset admits a supremum.
The two properties are easily shown to be equivalent to for a totally ordered set.
We define the \emph{set of real numbers} as a Dedekind complete ordered field, usually denoted by $\R$.
A \emph{real number} is an element of $\R$.
Since the multiplication by minus one reverses the order, we may only assume the existence of supremum for the Dedekind completeness of ordered fields.
Let $\R$ be a set of real numbers, and let $a,b\in\R$.
We give the following two basic properties of $\R$.
\begin{parts}
\item If $a>0$, then there is a positive integer $n$ such that $na>b$. In other words, $\R$ is \emph{archimedean}.
\item If $a<b$, then there is a rational number $r$ such that $a<r<b$. In other words, $\Q$ is \emph{dense} in $\R$.
\end{parts}
\end{prb}
\begin{pf}
(a)
Suppose not so that every $n\in\Z_{>0}$ satisfies $na\le b$.
Then, the existence of such $b\in\R$ means the set $S:=\{na:n\in\Z_{>0}\}$ is bounded above in $\R$.
Since $S$ is non-empty by $a\in S$, we have the least upper bound $s\in\R$ of $S$ by the Dedekind completeness of $\R$.
Since $s-a<s$ implies that $s-a$ cannot be an upper bound of $S$, there should be an element $na\in S$ such that $s-a<na$.
This is equivalent to $s<(n+1)a$, which is impossible because $(n+1)a\in S$ contradicts to that $s$ is an upper bound of $S$.
Therefore, there must be $n\in\Z_{>0}$ such that $na>b$.

(b)
Using the archimedean property of $\R$, take $n\in\Z_{>0}$ such that $n(b-a)>1$.
Consider a subset $S:=\{m\in\Z:m<nb\}$ of $\R$.
The subset $S$ is bounded above by an upper bound $nb$, and is non-empty because if we take $n'\in\Z_{>0}$ such that $n'>-nb$ using the archimedean property of $\R$, then $-n'<nb$ implies $-n'\in S$.
By the Dedekind completeness of $\R$, we have the least upper bound $s$ of $S$.

Since $s-1$ is not an upper bound of $S$, we can take $m\in S$ such that $s-1<m$.
Then, we have $nb-1\le m$ because its negation $m<nb-1$ implies $m+1\in S$, which is impossible by $s<m+1$ and the definition of least upper bound.
Therefore, it follows that $na<nb-1\le m<nb$.
If we define $r:=m/n\in\Q$, then we can check $a<r<b$.
\end{pf}







\begin{prb}[Uniqueness of real numbers]
Let $\R_1$ and $\R_2$ are Dedekind complete ordered fields.
Denote by $\Q_1$ and $\Q_2$ the subfields of $\R_1$ and $\R_2$ isomorphic to the field of rational numbers.
Let $\f:\Q_1\to\Q_2$ be the unique field isomorphism, and define $\tilde\f:\R_1\to\R_2$ such that
\[\tilde\f(a):=\sup\{\f(r)\in\Q_2:r<a,\ r\in\Q_1\},\qquad a\in\R_1.\]
\begin{parts}
\item well-definedess and order preserving
\item addition and multiplication
\item injective and surjective
\item uniqueness of field isomorphisms
\end{parts}
\end{prb}
\begin{pf}
(a)
First we prove $\f$ preserves order.

It is well-defined by the Dedekind completeness of $\R_2$ since for $n,n'\in\Z_{>0}$ such that $n>a$ and $n'>-a$ taken by the archimedean property of $\R$ we have an upper bound $\f(n)$ and an element $\f(-n')$ of the set $\{\f(r)\in\Q_2:r<a,\ r\in\Q_1\}$.


(b)
We always have if $r<a+b$
\[\f(r)\le\tilde\f(a)+\tilde\f(b)\]
We can find $r<a+b$ such that
\[\tilde\f(a)+\tilde\f(b)<\f(r)+\e\]

\end{pf}


\begin{prb}[Dedekind construction]
We prove the existence of the set of real numbers by a method called the Dedekind construction.
We assume that the set of rational numbers $\Q$ exists.
A \emph{Dedekind cut} is a non-empty proper subset $a$ of $\Q$ that is downward closed and does not have the maximum.
For example, $\{s\in\Q:s<r\}$ is a Dedekind cut for each $r\in\Q$.

Let $\R$ be the set of all Dedekind cuts.
Define a relation $\le$ on $\R$ such that $a\le b$ if and only if $a\subset b$ for $a,b\in\R$.
Define an operation $+:\R\times\R\to\R$ such that
\[a+b:=\{r+s:r\in a,\ s\in b\},\qquad -a:=\{r:r<-s\text{ for some }s\in\Q\setminus a\},\qquad a,b\in\R,\]
and define an operation $\cdot:\R\times\R\to\R$ such that
\[ab:=\begin{cases}
\{rs:r\in a\cap\Q_{\ge0},\ s\in b\cap\Q_{\ge0}\}\cup\Q_{<0}&\text{ if }a,b\ge0,\\
-(a(-b))&\text{ if }a\ge0,\ b<0,\\
-((-a)b)&\text{ if }a<0,\ b\ge0,\\
(-a)(-b)&\text{ if }a,b<0,
\end{cases},\qquad a,b\in\R.\]

\begin{parts}
\item $\R$ is a field.
\item $\R$ is an ordered field.
\item $\R$ is Dedekind complete.
\end{parts}
\end{prb}




\begin{prb}[Existence of $n$th roots]
\begin{parts}
\item There is no rational number $q$ such that $q^2=2$.
\item If $a\ge0$ and $n\in\Z_{>0}$, then there exists a unique $b\ge0$ such that $b^n=a$.
\end{parts}
\end{prb}
\begin{pf}
(a)

(b)
The uniqueness is clear if $0\le b_1<b_2$ then $b_1^n<b_2^n$ leads a contradiction to $b_1^n=a=b_2^n$.
If $a=0$, then we trivially have $b=0$, so assume $a>0$.
Let $S:=\{x\in\R_{\ge0}:x^n<a\}$, which is non-empty since $0\in S$, and bounded above since $1+a$ is an upper bound of $S$.
For $b:=\sup S$, which is clearly non-negative, we claim $b^n=a$.
It suffices to show $b^n<a$ and $b^n>a$ lead contradictions.

Assume $b^n<a$.
Then, we have $b+\e\in S$ by
\[(b+\e)^n=b^n+\e\sum_{k=0}^{n-1}(b+\e)^kb^{n-1-k}<b^n+\e n(b+\e)^{n-1}<b^n+\e n(b+1)^{n-1}<a,\]
where $\e$ is any real number taken sufficiently small such that
\[0<\e<\min\{1,\frac{a-b^n}{n(b+1)^{n-1}}\},\]
which contradicts to that $b$ is an upper bound of $S$.

Assume $b^n>a$.
Then, we can see $b-\e$ is an upper bound of $S$ by
\[(b-\e)^n=b^n-\e\sum_{k=0}^{n-1}b^k(b-\e)^{n-1-k}>b^n-\e nb^{n-1}>a,\]
where $\e$ is any real number taken sufficiently small such that
\[0<\e<\frac{b^n-a}{nb^{n-1}},\]
which contradicts to the minimality of $b$ among upper bounds of $S$.
\end{pf}


\begin{prb}[Uncountability of real numbers]
\end{prb}




\section{Real sequences}


\begin{prb}[Extended real numbers]
An \emph{extended real number} refers to an element of the disjoint union $\bar\R:=\R\cup\{\pm\infty\}$ of the set $\R$ of real numbers and the set of two distinguished elements denoted by $\infty$ and $-\infty$.
Note that $-\infty$ is not the additive inverse of $\infty$ and is nothing but a notation, because we do not have a well-defined a binary operation on the set $\bar\R$ which extends the addition of $\R$.
However, $\bar\R$ is given a canonical total order defined such that $-\infty\le a\le\infty$ for all $a\in\bar\R$.
In other words, we consider $\bar\R$ as a totally ordered set without any well-defined binary operations, so it is far from a field.
We can introduce the notations for open, closed, and half-closed intervals in extended real numbers as we did in real numbers, and in particular we have $\bar\R=[-\infty,\infty]$.

\begin{parts}
\item $\bar\R$ extends the four fundamental arithmetic operations of $\R$ such that associativity and commutativity of two operations, and other compatibility conditions for ordered fields are all satisfied as long as the occurring expressions are defined, except the followings: the addition is not defined for $(\infty,-\infty)$, $(-\infty,\infty)$, the multiplication is not defined for $(\infty,0)$, $(-\infty,0)$, $(0,\infty)$, $(0,-\infty)$, and the additive and multiplicative inverses are not defined for $\infty$ and $-\infty$.
\item The set of extended real numbers is a bounded totally ordered set that is Dedekind complete.
\end{parts}

\end{prb}
\begin{pf}
The field operations can be partially defined as follows: for $a\in\R\cup\{\pm\infty\}$,
\[a+\infty=\infty,\qquad a\ne-\infty\]
\[a-\infty=-\infty,\qquad a\ne\infty\]
\[a\cdot(\pm\infty)=(\pm\infty\cdot)a=\pm\sgn(a)\infty,\qquad,\qquad a\ne0,\]
\[\frac a{\pm\infty}=0,\qquad a\in\R\]
\[\frac{\pm\infty}a=\pm\sgn(a)\infty,\qquad a\ne0,\pm\infty.\]

The addition is not defined for $(\infty,-\infty)$, $(-\infty,\infty)$, the multiplication is not defined for $(\infty,0)$, $(-\infty,0)$, $(0,\infty)$, $(0,-\infty)$, and the multiplicative inverse is not defined for both the positive and negative infinity.

\end{pf}


\begin{prb}[Convergence of real sequences]
A \emph{sequence} in a set $X$ is a function to $X$ defined either on the totally ordered set $\Z_{\ge0}$ or $\Z_{>0}$.

When $\lim_{n\to\infty}a_n=\infty$, both expressions that $a_n$ converges to infinity and diverges to infinity, are possible.
\end{prb}


\begin{prb}[Monotone sequences]
preserving inequalities
monotone convergence theorem

Let $(a_n)$ be a sequence of extended real numbers.
The sequence $(a_n)$ is called \emph{monotonically increasing} or \emph{non-decreasing} if $a_n\le a_{n'}$ whenever $n\le n'$.
\begin{parts}
\item If $(a_n)$ converges in $[-\infty,\infty]$ and $a_n\ge0$ for all $n$, then $\lim_{n\to\infty}a_n\ge0$.
\item If $(a_n)$ is non-decreasing, then $\lim_{n\to\infty}a_n=\sup_na_n$ in $[-\infty,\infty]$.
\item
\end{parts}
\end{prb}


\begin{prb}[Limit superior and limit inferior]
Let $(a_n)$ be a sequence of extended real numbers.
We define the \emph{limit superior} and the \emph{limit inferior} of $(a_n)$ respectively as
\[\limsup_{n\to\infty}a_n:=\lim_{n\to\infty}\sup_{k\ge n}a_k,\qquad\liminf_{n\to\infty}a_n:=\lim_{n\to\infty}\inf_{k\ge n}a_k,\]
where the limits are taken in $[-\infty,\infty]$.
\begin{parts}
\item If $a_n\le b_n$ for all $n$, then $\limsup_{n\to\infty}a_n\le\limsup_{n\to\infty}b_n$.
\item $(a_n)$ converges if and only if $\liminf_{n\to\infty}a_n=\limsup_{n\to\infty}a_n$.
\end{parts}
\end{prb}
\begin{pf}
(a)
Since the sequence $\sup_{k\ge n}a_n$ is non-decreasing, we have
Taking supremum for $k\ge n$ and infimum for $n$ on the inequality $a_k\le b_k$, we easily get
\[\limsup_{n\to\infty}a_n=\inf_n\sup_{k\ge n}a_n\le\inf_n\sup_{k\ge n}b_n\le\limsup_{n\to\infty}b_n.\]

(b)
Let $a:=\lim_{n\to\infty}a_n\in\R$.
Fix $\e>0$ arbitrarily.
Then, there is $n_0$ such that
\[a_n-a<\e,\qquad n\ge n_0.\]
Taking the supremum on $n$, we get
\[\sup_{k\ge n}a_k-a\le\e,\qquad n\ge n_0.\]
As $n\to\infty$, we have
\[\limsup_{n\to\infty}a_n-a\le\e.\]
Since $\e>0$ is arbitrary, we finally obtain
\[\limsup_{n\to\infty}a_n-a\le0.\]
We can do same thing similarly for limit inferior to write an inequality
\[a\le\liminf_{n\to\infty}a_n\le\limsup_{n\to\infty}a_n\le a,\]
so the claim follows.

Conversely, assume that $\liminf_{n\to\infty}a_n=\limsup_{n\to\infty}a_n$ and let $a$ be the value of it, which is finite since the limit superior and limit inferior cannot be $\infty$ and $-\infty$ respectively.
Fix $\e>0$.
Take $n_1$ and $n_2$ such that 
\[-\e<\inf_{k\ge n}a_k-a,\qquad n\ge n_1,\qquad\qquad\sup_{k\ge n}a_k-a<\e,\qquad n\ge n_2.\]
If we let $n_0:=\max\{n_1,n_2\}$, then
\[-\e<\inf_{k\ge n}a_k-a\le a_n-a\le\sup_{k\ge n}a_k-a<\e,\qquad n\ge n_0,\]
hence the estimate $|a_n-a|<\e$ for all $n\ge n_0$.
Therefore, $\lim_{n\to\infty}a_n=a$.
\end{pf}

\begin{prb}[Asymptotics]
sufficiently large
asymptotic notations
growth and decay
\end{prb}




\section{Elementary functions}



\begin{prb}[Exponential functions]
The \emph{exponential function} is the function $\exp:\C\to\C$ defined by the power series
\[\exp(z):=\lim_{n\to\infty}\sum_{k=0}^n\frac{z^k}{k!},\qquad z\in\C.\]

\[e^z=\lim_{n\to\infty}\left(1+\frac zn\right)^{\frac1n}.\]
\begin{parts}
\item exponential law
\item image, periodicity
\end{parts}
\end{prb}












\section*{Exercises}
\begin{prb}[Real number fields]
\end{prb}

\begin{prb}[Fields of rational functions]
\end{prb}

\begin{prb}[Hyper-real numbers]
\end{prb}

\begin{prb}[Real division algebras]
The Frobenius theorem states that there are only three finite dimensional real division algebras.
\end{prb}

\begin{prb}[Cantor set]
Let $C\subset[0,1]$ be the ternary Cantor set.
\begin{parts}
\item $C$ is uncountable.
\item $C$ is closed in the sense that if every sequence in $C$ converges then the limit is in $C$.
\end{parts}
\end{prb}

\begin{prb}[Diophantine approximation]

\end{prb}

\begin{prb}[Irrational rotations]

\end{prb}


\begin{prb}[Ces\`aro mean]

\end{prb}

\begin{prb}[Non-linear recursive sequences]
Let $f:\R\to\R$ be a function, and $a\in\R$ be a real number.
We investigate the asymptotic behavior of a recursive real sequence $(a_n)$ defined such that
\[\left\{\begin{aligned}
a_{n+1}&=a_n-f(a_n)&&\text{ for }n\ge0,\\
a_0&=a.&&
\end{aligned}\right.\]
\begin{parts}
\item If $a_{n+1}=\frac12(a_n+a_n^{-1})$ for $n\ge0$ and $a_0=2$, then
The Newton method for $x\mapsto x^2-1$.
\item If $a_{n+1}=3a_n(1-a_n)$ for $n\ge0$ and $a_0=1/2$, then
\[a_n=\frac23+...\]
The first bifurcation of the logistic map.
\item If $a_{n+1}=\sin a_n$ for $n\ge0$ and $a_0=1$, then
\[a_n=\sqrt3n^{-\frac12}-\frac{3\sqrt3}{20}n^{-\frac32}+o(n^{-\frac32}).\]
\end{parts}
\end{prb}
\begin{pf}
(a)
We do not know the convergence of $(a_n)$ yet.
However, if we assume $a_n\to a$ as an ansatz for a guess, then we must have $a=1$ from $a=\frac12(a+a^{-1})$.
We first claim $a_n\to1$.

We have $a_n\ge1$.

Estimate the error as
\[a_{n+1}-1=\frac{(a_n-1)^2}{2a_n}.\]


Define $b_n:=\log\log(a_n-1)$.
Then,
\[b_{n+1}-b_n=\log\left(2-\frac{\log(2a_n)}{\log(a_n-1)}\right)\sim\log2\]
\[b_{n+2}-2b_{n+1}-b_n=\]
implies by the Ces\`aro mean
\[b_n\sim n\log2.\]
In this way, we cannot find the exact rate of convergence of $a_n-1$...



(c)
We have bounds $0<a_n\le1$ by induction.
By the monotone convergence theorem, we can see that $a_n$ is decreasing and covergent, and the limit is zero because it is the only fixed point of the sine function.
Define $b_n:=1/a_n^2$.
Since
\[\frac{a_{n+1}}{a_n}=\frac{\sin a_n}{a_n}\sim1,\]
we have
\[b_{n+1}-b_n=\frac1{a_{n+1}^2}-\frac1{a_n^2}=\frac{(a_n+a_{n+1})a_n^3}{a_{n+1}^2a_n^2}\frac{a_n-\sin a_n}{a_n^3}\sim2\cdot\frac16=\frac13.\]
By the limit of the Ces\`aro mean, we have
\[b_n=\frac n3+o(n).\]
It implies that
\[a_n-\sqrt{\frac 3n}=\frac{3a_n^2}{a_n+\sqrt{3/n}}\frac1n\left(b_n-\frac n3\right)=o(n^{-\frac12}).\]

\end{pf}



\section*{Problems}
\begin{enumerate}
\item Show that every real sequence $(a_n)$ has a subsequence $(a_{n_k})$, indexed by $k$, such that $\lim_{k\to\infty}a_{n_k}=\limsup_{n\to\infty}a_n$.
\item Show that a subset $E$ of $\R$ such that $(a+b)/2\in E$ for every $a,b\in E$ containing an interval is an interval.
\item Find all real numbers which can be the limit of a sequence in $\{(\sqrt m-\sqrt n)/(\sqrt m+\sqrt n):m,n\in\Z_{>0}\}$.
\item Show that a subset $E$ of $(0,\infty)$ such that $a/2\in E$ and $\sqrt{a^2+b^2}\in E$ for all $a,b\in E$ is dense in $(0,\infty)$.
\item Find all real numbers $x$ such that a sequence $(\sin(2^nx))$ coverges.
\item If $a_n\ge0$ and $\sum a_n$ diverges, then $\sum\frac{a_n}{1+a_n}$ also diverges.
\item If $a_n\ge0$ and $\sum a_n<\infty$, then there are sequences $b_n\downarrow0$ and $\sum c_n<\infty$ such that $a_n=b_nc_n$.
\end{enumerate}






\chapter{Metrics}
\section{Topology}
\begin{prb}[Metric spaces]
A \emph{metric} on a set $X$ is a function $d:X\times X\to\R_{\ge0}$ such that for all $x,y,z\in X$
\begin{parts}[(i)]
\item $d(x,y)=0$ if and only if $x=y$, \hfill(nondegeneracy)
\item $d(x,z)\le d(x,y)+d(y,z)$, \hfill(triangle inequality)
\item $d(x,y)=d(y,x)$. \hfill(symmetry)
\end{parts}
A \emph{metric space} is a set $X$ equipped with a metric on $X$.
A \emph{point} of a metric space $X$ is just an element of $X$.
\begin{parts}
\item A subset of a metric space is a metric space with a metric given by restriction.
\item If $d:\R\times\R\to\R_{\ge0}$ is defined by $d(a,b):=|a-b|$ for $a,b\in\R$, then $d$ is a metric on $\R$.
\item If $d:X\times X\to\R_{\ge0}$ is defined such that for $x,y\in X$ we have $d(x,y):=1$ if $x\ne y$ and $d(x,y):=0$ if $x=y$, then $d$ is a metric on any set $X$.
\end{parts}
\end{prb}

\begin{prb}[Neighborhood systems]
Metrics are often misguided to those that measure the distance between two points in the study of geoemtry.
However, virtually the most important role of a metric is to provide a reasonable system of small balls, the sets of points whose distances from specified center points are less than some fixed radii.
The balls centered at each point provide a concrete example of a system of neighborhoods at a point in a more intuitive sense.
In this viewpoint, a metric can be considered as a structure that allows someone to perceive the notion of neighborhoods more friendly.

Let $X$ be a metric space.
The set
\[B(x,\e)=B_\e(x):=\{y\in X:d(x,y)<\e\},\qquad x\in X,\ \e>0,\]
is called the \emph{open ball centered at $x$ with radius $\e$}.
Taking positive real numbers denoted by either $\e>0$ or $\delta>0$ in mathematical analysis usually really means taking a ball of the radius $\e$ or $\delta$.


\end{prb}


\begin{prb}[Convergence of sequences]
Let $X$ be a metric space, and let $(x_n)$ be a sequence in $X$.
As we did in $\R$, we can define a \emph{limit} of the sequence $(x_n)$ as a point $x\in X$ or the sequence $(x_n)$ \emph{converges} to a point $x\in X$ if for arbitrarily small $\e>0$ there exists $n_0$ such that $d(x_n,x)<\e$ for all $n\ge n_0$.
The choice of $n_0$ may depend on $\e$.
If it is satisfied, then we write $\lim_{n\to\infty}x_n=x$, or $x_n\to x$ as $n\to\infty$.
We say a sequence in a metric space is \emph{convergent} if it converges to a point.
If it does not converge to any points, then we say the sequence \emph{diverges}.
\begin{parts}
\item A sequence $x_n$ in a metric space $X$ converges to $x\in X$ if and only if $d(x_n,x)$ converges to zero.
\end{parts}
\end{prb}







\begin{prb}[Open sets and closed sets]
Let $X$ be a metric space.
A subset $U\subset X$ is called \emph{open} if for every point $x\in U$ there is $r>0$ such that $B(x,r)\subset U$.
A subset $F\subset X$ is called \emph{closed} if for every sequence $(x_n)$ in $F$ with $x_n\to x$ in $X$ we have $x\in F$.
The set of all open subsets of $X$ is called the \emph{topology} of $X$.

limit points, boundary, closure
\begin{parts}
\item If $U$ is a subset of $X$ and $F:=X\setminus U$ be the complement, then $U$ is open if and only if $F$ is closed.
\end{parts}
\end{prb}

\begin{prb}[Continuous functions]
Let $X$ and $Y$ be metric spaces, and let $f:X\to Y$ be a function.
We say $f$ is \emph{continuous at $x\in X$} if for any ball $B(f(x),\e)\subset Y$ of radius $\e>0$ there is a ball $B(x,\delta)\subset X$ of radius $\delta>0$ such that $f(B(x,\delta))\subset B(f(x),\e)$.
The function $f$ is called \emph{continuous} if it is continuous at every point on $X$.
\begin{parts}
\item $f$ is continuous if and only if $f^{-1}(V)$ is open in $U$ whenever $V$ is open in $Y$.
\item $f$ is continuous if and only if $f(x_n)\to f(x)$ whenever $x_n\to x$, where $(x_n)$ is a sequence in $X$.
\item 
Let $f:X\to Y$ be a function between two metric spaces.
If there is a constant $C$ such that $d(x,y)\le Cd(f(x),f(y))$ for all $x$ and $y$ in $X$, then $f$ is continuous.
In this case, $f$ is particularly called \emph{Lipschitz continuous} with the \emph{Lipschitz constant} $C$.
\end{parts}
\end{prb}

\begin{prb}[Equivalence of metrics]
topologically, uniformly, Lipschitz.
\end{prb}




\section{Completeness}
\begin{prb}[Cauchy sequences]
Cauchy continuous functions

\begin{parts}
\item $\R$ is a complete metric space.
\end{parts}
\end{prb}


\begin{prb}[Uniform continuous functions]
Lipschitz
Heine-Cantor theorem
\end{prb}


\begin{prb}[Continuous extension]
A subset $D$ of a metric space $X$ is called \emph{dense} in $X$ if $\bar D=X$, that is, for every $x\in X$ there is a sequence $(x_n)$ in $D$ such that $x_n\to x$ in $X$.
We frequently says that the point $x$ of the metric space $X$ is approximated by points of $D$.

If $D$ is a dense subset of a metric space $X$, and if $Y$ is a complete metric space, then a uniformly continuous function $f:D\to Y$ is uniquely extended to a continuous function $\tilde f:X\to Y$.
\end{prb}

\begin{prb}[Completion of metric spaces]
Let $X$ be a metric space.
A \emph{completion} of $X$ is a complete metric space $\bar X$ together with an isometry $X\to\bar X$ with dense image.
We will prove that there always exists a completion, and it is unique up to isometry.
To construct a completion, we consider an equivalence relation on the set of all Cauchy sequences in $X$, defined such that two Cauchy sequences $(x_n)_n$ and $(y_n)_n$ in $X$ are equivalent if and only if $d(x_n,y_n)\to0$ as $n\to\infty$.
Let $X^*$ be the set of all equivalence classes $[(x_n)_n]$ of Cauchy sequences $(x_n)_n$ in $X$.
\begin{parts}
\item $X^*$ has a metric $d^*$ defined such that $d^*([(x_n)_n],[(y_n)_n]):=\lim_{n\to\infty}d(x_n,y_n)$ for $[(x_n)_n],[(y_n)_n]\in X^*$.
\item The metric space $(X^*,d^*)$ is complete.
\item The constant sequence map $X\to X^*$ is an isometry with dense image.
\item The completion of $X$ is unique up to isometry.
\end{parts}
\end{prb}
\begin{pf}
(a)
First we can check the limit is well-defined since the real sequence $d(x_n,y_n)$ is Cauchy if $(x_n)_n$ and $(y_n)_n$ are Cauchy in $X$ due to the repeated applications of triangle inequalities
\begin{align*}
|d(x_n,y_n)-d(x_{n'},y_{n'})|
&\le|d(x_n,y_n)-d(y_n,x_{n'})|+|d(y_n,x_{n'})-d(x_{n'},y_{n'})|\\
&\le d(x_n,x_{n'})+d(y_n,y_{n'})\to0,\qquad n,n'\to\infty.
\end{align*}
The metric function $d^*:X^*\times X^*\to\R_{\ge0}$ is well-defined since if $[(x_n)_n]=[(x_n')_n]$ and $[(y_n)_n]=[(y_n')_n]$ in $X^*$, then
\begin{align*}
|d(x_n,y_n)-d(x_n',y_n')|
&\le|d(x_n,y_n)-d(y_n,x_n')|+|d(y_n,x_n')-d(x_n',y_n')|\\
&\le d(x_n,x_n')+d(y_n,y_n')\to0,\qquad n\to\infty.
\end{align*}
The non-degeneracy follows from the definition of the equivalence relation on the Cauchy sequences in $X$, and the symmetry can be trivially verified.
The triangle inequality is also not difficult to see by taking limit $n\to\infty$ on the triangle inequality $d(x_n,z_n)\le d(x_n,y_n)+d(y_n,z_n)$ for Cauchy sequences $(x_n)_n$, $(y_n)_n$, and $(z_n)_n$ in $X$.


(b)
Suppose $([(x_{nm})_n])_m$ is a Cauchy sequence in the metric space $X^*$ indexed by $m$, meaning that
\[\lim_{n\to\infty}d(x_{nm},x_{nm'})=d^*([(x_{nm})_n],[(x_{nm'})_n])\to0\qquad m,m'\to\infty.\]
Fix any $\e>0$.
Then, since there is $m_0$ such that
\[\lim_{n\to\infty}d(x_{nm},x_{nm'})<\e,\qquad m,m'\ge m_0,\]
there is $n_0$, by definition of limit $n\to\infty$, which satisfies
\[d(x_{nm},x_{nm'})<\e,\qquad m,m'\ge m_0,\quad n\ge n_0.\]
By letting $m'=n$, we obtain
\[d(x_{nm},x_{nn})<\e\qquad m\ge m_0,\quad n\ge\max\{m_0,n_0\}.\]

Define a sequence $(x_n)_n$ in $X$ by $x_n:=x_{nn}$.
First we prove $(x_n)_n$ is Cauchy in $X$ so that we can define a point $[(x_n)_n]$ of $X^*$.
Fix any sufficiently large $m$ with $m\ge m_0$.
Then, we can write
\begin{align*}
d(x_n,x_{n'})=d(x_{nn},x_{n'n'})&
\le d(x_{nn},x_{nm})+d(x_{nm},x_{n'm})+d(x_{n'm},x_{n'n'})\\
&<\e+d(x_{nm},x_{n'm})+\e,\qquad n,n'\ge\max\{m_0,n_0\}.
\end{align*}
Since, by definition of $X^*$, the sequence $(x_{nm})_n$ indexed by $n$ in $X$ is Cauchy in $X$ for each $m$, if we take the limit superior $n,n'\to\infty$ on the previous inequality, then we have
\[\limsup_{n,n'\to\infty}d(x_n,x_{n'})\le2\e,\]
so we can further take the limit $\e\to0$ to see that $(x_n)_n$ is Cauchy in $X$.

Next, we check the convergence $[(x_{nm})_n]\to[(x_n)_n]$ as $m\to\infty$ in $X^*$.
Since
\[d^*([(x_{nm})_n],[(x_n)_n])=\lim_{n\to\infty}d(x_{nm},x_{nn})\le\e,\qquad m\ge m_0,\]
if we take the limit superior $m\to\infty$ to get
\[\limsup_{m\to\infty}d^*([(x_{nm})_n],[(x_n)_n])\le\e,\]
then the limit $\e\to0$ shows that $[(x_{nm})_n]\to[(x_n)_n]$ as $m\to\infty$ in $X^*$.
Therefore, the metric space $X^*$ is complete.

(c)
If the equivalence classes of constant sequences $[(x)_n]$ and $[(y)_n]$ are elements of $X^*$ induced from points $x,y\in X$, then
\[d^*([(x)_n],[(y)_n])=\lim_{n\to\infty}d(x,y)=d(x,y),\]
so the map $X\to X^*$ is an isometry.
Also for any $[(x_n)_n]\in X^*$, the sequence $([(x_m)_n])_m$ defined by the image, for each $m$, of the $m$-th point $x_m\in X$ in the sequence $(x_n)_n$ under the isometry $X\to X^*$ converges to $[(x_n)_n]$ because $(x_n)_n$ is Cauchy in $X$ so that
\[\lim_{m\to\infty}d^*([(x_n)],[(x_m)])=\lim_{m\to\infty}\lim_{n\to\infty}d(x_n,x_m)=0.\]
Therefore, the isometry $X\to X^*$ has dense range.

(d)
\end{pf}


\begin{prb}[Separable metric spaces]
separable iff second countable iff lindelof
\end{prb}


\section{Compactness}

\begin{prb}[Open cover]

compact image
\end{prb}

\begin{prb}[Finite intersection property]
Nested intervals
\end{prb}

\begin{prb}[Sequential compactness]
Bolzano-Weierstrass
\end{prb}

\begin{prb}[Total boundedness]
\end{prb}


\begin{prb}[Heine-Borel theorem]
\end{prb}



\section{Connectedness}



\section*{Exercises}
\section*{Problems}








\chapter{Norms}




\section{Series}


\begin{prb}[Normed spaces]
Vector spaces
\end{prb}

\begin{prb}[Noremd algebras]
Algebras
\end{prb}

A \emph{series} is technically nothing but a sequence $(a_n)$, but written in the form $\sum_ka_k$.
In analysis, this notation of series alludes that we will investigate the limit behavior of sequence of partial sums $(\sum_{k\le n}a_k)$.


convergence tests
comparison
limit comparison
cauchy condensation
integral....

ratio
root


\begin{prb}[Unconditional convergence]
\end{prb}



\begin{prb}[Abel transform]
\[A_n(B_n-B_{n-1})+(A_n-A_{n-1})B_{n-1}=A_nB_n-A_{n-1}B_{n-1}\]
\[\sum_{m<k\le n}A_kb_k=A_nB_n-A_mB_m-\sum_{m<k\le n}a_kB_{k-1}.\]
abel test
\end{prb}



\begin{prb}[Dirichlet test]
\end{prb}


\begin{prb}[Mertens theorem]
Let $E$ be a normed algebra.
If a series $\sum_{k\ge0}a_k$ converges to $A$ absolutely and a series $\sum_{k\ge0}b_k$ converges to $B$, then their Cauchy product $\sum_{k\ge0}c_k$ with $c_k:=\sum_{l\le k}a_lb_{k-l}$ converges to $AB$.
Let
\[A_n:=\sum_{k\le n}a_k,\quad B_n:=\sum_{k\le n}b_k,\quad\text{ and }\quad C_n:=\sum_{k\le n}c_k.\]
\end{prb}
\begin{pf}
Note that $\sup_n\|B_n\|<\infty$ is bounded.
Write
\[\|C_n-AB\|\le\|C_n-A_nB_n\|+\|(A_n-A)B_n\|+\|A(B_n-B)\|.\]
Fix any $\e>0$.
Fix $m$ such that
\[\sum_{k=m+1}^\infty\|a_k\|<\e.\]
Then, since
\[C_n=\sum_{k\le n}\sum_{l\le k}a_lb_{k-l}=\sum_{l\le n}\sum_{l\le k\le n}a_lb_{k-l}=\sum_{l\le n}\sum_{k\le n-l}a_lb_k=\sum_{l\le n}a_lB_{n-l},\]
we can write
\begin{align*}
\|C_n-A_nB_n\|
&=\|\sum_{l\le n}a_l(B_n-B_{n-l})\|\\
&\le\|\sum_{l\le m}a_l(B_n-B_{n-l})\|+\|\sum_{m<l\le n}a_l(B_n-B_{n-l})\|\\
&\lesssim\sum_{l\le m}\|a_l\|\|B_n-B_{n-l}\|+\sum_{m<l\le n}\|a_l\|\\
&<\sum_{l\le m}\|a_l\|\|B_n-B_{n-l}\|+\e,
\end{align*}
so the limit superior $n\to\infty$ and the limit $\e\to0$ obtain $\|C_n-A_nB_n\|\to0$.
Since the other two terms $\|(A_n-A)B_n\|$ and $\|A(B_n-B)\|$ clearly converge to zero, the claim follows.
\end{pf}




\section{Banach spaces}

\begin{prb}
Let $E$ be a normed space.
Then, $E$ is Banach if and only every absolutely convergent series converges.
\end{prb}
\begin{pf}
($\Rightarrow$)
Let $\sum_ka_k$ be an absolutely convergent series.
Then, the partial sum $A_n:=\sum_{k\le n}a_k$ is a Cauchy sequence since the absolute convergence implies
\[\|A_n-A_{n'}\|\le\sum_{n'<k\le n}\|a_k\|=\sum_{k\le n}\|a_k\|-\sum_{k\le n'}\|a_k\|\to0,\qquad n\ge n'\to\infty.\]
Since $E$ is Banach, $A_n$ converges.

($\Leftarrow$)
Let $a_n$ be a Cauchy sequence.
Then, we can take a subsequence $a_{n_k}$ of $a_n$ such that
\[a_{n_0}=0,\qquad\|a_{n_k}-a_{n_{k'}}\|<\frac1{2^{k'}},\qquad k\ge k'>0.\]
Consider a series $\sum_{j>0}(a_{n_j}-a_{n_{j-1}})$, which is absolutely convergent.
Then, the partial sum
\[\sum_{0<j\le k}(a_{n_j}-a_{n_{j-1}})=a_{n_k}-a_{n_0}=a_{n_k}\]
converges to some point $a\in E$ by the assumption.
Then, fixing sufficiently large $k>0$ for any $\e>0$ such that $\|a_{n_k}-a\|<\e$ and $2^{-k}<\e$, we have
\[\|a_n-a\|\le\|a_n-a_{n_k}\|+\|a_{n_k}-a\|<\frac1{2^k}+\e<2\e,\qquad n\ge n_k,\]
so the limit superior $n\to\infty$ and the limit $\e\to0$ imply the convergence $a_n\to a$.
\end{pf}



\begin{prb}[Space of linear operators]
\end{prb}

\section{Hilbert spaces}


\begin{prb}
Projections and closed linear subspaces.

Let $H$ be a Hilbert space and $V$ be a closed linear subspace of $H$.

It is a big theorem that every closed subspace is complemented in a Hilbert space.


\begin{parts}
\item For $x\in H$, there is a unique $y\in V$ such that $x-y\in V^\perp$.
\item $V^{\perp\perp}=V$.
\end{parts}
\end{prb}
\begin{pf}
(a)


(b)
If $x\in V^{\perp\perp}$, then there is $y\in V$ such that $x-y\in V^\perp$, but since $x-y\in V^{\perp\perp}$ implies $x-y\in V^\perp\cap V^{\perp\perp}=0$, so $x\in V$.

\end{pf}

\section{Sequence spaces}

$\ell^1$, $\ell^\infty$, $\ell^p$, $c_0$, $c_c$



\section*{Exercises}


\section*{Problems}




\part{Functions}

\chapter{Continuity}


\section{Uniform convergence}

\begin{prb}
Let $X$ and $Y$ be metric spaces.
The set of all continuous functions $f:X\to Y$ is denoted by $C(X,Y)$.
If $Y=\R$ or $Y=\C$, then we usually write $C(X)$ to denote $C(X,Y)$.
\begin{parts}
\item If $X$ is compact and $Y$ is complete, then $C(X,Y)$ is complete.
\end{parts}
\end{prb}
\begin{pf}
(a)
Suppose $(f_n)$ is a Cauchy sequence in $C(X)$.
Since it is pointwise Cauchy, we have a function $f:X\to\R$ such that $f_n\to f$ pointwisely.
We first claim that $f_n\to f$ uniformly.
Fix $\e>0$.
Write
\begin{align*}
|f_n(x)-f(x)|
&\le|f_n(x)-f_{n'}(x)|+|f_{n'}(x)-f(x)|\\
&\le\|f_n-f_{n'}\|+|f_{n'}(x)-f(x)|\qquad n,n'\ge0,\ x\in X.
\end{align*}
Since $(f_n)$ is uniformly Cauchy, there is $n_0$ such that
\[\|f_n-f_{n'}\|<\e,\qquad n,n'>n_0,\]
so that the pointwise limit $n'\to\infty$ has the inequality
\[|f_n(x)-f(x)|\le\e,\qquad n>n_0,\ x\in X.\]
The limit $\e\to0$ after the supremum over $x\in X$ and the limit superior $n\to\infty$ implies the uniform limit $f_n\to f$.

Now we claim $f$ is continuous.
Let $a\in X$ and fix $\e>0$.
Divide the error as
\begin{align*}
|f(x)-f(a)|
&\le|f(x)-f_n(x)|+|f_n(x)-f_n(a)|+|f_n(a)-f(a)|\\
&\le2\|f-f_n\|+|f_n(x)-f_n(a)|,\qquad n\ge0,\ x\in X.
\end{align*}
Using the uniform convergence $f_n\to f$, we can fix $n$ such that
\[|f(x)-f(a)|<\e+|f_n(x)-f_n(a)|,\qquad x\in X.\]
Then, taking the limit superior $x\to a$ and $\e\to0$, the continuity of $f$ follows from the continuity of $f_n$.

(b)

\end{pf}

\begin{prb}[Dini theorem]
\end{prb}





\section{Arzela-Ascoli theorem}

\begin{prb}[Arzela-Ascoli theorem]
Let $X$ be a compact metric space, and $Y$ be a metric space.
Let $\cF\subset C(X,Y)$ be an \emph{equi-continuous} family of functions in the sense that at each point $x\in X$ for every $\e>0$ there is a $\delta>0$ such that
\[\sup_{f\in\cF}d(f(x),f(x'))<\e,\qquad x'\in B(x,\delta).\]
Suppose further that $\cF$ is pointwisely totally bounded.
\begin{parts}
\item $\cF$ is uniformly totally bounded.
\end{parts}
\end{prb}
\begin{pf}
(a)
Let $(f_n)$ be a sequence in $\cF$.
We prove the existence of a Cauchy subsequence.
Fix $\e>0$.
Since $\cF$ is equi-continuous, each point $x\in X$ has an open neighborhood $U_x$ such that
\[d(f_n(x),f_n(x'))<\e,\qquad x'\in U_x,\ n\ge0.\]
Since $\{U_x\}_{x\in X}$ is an open cover of a compact set $X$, there is finite subcover $\{U_{x_j}\}_j$.
Since $\cF$ is pointwisely totally bounded, we can take a subsequence $(f_{n_k})_k$ of $(f_n)$ finitely many times such that $(f_{n_k}(x_j))_k$ is a Cauchy sequence for each $j$.
For any $x\in X$, since there is $j$ such that $x\in U_{x_j}$, we have
\begin{align*}
d(f_{n_k}(x),f_{n_{k'}}(x))
&\le d(f_{n_k}(x),f_{n_k}(x_j))+d(f_{n_k}(x_j),f_{n_{k'}}(x_j))+d(f_{n_{k'}}(x_j),f_{n_{k'}}(x))\\
&<\e+\max_jd(f_{n_k}(x_j),f_{n_{k'}}(x_j))+\e
\end{align*}
for all $k,k'$.
The right-hand side does not depend on $x$, so after taking supremum for $x\in X$, by limiting $k,k'\to\infty$ and $\e\to0$, we get
\[\sup_{x\in X}d(f_{n_k}(x),f_{n_{k'}}(x))\to0,\qquad k,k'\to\infty.\]
Therefore, the subsequence $(f_{n_k})_k$ is uniformly Cauchy.
\end{pf}



\begin{prb}
Let $X$ be a topological space and $Y$ be a uniform space.
Consider the compact-open topology on $C(X,Y)$, which is the topology of compact convergence since $Y$ is a uniform space.
If $\cF\subset C(X,Y)$ is pointwise totally bounded and locally equi-continuous, then it is totally bounded.
\begin{parts}
\item
\end{parts}
\end{prb}
\begin{pf}
We may assume $X$ is compact so that $\cF$ is globally equi-continuous.
Let $f_i$ be a net in $\cF$ and fix an entourage $E$ of $Y$ with a smaller entourage $E'$ such that $E'^3\subset E$.
Since $\cF$ is globally equi-continuous, each point $x\in X$ has an open neighborhood $U_x$ such that
\[(f(x),f(x'))\in E',\qquad x'\in U_x,\ f\in\cF.\]
Take a finite subcover $\{U_{x_j}\}$.
Since $\cF$ is pointwisely totally bounded, taking subnets repeatedly but finitely many, we may assume that $f_i(x_j)$ is Cauchy for each $j$.
Find $i_0$ such that for all $j$
\[(f_i(x_j),f_{i'}(x_j))\in E',\qquad i,i'\succ i_0.\]
Then, for every $x\in X$ there is $j$ with $x\in U_{x_j}$ so that
\[(f_i(x),f_{i'}(x))=(f_i(x),f_i(x_j))\circ(f_i(x_j),f_{i'}(x_j))\circ(f_{i'}(x_j),f_{i'}(x))\in E'^3\subset E,\qquad i,i'\succ i_0.\]
Therefore, $f_i$ is uniformly Cauchy.
\end{pf}


\begin{prb}[Topology of compact convergence]
Let $X$ and $Y$ be metric spaces.
\end{prb}


\section{Stone-Weierstrass theorem}

\begin{prb}[Polynomial approximation of square root]
\end{prb}
\begin{pf}
Let $(p_n)$ be the sequence of functions in $C[0,1]$ defined recursively such that
\[p_{n+1}(s):=p_n(s)+\frac12(s-p_n^2(s)),\qquad p_0(s)=0,\qquad n\ge0,\ s\in[0,1].\]
Then, we can check $p_n$ is non-decreasing sequence of real polynomials satisfying $p_n(s)\le\sqrt s$ for all $n\ge0$ and $s\in[0,1]$ by the induction, as we have $p_0(s)=0\le\sqrt s$ and the inductive hypothesis $p_n(s)\le\sqrt s$ implies
\begin{align*}
p_{n+1}(s)
&=p_n(s)+\frac s2-\frac12p_n^2(s)\\
&=-\frac12(1-p_n(s))^2+\frac12(1+s)\\
&\le-\frac12(1-\sqrt s)^2+\frac12(1+s)=\sqrt s,\qquad s\in[0,1].
\end{align*}
Then,
\[\sqrt s-p_{n+1}(s)=(\sqrt s-p_n(s))\left(1-\frac12(\sqrt s+ p_n(s))\right)\le(\sqrt s-p_n(s))\left(1-\frac{\sqrt s}2\right)\]
implies the inequality
\[\sqrt s-p_n(s)\le\sqrt s\left(1-\frac{\sqrt s}2\right)^n,\qquad n\ge0,\ s\in[0,1].\]
By the Dini theorem or the computation $\sup_{t\in[0,1]}t(1-t)^n=n^n/(n+1)^{n+1}$, the right hand side converges to zero uniformly on $[0,1]$, so $p_n$ converges uniformly to the square root function on $[0,1]$.
\end{pf}

\begin{prb}[Stone-Weierstrass theorem]
Let $X$ be a compact metric space.
Suppose $A$ is a unital subalgebra of $C(X,\R)$ that separates points of $X$.
In other words, suppose $A$ is a linear subspace containing a constant function and closed under the multiplication, and for every distinct pair $x\ne y$ in $X$ there is $f\in A$ such that $f(x)\ne f(y)$.


\begin{parts}
\item $\bar A$ is also a unital subalgebra.
\item $\bar A$ is a lattice.
\item $A$ is uniformly dense in $C(X,\R)$.
\end{parts}
\end{prb}
\begin{pf}
(a)

(b)
Since $|f|=\sqrt{f^2}$ and
\[\max\{f,g\}=\frac12(f+g+|f-g|),\qquad\min\{f,g\}=\frac12(f+g-|f-g|)\]
for $f,g\in\bar A$, it is enough to show $\sqrt f\in\bar A$ for any $f\in\bar A$ with $f\ge0$.
We may assume $f\le1$ because $\bar A$ is linear.

Let $(p_n)$ be a sequence of real polynomials uniformly convergent to the square root function in $C([0,1],\R)$.
Since $\bar A$ is a unital algebra, we have $p_n(f)\in\bar A$ for each $n$.
Then, the claim $\sqrt f\in\bar A$ follows from
\[\|p_n(f)-\sqrt f\|=\sup_{x\in X}|p_n(f(x))-\sqrt{f(x)}|\le\sup_{s\in[0,1]}|p_n(s)-\sqrt s|\to0,\qquad n\to\infty.\]

(c)
It suffices to show $\bar A$ is uniformly dense in $C(X,\R)$, since a closed dense subset is the whole set $\bar A=C(X,\R)$, which means that $A$ is dense in $C(X,\R)$.
Fix $f\in C(X,\R)$ and $\e>0$.
We will construct $g\in\bar A$ such that $\|f-g\|<\e$.
Because the linear space $A$ contains a constant and separates points, for each pair of points $x$ and $y$ in $X$ we always have $g_{x,y}\in A$ such that
\[g_{x,y}(x)=f(x),\qquad g_{x,y}(y)=f(y).\]
Let
\[U_{x,y}:=\{z\in X:|g_{x,y}(z)-f(z)|<\e\},\qquad x,y\in X.\]
Since $|g_{x,y}-f|$ is continuous and since $y\in U_{x,y}$, the collection $\{U_{x,y}\}_{y\in X}$ is an open cover of $X$ for each $x\in X$, so we have a finite subcover $\{U_{x,y_j}\}_j$ by the compactness of $X$.
Define
\[g_x:=\min_jg_{x,y_j},\]
which is an element of $\bar A$ by the part (b).
Then, because for any $z\in X$ there is $j$ satisfying $z\in U_{x,y_j}$ so that $g_x(z)\le g_{x,y_j}(z)<f(z)+\e$, we have
\[g_x(x)=f(x),\qquad g_x<f+\e.\]
Let
\[U_x:=\{z\in X:|g_x(z)-f(z)|<\e\},\qquad x\in X.\]
Then, similarly $\{U_x\}_{x\in X}$ is an open cover of $X$, so we have a finite subcover $\{U_{x_j}\}_j$.
Define
\[g:=\max_jg_{x_j},\]
which is also an element of $\bar A$ by the part (b).
Then, because for every $j$ we have $g_{x_j}<f+\e$ and for any $z\in X$ there is $j$ satisfying $z\in U_{x_j}$ so that $g(z)\ge g_{x_j}(z)>f(z)-\e$, we have
\[f-\e<g<f+\e.\]
Therefore, we have $\|f-g\|<\e$.
\end{pf}


Locally compact version:
We say $A\subset C_0(X,\R)$ \emph{vanishes nowhere} if for every $x$ in $X$ there is $f\in A$ such that $f(x)\ne0$.
Then, $A$ vanishes nowhere if and only if $A+\C$ separates points of the one-point compactification $X_+$.
Let $A$ be a subalgebra of $C_0(X,\R)$ that separates points and vanishes nowhere.
Then, $A+\C$ is dense in $C(X_+,\R)$.
If $f\in C_0(X,\R)$ and $a_n+\lambda_n\in A+\C$ converges to $f$, then $\lambda_n\to0$ implies $a_n\to f$.
Therefore, $A$ is dense in $C_0(X,\R)$.




Complex version:
Let $A$ be a $*$-subalgebra of $C_0(X,\C)$ that separates points and vanishes nowhere.
Then, $A^{sa}$ is a subalgebra of $C_0(X,\R)$ that separates points and vanishes nowhere since $A^{sa}$ contains all real parts and imaginary parts of $A$.
Then, $A^{sa}$ is dense in $C_0(X,\R)$.
For $f\in C_0(X,\C)$, take $a_n,b_n\in A^{sa}$ such that $a_n\to\Re f$ and $b_n\to\Im f$.
Then, $a_n+ib_n\in A$ and $a_n+ib_n\to f$, so we are done.




\section{Normality of metric spaces}


\begin{prb}[Urysohn lemma]
Let $X$ be a metric space.
Let $A$ and $B$ be disjoint closed subsets of $X$.
\begin{parts}
\item A metric space is normal in the sense that there are disjoint open subsets $U$ and $V$ of $X$ such that $A\subset U$ and $B\subset V$.
\item There is a continuous function $f:X\to[0,1]$ such that $f(a)=0$ for $a\in A$ and $f(b)=1$ for $b\in B$.
\end{parts}
\begin{pf}
(a)

(b)
First we construct a collection $\{U_r:r\in\Q\cap[0,1]\}$ of open subsets such that
\[A\subset U_0,\quad\bar U_r\subset U_{r'},\quad\bar U_1\subset B,\qquad r<r'.\]
Since $\Q\cap[0,1]$ is countable, we can take its numeration, a seuqence $(r_n)$ indexed by $n\in\Z_{\ge0}$ such that $\{r_n:n\in\Z_{\ge0}\}=\Q\cap[0,1]$ and $r_n\ne r_{n'}$ whenever $n\ne n'$.
We may assume $r_0=0$ and $r_1=1$.
Let $U_1:=X\setminus B$, and using the normality of $X$, choose an open subset $U_0$ of $X$ such that $A\subset U_0$ and $\bar U_0\subset U_1$.
If we suppose we have constructed $U_{r_0},\cdots,U_{r_n}$, then taking $r,r'$ such that $\{r_0,\cdots,r_{n+1}\}\cap[r,r']=\{r,r_{n+1},r'\}$ and using the normality of $X$, we can choose an open subset $U_{r_{n+1}}$ of $X$ such that $\bar U_r\subset U_{r_{n+1}}$ and $\bar U_{r_{n+1}}\subset U_{r'}$.
Then we are done by induction.

Define $f:X\to[0,1]$ by
\[f(x):=\inf\{r\in\Q\cap[0,1]:x\in U_r\},\qquad x\in X.\]
We clearly have $f(a)=0$ for $a\in A$ and $f(b)=1$ for $b\in B$.
For any $x\in X$ and $\e>0$, if we take $r,r'\in\Q\cap[0,1]$ such that $f(x)-\e<r<f(x)<r'<f(x)+\e$, then $U:=U_{r'}\setminus\bar U_r$ satisfies $x\in U$ and $|f(x)-f(y)|<\e$ for all $y\in U$, so $f$ is continuous at every point of $X$.
\end{pf}

\end{prb}
\begin{prb}[Tietze extension]
Let $X$ be a metric space.
Let $A$ be a closed subset of $X$ and $f:A\to[-1,1]$ be a continuous function.
Then, there is a continuous extension $\tilde f:X\to[-1,1]$ of $f$.
\end{prb}
\begin{pf}
Initially, by the Urysohn lemma, we can take a continuous function $g_1:X\to[-\frac13,\frac13]$ such that $g_1(a)=-\frac13$ on $a\in f^{-1}([-1,-\frac13])$ and $g_1(b)=\frac13$ on $b\in f^{-1}([\frac13,1])$ so that
\[\sup_{x\in X}|g_1(x)|\le\frac13,\qquad\sup_{a\in A}|(f-g_1)(a)|\le\frac23.\]
Suppose we have constructed $g_1,\cdots,g_n$.

Define a sequence of continuous functions $f_n:A\to\left[-\left(\frac23\right)^n,\left(\frac23\right)^n\right]$ by $f_0:=f$ and
\[f_n(a):=f_{n-1}(a)-g_n(a)=f(a)-\sum_{k=1}^ng_k(a),\qquad a\in A,\ n\ge1.\]
Using the Urysohn lemma, take a continuous function
\[g_{n+1}:X\to\left[-\frac13\left(\frac23\right)^n,\frac13\left(\frac23\right)^n\right]\]
such that
\[g_{n+1}(a)=-\frac13\left(\frac23\right)^n,\qquad a\in f_n^{-1}\left(\left[-\left(\frac23\right)^n,-\frac13\left(\frac23\right)^n\right]\right)\]
and
\[g_{n+1}(b)=\frac13\left(\frac23\right)^n,\qquad b\in f_n^{-1}\left(\left[\frac13\left(\frac23\right)^n,\left(\frac23\right)^n\right]\right)\]
so that we have
\[\sup_{x\in X}|g_{n+1}(x)|\le\frac13\left(\frac23\right)^n,\qquad\sup_{a\in A}|(f_n-g_{n+1})(a)|\le\left(\frac23\right)^{n+1}.\]

\end{pf}


\begin{prb}[Continuous partition of unity]
Let $X$ be a metric space.
A family $\{F_i\}$ of subsets of $X$ is called \emph{locally finite} if every point $x\in X$ has an open neighborhood $U$ such that there are only finitely many $i$ satisfying $U\cap F_i\ne\varnothing.$
Let $\{U_i\}$ be an open cover of $X$.
A \emph{continuous partition of unity} subordinate to $\{U_i\}$ is a collection $\{\chi_i\}$ of continuous functions $\chi_i:X\to[0,1]$ such that $\supp\chi_i\subset U_i$ for each $i$ and $\sum_i\chi_i(x)=1$ for all $x\in X$.
Suppose $\{U_i\}$ is locally finite.
\begin{parts}
\item There is a locally finite open cover $\{V_i\}$ of $X$ such that $\bar V_i\subset U_i$ for each $i$.
\item A continuous partition of unity subordinate to $\{U_i\}$ exists.
\end{parts}
\end{prb}
\begin{pf}
(a)


(b)
Applying the part (a) twice, find locally finite open covers $\{V_i\}$ and $\{W_i\}$ such that $\bar W_i\subset V_i$ and $\bar V_i\subset U_i$ for each $i$.
By the Urysohn lemma, we can construct for each $i$ a continuous function $f_i:X\to[0,1]$ such that $f_i=0$ on $X\setminus V_i$ and $f_i=1$ on $\bar W_i$.
Note that we have $\supp f_i\subset\bar V_i\subset U_i$ and $1\le\sum_if_i(x)<\infty$ for all $x\in X$.
Define $\chi_i:X\to[0,1]$ by
\[\chi_i(x):=\frac{f_i(x)}{\sum_if_i(x)},\qquad x\in X.\]
Then, $\{\chi_i\}$ is a continuous partition of unity subordinate to the open cover $\{U_i\}$.
\end{pf}





\section*{Exercises}
\begin{prb}[Space filling curves]
\end{prb}
\begin{prb}[Bernstein polynomial]
We want to show $\R[x]$ is dense in $C([0,1],\R)$.
Let $f\in C([0,1],\R)$ and define \emph{Berstein polynomials} $B_n(f)\in\R[x]$ for each $n$ such that
\[B_n(f)(x):=\sum_{k=0}^nf\left(\frac kn\right)\binom nkx^k(1-x)^{n-k}.\]
\begin{parts}
\item $B_n(f)$ uniformly converges to $f$ on $[0,1]$.
\item There is a sequence $p_n\in\R[x]$ with $p_n(0)=0$ uniformly convergent to $x\mapsto|x|$ on $[-1,1]$.
\end{parts}
\end{prb}
\begin{pf}
(b)
Let
\[B_n(x):=\sum_{k=0}^n\left|1-\frac{2k}n\right|\binom nk(1-2x)^k(2x-1)^{n-k}.\]
Since $B_n(x)\to|x|$ uniformly on $[-1,1]$ and $B_n(0)\to0$, we have $B_n(x)-B_n(0)\to|x|$ uniformly on $[-1,1]$.
\end{pf}

\begin{prb}[Taylor series of square root]
We want to show the absolute value is approximated by polynomials in $C([-1,1],\R)$ in another way.
Let
\[f_n(x):=\sum_{k=0}^n a_k(x-1)^k\]
be the partial sum of the Taylor series of the square root function $\sqrt x$ at $x=1$.
\begin{parts}
\item By Abel's theorem, $f_n$ uniformly converges to $\sqrt x$ on $[0,1]$
\item There is a sequence $p_n\in\R[x]$ with $p_n(0)=0$ uniformly convergent to $x\mapsto|x|$ on $[-1,1]$.
\end{parts}
\end{prb}

\begin{prb}[Semi-continuity]
\end{prb}



\begin{prb}
Some examples
\begin{parts}
\item If $K\subset\C$, then $\C[z,\bar z]$ is dense in $C(K)$.
\item If $K\subset\R$ or $K\subset\T$, then $\C[z]$ is dense in $C(K)$.
\item If $K\subset\R\setminus\{0\}$, then $z\C[z]$ is dense in $C(K)$.
\end{parts}
\end{prb}



\section*{Problems}
\begin{enumerate}
\item* Show that a sequence of functions $f_n:[0,1]\to[0,1]$ that satisfies $|f_n(x)-f_n(y)|\le|x-y|$ whenever $|x-y|\ge\frac1n$ has a uniformly convergent subsequence.
\item Show that for a sequence of differentiable functions $f_n:\R\to\R$ that satisfies $|f_n'(x)|\le1$ for all $n\ge1$ and $x\in\R$ its pointwise limit is continuous if it exists.
\item Show that a sequence of $C^1$ functions $f_n:[0,1]\to\R$ such that $|f_n'(x)|\le x^{-\frac12}$ for $x\in(0,1]$ and $\int_0^1f_n(x)\,dx=0$ for all $n\ge1$ has a uniformly convergent subsequence.
\end{enumerate}


\begin{enumerate}
\item The set of local minima of a convex real function is connected.
\item Let $f:\R\to\R$ be continuous.
The equation $f(x)=c$ cannot have exactly two solutions for every constant $c\in\R$.
\item A continuous function that takes on no value more than twice takes on some value exactly once.
\item Let $f$ be a function that has the intermediate value property.
If the preimage of every singleton is closed, then $f$ is continuous.
\item If a continuous function $f:[0,\infty)\to\R$ has a limit at infinity, then it is uniformly continuous.
\item If $f:[0,1]^2\to\R$ is continuous, then $g:[0,1]\to\R$ defined by $g(x):=\max_{y\in[0,1]}f(x,y)$ is continuous.
\item A metric space $X$ is compact if and only if every continuous function $f:X\to\R$ has bounded image.
\end{enumerate}







\chapter{Differentiability}

\section{Taylor theorem}
\begin{prb}[Rolle theorem]
Let $f:[a,b]\to\R$ be a function that is continuous on $[a,b]$ and differentiable on $(a,b)$.
\begin{parts}
\item If $f(a)=f(b)=0$, then there is $c\in(a,b)$ such that $f'(c)=0$.
\item Suppose $f$ is $(n+1)$-times differentiable. If $f(a)=f'(a)=\cdots=f^{(n)}(a)=0$ and $f(b)=0$, then there is $c\in(a,b)$ such that $f^{(n+1)}(c)=0$.
\end{parts}
\end{prb}
\begin{pf}
(a)
If $f\equiv0$, then it is clear.
If not, we may assume there is $x\in(a,b)$ such that $f(x)>0$ by multiplying $-1$.
Since $f$ is continuous, by the extreme value theorem, there is $c\in(a,b)$ such that $c$ attains the maximum of $f$.
Then, $f'(c)=0$.

(b)
By the induction, we have $c_n\in(a,b)$ such that $f^{(n)}(c)=0$.
By applying Rolle's theorem (the part (a)) for $f^{(n)}$, we have $c_{n+1}\in(a,c_n)$ such that $f^{(n+1)}(c_{n+1})=0$.
\end{pf}

\begin{prb}[Mean value theorem]
\[\frac{f(b)-f(a)}{b-a}=f'(c)\]
\end{prb}

\begin{prb}[Taylor theorem]
\end{prb}


\section{Optimization}
\begin{prb}[L'hopital theorem]
\end{prb}

monotone and convex


\section{Differentiable classes}

\begin{prb}[Continuously differentiable functions]
A function $f:[a,b]\to\R$ is called \emph{continuously differentiable} if it is differentiable on the open interval $(a,b)$ and the derivative $f'$ is continuously extended to the closed interval $[a,b]$.

\[C^1([a,b])\subset C([a,b])\cap C^1((a,b)).\]
\begin{parts}
\item $C^1([a,b])$ is Banach.
\end{parts}
\end{prb}
\begin{pf}
(a)
Let $(f_n)$ be a Cauchy sequence in $C^1([a,b])$.
Then, we have $f_n\to f$ and $f_n'\to g$ uniformly in $C([a,b])$.
Fix $x\in[a,b]$.
For each $h\ne0$ and $n$, using the mean value theorem, we can find $c\in[a,b]$ such that $D_hf_n(x)=f_n'(c)$ with $|c-x|<|h|$, so we have
\begin{align*}
|D_hf(x)-g(x)|
&\le|D_hf(x)-D_hf_n(x)|+|D_hf_n(x)-f_n'(c)|+|f_n'(c)-g(c)|+|g(c)-g(x)|\\
&\le\frac2h\|f-f_n\|+0+\|f_n'-g\|+\sup_{y:|y-x|<|h|}|g(y)-g(x)|,\qquad h\ne0,\ n\ge0.
\end{align*}
As $n\to\infty$ and $h\to0$, the continuity of $g$ concludes that $f'=g$.
\end{pf}


\begin{prb}[H\"older spaces]
\end{prb}



\begin{prb}[Compact embeddings]
Let $X$ be a compact metric space.
We consider the topology of compact convergence on $C(X,\R)$.
Let $\cF\subset C(X,\R)$ be a family of functions with constants $C>0$ and $\alpha>0$ such that every point of $X$ has an open neighborhood $U$ satisfying
\[|f(x)-f(y)|\le Cd(x,y)^\alpha,\qquad f\in\cF,\ x,y\in U.\]
In other words, $\cF$ is bounded in $C^{0,\alpha}(X,\R)$.
Then, $\cF$ is totally bounded in $C(X,\R)$.
\end{prb}


\section{Smooth functions}

partition of unity
density


\begin{prb}[Smooth partition of unity]
\end{prb}




\section*{Exercises}
\begin{prb}[Variations on the mean value theorem]
Let $f$ be a differentiable function on the unit closed interval.
\begin{parts}
\item If $f(0)=0$ there is $c$ such that $cf'(c)=f(c)$. (Flett)
\item If $f(0)=0$ there is $c$ such that $cf(c)=(1-c)f'(c)$.
\end{parts}
\end{prb}
\begin{prb}[Nowhere differentiable function]
\end{prb}
\begin{prb}[Dini derivatives]
\end{prb}
\begin{prb}[Darboux theorem]
\end{prb}

\section*{Problems}
\begin{enumerate}
\item If $\lim_{x\to\infty}f(x)=a$ and $\lim_{x\to\infty}f'(x)=b$, then $a=0$.
\item Let $f$ be a real $C^2$ function with $f(0)=0$ and $f''(0)\ne0$.
Defined a function $\xi$ such that $f(x)=xf'(\xi(x))$ with $|\xi|\le|x|$, we have $\xi'(0)=1/2$.
\item Let $f$ be a $C^2$ function such that $f(0)=f(1)=0$.
We have $\|f\|\le\frac18\|f''\|$.
\item A smooth function such that for each $x$ there is $n$ having the $n$th derivative vanish is a polynomial.
\item If a real $C^1$ function $f$ satisfies $f(x)\ne0$ for $x$ such that $f'(x)=0$, then in a bounded set there are only finite points at which $f$ vanishes.
\item Let a real function $f$ be differentiable.
For $a<a'<b<b'$ there exist $a<c<b$ and $a'<c'<b'$ such that $f(b)-f(a)=f'(c)(b-a)$ and $f(b')-f(a')=f'(c')(b'-a')$.
\item Let $f:[1,\infty)\to\R$ satisfy that $f(1)=1$ and $f'(x)=(x^2+f(x)^2)^{-1}$. Show that $\lim_{x\to\infty}f(x)$ exists in the open interval $(1,1+\frac\pi4)$.
\item If $f:(0,\infty)\to\R$ is $C^2$ and satisfies $f'(x)\le0<f(x)$ for all $x>0$, then the boundedness of $f''$ implies $f'(x)\to0$ as $x\to\infty$.
\item If a function $f:[0,1]\to\R$ that is continuous on $[0,1]$ and differentiable on $(0,1)$ satisfies $f(0)=0$ and $0\le f'(x)\le2f(x)$, then $f$ is identically zero.
\item For $C^2$ function $f:\R\to\R$ we have $\|f'\|^2\le4\|f\|\|f''\|$.
\item For a smooth function $f:\R\to\R$ such that $f'''(x)<0$, we have $\frac{f'(x)+f'(y)}2<\frac{f(x)-f(y)}{x-y}$ for all $x\ne y\in\R$.
\end{enumerate}




\chapter{Analyticity}


% We do not consider isolated singularities and meromorphic functions.


\section{Power series}


\begin{prb}[Power series]
A \emph{power series} at $a\in\C$ is a series $\sum_{k\ge0}$

For a smooth function $f:I\to\R$ on an open interval $I\subset\R$ and a point $a\in I$, we can associate a power series
\[\sum_{k\ge0}\frac{f^{(k)}(a)}{k!}(x-a)^k\]

\end{prb}

uniform convergence and absolute convergence, abel theorem?
differentiation
convergence of radius, complex domain
sum, product, composition, reciprocal?
closed under uniform convergence

\begin{prb}[Analytic functions]
Let $U\subset\C$ be an open set.
A function $f:U\to\C$ is called \emph{analytic} at a point $a\in U$ if the Taylor series expanded at $a$ converges  on an open neighborhood of $a$, and \emph{analytic} if it is analytic at every point of $U$.


For $a\in\C$ and $r>0$, the space $\cO(a,r)$ can be defined by the set of complex sequences $(c_k)_{k\ge0}$ such that $c_kz^k\to0$ for each $|z|<r$.
For $f\in\cO(a,r)$, the associted sequence $c_k$ is written as $f^{(k)}(a)/k!$.
It is an Arens-Michael Fr\'echet algebra.

By definition, the set of functions analytic at $a$ is $\bigcup_{r>0}\cO(a,r)$.


For a complex sequence $(c_k)$, the supremum of $r_0$ such that $c_kz^k\to0$ for each $|z|<r$ is called the radius of convergence.
The density of polynomials
\end{prb}

\begin{prb}[Cauchy estimate]
\end{prb}

\begin{prb}[Identity theorem]
\end{prb}

\begin{prb}[Open mapping theorem]
\end{prb}


\section{Analytic continuation}


functional equations!


\section{}
\section{}
% every functions as power series
% root, exponential, logarithmic, trigonometric, hyperbolic
% hypergeometric, bessel, gamma, zeta are not treated before integrals





\section*{Problems}

	
\begin{enumerate}
\item Show that if $f:(-1,1)\to\R$ is a smooth function such that $|f^{(n)}(x)|\le1$ for all $n\ge1$ uniformly then $f$ is analytic.
\end{enumerate}



\end{document}