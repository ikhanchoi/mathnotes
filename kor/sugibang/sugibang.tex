\documentclass{../../large}
\usepackage[kor]{../../ikhanchoi}

\setlist[itemize]{leftmargin=16pt,itemsep=0pt,label=$\diamond$}
\setlist[enumerate]{leftmargin=16pt,itemsep=0pt}


\begin{document}
\title{수학의 기초와 방법}
\author{최익한}
\maketitle

\chapter*{머리말}

\section*{}


이 책의 예상독자는 자신의 전문분야에서 통계학이나 인공지능과 같은 도구를 사용하기 위해 수학이 필요하다는 이야기를 듣고 수학공부의 필요성을 느낀 인문사회계 학생들, 공학계 학생들, 그리고 자연계 학생들이다.
조금 더 정확하게 독자들에게 요구하는 배경지식을 기술하자면, 고등학교를 졸업할 때까지 이수과목으로서의 수학을 포기하지 않음으로써 대학입학시험을 치르는 데 큰 지장이 없었고, 아직 이차방정식과 피타고라스 정리를 비롯한 중학교 수학의 여러 개념들을 복습하여 정의들을 빠짐 없이 이해하고 있는 학생들을 상정한다.


수학적 감각을 얻기 위해 사고방식을 바꾸어 나가는 작업.
수학적 상황을 올바르게 파악하고 사고하는 능력.
수학적 언어를 올바르게 독해하고 구사하는 능력.


항상 어려운 것은 문제의식의 전달이다.
거의 모든 단어가 익숙하지 않을 것이다.
집합과 같이 실제와 멀리 떨어진 추상적인 개념들을 왜 이 정도까지 자세하게 알아야 하는지 공부해 보기 전까지는 깊게 공감할 수 없다.


우리의 삶에서 쉽게 발견할 수 있는 수학의 사용처들을 활용하여 수학을 지도해야 한다는 의견은 이미 수많은 현장에서 시행착오에 의해 설득력을 잃어버린 지 오래이다.
중학생들은 용돈을 예로 들어서, 직장인들은 자산형성을 예로 들어서, 양적방법론을 사용하는 인문사회계 학생들은 회귀분석을 예로 들어서, 실험을 하는 이공계 학생들은 수치적 계산방법을 예로 들어서 수학적 개념들을 설명해야 한다고 주장한다.
단언컨대, 많은 사람들의 주장과는 달리 수학이 아닌 타분야에 대한 응용사례들은 본질적으로 수학적 문제의식과 동기를 불러 일으킬 수 없으며 또한 올바른 이해로 이어질 수도 없다. (이렇게 강하게 쓰면 안 된다.)

이 책이 다루고자 하는 수학은 철저히 교양이다.




예상독자의 모든 생소할 만한 단어는 적어도 한 문단을 투자하여 설명하며 예시 또한 곁들이고자 하였다.



\section*{제 I 편: 수학의 기초}

\begin{itemize}
\item 수학적 모델의 변수들이 살고 있는 집합을 의식하여 주어진 상황을 파악할 수 있는 눈을 기르는 것.
\item 집합의 연산과 함수의 성질 및 수체계 등 현대수학에서 사용하는 기초적인 언어를 습득하는 것.
\end{itemize}


오지선다를 통한 개념확인. 수학을 공부함에 있어서 적절치 않은 방식이지만 우리나라의 학생들의 경험을 기반으로 진입장벽을 가능한 낮추고자 하는 의도.


\section*{제 II 편: 수학의 방법}

\begin{itemize}
\item 중학수학을 현대수학의 관점에서 엄밀하게 정의하고 증명하는 것.
\item 중학수학 수준의 구체적인 문제들에 직접 적용함으로써 추상적 개념들의 이해를 높이는 것.
\item 수학적 증명의 실제상황에서 읽고 쓸 수 있는 능력을 갖추는 것.
\end{itemize}

예제와 연습문제로 구성된 2편의 내용들. 확인문제는 없다.

너무 아는 내용들만으로 구성하면 늘어질 것을 염려해 선수지식이 필요없는 새로운 이론에 대한 이야기도.


\section*{이 책을 읽는 방법}

모든 독서는 선형적이지 않다.
1편의 내용이 너무 어렵다고 느껴진면 1.1절 집합, 1.2절 함수, 3.3절 자연수, 3.4절 실수에 해당하는 확인문제와 연습문제만 본인의 것으로 만든 뒤 바로 2편으로 넘어가 풍부한 예시들을 직접 살펴보는 것을 강력하게 추천한다.
1편의 내용을 제대로 이해하는 것은 대학에서 수학을 전공으로 하고자 하는 학생들에게도 결코 쉬운 내용이 아닐 것이라 생각한다.
앞에서 이야기했듯 1편의 내용은 독자들에게 전부 이해시키는 것을 목적으로 하고 있지 않으며, 책을 읽는 중 자연스럽게 떠오를 수 있는 가능한 모든 질문들에 대한 답을 포함하여 자기완전성을 추구하고자 결과적으로 어려운 내용들이 포함되었을 뿐이다.
1편의 본문을 이해하지 못했더라도 확인문제와 연습문제가 전달하고자 하는 내용들만이라도 완전히 이해했다는 생각이 든다면 1편에서의 목표는 달성했다고 보아도 좋다고 생각한다.
따라서 1편에서 다루고자 하는 문제의식들에 익숙하지 않은 경우, 2편을 읽으며 중간중간 1편을 확인하는 방식으로 책을 읽어 나가는 것이 바람직하지 않을까 생각한다.
실제로 2편의 내용들은 이와 같은 독서순서를 염두에 두고 서술되었으며 기초적인 집합과 함수 및 수체계에 대한 내용을 제외한 1편의 내용을 전제지식으로서 직접적으로 요구하지 않을 것이다.

2편은 모든 문제들에 대해 풀이와 그 풀이를 작성하게 된 해설을 가능한 상세하게 작성하였다.
마음에 드는 문제들에 대해 증명(풀이)을 외워서 스스로 쓸 수 있게 되는 것을 목표로 하는 것이 가장 이상적일 듯하다.
여기서 증명을 외운다는 것의 의미를 오해하지 않아야 한다.
수학에서 증명을 외운다는 것은 모범적인 증명으로서 제시된 글을 높은 일치도로 복원해 낼 수 있게 연습하는 것과 완전히 다르며 오히려 정반대에 가까운 행위이다.
수학에서 증명의 암기는 주어진 명제에 대한 논리적 하자가 없는 증명을 수단과 방법을 가리지 않고 재구성해낼 수 있게 되는 것을 말한다.
이것이 가능하다면 전혀 다른 기호 혹은 논리순서를 사용했거나 교과서의 증명과 한 글자도 똑같지 않은 형태로 머리에 지식이 남아 있다고 하더라도 매우 만족스럽고 성공적인 암기이며 증명을 외우지 못한 것이라고 절대로 말할 수 없다.
복원이 아닌 재구성이 핵심이다.
보통 전공수학수업에서 강의자의 발언이나 교재의 언급을 곧이곧대로 받아적고 받아들이는 식의 공부가 이해도에 매우 치명적인 이유이기도 하다.
증명의 암기를 실현하기 위해서는 주춧돌이 되는 아이디어를 한두 개 추출하여 암기한 후 나머지 빈 부분을 그 자리에서 메워 넣을 수 있기 위한 서사구성능력, 논리력, 수학적 언어구사력, 수학적 상황파악능력, 그리고 무엇보다도 경험을 갖추는 것이 반드시 필요하다.
일반적인 이론들은 1편에 정리하였으며, 2편에서 반복적인 연습을 통해 이러한 능력들을 자연스럽게 몸에 익힐 수 있게 하고자 하였다.
예를 들어 각 문제의 증명을 올바르게 암기하기까지에 이르는 사고의 흐름에 대한 가능한 플롯을 풀이의 해설에 되도록 포함시키고자 하였다.




\tableofcontents



\part{수학의 기초}


\chapter{집합과 구조}

우리는 왜 \emph{집합}(set)을 공부해야 하는가. 
한 가지 이유는 바로 수식들에서 나타나는 각 변수가 살고 있는 집합을 의식적으로 확인하는 습관이 수학적 상황을 올바르고 효과적으로 인식하고자 할 때 대단히 중요한 역할을 하기 때문이다.
이는 증명과정이 중요하지 않고 특정 값을 계산하기 위한 도구로서만 수학을 사용하는 응용분야라 할지라도 예외가 아니다.
예를 들어 $y=f(x)$라는 수식이 있다면 $x$가 독립변수이고 $y$가 종속변수라는 말만 기계적으로 떠올리고 말 것이 아니라 $x$와 $y$가 살고 있는 집합, 즉 실수인지 정수인지 아니면 어떤 구간 안에서만 생각해야 하는 경우인지를 고려해야 한다.
특히 여기서 변수의 종류와 개수가 늘어난다거나, $x$와 $y$의 자리에 실수가 아닌 복소수나 벡터나 행렬들이 오게 된다면, 각 변수가 값으로서 가질 가능성이 있는 것들의 집합을 분명하게 인식하는 것은 매우 중요해진다.
더 나아가 만약 함수 그 자체가 변수가 된다거나, 곡면과 같은 공간을 통째로 $x$로 받아서 네트워크와 같은 복잡한 조합론적 구조가 $y$로 나오는 무시무시한 경우라면 말할 것도 없다.
각 변수에 의미를 이야기하기 전에 정확히 각 변수들이 어떠한 종류의 것을 몇 개 포함하고 있는지 이해하고 이를 언어로 전달할 수 있어야 하며, 이 경우 집합에 대한 이해는 가히 필수적이다.

우리가 집합을 공부하는 또 하나의 이유는 바로 현대수학의 실질적으로 모든 수학적 대상이 원칙적으로 집합으로 기술되며 정의되기 때문이다.
이 책에서 말하는 현대수학이란 무엇인지, 그리고 집합으로 기술되며 정의된다는 것이 무슨 뜻인지에 대해서는 추후에 이 장에서 더 자세히 설명할 것이다.
위의 이유와는 다소 다른 분위기의 이유이지만, 이는 이 장에서 다루는 내용들을 공부하는 데에 굉장히 중요하며 도움이 되는 문제의식이기도 하다.
다소 극단적으로 말해 집합만 잘 알아도 현대수학은 전부 이해할 수 있다고 해도 좋다.
현대수학의 모든 개념과 이론들을 마치 원자와도 같은 집합들을 잘 다루기 위한 단축키들을 잘 디자인하는 게임에 지나지 않은 것으로 생각해도 원리적으로는 크게 틀리지 않다.
모든 현대수학은 집합론으로 환원될 수 있다고까지 말해도 과언은 아니다.

하지만 이는 컴퓨터가 수학을 할 때에 굉장히 효율적인 관점일지 몰라도 다양한 종류의 직관이라는 능력을 갖춘 인간에게는 매우 비효율적인 관점이다.
예를 들어 이러한 환원주의적 관점에서 수학을 연구할 때 컴퓨터를 이용한다면 절대적으로 참인 명제들을 엄청난 속도로 무한히 생성해낼 수 있을 것이다.
그러나 이런 방식으로 절대적 진리의 지평선을 무한정 확장시켜 나간다 한들, 무작위적 기호의 나열에 불과한, 그러나 절대적으로 참인, 그런 명제들이 과연 인간에게 유의미한 진리인가에 대해서는 그렇다고 할 수 없을 것이다.

예를 들어 함수를 생각해보자.
우리가 의무교육과정에서 배운 함수는 어떤 것을 입력했을 때 어떤 것이 출력되는 일종의 규칙 혹은 장치이지, 보통 집합이라고 생각되지는 않는다.
그러나 함수 또한 엄연히 수학적 대상이며, 따라서 모든 수학적 대상이 집합으로서 정의된다고 했던 의미에서는 함수가 특정조건을 만족하는 집합으로서 정의되어야 하고 실제로 그렇게 정의된다.
그렇다고 해서 우리가 함수를 생각할 때 이것을 추상적으로 조건지어진 집합이라고만 여기게 되면, 어떤 것을 입력했을 때 어떤 것이 출력된다는 함수의 직관과 의미를 잃게 될 것이다.

우리는 더 나아가 여러 수학적 대상들에 적절한 의미들을 효율적으로 부여하기 위하여, 관심이 있는 수학적 대상을 집합 그 자체만으로서 인식하고 이해하지 않는 대신, 주어진 집합 위에 다양하고 흥미로운 개념들을 추가적으로 정의하려고 할 때 필요한 독특한 형태의 정보, 즉 \emph{구조}(structure)라고 불리는 특별한 기능을 수행하는 수학적 대상이 본래의 집합 위에 올려져 있다고 생각한다.
변수 $x$나 $y$가 살고 있는 집합을 단순한 집합으로서만 인식하지 않고 구조라는 것들을 추가해야 비로소 변수들에 의미를 부여할 수 있는 길이 열린다는 것이다.

예를 들어 우리가 수라고 부르는 수학적 대상들에 대해 떠올려보자.
자연수, 정수, 유리수, 실수, 복소수 등이 바로 그것이다.
이런 수들을 모아놓은 것은 단순히 집합임을 넘어서 일반적으로 수들 간 덧셈이나 곱셈과 같은 연산을 허용한다.
또한 여기서 복소수를 제외한다면 수들 간의 순서, 즉 크기 비교 또한 가능하다.
연산이 정의되지 않아 덧셈조차 불가능한 단순한 집합을 수들의 집합이라고 부르기에는 다소 무리가 있을 것이다.
아무런 조건도 없이 집합을 하나 덩그러니 생각했을 때, 그리고 이 집합의 원소들을 특별한 종류의 수라고 의미를 붙여주고 싶을 때, 이때 이 수들의 덧셈이라는 것은 과연 무엇이며 순서라는 것은 과연 무엇인가.
바로 이 덧셈이라는 연산과 순서라는 관계가 수학적 구조의 예시들이 된다.
이렇게 한 집합 위에 구조라고 불리는 특별한 형태의 또다른 수학적 대상을 장착시킴으로써 수학적 대상들을 이해하는 관점이 현대수학의 일반적 관점이다.
위에서 함수를 생각했을 때와 같이 결국 덧셈이나 순서라는 구조들 또한 수학적 대상이기 때문에 환원주의적으로는 집합이어야겠지만, 이들을 집합으로 인식해서 딱히 이득을 볼 일이 없는 경우가 많기 때문에 단순한 집합과는 분리하여 구조라는 개념으로 이해하게 될 것이다.

이 장에서는 먼저 수학적으로는 완전히 엄밀하지는 않은 수준에서 현대수학에서 집합을 어떻게 인식하고 다루는지에 대해 먼저 살펴볼 것이다.
엄밀하지 않게 내용을 다루는 이유는 아직 현대수학에서 작동하는 논리에 관하여 논의하지 않았기 때문에 엄격한 엄밀성을 지키며 집합을 논하기에 어려움이 있기 때문이다.
따라서 수학적 논리에 대해 다루는 2장을 읽고 다시 1장을 읽었을 때 새로 보이는 것들이 많을 것이라 생각한다.
그 다음 절에서는 \emph{함수}(function)에 대해 공부한다.
함수는 여러 집합 사이를 이어주는 수단임과 동시에 다양한 수학적 구조들을 기술하기 위한 효과적인 도구이며, 집합과 함께 수학적 언어들을 구성하는 기본요소 역할을 할 것이다.
구체적으로는 함수가 집합으로서 엄밀하게 어떻게 정의되는지 알아보고 단사성과 전사성 등 이 책에서 앞으로 계속 언급될 성질들에 대해 공부할 것이다.
그리고 마지막으로 1.3절과 1.4절에서는 수학적 구조의 가장 기초적인 예시들인 \emph{관계}(relation)와 \emph{연산}(operation)에 관한 여러 정의와 성질들에 대해 이해하는 것을 목표로 할 것이다.

\section{집합}

우리는 의무교육과정에서 집합의 포함관계와 집합의 연산 등에 대하여 다루었다.
조건제시법과 원소나열법, 공집합과 부분집합, 합집합과 교집합 등이 바로 그것이다.
그런데 우리가 학습한 집합론만으로는 엄밀한 수학의 기초로서 충분한 역할을 할 수 없다는 문제가 이미 오래 전부터 제기되어 알려져 있다.
이는 여러 수학적 대상들을 모아 놓은 모든 종류의 추상적 관념을 아무런 제약 없이 집합이라는 수학적 개념으로 간주할 때 논리적 모순이 발생할 수 있기 때문이다.
예를 들어, 집합 $R$을
\[R:=\{x:\text{$x$는 $x\notin x$를 만족하는 집합}\}\]
와 같이 정의할 수 있다고 가정해 보자.
다시 말해, 자기 자신을 원소로 갖지 않는 모든 집합은 $R$의 원소가 되며, $R$의 모든 원소는 자기 자신을 원소로 가져야 한다.
이때 만약 $R$이 $R\in R$을 만족한다면 $R$은 $R$의 원소이므로 $R\notin R$이 성립해야 하며, 만약 $R\notin R$을 만족한다면 역시 $R$의 정의에 의해 $R$이 $R$의 원소, 즉 $R\in R$을 만족해야 한다.
이 모순을 \emph{Russell의 역설}(Russell's paradox)이라고 한다.
이와 같은 종류의 모순은 주로 ``모든 집합들의 집합''처럼 안이하게 정의된 집합들, 조금 더 자세히 말하자면 전체집합이라고 부를 만한 집합을 하나 고정하여 그 안에서 기술하는 것이 불가능한 ``큰 집합''들에 대하여 나타나는 경향이 있다.
이를 해결하는 일반적인 방법은, 수학적 이론을 전개하는 데에 지장이 없을 정도로 상식적인 집합의 조작들과 연산들이 가능하게끔 한 채, 매우 엄격한 기준을 통해 엄밀한 의미의 집합을 좁게 정의함으로써 $R$과 같은 대상들을 집합의 정의에서 배제하는 것이다.

우리는 \emph{집합}(set)을 아래에 제시될 9가지의 공리를 만족하는 어떠한 것, 즉 무정의용어를 가리키는 말로 사용할 것이다.
다르게 말하여, 9가지의 공리로부터 출발하는 형식적 이론의 대상으로 정의한다고도 할 수 있을 것이다.
집합이 9가지 공리를 만족한다는 약속을 통하여 이 모든 공리들은 항상 참인 명제로 받아들여질 것이다.
이처럼 집합의 정의를 공리적 접근으로써 제시하는 이론을, 우리가 의무교육과정에서 배운 엄밀하지 못한 집합론, 즉 \emph{소박한 집합론}(naive set theory)과 구별하여 \emph{공리적 집합론}(axiomatic set theory)이라고 한다.
이 9가지 공리에 의해 결정되는 집합론은 현대수학에서 가장 보편적으로 사용되고 대부분의 수학분야에서 표준으로 인정되고 있는 것으로서, \emph{선택공리를 포함한 }\textbf{Zermelo-Fraenkel}\emph{ 집합론}(Zermelo-Fraenkel set theory with choice), 줄여서 ZFC라고 불리지만, 여러 공리를 제거하거나 추가함으로써 다른 공리적 집합론을 생각하는 것도 가능하다.

조금 더 구체적으로, 공리적 집합론에서는 집합의 정의를 엄밀하게 만들기 위하여 \emph{상등}(equality)과 \emph{소속}(membership)이라는 두 가지의 관계만을 기본개념으로 가정하여 집합이 만족해야 할 공리들을 논한다.
두 집합 $x$와 $y$에 대하여, 상등관계는 두 집합이 완전히 같은지를 말하는 것으로 기호로 $x=y$와 같이 쓰며, 소속관계는 $x$가 $y$에 속해 있는지, 즉 $x$가 $y$의 \emph{원소}(element, member)인지를 말하는 것으로 기호로 $x\in y$와 같이 쓴다.
집합론의 공리들은 집합 간 소속관계가 만족해야 하는 조건들을 기술하며, 상등관계의 정의는 엄밀히 말하여 집합론의 공리에 의해 정의되는 개념이라기보다는 논리체계의 설정에서부터 비롯된다.
수학에서 집합론적으로 같다라는 표현은 가장 엄격한 차원에서 어떠한 기준으로도 완전히 동일함을 의미한다.
예를 들어 두 집합 $x$와 $y$가 서로 같고 집합 $z$가 $x$의 원소일 때 $z$가 $y$의 원소이기도 한 것은, 집합론의 공리가 아니라 우리가 사용할 형식적 논리체계의 공리로서 받아들여지는 것이다.
이렇게 민감하고 추상적인 논의들의 난해함 때문에 원래는 공리적 집합론을 일차술어논리 하의 엄밀한 형식언어를 사용하여 기술하는 것이 바람직하지만, 집합론 자체에 대한 수리논리학적 이론을 깊게 이해하기보다는 현대수학에서의 집합론의 기능과 경계를 직관적으로 전달하여 앞으로의 수학공부에 도움이 되고자 하는 의도에 맞추어, 형식언어에 관한 용어들의 정의 없이 자연언어로 공리들을 기술하고자 한다.
여기서 강조할 것은, 우리의 목표가 9가지 공리들이 각각 어떤 의미를 가지며 어떤 역할을 수행하는지 깊게 이해하는 것이 아니라, 함수와 수체계처럼 우리에게 익숙한 수학적 개념들이 사실은 이러한 공리들로부터 출발하여 구성되는 것임을 감각적으로 받아들이는 것이다.


\begin{axiom*}[ZFC 공리계]
무정의 용어는 집합, 소속관계.
\emph{선택공리를 포함한 }\textbf{Zermelo-Fraenkel}\emph{ 집합론}(Zermelo-Fraenkel set theory with choice)이란 다음 9가지의 공리 및 공리꼴로 구성되는 이론이다.

\begin{itemize}
\item 집합의 기본성질에 관한 공리군
\begin{enumerate}
\item (확장공리, extensionality)
집합 $x$와 $y$에 대하여, 만약 $x$의 모든 원소가 $y$의 원소이고 $y$의 모든 원소가 $x$의 원소이면, $x$와 $y$는 같은 집합이다.
\item (정칙공리, regularity)
집합 $x$에 대하여, $x$가 원소를 가진다면, $x$와 공통된 원소를 가지지 않는 $x$의 원소가 존재한다.
\end{enumerate}
\item 집합의 구성방법에 관한 공리군
\begin{enumerate}
\setcounter{enumi}{2}
\item (함축공리꼴, comprehension)
집합 $x$와 한 개의 집합에 관한 조건 $p$에 대하여, 집합 $z$가 $y$의 원소인 것과 $z$가 $x$의 원소이고 $p(z)$가 참인 것이 동치인 집합 $y$가 존재한다.
\item (치환공리꼴, replacement)
집합 $x$와 두 개의 집합에 관한 조건 $p$에 대하여, $x$의 각 원소 $z$에 대하여 $p(z,w)$가 참인 집합 $w$가 유일하게 존재할 때, 이러한 $w$들을 모두 원소로 갖는 집합 $y$가 존재한다.
\item (짝공리, pair)
집합 $x$와 $y$에 대하여, $x$와 $y$를 원소로 갖는 집합 $z$가 존재한다.
\item (합집합공리, union)
집합 $x$에 대하여, $x$의 원소의 원소가 $y$의 원소인 집합 $y$가 존재한다.
\item (멱집합공리, power set)
집합 $x$에 대하여, 집합 $z$의 모든 원소가 $x$의 원소일 때 $z$가 $y$의 원소인 집합 $y$가 존재한다.
\end{enumerate}
\item 무한집합의 존재에 관한 공리군
\begin{enumerate}
\setcounter{enumi}{7}
\item (무한공리, infinity)
원소가 존재하지 않는 집합을 원소로 가지며, 집합 $y$가 원소이면 $y$와 $y$의 원소들만을 원소로 갖는 집합도 원소인 집합 $x$가 존재한다.
\item (선택공리, choice)
집합 $x$에 대하여, 임의로 고른 $x$의 서로 다른 두 원소가 공통된 원소를 가지지 않을 때, 임의의 $x$의 원소와 단 하나의 공통된 원소를 가지는 집합 $y$가 존재한다.
\end{enumerate}
\end{itemize}
\end{axiom*}

특별히 세 번째와 네 번째 공리를 공리꼴이라고 부르는 것은 공리 안에 등장하는 조건 $p$를 다르게 설정할 때마다 다른 문장이 되므로 실질적으로 공리로서 집합들이 만족해야 하는 문장이 무한 개가 되기 때문이다.
역사적으로 Zermelo에 의해 처음 공리적인 집합론의 체계가 제안되었을 때 정칙공리, 치환공리꼴, 선택공리는 포함되지 않았다.
이 중 정칙공리와 치환공리꼴은 공리적 집합론의 관점에서 수학적 기초를 구성할 때 중요한 역할을 하지만 일반적으로 현대수학에서 다루는 수학적 정의와 증명들에는 사용되지 않는다.

소박한 집합론에서 다루었던 집합에 관한 개념들을 공리적 집합론의 관점에서 상등과 소속관계만을 가지고 다시 한 번 정의하고 넘어가고자 한다.
우리는 여기서 다음 개념들의 뜻을 사전적으로 정의할 뿐 각 개념들에 대하여 존재성과 유일성을 증명하지 않을 것이다.
따라서 엄밀한 관점에서는 아직 이 개념들을 자유롭게 사용해서는 안 되고 이 존재성과 유일성에 관한 주장들은 9가지의 공리로부터 증명되어야 하는 정리로서 인식되어야 한다.
그러나 이들의 증명은 우리의 목표와 다소 거리가 있기 때문에 이 개념들의 존재성 및 유일성 등은 이미 증명된 것이라고 가정하여 앞으로의 논의들을 진행할 때 자유롭게 사용할 것이다.

\begin{definition}[공집합]
집합 $x$에 대하여, $x$가 \emph{공집합}(empty set)이라는 것은, 모든 집합 $y$에 대하여 $y\notin x$인 것이다.
확장공리에 따르면 공집합인 집합은 유일하며 함축공리꼴에 따르면 공집합인 집합은 존재한다.
이 집합을 공집합이라고 하며 기호로 $\varnothing$와 같이 쓴다.
\end{definition}
\begin{definition}[부분집합]
집합 $x$와 $y$에 대하여, $x$가 $y$의 \emph{부분집합}(subset)이라는 것은, 모든 집합 $z$에 대하여 $z\in x$이면 $z\in y$인 것이며, 기호로 $x\subset y$와 같이 쓴다.
\end{definition}
\begin{definition}[조건제시법]
함축공리꼴에 따르면 $\{z\in x:p(z)\}$
\end{definition}
\begin{definition}[원소나열법]
유한 개의 서로 다른 집합 $x_1,\cdots,x_n$에 대하여, 임의의 집합 $y$에 대하여 $y\in x$와 어떤 자연수 $1\le i\le n$이 존재하여 $y=x_i$인 것이 서로 동치인 집합 $x$가 유일하게 존재한다면, 집합 $x$를 기호로 $\{x_1,\cdots,x_n\}$와 같이 쓴다.
\end{definition}
\begin{definition}[곱집합]
두 집합 $x$와 $y$에 대하여, 집합 $\{\{x\},\{x,y\}\}$를 $x$와 $y$의 \emph{순서쌍}(ordered pair)이라고 부르고 기호로 $(x,y)$와 같이 쓰며, $x$의 원소와 $y$의 원소로 이루어진 순서쌍들의 집합 $\{(z,w):\text{$z\in x$이고 $w\in y$}\}$를 $x$와 $y$의 \emph{곱집합}(product set)이라고 부르고 기호로 $x\times y$와 같이 쓴다.
\end{definition}
\begin{definition}[집합의 연산]
두 집합 $x$와 $y$에 대하여, 

만약 전체집합이라고 부를 만한 집합 $u$가 문맥상 혼동의 여지 없이 주어져 있을 경우, $u$의 부분집합 $x$에 대하여 $x$의 원소가 아닌 $u$의 원소들의 집합 $\{z\in u:z\notin x\}$을 $x$의 \emph{여집합}(complement)이라고 부르고 기호로 $x^c$와 같이 쓴다.
즉, 여집합 $x^c$는 $x$에도 $u$에도 의존하며, 이 혼동을 줄이고자 차집합 $u\setminus x$를 통해 표현하는 경우도 많다.
\end{definition}





\section{함수}

\begin{definition}[함수]
집합 $X$와 $Y$에 대하여, $X$에서 $Y$로 가는 \emph{함수}(function)란, 모든 $x\in X$에 대하여 $(x,y)\in f$를 만족하는 $y\in Y$가 유일하게 존재하는 $X\times Y$의 부분집합 $f$이며, 이때 $f$가 함수라는 것을 기호로 $f:X\to Y$와 같이 쓰고, 원소 $x\in X$에 대하여 $(x,y)\in f$를 만족하는 유일한 $y\in Y$를 기호로 $f(x)$와 같이 쓴다.
함수 $f:X\to Y$에 대하여, 집합 $X$를 $f$의 \emph{정의역}(domain), 집합 $Y$를 $f$의 \emph{공역}(codomain)이라고 하며, 부분집합 $f\subset X\times Y$를 특히 함수 $f$의 \emph{그래프}(graph)라고 부른다.
\end{definition}

전사 단사 전단사
상과 역상

합성과 결합법칙



\section{관계}

\begin{definition}[관계]

\end{definition}

모든 집합들의 모임은 Russel의 역설에 의해 수학적 의미의 집합이 될 수 없으므로 공리적 집합론에서 등장한 $=$, $\in$, $\subset$이라는 관계들은 전체집합을 설정하지 않는 한 방금 정의한 집합 위의 관계라는 개념으로는 이해할 수 없다.


순서관계

동치관계
분할



\section{연산}


반군 모노이드 군 아벨군,
군환체의 정의



\twocolumn
\section*{확인문제}

\subsection*{집합}
\begin{problem}
다음 중 옳지 않은 것을 고르시오.
\begin{enumerate}
\item[①] $\varnothing\cap\{\varnothing\}\subset\varnothing$
\item[②] $\{\varnothing,\{\varnothing\}\}\cap\{\varnothing\}=\varnothing$
\item[③]
\item[④]
\item[⑤]
\end{enumerate}
\end{problem}

\begin{problem}
다음 중 옳지 않은 것을 고르시오.
\begin{enumerate}
\item[①]
\item[②]
\item[③]
\item[④]
\item[⑤]
\end{enumerate}
\end{problem}

\subsection*{함수}

\subsection*{관계}
\subsection*{연산}



\onecolumn
\section*{연습문제}
\setcounter{problem}{0}

\subsection*{집합}
\begin{problem}[대칭차집합]
집합 $A,B$에 대하여, 집합
\[A\mathrel{\triangle}B:=(A\cup B)\setminus(A\cap B)\]
를 $A$와 $B$의 \emph{대칭차집합}(symmetric difference)라고 한다.
집합 $A,B,C$, 그리고 $D$를 생각하자.
\begin{parts}
\item $(A\mathrel{\triangle}B)\mathrel{\triangle}C=A\mathrel{\triangle}(B\mathrel{\triangle}C)$가 성립함을 보여라.
\item $(A\mathrel{\triangle}B)\cap C=(A\cap C)\mathrel{\triangle}(B\cap C)$가 성립함을 보여라.
\item $(A\mathrel{\triangle}B)\cup(C\mathrel{\triangle}D)\supset(A\cup C)\mathrel{\triangle}(B\cup D)$가 성립함을 보여라.
\item $(A\mathrel{\triangle}B)\cup(C\mathrel{\triangle}D)\subset(A\cup C)\mathrel{\triangle}(B\cup D)$가 일반적으로 성립하지 않음을 보여라.
\end{parts}
\end{problem}
\begin{proof}[풀이]
\end{proof}

\begin{problem}[무한 곱집합]
수열들의 집합
\[c_c,\qquad\ell^\infty,\qquad\R^\infty\]
\end{problem}
\begin{proof}[풀이]
\end{proof}


\subsection*{함수}
\subsection*{관계}
\subsection*{연산}









\chapter{술어논리}

이 장에서는 참이란 무엇인가, 증명이란 무엇인가, 명제 또는 문장이란 무엇인가, 조건이란 무엇인가, 그리고 변수란 무엇인가와 같은 굉장히 추상적인 질문들에 답한다.
이 질문들은 여러 상황에서 등장하는 수식들 속에 복잡하게 얽혀 있는 각 변수들의 역할과 관계를 분명히 파악하고, 이로부터 파생되는 수학적 표현들을 틀리지 않고 자유자재로 구사할 수 있게 되기 위해서 반드시 답해야 하는 중요한 질문들이다.

아쉽게도 일반적으로 우리는 의무교육과정 동안 이 질문들에 대한 답을 직접적으로 배우지 못한다.
다양한 수학문제들을 스스로 해결하는 경험을 반복하고 수학적 언어능력을 감각적으로 체득함으로써 간접적으로 답에 대한 이미지를 느낄 뿐이다.
아무리 교육과정을 충실히 따른 학생들이라도 따로 공부를 하지 않은 이상 참과 증명, 명제와 조건, 그리고 변수의 정의에 대해서 명확한 대답을 하는 것은 거의 불가능에 가까울 것이라 생각한다.
다만 학생들이 직감적으로 이 추상적인 개념들을 어떻게 활용하고 조작해야 하는지를 알게 하는, 어떤 의미로는 수학에 대한 원어민과 같은 능력을 갖추게 하는 것이 의무교육과정의 최종적인 목표라고 한다면 이 질문들에 답을 하지 못하는 것이 그렇게 이상한 일은 아닐 것이다.
이 목표를 향해 단기간에 압축시켜 훈련하고자 하는 것이 이 책 2편의 목표이기도 하다.

이 장에서 우리는 원어민보다 언어학자가 되기를 택하여 이 질문들에 대해 마치 문법을 알고리즘처럼 풀어내듯 완전한 정답을 제시하고자 한다.
이 질문들에 충분히 답을 할 수 있는 수준에 이른다면 우리는 수식들을 단순히 추상적인 기호들의 나열이나 구하고자 하는 값을 계산하기 위한 수단에 불과한 것이 아니라는 사실을 직시하게 될 것이다.
사소해 보이는 수식들 하나하나도 명확하게 주장하고자 하는 바가 있는 표현들이라는 것과, 그 주장하고자 하는 바를 어떻게 해석해내야 하는지를 분명하게 이해할 수 있게 될 것이다.
그리고 그 어떤 수식을 보더라도 그 안에 어느 변수가 역동적으로 살아 움직이고 있으며 어느 변수가 배경처럼 묵묵히 자리를 지키고 있는지가 한 눈에 들어오며 다양한 수식의 조작들에 자신감이 붙는 놀라운 경험을 할 수 있을 것이라고 기대한다.
그러나 이정표를 확실하게 의식하고 나아가지 못한다면 다소 어려운 길이 될 것이라고 생각한다.
길을 잃지 않을 자신이 부족하다면 4장을 먼저 읽어 이 장에서 이야기하고자 하는 목적의식을 직접 느끼고 오는 것도 매우 좋은 방법일 수 있다.
이 장의 내용은 언어의 사용법 자체보다 문법에 대한 이론서의 역할을 할 것이기 때문이다.

우리는 먼저 논리 그 자체를 수학의 언어로 풀어낼 것이다.
이를 위하여 \emph{논리체계}(logical system)를 하나 고정해야 한다.
하나의 논리체계는 비형식적인 의미에서 하나의 \emph{통사론}(syntax), 하나의 \emph{의미론}(semantics), 그리고 하나의 \emph{연역체계}(deductive system)를 갖는다.
여러 논리체계들 중 현대수학에서 가장 보편적으로 사용되는 것이 \emph{술어논리}(predicate logic) 혹은 \emph{일차논리}(first-order logic)이라고 불리는 논리체계이다.




\section{언어}



형식언어와 명제논리의 통사론
부호수와 일차논리의 통사론


술어는 \emph{관계}(relation) 또는 \emph{조건}(condition)이라고도 하며, 이 책에서는 조건이라는 단어를 즐겨 사용할 것이다.


우리나라의 교육과정에서 한 개의 항수를 가지는 두 조건 $p$와 $q$에 대하여 명제 $\forall x(p(x)\to q(x))$를 $p\to q$와 같이 쓰도록 지도하고 있으나, 이는 사실 바람직하지 않은 표기법이다.
예를 들어 한 개의 항수를 가지는 조건 $r$을 하나 더 생각할 때, 기호들의 나열 $(p\to q)\wedge(q\to r)$을 서로 다른 두 명제
\[\forall x((p(x)\to r(x))\wedge(q(x)\to r(x))),\qquad\forall x(p(x)\to r(x))\wedge\forall y(q(y)\to r(y))\]
중 어떤 것으로 해석해야 할지 분명하지 않기 때문이다.


\section{모형}
모형의 정의
\section{증명}
증명의 정의, 추론 규칙, 비형식적 증명에 대한 첨언


비형식적 증명에 관하여:
수업에서 다루지 않은 모든 기호와 개념은 앞서 정의될 것,
모든 문장은 주어와 서술어를 가질 것,
식은 체언이 될 수도 서술어가 될 수도 있음을 알아둘 것,
논리에 의한 접속사를 빠뜨리지 말 것,
모든 변수는 양화가 언급될 것,
기호로 문장을 시작하지 말 것,
Def, Thm, Prop, Lem, Cor, Rmk 등의 설명

\section{불완전성}



\twocolumn
\section*{확인문제}

\subsection*{언어}
\begin{problem}
다음 중 옳지 않은 것을 고르시오.
\begin{enumerate}
\item[①] 공집합 $A$에 대해 ``집합 $B$에 대해 $A$는 $B$의 부분집합이다.''는 참인 명제이다.
\item[②] 공집합 $A$와 집합 $B$에 대해 ``$A$는 $B$의 부분집합이다.''는 $B$에 대한 조건이다.
\item[③] 집합 $A$에 대해 ``어떤 집합 $B$에 대해 $A$는 $B$의 부분집합이다.''는 참인 명제이다.
\item[④] 집합 $A$에 대해 ``어떤 집합 $B$에 대해 $A$는 $B$의 부분집합이다.''는 $A$에 대한 조건이다.
\item[⑤] 공집합 $A$에 대해 ``어떤 집합 $B$에 대해 $A$는 $B$의 부분집합이다.''는 $B$에 대한 조건이 아니다.
\end{enumerate}
\end{problem}
\subsection*{모형}
\subsection*{증명}
\subsection*{불완전성}





답:\\
2.1. ③


\onecolumn
\section*{연습문제}

\subsection*{언어}
\subsection*{모형}
\subsection*{증명}
\subsection*{불완전성}



\chapter{수체계}
\section{서수}
정렬집합
\section{기수}


\section{자연수}
존재성과 유일성,
정수와 유리수
\section{실수}
순서와 체는 이미 정의함,
존재성과 유일성,
복소수와 사원수








\part{수학의 방법}

\chapter{대수의 방법}




\section{방정식과 부등식}

방정식과 부등식을 푼다는 것은 무슨 뜻인가.


\begin{example}
실수 $x$에 대한 방정식 $|x^2-1|=x+t$가 두 개의 실근을 갖는 실수 $t$의 값을 구하여라.
\end{example}
\begin{linenumbers*}
\begin{proof}[풀이]

\end{proof}
\end{linenumbers*}
\begin{proof}[해설]

특별히 문제에서 값을 아무것이나 하나만 구하라는 언급이 없는 경우, 우리는 가능한 값을 빠짐 없이 전부 구하여야 한다.
만약 없다면 없다고 풀이에 작성하여야 정답으로 인정받을 수 있다.



단, 증명이 쉽다고 해서 이 단계를 생략해서는 절대 안 된다는 것을 강조하고자 한다.
답으로 구한 값들이 실제로 조건을 만족한다는 것을 증명하는 과정에 대해서, 누구나 쉽게 확인할 수 있는 것을 굳이 길게 풀어쓰는 것이 오히려 증명의 가독성을 과도하게 떨어뜨릴 정도라고 하더라도, 쉬우면 쉬운 대로 쉽다고 반드시 언급하고 넘어가야 한다.
이 단계를 생략해도 된다는 뜻은, 곧 명제
\[\text{``실수 $t$에 대하여, 실수 $x$에 대한 방정식 $|x^2-1|=x+t$가 두 개의 실근을 갖는다면, $t$는 실수이다.''}\]
가 참이니까 답이 모든 실수라는 결론을 내려도 된다는 뜻이 되어 버리기 때문이다.
따라서 논리적으로 올바른 증명에는 반드시 구한 답이 문제의 조건을 만족한다는 것을 확인하는 역과정이 포함되어야 한다.

가정이 거짓이면 결론이 거짓이어도 명제는 참이 된다.




증명의 논리에 하자가 없다는 가정 하에, 실제로 증명을 작성할 때 어떤 부분을 생략해도 될지 안 될지 판단이 잘 안 서는 경우에는 항상 다음과 같은 대원칙을 생각하면 된다.
\[\tab{``읽는 사람이 조금이라도 헷갈릴 것 같으면 귀찮아도 다 써주고,\\\quad
읽는 사람이 조금이라도 안 헷갈릴 것 같으면 과감히 생략하자.''}\]
\end{proof}

\begin{example}
양의 실수 $x,y$가 $x+y=x^3+y^3=x^5+y^5$를 만족할 때 $x+y$의 값을 구하여라.
\end{example}
\begin{linenumbers*}
\begin{proof}[풀이]
조건 $x+y=x^3+y^3$으로부터 $(x-1)x(x+1)+(y-1)y(y+1)=0$을, 조건 $x+y=x^5+y^5$로부터 $(x-1)x(x+1)(x^2+1)+(y-1)y(y+1)(y^2+1)=0$을 얻는다.
이때
\begin{align*}
0&=(x-1)x(x+1)(x^2+1)+(y-1)y(y+1)(y^2+1)\\
&=(x-1)x(x+1)(x^2+1)-(x-1)x(x+1)(y^2+1)\\
&=(x-1)x(x+1)(x-y)(x+y)
\end{align*}
이고, 세 조건 $x=0$, $x+1=0$, $x+y=0$은 $x,y$가 양의 실수이기에 불가능하므로, 우리는 $x-1=0$ 또는 $x-y=0$을 얻는다.
첫 번째 경우는 $1+y=1+y^3$로부터 $y=1$이므로 $x+y=2$이고, 두 번째 경우는 $2x=2x^3$로부터 $x=1$이므로 $x+y=2x=2$이다.
따라서 $x+y=2$가 성립한다.

실제로 $x=y=1$라고 두면 조건 $x+y=x^3+y^3=x^5+y^5$을 만족하고 $x+y=2$이므로 답은 $2$이다.
\end{proof}
\end{linenumbers*}
\begin{proof}[해설]
풀이에 대한 해설에 들어가기 전, 문제의 상황을 올바르게 파악해보자.
만약에 시간이 촉박한 수능에서 ① $1$부터 ⑤ $5$까지의 오지선다로 이 문제가 나왔다고 가정해보자.
여기서 $x$와 $y$에 $1$을 대입하였을 때 조건이 성립한다는 것을 쉽게 확인할 수 있으므로, 많은 수험생들은 답으로 ②를 고르고 얼른 다음 문제로 넘어가는 것이 현명할 것이다.
다시 말해 위의 풀이에서 아홉 번째 줄만을 생각해야 시간적으로 유리할 것이다.
그러나 이 논리는 수학적으로 완전히 틀린 논리이기 때문에 서술형 문제의 답안으로 제출했을 경우 큰 감점을 피할 수 없다.
왜냐하면 이 문제의 답이 $2$라는 것을 주장하기 위해서는 명제
\[\text{``양의 실수 $x,y$가 $x+y=x^3+y^3=x^5+y^5$를 만족할 때 $x+y=2$이다.''}\]
가 참이라는 사실을 반드시 증명해야 하는데, 이를 충분히 증명하지 못했기 때문이다.
양의 실수 $x,y$가 주어진 조건 $x+y=x^3+y^3=x^5+y^5$을 만족함에도 불구하고 만에 하나 $x=1$이나 $y=1$이 성립하지 않을 가능성이 있다면, 우리는 $x+y=2$가 성립한다고 확실히 보장할 수 없게 된다.
따라서 $x=1$과 $y=1$을 대입하는 방법을 통해서 이 문제에 대한 유효한 증명을 완성하고자 한다면, 명제
\[\text{``양의 실수 $x,y$가 $x+y=x^3+y^3=x^5+y^5$를 만족할 때 $x=1$이고 $y=1$이다.''}\]
가 반드시 증명되어야 하며, 이를 통해 조건 $x=1$이나 $y=1$가 깨지게 될 가능성을 완전하게 배제해야 한다.
이것이 첫 번째 줄부터 여덟 번째 줄까지가 풀이에 포함되어야 하는 이유이다.

그다음 우리는 이렇게 답으로 제시한 값 $2$가 실제로 문제의 조건을 만족함을 증명하여야 한다.
이 과정이 왜 필요한가에 대해서는 앞의 예제에서 논의하였다.
그런데 이 문제에서 우리가 구해야 하는 것은 $x+y$의 값이지만 문제의 조건은 $x+y$에 대한 조건이 아닌 $x$와 $y$에 대한 조건이므로 구체적으로 무엇을 보여야 하는지 알기 어려울 수 있다.
여기서 $x+y$를 의미할 새로운 변수 $t$를 도입하여 생각해보자.
여기서 변수 $t$는 실수라고 해도, 양의 실수라고 해도, 딱히 상관은 없다.
문제상황에서 우리의 목적을 다시 상기하자.
우리는 양의 실수 $x,y$가 $x+y=x^3+y^3=x^5+y^5$를 만족할 때, $x+y$가 될 수 있는 값 $t$를 모두 구하는 것이 목적이다.
여기서 될 `수' 있는 값이라는 것에 주목하자.
그러면 양화사 `어떤'을 $x$와 $y$에 적용해 실수 $t$에 대한 조건
\[\text{``어떤 양의 실수 $x,y$가 존재하여 $x+y=x^3+y^3=x^5+y^5$를 만족하고 $x+y=t$이다.''}\]
를 생각하는 것이 타당하다.
따라서 이 조건에 우리의 답인 $t=2$를 대입한 명제를 증명하면, 답으로 제시한 값 $2$가 실제로 문제의 조건을 만족한다는 것을 보이는 것이 된다.
여기서 $t=2$를 대입하여 만들어진 명제는 $x,y$의 존재성을 주장하는 명제이므로 아홉 번째 줄에서 보이듯 $x,y$의 값을 직접 지정해줌으로써 간단하게 확인할 수 있다.


이제 이 풀이의 서술방식에 대하여 구체적으로 뜯어보도록 하자.
등호나 덧셈기호와 같이 표준으로 인정되는 기호를 제외한, 문제 상황에 의존하게 되는 모든 기호는 처음 등장할 때 반드시 정의되어야 한다는 원칙을 떠올리자.
그러나 첫 번째 줄에서 등장하는 변수 $x,y$는 이미 문제의 진술에서 정의되었기 때문에 이 $x,y$가 무엇인지, 구체적으로 이 변수가 속하는 집합이 무엇인지, 따로 언급하지 않아도 혼동의 여지가 없으므로 괜찮다.
즉, 한쪽 방향
\[\text{``양의 실수 $x,y$가 $x+y=x^3+y^3=x^5+y^5$를 만족할 때 $x+y=2$이다.''}\]
를 증명할 때에 한해서, 마치 $x$와 $y$가 상수인 것처럼 서술해도 큰 문제의 소지가 없다.
물론 $x,y$가 $x+y=x^3+y^3=x^5+y^5$를 만족하는 실수라는 것을 증명 안에서 다시 한 번 언급해주면 훨씬 분명해진다.

첫 번째 줄에서 $x+y=x^3+y^3$로부터 $(x-1)x(x+1)+(y-1)y(y+1)=0$를 얻었다는 표현을 보자.
이는 이항과 곱셈공식을 이용한 인수분해이다.
다섯 번째 줄에서 여섯 번째 줄로 갈 때에는 명제
\[\text{``실수 $x,y$에 대해 $xy=0$이면 $x=0$ 또는 $y=0$이다.''}\]
가 네 번 사용된다.
계산을 적절히 생략하는 것은 단순한 논리비약으로 생각되지 않기 때문에, 증명을 읽는 독자들의 지식 수준을 고려하여 증명의 아이디어가 한 눈에 들어올 수 있도록 하기 위해 충분히 허용될 수 있다.
예를 들어 곱셈공식에 의한 인수분해를 직접적으로 언급하지 않거나 구체적인 계산과정과 논리를 풀어쓰지 않은 것은 이 책이 중학수학에 익숙한 독자들을 상정하고 있기 때문이다.
그렇다고 해서 독자들의 입장에서 각 행간이 바로 와닿지 않을 정도의 너무 과한 생략을 하게 된다면, 이해하기 쉬운 증명을 작성해야 한다는 원칙에 어긋남과 동시에 논리비약으로 빠질 위험이 있으므로 주의해야 한다.

세 번째 줄부터 다섯 번째 줄까지에 이르는 계산식을 보자.




\end{proof}



산술기하, Cauchy-Schwarz 부등식

\begin{example}
양의 실수 $x,y,z$에 대하여 다음 부등식이 성립함을 보여라:
\[\frac x{y+z}+\frac y{z+x}+\frac z{x+y}\ge\frac32.\]
\end{example}
\begin{linenumbers*}
\begin{proof}[풀이 $1$]
\end{proof}
\end{linenumbers*}
\begin{linenumbers*}
\begin{proof}[풀이 $2$]
\end{proof}
\end{linenumbers*}
\begin{proof}[해설]
\end{proof}



최댓값과 최솟값의 정의

\begin{example}
실수 $x,y$에 대한 다음 식의 최댓값과 최솟값을 구하여라:
\[\frac{x^2-xy+y^2}{x^2+xy+y^2}.\]
\end{example}
\begin{linenumbers*}
\begin{proof}[풀이 $1$]
\end{proof}
\end{linenumbers*}
\begin{linenumbers*}
\begin{proof}[풀이 $2$]
\end{proof}
\end{linenumbers*}
\begin{proof}[해설]
\end{proof}




\section{실함수}

함수의 성질들, 연속함수, 코시함수방정식



\section{실수열}

재귀적으로 정의된 수열의 존재성

수학적 귀납법 제대로 쓰기

피보나치 수열의 일반항


\section{정수론}


나누어 떨어짐과 나머지

\begin{example}
정수 $x$에 대하여 $x^5-x$는 $30$의 배수임을 보여라.
\end{example}
\begin{linenumbers*}
\begin{proof}[풀이]
\end{proof}
\end{linenumbers*}
\begin{proof}[해설]
\end{proof}


소수



\begin{example}
다음 방정식을 만족하는 양의 정수들의 세 쌍 $(x,y,z)$를 모두 구하여라:
\[\frac1x+\frac1y+\frac1z=1.\]
\end{example}
\begin{linenumbers*}
\begin{proof}[풀이]
\end{proof}
\end{linenumbers*}
\begin{proof}[해설]
\end{proof}

합동

\begin{example}
다음 방정식을 만족하는 정수들의 세 쌍 $(x,y,z)$를 모두 구하여라:
\[x^4+y^4=z^4.\]
\end{example}
\begin{linenumbers*}
\begin{proof}[풀이]
\end{proof}
\end{linenumbers*}
\begin{proof}[해설]
\end{proof}




\begin{example}
양의 정수 $x,y$에 대하여 $x^2+y^2$가 $xy+1$로 나누어 떨어진다면
\[\frac{x^2+y^2}{xy+1}\]
는 완전제곱수임을 보여라.
\end{example}
\begin{linenumbers*}
\begin{proof}[풀이]
\end{proof}
\end{linenumbers*}
\begin{proof}[해설]
\end{proof}





\chapter{조합의 방법}

\section{경우의 수}
카운팅의 정수론적 방법은 왜 유효한가?
순열조합 증명, 군작용궤도, 생성함수
\section{알고리즘}
알고리즘의 형식화, 상태기계, 불변량과 단조량
\section{그래프}
더블카운팅, 비구성적 존재성(비둘기, 확률, 익스트리멀 이론)

상자와 공, 악수, 표, 테이블, 학급, 카드


\section{게임}
게임의 종류,
필승전략의 존재성





\chapter{기하의 방법}

\section{평면기하}

Euclid의 공리가 뒷받침되는 현대수학적 토대와 형식적 증명 체계는 어떻게 마련될 수 있는가?


\begin{axiom*}[Hilbert 공리계]
점과 직선.
결합관계(incidence), 순서관계(betweenness), 선분과 각의 합동관계(congruence)

선분, 각, 반직선, 반평면의 정의

다음 15개의 공리로 구성되는 이론을 \textbf{Euclid}\emph{ 평면기하}(Euclidean plane geometry)라고 한다.

\begin{itemize}
\item 결합관계에 관한 공리군
\begin{enumerate}
\item 서로 다른 두 점을 지나는 직선이 존재한다.
\item 한 직선 위에 서로 다른 두 점이 존재한다.
\item 한 직선 위에 있지 않은 세 점이 존재한다.
\end{enumerate}
\item 순서관계에 관한 공리군
\begin{enumerate}
\setcounter{enumi}{3}
\item 서로 다른 세 점 $A,B,C$에 대하여, $B$가 $A,C$ 사이에 있다면, $B$는 $C,A$ 사이에 있고, 세 점 $A,B,C$는 한 직선 위에 있다.
\item 서로 다른 두 점 $B,D$에 대하여, $B$가 $A,C$ 사이에 있고 $C$가 $B,D$ 사이에 있고 $D$가 $C,E$ 사이에 있는 점 $A,C,E$가 존재한다.
\item 한 직선 위에 있는 서로 다른 세 점 $A,B,C$에 대하여, 세 점 중 하나는 나머지 두 점 사이에 있다.
\item (Pasch) 한 직선 위에 있지 않은 서로 다른 세 점 $A,B,C$에 대하여, 한 직선이 세 점을 지나지 않고 $A,C$ 사이의 점을 지난다면, $A,B$ 사이의 점도 지나거나 $B,C$ 사이의 점도 지난다.
\end{enumerate}
\item 합동관계에 관한 공리군
\begin{enumerate}
\setcounter{enumi}{7}
\item (길이의 작도) 서로 다른 두 점 $A,B$와 점 $A'$에 대하여, $A'$에서 뻗어나가는 반직선 위의 점 $B'$가 존재하여 $AB$와 $A'B'$가 합동이다.
\item (길이의 동치) 추이성과 대칭성...
\item (길이의 합) $B$가 $A,C$ 사이에 있고 $B'$가 $A',C'$ 사이에 있으며, $AB$와 $A'B'$가 합동이고 $BC$와 $B'C'$가 합동일 때, $AC$와 $A'C'$도 합동이다.
\item (각도의 작도)
\item (SAS합동)
\end{enumerate}
\item 평행선에 관한 공리
\begin{enumerate}
\setcounter{enumi}{12}
\item (Playfair)
\end{enumerate}
\item 연속성에 관한 공리군
\begin{enumerate}
\setcounter{enumi}{13}
\item (Archimedes)
\item (Dedekind)
\end{enumerate}
\end{itemize}
\end{axiom*}


각돌리기, 길이비, 삼각형의 중심, 이차곡선

선분의 길이와 각의 크기는 실수로 표현되지 못한다.

\section{해석적 방법}


길이와 각을 실수의 값으로서 기술할 수 있는 평면기하의 모형.
해석학을 도입하지 않은 삼각비는 어떻게 다루어져야 하는가?

넓이의 정의

피타고라스 정리

삼각법, 복소평면, 배리센트릭
공간기하

\section{변환}

군

에를랑겐 프로그램

닮음, 반전, 사영

\section{}



\end{document}