\documentclass{../../large}
\usepackage{../../ikhanchoi}


\begin{document}
\title{Analysis II}
\author{Ikhan Choi}
\maketitle




\tableofcontents




\part{Integration}
\chapter{Riemann integral}


\section{Riemann integral}
We are concerned only with integral on a closed interval, until considering improper integral.



\begin{prb}
Let $[a,b]\subset\R$ be a closed interval.
\begin{parts}
\item The space of real-valued functions $[a,b]\to\R$ is Dedekind complete.
\item The space of continuous functions $C([a,b],\R)$ is not Dedekind complete.
\end{parts}
\end{prb}


\begin{prb}[Step functions]
A function $f:[a,b]\to\R$ is called a \emph{step function} if it is given by the real linear combination of indicator functions of closed intervals in $[a,b]$.
\end{prb}

\begin{prb}[Definition of Riemann integral]
Let $R$ be a vector space of some real-valued functions on a closed interval $[a,b]$ containing all step functions.
Let $R^+$ be the subset of all non-negative functions in $R$.
The \emph{Riemann integral} can be defined as a map $I:R^+\to[0,\infty]$ such that
\begin{parts}
\item it is additive and homogeneous,
\item it is normal...
\item $I(1_{[s,t]})=t-s$ for all closed intervals $[s,t]\subset[a,b]$.
\end{parts}
Such a linear functional is given, then we denote as
\[I(f)=\int_a^bf(x)\,dx,\qquad f\in R.\]

On the space of Riemann integrable functions, the Riemann integral uniquely exists.

\begin{parts}
\item The integral $\int_a^bs(x)\,dx:=\sum_{i=1}^nc_i(b_i-a_i)$, where $s(x)=\sum_{i=1}^nc_i1_{[a_i,b_i]}(x)$, is well-defined.
\item The integral $\int_a^bf(x)\,dx:=\lim_{n\to\infty}\int_a^bs_n(x)\,dx$ is well-defined.
\end{parts}
\end{prb}
\begin{pf}
\end{pf}


simple functions are norm dense in $L^\infty(I)$.
step functions are not norm dense in $L^\infty(I)$.
step functions are order dense(?) in $L^\infty(I)$.

% \R^I 안에서 클로져를 취해도 바나흐 완비는 되지 못하는 게 리만적분의 한계
% 그런데 Dedekind complete에서도 closure 개념이 있나?



For a given real function on interval, each (tagged) partition provides a step function.
Riemann integral: tagged partition
Darboux integral: partition

\section{Fundamental theorem of calculus}



\chapter{Lebesgue integral}


\section{Measurability}

\begin{prb}[Measurable sets]
\end{prb}

\begin{prb}[Measurable functions]
\end{prb}



\begin{prb}[Integral of complex-valued functions]
\end{prb}


\section{Improper integral}
It is about a infinite measure.
For integrable function, it has no problem.

An improper integral must be interpreted as an extension of operators from $L^1$.
There are various way to approximate the improper integral.
We need to be able to justify the reason why each specific approximation is reasonable or not.

principal values


\section*{Exercises}

\section*{Problems}
\begin{enumerate}
\item Find the value of $\lim_{n\to\infty}\frac1n\left(\sum_{k=1}^n\frac1nf\left(\frac kn\right)-\int_0^1f(x)\,dx\right)$.
\item Find all $a>0$ and $b>0$ such that $\int_0^\infty x^{-b}|\tan x|^a\,dx$ converges.
\item* If $xf'(x)$ is bounded and $x^{-1}\int_0^xf(t)\,dt\to L$ then $f(x)\to L$ as $x\to\infty$.
\item Show that for a continuous function $f:[0,1]\to\R$ we have $\int_0^1x^2f(x)\,dx=\frac13f(c)$ for some $c\in[0,1]$.
\end{enumerate}


\chapter{Lebesgue spaces}


\begin{pf}
\[\int fg\le C^p\int\frac{|f|^p}p+\frac1{C^q}\int\frac{|g|^q}q\]
Take $C$ such that
\[C^p\int\frac{|f|^p}p=\frac1{C^q}\int\frac{|g|^q}q.\]
Then,
\[C^p\int\frac{|f|^p}p+\frac1{C^q}\int\frac{|g|^q}q=2p^{-\frac1p}q^{-\frac1q}\Bigl(\int|f|^p\Bigr)^{\frac1p}\Bigl(\int|g|^p\Bigr)^{\frac1q}.\]
Note that we can show that $1\le2p^{-\frac1p}q^{-\frac1q}\le2$ and the minimum is attained only if $p=q=2$, so this method does not provide the sharpest constant.
\end{pf}









\part{Multi-variable calculus}
\chapter{Fr\'echet derivatives}
% 편미분이 그냥 계산도구라는 것을 배운다
% 편미분의 교환도 사실은 항상 다 되는 걸... 배울 수 있나
% 접공간 개념을 배운다
\section{Tangent spaces}
\begin{prb}[Vector fields]

\end{prb}

\section{Inverse function theorem}


\section{}




\chapter{Differential forms}

\section{De Rham complex}

\begin{prb}[Tensor product]
\end{prb}

\begin{prb}[Wedge product]
\end{prb}

\begin{prb}[One-forms]
\end{prb}

\begin{prb}[Exterior derivative]
\end{prb}


\section{Riemannian metrics}


\begin{prb}[Musical isomorphisms]
\end{prb}

\begin{prb}[Inner product of differential forms]
ONB
\end{prb}


\begin{prb}[Hodge star operator]
Identification of 2-forms and vector fields
\end{prb}

\section{Vector calculus}


\begin{prb}[Gradient, curl, and divergence]
\end{prb}

\begin{prb}[Potentials]
\end{prb}

\begin{prb}[Vector calculus identities]
\end{prb}



\section{Integral of differential forms}

\begin{prb}[Multiple integral]
volume forms,
stone weierstrass and fubini
\end{prb}

\begin{prb}[$C^1$ singular chains]

\end{prb}

\begin{prb}[Line integrals]
A $C^1$ singular 1-cycle is the formal sum of \emph{contours}, piecewise $C^1$ closed curves.
\end{prb}

\begin{prb}[Surface integrals]
A $C^1$ singular 2-cycles.
\end{prb}



\section*{Exercises}

\begin{prb}[Multivariable Taylor's theorem]
Symmetric product
\end{prb}

\begin{prb}[Vector analysis in two dimension]
\end{prb}

\begin{prb}[Geometric algebra]
\end{prb}








\chapter{Stokes theorem}


\section{}

embedded chains instead of manifolds
triangulation

\section{Local coordinates}


\begin{prb}[Spherical coordinates]
Let $U=\R^3\setminus\{\,(x,y,z):x=0,\ y\ge0\,\}$.
\[(x,y,z)=(r\sin\theta\cos\f,r\sin\theta\sin\f,r\cos\theta)\]
for $(r,\theta,\f)\in(0,\infty)\times(0,\pi)\times(0,2\pi)$.
Orthonormal bases are
\[\left\{\pd_r,\ \frac1r\pd_\theta,\ \frac1{r\sin\theta}\pd_\f\right\}\subset\fX(U),\]
\[\{dr,\ r\,d\theta,\ r\sin\theta\,d\f\}\subset\Omega^1(U),\]
\[\{r^2\sin\theta\,d\theta\wedge d\f,\ r\sin\theta\,d\f\wedge dr,\ r\,dr\wedge d\theta\}\subset\Omega^2(U).\]
\begin{parts}
\item
\item The Laplacian is given by
\[\Delta f=\frac1{r^2}\pd{r}\left(r^2\pd{f}{r}\right)+\frac1{r^2\sin\theta}\pd{\theta}\left(\sin\theta\pd{f}{\theta}\right)+\frac1{r^2\sin^2\theta}\pd[2]{f}{\f}.\]
\end{parts}
\end{prb}
\begin{pf}
Write $df$ in the orthonormal basis $\{dr,\ r\,d\theta,\ r\sin\theta\,d\f\}$ as
\begin{align*}
df&=\pd{f}{r}\,dr+\pd{f}{\theta}\,d\theta+\pd{f}{\f}\,d\f\\
&=\left(\pd{f}{r}\right)\,dr+\left(\frac1r\pd{f}{\theta}\right)\,r\,d\theta+\left(\frac1{r\sin\theta}\pd{f}{\f}\right)\,r\sin\theta\,d\f.
\end{align*}
After taking the Hodge star operator, write in the basis $\{d\theta\wedge d\f,d\f\wedge dr,dr\wedge d\theta\}$ as
\begin{align*}
{}*df&=\left(\pd{f}{r}\right)\,r^2\sin\theta\,d\theta\wedge d\f+\left(\frac1r\pd{f}{\theta}\right)\,r\sin\theta\,d\f\wedge dr+\left(\frac1{r\sin\theta}\pd{f}{\f}\right)\,r\,dr\wedge d\theta\\
&=r^2\sin\theta\pd{f}{r}\,d\theta\wedge d\f+\sin\theta\pd{f}{\theta}\,d\f\wedge dr+\frac1{\sin\theta}\pd{f}{\f}\,dr\wedge d\theta.
\end{align*}
Then, the differential is computed as
\begin{align*}
d*df&=d\left(r^2\sin\theta\pd{f}{r}\right)\,d\theta\wedge d\f+d\left(\sin\theta\pd{f}{\theta}\right)\,d\f\wedge dr+d\left(\frac1{\sin\theta}\pd{f}{\f}\right)\,dr\wedge\theta\\
&=\left[\sin\theta\pd{r}\left(r^2\pd{f}{r}\right)+\pd{\theta}\left(\sin\theta\pd{f}{\theta}\right)+\frac1{\sin\theta}\pd[2]{f}{\f}\right]\,dr\wedge d\theta\wedge d\f.
\end{align*}
Finally we have
\begin{align*}
\Delta f={}*d*df&=\frac1{r^2\sin\theta}\left[\sin\theta\pd{r}\left(r^2\pd{f}{r}\right)+\pd{\theta}\left(\sin\theta\pd{f}{\theta}\right)+\frac1{\sin\theta}\pd[2]{f}{\f}\right]\\
&=\frac1{r^2}\pd{r}\left(r^2\pd{f}{r}\right)+\frac1{r^2\sin\theta}\pd{\theta}\left(\sin\theta\pd{f}{\theta}\right)+\frac1{r^2\sin^2\theta}\pd[2]{f}{\f}
\end{align*}

\end{pf}





\section{Stokes theorems}
% 미분기하보다 피디이스럽게
\begin{prb}[Bump functions]
\end{prb}

\begin{prb}[Partition of unity]
\end{prb}

\begin{prb}
\end{prb}


\end{document}