\documentclass{../../large}
\usepackage[kor]{../../ikhanchoi}


\begin{document}
\title{미분적분학}
\author{최익한}
\maketitle
\tableofcontents


\part{극한}


\chapter{수열의 극한}

\section{}
\section{}
\section{}
\section{}


- 수열
    주어진 수열의 극한의 정의
    점근스케일
    스퀴즈의 적용
    스털링 공식, 조합식의 극한
    점화수열의 극한, 고정점방법, 존재성가정

- 급수
    텔레스코핑 예제들
    비교판정법
    근 판정법, 비 판정법
    교대급수의 판정법




\chapter{초월함수}

\section{}
\section{}
\section{}
\section{}



- 지수와 로그
    복소지수

- 삼각함수
    특수각, 복소지수
    삼각법, 사인 법칙과 코사인 법칙
    공식들(합차, 배각, 반각, 곱)
    귀납식과 대칭식의 증명
    쌍곡함수, 역삼각함수




\chapter{함수의 극한}

\section{}
\section{}
\section{}
\section{}


- 함수의 극한
    연속성을 위한 함숫값의 조건
    다항식의 근의 개수


- 초월함수의 극한




\part{미적분}

\chapter{미분}

\section{미분법}
미분계수, 순간속도, 미분가능성
미분법: 체인룰, 라이프니츠룰, 역함수, 로피탈
\section{단조성}
최댓값과 극댓값의 주의점, 부등식의 증명, 접선의 방정식, 평균값 정리
\section{볼록성}
볼록성, 젠센부등식
\section{멱급수 전개}
테일러 근사와 오차분석









\chapter{부정적분}

\section{적분법}
구분구적법, 기본정리, 그래프 넓이
부분분수, 역함수 적분
\section{치환적분}
\section{부분적분}
\section{}







\chapter{정적분}

\section{이상적분}
적분의 점근스케일
감마함수
\section{적분부등식}
\section{직교함수}
\section{적분변환}



\end{document}