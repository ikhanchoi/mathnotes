\documentclass{amsart}

%%% Start of the area for technical editor.
\newcommand{\publname}{Seoul Science High School}
%\newcommand{\doiname}{http://dx.doi.org/10.4134/JKMS}
\issueinfo{}% volume number
  {}%        % issue number
  {}%        % month
  {}%     % year
\pagespan{1}{}
%\received{Received January 5, 2005}
%\received{Received August 25, 2005;\enspace Revised October 20, 2005}
\copyrightinfo{}%              % copyright year
  {Seoul Science High School}% copyright holder
%%% End of the area for technical editor.

\usepackage{kotex}
\usepackage{iftex}
\ifPDFTeX
  \usepackage{dhucs-nanumfont}
\else\ifXeTeX
  \setmainhangulfont[Ligatures=TeX]{HCR Batang LVT}
  \setsanshangulfont[Ligatures=TeX]{HCR Dotum LVT}
\else\ifLuaTeX
  \setmainhangulfont[Ligatures=TeX]{HCR Batang LVT}
  \setsanshangulfont[Ligatures=TeX]{HCR Dotum LVT}
\fi\fi\fi
\usepackage{amssymb}

\allowdisplaybreaks

\theoremstyle{plain}
\newtheorem{theorem}{Theorem}[section]
\newtheorem{proposition}[theorem]{Proposition}
\newtheorem{lemma}[theorem]{Lemma}
\newtheorem{corollary}[theorem]{Corollary}

\theoremstyle{definition}
\newtheorem*{definition}{Definition}

\theoremstyle{remark}
\newtheorem{remark}[theorem]{Remark}

\begin{document}

\title[Fold-and-Cuttable Pattern]
{Simple Closed Curve that is Conically Fold-and-Cuttable in 3-Dimensional Space}

\author[I. Choi, J. W. Jung]{Ikhan Choi, Jong Wook Jung}
\address{address}
\email{email}

\subjclass{subclass}
\keywords{keywords}














\begin{abstract}
\iffalse
기초적 미적분학, 기하학에 의한 증명
\fi
In this paper, we defined an expansion of {\it fold-and-cut theorem} to 3-euclidean space by treating origami sheets as developable surfaces.
With basic calculus and geometry, we suggested a sufficient condition that given piecewise differentiable simple closed curve on a plane is to be conically fold-and-cuttable.

\end{abstract}

\maketitle

\section{Introduction}
\iffalse
-오리가미 분야소개

-주제에 대한 간략한 소개
크리즈패턴과 폴드컷정리
정리스테이트

-배경지식

로컬 폴더빌리티와 글로벌 폴더빌리티
폴더빌리티 :
 마에카와 : 엠-브이=2
 카와사키 : 180도
 두 낫 페네트레이트, 관통규칙

공간으로의 확장 : 접는 함수를 생각할 수 있다. 공간에서도 관통규칙을 그대로 생각해볼 수 있다.

데벨로퍼블 분류 : (렘마)

까지


-인 섹션 왓
\fi

A crease pattern is a finite planar straight-line graph drawn on a convex planar region (the paper). A crease is an edge of the planar graph.

Fold-and-cut 정리란 평면 위의 straight side로 만들어진 어떤 모양이라도 적당한 crease pattern이 존재하여 한 장의 종이를 납작하게 접고 직선으로 단 한번 자름으로써 얻을 수 있다는 정리이다.
우리는 미분가능한 곡선으로 이루어진 모양에 대하여 3차원 유클리드 공간에서 평면으로 단 한번 자름으로써 그 모양을 얻을 수 있는 조건에 대한 문제를 정의하였다.
특별한 경우에 대하여 local-foldability를 확장하여 conical folding을 정의하고 특수한 경우의 조각적 미분가능한 단순폐곡선에 대하여 이러한 straight cut의 존재성을 증명하였고 구체적으로 정리 \ref{36}으로 표현되었다.


어떤 crease pattern이 flat-foldable인지 판단하는 문제는 다음 규칙들로써 완성되었다.

\begin{lemma}
어떤 crease pattern이 flat-foldable할 조건은 다음과 같다: at any vertex,
\begin{itemize}
\item {\rm Maekawa's theorem:} The number of mountain folds minus the number of valley folds meeting at v is either $2$ or $−2$.
\item {\rm Kawasaki's theorem:} the sum of alternate angles around v is $\pi$.
\item a sheet can never penetrate itself.
\end{itemize}
\end{lemma}

Kawasaki's theorem에서 필요한 조건으로 flat-foldable한 crease pattern의 모든 점의 차수가 짝수여야 함은 Maekawa's theorem으로부터 알 수 있다.
종이가 자신을 관통하지 않는 것은 물리적으로 모델링하기 위해 필요한 조건이다.
Crease pattern에서 flat-foldability를 따질 때 vertex가 단 하나인 경우를 local-foldability, 아닌 경우는 global-foldability라 한다.
일반적으로 주어진 crease pattern이 global-foldability를 가지는지를 판별하는 알고리즘은 NP-hard라 알려져 있다 \cite{Be}.
공간에서의 local-foldability를 생각하였을 때 나머지 규칙들은 없애거나 따로의 확장이 필요하지만 sheet를 관통하지 않는 것에 대한 규칙은 sheet의 위상적 성질을 그대로 3차원으로 끌어올림으로써 자연스럽게 확장시킬 수 있다.


Developable surface란 모든 점에서 가우스 곡률이 0인 곡면을 말한다. 자명하게 평면과 isometric하며 국소적으로 다음 세 가지로 분류된다: cylinder, cone, tangent developable.
3차원에 매장된 developable surface는 일반적으로 generator라 불리는 one-parameter family of lines의 합집합으로 표현되는 ruled surface이다.
우리는 모든 점에서 같은 꼭지점을 공유하는 cone들의 일부가 되는 developable surface를 다루고자 한다.
이 때 곡면의 모든 genrator가 한 점을 꼭짓점으로 갖는 반직선의 일부이고 이 점-vertex-은 local flat-foldable한 crease pattern이 가지는 하나의 vertex에 대응된다.
그러나 구체적으로는 우리가 다루는 대상-sheet-은 모든 점에서 연속이고 거리를 보존하지만 미분불가능한 점이 존재하고 따라서 surface라 부르기엔 적절하지 않으며 이는 자세하게 2장에서 정의될 것이다.


2장에서는 문제를 정의하기 위해 folding, pattern, cutting을 정의하고 초등적 문제 해결을 위해 도입할 함수 transition의 정의와 간단한 성질을 설명할 것이다.

3장에서는 transition와 cuttability 사이에 어떤 관계가 있음을 보이고 최종적으로 정리 \ref{36}를 통해 제시될 특수한 조건을 만족하는 경우에 대하여 cut이 가능함을 보일 것이다.










\section{Cutting and Transition}





 문제를 수학적으로 접근하기 위하여 먼저 우리가 다루게 될 사상과 공간을 정의하자.
 
\begin{definition}
극점이 $O$이고 극좌표 $(r,\theta)$를 가지는 평면 $U$와 $z$축이 더해진 원통좌표계를 가지는 공간 $U\times {\mathbb R}$를 생각하자.
두 공간 $U$와 $U\times{\mathbb R}$에 euclidean metric을 줄 때 거리를 보존하는 조각적 미분동형사상 $\varphi : U\to M\subset U\times {\mathbb R}$에 대해 $\varphi(U)\cap U$가 열린 영역을 포함하지 않고 $\varphi(O)$이 $z$축 위에 오는 사상 $\varphi$를 {\it folding}이라 하고, 이 때 $\varphi(O)$의 $z$-좌표의 절댓값을 $\varphi$의 {\it height}라 하자.
\end{definition}

조각적 미분동형사상은 조각적으로 미분가능한 위상동형사상을 말하는 것으로 한다.
거리를 보존하므로 $M=\varphi(U)$을 적당히 가산 개의 영역으로 나누었을 때 각 영역은 developable surface이다.
사상 $\varphi$가 folding이면 단사함수이므로 $M$은 자연스럽게 자신을 관통하지 않는다.

\begin{definition}
Folding $\varphi$의 상 $M$을 분할하는 모든 developable surface가 $\varphi(O)$를 꼭짓점으로 하는 generalized cone일 때 folding $\varphi$를 {\it conical folding}이라 하자.
\end{definition}

Conical folding에 대한 개념은 vertex $\varphi(O)$에서 뻗어나가는 반직선들의 연속적 자취, 즉 generalized cone의 형태를 가지는 folding을 다루고자 정의한 것이며 flat origami model에서의 local-foldability를 공간으로 확장시킨 것으로 볼 수 있다.
여기서 점 $O$는 crease pattern의 vertex에 대응된다.


\begin{definition}
Fold-and-cut이 가능한지 결정될 조각적 미분가능한 단순폐곡선 $\gamma : [0,2\pi]\to U$를 {\it pattern}이라 하자.
어떤 folding $\varphi$와 임의의 $\theta\in [0,2\pi]$에 대하여 $\varphi(\gamma(\theta))$의 $z$-좌표가 $0$일 때, $\varphi$를 pattern $\gamma$의 {\it cutting}이라 하고, cutting $\varphi$가 conical folding이면 $\varphi$를 {\it conical cutting}이라 하자.
반대로 pattern $\gamma$에 대하여 어떤 folding $\varphi$가 존재하여 $\varphi$가 $\gamma$의 cutting이 될 때 이 $\gamma$를 {\it cuttable}하다고 하자.
\end{definition}

평면 $U$ 위에 주어진 pattern $\gamma$에 대하여 cutting $\varphi$가 존재할 때 상 $\varphi(\gamma)$ 또한 단순폐곡선이 된다.

우리는 평면 위에 어떤 단순폐곡선이 주어졌을 때 적절한 folding에 의해 평면을 접은 후 그 단순폐곡선을 평면으로 자를 수 있는가를 묻는 문제에 대하여 대답할 것이다.
``Cuttable''이란 이 문제를 수학적으로 개념화시킨 것이며 어떤 pattern에 대하여 cutting이 존재하는가에 대한 것만 묻고 있으므로 conical folding, cutting의 존재성을 증명하는 것으로 충분하다.
앞으로 이 논문에서 ``cutting''과 ``cuttable''은 각각 conical cutting과 주어진 pattern에 대해 conical cutting이 존재함을 의미하는 것으로 하자.

\begin{theorem}\label{nec1}%2.1
주어진 pattern $\gamma$가 cuttable하다면 $\gamma$ 위의 각 점에서 $\theta$-좌표로 가는 함수는 단사함수이다.
\end{theorem}

\begin{proof}
Pattern $\gamma$ 위의 서로 다른 두 점 ${\bold p}$와 ${\bold q}$가 존재하여 같은 $\theta$-좌표를 갖는다고 하자.
$\gamma$의 cutting $\varphi$에 대하여 $\varphi({\bold p}),\varphi({\bold q}),\varphi(O)$는 한 직선 $g$ 위에 있고 $M=\varphi(U)$은 $\varphi(O)$를 꼭짓점으로 하는 generalized cone이므로 $\varphi(g)$는 $M$ 의 generator이고 $U$ 위에 있다.
직선 $g$ 위에 있지 않은 임의의 점 ${\bold r}$에 대하여 $\varphi({\bold r})$도 $U$ 위에 있으므로 $\varphi({\bold r})$의 generator 또한 $U$ 위에 있다.
모든 generator들의 집합은 $M$의 부분집합이고 단순폐곡선 $\varphi(\gamma)$의 내부를 덮으므로 $M\cap U$는 열린 영역인 $\varphi(\gamma)$의 내부를 포함한다.
따라서 $\varphi$가 folding임에 모순이므로 $\gamma$ 위의 임의의 서로 다른 두 점 ${\bold p}$와 ${\bold q}$의 $\theta$-좌표는 다르다.
\end{proof}

\begin{corollary}%2.2
주어진 pattern $\gamma$가 cuttable하다면 $O$는 $\gamma$ 내부에 존재한다.
\end{corollary}

정리 \ref{nec1}에 의해 pattern $\gamma$가 cuttable일 필요조건은 조각적 미분가능한 연속함수 $r : [0,2\pi]\to (0,\infty)$가 존재하여 $\gamma(\theta)=(r(\theta),\theta)$로 두는 것이 가능한 것이다.
이와 같이 정의되는 함수 $r$을 pattern $\gamma$의 {\it radius}라 하고 반드시 cuttable하지는 않은 pattern $\gamma$에 대하여 radius가 정의되면 pattern $\gamma$를 {\it injective pattern}이라 하자.

\begin{theorem}\label{inc}%2.3
점 $O$를 중심으로 하는 원이 아닌 injective pattern $\gamma$의 radius $r$에 대하여 적당한 구간 $[a,b]$가 존재하여 $r(\theta)$가 $\theta$에 대하여 구간$[a,b]$에서 증가 또는 감소함수이다. 
\end{theorem}

\begin{proof}
조각적 미분가능인 pattern이 폐곡선이므로 미분불가능점은 많아야 유한 개이고 $r(\theta)$의 미분불가능점도 많아야 유한 개이다.
$\gamma$는 조건에 의해 상수함수가 아니므로 $r(\theta)$의 미분불가능점들의 집합을 $K$, $n+2$개의 실수 $k_0<k_1<k_2<\cdots<k_{n+1}$를 집합 $K\cup\{0,2\pi\}$의 원소로 정의할 때 집합족 $\{[k_i,k_{i+1}]\}_{i\in{\mathbb Z}\cap[0,n]}$의 원소가 되는 구간 중 $r$이 상수함수가 되지 않는 구간이 존재한다.
이 구간을 $[k_m, k_{m+1}]$이라 하자.
함수 $r$은 구간 $(k_m, k_{m+1})$에서 미분가능이므로 $r'(\theta_0)\ne0$를 만족하는 $\theta_0$가 존재한다.
다르부의 정리에 의해 $\theta_0$를 포함하는 적절한 근방$[a,b]$가 존재하여 이 구간에 속한 모든 $\theta$에 대하여 $r'(\theta)\ne0$이다.
이 때 $r'(\theta)>0$이면 $r$은 증가함수이고 $r'(\theta)$이면 $r$은 감소함수이다.
\end{proof}






\iffalse
자명상한의 존재성 설명
트랜지션의 정의 : 이변수함수(제트와 세타, 제트 0부터 자명상한까지 개구간, 특히 세타 정의 0부터 2파이 폐구간), 식으로, 양수 : 타우

트랜지션에 대한 설명 : 트랜지션 의미 - 뿔의 트라이앵귤레이션으로부터, 트랜지션은 패턴과 대응, 폴딩과는 독립적
자명상한 정의 : 수식으로 정의 triangulation

(정리)트랜지션의 성질
 제트세타 조각적 연속 유계 (정확히는 미분가능)
 세타 루프
 제트 0으로 가면 1로 감

(정리)트랜지션의 성질2
 제트 적당한 구간까지 강증가 필충조건 알프라임<알 (고전기하 좀 써야 할 듯)

(정리)커터블 필요조건2 : 원점을 지나는 접선 존재하면 안됨 - 즉 트랜지션이 발산하면 안됨
증명 : 자명상한이 0이 되므로

커터블 필요조건1과2는 유계인 트랜지션이 정의될 필충
%%%%%%%%%%%%%%%%%%%%%%%%%%%%%%%%55

사인드 트랜지션 : 트랜지션과 절댓값 같음, 피스와이즈 : 시그마
심플 트랜지션 정의 : 사인드 트랜지션, 페네트레이션 규칙 만족(알의 심플) : 시그마
커터블 트랜지션 정의 : 사인드 트랜지션, 심플 트랜지션, 어떤 제트에 대해 트랜지션 적분 : 시그마

사인드 설명 : 카와사키 정리
심플 설명 : (정리)강증가 or 강감소 구간 카파잡을 때 심플
커터블 설명 : 어떤 패턴에 대해 존재성 of (커터블 트랜지션과 커팅)이 동치임을 보일 것이다


(정리) 커터블 트랜지션 존재 iff 커팅 존재, 동치정리%트랜지션 존재는 커팅 존재의 필요충분조건이다
 커팅 조건 : 거리보존 단사(심플) 파이오 내포아님 제트사라짐

증명 : 트랜지션 존재가 커팅 존재의 충분조건
        적분되는 제트를 파이오로 두고 알이에프젠
       트랜지션 존재가 커팅 존재의 필요조건
        파이오를 제트로 두면
      
\fi










Cuttable pattern $\gamma$를 생각하자.
$\gamma$의 cutting $\varphi$와 radius $r$에 대하여 $\varphi$의 height를 $z$라 할 때 $z<\inf r(\theta)$이므로 $z$들의 집합은 상한이 존재한다.
이 상한을 $L$이라 하자.

\begin{definition}
주어진 injective pattern $\gamma$와 그 radius $r$에 대하여 $r(\theta)$의 미분불가능점들의 집합을 $K$라 할 때 {\it transition} $\tau : (0,L)\times\left([0,2\pi]\backslash K\right)\to[0,\infty]$를 다음과 같이 정의하자:
\begin{align}\label{tr1}
\tau(z,\theta)=\left(1+\frac{z^2}{r(\theta)^2 -z^2}\left(1-\frac{r'(\theta)^2}{r(\theta)^2 -z^2}\right)\right)^{\frac12}.
\end{align}
$r'$은 $r$의 도함수이다.
\end{definition}

주어진 pattern $\gamma$에 transition가 정의될 필요충분조건은 $\gamma$가 injective pattern인 것이다.

어떤 developable surface를 작은 삼각형들로 분할하여 근사시키는 것을 triangulation approximating이라 한다.
Pattern $\gamma$의 내부 영역을 $I$라 할 때 상 $\varphi(I)$는 generalized cone이므로 triangulation 과정에 있는 모든 삼각형은 꼭짓점 중 하나가 $\varphi(O)$에 있고 그 변이 $\varphi(\gamma)$ 위에 있게 분할할 수 있을 것이다.
각 $\theta$의 변화량 $\Delta\theta$를 생각하여 세 점 $\varphi(\gamma(\theta))$와 $\varphi(\gamma(\theta+\Delta\theta)), \varphi(O)$를 꼭짓점으로 갖는 삼각형을 $T$라 할 때 $T$는 삼각형 $(\gamma(\theta), \gamma(\theta+\Delta\theta), O)$와 합동이며 triangulation에 의한 분할의 원소 중 하나가 될 수 있다.

삼각형 $T$와 각 $\Delta\theta$를 평면 $U$에 정사영시킨 삼각형 $T^*$와 정사영된 각 $\Delta\theta^*$를 생각하자.
이 때 삼각형 $T^*$는 세 점 $\varphi(\gamma(\theta))$와 $\varphi(\gamma(\theta+\Delta\theta)), O$를 꼭짓점으로 가지는 삼각형이다.


$r_1=r(\theta), r_2=r(\theta+\Delta\theta)$라 할 때 코사인 법칙에 의하여 다음을 얻는다.
\begin{align*}
r_1^2+r_2^2-2r_1r_2\cos\Delta\theta\,&=\,\left(r_1^2-z^2\right)+\left(r_2^2-z^2\right)
\\&\quad-2\sqrt{r_1^2-z^2}\sqrt{r_2^2-z^2}\cos\Delta\theta^*.
\end{align*}
다음과 같이 식을 변형하자.
\begin{align*}
r_1r_2(1-\cos\Delta\theta)\,&=\,r_1r_2-z^2-\sqrt{r_1^2-z^2}\sqrt{r_2^2-z^2}
\\&\quad+\sqrt{r_1^2-z^2}\sqrt{r_2^2-z^2}(1-\cos\Delta\theta^*).
\end{align*}
삼각함수의 반각공식에 의하여 다음을 얻는다.
\begin{align*}
r_1r_2\sin^2\frac{\Delta\theta}2=\frac{(r_2-r_1)^2z^2}{r_1r_2-z^2+\sqrt{r_1^2-z^2}\sqrt{r_2^2-z^2}}+\sqrt{r_1^2-z^2}\sqrt{r_2^2-z^2}\sin^2\frac{\Delta\theta^*}2.
\end{align*}
양변을 $(\Delta\theta/2)^2$로 나누고 극한 $\lim_{\Delta\theta\to0}$를 취하면 $\lim_{\Delta\theta\to0}(r_2-r_1)/\Delta\theta=r'(\theta)$이고 $\lim_{\Delta\theta\to0}\sin\Delta\theta/\Delta\theta=1$이므로 다음을 얻는다.
\begin{align*}
r(\theta)^2=\frac{2r'(\theta)^2z^2}{r(\theta)^2-z^2}+\left(r(\theta)^2-z^2\right)\left(\frac{d\theta^*}{d\theta}\right)^2.
\end{align*}
간단한 계산을 통해 다음을 얻는다.
\begin{align}\label{tr2}
\frac{d\theta^*}{d\theta}=\left(1+\frac{z^2}{r(\theta)^2 -z^2}\left(1-\frac{r'(\theta)^2}{r(\theta)^2 -z^2}\right)\right)^{\frac12}.
\end{align}
식 (\ref{tr2})에 의해 함수 transition이 식 (\ref{tr1})과 같이 정의되었음을 확인하자.

\begin{definition}
각 $\psi\in[0,2\pi]$에 대하여 {\it transitioned angle} $\psi^*$를 다음과 같이 정의하자:
\begin{align*}
\psi^*=\int_0^{\psi}\tau(z,\theta)d\theta
\end{align*}
\end{definition}

삼각형 $T$의 두 꼭짓점 $\varphi(\gamma(\theta))$와 $\varphi(\gamma(\theta+\Delta\theta))$가 평면 $U$ 위에 있으므로 height의 값은 삼각형 $T$와 평면 $U$ 사이의 이면각이 직각일 때의 $\varphi(O)$의 $z$-좌표보다 커질 수 없다.
삼각형 $T$와 평면 $U$ 이면각이 직각일 때 다음과 같은 식이 성립한다.
\begin{align*}
{\rm area}\,{\rm of}\,T=\frac12r_1r_2\sin\Delta\theta&=\frac12\left|\sqrt{r_1^2-z^2}-\sqrt{r_2^2-z^2}\right|z
\\&=\frac12\frac{r_1^2-r_2^2}{\sqrt{r_1^2-z^2}-\sqrt{r_2^2-z^2}}z.
\end{align*}
마찬가지로 $\Delta\theta$가 $0$에 가까워질 때 적절한 식 변형을 통하여 다음을 얻을 수 있다.
\begin{align}\label{r'}
r(\theta)=\frac{\left|r'(\theta)\right|}{\sqrt{r(\theta)^2-z^2}}z \implies
z=\frac{r(\theta)^2}{\sqrt{r(\theta)^2+r'(\theta)^2}}.
\end{align}
이 $z$의 $\theta$에 대한 하한을 height의 상한 $L$로 정의하자.

\begin{definition}
주어진 injective pattern $\gamma$에 대하여 $\gamma$가 cuttable하기 위한 height의 {\it trivial supremum} $L$을 다음과 같이 정의하자:
\begin{align*}
L=\inf_{\theta}\frac{r(\theta)^2}{\sqrt{r(\theta)^2+r'(\theta)^2}}.
\end{align*}
\end{definition}

Transition은 folding에 의한 $\theta$가 이루는 공간의 확대와 축소를 나타내는 지표로 볼 수 있다.
Transition이 정의되는 pattern에 대해 transition과 trivial supremum 모두 유일하게 존재하고 folding 및 cutting과 독립적이다.

\begin{theorem}%2.4
주어진 injective pattern $\gamma$에 대하여 $\gamma$의 transition $\tau$는 $z$에 대하여 연속이고 $\theta$에 대하여 조각적으로 연속이며 미분가능이다.
또한 각각 임의의 $z$와 $\theta$에 대하여 $\tau(z,0)=\tau(z,2\pi)$와 $\lim_{z\to0}\tau(z,\theta)=1$가 성립한다.
\end{theorem}





\iffalse
\begin{theorem}
주어진 injective pattern $\gamma$를 생각하자. 양의 실수 $Z<L$가 존재하여 구간 $(0,Z)$ 안의 모든 $z$에 대하여 $\tau(z,\theta)$는 상수함수일 수 없고 증가함수일 필요충분조건은 $|r'(\theta)|<r(\theta)$이다.
\end{theorem}

\begin{proof}
한 변 $\varphi(\Delta r)$을 공유하는 두 삼각형 $T$와 $T^*$를 같은 평면 위에 놓고자 한다.
공간에서 변 $\varphi(\Delta r)$의 위치를 고정시킨 $T^*$의 합동변환은 $\varphi(\Delta r)$를 축으로 하는 회전변환이다.
이 회전변환에 의한 삼각형 $T^*$의 상들 중 삼각형 $T$와 같은 평면에 있는 삼각형은 두 개이다.
이 두 삼각형 중 $\varphi(\Delta r)$을 기준으로 $T$와 같은 쪽에 있는 삼각형을 $T'$라 하자.
이 회전변환에 의해 점 $O$가 이동한 점들은 항상 직선 $\varphi(\Delta r)$에 수직이고 $z$축을 포함하는 평면 위에 있고 이 점들 중 삼각형 $T$와 같은 평면에 있는 점 두 개중 $T'$의 꼭짓점이 되는 하나를 $O'$라 하면 두 점 $\varphi(O)$와 $O'$을 지나는 직선은 $\varphi(\Delta r)$에 수직이다.

삼각형 $T$를 평면 $U$에 정사영시키는 과정에서 넓이 및 길이가 축소되기 때문에 직선 $\varphi(\Delta r)$부터 $\varphi(O)$까지의 거리보다 $O'$까지의 거리가 더 짧고 height $z$의 값이 커질수록 정사영에 의한 축소 효과는 커지므로 두 거리 사이의 차 또한 커진다.
$\Delta\theta$가 $0$에 가까워질 때 삼각형 $T$의 외접원 내부에 $O'$가 존재하면 $\Delta\theta<\Delta\theta^*$이므로 $\tau=\Delta\theta^*/\Delta\theta$가 증가하고  점 $O$가 $\varphi(\Delta r)$ 쪽으로 충분히 작은 거리를 이동하여 외접원 내에 있을 필요충분조건은 $r(\theta)>\sqrt2 z$이다.
이를 식 (\ref{r'})에 대입하면 $|r'(\theta)|<r(\theta)$를 얻는다.
\end{proof}
\fi


\begin{theorem}\label{nec2}%2.5
주어진 pattern $\gamma$가 cuttable하다면 점 $O$를 지나고 $\gamma$에 접하는 직선이 존재하지 않는다.
\end{theorem}

\begin{proof}
점 $O$를 지나고 $\gamma$에 접하는 직선이 존재한다고 가정하고 이 접점을 $\gamma(\theta)=(r(\theta),\theta)$라 하면 $r'(\theta)$의 값은 무한대로 발산해 존재하지 않는다.
이 때 height의 값은 $0$이고 따라서 이 pattern $\gamma$의 trivial supremum $L$은 $0$이므로 $z$는 정의역을 가질 수 없다.
\end{proof}

정리 \ref{nec1}과 \ref{nec2}에 제시된 조건은 주어진 pattern $\gamma$에 대해 $\gamma$가 cuttable할 필요조건임과 동시에 유계인 transition이 정의될 필요충분조건이다.

\bigskip










\section{Cuttable Transition and Conditions of Cuttability}

\begin{definition}
주어진 injective pattern $\gamma$의 transition $\tau$에 대하여 유한 개의 닫힌 구간들의 합집합 $\kappa\subset[0,2\pi]$가 존재하여 다음이 성립하는 함수 $\sigma(z,\theta) : (0,L)\times[0,2\pi]\backslash K\to{\mathbb R}$를 {\it signed transtion}이라 하자:
\begin{align*}
\sigma(z,\theta)=s_{\kappa}(\theta)\tau(z,\theta)\quad{\rm where}\quad s_{\kappa}(\theta)=
\begin{cases}
-1&,\theta\in\kappa\\
1&,{\rm otherwise.}
\end{cases}
\end{align*}
만약 어떤 signed transition $\sigma$와 적당한 $z\in(0,L)$에 대하여 {\it generated folding by $\sigma$}, $\varphi : U\to M$, $U=[0,\infty)\times[0,2\pi)$을 다음과 같이 정의할 때 $\theta\in[0,2\pi)$에서 곡선 $\varphi(\gamma(\theta))$가 단순곡선이면 이 $\sigma$를 $z$에 대한 {\it simple transition}이라 하자:
\begin{align}\label{gen}
\varphi(\rho,\psi)=\left(\rho\sqrt{1-\frac{z^2}{r(\psi)^2}},\int_0^{\psi} \sigma(z,\theta)d\theta,z\left(1-\frac{\rho}{r(\psi)}\right) \right).
\end{align}
어떤 $z$에 대한 simple transition $\sigma$에 대하여 다음 식이 성립하면 이 $\sigma$를 $z$에 대한 {\it cuttable transition}이라 하자:
\begin{align}
\int_0^{2\pi} \sigma(z,\theta)d\theta\equiv0\quad\left({\rm mod}\, 2\pi\right).
\end{align} 
\end{definition}

\begin{theorem}\label{iso}%3.1
임의의 simple transition $\sigma$와 $z\in(0,L)$에 대하여 generated folding $\varphi$는 local piecewise isometry이다.
\end{theorem}

\begin{proof}
점 $\varphi(O)$가 $z$축에 있는 것과 $\varphi(U)\cap U$가 열린 영역을 포함하지 않는 것은 자명하다.
또한 $\varphi$의 각 성분이 연속이므로 $\varphi(O)$는 연결되어 있다.
Generated folding $\varphi$의 상 $M$에 대하여 $M$의 미분불가능한 점은 면적을 가지지 않고 radius $r$이 미분불가능한 점과 같으므로 transition $\tau$가 정의되는 점에 대해서 거리가 보존됨을 보이면 충분하다.

$\varphi$의 편도함수 $\varphi_{\rho},\varphi_{\psi}$는 다음과 같다.
\begin{align*}
\varphi_{\rho}=\left(\sqrt{1-\frac{z^2}{r^2}},\,0,\,-\frac{z}r\right),\ 
\varphi_{\psi}=\left(\rho\frac{z^2r'}{r^2\sqrt{r^2-z^2}},\,\sigma(z,\psi),\,\rho\frac{zr'}{r^2}\right).
\end{align*}
극좌표를 가지는 공간 $U$와 원통좌표계를 가지는 공간 $U\times{\mathbb R}$의 거리 텐서는 각각
\begin{align*}
{\bold I}_{U}=\begin{pmatrix}1&0\\0&\rho^2\end{pmatrix},\ 
{\bold I}_{U\times{\mathbb R}}=\begin{pmatrix}1&0&0\\0&\rho\sqrt{1-\frac{z^2}{r^2}}&0\\0&0&1\end{pmatrix}
\end{align*}
이므로 $M$ 위의 거리 텐서의 성분을 어느 정도의 계산을 통해 다음과 같이 구할 수 있다.
\begin{align*}
E&=\left\langle\varphi_{\rho},\varphi_{\rho}\right\rangle=1\\
F&=\left\langle\varphi_{\rho},\varphi_{\psi}\right\rangle=0\\
G&=\left\langle\varphi_{\psi},\varphi_{\psi}\right\rangle=\rho^2.
\end{align*}
이 때의 내적은 $U\times{\mathbb R}$ 위에서의 내적이다. 우리는 다음을 얻는다.
\begin{align*}
{\bold I}_M=\begin{pmatrix}E&F\\F&G\end{pmatrix}=\begin{pmatrix}1&0\\0&\rho^2\end{pmatrix}={\bold I}_U
\end{align*}
따라서 $\varphi$는 $r$이 미분가능한 점에서 거리를 보존하므로 local piecewise isometry이다.
\end{proof}

Generated folding의 상은 generalized cone임을 확인하자.

Flat origaami model의 local-foldability의 규칙 중 Kawasaki theorem은 구간 $[0,2\pi]$를 양의 방향의 각도와 음의 방향의 각도로 분할하여 모두 더할 때 $0$이 되게끔 해야 함을 의미하고 이를 공간으로 확장할 때 signed transition의 정의는 이러한 구간의 분할을 결정한다.
정리 \ref{inc}와 관련하여 다음의 simple transition에 관한 정리를 보자.

\begin{theorem}%3.2
주어진 injective pattern $\gamma$의 transition $\tau$와 적당한 $z\in(0,L)$에 대하여 $\gamma$의 radius $r(\theta)$가 $\theta\in[a,b]$에서 증가 또는 감소함수이고 각도 $\alpha, \beta$가 $\alpha^*=\frac{2a^*+b^*}3, \beta^*=\frac{a^*+2b^*}3$를 만족할 때 $\kappa=[\alpha,\beta]$라 하면 signed transition $\sigma(z,\theta)=s_{\kappa}(\theta)\tau(z,\theta)$는 $z$에 대한 simple transition이다.
\end{theorem}

\begin{proof}
구간 $\kappa$에서 정의된 signed transition $\sigma$가 simple이 아니라고 가정하면 어떤 두 실수 $\psi_1<\psi_2\in[0,2\pi]$가 존재해 식 (\ref{gen})에 의해 $\varphi(\gamma(\psi_1))=\varphi(\gamma(\psi_2))$이므로 다음이 성립한다.
\begin{align*}
r(\psi_1)=r(\psi_2),
\end{align*}
\begin{align}\label{psieq}
\int_0^{\psi_1} \sigma(z,\theta)d\theta=\int_0^{\psi_2} \sigma(z,\theta)d\theta
\iff \int_{\psi_1}^{\psi_2} \sigma(z,\theta)d\theta=0.
\end{align}

$\psi_2<\alpha$라 가정하면 $[\psi_1,\psi_2]\cap[\alpha,\beta]=\varnothing$이므로
\begin{align*}
\int_{\psi_1}^{\psi_2} \sigma(z,\theta)d\theta=\int_{\psi_1}^{\psi_2} \tau(z,\theta)d\theta>0
\end{align*}
식 (\ref{psieq})에 모순되어 $\psi_2\ge\alpha$를 얻는다.

$\psi_1<a$라 가정하고 다음과 같이 식을 전개하자.
\begin{align*}
\int_{\psi_1}^{\psi_2} \sigma(z,\theta)d\theta=\left(a^*-\psi_1^*\right)+\left(\alpha^*-a^*\right)+\int_{\alpha}^{\psi_2} \sigma(z,\theta)d\theta.
\end{align*}
다음 부등식에 의하여
\begin{align*}
\int_{\alpha}^{\psi_2} \sigma(z,\theta)d\theta
\ge\int_{\alpha}^{\beta} \sigma(z,\theta)d\theta
=-\int_{\alpha}^{\beta} \tau(z,\theta)d\theta=-\beta^*+\alpha^*
\end{align*}
다음과 같은 식 (\ref{psieq})에 대한 모순을 얻는다.
\begin{align*}
\int_{\psi_1}^{\psi_2} \sigma(z,\theta)d\theta\ge\left(a^*-\psi_1^*\right)+\left(\alpha^*-a^*\right)-\left(\beta^*-\alpha^*\right)=a^*-\psi_1^*>0.
\end{align*}
따라서 $\psi_1\ge a$이고 비슷하게 $\psi_2\le b$를 증명할 수 있으므로 $[\psi_1,\psi_2]\subset[a,b]$이고 함수 $r(\theta)$는 구간 $[a,b]$에서 증가 또는 감소함수이므로 $r(\psi_1)=r(\psi_2)$가 성립하지 않는다.
구간 $\kappa$에서 정의된 $\sigma$가 simple이 아니라는 가정은 조건에 위배된다.
\end{proof}

만약 $\gamma$의 radius $r$이 증가 또는 감소함수가 되는 구간이 존재한다면 적당한 $\kappa$가 존재하여 simple한 signed transition을 잡을 수 있다.
다음은 transtion을 정의한 이유에 대한 정리이다.

\begin{theorem}%3.3
임의의 cuttable transition $\sigma$와 $z\in(0,L)$에 대하여 generated folding $\varphi$는 piecewise isometry, 즉 conical folding이다.
\end{theorem}

\begin{proof}
동점 ${\bold p}=(\rho,\psi)$에 대하여 반평면 $\theta=\psi,r>0$에서 점 $\varphi({\bold p})$의 자취는 다음과 같이 $\rho$에 대한 매개변수 방정식으로 나타난다.
\begin{align*}
\varphi({\bold p})_r=\rho\sqrt{1-\frac{z^2}{r(\psi)^2}},\ 
\varphi({\bold p})_z=z\left(1-\frac{\rho}{r(\psi)}\right).
\end{align*}
따라서 $\varphi$의 상 $M$은 $\theta$에 따른 직선
\begin{align*}
\varphi({\bold p})_z=-\frac{z}{\sqrt{r(\psi)^2-z^2}}\varphi({\bold p})_r+z
\end{align*}
의 자취가 되므로 generalized cone이다.

$\sigma$가 cuttable transition이므로
\begin{align*}
\int_0^{2\pi}\sigma(z,\theta)d\theta\equiv0\quad\left({\rm mod}\, 2\pi\right)\iff\varphi(\rho,0)=\varphi(\rho,2\pi)
\end{align*}
이고 generated folding $\varphi$가 ${\bold p}\in[0,\infty)\times[0,2\pi]$에서 정의가능하고 이 때 새로 정의된 $\varphi$는 연속이고 전단사이므로 위상동형사상이다.

정리 \ref{iso}에 의해 $\varphi$는 거리를 보존하므로 piecewise isometry이고 곡면이 generalized cone을 이루므로 conical folding이다.
\end{proof}

\begin{corollary}\label{34}%3.4
주어진 injective pattern $\gamma$에 대하여 적당한 $z\in(0,L)$에 대한 cuttable transition이 존재하면 $\gamma$는 cuttable하다.
\end{corollary}

\begin{proof}
식 (\ref{gen})를 따라 generated된 folding에 pattern $\gamma$를 대입하자.
\begin{align*}
\varphi\left(r(\psi),\psi\right)=\left(\sqrt{r^2-z^2},\int^{\psi}_0 \sigma(z,\theta)\,d\theta,0\right).
\end{align*}
$z$성분이 $0$이므로 generated folding $\varphi$는 cutting이다.
\end{proof}







\iffalse
(정리)커터블 충분조건 : 상수가 아니고 세타에 무관하게 제트에 대해 전구간 강증가하는 트랜지션 정의되면 커터블 트랜지션 존재
증명 : 


트랜지션의 성질 앞의 다섯가지

정리에 의해 나오는 트랜지션 성질-커터블 : 타우
 사인드 트랜지션
 심플 트랜지션
 트랜지션 적분




결론정리 : 원이 아닌 패턴이 커터블일 필요충분조건은 유계 트랜지션이 정의되는 것, 즉 필요조건1,2이다

리마크 : 폴딩에서 페네트레이션 규칙을 유지하며 단사조건을 배제하면 원도 커터블
\fi
\begin{theorem}\label{35}%3.5
점 $O$를 중심으로 하는 원이 아닌 injective pattern $\gamma$에 대하여 유계인 transition $\tau$이 정의될 때 $\tau$에 대하여 다음이 성립하면 적절한 $z_0\in(0,L)$이 존재하여 $z_0$에 대한 cuttable transition $\sigma$가 존재한다.
\begin{align*}
\sup_z\int_0^{2\pi}\tau(z,\theta)d\theta\ge0.
\end{align*}
\end{theorem}

\begin{proof}
적절한 $z_0\in(0,L)$가 존재하여
\begin{align*}
\int_0^{2\pi}\tau(z_0,\theta)d\theta=0
\end{align*}
이 성립하면 $\kappa=\varnothing$이라 할 때 다음이 성립한다.
\begin{align*}
\int_0^{2\pi}\sigma(z,\theta)d\theta=\int_0^{2\pi}\tau(z,\theta)d\theta=2\pi
\end{align*}
이 때 $\sigma$는 cuttable transition이다.

모든 $z\in(0,L)$에 대하여
\begin{align*}
\int_0^{2\pi}\tau(z,\theta)d\theta>2\pi
\end{align*}
이 성립하는 경우를 생각하자.
적절한 실수 $Z\in(0,L)$에 대하여 다음과 같이 양수 $\lambda$를 정의하자:
\begin{align*}
\lambda=\int_0^{2\pi}\tau(Z,\theta)d\theta-2\pi>0.
\end{align*}
실수 $z\in(0,Z)$와 구간 $[\alpha,\beta]\subset[0,2\pi]$에 대하여 다음과 같이 함수 $\Sigma(z;\alpha,\beta)$를 정의하자:
\begin{align*}
\Sigma(z;\alpha,\beta)=\int_0^{2\pi}\tau(z,\theta)d\theta-2\int_{\frac{2\alpha+\beta}3}^{\frac{\alpha+2\beta}3}\tau(z,\theta)d\theta.
\end{align*}
$\tau$가 유계이므로 $\tau(Z,\theta)$의 상한 $\sup_{\theta}\tau(Z,\theta)$를 ${\rm Sup}$라 간단히 쓸 때 다음이 성립한다.
\begin{align*}
\int_{\frac{2\alpha+\beta}3}^{\frac{\alpha+2\beta}3}\tau(Z,\theta)d\theta\le\frac{\beta-\alpha}3 {\rm Sup}
\end{align*}
정리 \ref{inc}에 의해 존재하는 radius $r$이 증가 또는 감소하는 구간 $[a,b]$에 대하여 $\alpha^*=\frac{2a^*+b^*}3, \beta^*=\frac{a^*+2b^*}3$라 하고 $[\alpha_0,\beta_0]\subset[\alpha,\beta]$와 $\beta_0-\alpha_0<3\lambda/2{\rm Sup}$를 만족하는 구간 $[\alpha_0,\beta_0]$를 잡으면 다음과 같은 식이 성립한다.
\begin{align*}
\Sigma(Z;\alpha_0,\beta_0)
&=\int_0^{2\pi}\tau(Z,\tau)d\theta-2\int_{\frac{2\alpha_0+\beta_0}3}^{\frac{\alpha_0+2\beta_0}3}\tau(Z,\theta)d\theta\\
&\ge\int_0^{2\pi}\tau(Z,\theta)d\theta-\frac23(\beta_0-\alpha_0){\rm Sup}\\
&>\int_0^{2\pi}\tau(Z,\theta)d\theta-\lambda=2\pi.
\end{align*}
반면 $z=0$을 대입하면 다음을 얻는다.
\begin{align*}
\Sigma(0;\alpha_0,\beta_0)=2\pi-\frac23(\beta_0-\alpha_0)<2\pi.
\end{align*}
함수 $\Sigma(z;\alpha_0,\beta_0)$는 $z$에 대하여 연속이므로 중간값 정리에 의하여 다음 식이 성립하는 $z_0\in(0,Z)$이 존재한다.
\begin{align*}
\Sigma(z_0;\alpha_0,\beta_0)=2\pi
\end{align*}
또한 $[\alpha_0,\beta_0]\subset[\alpha,\beta]$이므로 $\alpha_0^*=\frac{2a_0^*+b_0^*}3, \beta_0^*=\frac{a_0^*+2b_0^*}3$을 만족하는 구간 $[a_0,b_0]\subset[a,b]$가 존재하고 이 구간에서 $r$은 증가 또는 감소함수이다.
$\kappa$를 구간 $[\alpha_0,\beta_0]$로 정의하면
\begin{align*}
\int_0^{2\pi}\sigma(z_0,\theta)d\theta=\Sigma(z_0;\alpha_0,\beta_0)=2\pi
\end{align*}
이므로 $\sigma$는 simple transition이고 cuttable transition이다.
\end{proof}

\begin{theorem}\label{36}%3.6
점 $O$를 중심으로 하는 원이 아닌 injective pattern $\gamma$에 대하여 유계인 transition $\tau$이 정의될 때 $\tau$에 대하여 다음이 성립하면 $\gamma$는 cuttable하다.
\begin{align*}
\sup_z\int_0^{2\pi}\tau(z,\theta)d\theta\ge0.
\end{align*}
\end{theorem}

\begin{proof}
정리 \ref{34}와 정리 \ref{35}에 의해 자명하다.
\end{proof}






\iffalse
\section{Continuous distribution of Cutting}

커팅들의 집합을 잘 정의해야 되는데...

(정리)제트제로에 대한 커터블 트랜지션이 존재하면
 제트제로보다 작은 모든 제트에 대해서도 커터블 트랜지션존재


\fi





\begin{thebibliography}{99}

\bibitem{Be} M. Bern, B. Hayes {\it The complexity of flat origami}, Proceedings of the seventh annual ACM-SIAM symposium on Discrete algorithms (SODA `96), Society for Industrial and Applied Mathmatics, Philadelphia, 175-183, 1996.

\bibitem{Bo} F. Borceux {\it A Differential Approach to Geometry: Geometric Trilogy III}, Springer, 2014.

\bibitem{ON} B. O'Neill, {\it Elementary Differential Geometry}, Elsevier, 2006.

\end{thebibliography}

\end{document}














