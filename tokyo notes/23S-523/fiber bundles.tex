\documentclass{../../small}
\usepackage{../../ikhanchoi}

\begin{document}
\title{Fiber Bundles}
\author{Ikhan Choi\\Lectured by Takuya Sakasai\\University of Tokyo, Spring 2023}
\maketitle
\tableofcontents

\newpage
\section{Day 1: April 10}

References:
Steenrod, \emph{The topology of fiber bundles}, and Tamaki, \emph{Fiber bundles and homotopy} (Japanese)

\subsection*{1. Introduction}

An ultimate goal of topology is to classify topological spaces, up to homeomorphism.
If you want to show two spaces are homeomorphic, we should construct a homeomorphism: \emph{Shokuninwaza} (wild knot, Casson handle).
If you want to show two spaces are not homeomorphic, then we can investigate topological \emph{properties}, and as their quantitative comparison, we can investigate topological \emph{invariants}
Some examples include
\begin{itemize}
\item the number of connected componenets,
\item the Euler characteristic,
\item homology groups,
\item homotopy groups,
\item the minimal number of open contractible sets to cover the spaces (Lusternik-Schnirelmann category, topological complexity),
\item Gelfand-Naimark theorem: $C(X)\cong C(Y)$ implies $X\cong Y$ if they are compact Hausdorff.
\end{itemize}


We will restrict objects to study.
For example, metric spaces, manifolds, CW-complexes.
As the assumptions change, invariants may have different appearances.
For a manifold $X$,
\[\chi(X)=\sum_{q=0}^\infty(-1)^q\rk_\Z H_q(X)=\sum_{q=0}^\infty(-1)^qb_q(X).\]
For a CW-complex $X$,
\[\chi(X)=\sum_{q=0}^\infty(-1)^q(\text{the number of $q$-cells}).\]


\smallskip\hrule\bigskip

Let $M$ be an connected closed $n$-dimensional manifold.
Some classification results are as follows(up to both homeomorphisms and diffeomorphisms, because $d\le2$):
\begin{itemize}
\item ($n=0$) $M\cong *$, and $\chi(*)=1$.
\item ($n=1$) $M\cong S^1$, and $\chi(S^1)=0$.
\item ($n=2$)
	\begin{itemize}
	\item If $M$ is orientable, then $M\cong\Sigma_g$ for $g\ge0$, and $\chi(\Sigma_g)=2-2g$.\\
	$\Sigma_0\cong S^2$, $\Sigma_1\cong T^2$.
	\item If $M$ is not orientable, then $M\cong(\RP^2)^{\# h}$ for $h\ge1$, and $\chi((\RP^2)^{\#h})=2-h$.\\
	$\RP^2(\cong\text{M\"obius strip}\cup D^2)$, $K=\RP^2\#\RP^2$
	\end{itemize}
\end{itemize}

\textbf{Problem 1.} Show $\RP^2\#T^2\cong\RP^2\#K$.
\bigskip

Here are some facts about triangulability:
\begin{itemize}
\item Cairns(1935), Whitehead (1940): every $C^1$-manifold is triangulable (unique as a PL-manifold).
\item Rado(1925, $n=2$), Moise(1952, $n=3$): for $n\le3$, every $C^0$-manifold is triangulable (unique as a PL-manifold).
\item Kirby-Siebermann(1966, $n\ge5$): for $n\ge4$, there is a non-triangulable PL-manifold.
\item Donaldson, Freedman, Casson: for $n=4$, there is a non-triangulable manifold as a topological space.
\item Manolescu(2013): for $n\ge5$, there is a non-triangulable manifold as a topological space.
\end{itemize}

Orientability?
For a connected closed surface $S$, it is orientable iff $H_2(S)\cong\Z$, not orientable iff $H_2(S)\cong0$. The generator of $H_2(S)$ is called the fundamental class.
Orientability asks if the tubular neighborhood of every simple closed curve is homeomorphic to an anulus.
It is described by the first Stiefel-Whitney class:
\[w_1(S)\in H^1(S;\Z/2\Z)\cong\Hom(H^1(S),\Z/2\Z)\cong\Hom(\pi_1(S),\Z/2\Z).\]

\smallskip\hrule

\subsection*{Euler characteristic of manifolds}

\subsubsection*{(0) Odd-dimensional manifolds}

\begin{thm*}
For an odd-dimensional closed connected manifold, $\chi(M^{2n+1})=0$.
\end{thm*}
\begin{pf}
If orientable, then $b_0(M)=1$, $b_3(M)=1$, $b_1(M)=b_2(M)$ by the Poincar\'e duality.
If not, a double cover is orientable, and $\chi(\tilde M)=2\chi(M)$.
\end{pf}


\subsubsection*{(1) Gauss-Bonnet theorem}
\begin{thm*}[Gauss-Bonnet]
If a smooth manifold $M^n$ embeds into $\R^{n+1}$ (hypersurface), then it is orientable and the Euler characteristic is given by
\[\chi(M)=\frac2{\vol(S^n)}\int_MK\,d\vol_M.\]
\end{thm*}





\section{Day 2: April 17}

We have a cohomological interpretation.
In the Chern-Weil theory, we have a generalized version of the Gauss-Bonnet theorem for a general compact manifold using the theory of connections.
We can interpret $2\vol(S^n)^{-1}K\cdot d\vol_M$ as a differential form which provides with the Euler characteristic.
In the context of the de Rham theorem, we will eventually call the equivalence class of this differential form as the \emph{Euler class}.

\subsubsection*{(2) Poincar\'e-Hopf theorem}

Let $M^n$ be a orientable connected smooth closed manifold.
Let $X$ be a smooth vector field on $M$ such that there are only finitely many zeros $\{p_1,\cdots,p_m\}$.
For each $p_j$, define the index $\Ind(X,p_j)$ as follows: 
seeing $X$ as a vector field on $\f_j(U_j)$ for a chart $(U_j,\f_j)$ not containing zeros of $X$ but $p_j$ and mapping $p_j$ to zero in $\R^n$, we define $\Ind(X,p_j)=\deg f_j$, where $f_j:S_\e(\approx S^{n-1})\to S^{n-1}:x\mapsto X_x/\|X_x\|$.

\begin{ex*}
Let $n=2$.
We have indices $1, 1, 1, -1, 0, 2$ for
\[X_1(x,y)=(x,y),\quad X_2(x,y)=(-x,-y),\quad X_3(x,y)=(-y,x),\]
\[X_4(x,y)=(-x,y),\quad X_5(x,y)=\sqrt{x^2+y^2}(1,1),\quad X_6(x,y)=(x^2-y^2,2xy).\]
\end{ex*}

\begin{thm*}[Poincar\'e-Hopf]
\[\sum_{j=1}^m\Ind(X,p_j)=\chi(M).\]
\end{thm*}

We have a cohomological interpretation.
Let $c=\sum_{j=1}^m\Ind(X,p_j)p_j$ be a singular 0-cycle on $M$.
Then, the Poincar\'e-Hopf theorem states that we have
\[\begin{array}{ccc}
H_0(M)&\xrightarrow{\sim}&\Z\\
p_j&\mapsto&1\\
c&\mapsto&\chi(M).
\end{array}\]
By the Poincar\'e duality, we can identify the homology class $[c]$ with a de Rham cohomology class, and the above map is just an integration map.

The cycle $c$ tells us the information of intersections of $X$ and zero section(of the tangent bundle).
If $TM$ is trivial, then the zero section does not self-intersection(?) so that $c=0$.
The Euler characteristic measures the twist of a bundle, and the characteristic class generalizes this wakugumi.


\subsection*{2. Fiber bundles}

From now we will only consider paracompact Hausdorff spaces.
Recall that a space is paracompact iff for every open cover there is a locally finite refinement.
\begin{ex*}
Open sets of $\R^n$, metric spaces, CW-complexes, countable inductive limit of compact spaces are paracompact.
\end{ex*}
\begin{thm}
For any open cover of a paracompact Hausdorff space $X$, there is a partition of unity subordinate to it.
\end{thm}

\textbf{Problem 2.} Prove the above theorem.

\begin{defn}
Let $B$ be connected(for simplicity).
A map $E\to B$ is called a fiber bundle with fiber $F$, or just a $F$-bundle, if it is locally trivial: every point $x\in B$ has an open neighborhood $U_x$ such that there is a homeomorphism $\f:p^{-1}(U_x)\to U_x\times F$ with $p=\pr_{U_x}\circ\f$.

For each $y\in B$ $E_y:=p^{-1}(y)$ is homeomorphic to $F$, and is called the fiber at $y$.
Also, $E$ and $B$ are called the total space and the base space.
We somtimes write as $\xi=(F\to E\xrightarrow{p}B)$.
\end{defn}
\begin{ex*}\,
\begin{parts}
\item We say $\pr_1:B\times F\to B$ is the product or bundle.
\item $p:\R\to S^1=\R/\Z:t\mapsto[t]$ is a $\Z$-bundle. In general, a fiber bundle with a discrete fiber is called a covering space.
\item $p_1:S^n\to\RP^n=S^n/(x\sim-x)$ is a $\Z/2\Z$-bundle.
\item $p:S^{2n+1}\to\CP^n=S^{2n+1}/(z\sim uz)$ for $u\in S^1$ is a $S^1$-bundle. (a generalization of Hopf bundles)
\item Let $M^n$ be a smooth manifold. Then, the tangent and the contangent bundles are $\R^n$-bundles.
\end{parts}
\end{ex*}

\textbf{Problem 3.} Show that $p:S^{2n+1}\to\CP^n$ is a $S^1$-bundle by checking concretely its local triviality.

\begin{defn}
If $F,E,B$ are $C^r$, $p:E\to B$ is $C^r$, and the local trivialzation is $C^r$, then we say the fiber bundle is $C^r$.
\end{defn}
\begin{defn}
For $\xi_1=(F\to E_1\xrightarrow{p_1}B_1)$, $\xi_2=(F\to E_2\xrightarrow{p_2}B_2)$, a bundle map $\Phi=(\tilde f,f):\xi_1\to\xi_2$ is a pair of maps $\tilde f:E_1\to E_2$ and $f:B_1\to B_2$ such that $f\circ p_1=p_2\tilde f$ and the restriction $\tilde f:p_1^{-1}(b)\to p_2^{-1}(f(b))$ is a homeomorphism for every $b\in B$.

If both $f$ and $\tilde f$ are homeomorphisms, then $\Phi$ is called a bundle isomorphism.
If a bundle is isomorphic to a product bundle, then it is called to be trivial.
\end{defn}

\textbf{Problem 4} For a bundle map $\Phi$, is $\tilde f$ homeomorphic if $f$ is homeomorphic? (If we are doing in the category of smooth manifolds, then the inverse function theorem may be helpful.)


\newpage
\section{Day 3: April 24}


\subsubsection*{Transition maps and structure groups}
Let $\xi=(F\to E\xrightarrow{p}B)$ be an $F$-bundle.
We have an open cover $\{U_\alpha\}$ such that for each $\alpha$ we have a local trivialization $p^{-1}(U_\alpha)\xrightarrow{\sim}U_\alpha\times F$.
For $U_\alpha\cap U_\beta\ne\varnothing$, we have a map
\[\f_\alpha\circ\f_\beta^{-1}:(U_\alpha\cap U_\beta)\times F\to(U_\alpha\cap U_\beta)\times F,\]
by which we can define $\tilde g_{\alpha\beta}:(U_\alpha\cap U_\beta)\times F\to F$ such that $\f_\alpha\circ\f_\beta^{-1}(b,f)=:(b,\tilde g_{\alpha\beta}(b,f))$.
The map $\tilde g_{\alpha\beta}$ is continuous, and we have for each $b$ a homeomorphism
\[g_{\alpha\beta}(b):F\to F:f\mapsto\tilde g(b,f),\]
that is, $g_{\alpha\beta}:U_\alpha\cap U_\beta\to\Homeo(F)$.
If we endow the compact-open topology on $\Homeo(F)$, then $g_{\alpha\beta}$ is continuous.

From definition, $g_{\alpha\beta}(b)\circ g_{\beta\alpha}(b)=\id_F$ for $b\in U_\alpha\cap U_\beta\ne\varnothing$, and $g_{\alpha\beta}(b)\circ g_{\beta\gamma}(b)=g_{\alpha\gamma}(b)$ for $b\in U_\alpha\cap U_\beta\cap U_\gamma\ne\varnothing$
(Note that the second relation implies the first.).
The second condition is called the cocycle condition.
The maps $\{g_{\alpha\beta}\}$ are called transition maps.

\setcounter{section}{2}
\setcounter{thm}{4}
\begin{thm}
Let $\{U_\alpha\}$ be an open cover of a connected space $B$.
Suppose we have a collection of continuous maps
\[\{g_{\alpha\beta}:U_\alpha\cap U_\beta\to\Homeo(F)\}_{(\alpha,\beta):U_\alpha\cap U_\beta\ne\varnothing}\] satisfying the cocycle condition.

($\spadesuit$) Suppose also that $F$ is locally compact, or there exists a topological transformation group $G$(i.e. $G$ is a topological group such that the group action $G\times F\to F$ is continuous) with
\[\bigcup_{\alpha,\beta}g_{\alpha\beta}(U_\alpha\cap U_\beta)\subset G\subset\Homeo(F).\]

Then, there exists a unique $F$- bundle $(F\to E\xrightarrow{p}B$ such that it is locally trivializable over $\{U_\alpha\}$ and $\{g_{\alpha\beta}\}$ is the transition maps of the bundle.
\end{thm}

The viewpoint of the above theorem is more likely to be the physicist's way of defining manifolds in the sense that they sometimes deifne a manifold as a collection of open subsets of a Euclidean space and transition maps between them.

The condition ($\spadesuit$) gaurantees for the second map in
\[\tilde g_{\alpha\beta}:(U_\alpha\cap U_\beta)\times F\to(U_\alpha\cap U_\beta)\times\Homeo(F)\times F\to(U_\alpha\cap U_\beta)\times F\]
\[(b,f)\mapsto(b,g_{\alpha\beta}(b),f)\mapsto(b,g_{\alpha\beta}(f))\]
to be continuous.

\begin{pf}(Sketch)
Define
\[\tilde E:=\coprod U_\alpha\times F\]
and $E:=\tilde E/\sim$, where the equivalence relation $\sim$ is generated by: for each $(b_1,f_1)\in U_\alpha\times F$ and $(b_2,f_2)\in U_\beta\times F$ we have $(b_1,f_1)\sim(b_2,f_2)$ iff $b_1=b_2$ and $f_1=g_{\alpha\beta}(b_2)(f_2)$.
Let $\pi:\tilde E\to E$ be the canonical projection.
Define also
\[\f_\alpha:p^{-1}(U_\alpha)\to U_\alpha\times F:[(b,f)\in U_\alpha,F]\mapsto (b,f),\] which are homeomorphisms by the assumption ($\spadesuit$), satisfying $\pr_1\circ\f_\alpha=p$.
\end{pf}

For the second condition in (\spadesuit), $G$ is called a structure group of the $F$-bundle.
From now on, whenever we consider a fiber bundle along with a structure group $G$, we assume it includes the data of local trivialization.

\begin{rmk*}
We will always think of $G$ for bundle maps between fiber bundles with structure group $G$.
We will frequently consider the maximal transition data and compatible(i.e. satisfying the cocycle condition) local trivializations.
\end{rmk*}
\begin{ex*}\,
\begin{enumerate}
\item Let $F=V\cong\R^n$ be a real vector space, and $G\in\{\GL(V),\SL(V)\}$ or $G\in\{\rO(V),\SO(V)\}$ with a fixed inner product on $V$. These fiber bundles are called real vector bundles.
\item Let $F=V\cong\C^n$ be a complex vector space, and $G\in\{\GL_\C(V)\}$ or $G\in\{\rU(V)\}$ with a fixed inner product on $V$. These fiber bundles are called complex vector bundles.
\item $F=G$ be a Lie group. Then, $G$-bundle with structure group $G$ is called a principal bundle.
\item Let $F$ be a nice smooth manifold and $G=\Diff^{C^\infty}(F)$ be the group of smooth diffeomorphisms together with the Fr\'echet topology. Then, we have smooth $F$-bundles.
\end{enumerate}
\end{ex*}

\begin{defn}
Let $G$ be a structure group and $B$ be a topological space.
If an $F$-bundle $\xi=(F\to E\to B,G)$ and an $F'$-bundle $\xi=(F'\to E'\to B,G)$ has the same transition data, then they are called associated bundles.
\end{defn}


\begin{ex*}
Let $F=\R^n$ be a real vector space with the standard inner product.
Let $G=O(n)$.
With $S^{n-1}\subset F$, the sphere bundle inside a real vector bundle is an associated bundle of the original real vector bundle.
In particular for $n=2$ and $G=\SO(2)$, then the circle bundle can be recognized as a principal $SO(2)$-bundle associated to a real plane bundle, and if we see the plane bundle as a complex line bundle, then it corresponds to a pricipal $U(1)$-bundle.
\end{ex*}

\begin{prop}
Let $G$ be a topological group and $\xi=(G\to E\to B,G)$ be a principal $G$-bundle.
Then, there is a natural right action of $G$ on $E$ which is free and the orbit space $E/G$ is homeomorphic to $B$(transitively act on each fiber).
\end{prop}
\begin{pf}
Let $u\in E$ and $\f_\alpha$ a local trivialization containing $u$ such that
\[\f_\alpha:p^{-1}(U_\alpha)\to U_\alpha\times G:u\mapsto(p(u),h).\]
We can check the well-definedness of $ug=\f_\alpha^{-1}(p(u),hg)$ by
\[\f_\beta(ug)=\f_\beta\circ\f_\alpha^{-1}(p(u),hg)=(p(u),g_{\beta\alpha}(p(u))(hg))=(p(u),h'g).\]

The right action of $G$ on $G$ is continuous, free, and transitive.
The right action of $G$ on $E$ is continuous and free, and $\bar p:E/G\to B$ is continuous and bijective.
\end{pf}

\textbf{Problem 5.} Show that $\bar{p}^{-1}$ is also continuous.

\begin{rmk*}
A principal $G$-bundle may also be defined as follows: a $G$-bundle such that (1) there is a continuous free right action of $G$ on $E$ which is (2) fiber-preserving and fiberwise transitive, and (3) we can choose \emph{$G$-equivariant} local trivialization such that $\f_\alpha(u)=(p(u),h)$ implies $\f_\alpha(ug)=(p(u),hg)$.
\end{rmk*}

\newpage
\setcounter{section}{3}
\section{Day 4: May 1}
\setcounter{section}{2}
\setcounter{thm}{7}

Let $G$ be a topological group.
A pricipal $G$-bundle $(G\to E\to B,G)$ has a continuou free action of $G$ on $E$.

\begin{rmk*}
For two principal $G$-bundles, $(\tilde f,f)$ is a bundle map if and only if $\tilde f$ is $G$-equivariant.
\end{rmk*}

\begin{defn}
Let $\xi=(F\to E\xrightarrow{p}B)$ be a fiber bundle.
A continuous map $s:B\to E$ such that $p\circ s=\id_B$ is called a section or a cross section.

An important question asks if there is a section globally defined on the whole $B$.
\end{defn}

\begin{prop}
Let $\xi=(G\to E\to B,G)$ be a principal $G$-bundle.
Then, $\xi$ is trivial if and only if it admits a global section.
\end{prop}

\begin{pf}
($\Rightarrow$)
Clear.

($\Leftarrow$)
Let $s:B\to E$ be a global section.
Define
\[\Phi:B\times G\to E:(b,g)\mapsto s(b)g.\]
Then, it is an $G$-equivariant isomorphism.
\end{pf}


Let $X$ be a right $G$-space which is free.
Then, is $X/G$ a principal $G$ bundle?
We have two problems:
\begin{parts}
\item Is the inverse image($=$orbit) of each point of $X/G$ homeomorphic to $G$?
No, the dynamics $\T^2\curvearrowright\R$ with irrational slope.
\item Does it satisfy the local triviality?
No, the translation $\R\leftarrow\Q$.
\end{parts}

\begin{prop}
Let $X$ be a right $G$-space which is free.
The quotient map $\pi:X\to X/G$ defines a principal $G$-bundle if and only if $X\curvearrowright G$ strongly freely(i.e. $X\times X\to G:(x,xg)\mapsto g$ is continuous) and there is a local trivialization for some $y\in X/G$.
\end{prop}
\begin{pf}
($\Rightarrow$) Clear.

($\Leftarrow$)
\[\pi^{-1}(U)\to U\times G:s(x)g\mapsto(x,g)\]
is continuous by the strongly free action.
It defines local trivializations.
\end{pf}

\begin{thm}[Gleason, 1950]
Let $M$ be a smooth manifold and $G$ a compact Lie group which gives a free right smooth action on $M$.
Then, $M/G$ is a smooth manifold such that $M\to M/G$ is a principal $G$-bundle.
\end{thm}
(Compactness of $G$ implies the properness of the action, and smoothness implies the local triviality)
\begin{cor}[Samelson, 1941]
Let $H$ be a compact Lie subgroup of a Lie group $G$.
Then, $G\to G/H$ is a principal $H$-bundle.
In fact, it is sufficient for $H$ to be a closed subgroup of $G$, even if it is not compact.
\end{cor}
\begin{ex*}\,
\begin{parts}
\item
With an action $S^{2n+1}\curvearrowright S^1$ such that $(z_0,\cdots,z_n)w=(z_1w,\cdots,z_nw)$, we have an $S^1$-bundle
\[S^{2n+1}\to\CP^n:(z_0,\cdots,z_n)\mapsto[z_0:\cdots:z_n].\]
\item
For $k\le n$, the Stiefel variety is
\[V_k(\R^n):=\{M\in M_{n,k}(\R):\rk M=k\}.\]
Also define
\[V_k^0(\R^n):=\{M\in V_k(\R^n):\text{column vectors of $M$ are orthonormal}\}\]
and the Grassmannian manifold
\[G_k(\R^n):=\{\text{$k$-dimensional subspaces of $\R^n$}\}.\]
With an action $V_k(\R^n)\curvearrowright\GL(k,\R)$ such that $(v_1,\cdots,v_k)X=(v_1X,\cdots,v_kX)$, we have $G_k(\R^n)\cong V_k(\R^n)/\GL(k,\R)$ and $G_k(\R^n)\cong V_k^0(\R^n)/\rO(k)$.
Then, $(\rO(k)\to V_k^0(\R^n)\to G_k(\R^n))$ and $(\GL(k,\R)\to V_k(\R^n)\to G_k(\R^n))$ are principal bundles.
\item
As a complex version of (b), we have principal bundles $(\rU(k)\to V_k^0(\C^n)\to G_k(\C^n))$ and $(\GL(k,\C)\to V_k(\C^n)\to G_k(\C^n))$.
\end{parts}
\end{ex*}


\begin{thm}
Let $M$ be smooth manifold and suppose we have a transitive smooth left action of a Lie group $G$ on $M$.
Let $H$ be \emph{the} isotropy group.
Then, $G/H\to M$ defines a diffeomorphism and $(H\to G\to M)$ is a principal bundle.
Such $M$ is called a homogeneous space.
\end{thm}

\begin{ex*}
With an action $\SO(n)\curvearrowleft S^{n-1}$, since the isotropy group is isomorphic to $\SO(n-1)$, we have a principal bundle $SO(n-1)\to SO(n)\to S^n$.

We can also see the examples above(Grassmann and Steifel manifolds) as principal bundles on homogeneous spaces with a diffeomorphsim $\rO(n-k)\setminus\rO(n)\to V_k^0(\R^n):[A]\mapsto(Ae_1,\cdots,Ae_k)$ and $\rO(n)/\rO(n-k)\times\rO(k)\cong G_k(\R^n)$: principal $\rO(k)$-bundle
\begin{cd}
\rO(k)\ar{r} & V_k^0(\R^n) \ar{d}\ar{r}{\sim}& \rO(n)/\rO(n-k)\\
& G_k(\R^n) \ar{r}{\sim} & \rO(n)/\rO(n-k)\times\rO(k).
\end{cd}
We also have a complex version.


\end{ex*}

\end{document}