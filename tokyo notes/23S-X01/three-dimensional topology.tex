\documentclass{../../small}
\usepackage{../../ikhanchoi}

\begin{document}
\title{Three-dimensional Topology}
\author{Ikhan Choi\\Lectured by Takahiro Kitayama\\University of Tokyo, Spring 2023}
\maketitle
\tableofcontents

\newpage
\section{Day 1: April 11}

Plan:
\begin{enumerate}
\item Fundamental groups of manifolds
\item Examples and constructions
\item Prime decomposition
\item Loop and sphere theorems
\item Haken manifolds
\item Seifert manifolds
\item JSJ composition
\item Geometrization
\item Virtually special theorems
\end{enumerate}

\noindent References:
\begin{enumerate}
\item J. Hempel, 3-manifolds
\item W. Jaco, Lectures on three-manifold topology
\item B. Martelli, An introduction to geometric topology
\item Morimoto, An introduction to three-dimensional manifolds (Japanses)
\end{enumerate}

\noindent Grading:
Submit a report for any three among the exercises given in the lecture (ITC-LMS Kadai).
Cancellation of class: 5/2, 7/11(maybe)

\noindent Convention:
\begin{itemize}
\item manifold = connected compact orientable smooth manifold
\item surface = connected compact orientable smooth 2-dimensional manifold
\item tub nbd, isotopy extension, transversality, triangulation,$\cdots$
\end{itemize}

\subsection*{1. Fundamental group}
\subsubsection*{1.1. Fundamental groups of higher dimensional manifolds}
\begin{prop}
Let $\pi$ be a finitely presented group.
Then, for every $d\ge4$ there is a $d$-manifold $X$ such that $\pi_1(X)\cong\pi$.
\end{prop}
\begin{pf}
Let $\pi=\<x_1,\cdots,x_n\mid r_1,\cdots r_m\>$ be given.
If $Y=(S^1\times S^{d-1})^{\#n}$, then $\pi_1(Y)\cong\<x_1,\cdots\,x_n\>$ by the van Kampen theorem.
Let
\[Z=Y\setminus(\coprod_{i=1}^m\nu(l_i)),\]
where $l_i\subset Y$ is embedded loops representing $r_i$ and $\nu$ denotes the open tubular neighborhood.
Then, $\pd Z=\coprod_{i=1}^ml_i\times S^{d-2}$.
Since $d\ge4$, any loops and disks can be pushed off $l_1\cdots,l_n$, we have an isomorphism $\pi_1(Z)\to\pi_1(Y)$.
Then, if we let
\[X=Z\cup_{\pd}(\coprod_{i=1}^mD^2\times S^{d-2}),\] then $l_i\times*=\pd(D^2\times*)$, we have $\pi_1(X)\cong\pi_1(Y)/\<[l_1],\cdots,[l_m]\>\cong\pi$.
\end{pf}

\subsubsection*{1.2. Surfaces and their groups}

\begin{thm}[Rad\'o, Whitehead]
Every topological surface admits a unique smooth and PL structure.
\end{thm}

\begin{thm}
Every surface is diffeomorphic to only one of $\Sigma_{g,b}$, where $\Sigma_{g,b}=(T^2)^{\#g}\#(D^2)^{\#b}$.
\end{thm}

\begin{cor}
$S=S^2,T^2,D^2$ are prime, that is, $S=S_1\#S_2$ implies $S_i\approx S^2$ for $i=1$ or $i=2$.
\end{cor}

\begin{rmk}
If the orientability is reduced out, then $\RP^2$ is prime.
Also note that $T^2\#\RP^2\approx(\RP^2)^{\#3}$.
\end{rmk}

\begin{thm}[Uniformization]
For every surface $S\ne D^2$, its interior admits a complete Riemannian metric of constant curvature
\[\begin{cases}1,&\chi(S)>0\\0,&\chi(S)=0\\-1,&\chi(S)<0\end{cases}\]
with universal covering $S^2$, $\R^2$, $\H^2$, respectively.
\end{thm}

The hyperbolic plane is $\H^2=\{(x,y)\in\R^2:y>0\}$ with the Riemannian metric $ds^2=(dx^2+dy^2)/y^2$, and $\Isom^+(\H^2)=\PSL_2(\R)$.

\begin{prop}
If a surface $S$ has $\chi(S)<0$, then there is a discrete group $\Gamma\le\PSL_2(\R)$ such that $S\approx\H^2/\Gamma$.
In particular, $\pi_1(S)$ is isomorphic to $\Gamma$.
\end{prop}
We have
\[\pi_1(\Sigma_{g,b})\cong F_{2g+b-1}\quad\text{ and }\quad\pi_1(\Sigma_g)\cong\<a_1,b_1,\cdots,a_g,b_g\mid[a_1,b_1],\cdots,[a_g,b_g]\>.\]

\begin{prop}
$\pi_1(\Sigma_g)$ is torsion free.
\end{prop}

\textbf{Exercise 1.} Prove Proposition 1.7.


\begin{thm}[Newman]
$\pi_1(\Sigma_{g,b})$ is linear over $\Z$, that is, is isomorphic to a subgroup of $\GL_n(\R)$.
For example, we can check $F_n\hookrightarrow F_2\hookrightarrow \SL_2(\Z)$ according to the pingpong lemma.
\end{thm}


Over $\R$, we may embed $\pi_1(S)\hookrightarrow\PSL_2(\Z)\cong\SO_{1,2}^+(\R)<\GL_3(\R)$ if $\chi(S)<0$,

\begin{defn*}
A group $\pi$ is called residually finite(RF) if for every $1\ne\gamma\in\pi$ there is a group homomorphism $\f:\pi\to G$ to a finite group $G$ such that $\f(\gamma)\ne1$.
A subgroup $\pi'<\pi$ is called separable if there is a group homomorphism $\f:\pi\to G$ to a finite group $G$ such that $\f(\gamma)\notin\f(\pi')$.
In particular, $\pi$ is residually finite if the trivial subgroup is separable in $\pi$.
A group $\pi$ is called locally extended residually finite(LERF) if every finitely generated subgroup of $\pi$ is separable.
\end{defn*}

\begin{thm}[Mal'cev]
Every finitely generated linear group over a field is residually finite.
\end{thm}

Over $\Z$, if $1\ne(a_{ij})\in\pi<\GL_n(\Z)$ given, then for $m>\max_{i,j}|a_{ij}|$ if we let $\f_m:\pi\hookrightarrow\GL_n(\Z)\twoheadrightarrow\GL_n(\Z/m\Z)$, then $\f_m((a_{ij}))\ne1$.

\begin{thm}[Scott]
$\pi_1(\Sigma_{g,b})$ is LERF.
\end{thm}


\section{Day 2: April 18}

\subsection*{Examples and constructions of 3-manifolds}

\begin{thm}[Moise]
Every topological 3-manifold(not neccesarily compact, connected, orientable) admits a unique smooth and PL structure.
\end{thm}

\subsection{Spherical manifolds}

Recall
\begin{align*}
S^3:&=\{x\in\R^4:|x|=1\}\\
&=\{(z,w)\in\C^2:|z|^2+|w|^2=1\}\\
&=\{a+bi+cj+dk\in\H:a^2+b^2+c^2+d^2=1\}.
\end{align*}
\subsubsection*{Lens spaces}
Let $p,q\in\Z$, $p>0$, $(p,q)=1$.
Then, $\Z/p\Z=\<\zeta=\exp(2\pi\sqrt{-1}/p)\>$ acts on $S^3$ such that $\zeta\cdot(z,w)=(\zeta z,\zeta^qw)$.
Then, the Lens spaces are defined as
\[L(p,q):=S^3/(\Z/p\Z)\quad\text{ with }\quad\pi_1(L(p,q))=\Z/p\Z.\]
For example, $L(1,1)=S^3$ and $L(2,1)=\RP^3$.


\begin{thm}[Reidemeister]\,
\begin{parts}
\item $L(p,q)\simeq L(p,q')$ (homotopy equiv) if and only if there is $a\in\Z$ such that $qq'\equiv\pm a^2\pmod p$.
\item $L(p,q)\approx L(p,q')$ (diffeo) if and only if $q'\equiv\pm q^{\pm1}\pmod p$.
\end{parts}
\end{thm}

For example, $L(7,1)\simeq L(7,2)$ since $1\cdot2\equiv 3^2\pmod7$, but $L(7,1)\approx L(7,2)$ since $2\not\equiv\pm1\pmod7$.
\begin{pf}[Sketch of ($\Leftarrow$)]
Direct construction.
(a) With the linking form $H_1(L)\times H_1(L)\to\Q/\Z$.
(b) Reidemeister torsion. 
\end{pf}

\subsubsection*{General quotients}
A spherical manifold is the orbit space $S^3/\Gamma$, where $\Gamma$ is a finite subgroup of $\SO(4)$ and $\Gamma\curvearrowleft S^3$ freely.

\begin{ex*}
With an action $\<-1,i,j,k\>\curvearrowleft S^3$, we obtain the prism manifold.
\end{ex*}
\begin{ex*}
With an action of the binary icosahedral group $\Gamma=\Z/2\Z\rtimes A_5$ on $S^3$, we obtain the Poincar\'e sphere.
We have $H_*(S^3/\Gamma)\cong H_*(S^3)$.
If we take 3/10 turn instead of 1/10, we have the Seifert-Weber space.
\end{ex*}

\subsection{Fibered manifolds}


\subsubsection*{Twisted bundles}
\[N_{g,b}=(\RP^2)^{\#g}\#(D^2)^{\#b}.\]

Let $D$ be a polygon with oriented sides $a_1,a_1',a_2,a_2',\cdots,a_g,a_g'$.
\[N_g\tilde\times[0,1]:=D\times[0,1]/\sim,\] where $(x,t)\sim(x',1-t)$ for $x\in a_i$, $x'\in a_i'$, $t\in[0,1]$ with $[x]=[x']\in N_g$, and it is orientable.
\[N_g\tilde\times S^1:=N_g\tilde\times[0,1]/(x,0)\sim(x,1),\ x\in N_g\]
\[N_{g,b}\tilde\times S^1:=N_g\tilde\times S^1\setminus\nu(b\text{ fibers}).\]

\textbf{Exercise 2.} Show the following:
\begin{parts}
\item $\RP^2\tilde[0,1]\approx\RP^3\setminus\text{(open ball)}$.
\item $\RP^2\tilde\times S^1\approx\RP^3\#\RP^3$.
\item $N_{1,1}\tilde\times S^1\approx N_2\tilde\times[0,1]$.
\end{parts}

\subsubsection*{Mapping tori}
The mapping class group is
\[\cM_{g,b}:=\mathrm{Diff}^+(\Sigma_{g,b},\pd\Sigma_{g,b})/\text{isotopy relative to }\pd.\]

\begin{thm}[Dehn, Lickorish]
$\cM_{g,b}$ is finitely generated by Dehn twists.
\end{thm}
For examples, $\cM_0=\cM_{0,1}=1$ by the Alexander trick, and $\cM_{0,2}\cong\Z$, $\cM_1=\cM_{1,1}\cong\SL_2(\Z)$.

Let $\f\in\cM_{g,b}$.
Then, a mapping torus is defined by
\[M_\f:=\Sigma_{g,b}\times[0,1]/(\f(x),0)\sim(x,1).\]

\subsection{Heegaard decomposition}

A manifold with a boundary
\[H_g:=D^3\cup(D^2\times[0,1])^{\sqcup g}\]
is called the handle body with genus $g$.
Then, $\pi_1(H_g)\cong F_g$ and $\pd H_g\approx\Sigma_g$.
Let $\f:\pd H_g\to\pd H_g$ be an orientation-preserving diffeomorphism, i.e. an element of the mapping class group $\cM_g$.
If a 3-manifold $M$ satisfies
\[M\approx H_g\approx H_g,\]
then the right-hand side is called the Heegaard decomposition(splitting) of $M$.
\begin{prop}
Every closed 3-manifold admitting a Heegaard decomposition of genus 0 is diffeomorphic to $S^3$.
\end{prop}
\begin{pf}
$\cM_0=1$.
\end{pf}
\begin{prop}
Every closed 3-manifold admitting a Heegaard decomposition of genus 1 is diffeomorphic to $S^3$, $S^2\times$, or $L(p,q)$.
\end{prop}
\textbf{Exercise 3.} Prove the above proposition.
\begin{thm}
Every closed 3-manifold $M$ admits a Heegaard decomposition along some $\Sigma_g$.
\end{thm}
\begin{pf}
Pick a triangulation $T$ of $M$.
Then, $H:=\bar{\nu(T^{(1)})}$ is a handlebody.
Then, $H':=M\setminus\mathrm{Int}H$ is also a handlebody.
Since $M$ is orientable, so are $H$ and $H'$, thus we are done.
\end{pf}
There is another proof using Morse theory.

\begin{cor}
The fundamental group of every closed 3-manifold admits a finite presentation of deficiency 0, i.e. the number of generators is equal to the number of relations.
\end{cor}
\begin{pf}
Apply the van Kampen theorem to
\[M=H_g\cup H_g=H_g\cup(D^2\times[-\e,\e])^{\sqcup g}\cup D^3.\qedhere\]
\end{pf}

\subsection{Dehn surgery}
Let $L$ be a link.
The link exterior is the set $E_L=S^3\setminus\nu(L)$.

\begin{prop}
Let $M$ be a 3-manifold, and $T\subset\pd M$ a torus component.
Let $h:\pd(D^2\times S^1)\to T$ be a diffeomorphism.
Then, $M\cup_h(D^2\times S^1)$ is determined only by $\pm[h(\pd D^2\times*)]\in H_1(T)$.
\end{prop}
\begin{pf}
Write
\[M\cup_h(D^2\times S^1)=M\cup_h(D^2\times(-\e,\e))\cup D^3.\qedhere\]
\end{pf}
For a knot $K$, there are two generators $\mu$ and $\lambda$, called the meridian and the longitude, of $H_1(\pd E_K)$ such that $\ker(H_1(\pd E_K)\to H_1(E_K)=\Z\mu)$ is generated by $\lambda$.

\textbf{Exercise 4.} Show that $L(p,q)$ with $p\ne0$ and $S^2\times S^1$ are obtained by the $p/q$-Dehn surgery along the unknot.

\begin{thm}[Lickorish-Wallace]
Every closed 3-manifold can be obtained by an (integral)-Dehn surgery along some link in $S^3$.
\end{thm}
\begin{pf}[Sketch]
Heegaard decomposition and $\cM_g=\<\text{Dehn twists}\>$.
Each Dehn twist realizes the Dehn surgery steps.
\end{pf}

\newpage
\section{Day 3: April 25}
\subsection*{Prime decomposition}

\subsection{Alexander's theorem}
\begin{thm}
Every (smooth) embedding $S^2\subset\R^3$ bounds some (smooth) embedding $D^3\subset\R^3$.
\end{thm}
\begin{rmk*}
The above theorem does not hold in the category of topological spaces.
Alexander's horned sphere is one of the counterexamples.

If $\R^d$ for $d\ge5$, then more complicated result such as h-cobordism theorem must be used to obtain the same conclusion.
\end{rmk*}
\begin{pf}[Sketch]
Isotope such a sphere $S$ so that the coordinate $z:S\to\R$ is a Morse function.
Assume that for all $p\ne q\in\mathrm{Crit}(z)$, then $z(p)\ne z(q)$.
We use induction on $(m,n)$, where $m$ is the number of saddles and
\[n:=\min\{\#\pi_0(S\cap z^{-1}(r)):\text{$r$ is a regular value s.t. $z(p)<r<z(q)$ for some saddles $p,q$}\}.\]
Note that $\#(\text{minima of $z$})-m+\#(\text{maxima of $z$})=\chi(S)=2$.

For the case $m=0$ so that there are only one minimum and maximum, then we can construct a ball by applying the Jordan-Sch\"onflies theorem to each level.

For the case $m=1$, then only four types appear: a jelly bean, a red blood cell, and their upside down versions.
Apply the Jordan Sch\"onflies again.

For the case $m\ge2$, let $r$ be a regular value realizing the value of $n$.
Let $D$ be union of the closure of the interior of the innermost circles of $S\cap z^{-1}(r)$.
Replace $S$ by $(S\setminus\pd D\times(-\e,\e))\cup(D\times\{-\e,\e\})$.
Then, each connected component has lower $(m,n)$ so that it bounds a ball.
Attaching all balls bounded by the components with balls $D\times[-\e,\e]$, $S$ also bounds a ball.
\end{pf}

\subsection{Irreducible manifolds}

The connected sum is defined as
\[M\# N:=(M\setminus(\text{open ball}))\cup_\partial(N\setminus(\text{open ball})).\]
\begin{prop}\,
\begin{parts}
\item $M\# N\approx N\# M$.
\item $(M_1\# M_2)\# M_3\approx M_1\#(M_2\# M_3)$.
\item $M\# S^3\approx M$.
\end{parts}
\end{prop}

We say a manifold $M$ is \emph{prime} if $M=N_1\# N_2$ implies $N_1\approx S^3$ or $N_2\approx S^3$.
We say a 3-manifold $M$ is \emph{irreducible} if every embedding $S^2\subset M$ bounds some embedding $D^3\subset M$.
In other words, in a prime manifold every separating sphere bounds a ball, in an irreducible manifold every sphere bounds a ball.

\begin{cor}
Every irreducible 3-manifold is prime.
\end{cor}
\begin{cor}
By Theorem 3.1, $S^3$ is irreducible.
\end{cor}
\begin{thm}\,
\begin{parts}
\item $S^2\times S^1$ is prime, but is not irreducible.
\item Every closed prime 3-manifold which is not irreducible is diffeomorphic to $S^2\times S^1$.
\end{parts}
\end{thm}
\begin{pf}
(a)
A sphere $S^2\times *$ cannot bound any $D^3\subset S^2\times S^1$ because $[S^2\times *]\ne0\in H_2(S^2\times S^1)$, so $S^2\times S^1$ is not irreducible.

Suppose $S^2\times S^1=N_1\# N_2$.
Since $\pi_1(N_1)*\pi_1(N_2)\cong\Z$, one of $\pi_1(N_1)$ or $\pi_1(N_1)$ is trivial.
Assume $\pi_1(N_1)$ is trivial and let $B:=N_1\setminus(\text{open ball})$.
Since $B$ is also simply connected, it lifts diffeomorphically into the universal cover $S^2\times\R$ of $S^2\times S^1$.
Because $S^2\times\R\approx\R^3\setminus\{0\}$, we have an embedding $B\subset\R^3$.
Because $\partial B\approx S^2$, by Theorem 3.1 we have $B\approx D^3$, so $N_1\approx S^3$.

(b)
If every sphere in $M$ is separating, then it has to be irreducible since $M$ is prime, so such $M$ contains a nonseparating sphere $S$.
Let $\gamma$ be an arc connecting the inside and the outside of $\pd\nu(S)$.
If we let $M':=\bar{\nu(S)}\cup\bar{\nu(\gamma)}$, then $\pd M'\approx S^2\#(S^1\times I)\#S^2\approx S^2$ is a separating sphere.
Since $M\setminus\mathrm{Int}M'\approx D^3$ because $M'$ is not simply connected and $M$ is prime, $M$ is diffeomorphic to $S^2\times S^1$.
\end{pf}

\begin{prop}
If a covering space $\tilde M$ of $M$ is irreducible, then so is $M$.
\end{prop}

\noindent\textbf{Exercise 4.} Prove Proposition 3.6.
\begin{rmk}
The converse of Proposition 3.6 is also known to be true.
\end{rmk}

\subsection{Normal surfaces}

Fix a 3-manifold $M$ and its triangulation $T$.
A (possibly disconnected) subsurface $S\subset M$ is called a \emph{normla surface} with respect to $T$ if $S$ is a union of \emph{normal disks}, defined as seven types of disks in a given tetrahedron: four triangles and three quadrilaterals.

\begin{prop}
Every (possibly disconnected) subsurface $S\subset M$ becomes a normal surface with respect to $T$ by isotopies and the following operations:
\begin{enumerate}[(i)]
\item Replace $S$ by $(S\setminus\partial D\times(-\e,\e))\cup(D\times\{\pm\e\})$ for a disk $D$ satisfying $D\cap(S\cup\partial M)=\partial D$.
\item Remove a components of $S$ contained in a ball in $M$.
\end{enumerate}
\end{prop}
\begin{pf}
Isotope $S$ so that $S\pitchfork T$.
It is sufficient to realize the following:
\begin{parts}
\item For every tetrahedron $\Delta\subset T$, $S\cap\Delta=\coprod(\text{disks})$.
\item For every disk $D$ in (a) and for every edge $e\in T^{(1)}$, $\#(D\cap e)\le1$.
\item For every triangle $\tau\subset T^{(2)}$, $S\cap\tau=\coprod(\text{arcs})$.
\end{parts}

For (a), if there is a non-disk component of $S\cap\Delta$, perform (i) along innermost loops in $S\cap\partial\Delta$.
Then, perform (ii) for closed components of $S$.

For (b), if there is a disk component of $S\cap\Delta$ which intersects an edge more than twice, inner most pair of two points connected by an arc in $D$, push the arc outside $\Delta$ with an ambient isotopy.
(To be continued)
\end{pf}


\subsection{Prime decomposition theorem}

\end{document}