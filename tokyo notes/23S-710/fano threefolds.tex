\documentclass{../../small}
\usepackage{../../ikhanchoi}

\DeclareMathOperator{\Pic}{Pic}
\DeclareMathOperator{\Div}{Div}

\begin{document}
\title{Fano Threefolds}
\author{Ikhan Choi\\Lectured by Hiromu Tanaka\\University of Tokyo, Spring 2023}
\maketitle
\tableofcontents

\newpage
\section{Day 1: April 6}

Grade: solve 2\sim4 exercises (report)

Throughout this lecture,
\begin{itemize}
\item we work over $\C$.
\item A projective scheme is a projective scheme over $\C$, i.e. a closed subscheme of $\P_\C^N$ for some $N$.
\item A variety is an integral scheme which is separated and of finite type over $\C$.
\end{itemize}


\smallskip\hrule

\begin{defn}
A Fano variety is a smooth projective variety $X$ such that $-K_X$ is ample.
\end{defn}
\begin{defn}
Let $X$ be a smooth variety.
A canonical divisor $K_X$ is a Weil divisor such that $\cO_X(K_X)\cong\omega_X:=\bigwedge^{\dim X}\Omega_X^1\in\Pic(X)$.
($\Omega$ is a locally free sheaf of rank($=\dim X$))
the canonical divisor
\end{defn}

\begin{ex}
If $X$ is a smooth projective curve, then $X$ is Fano iff $X\equiv\P^1$.
\end{ex}
\begin{pf}
\begin{enumerate}
\item A divisor $D$ on $X$ is ample iff $\deg D>0$.
($\deg D=\sum a_i$ for $D=\sum a_iP_i$)
\item $\deg K_X=2g-2$, ($g:=h^1(X,\cO_X)\in\Z_{2n}$)
\item $g=0$ iff $X=\P^1$.
\end{enumerate}
Moreover, $\P^n$ is Fano.
\end{pf}

\begin{ex}
Let $X\subset\P^N$: smooth hypersurface of $\deg d$.
For example, we may consider $X=\{x_0^d+\cdots+x_N^d\}$.
Then, $X$ is Fano iff $d\le N$.
\end{ex}
\begin{pf}
(Sketch)
By the adjunction formula,
\[\cO_X(K_X)\cong\cO_{\P^N}(K_{\P^N}+X)|_X\cong\cO_{\P^N}(-N-1-d)|_X.\]
Then, $\Pic\P^N=\{\cO_{\P^N}(m)|m\in\Z\}\cong\Z$ (group isomoprhism).
\end{pf}

\smallskip
\hrule
\bigskip

Why 3-folds?
It is started by Gino Fano (1904\sim), and the following theorem gives a motivation:
\begin{thm}[L\"uroth,1876]
$\C\subset K\subset\C(x)$ be field extensions.
Assume the trenscendental degree of $K$ is one.
Then, $K\cong\C(y)$.
\end{thm}
The L\"uroth problem states that: if $\C\subset K\subset\C(x_1,\cdots,x_n)$ field extensions, assuming the trenscendental degree of $K$ is $n$, then $K\cong\C(y_1,\cdots,y_n)$?

\begin{thm}[Castelnuovo, 1886]
The L\"uroth problem is true if $n=2$.
\end{thm}
The idea of this theorem is to convert L\"uroth problem into a geometric version.
A field extension $K\subset\C(x)$ corresponds to a dominant rational map $\P_\C^1\to X$, and the trenscendental degree one is equivalent to that $X$ is curve.
Here we may assume $X$ to be a smooth projective curve.
So, the L\"uroth theorem can be restated as
\begin{thm}
If $\P_\C^1\twoheadrightarrow X$ for a smooth projective curve $X$, then $X\cong\P_\C^1$.
\end{thm}
For $n=2$, we consider the rationality criterion.
\begin{thm}
Let $X$ be a smooth projective surface.
Then, $X$ is rational iff $H^1(X,\cO_X)=H^0(X,2K_X)=0$
\end{thm}
\begin{ex}
If a surface $X$ is del Pezzo($=$Fano surface), then $X$ is rational.
It is because if $-K_x$ is ample then $H^0(X,2K_X)=0$ ($\because$ if not, then $2K_X$ is linearly equivalent to an effective divisor $D$, and $2(-K_X)^2=2K_X\cdot K_X=D\cdot K_X=\sum a_iC_i\cdot K_x\ge0$.)
Also, by the Kodaira vanishing, we have $H^1(X,\cO_X)=H^1(X,\cO_X(K_X+(-K_X)))=0$.
\end{ex}

How about $n=3$?
We may consider
\begin{itemize}
\item Three-dimensional rationality criterion?
\item Fano hypersurface $X\subset\P^4$ are rational?
\end{itemize}
To settle the second question, Fano studied similar and easier Fano threefolds.
\begin{thm}
There is a counterexample to L\"uroth's problem.
Specifically, if $X$ is the complete intersection of deg 2 hypersurface and deg 3 hypersurface in $\P^5$, $X$ is not rational (1908, Fano), but $X$ is unirational (1912, Enriques).
\end{thm}
\begin{thm}[1942, G. Fano]
There is a hypersurface of degree 3 $X\subset\P^4$ which is not rational but unirational.
\end{thm}
\begin{rmk}
The proof by Fano is not rigorous, so the second question(rationality of hypersurface) is now considered as results of
\begin{itemize}
	\item Clemes-Griffiths (deg$=3$)
	\item Iskovskih-Manin (deg$\ge4$)
\end{itemize}
\end{rmk}

\smallskip\hrule

\subsection*{Classificaiton of Fano 3-folds}
Two invariants: Picard number $\rho$ and index $r$.
\begin{defn}
Let $X$ be a smooth projective variety.
\[\rho=\rho(X):=\dim_\Q((\Pic X\otimes_\Z\Q)/\equiv)\in\Z_{\ge0}.\]
It is equal to $\dim_\Q((\mathrm{Div}X\otimes_\Z\Q)/\equiv$, where $\mathrm{Div}X$ is the group of Weil divisors so that $\mathrm{Div}X\otimes_\Z\Q$ contains the formal linear combinations of prime divisors over $\Q$, and where the quivalence relation is given by $D\equiv D'$ iff $D\cdot C=D'\cdot C$ for every curve on $X$.
From the intersection theory, $D\cdot C=\cO_X(D)\cdot C=\deg(\mu^*\cO_X(D))$ for $\mu:C^N\to C\hookrightarrow X$(composition of normal and closed immersion).
Then, $D\in\mathrm{Div}X\otimes_\Z\Q$ implies that there is $m\in\Z_{\ge0}$ such that $mD\in\Div X$, then $D\cdot C:=\frac1m((mD)\cdot C)$.
\begin{rmk}
Let $X$be a Fano variety.
Then, $\Pic X\cong \Pic X/\equiv\cong\Z^{\oplus\rho(X)}$.
In particular, $D\sim D'$ implies $D\equiv D'$.
\end{rmk}
\end{defn}

\begin{defn}
Let $X$ be a Fano variety.
\[r=r_X:=\text{the largest positive integer that divides $K_X$},\]
that is, there is a divisor $H$ such that $-K_X\sim rH$, but for $s>r$ there is no divisor $H$ such that $-K_X\sim sH$.
\end{defn}

We shall prove $1\le r\le\dim X+1$(for $\dim X=3$, then $r=1,2,3,4$).
\begin{ex}
Let $X=\P^3$.
Then, $\Pic X\cong \Z H$, where $H$ is a hyperplane, and $-K_x\equiv\sim 4H$, hence $\rho=1$ and $r=4$.
\end{ex}

So here is the outline:
\begin{enumerate}
\item $r\ge2$: Iskovskih, Fujita
\item $\rho=r=1$: Iskovskih, Fujita
\item $\rho\ge2$: Mori-Mukai
\end{enumerate}
For 1, $\Delta$-genus(Fujita) is used, and for 2 and 3, the cone theorem(minimal model program) is used.
When $\dim X=2$, using MMP, a del Pezzo surface $X$ is reduced to $\P^2$ or $\P^1\times\P^1$.
When $\dim X=3$, we have primitive Fano threefolds.

Our plan:
\begin{enumerate}
\item Cone theorem(mainly 2-dim)
\item $r\ge2$
\item $\rho=r=1$
\item $\rho\ge2$ (primitive)
\item $\rho\ge2$ (imprimitive)
\end{enumerate}
\smallskip\hrule

\subsection*{Cone theorem}
\begin{thm}[Cone theorem, Mori, 1982]
Let $X$ be a Fano variety.
Then, there is rational curves $l_1,\cdots,l_m$ such that
\[NE(X)=\sum_{i=1}^m\R_{\ge0}[l_i]\quad\text{and}\quad-K_X\cdot l_i\le\dim X+1.\]
\end{thm}
When $\rho=3$, $NE(X)\subset N_1(X)\cong\R^{\rho(X)}$ is a triangular pyramid.

\begin{defn}
Let $X$ be a smooth projective variety.
\begin{enumerate}
\item $Z_1(X):=\bigoplus_{C:\text{curve on }X}\Z C$,
\item $N_1(X):=(Z_1(X)\otimes_\Z\R)/\equiv$, where $Z\equiv Z'$ iff $L\cdot Z=L\cdot Z'$ for all $L\in\Pic X$.
\end{enumerate}
\end{defn}
It is well-known that
\[N_1(X)\times\left(\frac{\Pic X\otimes_\Z\R}{\equiv}\right)\to\R\]
induces a bijection
\[N_1(X)\to\Hom_\R\left(\frac{\Pic X\otimes_\Z\R}{\equiv},\R\right),\]
therefore $\dim_\R N_1(X)=\rho(X)$.

\begin{defn}
Let $X$ be a smooth projective variety.
\begin{enumerate}
	\item For $Z\in Z_1(X)\otimes\R$, denote by $[Z]\in N_1(X)$ the numerical equivalence class of $Z$.
	\item For $Z\in Z_1(X)\otimes\R$ is an effective 1-cycle.
	\item $NE(X):=\{[Z]\in N_1(X): Z\text{ effective 1-cycles}\}$
\end{enumerate}
\end{defn}
\begin{rmk}
$NE(X)$ is a convex cone.
\end{rmk}

\begin{ex}
Let $X:=\P^1\times\P^1$.
Let $l_i=\pi_i^{-1}(*)$ for $i=1,2$ be any fibers.
Then, $NE(X)=\R\ge_0[l_1]+\R_{\ge0}[l_2]$.
One direction is clear, and for the opposite, pick $[D]=[a_1C_1+\cdots+a_rC_r]\in NE(X)$ ($a_i\ge0$).
It is enough to show $C_i\equiv b_1l_1+b_2l_2$ for some $b_1,b_2\ge0$.
Fix a curve $C$ on $X$.
Note that since $\Pic X=\Z l_1\oplus\Z l_2$, we have $C\equiv b_1l_1+b_2l_2$, so $0\le C\cdot l_i=(b_1l_1+b_2l_2)\cdot l_i=b_il_1\cdot l_2>0$, we are done.
\end{ex}

References for surfaces:
\begin{itemize}
\item Beauville: Complex algebraic surfaces (over $\C$), 
\item B\v adescu: Algebraic surfaces
\end{itemize}

References for cone thm:
\begin{itemize}
\item Koll\'ar-Mori: Birational geometry of algebraic varieties 
\item Debarre: Higher-dimensional algebraic geometry
\end{itemize}


\section{Day 2: April 13}

\subsection*{Extremal rays}

\begin{defn}
Let $X$ be a Fano variety.
A ray $R$ is called an extremal ray (of $NE(X)$ or of $X$) if $\zeta,\xi\in NE(X)$ and $\zeta+\xi\in R$ imply $\zeta,\xi\in R$.
\end{defn}

\begin{thm}[Contraction theorem]
Let $X$ be a Fano variety, $R=\R_{\ge0}[l]$ an extremal ray for a curve $l$ on $X$.
Then, there is a unique morphism $f:X\to Y$ such that
\begin{enumerate}[(i)]
\item $Y$ is a projective normal variety,
\item $f_*\cO_X=\cO_Y$,
\item For a curve $C$ on $X$, $f(C)$ is point iff $[C]\in R$.
\end{enumerate}
Note that such $f$ can define the associated extremal ray.
Moreover, we have $\rho(X)=\rho(Y)+1$ and an exact sequence $0\to\Pic Y\xrightarrow{f^*}\Pic X\xrightarrow{\cdot l}\Z$.
The morphism $f$ is called the contraction morphism of R.
\end{thm}
\begin{pf}
See [Koll\'ar-Mori].
\end{pf}

\begin{thm}
Let $X$ be a del Pezzo surface.
Let $R=\R_{\ge0}[l]$ be an extremal ray for a curve $l$ on $X$ and $f:X\to Y$ be its contraction.
Then, one of the following holds:
\begin{enumerate}[(A)]
\item $l$ is a $(-1)$-curve and $f$ is a blow down of $l$ (hence $\dim Y=2$),
\item $\dim Y=1$ (i.e. $Y$ is a smooth projective curve) and $\rho(X)=2$, and $f$ is a $\P^1$-bundle with fiber $l$.
\item $\dim Y=0$ (i.e. $Y=\Spec\C$) and $\rho(X)=1$.
\end{enumerate}
\end{thm}

\begin{rmk}
Let $Y$ be a smooth projective surface and $f:X\to Y$ be the blowup at a point $P\in Y$.
Then, $l:=f^{-1}(p)$ satisfies $l\cong\P^1$ and $l^2=-1$; called $(-1)$-curve.
In this case we say $f$ is the blowdown of $l$.
\end{rmk}

\begin{rmk}
Let $X$ be a del Pezzo surface and $\rho(X)=1$.
Then, it is known that $X\cong\P^2$.
\end{rmk}

\begin{exe}
Show the above remark.
\end{exe}

\begin{rmk}
Let $X$ be a smooth projective rational surface.
If there is no $(-1)$-curve on $X$, then $X\cong \P^2$ or $X$ is isomorphic to the Hirzeburch surface $\P_{\P^1}(\cO_{\P^1}\oplus\cO_{\P^1}(n))$, where $n\in\Z_{\ge0}\setminus\{1\}$.
\end{rmk}

\begin{rmk}
Let $X$ be a del Pezzo surface and $f:X\to Y$ be a $\P^1$-bundle on a smooth projective curve $Y$.
Then, $Y=\P^1$ and $X\cong\P_{\P^1}(\cO_{\P^1}\oplus\cO_{\P^1}(n))$, $n\in\{0,1\}$.
\begin{pf}[Sketch]
Leray spectral sequence gives $H^1(Y,f_*\cO_X(=\cO_Y))\hookrightarrow H^1(X,\cO_X)=0$, so $H^1(Y,\cO_Y)=0$ implies $Y=\P^1$.

Also, $\P^1$-bundle, $X\cong\P_{\P^1}(E)$ of rank two, it is well known that $E\cong\cO_{\P^1}(a)\oplus\cO_{\P^1}(b)$ and $X\cong\P_{\P^1}(\cO(a)\oplus\cO(b))\cong\P_{\P^1}(\cO\oplus\cO(b-a))$ for $n:=b-a\ge0$.
It is known that for a $\P^1$-bundle over $\P^1$ there is a section $c$ such that $c^2=-n$, then $n\in\{0,1\}$.
\end{pf}
\end{rmk}

\begin{lem}
Let $X$ be a del Pezzo surface and $C$ a curve on $X$.
Then, $C^2\ge-1$.
\end{lem}
\begin{pf}
Write $(K_X+C)\cdot C=2h^1(C,\cO_C)-2$.
Recall that $(\omega_X\otimes\cO_X(C))|_C\cong\omega_C$ holds even if $C$ is a singular curve.
Hence, $C^2\ge-K_X\cdot C-2\ge1-2=-1$.
\end{pf}

\begin{ex}
Let $X=\P^1\times\P^1$ and $l_i=\pi_i^{-1}(*)$ fibers.
Then, each projection map $\pi_i$ corresponds to the extremal rays $\R_{\ge0}[l_i]$.
\end{ex}
\begin{ex}
Let $X=\P^2$.
Then, $NE(X)=\R_{\ge0}[l]=\R_{\ge0}[l']=\cdots$ since $N_1(X)=\R^{\rho(X)}=\R$.
\end{ex}
\begin{ex}
Let $X=\P_{\P^1}(\cO_{\P^1}\oplus\cO_{\P^1}(1))$, which is del Pezzo.
Then, if $f$ is a blowdown of a section $l\cong\P^1$, then $\rho(Y)=1$ and $Y\cong\P^2$.
Then, we have two extremal rays $[l]$ and $[l']$ which correspond to $f$ and $\pi$ respectively.
\end{ex}

\begin{rmk}
Let $X$ be a del Pezzo surface with $\rho(X)\ge3$.
Then,
\[\{\text{extremal rays}\}\leftrightarrow\{\text{$(-1)$-curves}\}.\]
Therefore, a del Pezzo surface has a finitely many $(-1)$-curves.
\end{rmk}

\begin{ex}
Let $f:X\to\P^2$ be a blowup at two points $P$ and $Q$ with $l_P=f^{-1}(P)$ and $l_Q=f^{-1}(Q)$.
Lifting a line $m$ passing through $P$ and $Q$, we obtain $m_X$ the proper transform of $m$.
Then, $\rho(X)=3$ and $NE(X)=\R_{\ge0}[l_P]+\R_{\ge0}[l_Q]+\R_{\ge0}[m_X]$.
\end{ex}
\begin{rmk}
Let $X\subset\P^3$ be a smooth cubic surface, for example, $X:x^3+y^3+z^3+w^3=0$.
It is well-known that $X$ has exactly 27 $(-1)$-curves so that $NE(X)=\sum_{i=1}^{27}\R_{\ge0}[l_i]$.
\end{rmk}
\begin{rmk}
Minimal model program for del Pezzo surfaces.
\begin{cd}
&X:\text{ del Pezzo surface} \ar{d}&\\
&\exists(-1)\text{-curve on }X?\ar[swap]{dl}{YES}\ar{dr}{NO}&\\
f:X\to Y\text{ a blowdown of a $(-1)$-curve}\ar{d}&&\exists\text{ extremal }\R_{\ge0}[l]\text{ s.t. }l\text{ is not $(-1)$-curve}\ar{d}\\
\text{we can show $Y$ is del Pezzo}&&f:X\to Y\text{ the contraction of $\R_{\ge0}[l]$} \ar[swap]{dl}{\dim Y=1}\ar{d}{\dim Y=0}\\
&X\cong\P^1\times\P^1&X\cong\P^2\text{ (2.5)}
\end{cd}
\end{rmk}
\begin{rmk*}
Let $X\cong\P_{\P^1}(\cO_{\P^1}\oplus\cO_{\P^1}(n))$ with $n\in\{0,1\}$.

If $n=0$, then $X\cong\P_{\P^1}(\cO_{\P^1}\oplus\cO_{\P^1})\cong\P^1\times\P^1$.

If $n=1$, then $X\cong\P_{\P^1}(\cO_{\P^1}\oplus\cO_{\P^1}(1))$, there is a $(-1)$-curve on $X$ (cf.(2.11))
\end{rmk*}

\begin{pf}[Outline of (2.3)]
For an extremal ray $R=\R_{\ge0}[l]$, (A) for $l^2<0$, (B) for $l^2=0$, (C) for $l^2>0$.
\end{pf}

\begin{prop}
Let $X$ be a del Pezzo surface and $l$ be a curve on $X$ with $l^2<0$.
Then,
\begin{parts}
\item $l$ is a $(-1)$-curve,
\item $\R_{\ge0}[l]$ is an extremal ray,
\item the contraction of $R$ is the blowdown of $l$.
\end{parts}
In particular, $\dim Y=\dim X=2$.
\end{prop}
\begin{pf}
(a)
We will show the following statements are equivalent:
\begin{enumerate}[(i)]
\item $l$ is a $(-1)$-curve,
\item $l\cong\P^1$ and $l^2=-1$,
\item $K_X\cdot l=l^2=-1$,
\item $K_X\cdot l<0$ and $l^2<0$.
\end{enumerate}
Here $X$ is a smooth projective surface and $l$ a curve on it.
Note (i) and (ii) are equivalent by definition.
The equivalence between (ii) and (iii) is due to $(K_X+l)\cdot l=2h^1(l,\cO_l)-2\ge-2$.
The equivalence between (iii) and (iv) is clear.

(b) Omitted.

(c) Let $f:X\to Y$ blowdown of $l$ and $P:=f(l)$.
Recall that $f$ is a contraction of $R$ iff
\begin{enumerate}[(i)]
\item $Y$ is a projective normal variety,
\item $f_*\cO_X=\cO_Y$,
\item for a curve $C$ on $X$, $f(C)$ is a point iff $[C]\in\R_{\ge0}[l]$.
\end{enumerate}
It follows (ii) from the following lemma (2.18).
For (iii), ($\Rightarrow$) is clear.
($\Leftarrow$) Suppose $[C]\in\R_{\ge0}[l]$ and $C\ne l$ so that $C\cdot l\ge0$.
Then, $C\equiv al$ for $a\in\R_{\ge0}$, and $a>0$ since $C\cdot H=al\cdot H$ for ample $H$.
Now $0\le C\cdot l=al\cdot l=a(>0)\cdot l^2(=-1)<0$, a contradiction.
\end{pf}

\begin{lem}
If $f$ is a projective birational morphism of normla varieties, then $f_*\cO_X=\cO_Y$.
\end{lem}
\begin{pf}
Consider the Stein factorization
\begin{cd}
X\ar{dr}{g}\ar{rr}{f}&&Y\\
&Z\ar{ur}{h}
\end{cd}
such that $g_*\cO_X=\cO_Z$ and $h$ finite.
Then,
\begin{cd}
K(X)&&K(Y)\ar[swap]{ll}{\cong}\ar{dl}\\
&K(Z)\ar{ul}
\end{cd}
implies $Z\xrightarrow{h}Y$ is finite birational morphism, and $A\hookrightarrow B$ is integral extension with $K(A)=K(B)$ where $\Spec A\subset Y$ is affine open and $\Spec B$ is given by the pullback(inverse image of $h$), hence $A=B$.
\end{pf}

\begin{lem}
Let $X$ be a del Pezzo surface and $\R_{\ge0}[l]$ be an extremal ray for a curve $l$ on $X$, whose contraction is $f:X\to Y$.
Then,
\begin{enumerate}[(A)]
\item $l^2<0$ iff $\dim Y=2$,
\item $l^2=0$ iff $\dim Y=1$,
\item $l^2>0$ iff $\dim Y=0$.
\end{enumerate}
\end{lem}
\begin{pf}
Next lecture.
\end{pf}

\begin{prop}[(B)]
If $l^2=0$, then the fiber is isomorphic to $\P^1$.
\end{prop}
\begin{pf}
For $P\in Y$, let $F:=f^*P=\sum_{i=1}^ra_iC_i$ with $a_i\in\Z_{>0}$ and $C_i$ prime divisors.
\begin{clm}
Every fiber is irreducible.
\end{clm}
\begin{pf}
If it is reducible, then there are $C_1\ne C_2$ in the fiber, then
\[F\cdot C_1=(\sum_{i=1}^ra_iC_i)\cdot C_1=a_1C_1^2+(\text{positive}),\]
so $C_1^2<0$.
Then, $C_i\equiv b_il$, so $C_1^2<0$ implies $l^2<0$ and $C_1\cdot C_2\ge0$ implies $l^2\ge0$, a contradiction.
\end{pf}
We can show that every fiber $F$ is reduced:
\[(K_X+F)\cdot F=K_X\cdot F+F^2=K_X\cdot F+0<0,\]
by the adjunction, $F\cong\P^1$.
\end{pf}



\newpage
\section{Day 3: April 20}

\subsection*{Nef divisors and big divisors}
Our today's goal is to prove Lemma 2.19.


\begin{rmk}
Since $f_*(\cO_X)=\cO_Y$, $f:X\to Y$ is surjective so that $\dim Y\in\{0,1,2\}$.
If we prove (A) and (C) in the Lemma 2.19, then we are enough.
\end{rmk}

\begin{pf}[Proof of Lemma 2.19 (A)]
($\Rightarrow$) Proposition 2.17.

($\Leftarrow$)
Note that $\dim X=\dim Y$ and $f_*(\cO_X)=\cO_Y$ imply $f$ is birational.
For an ample Cartier divisor $A_Y$ on $Y$, $f^*A_Y$ is a big divisor(defined later).
Then,
\[f^*A_Y\cdot l=\deg(f^*A_Y|_l)=\deg(i^*f^*A_Y)=\deg((f|_l)^*j^*A_Y)=\deg((f|_l)^*\cO_{f(l)})=\deg\cO_l=0,\]
where $i:l\hookrightarrow X$ and $j:f(l)=*\hookrightarrow Y$ such that $fi=jf|_l$.

We can define $f^*A_Y$ to be a big divisor if and only if there is $m\in\Z_{>0}$ such that $mf^*A_Y$ is the sum of an ample divisor $A$ and an effective divisor $E$.
Then, $A\cdot l+E\cdot l=0$ implies $E\cdot l<0$, so if we write $E=\sum a_iC_i$, then $l=C_i$ for some $i$, hence $l^2<0$.
\end{pf}

\begin{defn}
Let $X$ be a projective normal variety and $D$ a Cartier divisor.
Then, $D$ is called to be big if and only if there are $m\in\Z_{>0}$, an ample Cartier divisor $A$, and an effective Cartier divisor $E$ such that $mD=A+E$.
\end{defn}

\begin{rmk}
In the above definition, the equality $mD=A+E$ can be replaced by $\sim$ or $\equiv$.
\end{rmk}

\begin{rmk}
A divisor $D$ is big iff $nD$ is big for all $n\in\Z_{>0}$ iff $nD$ is big for some $n\in\Z_{>0}$.
\end{rmk}

\begin{prop}
Let $f:X\to Y$ be a birational morphism of projective normal varieties.
For a Cartier divisor $D$ on $Y$, $f^*D$ is big iff $D$ is big.
\end{prop}
\begin{pf}
Since $f_*\cO_X=\cO_Y$, by tensoring $\cO_Y(mD)$ we get
\[\cO_Y(mD)=(f_*\cO_X)\otimes_{\cO_Y}\cO_Y(mD)=f_*(\cO_X\otimes_{\cO_X}f^*\cO_Y(mD))=f_*f^*\cO_Y(mD)\]
(the second equality is due to the projection formula),
so
\[H^0(Y,\cO_Y(mD))=H^0(Y,f_*f^*\cO_Y(mD))=H^0(X,f^*\cO_Y(mD))=H^0(X,\cO_X(mf^*(D))).\]
Therefore, $f^*D$ is big iff $D$ is big by Proposition 3.6.
\end{pf}

\begin{prop}
Let $X$ be a projective normal variety and $D$ a Cartier divisor on $X$.
Then $D$ is big iff there is $c\in\Q_{>0}$ such that for all sufficiently large $m$ we have
\[h^0(X,\cO_X(mD))>c\cdot m^{\dim X}.\]
\end{prop}
\begin{pf}
($\Rightarrow$)
We may assume $D=A+E$ with $A$ ample and $E$ effective.
Then, $H^0(X,mD)=H^0(X,m(A+E))\hookleftarrow H^0(X,mA)$ by
\[0\to \cO_X(-mE)\to\cO_X\to\cO_{mE}\to0.\]
Thus $h^0(X,mA)\le h^0(X,mD)$ implies that we may assume $D$ is ample.

It is well-known that
\[\chi(X,mD)=\frac{D^{\dim X}}{(\dim X)!}m^{\dim X}+O(m^{\dim X-1})\in\Z[m]\]
from the Riemann-Roch, and by the Serr vanishing we have $\chi(X,mD)=h^0(X,mD)$ for large $m$, and we also have $D^{\dim X}>0$ by Nakai's criterion.

($\Leftarrow$)
Fix $A$ a very ample divisor on $X$.
We may assume by Bertini that $A$ is a normal prime divisor.
We have
\[0\to\cO_X(mD-A)\to\cO_X(mD)\to\cO_X(mD)|_A\to0,\]
and $\cO_X(mD)|_A\cong\cO_A(mD_A)$ for some Cartier divisor $D_A$ on $A$ such that $\cO_X(D)|_A\cong \cO_A(D_A)$.

Write
\[0\to H^0(X,mD-A)\to H^0(X,mD)\to H^0(A,mD_A).\]
Here $h_0(X,mD)\ge c\cdot m^{\dim X}$ and $h^0(A,mD_A)\le b\cdot m^{\dim A}$ by the Exercise 3.7, we have $H^0(X,mD-A)\ne0$ for some $m>0$, i.e. $mD-A$ is linearly equivalent to an effective divisor.
\end{pf}

\textbf{Exercise 3.7.} Let $Z$ be a projective normal variety and $D$ a Cartier divisor on $Z$.
Show that there exists $b>0$ such that $h^0(Z,mD)\le b\cdot m^{\dim Z}$ for all $m\in\Z_{>0}$.
If you want, you may assume that $Z$ is smooth.

\begin{pf}[Proof of Lemma 2.19 (C)]
($\Leftarrow$)
Let $\dim Y=0$ i.e. $Y=\Spec\C$ with $\rho(X)=\rho(Y)+1=1$, which implies that $l\equiv cA$ for some $c\in\Q$ and an ample divisor $A$ on $X$ because every projective variety has an ample divisor.
Then, we can prove $c>0$ from $A\cdot l=A\cdot(cA)=cA^2$, hence $l^2=(cA)\cdot(cA)=c^2A^2>0$.

($\Rightarrow$)
Let $l^2>0$.
Note that if $l$ is a curve on a smooth projective surface $X$ such that $l^2>0$, then $l$ is nef because $l\cdot C>0$ if $l=C$ and $l\cdot C\ge0$ if $l\ne C$, and furthermore $l$ is big by Proposition 3.9.
Fix $C$ a curve on $X$.
We are enough to show $[C]\in\R_{\ge0}[l]$.
Then, $N_1(X)=\bigoplus_C\R_C/\equiv$ is generated by $[l]$, we get $\rho(X)=\dim N_1(X)=1$ and $\dim Y=0$.

Let $l$ be a big divisor so that there is a sufficiently large $m$ with a rational map $f:X\dashrightarrow\P^N$ defined by the complete linear system $|ml|$ whose image is a surface.
By considering the defining polynomials of $\f(C)=\bar{V_+}(f_1,\cdots,f_r)$ such that $\f(ml)$ is a hyperplane section, there must be $f_i$ not vanishing on $X$, so we have $f_i$ with $\bar{V_+}(f_i)\cap\f(X)=\f(C)+\f(E)$, where $E=\f^{-1}(\f(E))$.
Then, since $\bar{V_+}(f_i)\sim\f((\deg f_i)ml)$, which implies $(\deg f_i)ml\sim C+E$.
Thus, using the definition of extremal rays, we have $[C]\in\R_{\ge0}[l]$.
\end{pf}

\setcounter{thm}{7}
\begin{defn}
Let $X$ be a projective normal variety.
A Cartier divisor $D$ is called nef iff $D\cdot C\ge0$ for all curves $C$ on $X$.
\end{defn}
\begin{prop}
Let $X$ be a projective normal variety and $D$ a nef Cartier divisor.
Then, $D$ is big iff $D^{\dim X}>0$.
\end{prop}


\begin{pf}
For simplicity, assume $\dim X=2$.

($\Rightarrow$)
Let $mD=A+E$ with $z\in\Z_{>0}$, $A$ ample, $E$ effective.
Since $mD\cdot E\ge0$ from that $D$ is nef and $mD\cdot A=A^2+E\cdot A>0$ from that $A$ is ample, we have $(mD)^{\dim X}=(mD)^2=mD\cdot A+mD\cdot E>0$.

($\Leftarrow$)
We may assume $X$ is smooth by taking a resolution of $X$(the pullback via a rational map of a nef or big divisor is also nef of big respectively).
Take $H$ a very ample divisor on $X$.
We also may assume $H-K_X$ is ample by the Serre criterion.
Then,
\[0\to\cO_X(mD)\to\cO_X(mD+H)\to\cO_X(mD+H)|_H\to0\]
and
\[0\to H^0(\cO_X(mD))\to H^0(\cO_X(mD+H))\to H^0(\cO_X(mD+H)|_H)\]
are exact.
Note that we have
\[h^0(\cO_X(mD+H))=\chi(X,mD+H)=\frac{(mD+H)^2}{2!}+O(m)\ge c\cdot m^2\]
by the Kodaira vanishing
\[H^i(X,mD+H)=H^i(X,K_X+(mD)_{(\text{it is nef})}+(H-K_X)_{(\text{it is ample})})=0\]
(sum of nef and ample is ample $\because$ Corollary 3.12.)
and $h^0(\cO_X(mD+H)|_H)\le b\cdot m^{\dim H}=b\cdot m$.
Therefore, $h^0(X,\cO(mD))\ge c'\cdot m^2$ for some $c'$ and sufficiently large $m$.
\end{pf}

\begin{rmk}
Let $X$ be a projective normal variety with a nef divisor $D$.
Then,
\begin{parts}
\item $D\cdot{}^\forall(\text{curve})\ge0$ (by def),
\item $D\cdot{}^\forall(\text{effective 1-cycle})\ge0$.
\end{parts}
In particular, $NE(X)\subset D^{\ge0}:=\{\zeta\in N_1(X):D\cdot\zeta\ge0\}=D^{>0}\cup D^\perp$.
In fact,
\begin{parts}
\item[(c)] The Kleiman-Mori cone is contained in $D^{\ge0}$, i.e. $\bar{NE(X)}\subset D^{\ge0}$.
\end{parts}
\end{rmk}

\begin{thm}[Kleiman's ampleness criterion]
Let $X$ be a projective normal variety and $D$ a Cartier divisor.
Then, $D$ is ample iff $\bar{NE(X)}\setminus\{0\}\subset D^{>0}$.
\end{thm}
\begin{pf}
Omitted.
\end{pf}
\begin{cor}
If $N$ is nef and $A$ is ample, then $N+A$ is ample.
\end{cor}
\begin{pf}
$\zeta\in\bar{NE(X)}\setminus\{0\}$ implies $(N+A)\cdot\zeta=N\cdot\zeta+A\cdot\zeta>0$ because $N\cdot\zeta\ge0$ and $A\cdot\zeta>0$.
\end{pf}

\begin{rmk}
It is useful to use $\Q$-divisors.
For $D\in\Div X\otimes_\Z\Q$, $D$ is defined to be nef if there is $m\in\Z_{>0}$ such that $D$ is a nef Cartier divisor, and defined to be ample if there is $m\in\Z_{>0}$ such that $D$ is a ample Cartier divisor.
Then, a nef divisor can be approximated by $D=\lim_{\e\to0+}(D+\e A)$.
\end{rmk}

\begin{thm}[Nakai-Moishezon]
Let $X$ be a projective normal variety and $D$ a Cartier divisor.
Then, $D$ is ample (resp.~nef) iff for a subvariety $Y\subset X$ we have $Y\cdot D^{\dim Y}>0$ (resp.~$\ge0$).
\end{thm}
\begin{pf}
For amples, well-known.
For nefs, it follows from $Y\cdot D^{\dim Y}=\lim_{\e\to0+}Y\cdot(D+\e A)^{\dim Y}\ge0$.
\end{pf}


\newpage
\section{Day 4: April 27}
\subsection*{$\Delta$-genus}

We study $\Delta$-genus to classify Fano 3-folds with index $r\ge2$.

\subsection{Index}

\begin{defn}
Let $X$ be a Fano 3-fold.
The index $r=r_X\in\Z_{>0}$ is defined such that there is a divisor $H$ with $-K_X\sim rH$ but no divisors $H$ satisfy $-K_X\sim sH$ for $s\in\Z_{r>0}$.
\end{defn}
\begin{lem}
$1\le r\le 4$.
\end{lem}
\begin{pf}
Cone theorem implies $NE(X)=\sum_{i=1}^m\R_{\ge}[l_i]$
with $0<-K_X\cdot l_i\le\dim X+1=4$.
Then, since $r\le-K_X\cdot l_i$, we are done.
\end{pf}

Today's goal: $r=4$ implies $X\cong\P^3$, $r=3$ implies $X\cong\text{(quadratic)}\subset\P^4$.

Outline\\
$r=4$ => $\Delta(X,H)=0$ with $-K_X\sim4H$ => $|H|$ is very ample with $H^3=1$. => $X\cong\P^3$.\\
$r=3$ similar.

\subsection{$\Delta$-genus: definition and examples}
\begin{defn}
A pair $(X,D)$ is called a polarized variety if $X$ is a projective variety and $D$ is an ample divisor(or invertible sheaf) on $X$.
\end{defn}
\begin{defn}
Let $(X,D)$ be a polarized variety.
Then,
\[\Delta(X,D):=\dim X+D^{\dim X}-h^0(X,D).\]
\end{defn}
\begin{ex}\,
\begin{enumerate}[(i)]
\item
Let $n\in\Z_{>0}$.
Then,
\begin{align*}
\Delta(\P^1,\cO_{\P^1}(n))&=\dim\P^1+\deg\cO_{\P^1}(n)-h^0(\P^1,\cO_{\P^1}(n))\\&=1+n-(n+1)=0.\end{align*}
\item
Let $X$ be an elliptic curve and $D$ an ample divisor on $X$.
Then, by the Riemann-Roch
\[h^0(X,D)-h^1(X,D)=\chi(X,D)=\deg D+1-g=\deg D\]
and the Serre duality $h^1(X,D)=h^0(X,-D)=0$, we have
\[\Delta(X,D)=1+\deg D-\deg D=1.\]
\end{enumerate}
\end{ex}

\begin{ex}
Let $X$ be a del Pezzo surface.
Then, $\Delta(X,-K_X)=1$.
\end{ex}
\begin{pf}
By the Riemann-Roch
\[\chi(X,D)=\chi(X,\cO_X)+\frac12(-K_X)\cdot(-K_X-K_X)\]
and the Kodaira vanishing
\[\chi(X,-K_X)=\chi(X,K_X+(-2K_X))=h^0(X,-K_X),\qquad\chi(X,\cO_X)=h^0(X,\cO_X),\]
we have $h^0(X,-K_X)=K_X^2+1$.
Therefore,
\[\Delta(X,-K_X)=\dim X+(-K_X)^2-h^0(X,-K_X)=1.\]
\end{pf}

\begin{prop}
Let $X$ be a Fano 3-fold.
Pick a divisor $H$ such that $-K_X\sim rH$.
\begin{parts}
\item If $r=4$, then $\Delta(X,H)=0$ and $H^3=1$.
\item If $r=3$, then $\Delta(X,H)=0$ and $H^3=2$.
\end{parts}
\end{prop}
\begin{prop}[Riemann-Roch for 3-folds]
Let $X$ be a smooth projective 3-fold and $D$ a divisor.
Then,
\begin{parts}
\item \[\chi(X,D)=\frac1{12}D\cdot(D-K_X)\cdot(2D-K_X)+\frac1{12}D\cdot C_2(X)+\chi(X,\cO_X).\]
\item \[-K_X\cdot C_2(X)=24\chi(X,\cO_X).\]
\end{parts}
\end{prop}
\begin{pf}
Omitted.
\end{pf}
\begin{cor}
Let $X$ be Fano 3-fold and $H$ an ample divisor such that $H\equiv -qK_X$ with $q\in\Q_{>0}$.
Then,
\[h^0(X,H)=\chi(X,H)=\frac1{12}q(q+1)(2q+1)(-K_X)^3+2q+1.\]
\end{cor}
As a comment for $\Q_{>0}$, in most cases we have $q^{\pm1}\in\Z_{>0}$.
For example, $H\equiv-\frac1rK_X$ iff $rH\equiv-K_X$.
\begin{pf}
By Propositioin 4.8 and the Kodaira vanishing
\[\chi(X,H)=h^0(X,H),\quad\chi(X,\cO_X)=1,\]
we can complete the proof by simple computation.
\end{pf}
\begin{thm}
Let $(X,D)$ be a polarized variety.
Then, $\Delta(X,D)>\dim Bs|D|$, where $\dim\varnothing:=-1$.
In particular, $\Delta(X,D)\ge0$.
\end{thm}
\begin{pf}
We will do if time permits.
\end{pf}

\begin{pf}[Proof of Proposition 4.7.]
We only show (a).
Note that
\[h^0(X,H)=^{(4.9)}\frac1{12}q(q+1)(2q+1)(-K_X)^3+2q+1=h^0(X,H)=\frac52H^3+\frac32\]
since $q=\frac14$ and $(-K_X)^3=(4H)^3=64H^3$, so
Then, Theorem 4.10 and $H^3\ge1$ imply
\[0\ge\Delta(X,H)=\dim X+H^3-h^0(X,H)=\frac32(1-H^3)\le0.\]
Therefore, $H^3=1$ and $\Delta(X,H)=0$.
\end{pf}

\begin{rmk}
If $r=4$ and $-K_X\sim4H$, then $h^0(X,H)=4$.
If $|H|$ is very ample, then $X\hookrightarrow\P^{4-1}=\P^3$, hence $X\cong\P^3$.
Thus we are enough to show the complete linear system $|H|$ is very ample.
\end{rmk}

\begin{thm}
Let $(X,D)$ be a polarized variety with $\Delta(X,D)=0$.
Then,
\begin{parts}
\item $N_1$ property holds: $\bigoplus_{m=0}^\infty H^0(X,mD)$ is generated by $H^0(X,D)$ as a $\C$-algebra.
\item $|D|$ is very ample.
\end{parts}
\end{thm}

\begin{exe}
Show that under the $N_1$ property, if $D$ is ample, then $|D|$ is very ample.
\end{exe}

\begin{prop}
Let $(X,L)$ be a polarized variety with invertible sheaf $L$.
Let $Y$ be an integral closed subscheme in $|L|$.
For example, if $X$ is normal with $L\cong\cO_X(D)$, then $D\sim Y$, and it is a prime divisor.
Then,
\begin{parts}
\item $L^{\dim X}=(L|_Y)^{\dim X-1}$.
\item $0\le\Delta(X,L)-\Delta(Y,L|_Y)\le h^1(X,\cO_X)$.
\item $H^0(X,L)\to H^0(Y,L|_Y)$ is surjective iff $\Delta(X,L)=\Delta(Y,L|_Y)$.
\item Assume the condition in the part (c). Then, if $L|_Y$ satisfies $N_1$ property, then so does $L$.
\end{parts}
\end{prop}

\begin{pf}[Proof of Proposition 4.12 assuming Proposition 4.14.]
For simplicity, we assume $X$ is smooth.
The complete linear system $|D|$ is base point free by $\Delta(X,D)=0$ and Theorem 4.10 ($\dim Bs|D|<\Delta(X,D)$).
Let $Y\in|D|$ be a general member.
By Bertini, $Y$ is smooth and connected($D$ is ample), hence $Y$ is a smooth prime divisor.
Applying Proposition 4.14, we have $0\le\Delta(Y,D|_Y)\le\Delta(X,D)\le0$.
By Proposition 4.14 (d), $D$ satisfies $N_1$ property from applying the induction hypothesis.
\end{pf}

\begin{rmk}
We can check that for a projective curve $X$ we have TFAE:
\begin{enumerate}[(i)]
\item $X\cong\P^1$,
\item $\Delta(X,D)=0$ for every ample $D$,
\item $\Delta(X,D)=0$ for an ample $D$.
\end{enumerate}
\end{rmk}

\begin{pf}[Proof of Proposition 4.14]
Write $n:=\dim X$.

(a)
$L^n=L^{n-1}\cdot Y=(L|_Y)^{n-1}$

(b)
$\Delta(X,L)=n+L^n-h^0(X,L)$ and $\Delta(Y,L|_Y)=(n-1)+(L|_Y)^{n-1}-h^0(Y,L|_Y)$ imply
\[\Delta(X,L)-\Delta(Y,L|_Y)=1+h^0(Y,L|_Y)-h^0(X,L).\]
By taking $-\otimes L$ on
\[0\to\cO(-Y)\to\cO_X\to\cO_Y\to0,\]
we have exact sequences
\[0\to \cO_X\to L\to L|_Y\to0\]
and
\[0\to H^0(X,\cO_X)\to H^0(X,L)\to H^0(Y,L|_Y)\xrightarrow{\delta}H^1(X,\cO_X).\]
Then,
\[h^1(X,\cO_X)\ge\dim\im\delta=1+h^0(Y,L|_Y)-h^0(X,L)=\Delta(X,L)-\Delta(Y,L|_Y)\]
and $\dim\im\delta\ge0$ implies the desired result.

(c)
We have $\delta=0$ if and only if $\Delta(X,L)=\Delta(Y,L|_Y)$, which is also equivalent to that $H_0(X,L)\to H^0(Y,L|_Y)$ is surjective.

(d)
Note that we have a surjection $H^0(X,L)\to H^0(Y,L|_Y)$.
Suppose $L|_Y$ satisfies $N_1$ property.
If $\zeta\in H^0(Y,mL|_Y)$, then $\zeta=\sum c\xi_1\cdots\xi_m$ for $c\in\C$ and $\xi_i\in H^0(Y,L|_Y)$, so we can show the map $H^0(X,mL)\to H^0(Y,mL|_Y)$ is surjective.
\end{pf}


\begin{pf}[Proposition 4.12 (b)..?]
It is enough to show $H^0(X,mL)\otimes_\C H^0(X,L)\to H^0(X,(m+1)L)$ is surjective.
\begin{cd}
\,&
\,&
H^0(X,mL)\otimes_\C H^0(X,L)\ar[two heads]{r}\ar{d}{\mu_X}&
H^0(Y,mL|_Y)\otimes_\C H^0(Y,L|_Y)\ar[two heads]{d}{\mu_Y}\\
0\ar{r}&
H^0(X,mL)\ar{r}&
H^0(X,(m+1)L)\ar{r}&
H^0(Y,(m+1)L|_Y)
\end{cd}
For $\zeta\in H^0(X,(m+1)L)$, we have $\zeta_Y\in H^0(Y,(m+1)L|_Y)$ and there is $\sum c\xi_Y\otimes\eta_Y\in H^0(Y,mL|_Y)\otimes_\C H^0(Y,L|_Y)$ and back to obtain $\sum c\xi_X\otimes\eta_X\in H^0(X,mL)\otimes_\C H^0(X,L)$ with surjectivity.
If we define $\tilde \zeta:=\zeta-\mu_X(\sum c\xi_X\otimes \eta_X)$,
then there is $\tilde{\tilde\zeta}\mapsto$
% 写真を見よう
\end{pf}


We now prove Theorem 4.10.

\begin{defn}
Let $X$ be a projective variety and $L$ an ample invertible sheaf.
Let $V\subset H^0(X,L)$ be a $\C$-linear subspace.
Let $\Delta(X,L,V):=\dim X+L^{\min X}-\dim_\C V$.
(Note $\Delta(X,L)=\Delta(X,L,H^0(X,L)$)
\end{defn}
\begin{thm}
$\Delta(X,L,V)>\dim Bs|V|$, where $|V|$ is the linear system corresponding to $V$.
\end{thm}
\begin{pf}
We may assume that $X$ is normal and $V=H^0(X,L)$.
the normalization of the resolution of the inderminacies of $\f_{|L|}$..
\begin{cd}
X \ar[dashed]{r}{\f_{|L|}} & \P_\C^N & Z:=\psi(Y) \ar[hook]{l}{subsp}\\
Y\ar{u}{\mu}\ar{ru}{\psi}\ar{rru}{\psi}
\end{cd}

One of the following holds:
\begin{enumerate}[(i)]
\item $\dim Bs|L|=n$, where $n=\dim X$,
\item $\dim Z=1$,
\item $\dim Z\ge2$ and $\dim Bs|L|=n-1$,
\item $\dim Z\ge2$ and $\dim Bs|L|\le n-2$,
\end{enumerate}

For the case (i), since $\dim Bs|L|=n$ iff $H^0(X,L)=0$, we have
\[\Delta(X,L)=n+L^n-h^0(X,L)>n=\dim Bs|L|.\]

For the case (ii), we have $\Delta(X,L)=n+L^n-h^0(X,L)$.
Then, $\mu^*L=M+F$ is decomposed into base point free $M$ and fixed part $F$ by $L\mapsto\mu*L$ and $L_Z:=\cO_{\P^1}(1)|_Z\mapsto M$.
Then, with normal $X$ and $\mu$ birational we have
\[H_0(X,L)\cong H^0(Y,\mu^*L)\cong H^0(Y,M).\]
Also $H^0(Y,M)\cong H^0(Z,L_Z)$ since the injectivity follows from $\psi_*\cO_Y\hookleftarrow\cO_Z$ and the surjectivity is due to the fact that the composition $H^0(Y,M)\leftarrow H^0(Z,L_Z)\leftarrow H^0(\P^N,\cO(1))$ is bijective.
Now
\[0\le\Delta(Z,L_Z)=1+\deg L_Z-h^0(Z,L_Z)\]
and
\[(\mu^*L)^{n-1}\cdot(\psi^*L_Z)=(\deg L_Z)\cdot(\mu^*L)^{n-1}\cdot(\text{a general fiber of $\psi$})\ge\deg L_Z\]
because $\mu^*L$ is nef and big.
\[L^n=(\mu^*L)^n=(\mu^*L)^n\cdot(M+F)\ge\deg L_Z+(\mu^*L)^{n-1}F,\]
\[\Delta(X,L)=n+L^n-h^0(X,L)\ge n+\deg L_Z+(\mu^*L)^{n-1}\cdot F-h^0(Z,L_Z)\gen-1+(\mu^*L)^{n-1}\cdot F\ge n-1\]
If $\dim Bs|L|\le n-2$, then we are done.
If $\dim Bs|L|=n-1$, then $(\mu^*L)^{n-1}\cdot F>0$ because $\mu(F)$ has dimension $n-1$, so $\Delta(X,L)\cdots>n-1$.

For the case (iii) and (iv), see [Fujita].
\end{pf}
\begin{itemize}
	\item T. Fujita, Classification $\cdots$ of polarized varieties (Book)
	\item T. Fujita, On the structure $\cdots$ with $\Delta$-genus zero (Many papers by Fujita)
\end{itemize}


% C_2(X)?
% Bs|D|?
% complete linear system is very ample?



\end{document}