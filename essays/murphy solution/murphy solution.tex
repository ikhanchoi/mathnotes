\documentclass[12pt]{article}
\usepackage{../../ikany}
\usepackage[margin=100pt]{geometry}
\usepackage[T1]{fontenc}
\usepackage[bitstream-charter,cal]{mathdesign}
\linespread{1.1}

\DeclareMathOperator{\Inv}{Inv}
\DeclareMathOperator{\ind}{ind}

\begin{document}
\title{Solution Manual for\\C*-algebras and Operator Theory\\by Gerard J. Murphy}
\author{Ikhan Choi}
\maketitle
\tableofcontents


\section{Elementary spectral theory}

\begin{prb}
Let $(A_\lambda)_{\lambda\in\Lambda}$ denote a family of Banach algebras.
The \emph{direct sum} $A=\bigoplus_\lambda A_\lambda$ is the set of all $(a_\lambda)\in\prod_\lambda A_\lambda$ such that $\|(a_\lambda)\|=\sup_\lambda\|a_\lambda\|$ is finite.
Show that this is a Banach algebra under the pointwise-defined operations
\begin{gather*}
(a_\lambda)+(b_\lambda)=(a_\lambda+b_\lambda)\\
\mu(a_\lambda)=(\mu a_\lambda)\\
(a_\lambda)(b_\lambda)=(a_\lambda b_\lambda),
\end{gather*}
and norm given by $(a_\lambda)\mapsto\|(a_\lambda)\|$.
Show that $A$ is unital or abelian if this is the case for all of the algebras $A_\lambda$.

The \emph{restricted sum} $B=\bigoplus_\lambda^{c_0}A_\lambda$ is the set of all elements $(a_\lambda)\in A$ such that for each $\e>0$ there exists a finite subset $F$ of $\Lambda$ for which $\|a_\lambda\|<\e$ if $\lambda\in\Lambda\setminus F$.
Show that $B$ is a closed ideal in $A$.
\end{prb}
\begin{sol}
First we show $A$ is a Banach algebra.
Banach algebras are defined to have \emph{four} structures(the addition, scalar multiplication, vector multiplication, and norm), which indicates that we need to show the following four conditions for $A$: it is closed under the three algebraic operations, and it is complete with the norm.
Here the only non-trivial one is completeness.
Suppose $((a_\lambda^n))_n$ is a Cauchy sequence in $A$.
Let $\e>0$ and find $n_0$ such that $n,n'>n_0$ implies $\|(a_\lambda^n)-(a_\lambda^{n'})\|<\e$.
For each $\lambda\in\Lambda$, the sequence $(a_\lambda^n)_n$ is Cauchy because $\|a_\lambda^n-a_\lambda^{n'}\|\le\|(a_\lambda^n)-(a_\lambda^{n'})\|$ for all $n$ and $n'$.
Define $a_\lambda:=\lim_{n\to\infty}a_\lambda^n$.
Then, limiting $n'\to\infty$ on the inequality $\|a_\lambda^n-a_\lambda\|<\e+\|a_\lambda^{n'}-a_\lambda\|$, we have $\|a_\lambda^n-a_\lambda\|<\e$ for arbitrary $\lambda$ and $n>n_0$, and it means $\|(a_\lambda^n)-(a_\lambda)\|\to0$ as $n\to\infty$.
Therefore, $A$ is complete.
An alternative solution uses extraction of a subsequence to assume $\|(a_\lambda^n)-(a_\lambda^{n+1})\|<\frac1{2^n}\e$.

The existence of multiplicative unit and the commutativity on $A$ is clear.

Next we show $B$ is a closed ideal of $A$.
Let $b\in B$ and $a\in A$.
Find a finite $F\in\Lambda$ such that $\lambda\in\Lambda\setminus F$ implies $\|b_\lambda\|<\frac1{\|a\|}\e$.
Then, $\max\{\|a_\lambda b_\lambda\|,\|b_\lambda a_\lambda\|\}\le\|a\|\|b_\lambda\|<\e$ implies $ab$ and $ba$ are in $B$.
For closedness, let $(b^n)_n$ be a sequence in $B$ converges to $a\in A$.
Let $\e>0$ and take $n$ be such that $\|b^n-a\|<\frac12\e$.
Find a finite $F\in\Lambda$ such that $\lambda\in\Lambda\setminus F$ implies $\|b_\lambda^n-a_\lambda\|<\frac12\e$.
Then, $\|a_\lambda\|<\|a_\lambda-b_\lambda^n\|+\|b_\lambda^n\|<\e$ proves $a\in B$.
\end{sol}
\begin{rmk}
Let $(X_\lambda)_{\lambda\in\Lambda}$ be a family of Banach spaces.
We may define their \emph{direct sum} as the completion of the algebraic direct sum of them.
The problem in here is the completion depends on the choice of norms on the algebraic direct sum.
Consider some function spaces on the index set $\Lambda$ such as $V=\ell^p(\Lambda)$ ($1\le p\le\infty$) or $c_0(\Lambda)$.
We define \emph{$V$-direct sum} of $(X_\lambda)$ as
\begin{align*}
\bigoplus_\lambda^{\ell^p}X_\lambda&:=\{\,(x_\lambda)\in\prod_\lambda X_\lambda:\sum_\lambda\|x_\lambda\|^p<\infty\,\},\quad(1\le p<\infty)\\
\bigoplus_\lambda^{\ell^\infty}X_\lambda&:=\{\,(x_\lambda)\in\prod_\lambda X_\lambda:\sup_\lambda\|x_\lambda\|<\infty\,\},\\
\bigoplus_\lambda^{c_0}X_\lambda&:=\{\,(x_\lambda)\in\prod_\lambda X_\lambda:\forall\e>0\ \exists F\subset\Lambda,\ |F|<\infty,\ \lambda\in\Lambda\setminus F\Rightarrow\|x_\lambda\|<\e\,\}.
\end{align*}
Note that $\ell^p$-direct sum for $1\le p<\infty$ and $c_0$-direct sum are completions of the algebraic direct sum with respect to the norms $\|(x_\lambda)\|=(\sum_\lambda\|x_\lambda\|^p)^{1/p}$ or $\sup_\lambda\|x_\lambda\|$, but $\ell^\infty$ cannot be recognized as a completion in general.
The spaces $V=\ell^p$ and $c_0$ are not properly generalized to arbitrary Banach spaces.
There are several ways to generalize the construction method mimicking the direct sum such as direct integrals or $\psi$-direct sums of Bonsall and Duncan.

For Banach algebras or C*-algebras, the case $V=\ell^\infty$ or $c_0$ is often chosen in order to make the multiplication closed.
The \emph{direct sum} and \emph{restricted sum} in this problem refer to $\ell^\infty$-direct sum and $c_0$-direct sum respectively, and the $\ell^\infty$-direct sum is usually in fact called \emph{product} rather than direct sum because it satisfies the universal property of categorical products in the category of Banach spaces in which morphisms are set to be norm-decreasing.
\end{rmk}

\begin{prb}
Let $A$ be a Banach algebra and $\Omega$ a non-empty set.
Denote by $\ell^\infty(\Omega,A)$ the set of all bounded maps $f$ from $\Omega$ to $A$.
Show that $\ell^\infty(\Omega,A)$ is a Banach algebra with the pointwise-defined operations and the sup-norm $\|f\|=\sup\{\,\|f(\omega)\|:\omega\in\Omega\,\}$.
If $\Omega$ is a compact Hausdorff space, show that the set $C(\Omega,A)$ of all continuous functions from $\Omega$ to $A$ is a closed subalgebra of $\ell^\infty(\Omega,A)$.
\end{prb}
\begin{sol}
The completeness of $\ell^\infty(\Omega,A)=\bigoplus_{\omega\in\Omega}A$ follows from the previous problem, so we will only show the closedness of $C(\Omega,A)$, and the rest is trivial.
the compactness of $\Omega$ is required to satisfy $C(\omega,A)\subset\ell^\infty(\Omega,A)$.

We follow the proof of a well-known fact: the uniform limit of continuous functions is continuous.
Let $(f_n)_n$ be a sequence of $C(\Omega,A)$ convergent to $f\in\ell^\infty(\Omega,A)$.
Fix $\omega\in\Omega$ and $\e>0$.
Take $n$ such that $\|f_n-f\|<\frac13\e$.
Let $\delta>0$ be such that $\|\omega-\omega'\|<\delta$ implies $\|f_n(\omega)-f_n(\omega')\|<\frac13\e$.
Combining inequalities, we easily get $\|\omega-\omega'\|<\delta$ implies $\|f(\omega)-f(\omega')\|<\e$.
\end{sol}


\begin{prb}
Give an example of a unital non-abelian Banach algebra $A$ in which 0 and $A$ are the only closed ideals.
\end{prb}
\begin{sol}
Let $A=M_2(\C)$.
Clearly $A$ is unital and non-abelian Banach algebra.
Let $I\subset A$ be a non-zero ideal and take $0\ne i\in I$.
The principal ideal
\[(i)=\{\,\sum_{k=1}^na_kib_k:a_k,b_k\in A,\ n\in\N\,\}\]
is also a non-zero ideal.
Suppose $i=\mat{x&y\\z&w}$.
Then, all of the following four matrices
\[\mat{x&0\\0&0},\ \mat{0&y\\0&0},\ \mat{0&0\\z&0},\ \mat{0&0\\0&w}\]
are of the form $aib$ for some $a,b\in A$ and hence belong to the ideal $(i)$, and at least one of them is non-zero.
By repeated swapping their columns and rows, and by summing them with appropriate scaling, we obtain all elements of $A$, which proves $(i)=A$.
Therefore, the only non-zero ideal of $A$ is $A$.
\end{sol}
\begin{rmk}
When a Banach algebra has only two closed ideals, 0 and itself, we say it is \emph{simple}.
In the same way as the solution, we can easily show $B(\C^n)=M_n(\C)$ is simple.
If a Banach space $X$ is infinite-dimensional, then $B(X)$ is not simple because it has a closed ideal $K(X)$, the space of compact operators.
If $X$ is further a Hilbert space, then $K(X)$ is the smallest non-zero closed ideal of $B(X)$ (Theorem 2.4.5 and 2.4.7).
If $X$ is further separable, then $K(X)$ is the unique proper non-zero closed ideal of $B(X)$ (Theorem 4.1.15).
\end{rmk}


\begin{prb}
Give an example of a non-modular maximal ideal in an abelian Banach algebra.
(If $A$ is the disc algebra, let $A_0=\{\,f\in A:f(0)=0\,\}$.
Then $A_0$ is a closed subalgebra of $A$ and admits an ideal of the type required.)
\end{prb}
\begin{sol}
Let $A_1:=\{\,f\in A_0:f'(0)=0\,\}$.
We will show $A_1$ is non-modular maximal ideal in $A_0$.
Note that $A_0^2\subset A_1$ because $(fg)'(0)=f'(0)g(0)+f(0)g'(0)=0$ for $f,g\in A_0$.
Thus, $A_1$ is a maximal ideal since $A_0A_1\subset A_0^2\subset A_1$ and $A_0/A_1\cong\C$ is simple.
Assume $A_1$ is modular so that there is $u\in A_0$ satisfying $f-fu\in A_1$ for all $f\in A_0$.
Then, $fu\in A_0^2\subset A_1$ implies $f\in A_1$, which is a contradiction.
Therefore, $A_1$ is modular.
\end{sol}


\begin{prb}
Let $A$ be a unital abelian Banach algebra.
\begin{parts}
\item
Show that $\sigma(a+b)\subset\sigma(a)+\sigma(b)$ and $\sigma(ab)\subset\sigma(a)\sigma(b)$ for all $a,b\in A$.
Show that this is not true for \emph{all} Banach algebras.
\item
Show that if $A$ contains an \emph{idempotent} $e$ (that is, $e=e^2$) other than 0 and 1, then $\Omega(A)$ is disconnected.
\item
Let $a_1,\cdots,a_n$ generate $A$ as a Banach algebra.
Show that $\Omega(A)$ is homeomorphic to a compact subset of $\C^n$.
More precisely, set
\[\sigma(a_1,\cdots,a_n)=\{\,(\tau(a_1),\cdots,\tau(a_n)):\tau\in\Omega(A)\,\}.\]
Show that the canonical map from $\Omega(A)$ to $\sigma(a_1,\cdots,a_n)$ is a homeomorphism.
\end{parts}
\end{prb}
\begin{sol}
(a)
It easily follows from
\[\sigma(a+b)=\im(\hat a+\hat b)\cup\{0\}\subset(\im\hat a\cup\{0\})+(\im\hat b\cup\{0\})=\sigma(a)+\sigma(b),\]
and
\[\sigma(ab)=\im(\hat a\hat b)\cup\{0\}\subset(\im\hat a\cup\{0\})(\im\hat b\cup\{0\})=\sigma(a)\sigma(b),\]
even for non-unital case.
If it is non-abelian such as $A=M_2(\C)$, then
\[\sigma((\mat[small]{0&1\\1&0}))=\{\pm1\}\not\subset\{0\}=\sigma((\mat[small]{0&1\\0&0}))+\sigma((\mat[small]{0&0\\1&0}))\]
and
\[\sigma((\mat[small]{1&0\\0&0}))=\{0,1\}\not\subset\{0\}=\sigma((\mat[small]{0&1\\0&0}))\sigma((\mat[small]{0&0\\1&0})).\]

(b)
By the Gelfan representation we have $\im\hat e=\{0,1\}$ so that the two closed subsets $\hat e^{-1}(0)$ and $\hat e^{-1}(1)$ partition the character space $\Omega(A)$.
Since $e$ is neither 0 nor 1, the both subsets are non-empty, so $\Omega(A)$ is disconnected.

(c)
The set $\sigma(a_1,\cdots,a_n)$ is called the \emph{joint spectrum} of $a_1,\cdots,a_n$.
For the canonical map $\Omega(A)\to\sigma(a_1,\cdots,a_n)$, the surjectivity is obvious by definition, and the continuity follows from the fact that the topology of $\Omega(A)$ is induced from the weak* topology of the dual $A^*$.
The injectivity is due to the assumption that $a_1,\cdots,a_n$ are generators of $A$: if $\tau(a_i)=\tau'(a_i)$ for $i=1,\cdots,n$, then it must be that $\tau(a)=\tau(a)$ for all $a\in A$, and eventually $\tau=\tau'$.
Since a continuous bijection from a compact set to a Hausdorff space is homeomorphism, the canonical map is homeomorphism.
\end{sol}


\begin{prb}
Let A be a unital Banach algebra.
\begin{parts}
\item
If $a$ is invertible in $A$, show that $\sigma(a^{-1})=\{\,\lambda^{-1}:\lambda\in\sigma(a)\,\}$.
\item
For any element $a\in A$, show that $r(a^n)=(r(a))^n$.
\item
If $A$ is abelian, show that the Gelfand representation is isometric if and only if $\|a^2\|=\|a\|^2$ for all $a\in A$.
\end{parts}
\end{prb}
\begin{sol}
(a), (b)
It easily follows from the Gelfand representation.

(c)
If the Gelfand representation is isometric, then
\[\|a^2\|=\|(a^2)^{\hat{\ }}\|=r(a^2)=r(a)^2=\|\hat a\|^2=\|a\|^2.\]
Conversely,
\[\|\hat a\|=r(a)=\lim_{n\to\infty}\|a^n\|^{1/n}=\lim_{n\to\infty}\|a^{2^n}\|^{1/2^n}=\|a\|.\qedhere\]
\end{sol}


\begin{prb}
Let $A$ be a Banach algebra.
Show that the spectral radius function $r:A\to\R$ is upper semi-continuous.
(One can show that $r$ is not in general continuous [Hal, Problem 104].)
\end{prb}
\begin{sol}
We may assume $A$ is unital since the spectral radius function of non-unital Banach algebra is the restriction of spectral radius function of its unitization.
Suppose $r$ is not upper semi-continuous so that there is a sequence $a_n$ in $A$ that converges to $a$ and satisfies $r(a_n)>r(a)+\e$ for all $n$ where $\e>0$.
The compactness of spectrum lets some $\lambda_n\in\sigma(a_n)$ satisfy $|\lambda_n|=r(a_n)$ for each $n$.
Since $r(a_n)$ is uniformly bounded by $\sup_n\|a_n\|$, we may assume $\lambda_n$ converges to $\lambda$ by extracting subsequence.
The limit $\lim_{n\to\infty}(a_n-\lambda_n)=a-\lambda$ is not invertible because the set of invertible elements is open, thus we have $\lambda\in\sigma(a)$.
Limiting $|\lambda_n|=r(a_n)>r(a)+\e$, we obtain a contradiction $|\lambda|>r(a)$.
Therefore, $r$ is upper semi-continuous.
\end{sol}


\begin{prb}
Show that if $B$ is a maximal abelian subalgebra of a unital Banach algebra $A$, then $B$ is closed and contains the unit.
Show that $\sigma_A(b)=\sigma_B(b)$ for all $b\in B$.
\end{prb}
\begin{sol}
If $A$ is abelian, $A=C([-1,1])$ for example, then $\{\,f\in A:f(0)=0\,\}$ is a proper maximal abelian subalgebra of $A$ that does not contain the unit of $A$.
Henceforth, we might as well understand the maximality of $B$ as something may not be proper.
Nevertheless, in the only non-trivial situation that $A$ is non-abelian, $B$ is always proper so that we are allowed not to dig deeper.

Assume $A$ is non-abelian.
Let $B'$ be the \emph{commutant} of $B$ which is defined as the set containing all $a\in A$ which commute every $b\in B$.
Since $B$ is abelian, $B\subset B'$.
If $b\in B'\setminus B$ exsits, then the Banach subalgebra generated by $B\cup\{b\}$ is abelian subalgebra of $A$ contradicting the maximality of $B$, thus we have $B=B'$.
We can easily show that $B'$ is closed and contains unit of $A$.

If $b\in B$ is invertible in $A$, then its inverse is also in $B'=B$, and it easily deduces $\sigma_A(b)=\sigma_B(b)$ for $b\in B$.
\end{sol}


\begin{prb}
Let $(\Omega,\mu)$ be a measure space.
Show that the linear span of the idempotents is dense in $L^\infty(\Omega,\mu)$.
Show that the spectrum of the Banach algebra $L^\infty(\Omega,\mu)$ is totally disconnected, by showing that if $A$ is an arbitrary abelian Banach algebra in which the idempotents have dense linear span, its spectrum $\Omega(A)$ is totally disconnected.
\end{prb}
\begin{sol}
Idempotents exactly coincide with measurable characteristic functions in $L^\infty(\Omega,\mu)$, that is, the linear span of idempotents means the space of simple functions.
For $f\in L^\infty(\Omega,\mu)$ and $\e>0$, if we let $E_k:=f^{-1}([(k-\frac12)\e,(k+\frac12)\e))$, then at most finite number of them has positive measure so that a function $g:=\sum_{k=-\infty}^{\infty}\chi_{E_k}$ is equal to a simple function almost everywhere, and we have $\|f-g\|\le\frac12\e<\e$.
This proves the linear span of idempotents is dense in $L^\infty(\Omega,\mu)$

Let $A$ be an abelian Banach algebra in which the idempotents have dense linear span.
In order to show the total disconnectedness, we will separate arbitrary pair of distinct points of the spectrum $\Omega(A)$ with clopen sets, using the Gelfand representation of idempotents.
Let $\tau$ and $\tau'$ be distinct in $\Omega(A)$.
For any idempotent $e$, we have $\tau(e)=\tau'(e)$, because $\hat e^{-1}(\tau(e))$ and $\hat e^{-1}(\tau'(e))$ are clopen subsets of $\Omega(A)$ that separate $\tau$ and $\tau'$ if it is not the case.
The desity of idempotents obtains the equality $\tau(a)=\tau'(a)$ for all $a\in A$, and this contradicts to the assumption $\tau\ne\tau'$.
Therefore, the spectrum $\Omega(A)$ is totally disconnected.
\end{sol}


\begin{prb}
Let $A=C^1([0,1])$, as in Example 1.2.6.
Let $x:[0,1]\to\C$ be the inclusion.
Show that $x$ generates $A$ as a Banach algebra.
If $t\in[0,1]$, show that $\tau_t$ belongs to $\Omega(A)$, where $\tau_t$ is defined by $\tau_t(f)=f(t)$, and show that the map $[0,1]\to\Omega(A),\ t\mapsto\tau_t$, is a homeomorphism.
Deduce that $r(f)=\|f\|_\infty\ (f\in A)$.
Show that the Gelfand representation is not surjective for this example.
\end{prb}
\begin{sol}
Note that $x$ does not generate $A$, but we will show $\{1,x\}$ generates $A$.
Let $f\in C^1([0,1])$ and $\e>0$.
By the Stone-Weierstrass theorem, there is $h\in\C[x,\bar x]\subset A$ such that $\|f'-h\|_\infty<\frac12\e$ since $f'\in C([0,1])$.
Because $x=\bar x$ on the real interval $[0,1]$, the function $h$ is a polynomial of $x$.
Define a polynomial $g$ such that $g(t):=f(0)+\int_0^th(s)\,ds$.
Then,
\[|f(t)-g(t)|\le\int_0^t|f'(s)-h(s)|\,ds\le\|f'-h\|_\infty<\frac12\e\]
implies $\|f-g\|=\|f-g\|_\infty+\|f'-h\|_\infty<\e$.
Thus the set of polynomials is dense in $A$, that is, the set $\{1,x\}$ generates $A$.

The functional $\tau_t$ on $A$ is clearly a character, to show this, we only need to check that it is non-zero and it preserves the three algebraic structures of Banach algebras.
For the map $t\mapsto\tau_t$ given in the problem, the injectivity and continuity is also clear.
We are mainly required to show its surjectivity, and suppose not.
For some $\tau\in\Omega(A)$, there can find $f_t\in A$ such that $\tau_t(f_t)\ne\tau(f_t)$ for each $t\in[0,1]$.
Replacing $f_t$ to $f_t-\tau(f_t)$, we may assume $\tau_t(f_t)\ne\tau(f_t)=0$.
Consider an open cover $\{U_t\}_{t\in[0,1]}$ of $[0,1]$ defined by $U_t:=f_t^{-1}(\C\setminus\{0\})$.
From the compactness of $[0,1]$, we have a finite subset $F\subset[0,1]$ such that $\{U_t\}_{t\in F}$ is still a cover.
Define $f:=\sum_{t\in F}|f_t|^2=\sum_{t\in F}\bar f_tf_t\in A$, which satisfies $f(t)>0$ for all $t\in[0,1]$.
Then, $f$ is invertible in $A$ since the inverse $1/f$ itself and its derivative $(1/f)'=-f'/f^2$ are both continuous, however, we have $\tau(f)=\sum_{t\in F}\tau(\bar f_t)\tau(f_t)=0$, which is a contradiction.
Therefore, the map $[0,1]\to\Omega(A)$ is surjective, and thus is a continuous bijection between compact Hausdorff spaces, so we are done.

For the equation $r(f)=\|f\|_\infty$, we have two ways to show it.
First, it follows from the fact that $\lambda\in\im f$ if and only if $f-\lambda$ is invertible in $A$.
Second, since the composition of the Gelfand representation of $A$ and the isomorphism induced by the above homeomorphism
\[A=C^1([0,1])\to C(\Omega(A))\xrightarrow{\sim}C([0,1])\]
is realized as the inclusion $f\mapsto f$ by the relation $f(t)=\tau_t(f)=\hat f(\tau_t)$, the spectral radius $r(f)$ eqauls to be the norm $\|f\|_\infty$ on $C([0,1])$.
Obviously, we can check the Gelfand representation is not surjective.
\end{sol}


\begin{prb}
Let $A$ be a unital Banach algebra and set
\[\zeta(a)=\inf_{\|b\|=1}\|ab\|\qquad(a\in A).\]
We say that an element $a$ of $A$ is a \emph{left topological zero divisor} if there is a sequence of unit vectors $(a_n)$ of $A$ such that $\lim_{n\to\infty}aa_n=0$.
Equivalently, $\zeta(a)=0$.
\begin{parts}
\item
Show that left topological zero divisors are not invertible.
\item
Show that $|\zeta(a)-\zeta(b)|\le\|a-b\|$ for all $a,b\in A$.
Hence, $\zeta$ is a continuous function.
\item
If $a$ is a boundary point of the set $\Inv(A)$ in $A$, show that there is a sequence of invertible elements $(v_n)$ converging to a such that $\lim_{n\to\infty}\|v_n^{-1}\|^{-1}=0$.
Using the continuity of $\zeta$, deduce that $\zeta(a)=0$.
Thus, boundary points of $\Inv(A)$ are left topological zero divisors.
In particular, if $\lambda$ is a boundary point of the spectrum of an element $a$ of $A$, then $\lambda-a$ is a left topological zero divisor.
\item
Let $\Omega$ be a compact Hausdorff space and let $A=C(\Omega)$.
Show that in this case the topological zero divisors are precisely the non-invertible elements (if $f$ is non-invertible, then 0 is a boundary point of the spectrum of $\cl ff$).
\item
Give an example of a unital Banach algebra and a non-invertible element that is not a left topological zero divisor.
\end{parts}
\end{prb}
\begin{sol}
(a)
Let $a$ is a left topological zero divisor and $a_n$ be a sequence with $aa_n\to0$ as $n\to\infty$.
If $a^{-1}$ exists, then $\|a_n\|\le\|a^{-1}\|\|aa_n\|\to0$ implies $a=0$, a contradiction.

(b)
We have
\begin{align*}
\zeta(a)=\inf_{\|c\|=1}\|ac\|
&\le\inf_{\|c\|=1}(\|ac-bc\|+\|bc\|)\\
&\le\inf_{\|c\|=1}(\|a-b\|+\|bc\|)=\|a-b\|+\zeta(b).
\end{align*}
Exchanging $a$ and $b$, we get the desired inequality.

(c)
We show a little bit stronger result that any sequences $(v_n)_n$ in $\Inv(A)$ that converge to $a$ satisfy $\lim_{n\to\infty}\|v_n^{-1}\|^{-1}=0$.
Suppose not and find a bound $\|v_n^{-1}\|\le M$ by taking a subsequence of $v_n$, if necessary.
Then, since
\[\|v_n^{-1}-v_{n'}^{-1}\|\le\|v_n^{-1}\|\|v_n-v_{n'}\|\|v_{n'}^{-1}\|\le M^2\|v_n-v_{n'}\|,\]
the sequence $v_n^{-1}$ is Cauchy.
It limit must be the inverse of $a$ because
\[\max\{\|av_n^{-1}-1\|,\|v_n^{-1}a-1\|\}\le\|v_n^{-1}\|\|a-v_n\|\to0\]
as $n\to\infty$.
It contradicts to the fact $\Inv(A)$ is open.
The boundary point $a$ is a left topological zero divisor becasue $\zeta(a)\le\|av_n^{-1}\|\|v_n^{-1}\|^{-1}\to0$ when $n\to\infty$.

(d)
Let $f$ be a non-invertible element of $A=C(\Omega)$ and take $\omega_0\in\Omega$ such that $f(\omega_0)=0$.
We suggest two solutions.
For the first, construct a sequence $(f_n)_n$ in $A$ as
\[f_n(\omega):=\begin{cases}1-\frac{\|\omega-\omega_0\|}n&,\|\omega-\omega_0\|\le\frac1n\\0&,\|\omega-\omega_0\|>\frac1n\end{cases}.\]
They have $\|f_n\|=1$.
For $\e>0$, find $\delta>0$ such that $|f(\omega)|<\e$ for $\|\omega-\omega_0\|<\delta$ and take $n>\delta^{-1}$.
Then, $|f(\omega)f_n(\omega)|<\e$ for all $\omega\in\Omega$, so $f$ is a topological zero divisor.

Second, since $\bar ff$ is not invertible and has a spectrum on the positive real line, $\sigma(\bar ff)$ has 0 as a boundary point and we can use the part (c) of this problem to deduce $\bar ff$ is a topological zero divisor.
If a sequence $(g_n)_n$ of unit vectors in $A$ such that $\bar ffg_n$ converges to 0, then
so does $|fg_n|^2=\bar g_n\bar ffg_n$, so we can find out that $f$ is a topological zero divisor.

(e)
Let $A$ be the disc algebra, the algebra of continuous functions on the closed unit disc $\D$ and holomorphic on the interior.
If $\|f\|=1$, then by the maximum modulus principle we have $|z|=1$ such that $|f(z)|=1$, thus $|zf(z)|=1$ and $\|if\|=1$, where $i:\D\to\C$ refers to the inclusion, which is non-invertible.
This argues that $i\in A$ is never a topological divisor.
\end{sol}


\begin{prb}
A derivation on an algebra $A$ is a linear map $d:A\to A$ such that $d(ab)=adb+d(a)b$.
Show that the Leibnitz formula,
\[d^n(ab)=\sum_{r=0}^n\binom nrd^r(a)d^{n-r}(b)\qquad(n=1,2,\cdots),\]
holds.
\end{prb}
\begin{sol}
Use the induction argument.
Clear for $n=1$.
If the formula holds for $n$, then
\begin{align*}
d^{n+1}(ab)
&=\sum_{r=0}^n\binom nrd^r(a)d^{n-r+1}(b)+\sum_{r=0}^n\binom nrd^{r+1}(a)d^{n-r}(b)\\
&=\sum_{r=0}^n\binom nrd^r(a)d^{n-r+1}(b)+\sum_{r=1}^{n+1}\binom n{r-1}d^r(a)d^{n-r+1}(b)\\
&=\sum_{r=0}^{n+1}\binom{n+1}rd^r(a)d^{n-r+1}(b)
\end{align*}
since $\binom nr+\binom n{r-1}=\binom{n+1}r$.
\end{sol}


\begin{prb}
Suppose that $d$ is a bounded derivation on a unital Banach algebra $A$ and $\lambda\in\C\setminus\{0\}$ such that $da=\lambda a$.
Show that $a$ is nilpotent, that is, that $a^n=0$ for some positive integer $n$ (use the boundedness of $\sigma(d)$).
\end{prb}
\begin{sol}
The derivation is a bounded linear operator on a Banach space $A$, so we can consider the spectrum $\sigma(d)$ in a unital Banach algebra $B(A)$.
If $a^n\ne0$ for all positive integer $n$, then $d(a^n)=na^{n-1}d(a)=\lambda na^n$ implies $\lambda n\in\sigma(d)$, which is a contradiction to the boundedness of $\sigma(d)$.
\end{sol}


\begin{prb}
Suppose that $d$ is a bounded derivation on a unital Banach algebra $A$, and that $a\in A$ and $d^2a=0$.
Show that $da$ is quasinilpotent.
(Hint: Show that $d^{n+1}(a^n)=0$ and hence, $d^n(a^n)=n!(da)^n$.)
For $a\in A$, the map $b\mapsto[a,b]=ab-ba$ is a bounded derivation on $A$.
Therefore, the Kleinecke-Shirokov theorem holds: If $[a,[a,b]]=0$, then $[a,b]$ is quasinilpotent.
\end{prb}
\begin{sol}
We have $d^n(a^n)=d^{n-1}(na^{n-1}da)=d^{n-2}(n(n-1)a^{n-2}(da)^2)=\cdots=n!(da)^n$.
Then,
\[\|(da)^n\|=\frac1{n!}\|d^n(a^n)\|\le\frac1{n!}(\|d\|\|a\|)^n\to0\]
as $n\to\infty$, so $da$ is quasi-nilpotent.
\end{sol}


\begin{prb}
Let $H$ be a Hilbert space with an orthonormal basis $(e_n)_{n=1}^\infty$ and let $u$ be an operator in $B(H)$ diagonal with respect to $(e_n)$ with diagonal the sequence $(\lambda_n)$.
Show that $u$ is compact if and only if $\lim_{n\to\infty}\lambda_n=0$.
\end{prb}
\begin{sol}
($\Rightarrow$)
Suppose the sequence $(\lambda_n)_n$ does not converge to 0.
We may assume, taking a subsequence, there is $M>0$ such that $|\lambda_n|\ge M$ for all $n$.
Then, the sequence $u(e_n)=\lambda_ne_n$ has no convergent subsequences since $\|\lambda_ne_n-\lambda_{n'}e_{n'}\|^2=\|\lambda_ne_n\|^2+\|\lambda_{n'}e_{n'}\|^2\ge 2M^2$ for all pairs $n$ and $n'$, hence $u$ is not compact.

($\Leftarrow$)
Let $(x^k)_k$ be an arbitrary sequence of $H$ with $\|x^k\|\le1$.
Using the diagonal argument, take a subsequence $(u(x^j))_j$ of $(u(x^k))_k$ such that for each $n$ we have $\<u(x^j),e_n\>=\lambda_n\<x^j,e_n\>\to y_n$ for some $y_n$ as $j\to\infty$.
Note that for any finite $F\subset\N$ we have
\[\sum_{n\in F}|y_n|^2=\lim_{j\to\infty}\sum_{n\in F}|\lambda_n\<x^j,e_n\>|^2\le\max_{n\in F}|\lambda_n|^2\]
since $\|x^j\|\le1$.
Limiting $F\uparrow\N$, the sum $y:=\sum_{n=1}^\infty y_ne_n$ can be shown to be in $H$.
We will show the sequence $(u(x^j))_j$ converges to $y$.

For $\e>0$, let $n_0$ be such that $|\lambda_n|<\frac1{2\sqrt2}\e$ for $n>n_0$.
With this $n_0$, define $j_0$ such that $j>j_0$ implies $|\<u(x^j),e_n\>-y_n|<\frac1{\sqrt{2n_0}}\e$ for each $1\le n\le n_0$.
Then we have
\begin{align*}
\|u(x^j)-y\|^2
&=\sum_{n\le n_0}|\<u(x^j),e_n\>-y_n|^2+\sum_{n>n_0}|\<u(x^j),e_n\>-y_n|^2\\
&\le\frac12\e^2+\sum_{n>n_0}2(|\<u(x^j),e_n\>|^2+|y_n|^2)\\
&\le\frac12\e^2+\sup_{n>n_0}2(|\lambda_n|^2+|\lambda_n|^2)<\e^2
\end{align*}
for all $j>j_0$.
The second inequality is due to limiting $F\uparrow\N_{>n_0}$.
Therefore, $u(x^j)$ converges to $y$, and hence $u$ is compact.
\end{sol}


\begin{prb}
Let $X$ be a Banach space.
If $p\in B(X)$ is a compact idempotent, show that its rank is finite.
\end{prb}
\begin{sol}
For arbitrary $x\in X$, we have $x=p(x)+(1-p(x))$.
Moreover, if $x=p(y)=z-p(z)$ for some $y$ and $z\in X$, then $x=p(y)=p^2(y)=p(z-p(z))=p(z)-p^2(z)=0$.
Thus we have that $X=\im p\oplus\im(1-p)$.
Since $1-p$ is Fredholm of index zero, the image of $1-p$ has finite condimension, which means $\im p$ has finite dimension.
\end{sol}


\begin{prb}
Let $u:X\to Y$ be a compact operator between Banach spaces.
Show that if the range of $u$ is closed, then it is finite-dimensional.
(Hint: Show that the well-defined operator
\[X/\ker(u)\to u(X),\quad x+\ker(u)\mapsto u(x),\]
is an invertible compact operator.)
\end{prb}
\begin{sol}
Let $\tilde u$ be the map given in the hint.
By definition, $\tilde u$ is a bounded bijection.
Since $u(X)$ is closed and hence a Banach space, $\tilde u$ is invertible by the open mapping theorem.
Also, $\tilde u$ is compact beacuse $\tilde u(\cl B_{X/\ker(u)})=u(\cl B_X)$.
The existence of an invertible compact operator $T:X\to Y$ implies that $\id_X=T^{-1}T$ and $\id_Y=TT^{-1}$ are compact, therefore $X$ and $Y$ are finite-dimensional.
\end{sol}


\begin{prb}
Let $X,Y$ be Banach spaces and suppose that $u\in B(X,Y)$ has compact transpose $u^*$.
Show that $u$ is compact using the fact that $u^{**}$ is compact.
\end{prb}
\begin{sol}
If we think of as $X\subset X^{**}$ with the natural isometric embedding, $u=u^{**}|_X$.
Since the transpose and restriction of compact operators are compact, $u$ is compact.
\end{sol}


\begin{prb}
Let $u:X\to Y$ and $u':X'\to Y'$ be bounded operators between Banach spaces.
Show that the linear map
\[u\oplus u':X\oplus X'\to Y\oplus Y',\quad(x,x')\mapsto(u(x), u'(x')),\]
is bounded with norm $\max\{\|u\|,\|u'\|\}$.
Show that if $u$ and $u'$ are Fredholm operators, so is $u\oplus u'$, and $\ind(u\oplus u')=\ind(u)+\ind(u')$.
\end{prb}
\begin{sol}
Assume $\|u\|\ge\|u'\|$.
For $\|(x,x')\|=\|x\|+\|x'\|=1$, we have $\|(u\oplus u')(x,x')\|=\|u(x)\|+\|u'(x')\|\le\|u\|\|x\|+\|u'\|\|x'\|\le\|u\|$.
Also, for $\e>0$, if we choose a unit vector $x$ such that $\|u(x)\|>\|u\|+\e$, then we have $\|(x,0)\|=1$ and $\|u\|+\e<\|u(x)\|=\|(u\oplus u')(x,0)\|\le\|u\oplus u'\|$.
Thus, $\|u\oplus u'\|=\|u\|$.

Since $\ker(u\oplus u')=\ker u\oplus\ker u'$ and $\im(u\oplus u')=\im u\oplus\im u'$, it follows that nullity, defect, and index are all additive.
\end{sol}


\begin{prb}
If $X$ is an infinite-dimensional Banach space and $u\in B(X)$, show that
\[\bigcap_{v\in K(X)}\sigma(u+v)=\sigma(u)\setminus\{\,\lambda\in\C:u-\lambda\text{ is Fredholm of index zero}\,\}.\]
\end{prb}
\begin{sol}
($\subset$)
Let $\lambda\in\sigma(u+v)$ for all $v\in K(X)$.
Letting $v=0$, $\lambda\in\sigma(u)$.
If $u-\lambda$ is Fredholm of index zero, then there is an invertible $w$ such that $v:=1-w(u-\lambda)$ is finite-rank so that $u+w^{-1}v-\lambda=w^{-1}$ is invertible, hence $\lambda\notin\sigma(u+w^{-1}v)$ for compact $w^{-1}v$.

($\supset$)
Let $\lambda$ belong to the right-hand side.
Then, $u-\lambda$ is Fredholm with non-zero index, thus the element $u+v-\lambda$ cannot be invertible for any compact $v$.
\end{sol}





\section{C*-algebras and Hilbert space operators}

\begin{prb}
Let $A$ be a Banach algebra such that for all $a\in A$ the implication
\[Aa=0\quad\text{or}\quad aA=0\quad\Rightarrow\quad a=0\]
holds.
Let $L,R$ be linear mappings from $A$ to itself such that for all $a,b\in A$,
\[L(ab)=L(a)b,\quad R(ab)=aR(b),\quad\text{and}\quad R(a)b=aL(b).\]
Show that $L$ and $R$ are necessarily continuous.
\end{prb}
\begin{sol}
The continuity of $L$ and $R$, in fact, only requires the condition $R(a)b=aL(b)$.
Let $(b_n)_n$ be a sequence of $A$ converging to 0.
Since for every $a\in A$ we have $aL(b_n)=R(a)b_n\to0$, we get $L(b_n)\to0$ as $n\to\infty$ and the continuity of $L$.
The continuity of $R$ is shown in the same manner.
\end{sol}


\begin{prb}
Let $A$ be a unital C*-algebra.
\begin{parts}
\item
If $a,b$ are positive elements of $A$, show that $\sigma(ab)\subset\R^+$.
\item
If $a$ is an invertible element of $A$, show that $a=u|a|$ for a unique unitary $u$ of $A$.
Give an example of an element of $B(H)$ for some Hilbert space $H$ that cannot be written as a product of a unitary times a positive operator.
\item
Show that if $a\in\Inv(A)$, then $\|a\|=\|a^{-1}\|=1$ if and only if a is a uni tary.
\end{parts}
\end{prb}
\begin{sol}
(a)
Using $\sigma(ab)\cup\{0\}=\sigma(ba)\cup\{0\}$, we have $\sigma(ab)=\sigma(a^{1/2}(a^{1/2}b))\subset\sigma(a^{1/2}ba^{1/2})\cup\{0\}\subset\R^+$.

(b)
Note that the absolute value for non-normal element $a$ is defined to be $|a|:=(a^*a)^{1/2}$.
Uniqueness follows from the invertibility of $a$ (or $|a|$).
It is enough to show $a|a|^{-1}$ is unitary, and it is true:
\[(a|a|^{-1})^*(a|a|^{-1})=|a|^{-1}a^*a|a|^{-1}=1,\]
and
\[(a|a|^{-1})(a|a|^{-1})^*=a|a|^{-2}a^*=a(a^*a)^{-1}a^*=1.\]

We claim that all non-invertible isometries cannot be written as a product of a unitary and a positive.
Let $b\in B(H)$ be a positive isometry.
Then, $b^*=b$ and $b^*b=1$ imply $b^2=1$, and since $a$ is positive, we get $b=(b^2)^{1/2}=1$.
Therefore, if we take a non-invertible isometry $a\in A$, then the assumption that $a=up$ for unitary $u$ and positive $p$ leads to a contradiction that $u^{-1}a$ is a positive isometry so that $a$ have to be $u$ and invertible.
One of the typical examples of non-invertible isometries is the unilateral shift on $\ell^2(\N)$.

(c)
The backward implication is trivial, so let $\|a\|=\|a^{-1}\|=1$.
The condition in the problem is equivalent to $\|a^*a\|=\|(a^*a)^{-1}\|=1$ by the C*-identity.
Since $a^*a$ is normal, we can apply the spectral radius formla $\|a^*a\|=r(a^*a)$, which proves that $\sigma(a^*a)$ is a subset of the unit circle.
Furthermore, we get $\sigma(a^*a)=\{1\}$ because $a^*a$ is positive, so $\sigma(a^*a-1)=\{0\}$ implies $\|a^*a-1\|=r(a^*a-1)=0$ and $a$ is finally a unitary.
\end{sol}
\begin{rmk}
We should be careful on the use of polar decomposition theorem becasue it may not hold in general C*-algebra if it is not of the form $B(H)$ for Hilbert space $H$.
Note the lack of normality implies $u$ and $v$ such that $a=u|a|^{-1}$ and $a=|a|^{-1}v$ may differ.
\end{rmk}


\begin{prb}
Let $\Omega$ be a locally compact Hausdorff space, and suppose that the C*-algebra $C_0(\Omega)$ is generated by a sequence of projections $(p_n)_{n=1}^\infty$.
Show that the hermitian element $h=\sum_{n=1}^\infty p_n/3^n$ generates $C_0(\Omega)$.
\end{prb}
\begin{sol}
If we show $h$ is injective and vanishes nowhere, then the single function $h=h^*$ generates the whole $C_0(\Omega)$ by the Stone-Weierstrass theorem for locally compact spaces.

Suppose $x$ and $y$ be distinct points of $\Omega$ such that $h(x)=h(y)$.
We claim that $p_n(x)=p_n(y)$ for all $n$.
Let $h_n:=\sum_{k=n}^\infty p_k/3^{k-n+1}$ and check that we have $h_1=h$, $3h_n=p_n+h_{n+1}$, and $0\le h_n\le\frac12$.
It is clear that $h_1(x)=h_1(y)$.
If $h_n(x)=h_n(y)$, then the following inequality
\[|p_n(x)-p_n(y)|\le3|h_n(x)-h_n(y)|+|h_{n+1}(x)-h_{n+1}(y)|\le3\cdot0+\frac12\]
proves $p_n(x)=p_n(y)$ since $p_n$ can only have either 0 or 1 as its value, and we also obtain $h_{n+1}(x)=h_{n+1}(y)$, which hence gets the claim by the induction.
Take any function $f\in C_0(\Omega)$ satisfying $f(x)\ne f(y)$ so that it cannot belong to the generated *-subalgebra by $(p_n)_n$, so there must be $n$ such that $p_n(x)\ne p_n(y)$, which is a contradiction.
Therefore, $h$ is injective.

If $h(x)=0$, then $p_n(x)=0$ for all $n$.
For a function $f\in C_0(\Omega)$ that does not vanish at $x$, it also cannot belong to the generated *-subalgebra by $(p_n)_n$, we have $h(x)>0$ for all $x\in\Omega$.
\end{sol}


\begin{prb}
We shall see in the next chapter that all closed ideals in C*-algebras are necessarily self-adjoint.
Give an example of an ideal in the C*-algebra $C(\D)$ that is not self-adjoint.
\end{prb}
\begin{sol}
Let $I:=\{zf(z):f\in C(\D)\}$ be the principal ideal of the inclusion $z:\D\to\C$ in $C(\D)$.
Then, $z\in I$ but $\cl z\notin I$ since the function $z/\cl z:\D\setminus\{0\}\to\C$ cannot be continuously extended to 0; $\lim_{x\to0}x/\cl{x}=1$ but $\lim_{x\to0}ix/\cl{ix}=-1$ where $x\in\R$.
\end{sol}


\begin{prb}
Let $\f:A\to B$ be an isometric linear map between unital C*-algebras $A$ and $B$ such that $\f(a^*)=\f(a)^*$ ($a\in A$) and $\f(1)=1$.
Show that $\f(A^+)\subset B^+$.
\end{prb}
\begin{sol}
Let $a\in A$ be a positive element and assume $0\le a\le2$ without loss of generality.
Then, $\f(a)$ is also hermitian and $\|1-\f(a)\|=\|\f(1-a)\|=\|1-a\|\le1$ implies $\f(a)\ge0$
\end{sol}
\begin{rmk}
Also for a $^*$-homomorphism $\f:A\to B$, we have $\f(A^+)\subset B^+$ since $\sigma(\f(a))\subset\sigma(a)\cup\{0\}$.
\end{rmk}

\begin{prb}
Let $A$ be a unital C*-algebra.
\begin{parts}
\item
If $r(a)<1$ and $b=(\sum_{n=0}^\infty a^{*n}a^n)^{1/2}$, show that $b\ge1$ and $\|bab^{-1}\|<1$.
\item
For all $a\in A$, show that
\[r(a)=\inf_{b\in\Inv(A)}\|bab^{-1}\|=\inf_{b\in A_{sa}}\|e^bae^{-b}\|.\]
\end{parts}
\end{prb}
\begin{sol}
(a)
The convergence of the series defining $b$ follows from the condition $r(a)<1$.
Since $a^{*n}a^n=(a^n)^*(a^n)\ge0$, we have $b=(1+\sum_{n=1}^\infty a^{*n}a^n)^{1/2}\ge1^{1/2}=1$, and the invertibility of $b$ is obtained.
Then, because $b^{-2}>0$,
\[\|bab^{-1}\|^2=\|(b^{-1}a^*b)(bab^{-1})\|=\|b^{-1}(b^2-1)b^{-1}\|=\|1-b^{-2}\|<1.\]

(b)
An element $a\in A$ has the form $e^b$ for a self-adjoint $b\in A$ if and only if $a$ is invertible and positive.
Since $r(a)=r(b^{-1}(ba))=r(bab^{-1})\le\|bab^{-1}\|$ for invertible $b$, we clearly have
\[r(a)\le\inf_{b\in\Inv(A)}\|bab^{-1}\|\le\inf_{b\in A_{sa}}\|e^bae^{-b}\|=\inf_{b\in\Inv(A)\cap A^+}\|bab^{-1}\|.\]
Let $\e>0$.
Since $r(\frac a{r(a)+\e})<1$, there is an invertible positive $b$ such that $\|b\frac a{r(a)+\e}b^{-1}\|<1$ by the part (a) of this problem.
Therefore, limiting $\e\to0$ on $\inf_{b\in\Inv(A)\cap A^+}\|bab^{-1}\|<r(a)+\e$, we get the desired statement.
\end{sol}


\begin{prb}
Let $A$ be a unital C*-algebra.
\begin{parts}
\item
If $a,b\in A$, show that the map $f:\C\to A,\ \lambda\mapsto e^{i\lambda b}ae^{-i\lambda b}$, is differentible and that $f'(0)=i(ba-ab)$.
\item
Let $X$ be a closed vector subspace of $A$ which is unitarily invariant in the sense that $uXu^*\subset X$ for all unitaries $u$ of $A$.
Show that $ba-ab\in X$ if $a\in X$ and $b\in A$.
\item
Deduce that the closed linear span $X$ of the projections in $A$ has the property that $a\in X$ and $b\in A$ implies that $ba-ab\in X$.
\end{parts}
\end{prb}
\begin{sol}
(a)
Define $g:\C\to A$ by $g(\lambda):=e^{i\lambda b}i(ba-ab)e^{-i\lambda b}$.
Then,
\begin{align*}
&\|f(\lambda+\mu)-f(\lambda)-\mu g(\lambda)\|\\
&\qquad=\|e^{i(\lambda+\mu)b}a(1-e^{i\mu b}+i\mu b)e^{-i(\lambda+\mu)b}
+e^{i\lambda b}(e^{i\mu b}-1-i\mu b)ae^{-i\lambda b}\\
&\qquad\qquad-\mu(e^{i(\lambda+\mu)b}iabe^{-i(\lambda+\mu)b}-e^{i\lambda b}iabe^{-i\lambda b})\|\\
&\qquad\le2\|a\|\|e^{i\mu b}-1-i\mu b\|+2|\mu|\|ab\|\to0
\end{align*}
as $\mu\to0$, hence $f$ is differentiable with $f'=g$ and $f'(0)=i(ba-ab)$.

(b)
For $a\in X$, $b\in A$, and $\lambda\in\C$, since $e^{i\lambda b}$ is unitary, we have $f(\lambda)\in X$.
Then limiting $(i\lambda)^{-1}(f(\lambda)-a)$, which is in $X$, we get $ba-ab\in X$ from the part (a).

(c)
We prove $X$ is unitarily invariant.
Let $u\in A$ be a unitary.
For every projection $p$ in $A$, the element $upu^*$ is also a projection so that we have $upu^*\in X$, which implies $uXu^*\subset X$ by the closedness of $X$.
Then, by the part (b), the closed subspace $X$ has the property we want.
\end{sol}


\begin{prb}
Let $a$ be a normal element of a C*-algebra $A$, and $b$ an element commuting with $a$. Show that $b^*$ also commutes with $a$ (Fuglede's theorem).
(Hint: Define $f(\lambda)=e^{i\lambda a^*}be^{- i\lambda a^*}$ in $\tilde A$ and deduce from Exercise 2.7 that this map is differentiable and $f'(0)=i(a^*b-ba^*)$.
Since $e^{i\cl\lambda a}$ and $b$ commute, $f(\lambda)=e^{2ic(\lambda)}be^{-2ic(\lambda)}$, where $c(\lambda)=\Re(\lambda a^*)$.
Hence, $\|f(\lambda)\|=\|b\|$, so by Liouville's theorem, $f(\lambda)$ is constant.)
\end{prb}
\begin{sol}
The hint is in detail enough.
\end{sol}


\bigskip
In the following exercises $H$ is a Hilbert space:

\begin{prb}
If $I$ is an ideal of $B(H)$, show that it is self-adjoint.
\end{prb}
\begin{sol}
Let $a\in I$.
The polar decomposition gives a partial isometry $u$ such that $a=u|a|$ and $u^*a=|a|$.
Taking adjoint on $a=u|a|$, we get $a^*=|a|u^*=u^*au^*\in I$.
\end{sol}


\begin{prb}
Let $u\in B(H)$.
\begin{parts}
\item
Show that $u$ is a left topological zero divisor in $B(H)$ if and only if it is not bounded below (\emph{cf.} Exercise 1.11).
\item
Define
\[\sigma_{ap}(u)=\{\,\lambda\in\C:u-\lambda\text{ is not bounded below}\,\}.\]
This set is called the \emph{approximate point spectrum} of $u$ because $\lambda\in\sigma_{ap}(u)$ if and only if there is a sequence $(x_n)$ of unit vectors of $H$ such that $\lim_{n\to\infty}\|(u-\lambda)(x_n)\|=0$.
Show that $\sigma_{ap}(u)$ is a closed subset of $\sigma(u)$ containing $\pd\sigma(u)$.
\item
Show that $u$ is bounded below if and only if it is left-invertible in $B(H)$.
\item
Show that $\sigma(u)=\sigma_{ap}(u)$ if $u$ is normal.
\end{parts}
\end{prb}
\begin{sol}
(a)
($\Rightarrow$)
By definition of the left topological zero divisors, there is a sequence $(u_n)_n$ in $B(H)$ such that $\|u_n\|=1$ and $uu_n\to0$ as $n\to\infty$.
Since $\|u_n\|=1$, we can construct a sequence $(x_n)_n$ in $H$ such that $\|x_n\|=1$ but $\|u_n(x_n)\|>\frac12$.
Then, since $\|u\frac{u_n(x_n)}{\|u_n(x_n)\|}\|<2\|uu_n\|\to0$ as $n\to\infty$, $u$ is not bounded below.

($\Leftarrow$)
Suppose $u$ is not bounded below so that there is a sequence $(x_n)_n$ of unit vectors in $H$ such that $u(x_n)\to0$ as $n\to\infty$.
Then, the rank-one projections $x_n\otimes x_n$ have norm one and satisfy $\|u(x_n\otimes x_n)\|=\|(ux_n)\otimes x_n\|\le\|ux_n\|\to0$ as $n\to\infty$, hence $u$ is a left topological zero divisor.

(b)
If $u-\lambda$ is bounded below with lower bound $M$, then $u-\lambda-\mu$ is also bounded below for $|\mu|<\frac M2$ because $\|u(x)-\lambda x-\mu x\|\ge\|u(x)-\lambda x\|-|\mu|\|x\|>\frac M2\|x\|$, so the set $\sigma_{ap}(u)$ is closed.

Let $\lambda\in\pd\sigma(u)$ so that there is a sequence $(\lambda_n)_n$ of $\C$ such that $u-\lambda_n$ is invertible.
By the part (c) of Exercise 1.11, $\|(u-\lambda_n)^{-1}\|^{-1}\to0$ as $n\to\infty$.
Then, we can find a sequence $(x_n)_n$ of unit vectors in $H$ such that $\|(u-\lambda_n)^{-1}x_n\|^{-1}\to0$ when $n\to\infty$.
\begin{align*}
\|(u-\lambda)\frac{(u-\lambda_n)^{-1}x_n}{\|(u-\lambda_n)^{-1}x_n\|}\|
&=\|(1+(\lambda-\lambda_n)(u-\lambda_n)^{-1})x_n\|\|(u-\lambda_n)^{-1}x_n\|^{-1}\\
&\le\|(u-\lambda_n)^{-1}x_n\|^{-1}+|\lambda-\lambda_n|\to0
\end{align*}
where $n$ tends to $\infty$.
Therefore, $u-\lambda$ is not bounded below for every $\lambda\in\pd\sigma(u)$.

(c)
($\Rightarrow$)
Note that a bounded below operator is injective and has closed range.
Let $p$ be the projection to the range of $u$.
For arbitrary $x\in H$, there is a unique $y\in H$ such that $u(y)=p(x)$ since $u$ is injective.
Define $v\in B(H)$ such that $v(x):=y$.
The boundedness of $v$ is immediately shown from the assumption that $u$ is bounded below.
Then, $uvu(x)=pu(x)=u(x)$ implies $vu(x)=x$ for all $x\in H$; $u$ has a left inverse $v$.

($\Leftarrow$)
Let $v$ be a left inverse of $u$.
Then, $u$ is bounded below since $\|x\|=\|vu(x)\|\le\|v\|\|u(x)\|$.

(d)
It suffices to show that a bounded below normal operator is always invertible.
For normal $u\in B(H)$, the equality $\|u(x)\|=\|u^*(x)\|$ holds for all $x\in H$, so $u^*$ is also bounded below if $u$ is bounded below.
Since the adjoint of bounded below operator is surjective, $u=(u^*)^*$ is surjective.
By the open mapping theorem, $u$ is invertible.
\end{sol}


\begin{prb}
Let $u\in B(H)$ be a normal operator with spectral resolution of the identity $E$.
\begin{parts}
\item
Show that $u$ admits an invariant closed vector subspace other than 0 and $H$ if $\dim(H)>1$.
\item
If $\lambda$ is an isolated point of $\sigma(u)$, show that $E(\lambda)=\ker(u-\lambda)$ and that $\lambda$ is an eigenvalue of $u$.
\end{parts}
\end{prb}
\begin{sol}
(a)
More strongly, we will show every normal operator $u\in B(H)$ admits a \emph{reducing} subspace that is non-trivial and proper if $\dim(H)>1$.
If $u=\lambda\id_H$ for some scalar $\lambda\in\C$, then every closed subspace of $H$ is invariant, we are done.
Otherwise, the spectrum $\sigma(u)$ has at least two distinct points, say $x$ and $y$, because $\sigma(u)=\{\lambda\}$ implies $\sigma(u-\lambda)=\{0\}$ and $\|u-\lambda\|=r(u-\lambda)=0$.
Since $\sigma(u)$ is compact Hausdorff and hence is normal, there exist disjoint open subsets of $\sigma(u)$ separating $x$ and $y$, and positive functions $f$ and $g$ in $C(\sigma(u))$ such that $f(x)=1$, $g(y)=1$, and $fg=0$ by the Urysohn lemma.
Construct $p=\chi_{g^{-1}(0)}(u)\in B(H)$ by the Borel functional calculus.
Then, $p$ is a projection such that $0\ne f(u)\le p$ and $0\ne g(u)\le 1-p$, so the image of $p$ is non-trivial and proper, and it is reducing for $u$ since $pu=up$.



(b)
The characteristic function $\chi_{\{\lambda\}}$ is a non-zero continuous function on $\sigma(u)$.
By the continuous functional calculus, a non-zero element $\chi_{\{\lambda\}}(u)$ exists in $B(H)$.
Let $f(z)=z-\lambda$ and $x\in H$ a vector in the image of $\chi_{\{\lambda\}}(u)$.
Then $f(z)\chi_{\{\lambda\}}(z)=0$ implies $(u-\lambda)\chi_{\{\lambda\}}(u)=0$ and $(u-\lambda)x=0$.
\end{sol}
\begin{rmk}
The part (a) can be seen as a corollary of the fact that an irreducible representation of an abelian C*-algebra is necessarily of dimension one.
Conversely, we can prove an abelian C*-algebra $A$ only has one-dimensional irreducible representations by the construction of a non-zero and proper projection not in $C(\sigma(u))$ but $C_0(\Omega(A))$, using a version of the Urysohn lemma for locally compact Hausdorff spaces and the continuous functional calculus.
\end{rmk}
\begin{rmk}
I could not understand what the notation $E(\lambda)$ in the part (b) of this exercise means.
The notation $E(\lambda)$ is seemed to be commonly used to refer to the projection $E((-\infty,\lambda])$, where $E$ is the resolution of the identity of a self-adjoint operator so that $E$ is defined on a subset of $\R$.
\end{rmk}


\begin{prb}
An operator $u$ on $H$ is \emph{subnormal} if there is a Hilbert space $K$ containing $H$ as a closed vector subspace and there exists a normal operator $v$ on $K$ such that $H$ is invariant for $v$, and $u$ is the restriction of $v$.
We call $v$ a \emph{normal extension} of $u$.
\begin{parts}
\item
Show that the unilateral shift is a non-normal subnormal operator.
\item
Show that if $u$ is subnormal, then $u^*u\ge uu^*$.
\item
A normal extension $v\in B(K)$ of a subnormal operator $u\in B(H)$ is a minimal normal extension if the only closed vector subspace of $K$ reducing $v$ and containing $H$ is $K$ itself.
Show that $u$ admits a minimal normal extension.
In the case that $v$ is a minimal normal extension, show that $K$ is the closed linear span of all $v^{*n}(x)$ ($n\in\N,\ x\in H$).
\item
Show that if $v\in B(K)$ and $v'\in B(K')$ are minimal normal extensions of $u$, then there exists a unitary operator $w:K\to K'$ such that $v'=wvw^*$ (so there is only one minimal normal extension).
\end{parts}
\end{prb}
\begin{sol}
(a)
Let $u\in B(\ell^2(\N))$ be the unilateral shift and $v\in B(\ell^2(\Z))$ the bilateral shift.
Then, $u$ is the restriction of $v$ onto $\ell^2(\N)$, and $u$ is not normal since $uu^*\ne1=u^*u$ but $v$ is normal.

(b)
Let $v\in B(K)$ be a normal extension of $u\in B(H)$.
Fix $x\in H$.
Define $w:=v^*-u^*$ so that we have $u^*(x)=v^*(x)-w(x)$ and $w(H)\subset H^\perp$.
Then,
\begin{gather*}
u^*u(x)=v^*u(x)-wu(x)=v^*v(x)-wu(x),\\
uu^*(x)=vu^*(x)=vv^*(x)-vw(x),
\end{gather*}
imply $u^*u-uu^*=vw-wu$ by the normality of $v$.
Since $\<wu(x),x\>=0$ and $\<w(x),u^*(x)\>=0$, we have $\<(u^*u-uu^*)(x),x\>=\<vw(x),x\>=\<w(x),v^*(x)\>=\|w(x)\|^2\ge0$.

(c)
Write $(v,K)$ to denotes a normal extension $v\in B(K)$ of $u$.
Let $(P,\prec)$ be the preordered set of all normal extensions $u$ with the preorder defined as $(v,K)\prec(v',K')$ if and only if $K\subset K'$, $v=v'|_K$, and $v'$ reduces $K$.
If $(v_\lambda,K_\lambda)_\lambda$ is a chain in $P$, then restriction of any $v_\lambda$ onto $\bigcap_\lambda K_\lambda$ gives lower bound in $P$, there is a minimal element $(v,K)$ of $P$ by the Zorn lemma.
It is actually a \emph{minimal} normal extension in the sense of definition in this problem since if $K$ has a reducing subspace $K'\supset H$ for $v$, then $(v|_{K'},K')\prec(v,K)$ is also a normal extension of $u$, which contradicts to the minimality of $(v,K)$ in $P$.

Let $(v,K)$ be a minimal normal extension and $V\subset K$ the linear span of all $v^{*n}(x)$ for $n\in\N$ and $x\in H$.
Here, we use convention $0\in\N$ so that we have $H\subset\cl V$.
The closed subspace $\cl V$ of $K$ is invariant for $v^*$ clearly by definition of $V$ and also for $v$ since $v(v^{*n}(x))=v^{*n}(u(x))\in \cl V$, so $\cl V$ is reducing for $v$.
Therefore $\cl V=K$ for the definition of minimal normal extension.

(d)
We have shown in the part (c) that an operator from $K$ is fully determined by the value on $V$, the linear span of $v^{*n}(x)$.
Note that the norm of elements in $V$ does not depend on the operator $v$ but only on the vectors $x\in H$:
If $(x_n)_n$ be a sequence in $H$ with $x_n=0$ for all $n$ except finitely many, then
\[\|\sum_n v^{*n}x_n\|^2
=\sum_{n,m}\<v^{*n}x_n,v^{*m}x_m\>
=\sum_{n,m}\<v^mx_n,v^nx_m\>
=\sum_{n,m}\<x_n,x_m\>.\]
Let $w:K\to K'$ be satisfy $w(v^{*n}(x)):=v'^{*n}(x)$, which is clearly unitary if it is well-defined.
For any possibilities satisfying $0=\sum_nv^{*n}x_n$, which is equivalent to $\sum_{n,m}\<x_n,x_m\>=0$, the norm of $w(0)$ also must be zero.
Therefore, $w$ sends zero to zero and is well-defined on $V$, hence on $\cl V=K$.
With this $w$, we can show $v'=wvw^*$ from $wvw^*(v'^{*n}(x))=wv(v^{*n}(x))=wv^{*n}(u(x))=v'^{*n}(u(x))=v'(v'^{*n}(x))$.
\end{sol}


\section{Ideals and positive functionals}

In Exercises 1 to 7, $A$ denotes an arbitrary C*-algebra.

\begin{prb}
Let $a,b$ be normal elements of a C*-algebra $A$, and $c$ an element of $A$ such that $ac=cb$.
Show that $a^*c=cb^*$, using Fuglede's theorem (Exercise 2.8) and the fact that the element
\[d=\mat{a&0\\0&b}\]
is normal in $M_2(A)$ and commutes with 
\[d'=\mat{0&c\\0&0}.\]
This more general result is called the Putnam-Fuglede theorem.
\end{prb}
\begin{sol}
Since
\[dd'=\mat{0&ac\\0&0}=\mat{0&cb\\0&0}=d'd,\]
we have
\[\mat{0&a^*c\\0&0}=d^*d'=d'd^*=\mat{0&cb^*\\0&0}\]
by the Fuglede theorem.
\end{sol}


\begin{prb}
Let $\tau$ be a positive linear functional on $A$.
\begin{parts}
\item
If $I$ is a closed ideal in $A$, show that $I\subset\ker(\tau)$ if and only if $I\subset\ker(\f_\tau)$.
\item
We say $\tau$ is \emph{faithful} if $\tau(a)=0\Rightarrow a=0$ for all $a\in A^+$.
Show that if $\tau$ is faithful, then the GNS representation $(H_\tau,\f_\tau)$ is faithful.
\item
Suppose that $\alpha$ is an automorphism of $A$ such that $\tau(\alpha(a))=\tau(a)$ for all $a\in A$.
Define a unitary on $H_\tau$ by setting $u(a+N_\tau)=\alpha(a)+N_\tau$ ($a\in A$).
Show that $\f_\tau(\alpha(a))=u\f(a)u^*$ ($a\in A$).
\end{parts}
\end{prb}
\begin{sol}
(a)
($\Rightarrow$)
For $i\in I$, since $b^*i^*ib\in I$, we have
\[\|\f_\tau(i)(b+N_\tau)\|^2=\|ib+N_\tau\|^2=\tau(b^*i^*ib)=0\]
for all $b+N_\tau\in A/N_\tau$.
Becasue $A/N_\tau$ is dense in $H_\tau$, we get $\f_\tau(i)=0$ and $I\subset\ker(\f_\tau)$.

($\Leftarrow$)
For $i\in I$, we have $\tau(ib)=0$ for all $b\in A$ since
\[|\tau(ib)|^2\le\|\tau\|\tau(b^*i^*ib)=\|\tau\|\|ib+N_\tau\|^2=\|\tau\|\|\f_\tau(i)(b+N_\tau)\|^2=0.\]
Let $(u_\lambda)_\lambda$ be an approximate unit for $I$.
Then, $\tau(i)=\tau(i-iu_\lambda)\le\|\tau\|\|i-iu_\lambda\|\to0$ implies $i\in\ker(\tau)$.

(b)
The representation $(H_\tau,\f_\tau)$ is faithful if and only if $\ker(\f_\tau)=0$.
If $\tau$ is faithful, then since every element of $A$ is a linear combination of four positive elements, we have $a=0$ for any $a\in A$ satisfying $\tau(a)=0$.
Thus, $\tau$ is faithful if and only if $\ker(\tau)=0$.
By the part (a), we have $\ker(\tau)=0$ if and only if $\ker(\f_\tau)=0$, which is the desired result.

(c)
The operator $u$ on $H_\tau$ is unitary because
\[\|u(a+N_\tau)\|^2=\tau(\alpha(a)^*\alpha(a))=\tau(\alpha(a^*a))=\tau(a^*a)=\|a+N_\tau\|^2.\]
For $a\in A$ and $b+N_\tau\in A/N_\tau$, we have
\[\f_\tau(\alpha(a))(b+N_\tau)=\alpha(a)b+N_\tau=\alpha(a\alpha^{-1}(b))+N_\tau=u\f_\tau(a)u^{-1}(b+N_\tau).\qedhere\]
\end{sol}


\begin{prb}
If $\f:A\to B$ is a positive linear map between C*-algebras, show that $\f$ is necessarily bounded.
\end{prb}
\begin{sol}
Let $\cl{B_A}$ denote the closed unit ball of $A$.
Note that Theorem 3.3.1 states that a positive linear functional on a C*-algeba is bounded.
For any positive linear functional $\tau$ on $B$, the composition $\tau\circ\f$ is a positive linear functional on $A$, so $\tau(\f(\cl{B_A}))\subset\C$ is bounded.
Since every bounded linear functional is the sum of four positive linear functionals, the set $\f(\cl{B_A})\subset B$ is weakly bounded, and hence bounded.
\end{sol}


\begin{prb}
Suppose that $A$ is unital.
Let $\alpha$ be an automorphism of $A$ such that $\alpha^2=\id_A$.
Define $B$ to be the set of all matrices
\[c=\mat{a&b\\\alpha(b)&\alpha(a)},\]
where $a,b\in A$.
Show that $B$ is a C*-subalgebra of $M_2(A)$.
Define a map $\f:A\to B$ by setting
\[\f(a)=\mat{a&0\\0&\alpha(a)}.\]
Show that $\f$ is an injective *-homomorphism.
We can thus identify $A$ as a C*-subalgebra of $B$.
If we set $u=\bigl(\mat[small]{0&1\\1&0}\bigr)$, then $u$ is a self-adjoint unitary and $B=A+Au$.
If $C$ is any unital C*-algebra with a self-adjoint unitary element $v$, and $\psi:A\to C$ is a *-homomorphism such that
\[\psi(\alpha(a))=v\psi(a)v^*\quad(a\in A),\]
show that there is a unique *-homomorphism $\psi':B\to C$ extending $\psi$ (that is, $\psi'\circ\f=\psi$) such that $\psi'(u)=v$.

(This establishes that $B$ is a (very easy) example of a crossed product, namely $B=A\times_\alpha Z_2$, the crossed product of $A$ by the two-element group $Z_2$ under the action $\alpha$.
The theory of crossed products is a vast area of the modern theory of C*-algebras.
One of its primary uses is to generate new examples of simple C*-algebras.
For an account of this theory, see [Ped].)
\end{prb}
\begin{sol}
For the map $\f$, it is trivial to check that it is an injective $^*$-homomorphism.

We first check $B$ is a C*-subalgebra of $M_2(A)$.
The set $B$ is clearly closed under summation and scalar multiplication.
For product and adjoint, we have
\[\mat{a&b\\\alpha(b)&\alpha(a)}\mat{a'&b'\\\alpha(b')&\alpha(a')}
=\mat{aa'+b\alpha(b')&ab'+b\alpha(a')\\\alpha(b\alpha(a')+ab')&\alpha(b\alpha(b')+aa')}\in B\]
and
\[\mat{a&b\\\alpha(b)&\alpha(a)}^*=\mat{a^*&\alpha(b)^*\\\alpha(\alpha(b)^*)&\alpha(a^*)}\in B.\]
For closedness with respect to the norm of $M_2(\C)$, by the inequality
\[\max\{\|a\|,\|b\|,\|c\|,\|d\|\}\le\left\lVert\mat{a&b\\c&d}\right\rVert\le\|a\|+\|b\|+\|c\|+\|d\|,\]
if a sequence $\mat{a_n&b_n\\\alpha(b_n)&\alpha(a_n)}$ in $B$ is Cauchy so that $a_n$ and $b_n$ converge to some element $a$ and $b$ respectively, then it converges to $\mat{a&b\\\alpha(b)&\alpha(a)}$ which is in $B$.

Next, let us show the unique existence of $\psi':B\to C$.
For the uniqueness, suppose $\psi'$ is a $^*$-homomorphism satisfying all requirements given in the problem.
Since $\psi'\circ\f=\psi$ and $\psi'(u)=v$, we have
\[\psi'\mat{a&b\\\alpha(b)&\alpha(a)}=\psi'(\f(a)+\f(b)u)=\psi(a)+\psi(b)v,\]
then it suffices to show the map $\psi'$ defined by $\psi'(\f(a)+\f(b)u):=\psi(a)+\psi(b)v$ is $^*$-homomorphism
The only non-trivial point is the preservation of multiplication, which is shown as
\begin{align*}
\psi'\Biggr(&\mat{a&b\\\alpha(b)&\alpha(a)}\mat{a'&b'\\\alpha(b')&\alpha(a')}\Biggr)
=\psi'\mat{aa'+b\alpha(b')&ab'+b\alpha(a')\\\alpha(ab'+b\alpha(a'))&\alpha(aa'+b\alpha(b'))}\\
&=\psi(aa'+b\alpha(b'))+\psi(ab'+b\alpha(a'))v\\
&=\psi(a)\psi(a')+\psi(b)\psi(\alpha(b'))+\psi(a)\psi(b')v+\psi(b)\psi(\alpha(a'))v\\
&=\psi(a)\psi(a')+\psi(b)v\psi(b')v+\psi(a)\psi(b')v+\psi(b)v\psi(a')\\
&=(\psi(a)+\psi(b)v)(\psi(a')+\psi(b')v)\\
&=\psi'\mat{a&b\\\alpha(b)&\alpha(a)}\psi'\mat{a'&b'\\\alpha(b')&\alpha(a')}.\qedhere
\end{align*}
\end{sol}


\begin{prb}
An element $a$ of $A^+$ is \emph{strictly positive} if the hereditary C*-subalgebra of $A$ generated by $a$ is $A$ itself, that is, if $(aAa)^-=A$.
\begin{parts}
\item
Show that if $A$ is unital, then $a\in A^+$ is strictly positive if and only if $a$ is invertible.
\item
If $H$ is a Hilbert space, show that a positive compact operator on $H$ is strictly positive in $K(H)$ if and only if it has dense range.
\item
Show that if $a$ is strictly positive in $A$, then $\tau(a)>0$ for all non-zero positive linear functionals $\tau$ on $A$.
\end{parts}
\end{prb}
\begin{sol}
(a)
($\Rightarrow$)
Suppose $a\in A^+$ is not invertible.
If an element $aba$ of $aAa$ were invertible, then $a$ is both left and right invertible because $((aba)^{-1}ab)a=1=a(ba(aba)^{-1})$, thus there is no invertibles in $aAa$.
Since the set of non-invertible elements is closed, our hereditary subalgebra $(aAa)^-$ also only contains non-invertibles, but $A$ is unital so that $(aAa)^-$ is proper in $A$.

($\Leftarrow$)
If $a\in A^+$ is invertible, then for any $b\in A$ we have $b=a(a^{-1}ba^{-1})a\in aAa$.

(b)
($\Rightarrow$)
If the range of $a\in K(H)^+$ is not dense, then every element of $(aK(H)a)^-$ has range contained in $\im(a)^-$.
If we choose $x\notin\im(a)^-$, then $x\otimes x\in K(H)$ but not in $(aK(H)a)^-$.

($\Leftarrow$)
Suppose $a\in K(H)^+$ has dense range.
We have $a=(a^3)^{1/3}\in(aK(H)a)^-$ by the functional calculus, there is a sequence $(b_n)_n$ in $A$ such that $\|ab_na-a\|\to0$ as $n\to\infty$.
Because $ab_n^*a$ and $a((b_n+b_n^*)/2)a$ also converges to $a$, we may assume for $b_n$ to be self-adjoint.
For every $x\in\im(a)$, we have $\|ab_nx-x\|\to0$ for $n\to\infty$, and the left-hand side of
\begin{align*}
\|ab_n(x\otimes x)b_na-x\otimes x\|
&\le\|(ab_nx-x)\otimes(ab_nx)\|+\|x\otimes(ab_nx-x)\|\\
&=(\|ab_nx\|+\|x\|)\|ab_nx-x\|
\end{align*}
also converges to zero.
Therefore, $x\otimes x\in(aK(H)a)^-$.
Since $\im(a)$ is dense in $H$, and since $\|x\otimes x-y\otimes y\|\le(\|x\|+\|y\|)\|x-y\|$, every rank-one projection belongs to $(aK(H)a)^-$, and hence $K(H)=(aK(H)a)^-$.

(c)
Suppose $a\in A^+$ is strictly positive and a positive linear functional $\tau$ satisfies $\tau(a)>0$.
Then, every $b\in A^+$ satisfies
\begin{align*}
|\tau(aba)|^2
&=|\tau(a^{\frac12*}(b^\frac12a^\frac12)^*(b^\frac12a^\frac12)a^\frac12)|^2\\
&\le\tau(a^{\frac12*}a^\frac12)\tau((b^\frac12a^\frac12)^*(b^\frac12a^\frac12))
=\tau(a)\tau(a^\frac12ba^\frac12)=0,
\end{align*}
so $\tau(aba)=0$ for all $b\in A$ because $b$ can be decomposed four positive elements of $A$.
The continuity of $\tau$ implies $\tau=0$, therefore for non-zero positive linear functionals the value at $a$ is positive.
\end{sol}
\begin{rmk}
The converse result of the part (c) holds.
Let $S$ be the set of all positive linear functionals $\tau$ on $A$ with $\|\tau\|\le1$.
Then, $S$ is weak* compact by the Banach-Alaoglu theorem.
(Unlike unital C*-algebras, neither the space of states nor pure states is not weak* compact for general non-unital algebras.)
Then, every element of $A$ gives rise to a positive weak* continuous function on a compact Hausdorff space $S$.
Since $\tau(a)>0$ for $\tau\in S$, there is a lower bound $M>0$ of $a$ in the sense that $\tau(a)\ge M$.
If we take arbitrary non-zero $b\in A^+$, then since $\tau(M\|b\|^{-1}b)\le M\le\tau(a)$ for $\tau\in S$, we have $M\|b\|^{-1}b\le a$ by Theorem 3.4.3 so that $b$ belongs to the hereditary C*-subalgebra $(aAa)^-$.
It implies that $A=(aAa)^-$ because element of $A$ is decomposed into four elements of $A^+$.
See Theorem 5.3.1 and Remark 5.3.1 in the textbook.
In fact, Theorem 5.3.1 does not require Theorem 3.2.6.
\end{rmk}% damn?


\begin{prb}
We say that $A$ is \emph{$\sigma$-unital} if it admits a sequence $(u_n)_{n=1}^\infty$ which is an approximate unit for $A$.
It follows from Remark 3.1.1 that every separable C* -algebra is $\sigma$-unital.
\begin{parts}
\item
Let $a$ be a strictly positive element of $A$, and set $u_n=a(a+1/n)^{-1}$ for each positive integer $n$.
Show that $(u_n)$ is an approximate unit for $A$.
(Hint: Define $g_n:\sigma(a)\to\R$ by $g_n(t)=t^2/(t+1/n)$.
Show that the sequence $(g_n)$ is pointwise-increasing and pointwise-convergent to the inclusion $z:\sigma(a)\to\R$, and use Dini's theorem to deduce that $(g_n)$ converges uniformly to $z$.
Hence, $a=\lim_{n\to\infty}au_n$.)
\item
If $(u_n)_{n=1}^\infty$ is an approximate unit for $A$, show that $a=\sum_{n=1}^\infty u_n/2^n$ is a strictly positive element of $A$.
Thus, $A$ is $\sigma$-unital if and only if it admits a strictly positive element.
\end{parts}
\end{prb}
\begin{sol}
Note that $u_n$ belongs to non-unital $A$ because it is defined by the functional calculus for $f_n:\sigma(a)\to\R$ with $f_n(t)=t(t+\frac1n)^{-1}$, not by the product of $a$ and $(a+\frac1n)^{-1}$.
Since $\|f_n\|_\infty<1$, we have $\|u_n\|<1$ for each $n$.

(a)
Let $b\in A$ and take $c\in aAa$ such that $\|b-c\|<\frac12\e$ for arbitrary $\e>0$.
Since $\|u_n\|\le1$ and $\lim_{n\to\infty}cu_n=c$, we have $\lim_{n\to\infty}bu_n=b$ from the inequality
\[\|bu_n-b\|\le\|(b-c)u_n\|+\|cu_n-c\|+\|c-b\|\le\e+\|cu_n-c\|.\]
In the same manner, we get $\lim_{n\to\infty}u_nb=b$, so $u_n$ is an approximate unit for $A$.

(b)
Since $(aAa)^-$ is hereditary and $u_n/2^n\le a$, we have $u_n\in(aAa)^-$.
Therefore, for any $b\in A$, we get $b=\lim_{n\to\infty}u_nbu_n\in(aAa)^-$.
\end{sol}


\begin{prb}
Let $\Omega$ be a locally compact Hausdorff space.
(a) Show that $C_0(\Omega)$ admits an approximate unit $(p_n)_{n=1}^\infty$ where all the $p_n$ are projections, if and only if $\Omega$ is the union of a sequence of compact open sets.
(b) Deduce that if a C*-algebra $A$ admits a strictly positive element $a$ such that $\sigma(a)\setminus\{0\}$ is discrete, then $A$ admits an approximate unit $(p_n)_{n=1}^\infty$ consisting of projections.
(Show that $C^*(a)$ is *-isomorphic to $C_0(\sigma(a)\setminus\{0\})$.)
\end{prb}
\begin{sol}
(a)
($\Rightarrow$)
Note that the set $p_n^{-1}(1)$ is open since $p_n^{-1}(1)=p_n^{-1}((1-\e,1+\e))$, and is compact since $p_n^{-1}(1)=\{\,x\in\Omega:p_n(x)\ge\e\,\}$, for some $0<\e<1$.
For any $x\in\Omega$, taking any $f\in C_0(\Omega)$ with $f(x)\ne0$, we can find $n$ such that $|f(x)p_n(x)-f(x)|\le\|fp_n-f\|<\e$ for $\e=\frac12|f(x)|$, and the inequality implies $p_n(x)=1$.
This proves $\Omega=\bigcup_{n=1}^\infty p_n^{-1}(1)$.

($\Leftarrow$)
Suppose $(\Omega_n)_{n=1}^\infty$ is a sequence of compact open subsets of $\Omega$ such that $\Omega=\bigcup_{n=1}^\infty\Omega_n$.
We may assume it to be increasing; $\Omega_n\subset\Omega_{n+1}$ for all $n$.
Define $p_n$ to be the characteristic function $\chi_{\Omega_n}$ of a compact open set, which satisfies $p_n\in C_0(\Omega)$.
Take any $f\in C_0(\Omega)$.
For arbitrary $\e>0$, the set $\{\,x\in\Omega:|f(x)|\ge\e\,\}$ is compact and has an open cover $\{\Omega_n\}_n$, so there is $n_0$ such that $|f(x)|\ge\e$ on $x\in\Omega_n$ for all $n>n_0$.
Then, finally we have $|f(x)p_n(x)-f(x)|=0<\e$ for $x\in\Omega_n$ and $|f(x)p_n(x)-f(x)|=|f(x)|<\e$ for $x\notin\Omega_n$, proving $\|fp_n-f\|<\e$ for $n>n_0$.
Therefore, $p_n$ is an approximate unit for $C_0(\Omega)$.

(b)
If $C^*(a)$ is unital so that $a$ is invertible, then since $a$ is self-adjoint so that $C^*(a)$ is the generated Banach subalgebra by $a$, we have a homeomorphism
\[\hat a:\Omega(C^*(a))\cong\sigma(a)=\sigma(a)\setminus\{0\}\]
by Theorem 1.3.7.
If $C^*(a)$ is not unital so that $a$ is not invertible, then since $C^*(1,a)$ is the unitization of $C^*(a)$, we have
\[\Omega(C^*(a))\cup\{\tau_\infty\}=\Omega(C^*(1,a))\cong\sigma_{C^*(1,a)}(a)=:\sigma(a),\] where $\tau_\infty(a)=0$, so $\tau_\infty\notin\Omega(C^*(a))$ and $\hat a(\tau_\infty)=0$ implies the homeomorphism $\Omega(C^*(a))\cong\sigma(a)\setminus\{0\}$.
Therefore, we have the result in the hint: $C^*(a)$ is *-isomorphic to $C_0(\sigma(a)\setminus\{0\}$) by the Gelfand representation.

We first note the facts that a discrete subset in a second countable space is countable, and compactness is equivalent to finiteness in a discrete space.
Hence we have a sequence $(x_n)_{n=1}^\infty$ that numerates $\sigma(a)\setminus\{0\}=\{x_n\}_n$.
Define projections $q_n:\sigma(a)\setminus\{0\}\to\C$ such that $q_n(x_k)=1$ if $k\le n$ and 0 otherwise, and it follows that $q_n\in C_0(\sigma(a)\setminus\{0\})$ from $q_n^{-1}(1)$ is compact and open.
For arbitrary $f\in C_0(\sigma(a)\setminus\{0\})$ and $\e>0$, there is a finite subset $F\subset\sigma(a)\setminus\{0\}$ such that $|f(x)|<\e$ for $x\notin F$, so $\|fq_n-f\|<\e$ for every $n>|F|$, which proves $(q_n)_n$ is an approximate unit on $C_0(\sigma(a)\setminus\{0\})$.
If we let $p_n:=q_n(a)$ by the functional calculus, then $(p_n)_n$ is an approximate unit for $C^*(a)$ consisting of projections.
\end{sol}


\begin{prb}
Let $z:\T\to\C$ be the inclusion map.
Let $\theta\in[0,1]$.
(a) Show that there is a unique automorphism $\alpha$ of $C(\T)$ such that $\alpha(z)=e^{i2\pi\theta}z$.
Define the faithful positive linear functional $\tau:C(\T)\to\C$ by setting $\tau(f)=\int f\,dm$ where $m$ is normalised arc length on $\T$.
(b) Show that $\tau(\alpha(f))=\tau(f)$ for all $f\in C(\T)$.
(c) Deduce from Exercise 3.2 that there is a unitary $v$ on the Hilbert space $H_\tau$ such that $\f_\tau(\alpha(f))=v\f_\tau(f)v^*$ for all $f\in C(\T)$.
Let $u$ be the unitary $\f_\tau(z)$.
(d) Show that $vu=e^{i2\pi\theta}uv$.
If $\theta$ is irrational, the C*-algebra $A_\theta$ generated by $u$ and $v$ is called an \emph{irrational rotation algebra}, and $A_\theta$ can be shown to be simple.
See [Rie] for more details concerning $A_\theta$.
These algebras form a very important class of examples in C*-algebra theory.
They are motivating examples in Connes' development of "non-commutative differential geometry," a subject of great future promise [Con 2].
\end{prb}
\begin{sol}
\end{sol}


\begin{prb}
Let $m$ be normalised Haar measure on $\T$.
If $\lambda\in\C$, $|\lambda|<1$, define $\tau_\lambda:H^1\to\C$ by setting
\[\tau_\lambda(f)=\int\frac{f(w)}{1-\lambda\cl w}\,dmw\quad(f\in H^1).\]
(a) Show that $\tau_\lambda\in(H^1)^*$.
By expanding $(1-\lambda\cl w)^{-1}$ in a power series, show that $\tau_\lambda(f)=\sum_{n=1}^\infty\hat f(n)\lambda^n$.
(b) Deduce that the function
\[\tilde f:\operatorname{int}\D\to\C,\quad\lambda\mapsto\tau_\lambda(f),\]
is analytic, where $\operatorname{int}\D=\{\lambda\in\C:|\lambda|<1\}$.
(c) If $f,g\in H^2$, show that $fg\in H^1$ and $\tau_\lambda(fg)=\tau_\lambda(f)\tau_\lambda(g)$.
(Hint: There exist sequences $(\f_n)$ and $(\psi_n)$ in $\Gamma^+$ converging to $f$ and $g$, respectively, in the $L^2$-norm.
Show that the sequence $(\f_n\psi_n)$ converges to $fg$ in the $L^1$-norm, and deduce the result by first showing it for functions in $\Gamma^+$. )
\end{prb}
\begin{sol}
\end{sol}


\begin{prb}
If $f:\operatorname{int}\D\to\C$ is an analytic function and $0<r<1$, define $f_r\in C(\T)$ by setting $f_r(\lambda)=f(r\lambda)$.
Set $\|f\|_2=\sup_{0<r<1}\|f_r\|_2$, and let $H^2(\D)$ denote the set of all analytic functi ons $f:\operatorname{int}\D\to\C$ such that $\|f\|_2<\infty$.
(a) If $f\in H^2(\D)$, show that $\|f\|_2=\sqrt{\sum_{n=0}^\infty|\lambda_n|^2}$, where $f(\lambda)=\sum_{n=0}^\infty\lambda_n\lambda^n$ is the Taylor series expansion of $f$.
(b) Show that $H^2(\D)$ is a Hilbert space with inner product $\<f,g\>=\sum_{n=0}^\infty\lambda_n\cl\mu_n$, where $\lambda_n=f^{(n)}(0)/n!$ and $\mu_n=g^{(n)}(0)/n!$ (the operations are pointwise-defined), and show also that the map 
\[H^2\to H^2(\D),\quad f\mapsto\tilde f,\]
is a unitary operator.
(Thus, the elements of $H^2$ can be interpreted as analytic functions on $\operatorname{int}\D$ satisfying a growth condition approaching the boundary.
A similar interpretation can be given for the other $H^p$-spaces.)
\end{prb}
\begin{sol}
\end{sol}


\begin{prb}
Show that if $\f$ is a function in $L^\infty(\T)$ not almost everywhere zero, then either $T_\f$ or $T_\f^*$ is injective (Coburn).
(Hint: If $f\in\ker(T_\f)$ and $g\in\ker(T_\f^*)$, show that $\f f\cl g$ and $\cl\f\cl fg\in H^1$.
Deduce that $\f fg=0$ a.e. and apply Theorem 3.5.4 to show that $f$ or $g=0$ a.e.)
Deduce that $T_\f$ is invertible if and only if it is a Fredholm operator of index zero.
\end{prb}
\begin{sol}
\end{sol}


\section{Von Neumann algebras}

\begin{prb}
Let $H$ be a separable Hilbert space with an orthonormal basis $(e_n)_{n=1}^\infty$.
Prove that the relative weak topology on the closed unit ball $S$ of $B(H)$ is metrisable by showing that the equation
\[d(u,v)=\sum_{n,m=1}^\infty\frac{|\<(u-v)(e_n), e_m\>|}{2^{n+m}}\]
defines a metric on $S$ inducing the weak topology.
\end{prb}
\begin{sol}
\end{sol}


\begin{prb}
Let $H$ be a Hilbert space.
\begin{parts}
\item
Show that a weakly convergent sequence of operators on $H$ is necessarily norm-bounded. \item
Show that if $(u_n)$ and $(v_n)$ are sequences of operators on $H$ converging strongly to the operators $u$ and $v$, respectively, then $(u_nv_n)$ converges strongly to $uv$.
\item
Show that if $(u_n)$ is a sequence of operators on $H$ converging strongly to $u$, and if $v\in K(H)$, then $(u_nv)$ converges in norm to $uv$.
Show that $(vu_n)$ may not converge to $vu$ in norm.
\end{parts}
\end{prb}
\begin{sol}
\end{sol}


\begin{prb}
Let $H$ be a Hilbert space with an orthonormal basis $(e_n)_{n=1}^\infty$.
\begin{parts}
\item
Denote by $\Lambda$ the set of all pairs $(n,U)$ where $n$ is a positive integer, and $U$ is a neighbourhood of 0 in the strong topology of $B(H)$.
For $(n,U)$ and $(n',U')$ in $A$, write $(n,U)\le(n',U')$ if $n\le n'$ and $U'\subset U$.
Show that $A$ is a poset under the relation $\le$, and that it is upwards-directed.
\item
Let $u$ denote the unilateral shift on $(e_n)$, and note that $(u^{n*})$ is strongly convergent to zero.
If $\lambda=(n_\lambda,U_\lambda)\in \Lambda$, then $\lim_{n\to\infty}(n_\lambda u^{n*})=0$ in the strong topology, so for some $n$ we have $n_\lambda u^{n*}\in U_\lambda$.
Set $u_\lambda=n_\lambda u^{n*}$ and $v_\lambda=\frac1{n_\lambda}u^n$.
Show that $\lim_\lambda u_\lambda=0$ in the strong topology and $\lim_\lambda v_\lambda=0$ in the norm topology.
(Since $u_\lambda v_\lambda=1$, this shows that the operation of multiplication
\[B(H)\times B(H)\to B(H),\quad(u,v)\mapsto uv,\]
is not jointly continuous in either the weak or the strong topologies.)
\item
Show that neither the weak nor the strong topologies on $B(H)$ are metrisable, using Exercise 4.2 and the nets $(u_\lambda)$ and $(v_\lambda)$ from part (b) of this exercise.
\end{parts}
\end{prb}
\begin{sol}
\end{sol}


\begin{prb}
Let $A$ be a von Neumann algebra on a Hilbert space $H$, and suppose that $\tau$ is a bounded linear functional on $A$.
We say that $\tau$ is \emph{normal} if, whenever an increasing net $(u_\lambda)_{\lambda\in\Lambda}$ in $A_{sa}$ converges strongly to an operator $u\in A_{sa}$, we have $\lim_\lambda\tau(u_\lambda)=\tau(u)$.
Show that every $\sigma$-weakly continuous functional $\tau\in A^*$ is normal
(use Theorem 4.2.10 and show that if $(u_\lambda)_\lambda$ is a bounded net strongly convergent to $u$, and if $v\in L^1(H)$, then $\lim_\lambda\|u_\lambda v-uv\|=0$.)
\end{prb}
\begin{sol}
Theorem 4.2.10 is about a necessary and sufficient condition for a linear functional $\tau$ on $A$ to be $\sigma$-weakly continuous: there is $v\in L^1(H)$ such that $\tau(u)=\tr(vu)$ for all $u\in A$.
Suppose an increasing net $u_\lambda\in A_{sa}$ strongly converges to $u$.
Then we are done if we prove $\|u_\lambda v-uv\|_1\to0$.

We may assume $u_\lambda\in L^1(H)^+$ and $u_\lambda\downarrow0$ in the strong operator topology by letting $(u-u_\lambda)|v|\mapsto u_\lambda$.
Here the polar decomposition of $v$ is used.
Our goal is then to show $\|u_\lambda\|_1\to0$.
Let $\e>0$.
Let $E$ be an orthonormal basis of $H$ and 
\[E_\lambda:=\{\,e\in E:\|u_\lambda^{1/2}(e)\|\le\e\|u_1^{1/2}(e)\|\,\}\]
for $\lambda\in\Lambda$.
Since
\[\|u_\lambda^{1/2}(x)\|^2=\<u_\lambda^{1/2}(x),u_\lambda^{1/2}(x)\>=\<u_\lambda(x),x\>\le\|u_\lambda(x)\|\|x\|\]
and $u_\lambda$ converges to 0 in strongly, we have $E_\lambda\uparrow E$.
Therefore, limiting
\begin{align*}
\|u_\lambda\|_1
&=\sum_{e\in E}\<u_\lambda(e),e\>\\
&=\sum_{e\in E_\lambda}\|u_\lambda^{1/2}(e)\|^2+\sum_{e\in E\setminus E_\lambda}\|u_\lambda^{1/2}(e)\|^2\\
&\le\e^2\sum_{e\in E_\lambda}\|u_1^{1/2}(e)\|^2+\sum_{e\in E\setminus E_\lambda}\|u_1^{1/2}(e)\|^2\\
&=\e^2\sum_{e\in E_\lambda}\<u_1(e),e\>+\sum_{e\in E\setminus E_\lambda}\<u_1(e),e\>,
\end{align*}
we obtain
\[\lim_\lambda\|u_\lambda\|_1\le\e^2\|u_1\|_1,\]
which solves the problem.
\end{sol}
\begin{rmk}
Theorem 4.2.10, in fact, is superfluous because the $\sigma$-weak continuity implies the existence of $\{v_k\}_{k=1}^n$ in $L^1(H)$ such that
\[\tau(u)\le\sum_{k=1}^n|\tr(uv_k)|\]
for all $u\in A$.
\end{rmk}
\begin{rmk}
If we add the monotonicity condition, then the hint in the problem can be interpreted as a noncommutative version of the monotone convergence theorem of Beppo Levi in measure theory: we may think the strong operator topology and trace-class norm have an analogy with the topology of almost everywhere convergence and $L^1$ norm respectively.
\end{rmk}


\begin{prb}
The existence and characterisation of extreme points is very important in many contexts (for example, we shall be concerned with this in the next chapter in connection with pure states).
See the Appendix for the definition of extreme points.
Let $H$ be a non-zero Hilbert space.
\begin{parts}
\item
Show that the extreme points of the closed unit ball of $H$ are precisely the unit vectors.
\item
Deduce that the isometries and co-isometries of $B(H)$ are extreme points of the closed unit ball of $B(H)$.
(It can be shown that these are all of the extreme points.
This follows from [Tak, Theorem 1.10.2].)
\item
Show that if $H$ is an infinite-dimensional Hilbert space, then the closed unit ball of $B(H)^+$ is not the convex hull of the projections of $B(H)$.
\end{parts}
\end{prb}
\begin{sol}
\end{sol}


\begin{prb}
Let $A$ be a C*-algebra.
\begin{parts}
\item
Show that if $A$ is unital, then its unit is an extreme point of its closed unit ball.
\item
If $p$ is a projection of $A$, show that it is an extreme point of the closed unit ball of $A^+$ (use the unital algebra $pAp$ and part (a)).
The converse of this result is also true, but more difficult.
It follows from [Tak, Lemma 1.10.1].
\end{parts}
\end{prb}
\begin{sol}
\end{sol}


\begin{prb}
Let $A$ be a C*-algebra.
Show that if $p,q$ are equivalent projections in $A$, and $r$ is a projection orthogonal to both (that is, $rp=rq=0$), then the projections $r+p$ and $r+q$ are equivalent.

If $H$ is a separable Hilbert space and $p$ is a projection not of finite rank, set $\operatorname{rank}(p)=\infty$.
If $p$ has finite rank, set $\operatorname{rank}(p)=\dim p(H)$.
Show that $p\sim q$ in $B(H)$ if and only if $\operatorname{rank}(p)=\operatorname{rank}(q)$.

Thus, the equivalence class of a projection in a C*-algebra can be thought of as its ``generalised rank''.

We say a projection $p$ in a C*-algebra $A$ is \emph{finite} if for any projection $q$ such that $q\sim p$ and $q\le p$ we necessarily have $q=p$.
Otherwise, the projection is said to be \emph{infinite}.
Show that if $p,q$ are projections such that $q\le p$ and $p$ is finite, then $q$ is finite.

A projection $p$ in a von Neumann algebra $A$ is \emph{abelian} if the algebra $pAp$ is abelian.
Show that abelian projections are finite.

A von Neumann algebra is said to be \emph{finite} or \emph{infinite} according as its unit is a finite or infinite projection.
If $H$ is a Hilbert space, show that the von Neumann algebra $B(H)$ is finite or infinite according as $H$ is finite- or infinite- dimensional.
\end{prb}
\begin{sol}
\end{sol}


\section{Representations of C*-algebras}


\begin{prb}
Let $\tau$ be a pure state on a C*-algebra $A$, and $y$ a unit vector in $H_\tau$ such that $\tau(a)=\<\f_\tau(a)(y),y\>$ for all $a\in A$.
Show that there is a scalar $\lambda$ of modulus one such that $y=\lambda x_\tau$. 
\end{prb}
\begin{sol}
\end{sol}


\begin{prb}
Let $H$ be a Hilbert space and $x$ a unit vector of $H$.
Show that the functional
\[\omega_x:B(H)\to\C,\quad u\mapsto\<u(x),x\>,\] is a pure state of $B(H)$.
Show that not all pure states of $B(H)$ are of this form if $H$ is separable and infinite-dimensional.
\end{prb}
\begin{sol}
\end{sol}


\begin{prb}
Give an example to show that a quotient C*-algebra of a primitive C*-algebra need not be primitive.
\end{prb}
\begin{sol}
\end{sol}


\begin{prb}
If $I$ is a primitive ideal of a C*-algebra $A$, show that $M_n(I)$ is a primitive ideal of $M_n(A)$.
(Thus, if $A$ is primitive, so is $M_n(A)$.) 
\end{prb}
\begin{sol}
\end{sol}


\begin{prb}
Let $A$ be a C*-algebra.
Show the following conditions are equivalent:
\begin{parts}
\item $A$ is prime.
\item If $aAb=0$, then $a$ or $b=0$ ($a,b\in A$).
\end{parts} 
\end{prb}
\begin{sol}
\end{sol}


\begin{prb}
Let $S$ be a set of C*-subalgebras of a C*-algebra $A$ that is \emph{upwards-directed}, that is, if $B,C\in S$, then there exists $D\in S$ such that $B,C\subset D$.
Show that $(\cup S)^-$ is a C*-subalgebra of $A$.

Suppose that all the algebras in $S$ are prime and that $A=(\cup S)^-$.
Show that $A$ is prime.
\end{prb}
\begin{sol}
\end{sol}


\begin{prb}
If $A$ is a C*-algebra, its \emph{center} $C$ is the set of elements of $A$ commuting with every element of $A$.
Show that $C$ is a C*-subalgebra of $A$.
Show that if $A$ is simple, then $C=0$ if $A$ is non-unital and $C=\C1$ if $A$ is unital.
\end{prb}
\begin{sol}
\end{sol}


\begin{prb}
Let $S$ be an upwards-directed set of closed ideals in a C*-algebra $A$ (\emph{cf.} Exercise 5.6 for the term \emph{upwards-directed}).
Suppose that $A=(\cup S)^-$, and that all of the algebras in $S$ are postliminal.
Show that $A$ is postliminal.
\end{prb}
\begin{sol}
\end{sol}


\begin{prb}
Let $A$ be a C*-algebra.
If $I,J$ are postliminal ideals in $A$ (that is, closed ideals that are postliminal C*-algebras), show that $I+J$ is postliminal also.
Deduce from this and Exercise 5.8 that there is a largest postliminal ideal $I$ in $A$ (which may, of course, be the zero ideal).
Show that $A/I$ has no non-zero postliminal ideals.
\end{prb}
\begin{sol}
\end{sol}

\end{document}


