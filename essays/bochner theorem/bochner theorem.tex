\documentclass[12pt]{article}
\usepackage{../../ikany}
\usepackage[margin=100pt]{geometry}
\usepackage[T1]{fontenc}
\usepackage[bitstream-charter,cal]{mathdesign}
\linespread{1.1}


\title{Three perspectives on Bochner's theorem:\\from Herglotz representation\\to Pontryagin duality}
\author{Ikhan Choi}


\begin{document}
\maketitle
\begin{abstract}
The Bochner theorem states that the image of finite Borel measures on an abelian topological group under the Fourier-Stieltjes transform is the set of continuous positive definite functions.
This thesis will describe, prove, and investigate several examples of the Bochner theorem in its historical contexts within three different fields of mathematics, complex analysis, probability theory, and representation theory.
\end{abstract}
\newpage
\tableofcontents

\subsection*{Acknowledgement}





% 투두리스트

% 1장
%  그냥 다 고쳐야

% 2장
%  헬리 정리 사용할 때 포트망토를 풀어 써줄 수 없나
%  디랙이랑 연속제한 예시 추가

% 3장
%. 
%


\newpage
\section{Introduction}


\begin{defn}
Let $G$ be a group.
A function $f:G\to\C$ is called \emph{positive definite} if for each positive integer $n$ a non-negativity condition
\[\sum_{k,l=1}^nf(x_l^{-1}x_k)\xi_k\bar\xi_l\ge0\]
is satisfied for every $n$-tuple $(x_1,\cdots,x_n)\in G^n$ and every vector $(\xi_1,\cdots,\xi_n)\in\C^n$.
\end{defn}
A function $f$ is positive definite if and only if bilinear forms defined by matrices $(f(x_l^{-1}x_k))_{k,l=1}^n$ for each positive integer $n$ are hermitian, and even more, positive \emph{semi}-definite, regardless of any choices of $(x_1,\cdots,x_n)\in G^n$.

We give some several remarkable properties and examples of positive definite functions as follows:

\begin{prop}[Algebraic properties]
Let $G$ be a group.
For the set of all positive definite functions, the following properties hold:
\begin{parts}
\item It is closed under complex conjugation. Furthermore, $\bar{f(x)}=f(x^{-1})$.
\item It is closed under positive scalar multiplication.
\item It is closed under pointwise summation.
\item It is closed under pointwise product.
\end{parts}
\end{prop}
\begin{pf}
(a)
Let $n=1$ and $\xi_1\ne0$.
Then, $0\le f(e)|\xi_1|^2$ implies $f(e)\in\R$.

Let $n=2$ with $x_1=e$ and $x_2=x$, and let $\xi_2=1$.
Then,
\begin{align*}
0&\le f(e)|\xi_1|^2+f(x^{-1})\xi_1\bar\xi_2+f(x)\xi_2\bar\xi_1+f(e)|\xi_2|^2\\
&=f(e)(1+|\xi_1|^2)+f(x^{-1})\xi_1+f(x)\bar\xi_1,
\end{align*}
so
\begin{align*}
0&=\Im(f(x^{-1})\xi_1+f(x)\bar\xi_1)\\
&=(\Re f(x^{-1})-\Re f(x))\Im\xi_1+(\Im f(x^{-1})+\Im f(x))\Re\xi_1
\end{align*}
for all $\xi_1\in\C$.
Therefore, $\bar f(x)=f(x^{-1})$.

(b) and (c) are clearly true by definition.

(d) It follows from the Schur product theorem, which states that the Hadamard product(componentwise product) of two positive semi-definite matrices is also positive semi-definite.
\end{pf}

\begin{prop}[Analytic properties]
Let $G$ be a group with identity $e$.
\begin{parts}
\item If $f$ is positive definite, then $\sup_{x\in G}|f(x)|\le f(e)$.
\item If $f_n$ is a sequence of positive definite functions, then the pointwise limit $\lim_{n\to\infty}f_n$ is also positive definite.
\item Let $G$ be a locally compact group. If $f_n$ is a sequence of positive definite functions that converges to $f$ pointwisely and $f_n(e)=1$, then $f_n$ converges to $f$ compactly.
\item Let $G$ be a locally compact group. If $f$ is positive definite and continuous at the $e$, then it is both-sided uniformly continuous. (It holds for $G=\R$, but I have not checked for arbitrary $G$. I suspect it holds.)
\end{parts}
\end{prop}
\begin{pf}
(a)
Let $n=2$ with $x_1=e$ and $x_2=x$, and let $|\xi_1|=|\xi_2|=1$.
Then,
\[0\le f(e)|\xi_1|^2+f(x^{-1})\xi_1\bar\xi_2+f(x)\xi_2\bar\xi_1+f(e)|\xi_2|^2=2f(e)+2f(x)\xi_2\bar\xi_1.\]
Taking $\xi_1$ and $\xi_2$ such that $\xi_2\bar\xi_1$ has the same argument with $\bar f(x)$, we obtain $|f(x)|\le f(e)$.

(b)
The defining property of positive definite functions is purely algebriac, so that it is preserved by pointwise limit.

(c) and (d) are too difficult to prove at this point, we will be proved later.
\end{pf}

\begin{ex}

\end{ex}

\bigskip

This thesis follows the historical flows to extract mathematical ideas behind the positive definite functions.
In particular, we are concerned with the results like the following \emph{Bochner-type theorems}:
\begin{thm}
A function $c:\Z\to\C$ is positive definite if and only if there is a unique finite regular Borel measure $\mu$ on $\T=\R/2\pi\Z$ such that
\[c(k)=\int_0^{2\pi}e^{-ik\theta}\,d\mu(\theta)\]
for all $k\in\Z$.
\end{thm}
\begin{thm}
A continuous function $\f:\R\to\C$ is positive definite if and only if there is a unique finite regular Borel measure $\mu$ on $\R$ such that
\[\f(t)=\int e^{itx}\,d\mu(x)\]
for all $t\in\R$.
\end{thm}
They have similar forms in that they describe the necessary and sufficient conditions for a function to have a Fourier-Stieltjes integral representation of a finite regular Borel measure.
One of our primary goals is to investigate the nature of positive definite functions and their harmonic-analytic relation to Borel measures within more familiar cases of $G=\Z$ or $\R$.
Now then, we finally extend the Bochner-type results in the more general setting, where $G$ is a locally compact group, and assign a new perspective of measures in terms of the representation theory of groups.

Each theorem above has its own taste in different subfields of mathematics.
Theorem 1.1, which is a corollary of the celebrated Herglotz-Riesz representation theorem, is related to a classical problem in complex analysis that asks to give a characterization of a special class of analytic functions on the open unit disk $\D$ called the Carath\'eodory class.
The positive definiteness arises as a property of coefficients of functions in the Caracth\'eodory class, and their connection to Fourier coefficients leads the complex analysis problem into harmonic analysis.
In Section 2, with the methods of elementary complex variable function theory, our first Bochner-type theorem will be proved, giving a geometric description of the space of positive definite functions in addition.

In Section 3, we review the well-known results of the positive definite functions on the real line and their ``weak convergence''.
They have been studied by probabilists, to attack the weak convergence of probaility measures.
Recall that a probaility distribution of a real-valued random variable is defined by a probability measure on $\R$.
The extended Fourier transform, but reversing the sign convention on the phase term, with respect to not only integrable functions but also finte measures, called Fourier-Stieltjes transform, of a probability measure $\mu$ is called a characteristic function of the distribution $\mu$.
In terms of probability theory, it is nothing but the function defined by the expectation $\f(t):=Ee^{itX}$, where $X$ is a random variable of law $\mu$.
The Bochner theorem states that the necessary and sufficient condition for being a characteristic function is the positive definiteness and continuity.





\newpage
\section{On the group $\Z$: complex analysis}



In this section, we are going to investigate the origin of positive definiteness that occurs in the context of complex analysis via establishing the following one-to-one correspondences:
\begin{figure}[h]
\begin{tikzcd}[column sep = 0]
&\begin{tabular}{c}Points in the closed convex hull of\\the curve $(e^{-i\theta},e^{-i2\theta},\cdots)$ in $\C^\N$\end{tabular}&\\
\begin{tabular}{c}Positive definite\\sequences $(c_k)_{k\in\Z}$\\with $c_0=1$\end{tabular}
&\text{Carath\'eodory functions}\lar[<->]\uar[<->]\rar[<->]
&\begin{tabular}{c}Probability Borel\\measures on $\T$\end{tabular}.
\end{tikzcd}
\end{figure}

The vertical, left, and right arrows are discussed in Section 2.1, 2.2, and 2.3 respectively.
The definition of each term will be given throughout this section, and Bochner's theorem on the additivie group $\Z$ will be finally deduced as a corollary of the above correspondences.

\subsection{The Carath\'eodory coefficient problem}

The concept of positive definiteness of functions were originally inspired by the ``Carath\'eodory coefficient problem'' in early complex analysis.
The problem asks the condition on the power series coefficients for an analytic function defined on the open unit disk to have values of positive real part.
In other words, the Carath\'eodory coefficient problem describes the power series coefficients of some special functions precisely defined as follows:

\begin{defn}
The \emph{Carath\'eodory class} is the set of all analytic functions $f$ that map the open unit disk into the region of positive real part, with normalization condition $f(0)=1$.
A function in the Carath\'eodory class will be often called a \emph{Carath\'eodory function}.
\end{defn}

\begin{ex}[M\"obius transforms]
Typical examples of functions in the Carath\'eodory class are given by the family of functions
\[f_\theta(z)=\frac{e^{i\theta}+z}{e^{i\theta}-z}=1+\sum_{k=1}^\infty2e^{-ik\theta}z^k\]
parametrized by $\theta\in[0,2\pi)$.
We can check they are eactly the M\"obius transformations that map the unit disk to the right half space having normalization $f(0)=1$.
This family of examples play a crucial role in the representation problem of functions in the Carath\'eodory class.
\end{ex}

\begin{ex}[Convex combinations]
Note the Carath\'eodory class is convex; if $f_0$ and $f_1$ belong to the Carath\'eodory class, then the real part of the image of the function
\[f_t(z)=(1-t)f_0(z)+tf_1(z)\]
is also positive for $0<t<1$ and $f_t(0)=(1-t)+t=1$, so $f_t$ also belongs to the Carath\'eodory class.
\end{ex}

\begin{ex}[Positive harmonic functions]
Let $f$ be in the Carath\'eodory class.
By definition, the real part $\Re f:\D\to\R$ is a positive harmonic function such that $f(0)=1$.
Conversely, since there is a unique harmonic conjugate up to constant, we can recover $f$ from its real part by letting $\Im f(0)=0$.
In other words, there is a one-to-one correspondence between the Carath\'odory class and the positive harmonic functions on the open uni disk that has the value one at zero.
\end{ex}

Carath\'eodory's result intuitively tells us that every function in the Carath\'eodory class can be constructed by convex combinations the M\"obius transforms $f_\theta$.
As a result, they can be viewed as ``extreme points'' in the Carath\'eodory class.
We discuss about the extreme points after the proof of the Carath\'eodory theorem.

Before the discussion, we develop a lemma as a preparation for the interplay between complex analysis and Fourier analysis.

\begin{lem}[Fourier coefficient of analytic functions]
Let $f$ be an analytic function on the open unit disk $\D$ with $f(0)\in\R$ with
\[f(z)=c_0+\sum_{k=1}^\infty2c_kz^k,\]
the power series expansion of $f$ at $z=0$.
Then, for $0\le r<1$ and $k\in\Z$ we have
\[c_kr^{|k|}=\frac1{2\pi}\int_0^{2\pi}\Re f(re^{i\theta})e^{-ik\theta}\,d\theta,\]
where we use the notation $c_{-k}:=\bar c_k$.
\end{lem}
\begin{pf}
Suppose $k>0$ first.
The Cauchy integral formula writes
\begin{align*}
2c_kk!=\pd[k]{f}{z}(0)=\frac{k!}{2\pi i}\int_{|z|=r}\frac{f(z)}{z^{k+1}}\,dz=\frac{k!}{2\pi i}\int_0^{2\pi}\frac{f(re^{i\theta})}{(re^{i\theta})^{k+1}}\,ire^{i\theta}\,d\theta,
\end{align*}
and it implies
\[2c_kr^k=\frac1{2\pi}\int_0^{2\pi}f(re^{i\theta})e^{-ik\theta}\,d\theta.\]
Since $f(z)\,z^k$ is analytic, the Cauchy theorem is applied to have
\[0=\frac1{2\pi i}\int_{|z|=r}f(z)\,z^k\,dz=\frac1{2\pi}\int_0^{2\pi}f(re^{i\theta})r^ke^{ik\theta}\,d\theta,\]
and it implies
\[0=\frac1{2\pi}\int_0^{2\pi}\bar{f(re^{i\theta})}e^{-ik\theta}\,d\theta.\]
By combining the above equations, we obtain the formula.
For $k=0$, applying the Cauchy theorem for $f$, we have
\[c_0=f(0)=\frac1{2\pi i}\int_{|z|=r}\frac{f(z)}z\,dz=\frac1{2\pi}\int_0^{2\pi}\Re f(re^{i\theta})\,d\theta.\]
For $k<0$, we can obtain the same formula by taking complex conjugation on the case $k>0$.

Alternaively, we can show the same result using the orthogonal relation of complex exponential functions.
Easy computation shows the identity
\begin{align*}
\Re f(re^{i\theta})
&=\frac12[f(re^{i\theta})+\bar{f(re^{i\theta})}]\\
&=\frac12\left[\left(1+\sum_{k=1}^\infty2c_k(re^{i\theta})^k\right)+\bar{\left(1+\sum_{k=1}^\infty2c_k(re^{i\theta})^k\right)}\right]\\
&=\frac12\left[\left(1+\sum_{k=1}^\infty2c_kr^ke^{ik\theta}\right)+\left(1+\sum_{k=1}^\infty2\bar{c_k}r^ke^{-ik\theta}\right)\right]\\
&=\sum_{k=-\infty}^\infty c_kr^{|k|}e^{ik\theta}.
\end{align*}
From the uniform convergence of the power series on the compact set $\{z:|z|\le(r+1)/2\}$ and the orthogonality
\[\frac1{2\pi}\int_0^{2\pi}e^{-ik\theta}e^{il\theta}\,d\theta=\begin{cases}1&\text{ if }k=l\\0&\text{ if }k\ne l\end{cases},\]
it follows that
\[\frac1{2\pi}\int_0^{2\pi}\Re f(re^{i\theta})e^{-ik\theta}\,d\theta=\sum_{l=-\infty}^{\infty}c_lr^{|l|}\frac1{2\pi}\int_0^{2\pi}e^{il\theta}e^{-ik\theta}\,d\theta=c_kr^{|k|}.\qedhere\]
\end{pf}

Now, we prove the theorem.
The original paper of Carath\'eodory deals with the functions analytic on a neighborhood of the closed unit disk, but the same idea is extended well to the functions that may have harsh behavior on the boundary.
Furthermore, by loosening the regularity requirement at boundary, we can establish the exact description of Carath\'eodory functions in terms of their coefficients.

\begin{thm}[Carath\'eodory]
Let $f$ be an analytic function on the open unit disk with the power series expansion
\[f(z)=1+\sum_{k=1}^\infty2c_kz^k.\]
Then, $f$ belongs to the Carath\'eodory class if and only if for each $n$ the point $(c_1,\cdots,c_n)\in\C^n$ belongs to the convex hull of the curve $(e^{-i\theta},\cdots,e^{-in\theta})\in\C^n$ parametrized by $\theta\in[0,2\pi)$.
\end{thm}
\begin{pf}
($\Leftarrow$)
Denote by $K_n$ the convex hull of the curve $\theta\mapsto(e^{-i\theta},\cdots,e^{-in\theta})\in\C^n$.
Suppose first that $(c_1,\cdots,c_n)\in K_n$.
For each $n$, there exists a finite sequence of pairs $(\lambda_{n,j},\theta_{n,j})_j$ having the following convex combination
\[(c_1,\cdots,c_n)=\sum_j\lambda_{n,j}(e^{-i\theta_{n,j}},\cdots,e^{-in\theta_{n,j}})\]
with coefficients $\lambda_{n,j}\ge0$ such that $\sum_j\lambda_{n,j}=1$.
Define
\[f_n(z):=\sum_j\lambda_{n,j}\frac{e^{i\theta_{n,j}}+z}{e^{i\theta_{n,j}}-z},\]
which has positive real part on $|z|<1$ because $\Re(e^{i\theta_{n,j}}+z)/(e^{i\theta_{n,j}}-z)>0$ for $|z|<1$.
Then,
\begin{align*}
f_n(z)
&=\sum_j\lambda_{n,j}(1+\sum_{k=1}^\infty2e^{-ik\theta_{n,j}}z^k)\\
&=1+\sum_{k=1}^n2c_kz^k+\sum_{k=n+1}^\infty\left(\sum_j2\lambda_{n,j}e^{-ik\theta_{n,j}}\right)z^k
\end{align*}
implies
\begin{align*}
|f_n(z)-f(z)|
&=\left|\sum_{k=n+1}^\infty\left(\sum_j2\lambda_{n,j}e^{-ik\theta_{n,j}}\right)z^k-\sum_{k=n+1}^\infty2c_kz^k\right|\\
&\le\sum_{k=n+1}^\infty\left|\left(\sum_j2\lambda_{n,j}e^{-ik\theta_{n,j}}\right)-2c_k\right||z|^k\\
&\le\sum_{k=n+1}^\infty4|z|^k
\end{align*}
converges to zero for $|z|<1$.
Therefore, $f$ has non-negative real part on the open unit disk.
The non-negativity is strengthen to the positivity by the open mapping theorem so that $f$ belongs to the Carath\'eodory class.

($\Rightarrow$)
Conversely, suppose that $f$ is in the Carath\'eodory class.
Let $(\gamma_1,\cdots,\gamma_n)$ be any point on the surface $\partial K_n$ of $K_n$ and $S$ any supporting hyperplane of $K_n$ tangent at $(\gamma_1,\cdots,\gamma_n)$.
Let $(u_1,\cdots,u_n)$ be the outward unit normal vector of the supporting hyperplane $S$.
Note that this unit normal vector is uniquely determined with respect to the induced real inner product structure on $2n$-dimensional space $\C^n$ described by
\[\<(z_1,\cdots,z_n),(w_1,\cdots,w_n)\>=\sum_{k=1}^n(\Re z_k\Re w_k+\Im z_k\Im w_k)=\Re\sum_{k=1}^nz_k\bar w_k.\]
Then, $\sum_{k=1}^n|u_k|^2=1$ and further that the maximum
\[M:=\max_{(x_1,\cdots,x_n)\in K_n}\ \Re\sum_{k=1}^nx_k\bar u_k>0\]
is attained at $(\gamma_1,\cdots,\gamma_n)$.
Our goal is to verify the bound
\[\Re\sum_{k=1}^nc_k\bar u_k\le M,\]
which implies that $(c_1,\cdots,c_n)$ is contained in every half space tangent to $K_n$ so that we finally obtain $(c_1,\cdots,c_n)\in K_n$.

Since for any $\theta\in[0,2\pi)$ the point $(e^{-i\theta},\cdots,e^{-in\theta})$ is in $K_n$ so that
\[\Re\sum_{k=1}^ne^{-ik\theta}\bar u_k\le M,\]
we have for arbitrarily small $\e>0$ that
\[\Re\sum_{k=1}^n\frac1{r^k}e^{-ik\theta}\bar u_k\le M+\e\]
for any $0<r<1$ sufficiently close to $1$, thus we can write
\begin{align*}
\Re\sum_{k=1}^nc_k\bar u_k
&=\Re\sum_{k=1}^n\frac1{2\pi r^k}\int_0^{2\pi}\Re f(re^{i\theta})e^{-ik\theta}\bar u_k\,d\theta\\
&=\frac1{2\pi}\int_0^{2\pi}\Re f(re^{i\theta})\Re\sum_{k=1}^n\frac1{r^k}e^{-ik\theta}\bar u_k\,d\theta\\
&\le\frac1{2\pi}\int_0^{2\pi}\Re f(re^{i\theta})\,d\theta\cdot(M+\e)\\
&=M+\e
\end{align*}
thanks to the positivity of $\Re f$, and by limiting $r\to1$ from left we get the bound
\[\Re\sum_{k=1}^nc_k\bar u_k\le M.\qedhere\]
\end{pf}

Here we introduce an infinite-dimentional version of this theorem.

\begin{prop}
Consider a sequence space $\C^\N$, endowed with the standard product topology.
Then, the condition addressed in Caracth\'eodory's theorem is equivalent to the following: the point $(c_1,c_2,\cdots)\in\C^\N$ belongs to the closed convex hull of the curve $(e^{-i\theta},e^{-i2\theta},\cdots)\in\C^\N$ parametrized by $\theta\in[0,2\pi)$.

Furthermore, the curve $(e^{-i\theta},e^{-i2\theta},\cdots)\in\C^\N$ is the set of extreme points of its closed convex hull.
\end{prop}
\begin{pf}
Denote by $K_n$ the convex hull of the curve $\theta\mapsto(e^{-i\theta},\cdots,e^{-in\theta})\in\C^n$, and by $K$ the closed convex hull of the curve $\theta\mapsto(e^{-i\theta},e^{-i2\theta},\cdots)\in\C^\N$.
If we assume the Carath\'eodory coefficient condition is true, then since for each $n$ we have a convex combination
\[(c_1,\cdots,c_n)=\sum_j\lambda_{n,j}(e^{-i\theta_{n,j}},\cdots,e^{-in\theta_{n,j}})\]
with coefficients such that $\lambda_{n,j}\ge0$ and $\sum_j\lambda_{n,j}=1$, the sequence
\begin{align*}
&(c_1,\cdots,c_n,\sum_j\lambda_{n,j}e^{-i(n+1)\theta_{n,j}},\sum_j\lambda_{n,j}e^{-i(n+2)\theta_{n,j}}\cdots)\\
&\qquad\qquad=\sum_j\lambda_{n,j}(e^{-i\theta_{n,j}},\cdots,e^{-in\theta_{n,j}},e^{-i(n+1)\theta_{n,j}},e^{-i(n+2)\theta_{n,j}},\cdots)
\end{align*}
is contained in  and converges to the point $(c_1,c_2,\cdots)$ in the product topology as $n\to\infty$, so we are done with the desired result.
For the opposite direction, let $(c_1,c_2,\cdots)\in K$.
By definition of $K$ we have an expression
\[c_k=\lim_{m\to\infty}\sum_{j=1}^m\lambda_{m,j}e^{-ik\theta_{m,j}}\]
with $\lambda_{m,j}\ge0$ and $\sum_{j=1}^m\lambda_{m,j}=1$, for each $k$.
Then,
\[(c_1,\cdots,c_n)=\lim_{m\to\infty}\sum_{j=1}^m\lambda_{m,j}(e^{-i\theta_{m,j}},\cdots,e^{-in\theta_{m,j}})\]
belongs to $K_n$ because $K_n$ is closed.


Fix $\theta\in[0,2\pi)$ and suppose two complex sequences $(c_1,c_2,\cdots)$ and $(d_1,d_2,\cdots)$ in $\C^\N$ are contained in $K$ and satisfy
\[\frac{c_k+d_k}2=e^{-ik\theta}\]
for all $k\in\N$.
For each $k$, since all components of $K$ are bounded by one so that $|c_k|\le1$ and $|d_k|\le1$, and since $e^{-ik\theta}$ is an extreme point of the closed unit disk $\bar\D\subset\C$, we have $c_k=d_k=e^{-ik\theta}$, we deduce that $(e^{-i\theta},e^{-i2\theta},\cdots)$ is an extreme point of $K$.
Conversely, every extreme point of $K$ is contained in the curve $(e^{-i\theta},e^{-i2\theta},\cdots)$ by Milman's ``converse'' theorem of the Krein-Milman theorem[citation: Phelps].
\end{pf}



\subsection{Toeplitz's algebraic condition}

Toeplitz discovered the coefficient condition addressed in the Carath\'eodory's paper which regards convex bodies enveloped by a curve can be equivalently described in terms of an algebraic condition that the hermitian matrices
\[H_n:=(c_{k-l})_{k,l=1}^n=\mat{c_0&c_1&c_2&\cdots&c_{n-1}\\c_{-1}&c_0&c_1&\cdots&c_{n-2}\\c_{-2}&c_{-1}&c_0&\cdots&c_{n-3}\\\vdots&\vdots&\vdots&\ddots&\vdots\\c_{-n+1}&c_{-n+2}&c_{-n+3}&\cdots&c_0}\]
of size $n\times n$ always have non-negative determinant for any $n$.
This algebraic condition is equivalent to that $H_n$ are all positive semi-definite matrices.
Since the principal minors of a positive semi-definite matrix is positive semi-definite, and since a hermitian matrix such that every leading principal minor has non-negative determinant is positive semi-definite, the bilateral sequence $(c_k)_{k=-\infty}^\infty$ is positive definite function when we consider it as a complex-valued function on $\Z$ that maps an integer $k$ to $c_k$ if and only if it is a positive definite \emph{sequence} in the following sense:

\begin{defn}
A bilateral complex sequence $(c_k)_{k=-\infty}^\infty$ is said to be \emph{positive definite} if
\[\sum_{k,l=1}^nc_{k-l}\xi_k\bar\xi_l\ge0\]
for each $n$ and $(\xi_1,\cdots,\xi_n)\in\C^n$.
\end{defn}

\begin{thm}[Carath\'eodory-Toeplitz]
Let $f$ be an analytic function on the open unit disk with the power series expansion
\[f(z)=1+\sum_{k=1}^\infty2c_kz^k.\]
Then, $f$ belongs to the Carath\'eodory class if and only if the sequence $(c_k)_{k=-\infty}^\infty$ is positive definite, where we use the notations $c_0=1$ and $c_{-k}=\bar{c_k}$.
\end{thm}
\begin{pf}
($\Rightarrow$)
If $f$ is in the Carath\'eodory class, then because
\[c_{k-l}r^{|k-l|}=\frac1{2\pi}\int_0^{2\pi}\Re f(re^{i\theta})e^{-i(k-l)\theta}\,d\theta,\]
we have
\[\sum_{k,l=1}^nc_{k-l}\xi_k\bar\xi_l
=\lim_{r\uparrow1}\frac1{2\pi}\int_0^{2\pi}\Re f(re^{i\theta})\left|\sum_{k=1}^ne^{-ik\theta}\xi_k\right|^2\,d\theta\ge0\]
for each $n$.

($\Leftarrow$)
Conversely, assume that the coefficient sequence $(c_k)_{k=-\infty}^\infty$ is positive definite.
Put $\xi_k=z^{k-1}$ and $z=re^{i\theta}$ to write
\begin{align*}
0&\le\sum_{k,l=1}^{n+1}c_{k-l}z^{k-1}(\bar z)^{l-1}\\
&=\sum_{k,l=0}^nc_{k-l}r^{k+l}e^{i(k-l)\theta}\\
&=\sum_{k,l=0}^nc_{k-l}r^{|k-l|}r^{2\min\{k,l\}}e^{i(k-l)\theta}\\
&=\sum_{k=-n}^nc_kr^{|k|}e^{ik\theta}\sum_{l=0}^{n-|k|}r^{2l}\\
&=\sum_{k=-n}^nc_kr^{|k|}e^{ik\theta}\frac{1-r^{2(n-|k|+1)}}{1-r^2}\\
&=\frac1{1-r^2}\sum_{k=-n}^nc_kr^{|k|}e^{ik\theta}
-\frac{r^{n+2}}{1-r^2}\sum_{k=-n}^nc_kr^{n-|k|}e^{ik\theta}.
\end{align*}
For $r=|z|<1$ the first term tends to
\[\lim_{n\to\infty}\frac1{1-r^2}\sum_{k=-n}^nc_kr^{|k|}e^{ik\theta}=\frac1{1-|z|^2}\Re f(z),\]
and $|c_k|\le c_0=1$ implies the second term vanishes as
\[\left|\frac{r^{n+2}}{1-r^2}\sum_{k=-n}^nc_kr^{n-|k|}e^{ik\theta}\right|\le\frac{r^{n+2}}{1-r^2}(2n+1)\to0\]
as $n\to\infty$.
It proves $\Re f(z)\ge0$ for $|z|<1$, and we obtain $\Re f(z)>0$ by the open mapping theorem.
\end{pf}


\subsection{The Herglotz-Riesz representation theorem}

Herglotz proved another equivalent condition for the Carath\'eodory class in 1911, considered as the first Bochner-type theorem, which states the correspondence between the Carath\'eodory class and probability Borel measure on the unit circle.
The Carath\'eodory theorem states that the function $f$ in the Carath\'eodory class is a limit of convex combinations of M\"obius transforms $z\mapsto(e^{i\theta}+z)/(e^{i\theta}-z)$.
Herglotz's theorem, which we now also often call as the Herglotz-Riesz representation theorem, states that in fact $f$ is directly represented by the integral of the M\"obius transforms with respect to a newly constructed probability measure, instead of limiting process of convex sums.

The essential difficulty comes from the construction of a measure, and here we resolve this in the aid of either Helly's selection theorem or the Riesz-Markov-Kakutani representation theorem.
Suppose the function $f$ is analytic on a neighborhood of the closed unit disk $\bar\D$.
In this case, by appropriately manipulate the identities for $r=1$ in Lemma 2.1, or by using the Cauchy integral formula along the unit circle, we can get
\[f(z)=\frac1{2\pi}\int_0^{2\pi}\frac{e^{i\theta}+z}{e^{i\theta}-z}\Re f(e^{i\theta})\,d\theta.\]
Based on this representation of $f$, we will try to approximate the measure $d\mu$ with the absolutely continuous measures $(2\pi)^{-1}\Re f(re^{i\theta})\,d\theta$ by limiting $r\uparrow1$.
More precisely, we will use the following lemma:
\begin{lem}
Let $f$ be an analytic function on the open unit disk.
For $|z|<1$,
\[f(z)=\lim_{r\uparrow1}\frac1{2\pi}\int_0^{2\pi}\frac{e^{i\theta}+z}{e^{i\theta}-z}\Re f(re^{i\theta})\,d\theta.\qedhere\]
\end{lem}
\begin{pf}
By 
\begin{align*}
\lim_{r\uparrow1}\frac1{2\pi}\int_0^{2\pi}\frac{e^{i\theta}+z}{e^{i\theta}-z}\Re f(re^{i\theta})\,d\theta
&=\lim_{r\uparrow1}\frac1{2\pi}\int_0^{2\pi}\left(1+\sum_{k=1}^\infty2e^{-ik\theta}z^k\right)\Re f(re^{i\theta})\,d\theta\\
&=1+\lim_{r\uparrow1}\sum_{k=1}^\infty2\left(\frac1{2\pi}\int_0^{2\pi}e^{-ik\theta}\Re f(re^{-i\theta})\,d\theta\right)z^k\\
&=1+\lim_{r\uparrow1}\sum_{k=1}^\infty2c_kr^kz^k\\
&=f(z).\qedhere
\end{align*}
\end{pf}


\begin{thm}[The Herglotz-Riesz representation theorem]
Let $f$ be a complex-valued function defined on the open unit disk.
Then, $f$ belongs to the Carath\'eodory class if and only if $f$ is represented as the following Stieltjes integral
\[f(z)=\int_0^{2\pi}\frac{e^{i\theta}+z}{e^{i\theta}-z}\,d\mu(\theta),\]
where $\mu$ is a probability Borel measure on $\T=\R/2\pi\Z$.
\end{thm}
\begin{pf}[First proof: using Helly's selection theorem]
($\Leftarrow$)
Take a probability Borel measure $\mu$ on $\T$.
Then, we can check the function defined by
\[f(z):=\int_0^{2\pi}\frac{e^{i\theta}+z}{e^{i\theta}-z}\,d\mu(\theta)\]
is analytic on the open unit disk easily by using Morera's theorem and Fubini's theorem.
Recall that $z\mapsto(e^{i\theta}+z)/(e^{i\theta}-z)$ has positive real part since it is a conformal mapping that maps open unit disk onto the right half plane.
The function $f$ belongs to the Carath\'eodory class by the open mapping theorem since
\[\Re f(z)=\int_0^{2\pi}\Re\frac{e^{i\theta}+z}{e^{i\theta}-z}\,d\mu(\theta)\ge0.\]

($\Rightarrow$)
Fix $z$ in the open unit disk $\D$.
Define $f_n(\theta):=(2\pi)^{-1}\Re f((1-n^{-1})e^{i\theta})$ and
\[F_n(\theta):=\int_0^\theta\Re f_n(\psi)\,d\psi\]
for $\theta\in[0,2\pi]$.
Note $F_n(0)=0$ and $F_n(2\pi)=1$ for all $n$.
Since $\Re f\ge0$, $F_n$ is also monotonically increasing.
Therefore, the sequence $(F_n)_n$ has a pointwise convergent subsequence $(F_{n_j})_j$ on $[0,2\pi]$ by the Helly's selection theorem.
Let
\[F(\theta):=\lim_{\psi\downarrow\theta}\lim_{j\to\infty}F_{n_j}(\psi).\]
Then, we have $F(0)=0$ and $F(2\pi)=1$, and $F_{n_j}$ converges to $F$ at every continuity point $\theta$ of $F$.
It means $F_{n_j}$ converges to $F$ weakly as $j\to\infty$, so by the Portemanteau theorem, we get
\[\int_0^{2\pi}\frac{e^{i\theta}+z}{e^{i\theta}-z}dF_{n_j}(\theta)\to\int_0^{2\pi}\frac{e^{i\theta}+z}{e^{i\theta}-z}dF(\theta)\]
as $j\to\infty$ since $\theta\mapsto(e^{i\theta}+z)/(e^{i\theta}-z)$ is continuous and bounded on $\T$.
On the other hand,
\[\int_0^{2\pi}\frac{e^{i\theta}+z}{e^{i\theta}-z}dF_{n_j}(\theta)
=\frac1{2\pi}\int_0^{2\pi}\frac{e^{i\theta}+z}{e^{i\theta}-z}\Re f((1-n_j^{-1})e^{i\theta})\,d\theta\to f(z)\]
as $j\to\infty$.
Therefore, by the uniqueness of limit, we have
\[f(z)=\int_0^{2\pi}\frac{e^{i\theta}+z}{e^{i\theta}-z}dF(\theta)=\int_0^{2\pi}\frac{e^{i\theta}+z}{e^{i\theta}-z}d\mu(\theta),\]
where $\mu$ is the probability measure on $\T$ defined by the distribution function $F$ as $\mu([0,\theta])=F(\theta)$.
\end{pf}

\begin{pf}[Second proof: using the Riesz representation theorem]
As we have seen in the first proof that uses Helly's selection theorem, one direction is trivial.
Suppose $f$ is a Carath\'eodory function.
Let $g\in C(\T)$ be a complex-valued test function.
Define a sequence of complex linear functionals $l_n$ on $C(\T)$ as
\[l_n[g]:=\frac1{2\pi}\int_0^{2\pi}g(\theta)\Re f((1-n^{-1})e^{i\theta})\,d\theta.\]
It is positive and bounded since $\Re f\ge0$ and $\|l_r\|=l_r[1]=1$.
By the Alaoglu theorem, the sequence has $(l_n)_n$ a subsequence $(l_{n_j})_j$ that converges in the weak$^*$ topology of $C(\T)^*$.
If we let $l$ be the limit, then $l[1]=\lim_{j\to\infty}l_{n_j}[1]=1$ because $1\in C(\T)$.
(Notice that it does not valid if the domain space, $\T$ here, is not compact, and we will see this problem more carefully in the next chapter.)

By the Riesz-Markov-Kakutani representation theorem, there is a probability Borel measure $\mu$ on $\T$ such that
\[l[g]=\frac1{2\pi}\int_0^{2\pi}g(\theta)\,d\mu(\theta)\]
for all $g\in C(\T)$.
Then, for each fixed $z$ in the open unit disk it follows from Lemma 2.5 that
\[\frac1{2\pi}\int_0^{2\pi}\frac{e^{i\theta}+z}{e^{i\theta}-z}d\mu(\theta)=l[g_z]=\lim_{j\to\infty}l_{n_j}[g_z]=f(z)\]
since $g_z(\theta):=(e^{i\theta}+z)/(e^{i\theta}-z)$ belongs to $C(\T)$.
\end{pf}

As a corollary of Herglotz' theorem, we finally arrive at:

\begin{cor}[Bochner's theorem on $\Z$]
A function $c:\Z\to\C$ is positive-definite and $c_0=1$ if and only if there is a probability Borel measure $\mu$ on $\T=\R/2\pi\Z$ such that
\[c_k=\int_0^{2\pi}e^{-ik\theta}\,d\mu(\theta).\]
\end{cor}
\begin{pf}
Let $\mu$ be a probability Borel measure on $\T$.
Then, the sequence defined in the statement is positive definite because
\begin{align*}
\sum_{k,l=1}^nc_{k-l}\xi_k\bar\xi_l
&=\sum_{k,l=1}^n\int_0^{2\pi}e^{-i(k-l)\theta}\,d\mu(\theta)\ \xi_k\bar\xi_l\\
&=\int_0^{2\pi}\left|\sum_{k=1}^ne^{-ik\theta}\xi_k\right|^2d\mu(\theta)\ge0
\end{align*}
for any $(\xi_1,\cdots,\xi_n)\in\C^n$, and $c_0=1$ is clear.

On the other hand, if the sequence $(c_k)_{k=-\infty}^\infty$ is positive definite and $c_0=1$, then the function $z\mapsto1+\sum_{k=1}^\infty2c_kz^k$ is in the Carath\'eodory class.
By the Herglotz-Riesz representation theorem, there is a probability regular Borel measure $\mu$ on $\T$ such that
\begin{align*}
1+\sum_{k=1}^\infty2c_kz^k
&=\int_0^{2\pi}\frac{e^{it}+z}{e^{it}-z}\,d\mu(t)\\
&=\int_0^{2\pi}\left(1+\sum_{k=1}^\infty2e^{-ik\theta}z^k\right)\,d\mu(t)\\
&=1+\sum_{k=1}^\infty2\left(\int_0^{2\pi}e^{-ik\theta}\,d\mu(\theta)\right)z^k
\end{align*}
in $z\in\D$, hence the desired result follows.
\end{pf}


\begin{ex}[Dirac measures]

\end{ex}
\begin{ex}[Continuous restrictions]

\end{ex}





\newpage
\section{On the group $\R$: probability theory}

We have seen the relation of positive definite sequences and measures on the unit circle $\T$.
On the real line $\R$, predictably, we can also prove that there exists a correspondence between measures and positive definite functions.
The previous chapter used measures to characterize certain complex functions and positive definite sequences, but from this section, we will see how the positive functions are used in studying measures.

The systematic study of positive definite functions to study measures virtually starts in probability theory by Paul L\'evy.
Recall that a probability distribution is defined as a measure of norm one on a ``state space'', which is $\R$ for usual random variables.
Some classical problems including central limit problems and laws of large numbers arisen in probability theory want to describe limit behaviors of probability distributions.
L\'evy's discovery was that it is easier to verify the convergence of probability distributions via the \emph{Fourier transforms} of probability measures, instead of the measures themselves.

The Fourier transform(often called as \emph{Fourier-Stieltjes} transform when we emphasize the \emph{measures}, the objects transformed) of a probability measure is called a \emph{characteristic function}.
One of possible statement of Bochner's theorem is that a complex function on a real line is a characteristic function of a probability measure if and only if it is continuous and positive definite, i.e. the theorem gives the one-to-one correspondence of probability measures on $\R$ and the continuous positive definite functions on $\R$, under the Fourier-Stieltjes transform.



\subsection{Topologies on the space of probability measures}


\begin{defn}[Weak convergence]
Let $(\mu_n)_n$ and $\mu$ be probability Borel measures on a metric space $S$.
We say $\mu_n$ \emph{weakly converges to} $\mu$ if
\[\int g\,d\mu_n\to\int g\,d\mu\]
as $n\to\infty$ for any $g\in C_b(S)$, where $C_b(S)$ denotes the space of continuous and bounded functions.
\end{defn}






\subsection{The Le\'vy continuity theorem}

The direct connection between convergences in two different realms, measures and positive definite functions, is encoded in the L\'evy continuity theorem.
In this section, we will prove Bochner's theorem on $\R$ with the aid of the L\'evy continuity theorem.
This theorem connects the weak convergence of probability measures and pointwise convergence of \emph{characteristic functions}.
A characteristic function is defined as the Fourier transform, but conventionally reversed the sign on the phase term, of a probability measure, and is the place where the positive definiteness comes.

\begin{defn}[Characteristic functions]
Let $\mu$ be a probability measure on $\R$ and $X$ a random variable of distributioin $\mu$.
Note that such random variable always exists.
The \emph{characteristic function} of $X$ is a function $\f:\R\to\C$ defined by $\f(t):=Ee^{itX}$.
Equilvalently, $\f$ is given by
\[\f(t):=\int e^{itx}\,d\mu(x).\]
\end{defn}
\begin{prop}[Basic properties of characteristic functions]
Let $\f$ be a characteristic function of a probability Borel measure $\mu$ on $\R$.
\begin{parts}
\item $\f$ is positive definite.
\item $\f$ is uniformly continuous.
\end{parts}
\end{prop}
\begin{pf}
\end{pf}
\begin{ex}[Mathias' examples]

\end{ex}
\begin{ex}[Polya]
\end{ex}

Characteristic functions take an advantage that we can learn the information about probability measures by investigating the continuous functions instead of studying measures directly.



\begin{defn}[Tight measures]

\end{defn}

In order for a family of probability measures to be tight, their tail probabilities ought to be uniformly controlled.
The following lemma is useful in bounding tail probabilities in terms of characteristic functions; the averaging of $1-\f$ near zero provides with a reasonable estimate of the tail probability.

\begin{lem}
Let $\mu$ be a probability Borel measure on $\R$ and $\f$ be its characteristic function.
Then,
\[\mu([-\tfrac2\delta,\tfrac2\delta]^c)\le2\cdot\frac1{2\delta}\int_{-\delta}^\delta(1-\f(t))\,dt\]
for any $\delta>0$.
In particular, a single measure is tight.
\end{lem}
\begin{pf}
Write the average with the sinc function as
\begin{align*}
\frac1{2\delta}\int_{-\delta}^\delta\f(t)\,dt
&=\int\frac1{2\delta}\int_{-\delta}^\delta e^{itx}\,dt\,d\mu(x)\\
&=\int\frac1{2\delta}\cdot\frac{e^{i\delta x}-e^{-i\delta x}}{ix}\,d\mu(x)\\
&=\int\frac{\sin\delta x}{\delta x}\,d\mu(x).
\end{align*}
Then, for appropriate constant $R>0$ we have the following estimate of the sinc function term
\begin{align*}
\int\frac{\sin\delta x}{\delta x}\,d\mu(x)
\le\int_{|x|\le R}1d\mu(x)+\int_{|x|>R}\frac1{|\delta x|}\,d\mu(x)\\
=1-\int_{|x|>R}\left(1-\frac1{|\delta x|}\right)\,d\mu(x).
\end{align*}
If we take $R=\frac2\delta$, then the Chebyshev inequality has
\[\frac12\mu([-\tfrac2\delta,\tfrac2\delta]^c)\le\int_{|x|>\frac2\delta}\left(1-\frac1{|\delta x|}\right)\,d\mu(x)\le1-\frac1{2\delta}\int_{-\delta}^\delta\f(t)\,dt,\]
so we are done.
\end{pf}


\begin{thm}[The L\'evy continuity theorem]
Let $(\mu_n)_{n=1}^\infty$ be a sequence of probability Borel measures on $\R$ and $\f_n$ their characteristic functions.
Then, $\mu_n$ converges weakly to a probability Borel measure $\mu$ if and only if $\f_n$ converges pointwise to a function $\f$ that is continuous at zero.
\end{thm}
\begin{pf}
($\Rightarrow$)

($\Leftarrow$)
For $\e>0$, take $\delta>0$ using the continuity of $\f$ such that
\[\frac1{2\delta}\int_{-\delta}^\delta(1-\f(t))\,dt<\frac\e4.\]
By the bounded convergence theorem, there is $N>0$ such that
\[\frac1{2\delta}\int_{-\delta}^\delta|\f_n(t)-\f(t)|\,dt<\frac\e4\]
so that we have
\[\mu_n([-\tfrac2\delta,\tfrac2\delta]^c)\le2\cdot\frac1{2\delta}\int_{-\delta}^\delta(1-\f_n(t))\,dt<\e\]
for all $n>N$.
For each $n\le N$, since every single measure is tight, there is compact $K_n\subset\R$ such that $\mu(K_n^c)<\e$.
If we define a compact set $K:=[-\frac2\delta,\frac2\delta]\cup\bigcup_{n=1}^NK_n$, then $\mu_n(K^c)<\e$ for all $n$, so the sequence $\mu_n$ is tight.

Let $(\mu_{n_j})_j$ be any subsequence that converges weakly to a probability measure.
The limit of this subsequence is independent on the choice of the subsequence since its characteristic function is given by the pointwise limit $\lim_{j\to\infty}\f_{n_j}=\f$, by the first half of this theorem.
Let $\mu$ be this unique limit.
Then, $\mu_n$ converges weakly to $\mu$ since the tightness guarantees that every subsequence of $\mu_n$ has a further subsequence, which converges to $\mu$ weakly.
\end{pf}


There are various ways to prove Bochner's theorem on $\R$.
For example, we can prove it using either Helly's selection theorem or the Riesz-Markov-Kakutani representation theorem in the same manner as we did in the previous chapter.
We introduce a new proof that follows from the Herglotz representation theorem, in order to see the relation of two Bochner's theorem on $\Z$ and $\R$.
In this proof, the L\'evy continuity theorem is used as a key lemma.

\begin{cor}[Bochner's theorem on $\R$]
A function $\f:\R\to\C$ is continuous and positive-definite such that $\f(0)=1$ if and only if there is a probability Borel measure $\mu$ on $\R$ such that
\[\f(t)=\int e^{itx}\,d\mu(x).\]
\end{cor}
\begin{pf}
Let $\mu$ be a probability Borel measure on $\R$.
Then, the function $\f$ defined in the statement is positive definite because
\begin{align*}
\sum_{k,l=1}^n\f(t_k-t_l)\xi_k\bar\xi_l
&=\sum_{k,l=1}^n\int e^{i(t_k-t_l)x}\,d\mu(x)\xi_k\bar\xi_l\\
&=\int\left|\sum_{k=1}e^{it_kx}\xi_k\right|^2\,d\mu(x)\ge0.
\end{align*}
It is continuous because a single probability measure $\mu$ is tight so that for every $\e>0$ there is $M>0$ such that
\begin{align*}
|\f(t)-\f(s)|&\le\int|e^{itx}-e^{isx}|\,d\mu(x)
=\int|2\sin(\frac{t-s}2x)|\,d\mu(x)\\
&\le\int_{|x|\le M}|(t-s)x|\,d\mu(x)+\int_{|x|>M}\,d\mu(x)\\
&\le M|t-s|+\frac\e2<\e
\end{align*}
whenever $|t-s|<\e/2M$.
The normalization condition $f(0)=1$ is clear.

Conversely, suppose $\f$ is continuous and positive definite.
For each small $u>0$, since the sequence $(\f(uk))_{k\in\Z}$ is positive definite, by the Herglotz-Riesz representation theorem, there is a finite regular Borel measure $\nu_u$ on $[-\pi,\pi)$ such that
\[\f(uk)=\int_{-\pi}^\pi e^{-ik\theta}\,d\nu_u(\theta)\]
for every $k\in\Z$.
If we define a measure $\mu_u$ on $\R$ such that the support is contained in $[-\pi/u,\pi/u]$ and $\mu_u(E):=\nu_u(-uE)$ for Borel sets $E\subset[-\pi/u,\pi/u)$, then
\[\f(uk)=\int_{-\pi/u}^{\pi/u}e^{iukx}\,d\mu_u(x)=\f_u(uk)\]
for every $k\in\Z$, where $\f_u$ is the characteristic function of $\mu_n$.

Note that $\nu_u$ converges to the Dirac measure $\delta_0$ as $u\to0$ in weak$^*$ topology of $C(\T)^*$ where $\T$ is identified with the interval $[-\pi,\pi)$.
This is because trigonometric polynomials are uniformly dense in $C(\T)$ and $\nu_u$ are uniformly bounded in norm; for any $\e>0$ and $g\in C(\T)$, there is a trigonometric polynomial $h=\sum_kc_ke^{-ik\theta}$ such that $\|g-h\|_{C(\T)}<\e/2$, which implies
\begin{align*}
|\<g,\nu_u\>-g(0)|
&\le|\<g-h,\nu_u\>|+|\<h,\nu_u\>-h(0)|+|h(0)-g(0)|\\
&\le\e+|\sum_kc_k\f(uk)-h(0)|
\end{align*}
and
\[\sum_kc_k\f(uk)\to\sum_kc_k=h(0)\]
as $u\to0$.

For each $t\in\R$ and $u>0$, take $t_u$ such that $|t-t_u|<u/2$ and $t_u\in u\Z$.
Then, we get
\begin{align*}
|\f_u(t)-\f_u(t_u)|
&=|\int(e^{itx}-e^{it_ux})\,d\mu_u(x)|\\
&=|\int_{-\pi}^\pi(e^{i\frac tu\theta}-e^{i\frac{t_u}u\theta})\,d\nu_u(\theta)|\\
&\le\int_{-\pi}^\pi\left|\left(\frac tu-\frac{t_u}u\right)\theta\right|\,d\nu_u(\theta)\\
&\le\frac12\int_{-\pi}^\pi|\theta|\,d\nu_u(\theta)\to0
\end{align*}
as $u\to0$ since the function $\theta\mapsto|\theta|$ is continuous function on $\T$ if we view it as $[-\pi,\pi)$.
Therefore, the pointwise convergence is verfied as
\begin{align*}
|\f_u(t)-\f(t)|&\le|\f_u(t)-\f_u(t_u)|+0+|\f(t_u)-\f(t)|\to0
\end{align*}
as $u\to0$, and since $\f$ is continuous at zero, we can conclude that $\f$ is a characteristic function by the L\'evy continuity theorem.
\end{pf}


\subsection{Bochner's theorem in infinite dimensions}

bochner
measure <=> pos def continuous

schwarts bochner (finite condition removed)
tempered measure <=> pos def tempered dist


on hilbert space
measure <=> pos def continuous + trace class



\subsection{Examples: Polya's criterion}
Mathias' examples







\section{On locally compact groups: representation theory}

\subsection{Dual group and Fourier transform}
\subsection{The Pontryagin duality}
\subsection{Examples: topological groups in number theory}
$p$-adic integer $\Z_p$ and the Pr\"ufer $p$-subgroup.
Profinite integers and $\Q/\Z$.
Tate thesis: locally compact group of adeles.

\subsection{Why not non-abelian?}



\bibliographystyle{acm}
\bibliography{bib}


\end{document}