\documentclass{article}
\usepackage{../../ikany}
\usepackage[margin=3cm]{geometry}
\usepackage[T1]{fontenc}
\usepackage[bitstream-charter,cal]{mathdesign}
\linespread{1.15}


\title{Global Existence of Classical Solutions\\to the Vlasov-Poisson System}
\author{Ikhan Choi}
\date{October 16, 2020.}


\begin{document}
\maketitle
\tableofcontents

\section*{Acknowledgement}
This report is written in Undergraduate Research Program of Postech during 2019 fall semester, under the support and advice of professor Donghyun Lee.


\clearpage
\section{The Vlasov-Poisson system}
Consider the following Cauchy problem for the \emph{Valsov-Poisson system}:
\begin{align}
\left\{\ \begin{alignedat}{2}
&\pd_tf+v\cdot\nabla_xf+\gamma E\cdot\nabla_vf=0,&&\qquad(t,x,v)\in(0,\infty)\times\R_x^3\times\R_v^3,\\
&E(t,x)=-\nabla_x\Phi,\\
&\Phi(t,x)=\int\frac{\rho(y)}{4\pi|x-y|}\,dy,\\
&\rho(t,x)=\int f\,dv,\\
&f(0,x,v)=f_0(x,v)\ge0,
\end{alignedat}\right.
\end{align}
where $\gamma=\pm1$.
For example, we have \emph{repulsive problem} $\gamma=+1$ for electrons in plasma theory and \emph{attractive problems} $\gamma=-1$ for galactic dynamics.
($\rho$ denotes mass density.)

This report is a review of Schaeffer's paper \cite{schaeffer1991global}, and is written thanks to Glassey's book \cite{glassey1996cauchy}.
We mainly investigate the local and global existence problem for a classical solution of the Cauchy problem for the Vlasov-Poisson system.
More precisely, we prove there is a unique global $C_c^1$ solution when given a $C_c^1$ initial data $f_0$.
Let us define our solution space.

\begin{defn*}
Let $f_0:\R^6\to[0,\infty]$ be a function.
A function $f:[0,T]\times\R^6\to\R$ is said to be a \emph{classical solution} of the Cauchy problem for the Vlasov-Poisson system with initial data $f_0$ if $f\in C^1([0,T];C_c^1(\R^6))$ and satisfies all equations in (1) on its domain.
Further, if $f\in C^1(\R^+;C_c^1(\R^6))$, then the classical solution $f$ is said to be \emph{global}.
\end{defn*}

The precise statement of the global existence theorem is as follows:
\begin{thm}
Let $f_0\in C_c^1(\R^6)$ with $f_0\ge0$.
Then, there exists a unique global classical solution of the Cauchy problem for the Vlasov-Poisson system with initial data $f_0$.
\end{thm}

Results in sections 1.1 and 1.2 provide basic ingredients that will be used in the whole article.
On the other hand, results in 1.3 cannot be used in any local existence proof because they assume the existence of solutions, but they help understand the fundamental nature of solutions of the Vlasov-Poisson system and are used in the proof of global existence.

\begin{notn*}
We use the asymptotic notation
\[g(t)\lesssim h(t)\iff\exists\,c=c(f_0),\quad g(t)\le c\,h(t)\]
and
\[g(t)\simeq h(t)\iff\exists\,c,\quad g(t)=c\,h(t).\]

This report does not contain any other norms except the $L^p$ norms so that double vertical bars always refer to the $L^p$ norms.
We also omit marginalized variables and the subscript $L$.
For example,
\[\|f(t)\|_p=(\iint|f(t,x,v)|^p\,dv\,dx)^{1/p},\quad\|\rho(t)\|_p=(\int|\rho(r,x)|^p\,dx)^{1/p}.\]
\end{notn*}



\subsection{The Poisson equation}
For the three-dimensional boundaryless problem of the Poisson equation
\[-\Delta\Phi(x)=\rho(x)\]
in which the solution $\Phi$ vanishes at infinity, it is well-known that
\[\Phi=\tfrac1{4\pi|x|}*\rho,\]
so the electric field in the Vlasov-Poisson system is given by
\[E=-\nabla_x\Phi=-\nabla_x(\tfrac1{4\pi|x|}*\rho)=\frac{x}{4\pi|x|^3}*\rho.\]
It can be rewritten as
\[E(t,x)=\frac1{4\pi}\int\frac{(x-y)\rho(t,y)}{|x-y|^3}\,dy.\]

The nonlinearity of the system is originated from the force field $E$, so its estimates play a crucial role in study of the nonlinear system.
Since it is given by the solution of the Poisson equation, estimates of the Riesz potential, the convolution with a kernel of the form $|x|^{-(d-\alpha)}$, are directly connected to estimates of the force field.

\begin{lem}[Estimates of Riesz potential]
Let $\rho\in C_c^1(\R^d)$.
\begin{parts}
\item
There is a field estimate
\[\|\tfrac1{|x|^{d-1}}*\rho\|_\infty\lesssim\|\rho\|_\infty^{1-1/d}\|\rho\|_1^{1/d}.\]
\item
For $\log^+(x):=\max\{0,\log x\}$, we have an estimate of derivative of the field
\[\|\nabla(\tfrac1{|x|^{d-1}}*\rho)\|_\infty\lesssim1+\|\rho\|_\infty\log^+\|\nabla\rho\|_\infty+\|\rho\|_1.\]
\end{parts}
\end{lem}

\begin{pf}
(a)
Let $0\le\frac1p<\frac\alpha d<\frac1q\le1$.
Since $(d-\alpha)p<d<(d-\alpha)q$,
\begin{align*}
|\tfrac1{|x|^{d-\alpha}}*\rho|
&=\int_{|x-y|<R}\frac{\rho(y)}{|x-y|^{d-\alpha}}\,dy+\int_{|x-y|\ge R}\frac{\rho(y)}{|x-y|^{d-\alpha}}\,dy\\
&\le\|\rho\|_{p'}(\int_{|y|<R}\frac{dy}{|y|^{(d-\alpha)p}})^{1/p}+\|\rho\|_{q'}(\int_{|y|\ge R}\frac{dy}{|y|^{(d-\alpha)q}})^{1/q}\\
&\simeq\|\rho\|_{p'}(\int_0^Rr^{d-1-(d-\alpha)p}\,dr)^{1/p}+\|\rho\|_{q'}(\int_R^\infty r^{d-1-(d-\alpha)q}\,dr)^{1/q}\\
&\simeq\|\rho\|_{p'}R^{\frac dp-d+\alpha}+\|\rho\|_{q'}R^{\frac dq-d+\alpha}.
\end{align*}
By choosing $R$ such that $\|\rho\|_{p'}R^{\frac dp-d+\alpha}=\|\rho\|_{q'}R^{\frac dq-d+\alpha}$, we get
\[\|\tfrac1{|x|^{d-\alpha}}*\rho\|_\infty\lesssim\|\rho\|_{p'}^{\frac{1-\frac\alpha d-\frac1q}{\frac1p-\frac1q}}\|\rho\|_{q'}^{\frac{\frac1p-1+\frac\alpha d}{\frac1p-\frac1q}},\]
so the inequality
\[\|\tfrac1{|x|^{d-\alpha}}*\rho\|_\infty^{\frac1q-\frac1p}\lesssim\|\rho\|_p^{\frac1q-\frac\alpha d}\|\rho\|_q^{\frac\alpha d-\frac1p}\]
is obtained by interchaning $p$ and $q$ with their conjugates.
The desired result gets $p=\infty$, $\alpha=1$, and $q=1$.

(b)
Let $0<R_a\le R_b$ be constants which will be determined later.
Divide the region radially
\begin{align*}
|\nabla(\tfrac1{|x|^{d-1}}*\rho)|\lesssim\nabla\int_{|x-y|<R_a}+\nabla\int_{R_a\le|x-y|<R_b}+\nabla\int_{R_b\le|x-y|}.
\end{align*}
For the first integral,
\begin{align*}
\int_{|y|<R_a}\frac{\nabla\rho(x-y)}{|y|^{d-1}}\,dy
&\le\|\nabla\rho\|_\infty\int_{|y|<R_a}\frac1{|y|^{d-1}}\,dy\\
&\simeq\|\nabla\rho\|_\infty\int_0^{R_a}1\,dr
=R_a\|\nabla\rho\|_\infty.
\end{align*}
For the second integral,
\begin{align*}
\int_{R_a\le|x-y|<R_b}\frac{\rho(y)}{|x-y|^d}\,dy
&\le\|\rho\|_\infty\int_{R_a\le|x-y|<R_b}\frac1{|x-y|^d}\,dy\\
&\simeq\|\rho\|_\infty\int_{R_a}^{R_b}\frac1r\,dr
=(\log\tfrac{R_b}{R_a})\|\rho\|_\infty.
\end{align*}
For the third integral,
\[\int_{R_b\le|x-y|}\frac{\rho(y)}{|x-y|^d}\,dy\le R_b^{-d}\|\rho\|_1.\]
Thus,
\[|\nabla(\tfrac1{|x|^{d-1}}*\rho)|\lesssim R_a\|\nabla\rho\|_\infty+(\log\tfrac{R_b}{R_a})\|\rho\|_\infty+R_b^{-d}\|\rho\|_1.\]

Assuming $\rho$ is nonzero so that $\|\nabla\rho\|_\infty>0$, let $R_a=\min\{1,\|\nabla\rho\|_\infty^{-1}\}$ and $R_b=1$.
Since
\[\log\tfrac1{R_a}\le\log^+\|\nabla\rho\|_\infty\quad\text{and}\quad R_a\lesssim\|\nabla\rho\|_\infty,\]
we have
\[\|\nabla(\tfrac1{|x|^{d-1}}*\rho)\|_\infty\lesssim1+\|\rho\|_\infty\log^+\|\nabla\rho\|_\infty+\|\rho\|_1.\qedhere\]
\end{pf}

\subsection{Characteristics and volume preservation}

The Vlasov-Poisson equation is quite hyperbolic.
What we mean here is that the method of characteristic curves is useful, and it does not regularizes the solution directly.
Although we do not have an explicit representation formula, solutions written by characteristic curves make appropriate estimates possible.

Moreover, since the Vlasov-Poisson system is a Hamiltonian system on the phase space $\R_x^3\times\R_v^3$ with the Hamiltonian $H(x,v)=\frac12v^2+\gamma\Phi(x,v)$, it has the volume preserving propoerty.
We, however, will show the volume preservation by computation of the Jacobian determinant for coordinates transformations through characteristic flows.

\begin{lem}
Let $f\in C^1([0,T];C_c^1(\R^6))$ be a classical solution of the Vlasov-Poisson system.
\begin{parts}
\item Fix $t,x,v$. The system of ordinary differential equations
\[\left\{\begin{aligned}
\dot X(s;t,x,v)&=V(s;t,x,v),\\
\dot V(s;t,x,v)&=\gamma E(s,X(s;t,x,v)),\\
X(t;t,x,v)&=x,\qquad V(t;t,x,v)=v,
\end{aligned}\right.\]
where the dot symbol denote the time derivative $\dd{s}$, has a solution $(X,V)$ in $C^1([0,T],\R^6)$.
\item Fix $t,x,v$. Then, $f(s,X(s;t,x,v),V(s;t,x,v))=\const$.
\item Fix $t$, and let
\[y(s;x,v):=X(s;t,x,v)\quad\text{and}\quad w(s;x,v):=V(s;t,x,v).\]
Then, the Jacobian of coordinates transform $(x,v)\mapsto(y,w)$ is 1 for all $s$.
\end{parts}
\end{lem}
\begin{pf}
(a)
Note that we have
\[\rho\in C^1([0,T];C_c^1(\R^6)),\quad\Phi\in C^1([0,T];C^{2,\alpha}(\R^6))\]
so that
\[E\in C^1([0,T];C^{1,\alpha}(\R^6))\]
by the H\"older regularity of the Poisson equation.
Since a map
\[(x,v)\mapsto(v,\gamma E(t,x))\]
is globally Lipschitz with respect to $(x,v)$ for each $t$, we can apply the Picard-Lindel\"of theorem.

(b)
Differentiate and use the chain rule to get
\begin{align*}
\dd{s}&f(s,y(s),w(s))\\
&=\pd_tf(s,y,w)+\dot X(s;s,y,w)\cdot\nabla_xf(s,y,w)+\dot V(s;s,y,w)\cdot\nabla_vf(s,y,w)\\
&=\pd_tf(s,y,w)+w\cdot\nabla_xf(s,y,w)+\gamma E(s,y)\cdot\nabla_vf(s,y,w)=0,
\end{align*}
where we denote $y(s)=X(s;t,x,v)$ and $w(s)=V(s;t,x,v)$.

(c)
Let $J(s)=\pd{(y(s),w(s))}{(x,v)}$ be the Jacobi matrix.
Because when $s=t$ the Jacobian is
\[\det J(t)=\det\pd{(x,v)}{(x,v)}=1,\]
we want to show
\[\det J(s)=\const.\]
Since
\begin{align*}
J^{-1}(s)\dd{s}J(s)
&=\pd{(x,v)}{(y(s),w(s))}\dd{s}\pd{(y(s),w(s))}{(x,v)}\\
&=\pd{(\dot y(s),\dot w(s))}{(y(s),w(s))}
=\pd{(w(s),\gamma E(s,y(s)))}{(y(s),w(s))}
=\mat{0&1\\\gamma\pd{E}{y}(s,y(s))&0},
\end{align*}
the Jacobi formula deduces that
\[\dd{s}\det J(s)=\det(s)\tr\left(J^{-1}(s)\dd{s}J(s)\right)=0.\qedhere\]
\end{pf}

\begin{cor}
Let $f\in C^1([0,T];C_c^1(\R^6))$ be a classical solution of the Cauchy problem for the Vlasov-Poisson system.
Then, for any measurable function $\beta:\R\to\R$ such that $\iint\beta\circ f_0(x,v)\,dv\,dx<\infty$, we have
\[\iint\beta\circ f(t,x,v)\,dv\,dx=\const.\]
In particular,
\[\|f(t)\|_p=\const\]
for $1\le p\le\infty$.
\end{cor}
\begin{pf}
Fix $t,s\in[0,T]$ and denote $y=X(s;t,x,v)$ and $w=V(s;t,x,v)$.
Then,
\begin{align*}
\iint\beta\circ f(t,x,v)\,dv\,dx
&=\iint\beta\circ f(s,y(s),w(s))\,dv\,dx\\
&=\iint\beta\circ f(s,y,w)\,dw\,dy
\end{align*}
for $s\le T$.
\end{pf}

To sum up our weapons obtained so far, we have the following.
\begin{cor}
If a function $f\in C^1([0,T],C_c^1(\R^6))$ satisfies
\[\iint f(t,x,v)\,dv\,dx=\const,\]
and if we let
\[\rho(t,x)=\int f(t,x,v)\,dv,\quad E(t,x)=\frac1{4\pi}\int\frac{(x-y)\rho(t,y)}{|x-y|^3}\,dy,\]
then
\begin{parts}
\item $\|\rho(t)\|_1=\const$,
\item $\|E(t)\|_\infty\lesssim\|\rho(t)\|_\infty^{2/3}$,
\item $\|\nabla E(t)\|_\infty\lesssim1+\|\rho\|_\infty\log^+\|\nabla\rho\|_\infty$.
\end{parts}
\end{cor}
These estimates will be applied not only to the global existence proof, which assumes the local existence, but also to approximate solutions.

\begin{rmk}
Note that the volume preservation is also yielded for a approximation scheme, which will be suggested in the next section, hence the same results in Corollary 1.4 for the approximate solutions in the same manner.
The proof will be omitted.
\end{rmk}


\subsection{Conservation laws and moment propagation}
Usual algebraic computations with Stokes' theorem get several conservations laws, particularly including energy conservation.

\begin{lem}
Let $f$ be a classical solution of the Vlasov-Poisson system.
\begin{parts}
\item(Continuity equation)
\[\rho_t+\nabla_x\cdot j=0,\quad\text{where}\quad j=\int vf\,dv.\]
\item(Energy conservation)
\[\iint|v|^2f\,dv\,dx+\gamma\int|E|^2\,dx=\const.\]
\end{parts}
\end{lem}
\begin{pf}
(a)
Integrate with respect to $v$ to get
\begin{align*}
0&=\int f_t\,dv+\int v\cdot\nabla_xf\,dv\\
&=\rho_t+\nabla_x\cdot\int vf\,dv\\
&=\rho_t+\nabla_x\cdot j.
\end{align*}

(b)
Multiply $|v|^2$ and integrate with respect to $v$ and $x$ to get
\begin{align*}
\dd{t}\iint|v|^2f\,dv\,dx
&=\iint|v|^2f_t\,dv\,dx=-\iint|v|^2\gamma E\cdot\nabla_vf\,dv\,dx\\
&=\iint2v\cdot\gamma Ef\,dv\,dx=-2\gamma\int\nabla_x\Phi\cdot j\,dx\\
&=2\gamma\int\Phi\nabla_x\cdot j\,dx=2\gamma\int\Phi\Delta_x\Phi_t\,dx\\
&=-\dd{t}\gamma\int|E|^2\,dx.
\end{align*}
Thus
\[\iint|v|^2f\,dv\,dx+\gamma\int|E|^2\,dx=\const.\qedhere\]
\end{pf}

Kinetic energy is a type of quantities which are called moments;
we call the quantities of the form
\[\iint|v|^kf(t,x,v)\,dv\,dx\]
\emph{moments}, with a positive real $k$.
The energy conservation proves the bound of the 2-moment, kinetic energy,
\[\iint|v|^2f(t,x,v)\,dv\,dx\lesssim1\]
if $\gamma=+1$.
In fact, a bound of kinetic energy exists even for $\gamma=-1$.
As a corollary, the $L^{5/3}$ norm of mass density $\|\rho\|_{5/3}$ gets bounded.

\begin{lem}[Bound for kinetic energy]
Let $f\in C^1([0,T],C_c^1(\R^6))$ be a solution of the Vlasov-Poisson system.
For $t\in[0,T]$,
\begin{parts}
\item $\|\rho(t)\|_{5/3}^{5/3}\lesssim\iint|v|^2f\,dv\,dx$.
\item $\iint|v|^2f\,dv\,dx\lesssim1$.
\end{parts}
\end{lem}
\begin{pf}
(a)
Note
\begin{align*}
\rho(t,x)=\int f(t,x,v)\,dv
&\le\int_{|v|<R}f\,dv+\frac1{R^2}\int_{|v|\ge R}|v|^2f\,dv\\
&\lesssim R^3+ R^{-2}\int|v|^2f\,dv.
\end{align*}
Set $R^3=R^{-2}\int|v|^2f\,dv$ to get
\[\rho(t,x)^{5/3}\lesssim\int|v|^2f\,dv.\]

(b)
It is trivial for $\gamma=+1$ from the energy conservation.
Suppose $\gamma=-1$.
By the Hardy-Littlewood-Sobolev inequality,
\[\frac1p+\frac\alpha d=\frac1q\]
for $p=2$, $d=3$, and $\alpha=1$ implies $q=6/5$, hence the bound of $\|E(t)\|_2$
\[\|E(t)\|_2\simeq\|\frac1{|x|^{d-\alpha}}*_x\rho(t,x)\|_{L_x^2}\lesssim\|\rho(t)\|_{6/5}.\]
So, interpolation with H\"older's inequality gives
\[\|E(t)\|_2\lesssim\|\rho\|_1^{7/12}\|\rho\|_{5/3}^{5/12}\simeq\|\rho\|_{5/3}^{5/12}.\]
Thus (1) gives
\[\iint|v|^2f\,dv\,dx=c+\|E(t)\|_2^2\lesssim c+(\iint|v|^2f\,dv\,dx)^{1/2},\]
so the kinetic energy $\iint f\,dv\,dx$ is bounded.\qedhere
\end{pf}

\begin{rmk}
If we had a bound of higher moment
\[\iint|v|^kf(t,x,v)\,dv\,dx\lesssim1\]
for some $k>6$ so that $\|\rho(t)\|_p\lesssim1$ for some $p=\frac{k+3}3>3$, then we would obtain
\[\|E(t)\|_\infty^{1-\frac1p}\lesssim\|\rho\|_p^{\frac23}\|\rho\|_1^{\frac13-\frac1p}\lesssim1.\]
We will see that this estimate proves the global existence immediately; this is the idea of the paper of Lions and Perthame \cite{lions1991propagation}.
We do not cover this in detail.
\end{rmk}




\section{Local existence}

The proof of local existence follows the following steps:
\begin{parts}
\item construction of an approximate solution,
\item establishment of estimates,
\item (subsequential) convergence of the approximate solution,
\item verification of the solvability for the limit.
\end{parts}
The Vlasov-Poisson system is good enough that we can show the usual convergence of approximate solutions, not in the sense of subsequences.


\subsection{Approximate solution}

\begin{defn}
We define an (global) \emph{approximate solution} as a sequence of functions $f_n\in C^1(\R^+;C_c^1(\R^6))$ such that
\[\left\{\ \begin{alignedat}{2}
&\pd_tf_{n+1}+v\cdot\nabla_xf_{n+1}+\gamma E_n\cdot\nabla_vf_{n+1}=0,\\
&E_n(t,x)=-\nabla_x\Phi_n,\\
&\Phi_n(t,x)=\int\frac{\rho_n(y)}{|x-y|}\,dy,\\
&\rho_n(t,x)=\int f_n\,dv,\\
&f_{n+1}(0,x,v)=f_0(x,v).
\end{alignedat}\right.\]
This definition is made in order to let the force field $E$ constant when solving $f_{n+1}$.
Note that it assumes for $f_0$ to be automatically $C_c^1$ by definition.
\end{defn}


\begin{prop}
An approximate solution exists for given initial term $f_0\in C_c^1(\R^6)$.
\end{prop}
\begin{pf}
Let $f_0(t,x,v)=f_0(x,v)$.
Notice that $f_0$ is clearly in $C^1(\R^+;C_c^1(\R^6))$.
Assume $f_n\in C^1(\R^+;C_c^1(\R^6))$ satisfies the approximate system.
We want to show that there is $f_{n+1}$ that satisfies the approximate system and $f_{n+1}\in C^1(\R^+;C_c^1(\R^6))$.

We have for $0<\alpha<1$ that
\[\rho_n\in C^1(\R^+;C_c^1(\R^6)),\quad\Phi_n\in C^1(\R^+;C^{2,\alpha}(\R^6)),\ \text{and}\ E_n\in C^1(\R^+;C^{1,\alpha}(\R^6))\]
by the H\"older regularity of the Poisson equation.
Since a map $(x,v)\mapsto(v,\gamma E_n(t,x))$ is globally Lipschitz with respect to $(x,v)$ for each $t$, the classical Picard iteration uniquely solves the characteristic equation
\[\left\{\ \begin{alignedat}{2}
\dot X_{n+1}(s;t,x,v)&=V_{n+1}(s,t,x,v)\\
\dot V_{n+1}(s;t,x,v)&=\gamma E_n(s,X_{n+1}(s;t,x,v))
\end{alignedat}\right.\]
with condition $(X_{n+1}(t;t,x,v),V_{n+1}(t;t,x,v))=(x,v)$, and proves the uniqueness and regularity of the solution $s\mapsto(X_{n+1},V_{n+1})(s;t,x,v)\in C^1(\R^+,\R^6)$.

Define
\[f_{n+1}(t,x,v):=f_0(X_{n+1}(0;t,x,v),V_{n+1}(0;t,x,v)).\]
Then, $f_{n+1}$ is clearly $C^1$, and we can show that
\begin{align*}
f_{n+1}(s,X_{n+1}(s;t,x,v),V_{n+1}(s;t,x,v))&\\
=f_0(X_{n+1}(0;t,x,v),V_{n+1}(0;t,x,v))&=\const
\end{align*}
and that $f_{n+1}$ satisfies the approximate system by the chain rule
\begin{align*}
0&=\left.\dd{s}f_{n+1}(s,X_{n+1}(s;t,x,v),V_{n+1}(s;t,x,v))\right|_{s=t}\\
&=\pd_tf_{n+1}(t,x,v)+\dot X_{n+1}(t;t,x,v)\cdot\nabla_xf_{n+1}(t,x,v)+\dot V_{n+1}(t;t,x,v)\cdot\nabla_vf_{n+1}(t,x,v)\\
&=\pd_tf_{n+1}(t,x,v)+v\cdot\nabla_xf_{n+1}(t,x,v)+\gamma E_n(t,x)\cdot\nabla_vf_{n+1}(t,x,v).
\end{align*}
Also, $f_{n+1}$ has compact support for each $t$ since the characteristic does not blow up; finally we have $f_{n+1}\in C^1(\R^+,C_c^1(\R^6))$.
\end{pf}
\begin{rmk}
Although the approximate solution is unique when given the initial term $f_0(t,x,v)=f_0(x,v)$, we do not care of its uniqueness, but only the existence.
\end{rmk}

In this section, we fix an approximate solution $f_n$.


\subsection{Local estimates on approximate solutions}
Recall that the characteristic curves of $f_n$ are solutions of the system
\[\left\{\ \begin{alignedat}{2}
\dot X_{n+1}(s;t,x,v)&=V_{n+1}(s;t,x,v)\\
\dot V_{n+1}(s;t,x,v)&=\gamma E_n(s,X_{n+1}(s;t,x,v)).
\end{alignedat}\right.\]
Firstly, the volume preserving property still holds for our approximate system.
Therefore, we have
\[\|\rho_n(t)\|_1=\const,\quad\|f_n(t)\|_p=\const.\]
Next, we prove local-time bounds on fields $E_n$ and its spatial derivative $\nabla_xE_n$.
The bounds crucially act in the proof of convergence of $f_n$.
Note that $\nabla_xE_n$ is a gradient of a vector field $E_n$, which is 9-dimensional.
Introduce the following quantity.
\begin{defn}
Define the \emph{velocity support} or \emph{maximal velocity} by
\begin{align*}
Q_n(t):&=\sup\{\,|v|:\exists s\in[0,t],\ f_n(s,x,v)\ne0\,\}\\
&=\sup\{\,|V_n(s;0,x,v)|:s\in[0,t],\ f_0(x,v)\ne0\,\}.
\end{align*}
In particular, $Q_0$ is independent on $t$.
\end{defn}

\begin{lem}
Let $T>0$ be a constant such that
\[T<(Q_0\|f_0\|_\infty^{2/3}\|f_0\|_1^{1/3})^{-1}.\]
Then, we have the following bounds:
\begin{parts}
\item
For $t\le T$,
\[\|\rho_n(t)\|_\infty+\|E_n(t)\|_\infty+Q_n(t)\lesssim1\]
independently on $n$.
In addition, the position support $|X_n(t;0,x,v)|$ is also uniformly bounded in time $t\le T$.
\item
For $t\le T$,
\[\|\nabla_x\rho_n(t)\|_\infty+\|\nabla_xE_n(t)\|_\infty\lesssim1\]
independently on $n$.
\end{parts}
\end{lem}
The control mechanism among uniform norms of each quantity including $\rho$ and $E$ can be summarized as follows:
\begin{rd}[column sep=huge]
$\log\|E(t)\|_\infty$ \rar[symbol=\lesssim,swap]{\parbox{10em}{\centering\quad\\[10pt]Elliptic regularity of\\Poisson's eqn}}&
$\log\|\rho(t)\|_\infty$ \rar[symbol=\lesssim]&
$\log Q(t)$,
\end{rd}
and
\begin{rd}
$Q(t)$ \rar[symbol=\lesssim,swap]{\parbox{10em}{\centering\quad\\[10pt]By def}}&
$|(X,V)(t)|$ \rar[symbol=\lesssim,swap]{\parbox{10em}{\centering\quad\\[10pt]Equations of\\characteristics}}&
$\int_0^t(1+\|E(s)\|_\infty)\,ds$.
\end{rd}
Then, Gronwall's inequality saves the game for the bound of $Q$.
Also, we can observe that all functions in here are controlled by the velocity support $Q$.
For detail explanations, see the following proof.

\begin{pf}
(a)
Since
\[\|\rho_n(t)\|_\infty\le Q_n^3(t)\|f_0\|_\infty,\]
a rough estimate for $\|E\|_\infty$ gives
\[\|E_n(t)\|_\infty\le\|\rho_n(t)\|_\infty^{2/3}\|\rho_n(t)\|_1^{1/3}\le Q_n^2(t)\|f_0\|_\infty^{2/3}\|f_0\|_1^{1/3}.\]
Let $c(f_0):=\|f_0\|_\infty^{2/3}\|f_0\|_1^{1/3}$ so that $\|E_n(t)\|\le cQ_n^2(t)$ and $cQ_0t<1$ for $t\le T$.
We claim that if $t\le T$, then
\[Q_n(t)\le\frac{Q_0}{1-cQ_0t}\]
for all $n$.
Easily checked for $n=0$; $Q_0(t)\equiv Q_0\le Q_0/(1-cQ_0t)$.

Assume $Q_n(t)\le\frac{Q_0}{1-cQ_0t}$ for $t\le T$.
Let $f_0(x,v)\ne0$.
Then,
\begin{align*}
|V_{n+1}(t;0,x,v)|
&=\Bigl|v+\int_0^t\gamma E_n(s,X_{n+1}(s;0,x,v))\,ds\Bigr|\\
&\le|v|+\int_0^t\|E_n(s)\|_\infty\,ds\\
&\le Q_0+c\int_0^tQ_n^2(s)\,ds
\end{align*}
leads to
\[Q_{n+1}(t)\le Q_0+c\int_0^tQ_n^2(s)\,ds\le Q_0+c\int_0^t\left(\frac{Q_0}{1-cQ_0s}\right)^2ds=\frac{Q_0}{1-cQ_0t}.\]
By induction, $Q_n(t)\le\frac{Q_0}{1-cQ_0t}\lesssim1$ for all $n$ and $t\le T$.
Furthermore,
\[\|\rho_n(t)\|_\infty\lesssim Q_n^3(t)\lesssim1,\quad\|E_n(t)\|_\infty\lesssim Q_n^2(t)\lesssim1.\]
The position support is bounded because
\[|X_n(t;0,x,v)|\le|x|+\int_0^t|V_n(s;0,x,v)|\,ds\le|x|+TQ_n(t)\lesssim1.\]

(b)
Since we already have bounds for $\|\rho_n\|_\infty$ and $\|\rho_n\|_1$, what we should estimate in order to bound $\|\nabla_xE_n\|_\infty$ is $\nabla_x\rho_n$.
To do this, we will consider $\nabla_xX_n$ and $\nabla_xV_n$.
In particular, we have
\begin{align*}
\nabla_xX_n(t;t,x,v)&=\nabla_xx=(1,0,0\,;0,1,0\,;0,0,1),\\
\nabla_xV_n(t;t,x,v)&=\nabla_xv=0.
\end{align*}
Two inequalities
\begin{align*}
|\nabla_xX_{n+1}(s;t,x,v)|
&=\Bigl|\underbrace{(1,0,\cdots,0,1)}_{9}-\int_s^t\nabla_xV_{n+1}(s';t,x,v)\,ds'\Bigr|\\
&\le\sqrt3+\int_s^t|\nabla_xV_{n+1}(s';t,x,v)|\,ds'
\end{align*}
and
\begin{align*}
|\nabla_xV_{n+1}(s;t,x,v)|
&=|\int_s^t\nabla_x[E_n(s',X_{n+1}(s';t,x,v))]\,ds'|\\
&\le\int_s^t|\nabla_xX_{n+1}(s';t,x,v)|\cdot\|\nabla_xE_n(s')\|_\infty\,ds'
\end{align*}
are combined as
\begin{align*}
&|\nabla_xX_{n+1}(s;t,x,v)|+|\nabla_xV_{n+1}(s;t,x,v)|\\
&\quad\le\sqrt3+\int_s^t(1+\|\nabla_xE_n(s')\|_\infty)(|\nabla_xX_{n+1}(s';t,x,v)|+|\nabla_xV_{n+1}(s';t,x,v)|)\,ds'.
\end{align*}
By the Gronwall inequality, we get
\[|\nabla_xX_{n+1}(s;t,x,v)|+|\nabla_xV_{n+1}(s;t,x,v)|\le\sqrt3\,e^{\int_s^t(1+\|\nabla_xE_n(s')\|_\infty)\,ds'}\]
for $0\le s\le t$.
Thus we have
\begin{align*}
|\nabla_x\rho_{n+1}(t,x)|
&=|\int\nabla_x[f_0(X_{n+1}(0;t,x,v),V_{n+1}(0;t,x,v))]\,dv|\\
&\le\|\nabla_{x,v}f_0\|_\infty\int(|\nabla_xX_{n+1}(0;t,x,v)|+|\nabla_xV_{n+1}(0;t,x,v)|)\,dv\\
&\lesssim\|\nabla_{x,v}f_0\|_\infty Q_{n+1}^3(t)\cdot e^{\int_0^t(1+\|\nabla_xE_n(s)\|_\infty)\,ds}
\end{align*}
so that
\[\|\nabla_x\rho_{n+1}(t)\|_\infty\lesssim e^{\int_0^t(1+\|\nabla_xE_n(s)\|_\infty)\,ds}.\]

Recall that
\begin{align*}
\|\nabla_xE_{n+1}(t)\|_\infty
&\lesssim(1+\|\rho_{n+1}(t)\|_\infty\log^+\|\nabla_x\rho_{n+1}(t)\|_\infty+\|\rho_{n+1}(t)\|_1)\\
&\lesssim1+\log^+\|\nabla_x\rho_{n+1}(t)\|_\infty
\end{align*}
for $t\le T$.
By inserting the estimate for $\|\nabla_x\rho_{n+1}(t)\|_\infty$, we can find a constant $c=c(f_0)$ such that
\begin{align*}
1+\|\nabla_xE_{n+1}(t)\|_\infty\le c[1+\int_0^t(1+\|\nabla_xE_n(s)\|_\infty)\,ds]
\end{align*}
in $t\le T$, where $T<(Q_0\|f_0\|_\infty^{2/3}\|f_0\|_1^{1/3})^{-1}$.
Without loss of generality, we may assume that the constant $c$ satisfies
\[\sup_{s\in[0,T]}(1+\|\nabla_xE_0(s)\|_\infty)\le c.\]
Then, induction obtains the bound
\[1+\|\nabla_xE_n(t)\|_\infty\le ce^{ct}\le ce^{cT}\lesssim1\]
for all $n$ and $t\le T$.
\end{pf}



\subsection{Convergence of approximate solution}
Although most of the nonlinear systems fail to have convergent approximate solutions so that compactness methods are often applied, the approximate solutions constructed and investigated in the previous subsections uniformly converges.
\begin{lem}
Let $T>0$ be a constant such that
\[T<(Q_0\|f_0\|_\infty^{2/3}\|f_0\|_1^{1/3})^{-1}.\]
\begin{parts}
\item
For $t\le T$ and $n\ge1$,
\[\|f_{n+1}(t)-f_n(t)\|_\infty\lesssim\int_0^t\|E_n(s)-E_{n-1}(s)\|_\infty\,ds.\]
\item
For $s\le T$ and $n\ge1$,
\[\|E_n(s)-E_{n-1}(s)\|_\infty\lesssim\|f_n(s)-f_{n-1}(s)\|_\infty.\]
\item $f_n$ converges to a function $f$ uniformly in $C([0,T]\times\R^6)$.
\item For each $t,x,v$, a sequence of maps
\[s\mapsto(X_n(s;t,x,v),V_n(s;t,x,v))\]
converges in $C^1([0,T],\R^6)$ so that its limit $(X,V)$ satisfies the equations
\[\dot X(s;t,x,v)=V(s;t,x,v),\quad\dot V(s;t,x,v)=\gamma E(s,X(s;t,x,v)),\]
where
\[E(t,x)=\frac1{4\pi}\iint\frac{(x-y)f(t,y,v)}{|x-y|^3}\,dv\,dy.\]
\end{parts}
\end{lem}
\begin{pf}
(a)
Denote
\[g(s):=|X_{n+1}(s;t,x,v)-X_n(s;t,x,v)|+|V_{n+1}(s;t,x,v)-V_n(s;t,x,v)|\]
for given $t,x,v$.
The $C^1$ regularity of $f_0$ gives
\begin{align*}
|f_{n+1}(t,x,v)-f_n(t,x,v)|
&=|f_0(X_{n+1}(0;t,x,v),V_{n+1}(0;t,x,v))-f_0(X_n(0;t,x,v),V_n(0;t,x,v))|\\
&\lesssim|X_{n+1}(0;t,x,v)-X_n(0;t,x,v)|+|V_{n+1}(0;t,x,v)-V_n(0;t,x,v)|\\
&=g(0).
\end{align*}
If an inequality
\[\sup_{s\in[0,t]}g(s)\lesssim\int_0^t\|E_n(s)-E_{n-1}(s)\|_\infty\,ds\]
is obtained, whose right-hand side does not depend on $x$ nor $v$, then we are done.

Let $0\le s\le t\le T$.
Because
\begin{align*}
X_n(s;t,x,v)&=x-\int_s^tV_n(s';t,x,v)\,ds',\\
V_n(s;t,x,v)&=v-\int_s^t\gamma E_{n-1}(s',X_n(s';t,x,v))\,ds',
\end{align*}
we have two inequalities
\begin{align*}
|V_{n+1}(s;t,x,v)-V_n(s;t,x,v)|
&\le\int_s^t|E_n(s',X_{n+1}(s';t,x,v))-E_{n-1}(s',X_n(s';t,x,v))|\,ds'\\
&\le\int_s^t(|E_n(s',X_{n+1})-E_n(s',X_n)|+|E_n(s',X_n)-E_{n-1}(s',X_n)|)\,ds'\\
&\le\int_s^t(\|\nabla_xE_n(s')\|_\infty|X_{n+1}(s')-X_n(s')|+\|E_n(s')-E_{n-1}(s')\|_\infty)\,ds'
\end{align*}
and
\begin{align*}
|X_{n+1}(s;t,x,v)-X_n(s;t,x,v)|\le\int_s^t|V_{n+1}(s';t,x,v)-V_n(s';t,x,v)|\,ds'
\end{align*}
for $s\in[0,t]$.
By combining the two inequalities above, we get
\begin{align}\label{ggw}
g(s)\le\int_s^ta(s')g(s')\,ds'+\int_s^t\|E_n(s')-E_{n-1}(s')\|_\infty\,ds',
\end{align}
where $a(s):=1+\|\nabla_xE_n(s)\|_\infty$.

Here we use a Gronwall-type inequality to bound $g(s)$.
In more detail, multiplying
\[a(s)e^{-\int_s^ta(s')ds'}\]
on the both-hand-side of (\ref{ggw}), and using $a\lesssim 1$ in $t\le T$, we have
\begin{align*}
-\dd{s}\left(e^{-\int_s^ta(s')\,ds'}\int_s^ta(s')g(s')\,ds'\right)
&\le a(s)e^{-\int_s^ta(s')ds'}\int_s^t\|E_n(s')-E_{n-1}(s')\|_\infty\,ds'\\
&\lesssim\int_s^t\|E_n(s')-E_{n-1}(s')\|_\infty\,ds'
\end{align*}
Integrate from $s$ to $t$ and bound $(t-s)\le T\lesssim1$ to get
\begin{align}\label{abd}
e^{-\int_s^ta(s')\,ds'}\int_s^ta(s')g(s')\,ds'\lesssim\int_s^t\|E_n(s')-E_{n-1}(s')\|_\infty\,ds'.
\end{align}
Since $e^{\int_s^ta(s')\,ds'}\le e^{T\sup_{s\in[0,t]}a(s)}\lesssim1$, the inequalities (\ref{ggw}) and (\ref{abd}) implies
\begin{align}\label{rii}
g(s)\lesssim\int_0^t\|E_n(s')-E_{n-1}(s')\|_\infty\,ds'
\end{align}
for arbitrary $s\in[0,t]$.

(b)
Notice that
\[\|E_n(t)-E_{n-1}(t)\|_\infty\lesssim\|\rho_n(t)-\rho_{n-1}(t)\|_1^{1/3}\|\rho_n(t)-\rho_{n-1}(t)\|_\infty^{2/3}.\]
For $L^\infty$-norm,
\[\|\rho_n(t)-\rho_{n-1}(t)\|_\infty\le\max\{Q_n^3(t),Q_{n-1}^3(t)\}\|f_n(t)-f_{n-1}(t)\|_\infty\lesssim\|f_n(t)-f_{n-1}(t)\|_\infty.\]
For $L^1$-norm, since the support of $f_n,f_{n-1}$ is bounded in both directions $x,v$ in finite time,
\[\|\rho_n(t)-\rho_{n-1}(t)\|_1\le\|f_n(t)-f_{n-1}(t)\|_1\lesssim\|f_n(t)-f_{n-1}(t)\|_\infty\]
for $t\le T$, where $T<\infty$ arbitrary.

(c)
From (a) and (b), there is a constant $c=c(f_0)$ such that,
\[\|f_{n+1}(t)-f_n(t)\|_\infty\le c\int_0^t\|f_n(s)-f_{n-1}(s)\|_\infty\,ds.\]
We can easily get with induction
\[\|f_{n+1}(t)-f_n(t)\|_\infty\le M\frac{(ct)^n}{n!},\]
where $M=\sup_{s\in[0,T]}\|f_1(s)-f_0(s)\|_\infty$.
Therefore,
\[\sum_{n=0}^\infty\|f_{n+1}-f_n\|_\infty\le\sup_{t\in[0,T]}Me^{ct}<\infty\]
implies $f_n$ uniformly converges in $C([0,T]\times\R^6)$.

(d)
Write
\[X_n(s)=X_n(s;t,x,v),\qquad V_n(s)=V_n(s;t,x,v)\]
for given $t,x,v$.
Recall that $g$ measures the difference between $(X_{n+1},V_{n+1})$ and $(X_n,V_n)$.
By the inequality $(\ref{rii})$ and the result in (b),
\begin{align*}
\sup_{s\in[0,T]}g(s)\lesssim\int_0^T\|E_n(s)-E_{n-1}(s)\|_\infty\lesssim T\|f_n-f_{n-1}\|_\infty.
\end{align*}
Then, the uniform convergence of characteristics $(X_n,V_n)$ is clear by the absolute convergence of the series $\sum\|f_{n+1}-f_n\|_\infty$.

Also for the uniform convergence of $(\dot X_n,\dot V_n)$, it is proved by the absolute convergence of the series $\sum\|f_{n+1}-f_n\|_\infty$ since
\begin{align*}
\|\dot X_{n+1}-\dot X_n\|_\infty&=\|V_{n+1}-V_n\|_\infty,\\
\|\dot V_{n+1}-\dot V_n\|_\infty&\le\|\nabla_xE_n\|_\infty\|X_{n+1}-X_n\|_\infty+\|E_n-E_{n-1}\|_\infty,
\end{align*}
yielding
\[\|\dot X_{n+1}-\dot X_n\|_\infty+\|\dot V_{n+1}-\dot V_n\|_\infty\lesssim\|f_n-f_{n-1}\|_\infty.\]
Then, by limiting the both-hand-side of equations
\[\dot X_n(s)=V_n(s),\qquad \dot V_n(s)=\gamma E_{n-1}(s,X_n(s)),\]
we easily get
\[\dot X(s)=V(s),\qquad \dot V(s)=\gamma E(s,X(s)).\qedhere\]
\end{pf}

\begin{thm}[Local existence]
Let $f_n$ be an approximate solution.
Then, there is a constant $T=T(f_0)>0$ be a constant such that the limit $f$ of $f_n$ is a classical solution of the Cauchy problem for the Vlasov-Poisson system with time domain $[0,T]$.
\end{thm}
\begin{pf}
Take $T$ such that $T<(Q_0\|f_0\|_\infty^{2/3}\|f_0\|_1^{1/3})^{-1}$.
Let $X(s;t,x,v)$ and $V(s;t,x,v)$ be the limits of $X_n(s;t,x,v)$ and $V_n(s;t,x,v)$ for given $t,x,v$.
Notice that
\[f(t,x,v)=\lim_{n\to\infty}f_n(t,x,v)=\lim_{n\to\infty}f_0(X_n(0;t,x,v),V_n(0;t,x,v))=f_0(X(0;t,x,v),V(0;t,x,v)),\]
which shows $f$ is $C^1$ since $f_0$ and $(X,V)$ are $C^1$.
We can check it solves the system by expand the right-hand-side of
\[0=\dd{s}f(s,X(s;t,x,v),V(s;t,x,v))|_{s=t}\]
using the chain rule.
The compact support is by the fact that characteristic curves do not blow up.
\end{pf}


\subsection{Uniqueness}
\begin{thm}[Uniqueness]
Suppose $f_1,f_2\in C^1([0,T];C_c^1(\R^6))$ are classical solutions of the Cauchy problem for the Vlasov-Poisson system with a common initial data $f_0$.
Then, $f_1=f_2$.
\end{thm}
\begin{pf}
As we did in (a) and (b) of Lemma 2.3, we can obtain
\[\|f_1(t)-f_2(t)\|_\infty\lesssim\int_0^t\|f_1(s)-f_2(s)\|_\infty\,ds\]
for $t\le T$.
By the Gronwall lemma, we get
\[\int_0^t\|f_1(s)-f_2(s)\|_\infty\le0.\qedhere\]
\end{pf}



\subsection{Prolongation criterion}
We give in this last subsection a sufficient condition for a local classical solution $f$ to be global.
\begin{defn}
Let $f\in C^1([0,T];C_c^1(\R^6))$.
Define for $t\in[0,T]$
\[Q(t):=\sup\{\,|v|:\exists s\in[0,t],\ f(s,x,v)\ne0\,\}.\]
\end{defn}
\begin{prop}
Let $f\in C^1([0,T];C_c^1(\R^6))$ be a classical solution of the Cauchy problem for the Vlasov-Poisson system.
If $Q(T)<\infty$, then $f$ is continued to a classical solution with a longer time interval.
\end{prop}
\begin{pf}
We are going to apply the local existence result for a new problem, in which we write $\tilde f$ for the solution, with initial condition $\tilde f(0,x,v):=f(t_0,x,v)$ for some $t_0<T$.
In Section 2.3, we have shown the length of time interval for existence $T$ is given by the condition
\[T<(Q_0\|f_0\|_\infty^{2/3}\|f_0\|_1^{1/3})^{-1}.\]
It means that, if we arrange it for the new solution $\tilde f$, the interval of existence of $\tilde f$ has in fact a lower bound $\tilde T>0$ that depends only on $Q(T)$ for any new initial time $t_0$; it is because the monotonicity of $Q$ says $Q(T)^{-1}<Q(t_0)^{-1}$ and the volume preservation implies $\|f_0\|_\infty=\|f(t_0)\|_\infty$ and $\|f_0\|_1=\|f(t_0)\|_1$.
In other words, we can take any $\tilde T$ such that
\[\tilde T<(Q(T)\|f_0\|_\infty^{2/3}\|f_0\|_1^{1/3})^{-1}.\]
Note that the bound does not depend on $t_0$ but only on its upper bound $T$.

Set $t_0=T-\frac12\tilde T$ so that $t_0<T<t_0+\tilde T$.
Then, we can construct a new solution in $C^1([0,t_0+\tilde T],C_c^1(\R^6))$ by pasting solutions $f\in C^1([0,T],C_c^1(\R^6))$ and $\tilde f\in C^1([t_0,t_0+\tilde T],C_c^1(\R^6))$.
\end{pf}

\begin{cor}
If the classical solution $f\in C^1([0,T];C_c^1(\R^6))$ with a given initial data $f_0\in C_c^1(\R^6)$ satisfies $Q(t)\le h(t)$ in $t\le T$ for a continuous function $h:[0,\infty)\to[0,\infty)$, then Theorem 1.1 is true.
\end{cor}
\begin{pf}
Suppose $f\in C^1([0,T_{max});C_c^1(\R^6))$ for $T_{max}<\infty$ is the maximal solution with initial data $f_0$.
Since $Q$ is bounded on $[0,T_{max}]$, we can apply the previous proposition, which contradicts to the maximality of $T_{\max}$.
Hence $T_{max}=\infty$, and the solution $f$ is prolonged forever.
\end{pf}












\section{Global existence}



\begin{thm*}[Schaeffer, 1991]
Let $f_0\in C_c^1(\R^6,[0,\infty))$ and $p>\frac{33}{17}$.
The classical solution $f\in C^1([0,T];C_c^1(\R^6))$ of the Cauchy problem for the Vlasov-Poisson system with an initial data $f_0$ has a constant $c=c(f_0,p)$ such that
\[Q(t)\le c(1+t)^p\]
for all $t\le T$.
\end{thm*}

\subsection{Time averaging}
Fix a (local) classical solution $f$.
If we had an integral inequality of the form
\[Q(t)-Q(t-\Delta)\lesssim\int_{t-\Delta}^tQ(s)^a\,ds\]
for some constant $0\le a\le1$, then we would be able to prove that
\begin{align}\label{gjg}
Q(t)\lesssim\begin{cases}(1+t)^{\frac1{1-a}}&,0\le a<1\\e^{ct}&,a=1\end{cases}
\end{align}
using the nonlinear Gronwall lemma.
To obtain this integral inequality, we may try as follows: if we got an estimate on the field
\[\|E(t)\|_\infty\lesssim Q(t)^a,\]
then for any fixed $t,\hat x,\hat v$ such that $f(t,\hat x,\hat v)\ne0$ and for any $\Delta>0$ we have
\[|\hat v-V(t-\Delta;t,\hat x,\hat v)|=|\int_{t-\Delta}^t\gamma E(s,X(s;t,\hat x,\hat v))\,ds|\lesssim\int_{t-\Delta}^tQ(s)^a\,ds,\]
so there would be a constant $c=c(f_0)$ such that
\[|\hat v|\le|V(t-\Delta;t,\hat x,\hat v)|+c\int_{t-\Delta}^tQ(s)^a\,ds\le Q(t-\Delta)+c\int_{t-\Delta}^tQ(s)^a\,ds,\]
which deduces
\[Q(t)\le Q(t-\Delta)+c\int_{t-\Delta}^tQ(s)^a\,ds.\]

However, an optimized modification of the estimate in (a) of Lemma 1.2 that uses $\|\rho(t)\|_{5/3}\lesssim1$ only gives the large exponent
\[\|E(t)\|_\infty\lesssim\|\rho(t)\|_\infty^{4/9}\|\rho(t)\|_{5/3}^{5/9}\lesssim(Q(t)^3)^{4/9}\cdot1^{5/9}=Q(t)^{4/3},\]
so we need another approach for suppression of the exponent $4/3$ down to $1$.
Our strategy is to average in the time direction.
Precisely, we estimate the averaged field
\[\frac1\Delta\int_{t-\Delta}^t|E(s,X(s;t,\hat x,\hat v))|\,ds\lesssim Q(t)^a\]
for arbitrary $t,\hat x,\hat v$ and for suitably chosen $\Delta$.
Then, we would get a weaker inequality
\[Q(t)-Q(t-\Delta)\lesssim \Delta\cdot Q(t)^a,\]
which is also able to deduce (\ref{gjg}).
The detailed proof of (\ref{gjg}) will be presented in Section 3.4.


\begin{notn*}
Fix $(t,\hat x,\hat v)\in\R^+\times\R^6$.
We will write
\[\hat X(s):=X(s;t,\hat x,\hat v)\quad\text{and}\quad\hat V(s):=V(s;t,\hat x,\hat v).\]
Also, we will use the notations
\[X(s):=X(s;t,x,v)\quad\text{and}\quad V(s):=V(s;t,x,v),\]
where $x,v$ are usually used in integration variable.
Symbols $y$ and $w$ are always used for $X(s)$ and $V(s)$ at time $s$ especially when applying volume preserving coordinates transformation $(x,v)\mapsto(X(s),V(s))=(y,w)$.
\end{notn*}

Now, our ultimate goal is to bound the integral
\begin{align*}
\int_{t-\Delta}^t|E(s,\hat X(s))|\,ds
&\le\frac1{4\pi}\int_{t-\Delta}^t\iint\frac{f(s,y,w)}{|y-\hat X(s)|^2}\,dw\,dy\,ds\\
&=\frac1{4\pi}\int_{t-\Delta}^t\iint\frac{f(t,x,v)}{|X(s)-\hat X(s)|^2}\,dv\,dx\,ds
\end{align*}
by velocity.
The main difficulty of this integral is that $|y-\hat X(s)|^{-2}$ is not integrable with respect to $y$ on the region where $|y|$ is large; the inverse square has too slow decay rate to be integrable in three-dimesional space $\R^3$.

We want to find a lower bound of the relative position vector $|X(s)-\hat X(s)|$ assuming it is large.
When the distance between $X(s)$ and $\hat X(s)$ is sufficiently large so that the interaction between particles at positions $X(s)$ and $\hat X(s)$ is sufficiently weak, the distance will change almost linearly in both velocity and time by their inertia.
Intuitively, we can write
\[|X(s)-\hat X(s)|\gg1\impl X(s)-\hat X(s)\approx (v-\hat v)(s-c_1)+c_2,\]
where $c_1,c_2$ are constants that depend on $(t,x,v,\hat x,\hat v)$.

Then, here the time averaging plays its role: interchange the integral as follows using the Tonelli theorem:
\[\int_{t-\Delta}^t\iint\frac{f(t,x,v)}{|X(s)-\hat X(s)|^2}\,dv\,dx\,ds=\iint f(t,x,v)\left(\int_{t-\Delta}^t\frac{ds}{|X(s)-\hat X(s)|^2}\right)\,dv\,dx.\]
The time integration of $|X(s)-\hat X(s)|^{-2}\approx|(v-\hat v)(s-c_1)+c_2|^{-2}$ on a set $\{s:|(v-\hat v)(s-c_1)+c_2|\ge r\}$ for a proper spatial radius $r$ approximately cannot exceed $(|v-\hat v|r)^{-1}$.
It means that the singularity issue of a spatial function is changed to an estimate problem for a function of velocity.
Finally by taking $r$ such that $(|v-\hat v|r)^{-1}\lesssim|v|^2$, it allows that the kinetic energy directly bound the quantity that we want to control.
This idea is embodied in the ``ugly set estimate'' in Proposition 3.3.


\subsection{Lemmas on relative velocity}
The following lemma suggests an appropriate way to choose the time averaging interval $\Delta$.
\begin{lem}
Let $P>0$.
Suppose $s\le[t-\Delta,t]$, where
\[\Delta\cdot\sup_{s\in[0,t]}\|E(s)\|_\infty\le\frac P4.\]
\begin{parts}
\item If $|v|<P$, then $|V(s)|<2P$.
\item If $|v|\ge P$, then $\frac12|v|\le|V(s)|\le2|v|$. 
\item If $|v-\hat v|<P$, then $|V(s)-\hat V(s)|<2P$.
\item If $|v-\hat v|\ge P$, then $\frac12|v-\hat v|\le|V(s)-\hat V(s)|\le2|v-\hat v|$.
\end{parts}
\end{lem}
\begin{pf}
Note that
\[|v-V(s)|\le\int_s^t|E(s',X(s'))|\,ds'\le\Delta\cdot\sup_{s'\in[0,t]}\|E(s')\|_\infty\le\frac P4,\]
and similarly
\[|\hat v-\hat V(s)|\le\frac P4.\]

For (a),
\[|V(s)|\le|v|+|v-V(s)|<P+\frac P4<2P.\]
For (b),
\[|V(s)|\ge|v|-|v-V(s)|\ge|v|-\frac P4\ge\frac12|v|\]
and
\[|V(s)|\le|v|+|v-V(s)|\le|v|+\frac P4\le2|v|.\]
For (c)
\[|V(s)-\hat V(s)|\le|V(s)-v|+|v-\hat v|+|\hat v-\hat V(s)|<\frac P4+P+\frac P4<2P.\]
For (d),
\[|V(s)-\hat V(s)|\ge-|V(s)-v|+|v-\hat v|-|\hat v-\hat V(s)|\ge-\frac P4+|v-\hat v|-\frac P4\ge\frac12|v-\hat v|\]
and
\[|V(s)-\hat V(s)|\le|V(s)-v|+|v-\hat v|+|\hat v-\hat V(s)|\le\frac P4+|v-\hat v|+\frac P4\le2|v-\hat v|.\]
\end{pf}


From now for $0\le\Delta\le t$, we always assume that it is sufficiently small such that
\[\Delta\cdot\sup_{s\in[0,t]}\|E(s)\|_\infty\le\frac P4.\]

\begin{lem}[Lower bound of relative position vector]
If $v$ satisfies $|v-\hat v|\ge P$, then there is $s_0\in[t-\Delta,t]$ such that
\[|X(s)-\hat X(s)|\ge\frac14|v-\hat v||s-s_0|\]
for all $s\in[t-\Delta,t]$ and $x\in\R^3$.
\end{lem}
\begin{pf}
Let $Z(s):=X(s)-\hat X(s)$ be the relative position vector.
Then,
\begin{align*}
Z'(s)&=V(s)-\hat V(s),\\
Z''(s)&=\gamma[E(s,X(s),V(s))-E(s,\hat X(s),\hat V(s))],
\end{align*}
so the Taylor expansion at $s_0\in[t-\Delta,t]$ gives
\[Z(s)=\left[Z(s_0)+Z'(s_0)(s-s_0)\right]+\left[\frac{Z''(\sigma)}2(s-s_0)^2\right]\]
for some $\sigma$ between $s,s_0$.

Choose
\[s_0=\argmin_{s\in[t-\Delta,t]}|Z(s)|.\]
If $s_0=t$ or $s_0=t-\Delta$, then $\dd{s}|Z(s_0)|^2\le0$ or $\dd{s}|Z(s_0)|^2\ge0$ respectively.
Otherwise, $s_0\in(t-\Delta,t)$, and $\dd{s}|Z(s_0)|^2=0$.
Hence
\[Z(s_0)\cdot Z'(s_0)(s-s_0)=\frac12\dd{s}|Z(s_0)|^2(s-s_0)\ge0\]
for $s\in[t-\Delta,t]$, and
\[|Z(s_0)+Z'(s_0)(s-s_0)|^2\ge|Z'(s_0)(s-s_0)|^2.\]

The condition $|v-\hat v|\ge P$ implies
\[|Z'(s)|\ge\frac12|v-\hat v|\]
for $s\in[t-\Delta,t]$.
Therefore,
\[|Z(s_0)+Z'(s_0)(s-s_0)|\ge|Z'(s_0)(s-s_0)|\ge\frac12|v-\hat v||s-s_0|,\]
and
\[|\frac{Z''(\sigma)}2(s-s_0)^2|\le\|E(t)\|_\infty(s-s_0)^2
\le\|E(t)\|_\infty\Delta|s-s_0|\le\frac P4|s-s_0|
\le\frac14|v-\hat v||s-s_0|\]
yields
\[|X(s)-\hat X(s)|=|Z(s)|\ge\frac14|v-\hat v||s-s_0|.\qedhere\]
\end{pf}


\subsection{Divide and conquer}

We estimate the integral of $|E(s,\hat X(s))|$ by dividing the integral region $[t-\Delta,t]\times\R^6$ into three regions as:
for $P\ge4\Delta\cdot\sup_{s\in[0,t]}\|E(s)\|_\infty$ and $R>0$, define
\begin{align*}
U:&=\{\,(s,x,v):|X(s)-\hat X(s)|\ge r,\,|v-\hat v|\ge P\,\},\\
B:&=\{\,(s,x,v):|X(s)-\hat X(s)|<r,\,|v-\hat v|\ge P,\,|v|\ge P\,\},\\
G:&=\{\,(s,x,v):\,\min\{|v-\hat v|,|v|\}<P\,\}\\
&=[t-\Delta,t]\times\R^6\setminus(U\cup B),
\end{align*}
where $r:=R\max\{|v|^{-3},|v-\hat v|^{-3}\}$.
The constants $P$ and $R$ will be determined later.
The conditions $|v-\hat v|\ge P$ on $U$ and $\min\{|v-\hat v|,|v|\}\ge P$ on $B$ are introduced in order for application of Lemma 3.2 and (b), (d) of Lemma 3.1 respectively.

\begin{prop}[Ugly set estimate]
\[\iiint_U\lesssim R^{-1}.\]
\end{prop}
\begin{pf}
Write
Then,
\[U=\{\,(s,x,v):s\in[t-\Delta,t],\quad|v-\hat v|\ge P,\quad |X(s)-\hat X(s)|\ge r\,\}.\]
Since $|X(s)-\hat X(s)|\ge r$ on $U$,
\[\int_{|s-s_0|<S}\frac{\chi_U(s,x,v)}{|X(s)-\hat X(s)|^2}\,ds
\le\frac1{r^2}\int_{|s-s_0|<S}ds=\frac{2S}{r^2},\]
and since $|v-\hat v|\ge P$ on $U$ so that $|X(s)-\hat X(s)|\ge\frac14|v-\hat v||s-s_0|$ by Lemma 3.2,
\[\int_{|s-s_0|\ge S}\frac{\chi_U(s,x,v)}{|X(s)-\hat X(s)|^2}\,ds
\le16\int_{|s-s_0|\ge S}\frac1{|v-\hat v|^2|s-s_0|^2}\,ds\\
=32\frac1{|v-\hat v|^2S}.\]
If we choose $S$ such that $2S/r^2=32/|v-\hat v|^2S$, then we obtain the estimate
\[\int\frac{\chi_U(s,x,v)}{|X(s)-\hat X(s)|^2}\,ds\lesssim\frac1{|v-\hat v|r}.\]

Then, by the definition of $r$, 
\[\frac1{|v-\hat v|r}=R^{-1}\frac{\min\{|v|^3,|v-\hat v|^3\}}{|v-\hat v|}\le R^{-1}|v|^2\]
so that we have
\begin{align*}
\iiint_U\frac{f(t,x,v)}{|X(s)-\hat X(s)|^2}\,dv\,dx\,ds
&=\iint f(t,x,v)\left(\int\frac{\chi_U(s,x,v)}{|X(s)-\hat X(s)|^2}\,ds\right)\,dv\,dx\\
&\lesssim R^{-1}\iint|v|^2f(t,x,v)\,dv\,dx\lesssim R^{-1}.\qedhere
\end{align*}
\end{pf}



\begin{prop}[Bad set estimate]
\[\iiint_B\lesssim\Delta\cdot R\log\frac{4Q(t)}P.\]
\end{prop}
\begin{pf}
Because $|X(s)-\hat X(s)|<r$ in $B$, we need to find estimates for the union of two regions
\[|X(s)-\hat X(s)|<R|v|^{-3}\quad\text{and}\quad|X(s)-\hat X(s)|<R|v-\hat v|^{-3}.\]
If we integrate $|X(s)-\hat X(s)|^{-2}$ with respect to $y$ on these regions, then we get integrands $|v|^{-3}$ and $|v-\hat v|^{-3}$, which has singularities on regions at which $|v|$, $|v-\hat v|$ are respectively small and large; an inverse cubic function is both sharp and broad in three dimensional space $\R^3$.
Fortunately, the integral of inverse cube on the region $|v|\sim\infty$ is bounded by $Q$, and the region $|v|\sim0$ is bounded by $P$.

For each $s\in[t-\Delta,t]$, we apply the transformation $(x,v)\mapsto(y,w)=(X(s),V(s))$.
Since $|v|\ge P$ implies
\[\frac12P\le|w|\le Q(s)\quad\text{and}\quad|w|\le2|v|\]
by Lemma 3.1, we have
\begin{align*}
\int_{|v|\ge P}&\int_{|X(s)-\hat X(s)|<R|v|^{-3}}\frac{f(t,x,v)}{|X(s)-\hat X(s)|^2}\,dx\,dv\\
&\lesssim\int_{\frac12P\le|w|\le Q(s)}\int_{|y-\hat X(s)|<R|V(t;s,y,w)|^{-3}}\frac1{|y-\hat X(s)|^2}\,dy\,dw\\
&\simeq\int_{\frac12P\le|w|\le Q(t)}R|V(t;s,y,w)|^{-3}\,dw\\
&\le8R\int_{\frac12P\le|w|\le Q(t)}|w|^{-3}\,dw\\
&\simeq R\log\frac{2Q(t)}P.
\end{align*}
Similarly but using $|v-\hat v|\ge P$, we have
\[\frac12P\le|w-\hat V(s)|\le 2Q(s)\quad\text{and}\quad|w-\hat V(s)|\le2|v-\hat v|,\]
and
\[\int_{|v-\hat v|\ge P}\int_{|X(s)-\hat X(s)|<R|v-\hat v|^{-3}}\frac{f(t,x,v)}{|X(s)-\hat X(s)|^2}\,dx\,dv\lesssim R\log\frac{4Q(t)}P.\]
Therefore,
\[\iiint_B\frac{f(t,x,v)}{|X(s)-\hat X(s)|^2}\,dv\,dx\,ds\lesssim\Delta\cdot R\log\frac{4Q(t)}P.\qedhere\]
\end{pf}


\begin{prop}[Good set estimate]
\[\iiint_G\lesssim\Delta\cdot P^{4/3}.\]
\end{prop}
\begin{pf}
As we have done in the bad set estimate, we need to control the integral on the union of two regions
\[|v|<P\quad\text{and}\quad|v-\hat v|<P.\]
We can use (a) and (c) of Lemma 3.1.
The coordinates transformation $(x,v)\mapsto(y,w)=(X(s),V(s))$ gives, since $|v|<P$ implies $|w|<2P$,
\[\iint_{|v|<P}\frac{f(t,x,v)}{|X(s)-\hat X(s)|^2}\,dv\,dx
\le\int\frac1{|y-\hat X(s)|^2}\int_{|w|<2P}f(s,y,w)\,dw\,dy.\]
If we write $\rho_P(s,y):=\int_{|w|<2P}f(s,y,w)\,dw$, then since its $L_y^{5/3}$ norm is bounded,
\[\int\frac{\rho_P(s,y)}{|y-\hat X(s)|^2}\,dy
\lesssim\|\rho_P(s,y)\|_{L_y^\infty}^{4/9}\cdot\|\rho_P(s,y)\|_{L_y^{5/3}}^{5/9}\\
\lesssim((2P)^3)^{4/9}\cdot1^{5/9}\simeq P^{4/3}.\]
Similarly on the region $|v-\hat v|<P$,
\[\iint_{|v-\hat v|<P}\frac{f(t,x,v)}{|X(s)-\hat X(s)|^2}\,dv\,dx\lesssim P^{4/3},\]
so we are done.
\end{pf}


\subsection{Bound on the velocity support}
Finally, with above estimates, we prove that $Q$ does not blow up.
We assume that the classical solution $f\in C^1([0,T);C_c^1(\R^6))$ is the maximally prolonged solution.
\begin{defn}
Define a function $\Delta:[0,T)\to[0,\infty)$ by
\[\Delta(t):=\frac1{cQ(t)^{4/3}}\frac{Q(t)^{4/11}}4=\frac1{4c}\,Q(t)^{-32/33}.\]
\end{defn}
\begin{cor}
Let $c=c(f_0)>0$ be a constant such that
\[\|E(s)\|_\infty\le cQ(s)^{4/3}\]
for all $s\in[0,T]$.
For $t<T$ such that $t-\Delta(t)>0$, and for any $a>\frac{16}{33}$, we have
\[Q(t)-Q(t-\Delta)\lesssim_a\Delta(t)\cdot Q(t)^a.\]
\end{cor}
\begin{pf}
Let $(d,e)=(\frac4{11},\frac{16}{33})$ and
\[P=Q(t)^d\quad\text{and}\quad R=Q(t)^e(\log\frac{4Q(t)}P)^{-1/2}.\]
Then, $\Delta(t)\cdot cQ(t)^{4/3}=\frac P4$.
Since
\[\Delta(t)\sup_{s\in[0,t]}\|E(s)\|_\infty=\frac P4\cdot\frac{\sup_{s\in[0,t]}\|E(s)\|_\infty}{cQ(t)^{4/3}}\le\frac P4,\]
we can use the estimates on $U$, $B$, and $G$ :
\begin{align*}
\int_{t-\Delta(t)}^t&|E(s,\hat X(s))|\,ds
\le\int_{t-\Delta(t)}^t\iint\frac{f(t,x,v)}{|X(s)-\hat X(s)|^2}\,dv\,dx\,ds\\
&\lesssim R^{-1}+\Delta(t)R\log\frac{4Q(t)}P+\Delta(t)P^{4/3}\\
&\simeq\Delta(t)\left(Q(t)^{4/3}P^{-1}R^{-1}+R\log\frac{4Q(t)}P+P^{4/3}\right)\\
&=\Delta(t)\left(Q(t)^{4/3-d-e}\sqrt{\log\frac{4Q(t)}P}+Q(t)^e\sqrt{\log\frac{4Q(t)}P}+Q(t)^{4d/3}\right).
\end{align*}
Because $d=\frac4{11}$ and $e=\frac{16}{33}$ satisfy
\[\tfrac43-d-e=e=\tfrac43d,\]
we get
\[\int_{t-\Delta}^t|E(s,\hat X(s))|\,ds\lesssim\Delta(t)Q(t)^{16/33}\log^{1/2}Q(t)\]
and the desired result by setting $\hat x$ and $\hat v$ to be arbitrarily but $f(t,\hat x,\hat v)\ne0$.
\end{pf}

\begin{rmk}
We must notice that the lower bound of $\Delta$ is given in this corollary.
Suppose $\Delta>0$ had no lower bound.
If we define an increasing function
\[j(z):=e^{\frac1{1-a}z^{1-a}},\]
that is, $j$ is defined as a solution of a differential equation $j'(z)=z^{-a}j(z)$, then the inequality in the above corollary
\[Q(t)-Q(t-\Delta)\le c\Delta\cdot Q(t)^a\]
with $c=c(f_0,a)$ would give
\begin{align*}
\tilde Q(t)-\tilde Q(t-\Delta)
&=j(Q(t))-j(Q(t-\Delta))\\
&\le j(Q(t))-j(Q(t)-c\Delta\cdot Q(t)^a)\\
&\le c\Delta\cdot Q(t)^a\ j'(Q(t))\\
&=c\Delta\cdot j(Q(t))=c\Delta\cdot\tilde Q(t),
\end{align*}
where $\tilde Q(t):=j(Q(t))$.
It derives an inequality including the left lower Dini's derivative
\[D_-(e^{ct}\tilde Q(t))\le0,\]
and this proves $\tilde Q(t)\le\tilde Q(0)e^{ct}$, which implies $Q(t)\lesssim_a(1+t)^{\frac1{1-a}}$.

However, unfortuately there is a lower bound for $\Delta$.
See the proof of Corollary 3.6, and check that the lower bound is required:
\[R^{-1}\lesssim\Delta\cdot Q(t)^{4/3}P^{-1}R^{-1}.\]
Thereby, we must use another discrete method to justify $Q(t)\lesssim_a(1+t)^{\frac1{1-a}}$.
\end{rmk}

\begin{thm*}[Schaeffer, 1991, restatement]
For $\frac{16}{33}<a<1$,
\[Q(t)\lesssim_a(1+t)^{\frac1{1-a}}.\]
\end{thm*}
\begin{pf}
Since
\[Q(t)-Q(s)\le\int_s^t\|E(s')\|_\infty\,ds'\]
so that $Q$ is a nondecreasing continuous function that diverges at time $T$, there is a unique $T_1=T_1(f_0)\in(0,T)$ such that $T_1=Q(T_1)^{-32/33}$.
We have $Q(t)\le Q(T_1)\lesssim1$ for $t\le T_1$, so assume $t\in(T_1,T)$.

Inductively define a decreasing sequence $\{t_i\}_i$ such that
\[t_0:=t,\qquad t_{i+1}:=t_i-\Delta(t_i).\]
The differences have a uniform lower bound
\[t_i-t_{i+1}=\Delta(t_i)=\frac1{4c}\,Q(t_i)^{-32/33}\ge\frac1{4c}\,Q(t)^{-32/33},\]
so there exists a positive integer $n$ such that $0<t_n\le T_1<t_{n-1}$.
By Corollary 3.6, $t_i-\Delta(t_i)>0$ implies
\[Q(t_i)-Q(t_{i+1})\lesssim_a\Delta(t_i)Q(t_i)^a\]
for $i<n$.
Then,
\begin{align*}
Q(t)-Q(T_1)&\le Q(t_0)-Q(t_n)
=\sum_{i=0}^{n-1}(Q(t_i)-Q(t_{i+1}))\\
&\lesssim_a\sum_{i=0}^{n-1}\Delta(t_i)\cdot Q(t_i)^a
\le\sum_{i=0}^{n-1}\Delta(t_i)\cdot Q(t)^a
\le tQ(t)^a
\end{align*}
yields
\[Q(t)\lesssim_a1+tQ(t)^a\lesssim(1+t)Q(t)^a.\]
Therefore, $Q(t)\lesssim(1+t)^{\frac1{1-a}}$.
\end{pf}


\bibliographystyle{acm}
\bibliography{bib}


\end{document}