\documentclass[12pt]{article}
\usepackage{../../ikany}
\usepackage[margin=100pt]{geometry}
\usepackage[T1]{fontenc}
\usepackage[bitstream-charter,cal]{mathdesign}
\linespread{1.1}


\title{Characterizations of Harmonic Functions Vanishing at Infinity on the Punctured Domain}
\author{Ikhan Choi}
\date{September 25, 2021.}


\begin{document}
\maketitle

\begin{thm*}
Let $d\ge3$.
A distribution $u\in\cD'(\R^d)$ is a harmonic function on $\R^d\setminus\{0\}$ and vanishes at infinity if and only if there is a distribution $\rho\in\cD'(\R^d)$ such that $u=\Phi*\rho$ and $\supp(\rho)\subset\{0\}$, where $\Phi$ denotes the fundamental solution of Laplace's equation.
\end{thm*}
\begin{pf}
($\Rightarrow$)
Define a distribution $\rho$ by
\[\<\rho,\f\>:=-\<u,\Delta\f\>\]
for $\f\in C_c^\infty(\R^d)$.
In other words, $\rho=-\Delta u$ in distributional sense.
Then, $\rho$ has the support contained in $\{0\}$ because if $\f\in C_c^\infty(\R^d\setminus\{0\})$ then
\[\<\rho,\f\>=-\<u,\Delta \f\>=-\int u(x)\Delta\f(x)\,dx=-\int\Delta u(x)\f(x)\,dx=0.\]
Therefore, we only need to verify $u=\Phi*\rho$ to complete the proof.

Let $\f\in C_c^\infty(\R^d)$.
Be cautious that the argument
\[\<\Phi*\rho,\f\>=\<\rho,\Phi*\f\>=-\<u,\Delta(\Phi*\f)\>=\<u,\f\>\]
fails to provide a proof because the function $\Phi*\rho$ is not compactly supported so that we cannot deduce $\<\rho,\Phi*\f\>=-\<u,\Delta(\Phi*\f)\>$, and here we use the condition that $u$ vanishes at infinity to justify the equality.
Define a cutoff function $\chi\in C_c^\infty(\R^d)$ such that
\[\chi(x)=\begin{cases}1&\text{ if }|x|\le\frac54\\0&\text{ if }|x|>\frac74\end{cases}.\]
If we denote $\chi_r(x):=\chi(\frac xr)$, then we have
\[\<\rho,(\Phi\chi_r)*\f\>=-\<u,\Delta((\Phi\chi_r)*\f)\>\]
by the definition of $\rho$.
We have the limit of the left-hand side
\[\lim_{r\to\infty}\<\rho,(\Phi\chi_r)*\f\>=\<\rho,\Phi*\f\>\]
because
\begin{align*}
\supp((\Phi(1-\chi_r)*\f)&\subset\supp(\Phi(1-\chi_r))+\supp(\f)\\
&\subset\R^d\setminus B(0,2R)+\cl B(0,R)=\R^d\setminus B(0,R)
\end{align*}
for all $r>2R$ so that the supports of $\Phi(1-\chi_r)*\f$ and $\rho$ are disjoint, where we define $R:=\sup_{x\in\supp(\f)}|x|$.
However, the right-hand limit
\[-\lim_{r\to\infty}\<u,\Delta((\Phi\chi_r)*\f)\>=-\<u,\Delta(\Phi*\f)\>\]
is not a trivial result.

Assuming $\chi(x)=\chi(-x)$ without loss of generality, we have
\[\<u,\Delta(\Phi(1-\chi_r)*\f)\>=\<u*\Delta(\Phi(1-\chi_r)),\f\>.\]
Because
\[\Delta_y\Bigl[\Phi(x-y)\bigl(1-\chi(\tfrac{x-y}r)\bigr)\Bigr]=0\]
for $|y|<R$ and $x\in\supp(\f)$ if $r>2R$, we can write
\[\<u*\Delta(\Phi(1-\chi_r)),\f\>
=\int\f(x)\int u(y)\Delta_y\Bigl[\Phi(x-y)\bigl(1-\chi(\tfrac{x-y}r)\bigr)\Bigr]\,dy\,dx.\]
We compute
\begin{align*}
\Delta_y\Bigl[\Phi(x-y)\bigl(1&-\chi(\tfrac{x-y}r)\bigr)\Bigr]
=2\nabla\Phi(x-y)\cdot\frac1r\nabla\chi(\tfrac{x-y}r)-\Phi(x-y)\frac1{r^2}\Delta\chi(\tfrac{x-y}r)\\
&=-\frac2{\omega_d}\frac{x-y}{|x-y|^d}\cdot\frac1r\nabla\chi(\tfrac{x-y}r)
-\frac1{(d-2)\omega_d}\frac1{|x-y|^{d-2}}\frac1{r^2}\Delta\chi(\tfrac{x-y}r).
\end{align*}
Then, since $\frac54r\le|x-y|\le\frac74r$ if $\nabla\chi(\tfrac{x-y}r)\ne0$ and $\Delta\chi(\tfrac{x-y}r)\ne0$, we obtain
\[\Bigl|\Delta_y\Bigl[\Phi(x-y)\bigl(1-\chi(\tfrac{x-y}r)\bigr)\Bigr]\Bigr|\le C\frac1{r^d}\psi(\frac{x-y}r)\]
for some constant $C>0$, where
\[\psi(y):=|\nabla\chi(y)|+|\Delta\chi(y)|.\]
For each $x\in\supp(\f)$, since we have $\frac54r\le|x-y|\le\frac74r$ implies $r\le|y|\le2r$ if $r>4R$, it follows that
\begin{align*}
|\int u(y)\Delta_y\bigl[\Phi(x-y)\bigl(1-\chi(\tfrac{x-y}r)\bigr)\bigr]\,dy|
&\le C\int|u(y)\frac1{r^d}\psi(\frac{x-y}r)|\,dy\\
&\le C\max_{r\le|y|\le2r}u(y)
\end{align*}
converges to zero as $r\to\infty$.
By the bounded convergence theorem, we can deduce
\[\lim_{r\to\infty}\int\f(x)\int u(y)\Delta_y\Bigl[\Phi(x-y)\bigl(1-\chi(\tfrac{x-y}r)\bigr)\Bigr]\,dy\,dx=0,\]
so we are done.

($\Leftarrow$)
Let $\f\in C_c^\infty(\R^d\setminus\{0\})$.
Since
\[\<\Phi*\rho,\Delta\f\>=\<\rho,\Phi*(\Delta\f)\>=\<\rho,\f\>=0,\]
the distribution $\Phi*\rho$ on $\R^d\setminus\{0\}$ is weakly harmonic, and by Weyl's lemma for distributions, it is a smooth harmonic function on $\R^d\setminus\{0\}$.

Since $\rho$ is supported at zero, we have a positive integer $k$ and constants $a_\alpha$ such that
\[|\<\rho,\f\>|\le\sum_{|a|\le k}|a_\alpha D^\alpha\f(0)|\]
for $\f\in C^\infty(\R^d)$.
Then, for $x\in\R^d$ with $|x|=r>0$, by taking $\chi\in C_c^\infty(\R^d)$ such that $\chi=1$ on $B(0,2r)$, we have
\[|\Phi*\rho(x)|=|(\Phi\chi)*\rho(x)|=|\<\rho(y),\Phi(x-y)\chi(x-y)\>_y|\le\sum_{|a|\le k}|a_\alpha D^\alpha\Phi(x)|=O(r^{2-d})\]
as $r\to\infty$.
Therefore, $\Phi*\rho$ vanishes at infinity.
\end{pf}

\begin{lem*}
Let $\rho$ be a distribution on $\R^d$ such that $\supp(\rho)\subset\{0\}$.
Then, there is a constant coefficient partial differential operator $P(D)$ such that $\rho=P(D)\delta$.
\end{lem*}
\begin{cor*}
Let $d\ge3$.
If a distribution $u\in\cD'(\R^d)$ is a harmonic function on $\R^d\setminus\{0\}$ and vanishes at infinity, then there are an integer $k\ge0$ and constants $a_\alpha$ such that
\[u(x)=\sum_{|a|\le k}a_\alpha D^\alpha\Phi(x)\]
for $x\ne0$, where $\Phi$ denotes the fundamental solution of Laplace's equation.
\end{cor*}


\end{document}