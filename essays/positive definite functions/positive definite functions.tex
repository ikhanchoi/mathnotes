\documentclass[12pt]{article}
\usepackage{../../ikany}
\usepackage[margin=100pt]{geometry}
\usepackage[T1]{fontenc}
\usepackage[bitstream-charter,cal]{mathdesign}
\linespread{1.1}


\title{Positive Definite Functions\\on Locally Compact Groups}
\author{Ikhan Choi}


\begin{document}
\maketitle
\tableofcontents

\begin{abstract}

\end{abstract}

\section{Introduction}

Let $G$ be a locally compact group.
We always assume that $G$ is Hausdorff.

\begin{defn}
A function $f:G\to\C$ is called \emph{positive definite} if for each positive integer $n$ a non-negativity condition
\[\sum_{k,l=1}^nf(x_l^{-1}x_k)\xi_k\bar\xi_l\ge0\]
is satisfied for every $n$-tuple $(x_1,\cdots,x_n)\in G^n$ and every vector $(\xi_1,\cdots,\xi_n)\in\C^n$.
\end{defn}
A function $f$ is positive definite if and only if bilinear forms defined by matrices $(f(x_ix_j^{-1}))_{i,j=1}^n$ for each positive integer $n$ are hermitian, and even more, positive \emph{semi}-definite, regardless of any choices of $(x_1,\cdots,x_n)\in G^n$.
Positive definite functions have several remarkable properties as follows:
\begin{prop}[Algebraic properties]$ $
\begin{parts}
\item $\cP(G)$ is closed under complex conjugation. Furthermore, $\bar{f(x)}=f(x^{-1})$.
\item $\cP(G)$ is closed under positive scalar multiplication.
\item $\cP(G)$ is closed under summation.
\item $\cP(G)$ is closed under product.
\end{parts}
\end{prop}
\begin{pf}
To be written
\end{pf}

\begin{prop}[Analytic properties]$ $
\begin{parts}
\item If $f\in\cP(G)$, then $\sup_{x\in G}|f(x)|\le f(e)$.
\item If $f_n\in\cP(G)$, then the pointwise limit $\lim_{n\to\infty}f_n$ is also in $\cP(G)$.
\item $f\in\cP(G)$ is both-sided uniformly continuous if it is continuous at the identity. (uncertain)
\item $f_n\in\cP(G)$ converges to $f$ compactly if it converges to $f$ pointwisely and $f_n(e)=1$.
\end{parts}
\end{prop}
\begin{pf}
To be written
\end{pf}

\bigskip

This thesis follows the historical flows to extract mathematical ideas behind the positive definite functions.
In particular, we are concerned with the results like the following \emph{Bochner-type theorems}:
\begin{thm}
A function $c:\Z\to\C$ is positive definite if and only if there is a unique finite regular Borel measure $\mu$ on $\T=\R/2\pi\Z$ such that
\[c(k)=\frac1{2\pi}\int_0^{2\pi}e^{-ik\theta}\,d\mu(\theta)\]
for all $k\in\Z$.
\end{thm}
\begin{thm}
A continuous function $\f:\R\to\C$ is positive definite if and only if there is a unique finite regular Borel measure $\mu$ on $\R$ such that
\[\f(t)=\int e^{itx}\,d\mu(x)\]
for all $t\in\R$.
\end{thm}
They have similar forms in that they describe the necessary and sufficient conditions for a function to have a Fourier-Stieltjes integral representation of a finite regular Borel measure.
One of our primary goals is to investigate the nature of positive definite functions and their harmonic-analytic relation to Borel measures within more familiar cases of $G=\Z$ or $\R$.
Now then, we finally extend the Bochner-type results in the more general setting, where $G$ is a locally compact group, and assign a new perspective of measures in terms of the representation theory of groups.

Each theorem has its own taste in different subfields of mathematics.
Theorem 1.1, which is a corollary of the celebrated Herglotz-Riesz representation theorem, is related to a classical problem in complex analysis that asks to give a characterization of a special class of analytic functions on the open unit disk $\D$ called the Carath\'eodory class.
The positive definiteness arises as a property of coefficients of functions in the Caracth\'eodory class, and their connection to Fourier coefficients leads the complex analysis problem into harmonic analysis.
In Section 2, with the methods of elementary complex variable function theory, our first Bochner-type theorem will be proved, giving a geometric description of the space of positive definite functions in addition.

In Section 3, we review the well-known results of the positive definite functions on the real line and their ``weak convergence''.
They have been well-studied by probabilists, to attack the weak convergence of probaility measures.
Recall that a probaility distribution of a real-valued random variable is defined by a probability measure on $\R$.
The extended Fourier transform, but reversing the sign convention on the phase term, with respect to not only integrable functions but also finte measures, called Fourier-Stieltjes transform, of a probability measure $\mu$ is called a characteristic function of the distribution $\mu$.
In terms of probability theory, it is nothing but the function defined by the expectation $\f(t):=Ee^{itX}$, where $X$ is a random variable of law $\mu$.
The Bochner theorem states that the necessary and sufficient condition for being a characteristic function is the positive definiteness.

Summaries for Section 4 and Section 5... will be here.






\section{On the group $\Z$: complex analysis}

Before the discussion of big theorems including the Carath\'eodory-Toeplitz theorem and the Herglotz-Riesz representation theorem, we develop a lemma as a preparation for the interplay between complex analysis and Fourier analysis.

\begin{lem}[Fourier series of analytic functions]
Let $f$ be an analytic function on the open unit disk $\D$ with $f(0)\in\R$ with
\[f(z)=c_0+\sum_{k=1}^\infty2c_kz^k,\]
the power series expansion of $f$ at $z=0$.
Use the notation $c_{-k}:=\bar c_k$.
\begin{parts}
\item For $0\le r<1$ and $0\le \theta<2\pi$, we have
\[\Re f(re^{i\theta})=\sum_{k=-\infty}^\infty c_kr^{|k|}e^{ik\theta}.\]
\item For $0\le r<1$, we have
\[c_kr^{|k|}=\frac1{2\pi}\int_0^{2\pi}\Re f(re^{i\theta})e^{-ik\theta}\,d\theta.\]
\end{parts}
\end{lem}
\begin{pf}
(a)
Easy computation shows the identity
\begin{align*}
\Re f(re^{i\theta})
&=\frac12[f(re^{i\theta})+\bar{f(re^{i\theta})}]\\
&=\frac12\left[\left(1+\sum_{k=1}^\infty2c_k(re^{i\theta})^k\right)+\bar{\left(1+\sum_{k=1}^\infty2c_k(re^{i\theta})^k\right)}\right]\\
&=\frac12\left[\left(1+\sum_{k=1}^\infty2c_kr^ke^{ik\theta}\right)+\left(1+\sum_{k=1}^\infty2\bar{c_k}r^ke^{-ik\theta}\right)\right]\\
&=\sum_{k=-\infty}^\infty c_kr^{|k|}e^{ik\theta}.
\end{align*}

(b)
It is clear from the uniform convergence of the series in the part (a) and the orthogonality
\[\frac1{2\pi i}\int_0^{2\pi}e^{-ik\theta}e^{il\theta}\,d\theta=\begin{cases}1&\text{ if }k=l\\0&\text{ if }k\ne l\end{cases}.\]
\end{pf}

\subsection{The Carath\'eodory coefficient problem}

The positive definiteness of functions were originally inspired by ``Carath\'eodory coefficient problem'' in complex analysis.
The problem asks the condition on the power series coefficients for an analytic function defined on the open unit disk to have values of positive real part.
The original paper deals with the functions analytic on a neighborhood of the closed unit disk, but the idea is extended well to the functions that has harsh behavior on the boundary.

\begin{defn}
The \emph{Carath\'eodory class} is the set of all analytic functions $f$ that map the open unit disk into the region of positive real part, with normalization condition $f(0)=1$.
\end{defn}


Typical examples of functions in the Carath\'eodory class are given by the family of functions
\[f_\theta(z)=\frac{e^{i\theta}+z}{e^{i\theta}-z}=1+\sum_{k=1}^\infty2e^{-ik\theta}z^k\]
parametrized by $\theta\in[0,2\pi)$.
We can check they are eactly the M\"obius transformations that map the unit circle to the imaginary axis satisfying $f(0)=1$.
Note the Carath\'eodory class is convex; if $f_0$ and $f_1$ are in the Carath\'eodory class, then the real part of the image of the function $f_t(z)=(1-t)f_0(z)+tf_1(z)$ is also positive for $0<t<1$ and $f_t(0)=(1-t)+t=1$, so $f_t$ also belongs to the Carath\'eodory class.
It clearly implies that the convex combination of $f_\theta$ is also in the Carath\'eodory class.
Carath\'eodory's result tells us that the converse statement also holds in a modified sense, so that $f_\theta$ can be viewed as ``extreme points'' in the Carath\'eodory class.
We discuss about the extreme points later.
Precisely, it is stated as follows:

\begin{thm}[Carath\'eodory]
Let $f$ be an analytic function on the open unit disk with the power series expansion
\[f(z)=1+\sum_{k=1}^\infty2c_kz^k.\]
Then, $f$ belongs to the Carath\'eodory class if and only if the point $(c_1,\cdots,c_n)\in\C^n$ belongs to the convex hull of the curve $(e^{-i\theta},\cdots,e^{-in\theta})\in\C^n$ parametrized by $\theta\in[0,2\pi)$ for each $n$.
\end{thm}
\begin{pf}
($\Leftarrow$)
Denote by $K_n$ the convex hull of the curve $\theta\mapsto(e^{-i\theta},\cdots,e^{-in\theta})\in\C^n$.
Suppose first that $(c_1,\cdots,c_n)\in K_n$.
For each $n$, there exists a finite sequence of pairs $(\lambda_{n,j},\theta_{n,j})_j$ having the following convex combination
\[(c_1,\cdots,c_n)=\sum_j\lambda_{n,j}(e^{-i\theta_{n,j}},\cdots,e^{-in\theta_{n,j}})\]
with coefficients $\lambda_{n,j}\ge0$ such that $\sum_j\lambda_{n,j}=1$.
Define
\[f_n(z):=\sum_j\lambda_{n,j}\frac{e^{i\theta_{n,j}}+z}{e^{i\theta_{n,j}}-z},\]
which has positive real part on $|z|<1$ because $\Re(e^{i\theta_{n,j}}+z)/(e^{i\theta_{n,j}}-z)>0$ for $|z|<1$.
Then,
\begin{align*}
f_n(z)
&=\sum_j\lambda_{n,j}(1+\sum_{k=1}^\infty2e^{-ik\theta_{n,j}}z^k)\\
&=1+\sum_{k=1}^n2c_kz^k+\sum_{k=n+1}^\infty\left(\sum_j2\lambda_{n,j}e^{-ik\theta_{n,j}}\right)z^k
\end{align*}
implies
\begin{align*}
|f_n(z)-f(z)|
&=\left|\sum_{k=n+1}^\infty\left(\sum_j2\lambda_{n,j}e^{-ik\theta_{n,j}}\right)z^k-\sum_{k=n+1}^\infty2c_kz^k\right|\\
&\le\sum_{k=n+1}^\infty\left|\left(\sum_j2\lambda_{n,j}e^{-ik\theta_{n,j}}\right)-2c_k\right||z|^k\\
&\le\sum_{k=n+1}^\infty4|z|^k
\end{align*}
converges to zero for $|z|<1$.
Therefore, $f$ has non-negative real part on the open unit disk.
The non-negativity is strengthen to the positivity by the open mapping theorem so that $f$ belongs to the Carath\'eodory class.

($\Rightarrow$)
Conversely, suppose that $f$ is in the Carath\'eodory class.
Let $(\gamma_1,\cdots,\gamma_n)$ be any point on the surface $\partial K_n$ of $K_n$ and $S$ any supporting hyperplane of $K_n$ tangent at $(\gamma_1,\cdots,\gamma_n)$.
Let $(u_1,\cdots,u_n)$ be the outward unit normal vector of the supporting hyperplane $S$.
Note that this unit normal vector is uniquely determined with respect to the induced real inner product structure on $2n$-dimensional space $\C^n$ described by
\[\<(z_1,\cdots,z_n),(w_1,\cdots,w_n)\>=\sum_{k=1}^n(\Re z_k\Re w_k+\Im z_k\Im w_k)=\Re\sum_{k=1}^nz_k\bar w_k.\]
Then, $\sum_{k=1}^n|u_k|^2=1$ and further that the maximum
\[M:=\max_{(x_1,\cdots,x_n)\in K_n}\ \Re\sum_{k=1}^nx_k\bar u_k>0\]
is attained at $(\gamma_1,\cdots,\gamma_n)$.
Our goal is to verify the bound
\[\Re\sum_{k=1}^nc_k\bar u_k\le M,\]
which implies that $(c_1,\cdots,c_n)$ is contained in every half space tangent to $K_n$ so that we finally obtain $(c_1,\cdots,c_n)\in K_n$.

Since for any $\theta\in[0,2\pi)$ the point $(e^{-i\theta},\cdots,e^{-in\theta})$ is in $K_n$ so that
\[\Re\sum_{k=1}^ne^{-ik\theta}\bar u_k\le M,\]
we have for arbitrarily small $\e>0$ that
\[\Re\sum_{k=1}^n\frac1{r^k}e^{-ik\theta}\bar u_k\le M+\e\]
for any $0<r<1$ sufficiently close to $1$, thus we can write
\begin{align*}
\Re\sum_{k=1}^nc_k\bar u_k
&=\Re\sum_{k=1}^n\frac1{2\pi r^k}\int_0^{2\pi}\Re f(re^{i\theta})e^{-ik\theta}\bar u_k\,d\theta\\
&=\frac1{2\pi}\int_0^{2\pi}\Re f(re^{i\theta})\Re\sum_{k=1}^n\frac1{r^k}e^{-ik\theta}\bar u_k\,d\theta\\
&\le\frac1{2\pi}\int_0^{2\pi}\Re f(re^{i\theta})\,d\theta\cdot(M+\e)\\
&=M+\e
\end{align*}
thanks to the positivity of $\Re f$, and by limiting $r\to1$ from left we get the bound
\[\Re\sum_{k=1}^nc_k\bar u_k\le M.\qedhere\]
\end{pf}


Here we introduce an infinite-dimentional description of this theorem, combining with the Herglotz representation theorem, suggests a rough but intuitional description of the space of probability measures on the unit circle $\T$.


\begin{prop}
Consider a sequence space $\C^\N$, endowed with the standard product topology.
Then, the condition addressed in Caracth\'eodory's theorem is equivalent to the following: the point $(c_1,c_2,\cdots)\in\C^\N$ belongs to the closed convex hull of the curve $(e^{-i\theta},e^{-i2\theta},\cdots)\in\C^\N$ parametrized by $\theta\in[0,2\pi)$.

Furthermore, the curve $(e^{-i\theta},e^{-i2\theta},\cdots)\in\C^\N$ is the set of extreme points of its closed convex hull.
\end{prop}
\begin{pf}
Denote by $K_n$ the convex hull of the curve $\theta\mapsto(e^{-i\theta},\cdots,e^{-in\theta})\in\C^n$, and by $K$ the closed convex hull of the curve $\theta\mapsto(e^{-i\theta},e^{-i2\theta},\cdots)\in\C^\N$.
If we assume the Carath\'eodory coefficient condition is true, then since for each $n$ we have a convex combination
\[(c_1,\cdots,c_n)=\sum_j\lambda_{n,j}(e^{-i\theta_{n,j}},\cdots,e^{-in\theta_{n,j}})\]
with coefficients such that $\lambda_{n,j}\ge0$ and $\sum_j\lambda_{n,j}=1$, the sequence
\begin{align*}
&(c_1,\cdots,c_n,\sum_j\lambda_{n,j}e^{-i(n+1)\theta_{n,j}},\sum_j\lambda_{n,j}e^{-i(n+2)\theta_{n,j}}\cdots)\\
&\qquad\qquad=\sum_j\lambda_{n,j}(e^{-i\theta_{n,j}},\cdots,e^{-in\theta_{n,j}},e^{-i(n+1)\theta_{n,j}},e^{-i(n+2)\theta_{n,j}},\cdots)
\end{align*}
is contained in  and converges to the point $(c_1,c_2,\cdots)$ in the product topology as $n\to\infty$, so we are done with the desired result.
For the opposite direction, let $(c_1,c_2,\cdots)\in K$.
By definition of $K$ we have an expression
\[c_k=\lim_{m\to\infty}\sum_{j=1}^m\lambda_{m,j}e^{-ik\theta_{m,j}}\]
with $\lambda_{m,j}\ge0$ and $\sum_{j=1}^m\lambda_{m,j}=1$, for each $k$.
Then,
\[(c_1,\cdots,c_n)=\lim_{m\to\infty}\sum_{j=1}^m\lambda_{m,j}(e^{-i\theta_{m,j}},\cdots,e^{-in\theta_{m,j}})\]
belongs to $K_n$ because $K_n$ is closed.


Fix $\theta\in[0,2\pi)$ and suppose two complex sequences $(c_1,c_2,\cdots)$ and $(d_1,d_2,\cdots)$ in $\C^\N$ are contained in $K$ and satisfy
\[\frac{c_k+d_k}2=e^{-ik\theta}\]
for all $k\in\N$.
For each $k$, since all components of $K$ are bounded by one so that $|c_k|\le1$ and $|d_k|\le1$, and since $e^{-ik\theta}$ is an extreme point of the closed unit disk $\bar\D\subset\C$, we have $c_k=d_k=e^{-ik\theta}$, we deduce that $(e^{-i\theta},e^{-i2\theta},\cdots)$ is an extreme point of $K$.
Conversely, every extreme point of $K$ is contained in the curve $(e^{-i\theta},e^{-i2\theta},\cdots)$ by Milman's ``converse'' theorem of the Krein-Milman theorem[citation: Phelps].
\end{pf}



\subsection{Toeplitz's algebraic condition}

Toeplitz discovered the coefficient condition addressed in the Carath\'eodory's paper which regards convex bodies enveloped by a curve can be equivalently described in terms of an algebraic condition that the hermitian matrices
\[H_n:=(c_{k-l})_{k,l=1}^n=\mat{c_0&c_1&c_2&\cdots&c_{n-1}\\c_{-1}&c_0&c_1&\cdots&c_{n-2}\\c_{-2}&c_{-1}&c_0&\cdots&c_{n-3}\\\vdots&\vdots&\vdots&\ddots&\vdots\\c_{-n+1}&c_{-n+2}&c_{-n+3}&\cdots&c_0}\]
of size $n\times n$ always have non-negative determinant for any $n$.
This algebraic condition is equivalent to that $H_n$ are all positive semi-definite matrices.
Since the principal minors of a positive semi-definite matrix is positive semi-definite, and since a hermitian matrix such that every leading principal minor has non-negative determinant is positive semi-definite, the bilateral sequence $(c_k)_{k=-\infty}^\infty$ is positive definite function when we consider it as a complex-valued function on $\Z$ that maps an integer $k$ to $c_k$ if and only if it is a positive definite \emph{sequence} in the following sense:

\begin{defn}
A bilateral complex sequence $(c_k)_{k=-\infty}^\infty$ is said to be \emph{positive definite} if
\[\sum_{k,l=1}^nc_{k-l}\xi_k\bar\xi_l\ge0\]
for each $n$ and $(\xi_1,\cdots,\xi_n)\in\C^n$.
\end{defn}

\begin{thm}[Carath\'eodory-Toeplitz]
Let $f$ be an analytic function on the open unit disk with the power series expansion
\[f(z)=1+\sum_{k=1}^\infty2c_kz^k.\]
Then, $f$ belongs to the Carath\'eodory class if and only if the sequence $(c_k)_{k=-\infty}^\infty$ is positive definite, where we use the notations $c_0=1$ and $c_{-k}=\bar{c_k}$.
\end{thm}
\begin{pf}
($\Rightarrow$)
If $f$ is in the Carath\'eodory class, then because
\[c_{k-l}r^{|k-l|}=\frac1{2\pi}\int_0^{2\pi}\Re f(re^{i\theta})e^{-i(k-l)\theta}\,d\theta,\]
we have
\[\sum_{k,l=1}^nc_{k-l}\xi_k\bar\xi_l
=\lim_{r\uparrow1}\frac1{2\pi}\int_0^{2\pi}\Re f(re^{i\theta})\left|\sum_{k=1}^ne^{-ik\theta}\xi_k\right|^2\,d\theta\ge0\]
for each $n$.

($\Leftarrow$)
Conversely, assume that the coefficient sequence $(c_k)_{k=-\infty}^\infty$ is positive definite.
Put $\xi_k=z^{k-1}$ and $z=re^{i\theta}$ to write
\begin{align*}
0&\le\sum_{k,l=1}^{n+1}c_{k-l}z^{k-1}(\bar z)^{l-1}\\
&=\sum_{k,l=0}^nc_{k-l}r^{k+l}e^{i(k-l)\theta}\\
&=\sum_{k,l=0}^nc_{k-l}r^{|k-l|}r^{2\min\{k,l\}}e^{i(k-l)\theta}\\
&=\sum_{k=-n}^nc_kr^{|k|}e^{ik\theta}\sum_{l=0}^{n-|k|}r^{2l}\\
&=\sum_{k=-n}^nc_kr^{|k|}e^{ik\theta}\frac{1-r^{2(n-|k|+1)}}{1-r^2}\\
&=\frac1{1-r^2}\sum_{k=-n}^nc_kr^{|k|}e^{ik\theta}
-\frac{r^{n+2}}{1-r^2}\sum_{k=-n}^nc_kr^{n-|k|}e^{ik\theta}.
\end{align*}
For $r=|z|<1$ the first term tends to
\[\lim_{n\to\infty}\frac1{1-r^2}\sum_{k=-n}^nc_kr^{|k|}e^{ik\theta}=\frac1{1-|z|^2}\Re f(z),\]
and $|c_k|\le c_0=1$ implies the second term vanishes as
\[\left|\frac{r^{n+2}}{1-r^2}\sum_{k=-n}^nc_kr^{n-|k|}e^{ik\theta}\right|\le\frac{r^{n+2}}{1-r^2}(2n+1)\to0\]
as $n\to\infty$.
It proves $\Re f(z)\ge0$ for $|z|<1$, and we obtain $\Re f(z)>0$ by the open mapping theorem.
\end{pf}


\subsection{The Herglotz-Riesz representation theorem}

Herglotz proved another equivalent condition in 1911, considered as the first Bochner-type theorem, in terms of regular Borel measures on the unit circle.
\begin{thm}[The Herglotz-Riesz representation theorem]
Let $f$ be a complex-valued function defined on the open unit disk.
Then, $f$ belongs to the Carath\'eodory class if and only if $f$ is represented as the following Stieltjes integral
\[f(z)=\frac1{2\pi}\int_0^{2\pi}\frac{e^{it}+z}{e^{it}-z}\,d\mu(t),\]
where $\mu:[0,2\pi]\to\R$ is a non-decreasing function with $\mu(2\pi)-\mu(0)=1$.
\end{thm}
\begin{pf}
To be written.
It uses Helly's selesction theorem.
\end{pf}


\begin{cor}[The Bochner theorem on $\Z$]
A function $c:\Z\to\C$ is positive-definite if and only if there is a finite regular Borel measure $\mu$ on $\T=\R/2\pi\Z$ such that $\hat\mu=c$.
\end{cor}
\begin{pf}
Let $\mu$ be a finite Borel measure on $\T$.
Then, $\hat\mu$ is positive definite because
\begin{align*}
\sum_{k,l=1}^n\hat\mu(k-l)\xi_k\bar\xi_l
&=\sum_{k,l=1}^n\frac1{2\pi}\int_0^{2\pi}e^{-i(k-l)\theta}\,d\mu(\theta)\ \xi_k\bar\xi_l\\
&=\frac1{2\pi}\int_0^{2\pi}\left|\sum_{k=1}^ne^{-ik\theta}\xi_k\right|^2d\mu(\theta)\ge0
\end{align*}
for any $(\xi_1,\cdots,\xi_n)\in\C^n$.

On the other hand, if the sequence $(c_k)_{k=-\infty}^\infty$ is positive definite, then, with the assumption $c_0=1$, the function $z\mapsto1+\sum_{k=1}^\infty2c_kz^k$ is in the Carath\'eodory class.
By the Herglotz-Riesz representation theorem, there is a probability regular Borel measure $\mu$ on $\cT$ such that
\begin{align*}
1+\sum_{k=1}^\infty2c_kz^k
&=\frac1{2\pi}\int_0^{2\pi}\frac{e^{it}+z}{e^{it}-z}\,d\mu(t)\\
&=\frac1{2\pi}\int_0^{2\pi}\left(1+\sum_{k=1}^\infty2e^{-ik\theta}z^k\right)\,d\mu(t)\\
&=1+\sum_{k=1}^\infty2\left(\frac1{2\pi}\int_0^{2\pi}e^{-ik\theta}\,d\mu(\theta)\right)z^k
\end{align*}
in $z\in\D$, hence that
\[c_k=\frac1{2\pi}\int_0^{2\pi}e^{-ik\theta}\,d\mu(\theta)=\hat\mu(k).\qedhere\]
\end{pf}

So far, we have proved the following one-to-one correspondences:
\begin{figure}
\begin{tikzcd}
&\begin{tabular}{c}Closed convex hull of\\the curve $(e^{-i\theta},e^{-i2\theta},\cdots)$\end{tabular}&\\
\begin{tabular}{c}Positive definite\\sequences $(c_k)_k$\\with $c_0=1$\end{tabular}
&\text{Carath\'eodory class}\lar[<->]\uar[<->]\rar[<->]
&\begin{tabular}{c}Probability Borel\\measures on $\T$\end{tabular}
\end{tikzcd}
\end{figure}

\section{On the group $\R$: probability theory}
We have seen the relation of positive definite sequences and measures on the unit circle $\T$.
Studying measures with positive definite functions virtually starts in probability theory, especially from the theory of weak convergence of probability distributions.

\subsection{Weak convergence of probability measures}

\begin{defn}[Weak convergence and vague convergence]
Let $\mu_n$ be a sequence of probability measures on a 
\end{defn}


\begin{thm}[L\'evy's continuity theorem]

\end{thm}

\subsection{Examples of positive definite functions}

Mathias' examples

Polya's criterion

\subsection{A proof of Bochner's theorem}

Riesz-Markov-Kakutani representation theorem

\subsection{Application: Stone-von Neumann theorem}






\section{On locally compact abelian groups}

\section{On locally compact non-abelian groups}



\bibliographystyle{acm}
\bibliography{bib}


\end{document}