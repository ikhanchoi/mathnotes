\documentclass[12pt]{article}
\usepackage{../../ikany}
\usepackage[margin=100pt]{geometry}
\usepackage[T1]{fontenc}
\usepackage[bitstream-charter,cal]{mathdesign}
\linespread{1.1}

\begin{document}
\tableofcontents






\section{Universal coefficient theorem}
\begin{lem}
Suppose we have a flat resolution
\begin{es}
0\>P_1\>P_0\>A\> 0.
\end{es}
Then, we have a exact sequence
\begin{es}
\cdots\>0\>\Tor_1^R(A,B)\>P_1\otimes B\>P_0\otimes B\>A\otimes B\>0.
\end{es}
\end{lem}


\begin{thm}
Let $R$ be a PID.
Let $C_\bullet$ be a chain complex of flat $R$-modules and $G$ be a $R$-module.
Then, we have a short exact sequence
\begin{es}
0\>H_n(C)\otimes G\>H_n(C;G)\>\Tor(H_{n-1}(C),G)\>0,
\end{es}
which splits, but not naturally.
\end{thm}

\begin{pf}[1]
We have a short exact sequence of chain complexes
\begin{es}
0\>Z_\bullet\>C_\bullet\>B_{\bullet-1}\>0
\end{es}
where every morphism in $Z_\bullet$ and $B_\bullet$ are zero.
Since modules in $B_{\bullet-1}$ are flat, we have a short exact sequence
\begin{es}
0\>Z_\bullet\otimes G\>C_\bullet\otimes G\>B_{\bullet-1}\otimes G\>0
\end{es}
and the associated long exact sequence
\begin{es}
\,\>H_n(B;G)\>H_n(Z;G)\>H_n(C;G)\>H_{n-1}(B;G)\>H_{n-1}(Z;G)\>\,
\end{es}
where the connecting homomomorphisms are of the form $(i_n\colon B_n\to Z_n)\otimes1_G$ (It is better to think diagram chasing than a natural construction).
Since morhpisms in $B$ and $Z$ are zero (if it is not, then the short exact sequence of chain complexes are not exact, we have
\begin{es}
\,\>B_n\otimes G\>Z_n\otimes G\>H_n(C;G)\>B_{n-1}\otimes G\>Z_{n-1}\otimes G\>\,.
\end{es}
Since
\begin{es}
0\>\Tor_1^R(H_n,G)\>B_n\otimes G\>Z_n\otimes G\>H_n\otimes G\>0
\end{es}
for all $n$, the exact sequence splits into short exact sequence by images
\begin{es}
0\>H_n\otimes G\>H_n(C;G)\>\Tor_1^R(H_{n-1},G)\>0.
\end{es}

For splitting,
\end{pf}

\begin{pf}[2]
Since $R$ is PID, we can construct a flat resolution of $G$
\begin{es}
0\>P_1\>P_0\>G\>0.
\end{es}
Since modules in $C_\bullet$ are flat so that the tensor product functors are exact and $P_1\to P_0$ and $P_0\to G$ induce the chain maps, we have a short exact sequence of chain complexes
\begin{es}
0\>C_\bullet\otimes P_1\>C_\bullet\otimes P_0\>C_\bullet\otimes G\>0.
\end{es}
Then, we have the associated long exact sequence
\begin{es}
\,\>H_n(C;P_1)\>H_n(C;P_0)\>H_n(C;G)\>H_{n-1}(C;P_1)\>H_{n-1}(C;P_0)\>\,.
\end{es}
Since flat tensor product functor commutes with homology funtor from chain complexes, we have
\begin{es}
\,\>H_n\otimes P_1\>H_n\otimes P_0\>H_n(C;G)\>H_{n-1}\otimes P_1\>H_{n-1}\otimes P_0\>\,.
\end{es}
Since
\begin{es}
0\>\Tor_1^R(G,H_n)\>H_n\otimes P_1\>H_n\otimes P_0\>H_n\otimes G\>0
\end{es}
for all $n$, the exact sequence splits into short exact sequence by images
\begin{es}
0\>H_n\otimes G\>H_n(C;G)\>\Tor_1^R(G,H_{n-1})\>0.
\end{es}
\end{pf}

Proof 3.
By tensoring $G$, we get the following diagram.
\begin{cd}[row sep={24pt,between origins}, column sep={36pt,between origins}]
H_n\otimes G  \ar{ddr}  &&&& H_{n-1}\otimes G \\
&&&& \\
& \coker\pd_{n+1}\otimes G  \ar[->>]{ddr} && \ker\pd_{n-1}\otimes G  \ar[->>]{uur}\ar{dr} & \\
C_n\otimes G  \ar[->>]{ur}\ar[->>]{drr} &&&& C_{n-1}\otimes G \\
&& \im\pd_n\otimes G  \ar{uur}\ar{urr} && \\
&&&& \\
&\Tor_1(H_{n-1},G)\ar[>->]{uur}&&&
\end{cd}
Every aligned set of consecutive arrows indicates an exact sequence.
Notice that epimorphisms and cokernals are preserved, but monomorphisms and kernels are not.
Especially, $\coker\pd_{n+1}\otimes G=\coker(\pd_{n+1}\otimes1_G)$ is important.

Consider the following diagram.
\begin{cd}[row sep={30pt,between origins}, column sep={60pt,between origins}]
H_n(C;G) \ar[>->]{dr} & H_n\otimes G \ar{d}&&&\\
& \coker\pd_{n+1}\otimes G \ar[->>]{dd}\ar[->>]{ddrr} && \ker\pd_{n-1}\otimes G \ar{dr}{\text{monic!}} & \\
&&&& C_{n-1}\otimes G \\
& \im\pd_n\otimes G \ar{uurr} && \im(\pd_n\otimes1_G) \ar[>->]{ur}\ar[dashed,>->]{uu} & \\
\Tor_1(H_{n-1},G) \ar[>->]{ur} &&&&
\end{cd}
Since $\ker\pd_{n-1}$ is free, 

If we show $\im(\pd_n\otimes1_G)\to\ker\pd_{n-1}\otimes G$ is monic, then we can get
\begin{align*}
H_n(C;G)&=\ker(\coker\pd_{n+1}\otimes G\to\im(\pd_n\otimes1_G))\\
&=\ker(\coker\pd_{n+1}\otimes G\to\ker\pd_{n-1}\otimes G).
\end{align*}





\section{Fundamental differential geometry}

\subsection{Manifold and Atlas}
\begin{defn}
A \emph{locally Euclidean space} $M$ of dimension $m$ is a Hausdorff topological space $M$ for which each point $x\in M$ has a neighborhood $U$ homeomorphic to an open subset of $\R^d$.
\end{defn}
\begin{defn}
A \emph{manifold} is a locally Euclidean space satisfying the one of following equivalent conditions: second countability, blabla%
\end{defn}

\begin{defn}
A \emph{chart} or a \emph{coordinate system} for a locally Euclidean space is a map $\varphi$ is a homeomorphism from an open set $U\subset M$ to an open subset of $\R^d$.
A chart is often written by a pair $(U,\varphi)$.
\end{defn}

\begin{defn}
An \emph{atlas} $\mathcal{F}$ is a collection $\mathcal{F}=\{(U_\alpha,\varphi_\alpha)\mid\alpha\in A\}$ of charts on $M$ such that $\bigcup_{\alpha\in A} U_\alpha=M$.
\end{defn}


\begin{defn}
A \emph{differentiable maifold} is a manifold on which a differentiable structure is equipped.
\end{defn}
The definition of differentiable structure will be given in the next subsection.
Actually, a differentiable structure can be defined for a locally Euclidean space.



\subsection{Definition of Differentiable Structure}


\begin{defn}
An atlas $\mathcal{F}$ is called \emph{differentiable} if any two charts $\varphi_\alpha,\varphi_\beta\in\mathcal{F}$ is \emph{compatible}: each \emph{transition function} $\tau_{\alpha\beta}\colon\varphi_\alpha(U_\alpha\cap U_\beta)\to\varphi_\beta(U_\alpha\cap U_\beta)$ which is defined by $\tau_{\alpha\beta}=\varphi_\beta\circ\varphi_\alpha^{-1}$ is differentiable.
\end{defn}
It is called a \emph{gluing condition}.

\begin{defn}
For two differentiable atlases $\mathcal{F},\mathcal{F}'$, the two atlases are \emph{equivalent} if $\mathcal{F}\cup\mathcal{F}'$ is also differentiable.
\end{defn}

\begin{defn}
An differentiable atlas $\mathcal{F}$ is called \emph{maximal} if the following holds:
if a chart $(U,\varphi)$ is compatible to all charts in $\mathcal{F}$, then $(U,\varphi)\in\mathcal{F}$.
\end{defn}

\begin{defn}
A \emph{differentiable structure} on $M$ is a maximal differentiable atlas.
\end{defn}

To differentiate a function on a flexible manofold, first we should define the differentiability of a function.
A differentiable structure, which is usually defined by a maximal differentiable atlas, is roughly a collection of differentiable functions on $M$.
When the charts is already equipped on $M$, it is natural to define a function $f\colon M\to\R$ differentiable if the functions $f\circ\varphi^{-1}\colon\R^d\to\R$ is differentiable.

The gluing condition makes the differentiable function for a chart is also differentiable for any charts because $f\circ\varphi_\alpha^{-1}=(f\circ\varphi_\beta^{-1})\circ(\varphi_\beta\circ\varphi_\alpha^{-1})
=(f\circ\varphi_\beta^{-1})\circ\tau_{\alpha\beta}$.
If a function $f$ is differentiable on an atlas $\mathcal{F}$, then $f$ is also differentiable on any atlases which is equivalent to $\mathcal{F}$ by the definition of the equivalence relation for differential atlases.
We can construct the equivalence classes respected to this equivalence relation.

Therefore, we want to define a differentiable structure as a one of the equivalence classes.
However the differentiable structure is frequently defined as a maximal atlas for the convenience since each equivalence class is determined by a unique maximal atlas.

\begin{ex}
While the circle $S^1$ has a unique smooth structure, $S^7$ has 28 smooth structures.
The number of smooth structures on $S^4$ is still unknown.
\end{ex}

\begin{defn}
A continuous function $f\colon M\to N$ is differentiable if $\psi\circ f\circ\varphi^{-1}$ is differentiable for charts $\varphi,\psi$ on $M,N$ respectively.
\end{defn}


\subsection{Curves}

\begin{defn}
For $f\colon M\to\R$ and $(U,\phi)$ a chart,
\[df\left(\pd{x^\mu}\right):=\pd{f\circ\phi^{-1}}{x^\mu}.\]
\end{defn}



\begin{defn}
Let $\gamma\colon I\to M$ be a smooth curve.
Then, $\dot\gamma(t)$ is defined by a tangent vector at $\gamma(t)$ such that
\[\dot\gamma(t):=d\gamma\left(\pd{t}\right).\]
Let $\phi\colon M\to N$ be a smoth map.
Then, $\phi(t)$ can refer to a curve on $N$ such that
\[\phi(t):=\phi(\gamma(t)).\]
Let $f\colon M\to\R$ be a smooth function.
Then, $\dot f(t)$ is defined by a function $\R\to\R$ such that
\[\dot f(t):=\dd{t}f\circ\gamma.\]
\end{defn}

\begin{prop}
Let $\gamma\colon I\to M$ be a smooth curve on a manifold $M$.
The notation $\dot\gamma^\mu$ is not confusing thanks to
\[(\dot\gamma)^\mu=\dot{(\gamma^\mu)}.\]
In other words,
\[dx^\mu(\dot\gamma)=\dd{t}x^\mu\circ\gamma.\]
\end{prop}


\subsection{Connection computation}

\begin{align*}
\nabla_XY&=X^\mu\nabla_\mu(Y^\nu\pd_\nu)\\
&=X^\mu(\nabla_\mu Y^\nu)\pd_\nu+X^\mu Y^\nu(\nabla_\mu\pd_\nu)\\
&=X^\mu\left(\pd{Y^\nu}{x^\mu}\right)\pd_\nu+X^\mu Y^\nu(\Gamma_{\mu\nu}^\lambda\pd_\lambda)\\
&=X^\mu\left(\pd{Y^\nu}{x^\mu}+\Gamma_{\mu\lambda}^\nu Y^\lambda\right)\pd_\nu.
\end{align*}
The covariant derivative $\nabla_XY$ does not depend on derivatives of $X^\mu$.

\[Y^\nu_{,\mu}=\nabla_\mu Y^\nu=\pd{Y^\nu}{x^\mu},\qquad Y^\nu_{;\mu}=(\nabla_\mu Y)^\nu=\pd{Y^\nu}{x^\mu}+\Gamma_{\mu\lambda}^\nu Y^\lambda.\]
\begin{thm}
For Levi-civita connection for $g$,
\[\Gamma^l_{ij}=\frac12(\pd_ig_{jk}+\pd_jg_{ki}-\pd_kg_{ij}).\]
\end{thm}
\begin{pf}
\begin{align*}
(\nabla_ig)_{jk}&=\pd_ig_{jk}-\Gamma^l_{ij}g_{lk}-\Gamma^l_{ik}g_{jl}\\
(\nabla_jg)_{kl}&=\pd_jg_{kl}-\Gamma^l_{jk}g_{li}-\Gamma^l_{ji}g_{kl}\\
(\nabla_kg)_{ij}&=\pd_kg_{ij}-\Gamma^l_{ki}g_{lj}-\Gamma^l_{kj}g_{il}\\
\end{align*}
If $\nabla$ is a Levi-civita connection, then $\nabla g=0$ and $\Gamma_{ij}^k=\Gamma_{ji}^k$.
Thus,
\[\Gamma^l_{ij}g_{kl}=\frac12(\pd_ig_{jk}+\pd_jg_{ki}-\pd_kg_{ij}).\]
\[\Gamma^l_{ij}=\frac12g^{kl}(\pd_ig_{jk}+\pd_jg_{ki}-\pd_kg_{ij}).\]
\end{pf}



\subsection{Geodesic equation}

\begin{thm}
If $c$ is a geodesic curve, then components of $c$ satisfies a second-order differential equation
\[\dd[2]{\gamma^\mu}{t}+\Gamma_{\nu\lambda}^\mu\dd{\gamma^\nu}{t}\dd{\gamma^\lambda}{t}=0.\]
\end{thm}
\begin{pf}
Note
\[0=\nabla_{\dot\gamma}\dot\gamma=\dot\gamma^\mu\nabla_\mu(\dot\gamma^\lambda\pd_\lambda)
=(\dot\gamma^\nu\pd_\nu\dot\gamma^\mu+\dot\gamma^\nu\dot\gamma^\lambda\Gamma_{\nu\lambda}^\mu)\pd_\mu.\]
Since
\[\dot\gamma^\nu\pd_\nu\dot\gamma^\mu=\dot\gamma(\dot\gamma^\mu)=d\dot\gamma^\mu(\dot\gamma)=d\dot\gamma^\mu\circ d\gamma\left(\pd{t}\right)=d\dot\gamma^\mu\left(\pd{t}\right)=\ddot\gamma^\mu,\]
we get a second-order differential equation
\[\dd[2]{\gamma^\mu}{t}+\Gamma_{\nu\lambda}^\mu\dd{\gamma^\nu}{t}\dd{\gamma^\lambda}{t}=0\]
for each $\mu$.
\end{pf}


\section{Vector calculus on spherical coordinates}
\[
\begin{tikzcd}[row sep=5pt, column sep=3pt]
V&=&(V_r,V_\theta,V_\phi)&&&&&&&&\\
&=&V_r&\hat{r}&+&V_\theta&\hat{\theta}&+&V_\phi&\hat{\phi}&(\text{normalized coords})\\
&=&V_r&\pd{r}&+&\frac1r\ V_\theta&\pd{\theta}&+&\frac1{r\sin\theta}\ V_\phi&\pd{\phi}&(\Gamma(TM))\\
&=&V_r&dr&+&r\ V_\theta&d\theta&+&r\sin\theta\ V_\phi&d\phi&(\Omega^1(M))\\
&=&r^2\sin\theta\ V_r&d\theta\wedge d\phi&+&r\sin\theta\ V_\theta&d\phi\wedge dr&+&r\ V_\phi&dr\wedge d\theta&(\Omega^2(M)).
\end{tikzcd}
\]

\[
\nabla\cdot V=\frac1{r^2\sin\theta}\left[  \pd{r}\left(r^2\sin\theta\ V_r\right) + \pd{\theta}\left(r\sin\theta\ V_\theta\right) + \pd{\phi}\left(r\ V_\phi\right)  \right]
\]
\[\Delta u=\frac1{r^2\sin\theta}\left[\pd{r}\left(r^2\sin\theta\pd{r}u\right)+\pd{\theta}\left(\sin\theta\pd{\theta}u\right)+\pd{\phi}\left(\frac1{\sin\theta}\pd{\phi}u\right)\right]\]

Let $(\xi,\eta,\zeta)$ be an orthogonal coordinate that is \emph{not} normalized.
Then,
\[\sharp=g=\diag(\|\pd_\xi\|^2,\|\pd_\eta\|^2,\|\pd_\zeta\|^2)\]
\[\hat x=\|\pd_x\|^{-1}\ \pd_x=\|\pd_x\|\ dx=\|\pd_y\|\|\pd_z\|\ dy\wedge dz\]
In other words, we get the normalized differential forms in sphereical coordinates as follows:
\[dr,\quad r\,d\theta,\quad r\sin\theta\,d\phi,\quad(r\,d\theta)\wedge(r\sin\theta\,d\phi),\quad(r\sin\theta\,d\phi)\wedge(dr),\quad(dr)\wedge(r\,d\theta).\]

\begin{align*}
\grad:\nabla&=\left[\ \frac1{\|\pd_x\|}\pd{x}\cdot-\ ,\ \frac1{\|\pd_y\|}\pd{y}\cdot-\ ,\ \frac1{\|\pd_z\|}\pd{z}\cdot-\ \right]\\
\curl:\nabla&=\left[\ \frac1{\|\pd_y\|\|\pd_z\|}\left(\pd{y}(\|\pd_z\|\cdot-)-\pd{z}(\|\pd_y\|\cdot-)\right)\ ,\right.\\
&\qquad\left.\frac1{\|\pd_z\|\|\pd_x\|}\left(\pd{z}(\|\pd_x\|\cdot-)-\pd{x}(\|\pd_z\|\cdot-)\right)\ ,\right.\\
&\qquad\left.\frac1{\|\pd_x\|\|\pd_y\|}\left(\pd{x}(\|\pd_y\|\cdot-)-\pd{y}(\|\pd_x\|\cdot-)\right)\ \right]\\
\div:\nabla&=\frac1{\|\pd_x\|\|\pd_y\|\|\pd_z\|}\left[\ \pd{x}\bigl(\|\pd_y\|\|\pd_z\|\cdot-\bigr)\ ,\ \pd{y}\bigl(\|\pd_z\|\|\pd_x\|\cdot-\bigr)\ ,\ \pd{z}\bigl(\|\pd_x\|\|\pd_y\|\cdot-\bigr)\ \right]\\
\Delta&=\frac1{\|\pd_x\|\|\pd_y\|\|\pd_z\|}\left[\ \pd{x}\left(\frac{\|\pd_y\|\|\pd_z\|}{\|\pd_x\|}\pd{x}\right)+\pd{y}\left(\frac{\|\pd_z\|\|\pd_x\|}{\|\pd_y\|}\pd{y}\right)+\pd{z}\left(\frac{\|\pd_x\|\|\pd_y\|}{\|\pd_z\|}\pd{z}\right)\ \right]
\end{align*}

\bigskip

\[\grad=\frac1{\|\cdot\|^1}\,(\nabla)\,\|\cdot\|^0\]
\[\curl=\frac1{\|\cdot\|^2}\,(\nabla\times)\,\|\cdot\|^1\]
\[\div=\frac1{\|\cdot\|^3}\,(\nabla\cdot)\,\|\cdot\|^2\]




\section{Bundles}
Show that $S^n$ has a nonvanishing vector field if and only if $n$ is odd.
\begin{sol}
Since $S^n$ is embedded in $\R^{n+1}$, the tangent bundle $TS^n$ can be considered as an embedded manifold in $S^n\times\R^{n+1}$ which consists of $(x,v)$ such that $\<x,x\>=1$ and $\<x,v\>=0$, where the inner product is the standard one of $\R^{n+1}$.

Suppose $n$ is odd.
We have a vector field
$(x_1,x_2,\cdots,x_{n+1};x_2,-x_1,\cdots,-x_n)$
which is nonvanishing.

Conversely, suppose we have a nonvanishing vector field $X$.
Consider a map
\[\phi:S^n\xrightarrow{X}TS^n\to S^n\times\R^{n+1}\to{\phi}\R^{n+1}\to S^n.\]

The last map can be defined since $X$ is nowhere zero. Since this map satisfies $\<x,\phi(x)\>=0$ for all $x\in S^n$, we can define homotopies from $\phi$ to the identity map and the antipodal map respectively. Therefore, the antipodal map must have positive degree, $+1$, so $n$ is odd.
\end{sol}


\begin{prop}
Independent commuting vector fields are realized as partial derivatives in a chart.
\end{prop}

\begin{prop}
Let $\{\pd_1,\cdots,\pd_k\}$ be an independent involutive vector fields.
We can find independent commuting $\{\pd_{k+1},\cdots,\pd_n\}$ such that union is independent.
(Maybe)
\end{prop}
\begin{prop}
Let $\{\pd_1,\cdots,\pd_k\}$ be an independent commuting vector fields.
We can find independent commuting $\{\pd_{k+1},\cdots,\pd_n\}$ such that union is independent and commuting.
(Maybe)
\end{prop}

\bigskip



The following theorem says that image of immersion is equivalent to kernel of submersion.
\begin{prop}
An immersed manifold is locally an inverse image of a regular value.
\end{prop}

\begin{prop}
A closed submanifold with trivial normal bundle is globally an inverse image of a regular value.
\end{prop}
\begin{pf}
It uses tubular neighborhood.
Pontryagin construction?
\end{pf}

\begin{prop}
An immersed manifold is locally a linear subspace in a chart.
\end{prop}

\begin{prop}
Distinct two points on a connected manifold are connected by embedded curve.
\end{prop}
\begin{pf}
Let $\gamma:I\to M$ be a curve connecting the given two points, say $p,q$.

\Step[1]{Constructing a piecewise linear curve}
For $t\in I$, take a convex chart $U_t$ at $\gamma(t)$.
Since $I$ is compact, we can choose a finite $\{t_i\}_i$ such that $\bigcup_i\gamma^{-1}(U_{t_i})=I$.
This implies $\im\gamma\subset\bigcup_iU_{t_i}$.
Reorganize indices such that $\gamma(t_1)=p$, $\gamma(t_n)=q$, and $U_{t_i}\cap U_{t_{i+1}}\ne\varnothing$ for all $1\le i\le n-1$.
It is possible since the graph with $V=\{i\}_i$ and $E=\{(i,j):U_{t_i}\cap U_{t_j}\ne\varnothing$ is connected.
Choose $p_i\in U_{t_i}\cap U_{t_{i+1}}$ such that they are all dis for $1\le i\le n-1$ and let $p_0=p$, $p_n=q$.

How can we treat intersections?

Therefore, we get a piecewise linear curve which has no self intersection from $p$ to $q$.

\Step[2]{Smoothing the curve}
\end{pf}

\begin{prop}
Let $M$ is an embedded manifold with boundary in $N$.
Any kind of sections on $M$ can be extended on $N$.
\end{prop}

\begin{prop}
Every ring homomorphism $C^\infty(M)\to\R$ is obtained by an evaluation at a point of $M$.
\end{prop}
\begin{pf}
Suppose $\phi:C^\infty(M)\to\R$ is not an evaluation.
Let $h$ be a positive exhaustion function.
Take a compact set $K:=h^{-1}([0,\phi(h)])$.
For every $p\in K$, we can find $f_p\in C^\infty(M)$ such that $\phi(f_p)\ne f_p(p)$ by the assumption.
Summing $(f_p-\phi(f_p))^2$ finitely on $K$ and applying the extreme value theorem, we obtain a function $f\in C^\infty(M)$ such that $f\ge0$, $f|_K>1$, and $\phi(f)=0$.
Then, the function $h+\phi(h)f-\phi(h)$ is in kernel of $\phi$ although it is strictly positive and thereby a unit.
It is a contradiction.
\end{pf}


\begin{prop}
The set of points that is geodesically connected to a point is open.
\end{prop}





\end{document}