\documentclass[11pt]{amsart}
\usepackage{../../ikany}
\usepackage[margin=1.5in]{geometry}

\title{Vlasov-Poisson system}
\author{Ikhan Choi}

\def\tint{{\textstyle\int}}
\def\loc{\mathrm{loc}}
\setlist[cond,1]{label=\rm{\roman*.}}

\begin{document}
\maketitle
\tableofcontents

\section{Vlasov-Poisson system}
Consider the following Cauchy problem for the \emph{Valsov-Poisson system}:
\begin{pde}
&f_t+v\cdot\del_xf+\gamma E\cdot\del_vf=0,\:(t,x,v)\in\R_t^+\x\R_x^3\x\R_v^3,\\
&E(t,x)=-\del_x\Phi,\\
&\Phi(t,x)=(-\Delta_x)^{-1}\rho,\\
&\rho(t,x)=\tint f\,dv,\\
&f(0,x,v)=f_0(x,v)\ge0,
\end{pde}
where $\gamma=\pm1$.
For example, we have \emph{repulsive problem} $\gamma=+1$ for electrons in plasma theory and \emph{attractive problems} $\gamma=-1$ for galactic dynamics.
($\rho$ denotes mass density.)

This report mainly investigates the local and global existence problem of the Cauchy problem for the Vlasov-Poisson system.
Results in 1.1 and 1.2 provide basic ingredients that will be used in the whole article.
On the other hand, results in 1.3 cannot be used in any local existence proof because they assume the existence of solutions, but they help understand the fundamental nature of solutions of the Vlasov-Poisson system and are used in the proof of global existence.

\begin{notn*}
We use the asymptotic notation
\[g(t)\les h(t)\iff\exists\,c=c(f_0),\quad g(t)\le c\,h(t)\]
and
\[g(t)\simeq h(t)\iff\exists\,c,\quad g(t)=c\,h(t).\]

This report does not contain any of Sobolev norms.
We omit marginalized variables and the $L$ character for subscript.
For example,
\[\|f(t)\|_p=(\iint|f(t,x,v)|^p\,dv\,dx)^{1/p},\quad\|\rho(t)\|_p=(\int|\rho(r,x)|^p\,dx)^{1/p}.\]
\end{notn*}



\subsection{Poisson equation}
For the three-dimensional boundaryless problem of the Poisson equation
\[-\Delta\Phi(x)=\rho(x)\]
in which the solution $\Phi$ vanishes at infinity, we have
\[\Phi=\tfrac1{4\pi|x|}*\rho,\]
so the electric field in the Vlasov-Poisson system is given by
\[E=-\del_x\Phi=-\del_x(\tfrac1{4\pi|x|}*\rho)=\frac{x}{4\pi|x|^3}*\rho.\]
It can be rewritten as
\[E(t,x)=\frac1{4\pi}\int\frac{(x-y)\rho(t,y)}{|x-y|^3}\,dy.\]

The nonlinearity of the system is originated from the force field $E$, so its estimates play the most important role in investigation of the nonlinear system.
Since it is given by the solution of the Poisson equation, estimates of the Riesz potential is directly connected to estimates of the force field.


\begin{lem}[Estimates of Riesz potential]
Let $\rho\in C_c^1(\R^d)$.
\begin{cond}
\item(Field estimate)
\[\|\tfrac1{|x|^{d-1}}*\rho\|_\infty\les\|\rho\|_\infty^{1-1/d}\|\rho\|_1^{1/d}\]
\item(Field derivative estimate)
For $\log^+(x):=\max\{0,\log x\}$,
\[\|\del(\tfrac1{|x|^{d-1}}*\rho)\|_\infty\les1+\|\rho\|_\infty\log^+\|\del\rho\|_\infty+\|\rho\|_1.\]
\end{cond}
\end{lem}
\iffalse
\begin{rmk}
Note that the constants hidden in symbol $\les$ in these estimate cannot depend on $f_0$ since the theorem is actually not about the Vlasov-Poisson system but the Poisson equation; the constant only depends on the dimension $d=3$.
\end{rmk}
\fi

\begin{pfs}
\item
Let $0\le\frac1p<\frac\alpha d<\frac1q\le1$.
Since $(d-\alpha)p<d<(d-\alpha)q$,
\begin{align*}
|\tfrac1{|x|^{d-\alpha}}*\rho|
&=\int_{|x-y|<R}\frac{\rho(y)}{|x-y|^{d-\alpha}}\,dy+\int_{|x-y|\ge R}\frac{\rho(y)}{|x-y|^{d-\alpha}}\,dy\\
&\le\|\rho\|_{p'}(\int_{|y|<R}\frac{dy}{|y|^{(d-\alpha)p}})^{1/p}+\|\rho\|_{q'}(\int_{|y|\ge R}\frac{dy}{|y|^{(d-\alpha)q}})^{1/q}\\
&\simeq\|\rho\|_{p'}(\int_0^Rr^{d-1-(d-\alpha)p}\,dr)^{1/p}+\|\rho\|_{q'}(\int_R^\infty r^{d-1-(d-\alpha)q}\,dr)^{1/q}\\
&\simeq\|\rho\|_{p'}R^{\frac dp-d+\alpha}+\|\rho\|_{q'}R^{\frac dq-d+\alpha}.
\end{align*}
By choosing $R$ such that $\|\rho\|_{p'}R^{\frac dp-d+\alpha}=\|\rho\|_{q'}R^{\frac dq-d+\alpha}$, we get
\[\|\tfrac1{|x|^{d-\alpha}}*\rho\|_\infty\les\|\rho\|_{p'}^{\frac{1-\frac\alpha d-\frac1q}{\frac1p-\frac1q}}\|\rho\|_{q'}^{\frac{\frac1p-1+\frac\alpha d}{\frac1p-\frac1q}},\]
so the inequality
\[\|\tfrac1{|x|^{d-\alpha}}*\rho\|_\infty^{\frac1q-\frac1p}\les\|\rho\|_p^{\frac1q-\frac\alpha d}\|\rho\|_q^{\frac\alpha d-\frac1p}\]
is obtained by interchaning $p$ and $q$ with their conjugates.
The desired result gets $p=\infty$, $\alpha=1$, and $q=1$.

\item
Let $0<R_a\le R_b$ be constants which will be determined later.
Divide the region radially
\begin{align*}
|\del(\tfrac1{|x|^{d-1}}*\rho)|\les\del\int_{|x-y|<R_a}+\del\int_{R_a\le|x-y|<R_b}+\del\int_{R_b\le|x-y|}.
\end{align*}
For the first integral,
\begin{align*}
\int_{|y|<R_a}\frac{\del\rho(x-y)}{|y|^{d-1}}\,dy
&\le\|\del\rho\|_\infty\int_{|y|<R_a}\frac1{|y|^{d-1}}\,dy\\
&\simeq\|\del\rho\|_\infty\int_0^{R_a}1\,dr
=R_a\|\del\rho\|_\infty.
\end{align*}
For the second integral,
\begin{align*}
\int_{R_a\le|x-y|<R_b}\frac{\rho(y)}{|x-y|^d}\,dy
&\le\|\rho\|_\infty\int_{R_a\le|x-y|<R_b}\frac1{|x-y|^d}\,dy\\
&\simeq\|\rho\|_\infty\int_{R_a}^{R_b}\frac1r\,dr
=(\log\tfrac{R_b}{R_a})\|\rho\|_\infty.
\end{align*}
For the third integral,
\[\int_{R_b\le|x-y|}\frac{\rho(y)}{|x-y|^d}\,dy\le R_b^{-d}\|\rho\|_1.\]
Thus,
\[|\del(\tfrac1{|x|^{d-1}}*\rho)|\les R_a\|\del\rho\|_\infty+(\log\tfrac{R_b}{R_a})\|\rho\|_\infty+R_b^{-d}\|\rho\|_1.\]

Assuming $\rho$ is nonzero so that $\|\del\rho\|_\infty>0$, let $R_a=\min\{1,\|\del\rho\|_\infty^{-1}\}$ and $R_b=1$.
Since
\[\log\tfrac1{R_a}\le\log^+\|\del\rho\|_\infty\quad\text{and}\quad R_a\les\|\del\rho\|_\infty,\]
we have
\[\|\del(\tfrac1{|x|^{d-1}}*\rho)\|_\infty\les1+\|\rho\|_\infty\log^+\|\del\rho\|_\infty+\|\rho\|_1.\qedhere\]
\end{pfs}

\subsection{Characteristics and volume preservation}

The Vlasov-Poisson equation is quite hyperbolic.
What we mean here is that the method of characteristic curves is useful, and it does not regularizes the solution directly.
Although we do not have an explicit representation formula, solutions given by characteristics make appropriate estimates possible.

Moreover, since the Vlasov-Poisson system is a Hamiltonian system on the phase space $\R_x^3\x\R_v^3$ with the Hamiltonian $H(x,v)=\frac12v^2+\gamma\Phi(x,v)$, it has the volume preserving propoerty.
We, however, will show the volume preservation by computation of the Jacobian determinant for transformations through characteristic flows.

\begin{lem}
Let $f\in C^1([0,T],C_c^1(\R^6))$ be a solution of the Vlasov-Poisson system.
\begin{cond}
\item Fix $t,x,v$. The ordinary differential equation
\begin{gather*}
\dot X(s;t,x,v)=V(s;t,x,v),\quad\dot V(s;t,x,v)=\gamma E(t,X(s;t,x,v)),\\
X(t;t,x,v)=x,\qquad V(t;t,x,v)=v
\end{gather*}
with time variable $s$ has a solution $(X,V)$ in $C^1([0,T],\R^6)$.
\item Fix $t,x,v$. Then, $f(s,X(s;t,x,v),V(s;t,x,v))=\const$.
\item Fix $t$ and let
\[y(s,x,v):=X(s;t,x,v)\quad\text{and}\quad w(s,x,v):=V(s;t,x,v).\]
Then, the Jacobian of coordinates transform $(x,v)\mapsto(y,w)$ is 1 for all $s$.
\end{cond}
\end{lem}
\begin{pfs}
\item
Note that we have
\[\rho\in C^1([0,T];C_c^1(\R^6)),\quad\Phi\in C^1([0,T];C^{2,\alpha}(\R^6))\]
so that
\[E\in C^1([0,T];C^{1,\alpha}(\R^6))\]
by the H\"older regularity of the Poisson equation.
Since a map
\[(x,v)\mapsto(v,\gamma E(t,x))\]
is globally Lipschitz with respect to $(x,v)$ for each $t$, we can apply the Picard Lindel\"of theorem.

\item
Differentiate and use the chain rule to get
\begin{align*}
\dd{s}&f(s,y,w)\\
&=\pd_tf(s,y,w)+\dot X(s;s,y,w)\cdot\del_xf(s,y,w)+\dot V(s;s,y,w)\cdot\del_vf(s,y,w)\\
&=\pd_tf(s,y,w)+w\cdot\del_xf(s,y,w)+\gamma E(s,y)\cdot\del_vf(s,y,w)=0,
\end{align*}
where we denote $y=X(s;t,x,v)$ and $w=V(s;t,x,v)$.

\item
Let $J(s)=\pd{(y,w)}{(x,v)}$ be the Jacobi matrix.
Because when $s=t$ the Jacobian is
\[\det J(t)=\det\pd{(x,v)}{(x,v)}=1,\]
We want to show
\[\det J(s)=\const.\]
Since
\[J^{-1}(s)\dd{s}J(s)=\pd{(x,v)}{(y,x)}\dd{s}\pd{(y,w)}{(x,v)}=\pd{(\dot y,\dot w)}{(y,w)}=\mat{0&1\\\gamma\pd{E}{y}(s,y)&0},\]
the Jacobi formula deduces that
\[\dd{s}\det J(s)=\det(s)\tr\left(J^{-1}(s)\dd{s}J(s)\right)=0.\qedhere\]
\end{pfs}

\begin{cor}
Let $f\in C^1([0,T],C_c^1(\R^6))$ be a solution of the Cauchy problem for the Vlasov-Poisson system.
Then, for any measurable function $\beta:\R\to\R$ such that $\iint\beta\o f_0(x,v)\,dv\,dx<\infty$, we have
\[\iint\beta\o f(t,x,v)\,dv\,dx=\const.\]
In particular,
\[\|f(t)\|_p=\const\]
for $1\le p\le\infty$.
\end{cor}
\begin{pf}
Fix $t,s[0,T]$ and denote $y=X(s;t,x,v)$ and $w=V(s;t,x,v)$.
Then,
\begin{align*}
\iint\beta\o f(t,x,v)\,dv\,dx
&=\iint\beta\o f(s,X(s;t,x,v),V(s;t,x,v))\,dv\,dx\\
&=\iint\beta\o f(s,y,w)\,dw\,dy
\end{align*}
for $s\le T$.
\end{pf}
\begin{rmk}
Note that this result can be obtained in the approximation scheme, which will be suggested in the next section.
\end{rmk}

To sum up our weapons obtained in 1.1 and 1.2,
\begin{cor}
If a function $f\in C^1([0,T],C_c^1(\R^6))$ satisfies
\[\iint f(t,x,v)\,dv\,dx=\const,\]
and if we let
\[\rho(t,x)=\int f(t,x,v)\,dv,\quad E(t,x)=\frac1{4\pi}\int\frac{(x-y)\rho(t,y)}{|x-y|^3}\,dy,\]
then
\begin{cond}
\item $\|\rho(t)\|_1=\const$,
\item $\|E(t)\|_\infty\les\|\rho(t)\|_\infty^{2/3}$,
\item $\|\del E(t)\|_\infty\les1+\|\rho\|_\infty\log^+\|\del\rho\|_\infty$.
\end{cond}
\end{cor}
These estimates will be applied not only to the global existence proof, which assumes the local existence, but also to approximate solutions.



\subsection{Conservation laws and moment propagation}
Usual algebraic computations with Stokes' theorem get several conservations laws, particularly including energy conservation.

\begin{lem}
Let $f\in C^1([0,T],C_c^1(\R^6))$ be a solution of the Vlasov-Poisson system.
\begin{cond}
\item(Continuity equation)
\[\rho_t+\del_x\cdot j=0,\qquad\text{where}\quad j=\int vf\,dv.\]
\item(Energy conservation)
\[\iint|v|^2f\,dv\,dx+\gamma\int|E|^2\,dx=\const.\]
\end{cond}
\end{lem}
\begin{pfs}
\item
Integrate with respect to $v$ to get
\begin{align*}
0&=\int f_t\,dv+\int v\cdot\del_xf\,dv\\
&=\rho_t+\del_x\cdot\int vf\,dv\\
&=\rho_t+\del_x\cdot j.
\end{align*}
\item
Multiply $|v|^2$ and integrate with respect to $v$ and $x$ to get
\begin{align*}
\dd{t}\iint|v|^2f\,dv\,dx
&=\iint|v|^2f_t\,dv\,dx=-\iint|v|^2\gamma E\cdot\del_vf\,dv\,dx\\
&=\iint2v\cdot\gamma Ef\,dv\,dx=-2\gamma\int\del_x\Phi\cdot j\,dx\\
&=2\gamma\int\Phi\del_x\cdot j\,dx=2\gamma\int\Phi\Delta_x\Phi_t\,dx\\
&=-\dd{t}\gamma\int|E|^2\,dx.
\end{align*}
Thus
\[\iint|v|^2f\,dv\,dx+\gamma\int|E|^2\,dx=\const.\qedhere\]
\end{pfs}


Moments are quantities of the form
\[\iint|v|^kf(t,x,v)\,dv\,dx\]
for a positive real $k$.
The energy conservation proves the bound of the 2-moment, which is also called kinetic energy,
\[\iint|v|^2f(t,x,v)\,dv\,dx\les1\]
if $\gamma=+1$.
In fact, a bound of kinetic energy exists even for $\gamma=-1$.
As a corollary, the $L^{5/3}$ norm of mass density $\|\rho\|_{5/3}$ gets bounded.

\begin{lem}[Bound for kinetic energy]
Let $f\in C^1([0,T],C_c^1(\R^6))$ be a solution of the Vlasov-Poisson system.
For $t\in[0,T]$,
\begin{cond}
\item $\|\rho(t)\|_{5/3}\les\iint|v|^2f\,dv\,dx$.
\item $\iint|v|^2f\,dv\,dx\les1$.
\end{cond}
\end{lem}
\begin{pfs}
\item
Note
\begin{align*}
\rho(t,x)=\int f(t,x,v)\,dv
&\le\int_{|v|<R}f\,dv+\frac1{R^2}\int_{|v|\ge R}|v|^2f\,dv\\
&\les R^3+ R^{-2}\int|v|^2f\,dv.
\end{align*}
Set $R^3=R^{-2}\int|v|^2f\,dv$ to get
\[\rho(t,x)^{5/3}\les\int|v|^2f\,dv.\]

\item
It is trivial for $\gamma=+1$ from the energy conservation.
Suppose $\gamma=-1$.
By the Hardy-Littlewood-Sobolev inequality,
\[\frac1p+\frac\alpha d=\frac1q\]
for $p=2$, $d=3$, and $\alpha=1$ implies $q=6/5$, hence the bound of $\|E(t)\|_2$
\[\|E(t)\|_2\simeq\|\frac1{|x|^{d-\alpha}}*_x\rho(t,x)\|_{L_x^2}\les\|\rho(t)\|_{6/5}.\]
So, interpolation with H\"older's inequality gives
\[\|E(t)\|_2\les\|\rho\|_1^{7/12}\|\rho\|_{5/3}^{5/12}\simeq\|\rho\|_{5/3}^{5/12}.\]
Thus (1) gives
\[\iint|v|^2f\,dv\,dx=c+\|E(t)\|_2^2\les c+(\iint|v|^2f\,dv\,dx)^{1/2},\]
so the kinetic energy $\iint f\,dv\,dx$ is bounded.\qedhere
\end{pfs}

If we justify a bound of higher moment
\[\iint|v|^kf(t,x,v)\,dv\,dx\les1\]
for some $k>6$ so that we have $\|\rho(t)\|_p\les1$ for some $p=\frac{k+3}3>3$, then we obtain
\[\|E(t)\|_\infty^{1-\frac1p}\les\|\rho\|_p^{\frac23}\|\rho\|_1^{\frac13-\frac1p}\les1.\]
We will see that this estimate proves the global existence immediately; this is the idea of the paper of Lions and Perthame\cite{}.
We do not cover this in detail.





\section{Local existence}

The proof of local existence follows the following steps:
\begin{enumerate}
\item construction of an approximate solution,
\item establishment of a priori estimates,
\item (subsequential) convergence of the approximate solution,
\item verification of the solvability for the limit.
\end{enumerate}
The Vlasov-Poisson system is good enough to show cdirect convergence of approximate solutions, not in the sense of subsequences.


\subsection{Approximate solution}

\begin{defn}
We define an (global) \emph{approximate solution} as a sequence of functions $f_n\in C^1(\R^+,C_c^1(\R^6))$ such that
\begin{pde*}
&\pd_tf_{n+1}+v\cdot\del_xf_{n+1}+\gamma E_n\cdot\del_vf_{n+1}=0,\\
&E_n(t,x)=-\del_x\Phi_n,\\
&\Phi_n(t,x)=(-\Delta_x)^{-1}\rho_n,\\
&\rho_n(t,x)=\tint f_n\,dv,\\
&f_{n+1}(0,x,v)=f_0(x,v).
\end{pde*}
This definition is made in order to let the force field $E$ constant when solving $f_{n+1}$.
\end{defn}


\begin{prop}
An approximate solution exists.
\end{prop}
\begin{pf}
Let $f_0(t,x,v)=f_0(x,v)$.
Notice that $f_0$ is clearly in $C^1(\R^+;C_c^1(\R^6))$.
Assume $f_n\in C^1(\R^+;C_c^1(\R^6))$ satisfies the approximate system.
We want to show that there is $f_{n+1}$ that satisfies the approximate system and $f_{n+1}\in C^1(\R^+;C_c^1(\R^6))$.

We have
\[\rho_n\in C^1(\R^+;C_c^1(\R^6)),\quad\Phi_n\in C^1(\R^+;C^{2,\alpha}(\R^6)),\ \text{and}\ E_n\in C^1(\R^+;C^{1,\alpha}(\R^6))\]
by the H\"older regularity of the Poisson equation.
Since a map $(x,v)\mapsto(v,\gamma E_n(t,x))$ is globally Lipschitz with respect to $(x,v)$ for each $t$, the classical Picard iteration uniquely solves the characteristic equation
\begin{pde*}
\dot X_{n+1}(s;t,x,v)&=V_{n+1}(s,t,x,v)\\
\dot V_{n+1}(s;t,x,v)&=\gamma E_n(s,X_{n+1}(s;t,x,v))
\end{pde*}
with condition $(X_{n+1}(t;t,x,v),V_{n+1}(t;t,x,v))=(x,v)$ and proves the uniqueness and regularity of the solution $s\mapsto(X_{n+1},V_{n+1})(s;t,x,v)\in C^1(\R^+,\R^6)$.

Define
\[f_{n+1}(t,x,v):=f_0(X_{n+1}(0;t,x,v),V_{n+1}(0;t,x,v)).\]
Then, we can show that
\begin{align*}
f_{n+1}(s,X_{n+1}(s;t,x,v),V_{n+1}(s;t,x,v))&\\
=f_0(X_{n+1}(0;t,x,v),V_{n+1}(0;t,x,v))&=\const
\end{align*}
and that $f_{n+1}$ satisfies the approximate system by the chain rule
\begin{align*}
0&=\left.\dd{s}f_{n+1}(s,X_{n+1}(s;t,x,v),V_{n+1}(s;t,x,v))\right|_{s=t}\\
&=\pd_tf_{n+1}(t,x,v)+\dot X_{n+1}(t;t,x,v)\cdot\del_xf_{n+1}(t,x,v)\\
&\hspace{7.5em}+\dot V_{n+1}(t;t,x,v)\cdot\del_vf_{n+1}(t,x,v)\\
&=\pd_tf_{n+1}(t,x,v)+v\cdot\del_xf_{n+1}(t,x,v)+\gamma E_n(t,x)\cdot\del_vf_{n+1}(t,x,v).
\end{align*}
Also, $f_{n+1}$ has compact support for each $t$ since the characteristic does not blow up; finally we have $f_{n+1}\in C^1(\R^+,C_c^1(\R^6))$.
\end{pf}
\begin{rmk}
Although the approximate solution is unique when given the initial term $f_0(t,x,v)=f_0(x,v)$, we do not care of the uniqueness, but only the existence.
\end{rmk}



\subsection{Local a priori estimates}
Firstly, the volume preserving property still holds for our approximate system, so we have
\[\|\rho_n(t)\|_1=\const,\quad\|f_n(t)\|_p=\const.\]
Next, we prove local-time bounds on fields $E_n$.
Introduce the following quantity.
\begin{defn}
Define the \emph{velocity support} or \emph{maximal velocity}
\[Q_n(t)=\sup\{\,|v|:f_n(s,x,v)\ne0,\ s\in[0,t]\,\}.\]
\end{defn}

\begin{lem}
Let $f_n$ be the sequence of approximate solutions, and let $T>0$ be a constant such that
\[T<(Q_0\|f_0\|_\infty^{2/3}\|f_0\|_1^{1/3})^{-1}.\]
Then, we have the following bounds:
\begin{cond}
\item
For $t\le T$,
\[\|\rho_n(t)\|_\infty+\|E_n(t)\|_\infty+Q_n(t)\les1\]
indendent on $n$.
In addition, the support of $X_n$ is also uniformly bounded in $t\le T$.
\item
For $t\le T$
\[\|\del_x\rho_n(t)\|_\infty+\|\del_xE_n(t)\|_\infty\les1\]
independent on $n$.
\end{cond}
\end{lem}
\begin{pfs}
\item
Since
\[\|\rho_n(t)\|_\infty\le Q_n^3(t)\|f_0\|_\infty,\]
a rough estimate for $\|E\|_\infty$ gives
\[\|E_n(t)\|_\infty\le\|\rho_n(t)\|_\infty^{2/3}\|\rho_n(t)\|_1^{1/3}\le Q_n^2(t)\|f_0\|_\infty^{2/3}\|f_0\|_1^{1/3}.\]
Let $c(f_0)=\|f_0\|_\infty^{2/3}\|f_0\|_1^{1/3}$ be a constant such that $\|E_n(t)\|\le cQ_n^2(t)$.
We claim that
\[Q_n(t)\le\frac{Q_0}{1-cQ_0t}\]
for all $n$.
Easily checked for $n=0$; $Q_0(t)\equiv Q_0\le\frac{Q_0}{1-cQ_0t}$.

Assume $Q_n(t)\le\frac{Q_0}{1-cQ_0t}$.
Then,
\begin{align*}
|V_{n+1}(t;0,x,v)|
&\le|v|+\int_0^t|E_n(s;0,x,v)|\,ds\\
&\le Q_0+c\int_0^tQ_n^2(s)\,ds
\end{align*}
implies
\begin{align*}
Q_{n+1}(t)
&\le Q_0+c\int_0^tQ_n^2(s)\,ds\\
&\le Q_0+c\int_0^t\left(\frac{Q_0}{1-cQ_0s}\right)^2ds
=\frac{Q_0}{1-cQ_0t}.
\end{align*}
By induction, $Q_n(t)\le\frac{Q_0}{1-cQ_0t}\les1$ for all $n$ and $t\in[0,T]$, where $T<(cQ_0)^{-1}$.
Furthermore,
\[\|\rho_n(t)\|_\infty\les Q_n^3(t)\les1,\quad\|E_n(t)\|_\infty\les Q_n^2(t)\les1.\]
For the position support, we can bound it because
\[|X_n(t;0,x,v)|\le|x|+\int_0^t|V_n(s;0,x,v)|\,ds\le|x|+TQ_n(t)\les1.\]

\item
Two inequalities
\begin{align*}
|\del_xX_{n+1}(s;t,x,v)|
&=\Bigl|\underbrace{(1,\cdots,1)}_{9}-\int_s^t\del_xV_{n+1}(s';t,x,v)\,ds'\Bigr|\\
&\le3+\int_s^t|\del_xV_{n+1}(s';t,x,v)|\,ds'
\end{align*}
and
\begin{align*}
|\del_xV_{n+1}(s;t,x,v)|
&=|\int_s^t\del_xE_n(s',X_{n+1}(s';t,x,v))\,ds'|\\
&\le\int_s^t|\del_xX_{n+1}(s';t,x,v)|\cdot\|\del_xE_n(s')\|_\infty\,ds'
\end{align*}
are combined as
\begin{align*}
&\qquad|\del_xX_{n+1}(s;t,x,v)|+|\del_xV_{n+1}(s;t,x,v)|\\
&\le3+\int_s^t(1+\|\del_xE_n(s')\|_\infty)(|\del_xX_{n+1}(s';t,x,v)|+|\del_xV_{n+1}(s';t,x,v)|)\,ds'.
\end{align*}
By the Gronwall inequality, we get
\[|\del_xX_{n+1}(s;t,x,v)|+|\del_xV_{n+1}(s;t,x,v)|\le e^{\int_s^t(1+\|\del_xE_n(s')\|_\infty)\,ds'}\]
for $0\le s\le t$.

Note that
\begin{align*}
|\del_x\rho_{n+1}&(t,x)|
=|\int\del_xf_0(X_{n+1}(0;t,x,v),V_{n+1}(0;t,x,v))\,dv|\\
&\le\|\del_{x,v}f_0\|_\infty\int(|\del_xX_{n+1}(0;t,x,v)|+|\del_xV_{n+1}(0;t,x,v)|)\,dv\\
&\le\|\del_{x,v}f_0\|_\infty Q_{n+1}^3(t)\cdot e^{\int_0^t(1+\|\del_xE_n(s)\|_\infty)\,ds}.
\end{align*}
Recall that
\[\|\del_xE_{n+1}(t)\|\les(1+\|\rho_{n+1}(t)\|_\infty\log^+\|\del_x\rho_{n+1}(t)\|_\infty+\|\rho_{n+1}(t)\|_1).\]
By inserting the estimate for $|\del_x\rho_{n+1}(t,x)|$, we can find a constant $c=c(f_0)$ such that
\begin{align*}
1+\|\del_xE_{n+1}(t)\|_\infty\le c(1+\int_0^t(1+\|\del_xE_n(s)\|_\infty)\,ds)
\end{align*}
in $t\le T$, where $T<(Q_0\|f_0\|_\infty^{2/3}\|f_0\|_1^{1/3})^{-1}$.
Without loss of generality, we may assume that $c$ satisfies
\[c\ge\sup_{s\in[0,T]}(1+\|E_0(s)\|_\infty).\]
Then, induction obtains the bound
\[1+\|E_n(t)\|_\infty\le ce^{ct}\le ce^{cT}\les1\]
for all $n$ and $t\le T$.
The derivative of mass density is bounded since
\[\|\del_x\rho_{n+1}(t)\|_\infty\les e^{\int_0^t(1+\|\del_xE_n(s)\|_\infty)\,ds}.\qedhere\]
\end{pfs}



\subsection{Convergence of approximate solution}
Although most of the nonlinear systems fail to have convergent approximate solutions so that compactness methods are often applied, the constructed and investigated approximate solutions in the previous subsections uniformly converges.
\begin{lem}
Let $f_n$ be the sequence of approximate solutions, and let $T>0$ be a constant such that
\[T<(Q_0\|f_0\|_\infty^{2/3}\|f_0\|_1^{1/3})^{-1}.\]
\begin{cond}
\item
for $t\le T$ and $n\ge1$,
\[\|f_{n+1}(t)-f_n(t)\|_\infty\les\int_0^t\|E_n(s)-E_{n-1}(s)\|_\infty\,ds.\]
\item
for $t\le T$ and $n\ge1$,
\[\|E_n(s)-E_{n-1}(s)\|_\infty\les\|f_n(s)-f_{n-1}(s)\|_\infty.\]
\item $f_n$ converges to a function $f$ uniformly in $C([0,T]\x\R^6)$.
\item $(X_n,V_n)$ converges uniformly in $C([0,T]\x\R^6)$, and its limit $(X,V)$ satisfies the characteristic equation
\[\dot X=V,\quad\dot V=\gamma E,\]
where
\[E(t,x)=\frac1{4\pi}\iint\frac{(x-y)f(t,x,v)}{|x-y|^3}\,dv\,dx.\]
\end{cond}
\end{lem}
\begin{pfs}
\item
Denote
\[g(s):=|X_{n+1}(s;t,x,v)-X_n(s;t,x,v)|+|V_{n+1}(s;t,x,v)-V_n(s;t,x,v)|.\]
The $C^1$ regularity of $f_0$ gives
\begin{align*}
|f&_{n+1}(t,x,v)-f_n(t,x,v)|\\
&=|f_0(X_{n+1}(0;t,x,v),V_{n+1}(0;t,x,v))-f_0(X_n(0;t,x,v),V_n(0;t,x,v))|\\
&\les|X_{n+1}(0;t,x,v)-X_n(0;t,x,v)|+|V_{n+1}(0;t,x,v)-V_n(0;t,x,v)|\\
&=g(0).
\end{align*}
If an inequality
\[\sup_{s\in[0,t]}g(s)\les\int_0^t\|E_n(s)-E_{n-1}(s)\|_\infty\,ds\]
is obtained for $t\le T$, then we are done.

Let $0\le s\le t\le T$.
Because
\begin{align*}
X_n(s;t,x,v)&=x-\int_s^tV_n(s';t,x,v)\,ds',\\
V_n(s;t,x,v)&=v-\int_s^tE_{n-1}(s',X_n(s;t,x,v))\,ds',
\end{align*}
we have two inequalities
\begin{align*}
|V&_{n+1}(s;t,x,v)-V_n(s;t,x,v)|\\
&\le\int_s^t|E_n(s',X_{n+1}(s';t,x,v))-E_{n-1}(s',X_n(s';t,x,v))|\,ds'\\
&\le\int_s^t(|E_n(s',X_{n+1})-E_n(s',X_n)|+|E_n(s',X_n)-E_{n-1}(s',X_n)|)\,ds'\\
&\le\int_s^t(\|\del_xE_n(s')\|_\infty|X_{n+1}(s')-X_n(s')|+\|E_n(s')-E_{n-1}(s')\|_\infty)\,ds'
\end{align*}
and
\begin{align*}
|X_{n+1}(s;t,x,v)-X_n(s;t,x,v)|\le\int_s^t|V_{n+1}(s';t,x,v)-V_n(s';t,x,v)|\,ds'
\end{align*}
for $s\in[0,t]$.
By combining the two inequalities above, we get
\begin{align}\label{ggw}
g(s)\le\int_s^ta(s')g(s')\,ds'+\int_s^t\|E_n(s')-E_{n-1}(s')\|_\infty\,ds',
\end{align}
where $a(s):=1+\|\del_xE_n(s)\|_\infty$.

Here we use a Gronwall-type inequality.
In more detail, multiplying
\[a(s)e^{-\int_s^ta(s')ds'}\]
on the both-hand-side of (\ref{ggw}), and using $a\les 1$ in $t\le T$, we have
\begin{align*}
&-\dd{s}\left(e^{-\int_s^ta(s')\,ds'}\int_s^ta(s')g(s')\,ds'\right)\\
&\hspace{5em}\le a(s)e^{-\int_s^ta(s')ds'}\int_s^t\|E_n(s')-E_{n-1}(s')\|_\infty\,ds'\\
&\hspace{5em}\les\int_s^t\|E_n(s')-E_{n-1}(s')\|_\infty\,ds'
\end{align*}
Integrate from $s$ to $t$ and bound $(t-s)\le T\les1$ to get
\begin{align}\label{abd}
e^{-\int_s^ta(s')\,ds'}\int_s^ta(s')g(s')\,ds'\les\int_s^t\|E_n(s')-E_{n-1}(s')\|_\infty\,ds'.
\end{align}
Since $e^{\int_s^ta(s')\,ds'}\le e^{T\sup_{s\in[0,t]}a(s)}\les1$, the inequalities (\ref{ggw}) and (\ref{abd}) implies
\begin{align}\label{rii}
g(s)\les\int_0^t\|E_n(s)-E_{n-1}(s)\|_\infty\,ds.
\end{align}

\item
Notice that
\[\|E_n(t)-E_{n-1}(t)\|_\infty\les\|\rho_n(t)-\rho_{n-1}(t)\|_1^{1/3}\|\rho_n(t)-\rho_{n-1}(t)\|_\infty^{2/3}.\]
For $L^\infty$-norm,
\begin{align*}
\|\rho_n(t)-\rho_{n-1}(t)\|_\infty
&\le\max\{Q_n^3(t),Q_{n-1}^3(t)\}\|f_n(t)-f_{n-1}(t)\|_\infty\\
&\les\|f_n(t)-f_{n-1}(t)\|_\infty.
\end{align*}
For $L^1$-norm, since the support of $f_n,f_{n-1}$ is bounded in both directions $x,v$ in finite time,
\[\|\rho_n(t)-\rho_{n-1}(t)\|_1\le\|f_n(t)-f_{n-1}(t)\|_1\les\|f_n(t)-f_{n-1}(t)\|_\infty\]
for $t\le T$, where $T<\infty$ arbitrary.

\item
From (i) and (ii), there is a constant $c=c(f_0)$ such that for $t<T$,
\[\|f_{n+1}(t)-f_n(t)\|_\infty\le c\int_0^t\|f_n(s)-f_{n-1}(s)\|_\infty\,ds.\]

We can easily get with induction
\[\|f_{n+1}(t)-f_n(t)\|_\infty\le M\frac{(ct)^n}{n!},\]
where $M=\sup_{s\in[0,T]}\|f_1(s)-f_0(s)\|_\infty$.
Therefore,
\[\sum_{n=0}^\infty\|f_{n+1}(t)-f_n(t)\|_\infty\le Me^{ct}\le Me^{cT}<\infty\]
implies $f_n$ uniformly converges.

\item
The convergence of characteristics is clear by the inequality $(\ref{rii})$ and the convergence of $f_n$.
%%%
\qedhere
\end{pfs}

\begin{prop}[Local existence]
Let $f_n$ be the sequence of approximate solutions.
Then, there is a constant $T=T(f_0)$ be a constant such that the limit $f$ of $f_n$ is in $C^1([0,T],C_c^1(\R^6))$, and solves the Cauchy problem for the Vlasov-Poisson system.
\end{prop}
\begin{pf}
Take $T$ such that $T<(Q_0\|f_0\|_\infty^{2/3}\|f_0\|_1^{1/3})^{-1}$.
Let $X(s;t,x,v)$ and $V(s;t,x,v)$ be the limits of $X_n$ and $V_n$.
Notice that
\begin{align*}
f(t,x,v)=\lim_{n\to\infty}f_n(t,x,v)&=\lim_{n\to\infty}f_0(X_n(0;t,x,v),V_n(0;t,x,v))\\
&=f_0(X(0;t,x,v),V(0;t,x,v)).
\end{align*}
We can check it solves the system by expand the right-hand-side of
\[0=\dd{s}f(s,X(s;t,x,v),V(s;t,x,v))|_{s=t}\]
using the chain rule.
\end{pf}


\subsection{Uniqueness}



\subsection{Prolongation criterion}

\begin{prop}
If $Q(t)$ does not blow up, then the solution $f$ of the Vlasov-Poisson system is continued globally to the entire $\R^+$.
\end{prop}
\begin{pf}
Suppose $f\in C^1([0,T_{max}),C_c^1(\R^6))$ for $T_{max}<\infty$ is the maximal solution.
Since $Q$ does not blow up, we may define
\[Q(T_{max}):=\lim_{t\to T_{max}+}Q(t).\]
We are going to apply the local existence result for the new system with initial condition $\tld f(0,x,v)=f(t_0,x,v)$ for some $t_0<T_{max}$.
In subsection 2.3, we have shown the length of time interval for existence $T$ is given by the condition
\[T<(Q_0\|f_0\|_\infty^{2/3}\|f_0\|_1^{1/3})^{-1}.\]
It means that, if we arrange it for the new solution $\tld f$, the interval of existence of $\tld f$ has in fact a lower bound $\tld T>0$ that depends only on $Q(T_{max})$ for any new initial time $t_0$ since $Q$ is monotonically increasing and the volume preservation implies $\|f_0\|_\infty=\|f(t_0)\|_\infty$ and $\|f_0\|_1=\|f(t_0)\|_1$.

By setting $t_0=T_{max}-\frac12\tld T$ we can show there exists a solution $f\in C^1([0,T_{\max}+\frac12\tld T),C_c^1(\R^6))$, which contradicts to the maximality of $T_{\max}$.
Hence $T_{max}=\infty$, and the solution $f$ is prolonged forever.
\end{pf}













\section{Global existence}



\begin{thm*}[Schaeffer, 1991]
Let $f_0\in C_c^1(\R^6)$ and $f_0\ge0$.
Then, the Cauchy problem for the Vlasov-Poisson system has a unique $C_c^1$ global solution.
\end{thm*}

\subsection{Estimate on field}




\subsection{Lower bound on relative position vectors}
Our goal is to obtain a priori estimate like
\[\|E(t)\|_\infty\les Q(t)^a\qquad\text{for some }a<1.\]
Since the force field $E$ measures the maximal rate of changes in velocity, the estimate can be read very roughly as
\[Q'(t)\les Q(t)^a,\]
which leads its polynomial growth.
So we need to bound $E$.

Fix a time of existence $t$ and a point $(t,\hat x,\hat v)$ and let
\[\hat X(s):=X(s;t,\hat x,\hat v),\qquad\hat V(s):=V(s;t,\hat x,\hat v).\]
Decompose $[t-\Delta,t]\x\R_x^3\x\R_v^3$ as
\begin{align*}
U&=\left\{\,(s,x,v):\ |v-\hat V(t)|\ge P,\quad|y-\hat X(s)|\ge r\,\right\},\\
B&=\left\{\,(s,x,v):\ |v-\hat V(t)|\ge P,\quad|v|\ge P\,\right\}\setminus U,\\
G&=\left\{\,(s,x,v):\ |v-\hat V(t)|<P\quad\text{or}\quad|v|<P\,\right\}.
\end{align*}
(We can let $U\mapsto U\cap\{|v|\ge P\}$ to make the decomposition disjoint.)
Later we choose
\[P=Q^{4/11},\quad r=R\max\{|v|^{-3},\,|v-\hat V(t)|^{-3}\},\quad R=Q^{16/33}(\log^+Q)^{1/2}.\]
Also, later we choose $\Delta\cdot\sup_{s\le t}\|E(s)\|_\infty<\frac P4$.

Reading the proof, letting $y=X(s;t,x,v)$ and $w=V(s;t,x,v)$ be functions of time variable $s$, trace carefully the following four quantities:
\[|x-\hat X(t)|,\ |y-\hat X(s)|,\ |v-\hat V(t)|,\ |w-\hat V(s)|.\]
The following observation suggests a lower bound of relative position.
\begin{prop}
Fix $x,v$.
Let $P>0$ and $0<\Delta<t$ be constants such that
\[\Delta\cdot\sup_{s\le t}\|E(s)\|_\infty<\frac P4.\]
If $v$ satisfies $|v-\hat V(t)|\ge P$, then there is $s_0\in[t-\Delta,t]$ such that
\[|y-\hat X(s)|\ge\frac14|v-\hat V(t)||s-s_0|\]
for all $s\in[t-\Delta,t]$.
\end{prop}
\begin{pf}
Since $\Delta\|E(s)\|_\infty<\frac P4$, we have
\[|v-w|<\frac P4\quad\text{and}\quad|\hat V(t)-\hat V(s)|<\frac P4.\]
The condition $|v-\hat V(t)|\ge P$ implies
\[\frac12|v-\hat V(t)|\le|v-\hat V(t)|-\frac P4-\frac P4<|w-\hat V(s)|.\]

Let $Z(s):=y-\hat X(s)$ be the relative position vector.
Then,
\begin{align*}
Z'(s)&=w-\hat V(s),\\
Z''(s)&=\gamma[E(s,y,w)-E(s,\hat X(s),\hat V(s))].
\end{align*}
Let $s_0\in[t-\Delta,t]$ minimize $s\mapsto|Z(s)|$ and expand $Z$ as
\[Z(s)=Z(s_0)+Z'(s_0)(s-s_0)+\frac{Z''(\sigma)}2(s-s_0)^2\]
for some $\sigma$ between $s$ and $s_0$.
Then,
\[|Z(s_0)+Z'(s_0)(s-s_0)|\ge|Z'(s_0)(s-s_0)|\ge\frac12|v-\hat V(t)||s-s_0|\]
and
\begin{align*}
|\frac{Z''(\sigma)}2(s-s_0)^2|
&\le\|E(t)\|_\infty(s-s_0)^2
\le\|E(t)\|_\infty\Delta|s-s_0|\\
&\le\frac P4|s-s_0|
\le\frac14|v-\hat V(t)||s-s_0|
\end{align*}
proves
\[|y-\hat X(s)|=|Z(s)|\ge\frac14|v-\hat V(t)||s-s_0|.\qedhere\]
\end{pf}

We introduce time averaging to use the above lower bound.
\begin{prop}
Fix $x,v$.
Let $P>0$ and $0<\Delta<t$ be constants such that
\[\Delta\cdot\sup_{s\le t}\|E(s)\|_\infty<\frac P4.\]
If $v$ satisfies $|v-\hat V(t)|\ge P$, then
\[\int_{t-\Delta}^t\frac1{|y-\hat X(s)|^2}\chi_A(s)\,ds\les\frac{r^{-1}}{|v-\hat V(t)|},\]
where $A=\{s:|y-\hat X(s)|\ge r\}$.
\end{prop}
\begin{pf}
Since $|y-\hat X(s)|\ge\frac14|v-\hat V(t)||s-s_0|$,
\begin{align*}
\int_{t-\Delta}^t\frac1{|y-\hat X(s)|^2}\chi_A(s)\,ds
&\le16\int_{t-\Delta}^t\frac1{|v-\hat V(t)|^2|s-s_0|^2}\chi_A(s)\,ds\\
&\le32\int_r^\infty\frac1{|v-\hat V(t)|^3|s-s_0|^2}\,d(|v-\hat V(t)||s-s_0|)\\
&=32\,\frac{r^{-1}}{|v-\hat V(t)|}.\qedhere
\end{align*}
\end{pf}


\subsection{Divide and conquer}
\subsubsection{Ugly set estimate}

Therefore, if we let $r^{-1}\simeq\min\{|v|^3,|v-\hat V(t)|^3\}$, then
\[\int_{t-\Delta}^t\frac1{|y-\hat X(s)|^2}\chi_A(s)\,ds\les|v|^2\]
so that we have
\[\iiint_U\frac{f(s,y,w)}{|y-\hat X(s)|^2}\,dw\,dy\,ds\les R^{-1}\int|v|^2f(t,x,v)\,dv\,dx\les R^{-1}\]
when
\[U\subset\{\,(s,x,v):\ |v-\hat V(t)|\ge P,\quad|y-\hat X(s)|\ge R\max\{|v|^{-3},|v-\hat V(t)|^{-3}\}\,\}.\]

\subsubsection{Bad set estimate}
Consider $U^c$.
We need to control the union of two regions
\[|y-\hat X(s)|<R|v|^{-3}\quad\text{and}\quad|y-\hat X(s)|<R|v-\hat V(t)|^{-3}.\]
Without any conditions, the integration of fundamental solution with respect to $y$ gives
\[\int_{|y-\hat X(s)|<r}\frac1{|y-\hat X(s)|^2}\,dy\simeq r.\]
\begin{clm}
If $|v|\ge P$ and $|v-\hat V(t)|\ge P$, then
\[\int_{U^c}\frac1{|y-\hat X(s)|^2}\,dy\les\max\{|w|^{-3},|w-\hat V(s)|^{-3}\}\]
for $s\in[t-\Delta,t]$.
\end{clm}
\begin{pf}
It follows from
\[|w|\simeq|v|,\quad|w-\hat V(s)|\simeq|v-\hat V(t)|\]
for $|v|\ge P$ and $|v-\hat V(t)|\ge P$.
\end{pf}

\subsubsection{Good set estimate}

\subsection{Polynomial decay}
\begin{lem}
Along the time of existence we have
\[\|E(t)\|_{L_x^\infty}\les Q(t)^{4/3}.\]
\end{lem}
\begin{pf}
Note that we have
\[\|E\|_\infty\les\|\rho\|_\infty^{4/9}\|\rho\|_{5/3}^{5/9}.\]
Since the velocity support of $f$ is bounded by finite $Q(t)$,
\[\rho(t,x)=\int_{|v|<Q(t)}f(t,x,v)\,dv\les Q(t)^3\|f_0(x)\|_{L_v^\infty}\les Q(t)^3,\]
so
\[\|E(t)\|_{L_x^\infty}\les\|\rho(t)\|_{L_x^\infty}^{4/9}\les Q(t)^{4/3}.\qedhere\]
\end{pf}


\section*{Acknowledgement}
This report is written in Undergraduate Research Program of Postech, under the support and advice of professor Donghyun Lee.

\end{document}