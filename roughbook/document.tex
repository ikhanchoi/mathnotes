\documentclass[a4paper,12pt]{report}
\usepackage{../ikany}

\usepackage[margin=1.25in]{geometry}
\linespread{1.1}
\usepackage[bitstream-charter,cal]{mathdesign}
\let\circledS\undefined
\usepackage[T1]{fontenc}

\DeclareMathOperator{\Inv}{Inv}
\DeclareMathOperator{\ind}{ind}

\begin{document}




\begin{prb}
Let $(T_n)$ be a sequence in $B(X,Y)$.
If $T_n$ coverges then $\|T_n\|$ is bounded by the uniform boundedness principle.
\end{prb}





\begin{prb}
We show that there is no projection from $\ell^\infty$ onto $c_0$.
\begin{parts}
\item
Show that a Banach space $X$ is isometrically isomorphic to a Hilbert space if there is a bounded linear projection on every closed subspace of $X$.
\end{parts}
\end{prb}





\begin{prb}[Bounded below maps in Banach spaces]
Let $T:X\to Y$ be a bounded linear map between Banach spaces.
Show that the following statements are equivalent:
\begin{parts}
\item It is bounded below.
\item It is injective and has closed range.
\item It is a isometric isomorphism onto its image.
\end{parts}
\end{prb}


\begin{prb}[Bounded below maps in Hilbert spaces]
Let $T:H\to K$ be a bounded linear operator between Hilbert spaces.
Show that the following statements are equivalent:
\begin{parts}
\item It is bounded below.
\item It has a left inverse.
\item Its adjoint has right inverse.
\item The product $T^*T$ is invertible.
\end{parts}
In particular, a normal operator in $B(H)$ is bounded below if and only if it is invertible.
\end{prb}





\begin{prb}[Injectivity and surjectivity of dual map]
Let $T:X\to Y$ be a bounded linear operator between Banach spaces and $T^*:Y^*\to X^*$ be its dual.
\begin{parts}
\item
Show that $T^*$ is injective if and only if $T$ has dense range.
\item
Show that $T^*$ is surjective if and only if $T$ is bounded below.
\end{parts}
\end{prb}

\begin{prb}
For $T\in B(H)$, we have an obvious fact $(\im T)^\perp=\ker T^*$.
If $T$ is normal, then the kernel of $T$ and $T^*$ are equal.
\begin{parts}
\item
Show that if $T$ is surjective bounded operator, then $T$ is invertible.
\end{parts}
\end{prb}


\begin{prb}[Schur's property of $\ell^1$]
.
\end{prb}


\begin{prb}
Let $\f:L^\infty([0,1])\to\ell^\infty(\N)$ be an isometric isomorphism.
Suppose $\f$ is realised as a sequence of bounded linear functionals on $L^\infty$.
\begin{parts}
\item
Show that $\f^*(\ell^1)\subset L^1$ where $\ell^1$ and $L^1$ are considered as closed linear subspaces of $(\ell^\infty)^*$ and $(L^\infty)^*$ respectively.
\item Show that $\f^*$ is indeed an isometric isomorphism, and deduce $\f$ cannot be realised as bounded linear functionals on $L^\infty$.
\end{parts}
\end{prb}


\begin{prb}[Predual correspondence]
Let $X$ be a Banach space and $Z$ be a linear subspace of $X^*$.
Define $\f:X\to Z^*$ as the restriction of the dual map of inclusion $Z\subset X^*$.
\begin{parts}
\item
Show that if $\f$ is an isometric isomorphism, then closed ball of $X$ is compact Hausdorff in $\sigma(X,Z)$.
\item Show that the converse holds by using Goldstine's theorem.
\end{parts}
\end{prb}


\begin{prb}[Operator monotonicity of square and commitativity]
Let $\cA$ be a $C^*$-algebra in which the square function is operator monotone, that is, $0\le a\le b$ implies $a^2\le b^2$ for any positive elements $a$ and $b$ in $\cA$.
We are going to show that $\cA$ is necessarily commutative.
Let $a$ and $b$ denote arbitrary positive elements of $\cA$.
\begin{parts}
\item
Show that $ab+ba\ge0$.
\item
Let $ab=c+id$ where $c$ and $d$ are self adjoints.
Show that $d^2\le c^2$.
\item
Suppose $\lambda>0$ satisfies $\lambda d^2\le c^2$.
Show that $c^2d^2+d^2c^2-2\lambda d^4\ge0$.
\item
Show that $\lambda(cd+dc)^2\le(c^2-d^2)^2$.
\item
Show that $\sqrt{\lambda^2+2\lambda-1}\cdot d^2\le c^2$ and deduce $d=0$.
\item
Extend the result for general exponent: $\cA$ is commitative if $f(x)=x^\beta$ is operator monotone for $\beta>1$.
\end{parts}
\end{prb}

\begin{prb}[Compact left multiplications and SOT]
Let $T_n$ be a sequence of bounded linear operators on a Hilbert space that converges in SOT.
For compact $K$, $T_n K$ converges in norm, but $KT_n$ generally does not unless $T$ is self-adjoint.
\end{prb}


\begin{prb}
Let $X$ be a closed subspace of a Banach space $Y$ and \[i:X\to Y\] the inclusion.
Suppose $X$ and $Y$ have preduals $X_*$ and $Y_*$ respectively.
Let \[j:=i^*|_{Y_*}:Y_*\to Z\subset X^*,\]
where $Z:=i^*(Y_*)^-$.
Then we can show
\[j^*:Z^*\subset X^{**}\to Y\]
coincides with $i$ on $X\cap Z^*$.
From the existence of $X_*$ we have $X^{**}\to X$, which is restricted to define a map $k:Z^*\to X$.
\begin{cd}
&X\ar{r}{i}&Y\\
X^{**}\ar{ur}\ar{r}&Z^*\ar{u}{k}\ar{ur}{j}&
\end{cd}
We can show $k$ is an isomorphism so that we have
\[X_*\cong Y_*/Y_*\cap\ker(i^*).\]
\end{prb}

\begin{prb}[Injective *-homomorphism is an isometry]
%https://math.stackexchange.com/questions/434706/sufficient-condition-for-a-homomorphism-between-c-algebras-being-isometric/435105#435105
\end{prb}










\newpage
\section{Topological measures}
\begin{prb}
Let $X$ be compact.
A positive linear functional $\rho$ on $C(X)$ is bounded with norm $\rho(1)$.
\end{prb}
\begin{pf}
Since $0\le\rho(\|f\|\pm f)=\|f\|\rho(1)\pm\rho(f)$, we have $|\rho(f)|\le\rho(1)\|f\|$.
\end{pf}

\begin{prb}
Let $X$ be a locally compact Hausdorff space.
\begin{parts}
\item The Baire $\sigma$-algebra is generated by compact $G_\delta$ sets.
\item If $X$ is second countable, then every Baire set is Borel.
\end{parts}
\end{prb}
\begin{sol}
(b)
(A second countable locally compact space is $\sigma$-compact.

Since $X$ is $\sigma$-compact and Hausdorff, every closed set is a countable union of compact sets, so the Borel $\sigma$-algebra on $X$ is generated by compact sets.)

Since locally compact Hausdorff space is regular, the Urysohn metrization implies $X$ is metrizable, and every closed sets in metrizable space is $G_\delta$ set.
\end{sol}

\subsection{The Riesz-Kakutani theorem for positive linear functionals}
\begin{prb}
Let $X$ be compact.
There is a map from the set of finite Baire measures to the set of positive linear functionals on $C(X)$.
\end{prb}
\begin{sol}
A function in $C(X)$ is Baire measurable and bounded.
Thus the integration is well-defined.
\end{sol}

\begin{prb}
Let $X$ be compact.
There is a map from the set of positive linear functionals on $C(X)$ to the set of finite regular Borel measures.
\end{prb}
\begin{sol}
i. and ii. and iii. of Theorem 7.2.
\end{sol}


\begin{prb}
Let $X$ be compact.
Let $\rho$ be a positive linear functional on $C(X)$.
Let $\nu$ be the regular Borel measure associated to $\rho$.
Then, $\rho(f)=\int f\,d\nu$.
\end{prb}
\begin{sol}
iv. of Theorem 7.2.
\end{sol}

\begin{prb}
Let $X$ be compact.
Let $\nu$ be a finite regular Borel measure.
Let $\nu'$ be the regular Borel measure associated to the positive linear functional $f\mapsto\int f\,d\nu$.
Then, $\nu=\nu'$ on Borel sets.
\end{prb}
\begin{sol}
Theorem 7.8.
\end{sol}

The two results above establish the correspondence between positive linear functionals and regular Borel measures.
The following is an additional topic: Borel extension of Baire measures.
\begin{prb}
Let $X$ be compact.
Let $\mu$ be a finite Baire measure.
Let $\nu$ be the regular Borel measure associated to the positive linear functional $f\mapsto\int f\,d\mu$.
Then, $\mu=\nu$ on Baire sets.
\end{prb}
\begin{sol}
Let $\mu,\nu$ be finite Baire measures.
Enough to show if $\int f\,d\mu=\int f\,d\nu$ then $\mu=\nu$ according to the preceding two results.

Enough to show the regularity of Baire measures.
\end{sol}







\begin{itemize}
\item A second countable locally compact space is $\sigma$-compact.
\item A $\sigma$-compact locally compact space is paracompact.
\item A second countable regular space is paracompact.
\item A locally compact Hausdorff space is regular.
\end{itemize}

semiring
$\sigma$-finiteness implies the uniqueness






\newpage
\section{Problems}

\begin{prb}
Let $f:\R\to\R$ be differentiable.
Show that $f$ is identically zero if $f'(x)=f(x)^2$ for all $x$.
\end{prb}

\begin{prb}
Let $f(x)=x(1+x)^{-1}$ be a function on $\R_{\ge0}$.
Show that a C$^*$-algebra $\cA$ is commutative if and only if $f$ is operator subadditive in $\cA$.
\end{prb}

%L\"owner-Heinz inequality
\begin{prb}
Let $T$ be an invertible linear operator on a normed space.
Show that $T^{-2}+\|T\|^{-2}$ is injective if it is surjective.
\end{prb}

\begin{prb}[Diophantine euqations]
\begin{parts}
\item Show that there is no integral solution of the equation $x^7+7=y^2$.
\item Show that if $(x^2+y^2+z^2)/(xy+yz+zx)$ is an integer, then it is not divided by 3.
\item Show that there is no non-trivial integral solution of $x^4-y^4=z^2$.
\end{parts}
\end{prb}





\end{document}


