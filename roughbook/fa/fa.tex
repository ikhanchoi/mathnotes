\documentclass[12pt]{report}
\usepackage{../../ikany}
\usepackage[margin=100pt]{geometry}
\usepackage[T1]{fontenc}
\usepackage[bitstream-charter,cal]{mathdesign}
\linespread{1.1}


\begin{document}
\tableofcontents

\part{Traditional topics}
\chapter{Geometry of Banach spaces}
dentability, Bishop-Phelps, Diestel's book
https://gowers.wordpress.com/2009/02/07/a-remarkable-recent-result-in-banach-space-theory/
\chapter{Spectral theory}
\chapter{Hardy spaces}











\part{Operator algebra}

\chapter{Classification of C$^*$-algebras}

\section{K-theory and approximately finite algebras}
Elliott conjecture: amenable simple separable C$^*$-algerbas are classified by K-theory.
\section{Crossed products and C$^*$-dynamical systems}
\section{Abstract harmonic analysis}
Group C$^*$ algebras






\chapter{Classification of factors}
Direct integral of factors.

Type I factors.
It possess a minimal projection.
It is isomorphic to the whole $B(H)$ for some Hilbert space.
Therefore, it is classified by the cardinality of $H$.

Type II factors.
No minimal projection, but there are non-zero finite projections so that every projection can be ``halved'' by two Murray-von Neumann equivalent projections.

In type II$_1$ factors, the identity is a finite projection
Also, Murray and von Neumann showed there is a unique finite tracial state and the set of traces of projections is $[0,1]$.
Free probability theory attacks the free groups factors, which are type II$_1$.

In type II$_\infty$ factors
There is a unique semifinite tracial state up to rescaling and the set of traces of projections is $[0,\infty]$.

In type III factors no non-zero finite projections exists.
Classified the $\lambda\in[0,1]$ appeared in its Connes spectrum, they are denoted by III$_\lambda$.
Tomita-Takesaki theory.
It is represented as the crossed product of a type II$_\infty$ factor and $\R$.

Amenability, equivalently hyperfiniteness is a very nice condition in von Neumann algebra theory.
Group-measure space construction can construct them.
There are unique hyperfinite type II$_1$ and II$_\infty$ factors, and their property is well-known.
Fundamental groups of type II factors, discrete group theory, Kazhdan's property (T) are used.


Tensor product facctors such as Araki-Woods factors and Powers factors.

\section{Hyperfinite factors}

weight, trace, state.

finite trace$=$tracial state.

\begin{prb}[Uniformly hyperfinite algebras]
Let $\cA$ be a uniformly hyperfinite algebra.
\begin{parts}
\item Every matrix algebra admits a unique finite trace.
\item Every UHF algebra admits a unique finite trace.
\item Every hyperfinite 
\end{parts}
\end{prb}

\begin{prb}[Classification of UHF algebras]

\end{prb}








\chapter{Subfactor theory}
The way how quantum systems are decomposed.
Quite combinatorial!
And has Galois analogy.

\begin{prb}[Jones index theorem]
A \emph{subfactor} of a factor $M$ is a factor $N$ containing $1_M$.
\end{prb}

Tensor categories and topological invariants of 3-folds.
Ergodic flows.







\part{Mathematical quantum theory}

\chapter{Quantum statistical physics}
CCR and CAR representation problems
KMS state




\chapter{Algebraic quantum field theory}
\section{Wightman axioms}
\section{Nets of algebras}
Doplicher-Haag-Roberts theory defines the concept of ``tensor product'' of net of algebras by considering composition of endomorphisms between nets of algebras.
The commutativity, up to unitary equivalence, of this tensor product follows from the locality axiom.
It is figured by Longo that the DHR theory has an analogy with subfactor theory of Jones.

A net of algebras on the 1+1 Minkowski space is decomposed as a tensor product of two nets of algebras on the 1-dimensional space, which are called chirl nets.
Due to the disconnectivity of the completion of open sets in 1 dimensional space, the unitary mapping that connects ``tensor products'' of nets of algebras brings a braiding; their representations produce a baided tensor category.



\chapter{Conformal field theory}



\chapter{Strict deformation quantization}






\chapter{Quantum information theory}
\section{Operator spaces}




\end{document}