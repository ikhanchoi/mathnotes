\documentclass{../prb}
\usepackage{../../ikany}

\title{Physics I : Classical theory}

% brackets
\newcommand{\bket}[1]{\left\lvert #1\right\rangle}
\newcommand{\brak}[1]{\left\langle #1 \right\rvert}
\newcommand{\braket}[2]{\left\langle #1\middle\vert #2 \right\rangle}


\begin{document}
\maketitle
\tableofcontents


\chapter{Classical mechanics}


\section{Newtonian mechanics}


\subsection*{Uniformly accelerated motion}

\begin{prb}
Take $x$-axis to be the horizontal earth and $y$-axis to be vertical.
A boy is on the earth, and let the position of the boy be $(0,0)$.
A bird in the sky $(l,h)$ get a shot at time $t=0$ and falls freely.
Almost simultaneously, The boy throw a small rock with the initial velocity $v_0$ and the angle $\theta$ from the earth.
Let $g$ be the gravitational acceleration.
\begin{cond}
\item Find the time $t$ at which the rock reaches a vertical line $x=l$.
\item At this time, find the $y$-coordinate of the rock and the bird $y_r$, $y_{bird}$.
\item They collided. Find $\theta$.
\item Find the condition for $v_0$ to make them collide each other.
\item Show that if you are a bird, a rock is in a uniform rectilinear motion. Find the time for collision after throwing.
\end{cond}
\end{prb}
\begin{sol}\ \par
(1)
The horizontal component of the velocity is $v_0\cos\theta$.
Therefore, the time traveling the length $l$ is
\[t=\boxed{\frac l{v_0\cos\theta}}\,.\]

(2)
Let $t$ be the answer of the previous problem.
Their accelerations are $-g$.
Therefore,
\[y_{rock}=\frac12(-g)t^2+(v_0\sin\theta)t+(0)=\]
and
\[y_{bird}=\frac12(-g)t^2+(0)t+(h)=.\]

(3)

\end{sol}


\clearpage
In Hamiltonian mechanics, the phase space $M$ is defined to be cotangent bundle of a configuration manifold.
According to Newton's principle of determinacy, a particle at a specific time corresponds to a point in $M$, and the point contains all informations of a particle.
A function on $M$ is a physical quantity, such as position, momentum, angular momentum, etc.
Especially, positions and momenta with respect to each dimension provide with canonical coordinate functions on $M$.
Therefore, every function on $M$ can be realized by a function of positions and momenta.

A \emph{Hamiltonian function} $H$ is also just a function on $M$.
In physics, if a Hamiltonian function is given, the equation of motion is generated.
In other words, Hamiltonian function defines a physical problem.

\begin{defn}[Hamilton's equations of motion]
For a Hamiltonian function $H$, Hamilton's equations of motion are given by
\[\dot x=\pd{H}{p},\qquad\dot p=-\pd{H}{x}.\]
Using the Poisson bracket, the equations can be represented by
\[\dd{f}{t} = \{f,H\} + \pd{f}{t}.\]
\end{defn}

\begin{prb}[Harmonic oscillator]
Let $M=T^*\R$ and
\[H(x,p)=\frac{p^2}{2m}+\frac12kx^2.\]
This Hamiltonian function defines a problem of 1-dimensional harmonic oscillator.
The equations of motion are
\[\dot x=\pd{H}{p}=\frac pm,\qquad\dot p=-\pd{H}{x}=-kx.\]
Therefore, we get the familiar equation for a harmonic oscillator
\end{prb}
\begin{sol}\ \par
(1) Here is the proof.

(2) Here is a proof.
\end{sol}


If $H$ has a symmetry under transformations in time, namely, $H$ does not depend on $t$ explicitly, then 

A problem that $H$ is explicitly independent on $p$ is difficult to occur physically.

\chapter{Electromagnetism}

\chapter{Relativity theory}

\chapter{Statistical mechanics}

\chapter{Fluid dynamics}



\section{Quantum mechanics}
\begin{prb}[Hydrogen atom]
Hydrogen
\q hydrogen
\end{prb}


\section{Particle physics}
\begin{prb}[Yukawa potential]
The wave equation for a massive field is given by the Klein-Gordon equation
\[(\Box+m^2)u(t,x)=0,\]
where $m$ is mass.
\begin{cond}
\item Derive the Yukawa potential\[u(x)=k\frac{e^{-\frac rm}}r\]where $r=|x|$, as a Green function by assuming static case.
\item Letting $m=0$, discuss the relation with Coulomb potential.
\item By taking Fourier transform.
\item Find an approximate range of strong nuclear force and mass of pion.
\end{cond}
\end{prb}

\begin{prb}[Negative energy solution and antiparticles]
Dirac's interpretation.
\q negative energy solution of the Klein-Gordon equation
\q time reversal?
\end{prb}

\begin{prb}[Polarization of photon field]
\end{prb}

\begin{prb}[Aharonov-Bohm effect]
\end{prb}


\end{document}