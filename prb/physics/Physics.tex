\documentclass[11pt]{article}
\input{../../header}
\NeedsTeXFormat{LaTeX2e}
\ProvidesClass{prb}[2019/03/25 prb]
\LoadClass[11pt]{amsbook}
\makeatletter


\usepackage[margin=100pt,headheight=10pt]{geometry}

\newcommand{\korean}{
  \usepackage[hangul]{kotex}
  \geometry{headheight=20pt}
}


\newcommand{\+}{\begin{center}$\cdot\qquad\cdot\qquad\cdot\qquad\cdot\qquad\cdot$\end{center}}
% \blacklozenge


\let\@@author\author
\renewcommand{\author}[1]{
  \def\lauthor{#1}
  \@@author{\small Written by \lauthor \\ \small Solved by \lauthor}
}

\author{Ikhan Choi}

\usepackage{fancyhdr}
\pagestyle{fancy}
\fancyhf{}
\fancyhead[L]{\nouppercase{\leftmark}}
\fancyhead[R]{\nouppercase{\rightmark}}
\fancyfoot[C]{\scriptsize{\thepage}}

\let\@@section\section
\renewcommand{\section}{\newpage\@@section}




\setcounter{tocdepth}{2}
\def\l@subsection{\@tocline{2}{0pt}{2pc}{4pc}{}}



\usepackage{hyperref}


\makeatother

\def\ntitle{Physics}

% brackets
\newcommand{\bket}[1]{\left\lvert #1\right\rangle}
\newcommand{\brak}[1]{\left\langle #1 \right\rvert}
\newcommand{\braket}[2]{\left\langle #1\middle\vert #2 \right\rangle}

\begin{document}
\maketitle
\tableofcontents


\section{Classical mechanics}

In Hamiltonian mechanics, the phase space $M$ is defined to be cotangent bundle of a configuration manifold.
According to Newton's principle of determinacy, a particle at a specific time corresponds to a point in $M$, and the point contains all informations of a particle.
A function on $M$ is a physical quantity, such as position, momentum, angular momentum, etc.
Especially, positions and momenta with respect to each dimension provide with canonical coordinate functions on $M$.
Therefore, every function on $M$ can be realized by a function of positions and momenta.

A \emph{Hamiltonian function} $H$ is also just a function on $M$.
In physics, if a Hamiltonian function is given, the equation of motion is generated.
In other words, Hamiltonian function defines a physical problem.

\begin{defn}[Hamilton's equations of motion]
For a Hamiltonian function $H$, Hamilton's equations of motion are given by
\[\dot x=\pd{H}{p},\qquad\dot p=-\pd{H}{x}.\]
Using the Poisson bracket, the equations can be represented by
\[\dd{f}{t} = \{f,H\} + \pd{f}{t}.\]
\end{defn}

\begin{ex}[Harmonic oscillator]
Let $M=T^*\R$ and
\[H(x,p)=\frac{p^2}{2m}+\frac12kx^2.\]
This Hamiltonian function defines a problem of 1-dimensional harmonic oscillator.
The equations of motion are
\[\dot x=\pd{H}{p}=\frac pm,\qquad\dot p=-\pd{H}{x}=-kx.\]
Therefore, we get the familiar equation for a harmonic oscillator
\end{ex}



If $H$ has a symmetry under transformations in time, namely, $H$ does not depend on $t$ explicitly, then 

A problem that $H$ is explicitly independent on $p$ is difficult to occur physically.


\section{Quantum mechanics}
\begin{prb}[Hydrogen atom]
Hydrogen
\q hydrogen
\end{prb}

\section{Relativity theory}

\section{Particle physics}
\begin{prb}[Yukawa potential]
The wave equation for a massive field is the Klein-Gordon equation
\[(\Box+m^2)U=0.\]
\q deriving the Yukawa potential as a Green function by assuming static case.
\q relation with Coulomb potential.
\q the Fourier transformation.
\q range of (strong) nuclear force and mass of pion
\end{prb}

\begin{prb}[Negative energy solution and antiparticles]
Dirac's interpretation.
\q negative energy solution of the Klein-Gordon equation
\q time reversal?
\end{prb}

\begin{prb}[Polarization of photon field]
\mtprb
\end{prb}

\begin{prb}[Aharonov-Bohm effect]
\mtprb
\end{prb}


\end{document}