\documentclass{../prb}
\usepackage{../../ikany}

\title{Number Theory I : Classical therory}

\begin{document}
\maketitle
\tableofcontents

\chapter{Diophantine equations}

\section{Quadratic equations}


\begin{prb}
Consider a family of diophantine equations:
\[x^2+y^2-kxy-k=0\]
for $k\in\Z$.
\begin{cond}
\item Find the smallest three solutions such that $x>y\ge0$ when $k=4$.
\item Show that if $(x,y)$ is a solution, then $(y,ky-x)$ is also a solution.
\item Show that if it has a solution, then at least one solution satisfies $x>0\ge y$.
\item Show that the equation does not have a solution in the region $xy<0$.
\item Show that the equation has a solution if and only if $k$ is a perfect square.
\item Let $a$ and $b$ be integers. Deduce that if $ab+1$ divides $a^2+b^2$, then \[\frac{a^2+b^2}{ab+1}\]is a perfect square.
\end{cond}
\end{prb}
\begin{sol}
\item
Try for $y=0,1,\cdots,8$.
Then we get $(2,0)$, $(8,2)$, and $(30,8)$.
\item
By substitution, we have
\[y^2+(ky-x)^2-ky(ky-x)-k=y^2-2kxy+x^2+kxy-k=0.\]
\item
Suppose not.
By symmetry, we may assume we have a solution with $x>y>0$.
Take the solution such that $x+y$ is minimal.
Note that we have
\[0\le x^2+y^2=k(xy+1)\impl k\ge0,\]
and
\[2x^2>x^2+y^2=kxy+k\ge kxy\impl 2x>ky.\]
As we have seen, $(y,ky-x)$ is a solution, and $ky-x>0$ by the assumption.
Since $x+y>y+(ky-x)$, we obtain a contradiction for the minimality.
\item
Suppose $x,y\in\Z$ satisfy $xy<0$.
Since $xy\le-1$, \[x^2+y^2-kxy-k\ge x^2+y^2+k-k>0.\]
\item
\end{sol}
\begin{note}
In general, the transformation $(x,y)\mapsto(y,ky-x)$ preserving the image of hyperbola is not easy to find.
A strategy to find it in this problem is called the \emph{Vieta jumping} or \emph{root flipping}.
It gets the name by the following reason:
If $(a,b)$ is a solution with $a>b$, then a quadratic equation \[x^2-kbx+b^2-k=0\] has a root $a$, and the other root is $kb-a$ so that $(b,kb-a)$ is also a solution.
The last problem is from the International Mathematical Olympiad 1988, and the Vieta jumping technique was firstly used to solve it.
\end{note}

\begin{prb}
Consider a diophantine equation:
\[y^2=x^3+7.\]
Suppose $(x,y)$ is a solution.
\begin{cond}
\item Show that $x$ is even and $y$ is odd.
\item Show that $x^3+8$ is divided by a prime $p$ such that $p\equiv3\pmod{4}$.
\item Show that the equation has no solutions.
\end{cond}
\end{prb}








\end{document}